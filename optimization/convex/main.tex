\chapter{凸集}

\subsubsection{直线}
\begin{definition}
	设$x_1,x_2\in\mathbb{R}^{n}$且$x_1\ne x_2$。若$\theta\in[0,1]$,称所有满足形式:
	\begin{equation*}
		y=\theta x_1+(1-\theta)x_2
	\end{equation*}
	的点$y$构成的集合为\gls{LineSegment}。若$\theta\in\mathbb{R}^{}$,称所有满足上述形式的点$y$构成的集合为\gls{Line}。
\end{definition}
\begin{note}
	上述形式等价于:
	\begin{equation*}
		y=x_2+\theta(x_1-x_2)
	\end{equation*}
	这给出了另一种直观:$y$是基$x_2$和被$\theta$缩放的方向$x_1-x_2$的和。
\end{note}
\subsubsection{凸集}
\begin{definition}
	设$E\subseteq\mathbb{R}^{n}$。若对于任意不同的$x_1,x_2\in E$,由$x_1,x_2$确定的线段都在$E$中,则称$E$是一个\gls{ConvexSet}。
\end{definition}
\begin{definition}
	设$n\in\mathbb{N}^+$,$\seq{\theta}{n}\geqslant0$且$\sum\limits_{i=1}^{n}\theta_i=1$,$\seq{x}{n}\in\mathbb{R}^{n}$。称:
	\begin{equation*}
		\theta_1x_1+\theta_2x_2+\cdots+\theta_nx_n
	\end{equation*}
	为$\seq{x}{n}$构成的一个\gls{ConvexCombination}。
\end{definition}
\begin{definition}
	设$E\subseteq\mathbb{R}^{n}$。定义:
	\begin{equation*}
		\left\{\sum_{i=1}^{n}\theta_ix_i:x_i\in E,\;\theta_i\geqslant0,\;\sum_{i=1}^{n}\theta_i=1,\;n\in\mathbb{N}^+\right\}
	\end{equation*}
	为$E$的\gls{ConvexHull},记作$\operatorname{Conv}(E)$。
\end{definition}
\begin{property}\label{prop:ConvexSet}
	设$E\subseteq\mathbb{R}^{n}$。凸集具有如下性质:
	\begin{enumerate}
		\item $E$是凸集当且仅当$E$中任意有限个元素的凸组合都在$E$中;
		\item $E$的凸包是包含$E$的最小凸集;
		\item 凸集的交集是凸集;
	\end{enumerate}
\end{property}
\begin{proof}
	(1)\textbf{必要性:}当$n=2$时,由凸集的定义可知$E$中任意$2$个元素的凸组合都在$E$中。假设结论对$n-1$成立,下面证明对$n$个也成立。\par
	任取$\seq{\theta}{n}\geqslant0$且$\sum\limits_{i=1}^{n}\theta_i=1$,$\seq{x}{n}\in E$,则:
	\begin{equation*}
		\sum_{i=1}^{n}\theta_ix_i=\sum_{i=1}^{n-1}\theta_ix_i+\theta_nx_n=\sum_{i=1}^{n-1}\theta_ix_i+\left(1-\sum_{i=1}^{n-1}\theta_i\right)x_i
	\end{equation*}
	若$\sum\limits_{i=1}^{n-1}\theta_i=0$,则上式退化为$x_n$,结论成立。若$\sum\limits_{i=1}^{n-1}\theta_i\ne0$,则:
	\begin{equation*}
		y=\sum_{i=1}^{n-1}\frac{\theta_i}{\sum\limits_{j=1}^{n-1}\theta_j}x_i
	\end{equation*}
	是关于$\seq{x}{n-1}$的一个凸组合,由归纳假设可知$y\in E$。变形可得:
	\begin{equation*}
		\sum_{i=1}^{n}\theta_ix_i=\sum_{i=1}^{n-1}\theta_ix_i+\left(1-\sum_{i=1}^{n-1}\theta_i\right)x_i=\sum\limits_{j=1}^{n-1}\theta_jy+\left(1-\sum_{i=1}^{n-1}\theta_i\right)x_i\in E
	\end{equation*}
	必要性得证。\par
	\textbf{充分性:}任意有限个包含$2$个。\par
	(2)由(1)立即可得。\par
	(3)任取指标集$I$和凸集$C_i,i\in I$,令$C=\underset{i\in I}{\overset{}{\cap}}C_i$。任取$x_1,x_2\in C$和$\theta\geqslant0$,则$x_1,x_2\in C_i$。因为$C_i$是凸集,所以$\theta x_1+(1-\theta)x_2\in C_i$,于是$\theta x_1+(1-\theta)x_2\in C$,$C$是凸集。
\end{proof}
\subsubsection{仿射集}
\begin{definition}
	设$E\subseteq\mathbb{R}^{n}$。若对于任意不同的$x_1,x_2\in E$,由$x_1,x_2$确定的直线都在$E$中,则称$E$是一个\gls{AffineSet}。
\end{definition}
\begin{definition}
	设$n\in\mathbb{N}^+$,$\seq{\theta}{n}\in\mathbb{R}^{}$且$\sum\limits_{i=1}^{n}\theta_i=1$,$\seq{x}{n}\in\mathbb{R}^{n}$。称:
	\begin{equation*}
		\theta_1x_1+\theta_2x_2+\cdots+\theta_nx_n
	\end{equation*}
	为$\seq{x}{n}$构成的一个\gls{AffineCombination}。
\end{definition}
\begin{definition}
	设$E\subseteq\mathbb{R}^{n}$。定义:
	\begin{equation*}
		\left\{\sum_{i=1}^{n}\theta_ix_i:x_i\in E,\;\theta_i\in\mathbb{R}^{},\;\sum_{i=1}^{n}\theta_i=1,\;n\in\mathbb{N}^+\right\}
	\end{equation*}
	为$E$的\gls{AffineHull},记作$\operatorname{Aff}(E)$。
\end{definition}
\begin{property}\label{prop:AffineSet}
	设$E\subseteq\mathbb{R}^{n}$。仿射集具有如下性质:
	\begin{enumerate}
		\item $E$是仿射集当且仅当$E$中任意有限个元素的仿射组合都在$E$中;
		\item $E$是仿射集当且仅当存在一个$\mathbb{R}^{n}$的子空间$V$,使得$E=\{x+x_0:x\in V,\;x_0\in E\}$,并且表达式与$x_0$的选择无关;
		\item $E$的仿射包是包含$E$的最小仿射集;
		\item 仿射集是凸集;
	\end{enumerate}
\end{property}
\begin{proof}
	(1)完全类似于\cref{prop:ConvexSet}(1)。\par
	(2)\textbf{必要性:}设$E$是一个仿射集,任取$x_0\in E$,令$V=\{x-x_0:x\in E\}$,则$E=\{x+x_0:x\in V,\;x_0\in E\}$。\par
	任取$\alpha,\beta\in\mathbb{R}^{}$和$x_1,x_2\in V$,则:
	\begin{equation*}
		\alpha x_1+\beta x_2+x_0=\alpha(x_1+x_0)+\beta(x_2+x_0)+(1-\alpha-\beta)x_0\in E
	\end{equation*}
	于是$\alpha x_1+\beta x_2\in V$。由\cref{theo:Subspace}可知$V$是一个子空间。\par
	取不同于$x_0$的$x_1\in E$构成集合$V'=\{x-x_1:x\in C\}$。任取$V'$中的元素$\alpha=x-x_1$,则$\alpha=x+x_1-x_0$是$x,x_1,x_0$构成的一个仿射组合,于是$\alpha\in V$。由$\alpha$的任意性可知$V'\subseteq V$。同理可得$V\subseteq V'$,于是有$V=V'$,即表达式与$x_0$的选择无关。\par
	\textbf{充分性:}任取$E$中不同的两点$x_1+x_0,x_2+x_0$和$\theta\in\mathbb{R}^{}$,有:
	\begin{equation*}
		\theta(x_1+x_0)+(1-\theta)(x_2+x_0)=\theta x_1+(1-\theta)x_2+x_0
	\end{equation*}
	因为$V$是一个子空间,由\cref{theo:Subspace}可知$\theta x_1+(1-\theta)x_2\in V$,所以$\theta x_1+(1-\theta)x_2+x_0\in E$。由$x_1+x_0,x_2+x_0$和$\theta$的任意性可知$E$是一个仿射集。\par
	(3)由(1)立即可得。\par
	(4)由定义立即可得。
\end{proof}
\subsubsection{锥}
\begin{definition}
	设$E\subseteq\mathbb{R}^{n}$。若对于任意的$x\in E$和$\theta\geqslant0$,有$\theta x\in E$,则称$E$是一个\gls{Cone}。若锥$E$还满足对任意的$x_1,x_2\in E$和$\theta_1,\theta_2\geqslant0$,则称$E$为\gls{ConvexCone}。
\end{definition}
\begin{definition}
	设$n\in\mathbb{N}^+$,$\seq{\theta}{n}\geqslant0$,$\seq{x}{n}\in\mathbb{R}^{n}$。称:
	\begin{equation*}
		\theta_1x_1+\theta_2x_2+\cdots+\theta_nx_n
	\end{equation*}
	为$\seq{x}{n}$构成的一个\gls{ConicCombination}。
\end{definition}
\begin{definition}
	设$E\subseteq\mathbb{R}^{n}$。定义:
	\begin{equation*}
		\left\{\sum_{i=1}^{n}\theta_ix_i:x_i\in E,\;\theta_i\geqslant0,\;n\in\mathbb{N}^+\right\}
	\end{equation*}
	为$E$的\gls{ConicHull},记作$\operatorname{Cone}(E)$。
\end{definition}
\begin{property}
	设$E\subseteq\mathbb{R}^{n}$。锥具有如下性质:
	\begin{enumerate}
		\item $E$是锥包当且仅当$E$中任意有限个元素的锥组合都在$E$中;
		\item $E$的锥包是包含$E$的最小锥;
	\end{enumerate}
\end{property}
\begin{proof}
	(1)完全类似于\cref{prop:ConvexSet}(1)。\par
	(2)由(1)立即可得。
\end{proof}
\subsubsection{超平面与半空间}
\begin{definition}
	设$a\in\mathbb{R}^{n},\;b\in\mathbb{R}^{}$。称$\{x\in\mathbb{R}^{n}:a^Tx=b\}$为\gls{Hyperplane},$\{x\in\mathbb{R}^{n}:a^Tx\leqslant b\}$为闭\gls{Halfspace},$\{x\in\mathbb{R}^{n}:a^Tx< b\}$为开半空间。
\end{definition}
\begin{property}\label{prop:HyperplaneHalfspace}
	超平面与半空间具有如下性质:
	\begin{enumerate}
		\item 超平面是仿射集;
		\item 半空间是凸集;
	\end{enumerate}
\end{property}
\begin{proof}
	(1)任取超平面$\{x\in\mathbb{R}^{n}:a^Tx=b\}$和其中任意两点$x_1,x_2$,再任取$\theta\in\mathbb{R}^{}$,有:
	\begin{equation*}
		a^T[\theta x_1+(1-\theta)x_2]=\theta b+(1-\theta)b=b
	\end{equation*}
	所以$\theta x_1+(1-\theta)x_2\in\{x\in\mathbb{R}^{n}:a^Tx=b\}$,超平面是仿射集。\par
	(2)任取闭半空间$\{x\in\mathbb{R}^{n}:a^Tx\leqslant b\}$和其中任意两点$x_1,x_2$,再任取$\theta\geqslant0$,有:
	\begin{equation*}
		a^T[\theta x_1+(1-\theta)x_2]\leqslant\theta b+(1-\theta)b=b
	\end{equation*}
	所以$\theta x_1+(1-\theta)x_2\in\{x\in\mathbb{R}^{n}:a^Tx\leqslant b\}$,闭半空间是凸集。开半空间同理可得。
\end{proof}
\subsubsection{多面体}
\begin{definition}
	设$A\in M_{m\times n}(\mathbb{R}^{}),\;b\in\mathbb{R}^{n},\;C\in M_{p\times q}(\mathbb{R}^{}),\;d\in\mathbb{R}^{q}$。称$\{x\in\mathbb{R}^{n}:Ax\leqslant b,\;Cx=d\}$为\gls{Polyhedron}。
\end{definition}
\begin{property}
	多面体是凸集。
\end{property}
\begin{proof}
	由\cref{prop:ConvexSet}(3)、\cref{prop:HyperplaneHalfspace}(1)、\cref{prop:AffineSet}(4)和\cref{prop:HyperplaneHalfspace}(2)立即可得。
\end{proof}
\subsubsection{球与椭球}
\begin{definition}
	设$||\cdot||$是$\mathbb{R}^{n}$上的一个范数,$x_c\in\mathbb{R}^{n},\;r\in\mathbb{R}^{}$。称$\{x\in\mathbb{R}^{n}:||x-x_c||\leqslant r\}$为\gls{Ball}。
\end{definition}
\begin{definition}
	设$P\in M_{n}(\mathbb{R}^{})$是正定矩阵,$x_c\in\mathbb{R}^{n}$。由\cref{theo:PositiveDefinite}(6)可得$P$的可逆性,称$\{x\in\mathbb{R}^{n}:(x-x_c)^TP^{-1}(x-x_c)\leqslant1\}$为\gls{Elliposid}。
\end{definition}
\begin{property}
	球与椭球具有如下性质:
	\begin{enumerate}
		\item 球是凸集;
		\item 椭球$\{x\in\mathbb{R}^{n}:(x-x_c)^TP^{-1}(x-x_c)\leqslant1\}$等价于$\{x_c+P^{\frac{1}{2}}u:u\in\mathbb{R}^{n},\;||u||_2\leqslant1\}$,由$P$的正定性可得$P^{\frac{1}{2}}$的存在性;
		\item 椭球是凸集;
	\end{enumerate}
\end{property}
\begin{proof}
	(1)任取球$\{x\in\mathbb{R}^{n}:||x-x_c||\leqslant r\}$和其中任意两点$x_1,x_2$,再任取$\theta\geqslant0$,有:
	\begin{equation*}
		||\theta x_1+(1-\theta)x_2-x_c||=||\theta(x_1-x_c)+(1-\theta)(x_2-x_c)||\leqslant\theta||x_1-x_c||+(1-\theta)||x_2-x_c||\leqslant\theta r+(1-\theta)r=r
	\end{equation*}
	所以$\theta x_1+(1-\theta)x_2\in\{x\in\mathbb{R}^{n}:||x-x_c||\leqslant r\}$,球是凸集。\par
	(2)任取$x\in\{x_c+P^{\frac{1}{2}}u:u\in\mathbb{R}^{n},\;||u||_2\leqslant1\}$,由\cref{prop:Transpose}(4)可得:
	\begin{equation*}
		(x-x_c)^TP^{-1}(x-x_c)=(P^{\frac{1}{2}}u)^TP^{-1}P^{\frac{1}{2}}u=||u||_2\leqslant1
	\end{equation*}
	所以$\{x_c+P^{\frac{1}{2}}u:u\in\mathbb{R}^{n},\;||u||_2\leqslant1\}\subseteq\{x\in\mathbb{R}^{n}:(x-x_c)^TP^{-1}(x-x_c)\leqslant1\}$。\par
	任取$x\in\{x\in\mathbb{R}^{n}:(x-x_c)^TP^{-1}(x-x_c)\leqslant1\}$。由$P$的正定性可得$P^{-\frac{1}{2}}$的存在性,取$u=P^{-\frac{1}{2}}(x-x_c)$,根据\cref{prop:Transpose}(4)可得:
	\begin{equation*}
		||u||_2=[P^{-\frac{1}{2}}(x-x_c)]^TP^{-\frac{1}{2}}(x-x_c)=(x-x_c)^TP^{-1}(x-x_c)\leqslant1
	\end{equation*}
	所以$\{x\in\mathbb{R}^{n}:(x-x_c)^TP^{-1}(x-x_c)\leqslant1\}\subseteq\{x_c+P^{\frac{1}{2}}u:u\in\mathbb{R}^{n},\;||u||_2\leqslant1\}$。\par
	综上,$\{x\in\mathbb{R}^{n}:(x-x_c)^TP^{-1}(x-x_c)\leqslant1\}=\{x_c+P^{\frac{1}{2}}u:u\in\mathbb{R}^{n},\;||u||_2\leqslant1\}$。\par
	(3)任取椭球$\{x\in\mathbb{R}^{n}:(x-x_c)^TP^{-1}(x-x_c)\leqslant1\}$和其中任意两点$x_1=x_c+P^{\frac{1}{2}}u_1,x_2=x_c+P^{\frac{1}{2}}u_2$,再任取$\theta\geqslant0$,有:
	\begin{equation*}
		\theta x_1+(1-\theta)x_2=x_c+P^{\frac{1}{2}}[\theta u_1+(1-\theta)u_2]
	\end{equation*}
	由(1)可知$||\theta u_1+(1-\theta)u_2||_2\leqslant1$,所以$\theta x_1+(1-\theta)x_2\in\{x\in\mathbb{R}^{n}:(x-x_c)^TP^{-1}(x-x_c)\leqslant1\}$,椭球是凸集。
\end{proof}

\begin{definition}
	设$X$是一个欧氏空间,$x_1,x_2\in X$。若存在$\alpha_1,\alpha_2$满足$\alpha_1+\alpha_2=1,\;0\leqslant\alpha_i\leqslant1,\;i=1,2$,使得$x=\alpha_1x_1+\alpha_2x_2$,则称$x$为$x_1,x_2$的\gls{ConvexCombination}。若$0<\alpha_i<1,\;i=1,2$,称其为\gls{StrictlyConvexCombination}。
\end{definition}
\begin{definition}
	设$E$是欧式空间$X$中的一个凸集,$x\in E$。若$x$不能表示为$E$中不同的两点的严格凸组合,则称$x$是$E$的一个\gls{ExtremePoint}。
\end{definition}

\begin{definition}
	设$f$是一个定义在开区间$(a,b)$上的实值函数,$a,b\in\overline{\mathbb{R}^{}}$。若对任意的$a<x<y<b$和任意的$\alpha\in(0,1)$有:
	\begin{equation*}
		f[\alpha x+(1-\alpha)y]\leqslant\alpha f(x)+(1-\alpha)f(y)
	\end{equation*}
	则称$f$是\gls{ConvexFunction},若上式取严格的小于号则称$f$\gls{StrictlyConvex}。若$-f$是凸函数,则称$f$为\gls{ConcaveFunction}。
\end{definition}
\begin{property}\label{prop:ConvexFunction}
	设$f$是定义在开区间$(a,b)$上的函数。凸函数具有如下性质:
	\begin{enumerate}
		\item 若$f$在$(a,b)$上可微,则$f$是凸函数的充分必要条件为对任意的$a<x<y<b$有$f'(x)\leqslant f'(y)$,是严格凸函数的充分必要条件为对任意的$a<x<y<b$有$f'(x)<f'(y)$;
		\item 若$f$在$(a,b)$上二次可微,则$f$是凸函数的充分必要条件为对任意的$x\in(a,b)$有$f''(x)\geqslant0$,是严格凸函数的充分必要条件为对任意的$x\in(a,b)$有$f''(x)>0$;
		\item 若$f$是一个凸函数,$t\in(a,b)$,则存在经过点$(t,f(t))$的直线$g(x)=c(x-t)+f(t)$满足对任意的$x\in(a,b)$有$g(x)\leqslant f(x)$。当$f$为严格凸函数时,$g(x)=f(x)$当且仅当$x=t$。
	\end{enumerate}
\end{property}

\subsection{映射的上下极限与半连续性}
\begin{definition}
	$(X,\rho)$是一个度量空间,$E$是$X$的子空间,$f$是$E$到$\mathbb{R}$上的映射。对于$E$中的任一聚点$a$,定义:
	\begin{gather*}
		\liminf_{x\to a}f(x)=\lim_{\varepsilon\to0}\Bigl(\inf\{f(x):x\in U(a,\varepsilon)\}\backslash\{a\}\Bigr) \\
		\limsup_{x\to a}f(x)=\lim_{\varepsilon\to0}\Bigl(\sup\{f(x):x\in U(a,\varepsilon)\}\backslash\{a\}\Bigr)
	\end{gather*}
\end{definition}
\begin{definition}
	$(X,\rho)$是一个度量空间,$f$是$X$到$\overline{\mathbb{R}}$上的映射。若:
	\begin{equation*}
		\limsup_{x\to a}f(x)\leqslant f(x_0)
	\end{equation*}
	则称$f(x)$在$a$点\gls{UpperSemicontinuous}。
	若:
	\begin{equation*}
		\liminf_{x\to a}f(x)\leqslant f(x_0)
	\end{equation*}
	则称$f(x)$在$a$点\gls{lowerSemicontinuous}。
\end{definition}
