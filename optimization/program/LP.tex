\section{线性规划}

\subsection{线性规划的标准形}
线性规划问题的数学模型有各种不同的形式,如:
\begin{enumerate}
	\item 最大化或最小化目标函数;
	\item 约束条件有$\leqslant,\;\geqslant\;=$三种情况;
	\item 决策变量一般有非负性要求,但有时又可能没有。
\end{enumerate}
为了以统一的方式求解线性规划问题,规定了一种线性规划的标准形式,非标准形可以转化为标准形。
\begin{definition}
	规定线性规划问题的\textbf{标准形式}为:
	\begin{enumerate}
		\item 最大化目标函数;
		\item 约束条件为等式且右端常数项非负;
		\item 决策变量非负。
	\end{enumerate}
	于是线性规划问题可以写为如下矩阵形式:
	\begin{gather*}
		\max z=C^TX \\
		s.t.
		\begin{cases}
			AX=\beta \\
			X\geqslant\mathbf{0}
		\end{cases}
	\end{gather*}
	称$C=(c_1,c_2,\dots,c_n)$为\textbf{价值向量},其中每个分量为\textbf{价值系数},$\beta$为\textbf{资源向量}。系数矩阵$A\in M_{m\times n}(\mathbb{R})$行满秩,即$\operatorname{rank}(A)=m$。称$A$的秩$m$为LP问题的\textbf{阶数},$A$的列数$n$为LP问题的\textbf{维数}。
\end{definition}
\begin{theorem}
	每个线性规划问题都可以化成标准形式。
\end{theorem}
\begin{proof}
	考虑每种不符合标准形式的情况:\par
	\textbf{最小化目标函数:}若LP问题的目标函数为:
	\begin{equation*}
		\min z=C^TX
	\end{equation*}
	则可将其转化为:
	\begin{equation*}
		\max z'=-C^TX
	\end{equation*}
	显然转化前后的LP问题是等价的。\par
	\textbf{约束条件右端常数项为负数:}若LP问题的约束条件为$Ax=\beta$和$\alpha^Tx=b<0$,则可将其转化为:
	\begin{equation*}
		\begin{pmatrix}
			A \\
			-\alpha^T
		\end{pmatrix}
		x
		=
		\begin{pmatrix}
			\beta \\
			-b
		\end{pmatrix}
	\end{equation*}
	显然转化前后的LP问题是等价的。\par
	\textbf{约束条件中含$\leqslant$:}若LP问题的约束条件为$Ax=\beta$和$\alpha^Tx\leqslant b$,则可将约束条件转化为:
	\begin{gather*}
		\begin{pmatrix}
			A & \mathbf{0} \\
			\alpha^T & 1
		\end{pmatrix}
		\begin{pmatrix}
			x \\
			y
		\end{pmatrix}
		=
		\begin{pmatrix}
			\beta \\
			b
		\end{pmatrix} \\
		X'=(X,y)^T\geqslant\mathbf{0}
	\end{gather*}
	显然转化前后的LP问题是等价的。称新引入的变量$y$为\gls{SlackVariable}。\par
	\textbf{约束条件中含$\geqslant$:}若LP问题的约束条件为$Ax=\beta$和$\alpha^Tx\geqslant b$,则可将约束条件转化为:
	\begin{gather*}
		\begin{pmatrix}
			A & \mathbf{0} \\
			\alpha^T & -1
		\end{pmatrix}
		\begin{pmatrix}
			x \\
			y
		\end{pmatrix}
		=
		\begin{pmatrix}
			\beta \\
			b
		\end{pmatrix} \\
		X'=(X,y)^T\geqslant\mathbf{0}
	\end{gather*}
	显然转化前后的LP问题是等价的。称新引入的变量$y$为\gls{SurplusVariable}。\par
	\textbf{决策变量中含负值变量:}若LP问题为:
	\begin{gather*}
		\max z=(C,c)^T(X,x)^T \\
		s.t.
		\begin{cases}
			(A,\alpha)(X,x)^T=\beta \\
			X\geqslant\mathbf{0} \\
			x<0
		\end{cases}
	\end{gather*}
	则作变换$x'=-x$,将该LP问题转化为:
	\begin{gather*}
		\max z=(C,-c)^T(X,x')^T \\
		s.t.
		\begin{cases}
			(A,-\alpha)(X,x')^T=\beta \\
			X'=(X,x')^T\geqslant\mathbf{0}
		\end{cases}
	\end{gather*}
	显然转化前后的LP问题是等价的。\par
	\textbf{决策变量中含自由变量\footnote{即对该变量的取值无任何要求。}:}若LP问题为:
	\begin{gather*}
		\max z=(C,c)^T(X,x)^T \\
		s.t.
		\begin{cases}
			(A,\alpha)(X,x)^T=\beta \\
			X\geqslant\mathbf{0}
		\end{cases}
	\end{gather*}
	可引入新变量$x_1,x_2$,将该LP问题转化为:
	\begin{gather*}
		\max z=(C,c,-c)^T(X,x_1,x_2)^T \\
		s.t.
		\begin{cases}
			(A,\alpha,-\alpha)(X,x_1,x_2)^T=\beta \\
			X'=(X,x_1,x_2)^T\geqslant\mathbf{0}
		\end{cases}
	\end{gather*}
	显然转化前后的LP问题是等价的。
\end{proof}
\subsubsection{标准形的基本解}
\begin{definition}
	对于标准形LP问题中约束条件的系数矩阵$A=(\seq{\alpha}{n})\in M_{m\times n}(\mathbb{R})$,称由其列向量组的一个基$\alpha_{i_1},\alpha_{i_2},\dots,\alpha_{i_m}$构成的矩阵$(\alpha_{i_1},\alpha_{i_2},\dots,\alpha_{i_m})$为该LP问题的一个\textbf{基矩阵}(简称\textbf{基}),其中的每一个向量$\alpha_{i_j},\;j=1,2,\dots,m$称为该LP问题的\textbf{基向量},$A$中剩余的$n-m$个列向量$\alpha_{j_1},\alpha_{j_2},\dots,\alpha_{j_{n-m}}$称为\textbf{非基向量}。分别称基向量对应的变量和非基向量对应的变量为\textbf{基变量}、\textbf{非基变量}。分别称所有基变量构成的向量$(x_{i_1},x_{i_2},\dots,x_{i_m})$和非基变量构成的向量$(x_{j_1},x_{j_2},\dots,x_{j_{n-m}})$为\textbf{基变矢}、\textbf{非基变矢}。对系数矩阵$A$的列向量、决策变量$x$和资源向量$\beta=(b_1,b_2,\dots,b_n)^T$作顺序的重排,可以得到:
	\begin{align*}
		&\quad(\alpha_{i_1},\alpha_{i_2},\dots,\alpha_{i_m},\alpha_{j_1},\alpha_{j_2},\dots,\alpha_{j_{n-m}})(x_{i_1},x_{i_2},\dots,x_{i_m},x_{j_1},x_{j_2},\dots,x_{j_{n-m}})^T \\
		&=(b_{i_1},b_{i_2},\dots,b_{i_m},b_{j_1},b_{j_2},\dots,b_{j_{n-m}})^T
	\end{align*}
	将该方程重新写作:
	\begin{equation*}
		(B,C)(X_1,X_2)^T=\beta'
	\end{equation*}
	其中$B$对应基矩阵,$C$对应非基向量构成的矩阵,$X_1$为基变矢,$X_2$为非基变矢。令$X_2=\mathbf{0}$,可得出方程的一个解:
	\begin{equation*}
		\begin{cases}
			X_1=B^{-1}\beta' \\
			X_2=\mathbf{0}
		\end{cases}
	\end{equation*}
	称该解为约束方程组的一个关于基矩阵$B$的\textbf{基本解},也称之为该标准形LP问题的一个基本解。若基本解中存在基变量值为$0$,则称之为\textbf{退化基本解}。若某个基本解达到标准形LP问题的最优解,则称之为\textbf{最优基本解}。称满足非负性约束$X\geqslant\mathbf{0}$的基本解为\gls{BasicFeasibleSolution},若它是退化的,则称之为\textbf{退化基本可行解}。称基本可行解对应的基矩阵为\textbf{可行基},称最优基本解对应的基矩阵为\textbf{最优基}。
\end{definition}

\subsection{线性规划解的性质}
\begin{lemma}
	对于$m$阶$n$维标准形LP问题,$A=(\seq{a}{n})$、$E$分别为其系数矩阵和可行域,则$x=(x_1,x_2,\dots,x_n)\in E$是基本可行解的充分必要条件为:对于$I=\{i:x_i>0,i=1,2,\dots,n\}$,由$a_i,\;i\in I$构成的向量组线性无关。
\end{lemma}
\begin{proof}
	\textbf{必要性:}由基本可行解的定义直接得到。\par
	\textbf{充分性:}
\end{proof}

\begin{property}
	线性规划的解具有如下性质:
	\begin{enumerate}
		\item LP问题解的可行域是凸集;
		\item LP问题的基本可行解与可行域的极点一一对应;
		\item \textbf{线性规划基本定理}
		\begin{itemize}
			\item 若LP问题有可行解,则必有基本可行解;
			\item 若LP问题有最优解,则必有最优基本解。
		\end{itemize}
		\item 若LP问题的可行域不是空集,则可行域至少有一个极点;
		\item LP问题可行域的极点必为有限多个。
	\end{enumerate}
\end{property}
\begin{proof}
	设$m$阶$n$维LP问题如下:
	\begin{gather*}
		\max z=C^TX \\
		s.t.
		\begin{cases}
			AX=\beta \\
			X\geqslant\mathbf{0}
		\end{cases}
	\end{gather*}
	其可行域为$E$,$A=(\seq{a}{n})\in M_{m\times n}(\mathbb{R}),\;x=(x_1,x_2,\dots,x_n)^T$。\par
	(1)显然:
	\begin{equation*}
		E=\{x\in\mathbb{R}^{n}:Ax=\beta,x\geqslant\mathbf{0}\}
	\end{equation*}
	任取$x_1,x_2\in E$,则对任意的$0\leqslant\alpha\leqslant1$有:
	\begin{gather*}
		A[\alpha x_1+(1-\alpha)x_2]=\alpha Ax_1+(1-\alpha)Ax_2=\alpha\beta+(1-\alpha)\beta=\beta \\
		\alpha x_1+(1-\alpha)x_2\geqslant0
	\end{gather*}
	所以$\alpha x_1+(1-\alpha)x_2\in E$,$E$是一个凸集。\par
	(2)\textbf{基本可行解必是极点:}要证基本可行解$x$必是极点,即证$x$不能表示为$E$内不同的两点$x_1,x_2$的严格凸组合。假设$x$可以表示为$x_1,x_2$的严格凸组合,即存在$0<\alpha<1$,使得$x=\alpha x_1+(1-\alpha)x_2$。因为$x$是基本解,所以$x$的非基变量都为$0$,于是$x_1,x_2$对应部分也都为$0$。设$x$对应的基矩阵是$B$,则$x=B^{-1}\beta$,此时也必有$x_1=x_2=B^{-1}\beta$(考虑基本解的求解过程,$B$对应的非基变量全为$0$),于是$x=x_1=x_2$,所以$x$不可以表示为可行域内不同的两点的严格凸组合,矛盾,$x$是极点。\par
	\textbf{极点必是基本可行解:}要证极点$x$必是基本可行解,即证存在基矩阵$B$使得$x=B^{-1}\beta$。由极点定义,$x$不能表示为$E$内不同的两点$x_1,x_2$的严格凸组合,即不存在$0<\alpha<1$和$E$内不同的两点$x_1,x_2$使得$x=\alpha x_1+(1-\alpha)x_2$,
\end{proof}