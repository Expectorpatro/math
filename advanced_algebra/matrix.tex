\chapter{矩阵}

\begin{theorem}
	设$\alpha$是一个$3$维列向量,$\alpha\alpha^T=
	\begin{pmatrix}
		1 & -1 & 1 \\
		-1 & 1 & -1 \\
		1 & -1 & 1
	\end{pmatrix}$
	求$\alpha^T\alpha$。
\end{theorem}
\begin{proof}
	设$\alpha=(x,y,z)^T$,则:
	\begin{equation*}
		\alpha\alpha^T=
		\begin{pmatrix}
			x^2 & xy & xz \\
			yx & y^2 & yz \\
			zx & zy & z^2
		\end{pmatrix}
	\end{equation*}
	于是$\alpha^T\alpha=x^2+y^2+z^2=3$。
\end{proof}
\begin{theorem}
	设$A=I-\alpha\alpha^T$,$I$为$n$阶单位阵,$\alpha$是$n$维非零列向量,证明:
	\begin{enumerate}
		\item $A^2=A$的充要条件为$\alpha^T\alpha=1$;
		\item 当$\alpha^T\alpha=1$时,$A$不可逆。
	\end{enumerate}
\end{theorem}
\begin{proof}
	(1)显然:
	\begin{align*}
		A^2-A
		&=(I-\alpha\alpha^T)(I-\alpha\alpha^T)-I+\alpha\alpha^T \\
		&=I-2\alpha\alpha^T+\alpha\alpha^T\alpha\alpha^T-I+\alpha\alpha^T \\
		&=-\alpha\alpha^T+\alpha(\alpha^T\alpha)\alpha^T \\
		&=-\alpha\alpha^T+(\alpha^T\alpha)\alpha\alpha^T \\
		&=(\alpha^T\alpha-1)\alpha\alpha^T
	\end{align*}
	因为$\alpha\ne\mathbf{0}$,所以$A^2=A$的充分必要条件为$\alpha^T\alpha=1$。\par
	(2)设$A$可逆,则存在矩阵$B$,使得$AB=BA=I$。由(1)得$A^2=A$,两边同乘$B$可得
	\begin{equation*}
		BA^2=BAA=A=AB=I
	\end{equation*}
	而$\alpha\ne\mathbf{0}$,于是$A\ne I$,矛盾,所以$A$不可逆。
\end{proof}
\begin{theorem}
	设$A,B$为$n$阶方阵,且$AB=A+B$,证明$AB=BA$。
\end{theorem}
\begin{proof}
	显然:
	\begin{gather*}
		AB-A-B+E=E \\
		(A-E)(B-E)=E
	\end{gather*}
	所以$A-E$可逆,$B-E$为其逆,于是:
	\begin{equation*}
		(B-E)(A-E)=E
	\end{equation*}
	所以$BA-B-A+E=E$,$BA=B+A$,从而$AB=BA$。
\end{proof}

\section{求矩阵的幂}
\begin{enumerate}
	\item 数学归纳法
	\item 二项式公式,将矩阵分解为可交换得两个矩阵
	\item 将矩阵拆分为列向量的积
	\item 将矩阵分块,求分块对角阵的幂
	\item 利用相似标准形
\end{enumerate}
\begin{theorem}
	设$A=
	\begin{pmatrix}
		\lambda & 1 & 0 \\
		0 & \lambda & 1 \\
		0 & 0 & \lambda
	\end{pmatrix}$,求$A^k$。
\end{theorem}
\begin{proof}
	数学归纳法。
\end{proof}
\begin{theorem}
	设$A=
	\begin{pmatrix}
		1 & 0 & 1\\
		0 & 2 & 0 \\
		1 & 0 & 1
	\end{pmatrix}$,求$A^k$。
\end{theorem}
\begin{proof}
	(1)拆分为主对角线、副对角线两个矩阵。(2)相似。(3)数学归纳法。
\end{proof}
\begin{theorem}
	设矩阵$A=
	\begin{pmatrix}
		0 & -1 & 1 \\
		2 & -3 & 0 \\
		0 & 0 & 0 
	\end{pmatrix}$:
	\begin{enumerate}
		\item 求$A^{99}$;
		\item 设$3$阶矩阵$B=(\alpha_1,\alpha_2,\alpha_3)$满足$B^2=BA$,记$B^{100}=(\beta_1,\beta_2,\beta_3)$,将$\beta_1,\beta_2,\beta_3$分别表示为$\alpha_1,\alpha_2,\alpha_3$的线性组合。
	\end{enumerate}
\end{theorem}
\begin{proof}
	(1)求矩阵的相似标准形。\par
	(2)归纳总结$B^n=BA{n-1}$。
\end{proof}
\begin{theorem}
	设矩阵$A=
	\begin{pmatrix}
		2 & 4 & 0 & 0 \\
		1 & 2 & 0 & 0 \\
		0 & 0 & 2 & 0 \\
		0 & 0 & 4 & 2 
	\end{pmatrix}$,计算$A^n$。
\end{theorem}
\begin{proof}
	划分分块对角阵计算,将两个分块矩阵分别拆分为一个对角阵和另一个矩阵。
\end{proof}
\begin{theorem}
	设$A=
	\begin{pmatrix}
		1 & 0 & 0 \\
		1 & 0 & 1 \\
		0 & 1 & 0
	\end{pmatrix}$,证明$n\geqslant3$时,有:
	\begin{equation*}
		A^n=A^{n-2}+A^2-E
	\end{equation*}
	并且求$A^{100}$。
\end{theorem}
\begin{proof}
	数学归纳法,证明$n=3$时成立。假设对$n$成立,证明对$n+1$成立。
	\begin{align*}
		A^{n+1}&=AA^n=A(A^{n-2}+A^2-E)=A^{n-1}+A^3-A \\
		&=A^{n-1}+A+A^2-E-A=A^{n-1}+A^2-E
	\end{align*}
	显然有:
	\begin{equation*}
		A^{100}=A^{98}+A^2-E=A^{96}+2A^2-2E=A^2+49A^2-49E=50A^2-49E\qedhere
	\end{equation*}
\end{proof}
\begin{theorem}
	设$A=\begin{pmatrix}
		0 & -1 & 0 \\
		1 & 0 & 0 \\
		0 & 0 & -1
	\end{pmatrix}$,$B=P^{-1}AP$,其中$P$为三阶矩阵,求$B^{2020}-2A^2$。
\end{theorem}
\begin{proof}
	先计算$A^2$。
	\begin{equation*}
		B^{2020}-2A^2=P^{-1}A^{2020}P-2A^2=P^{-1}(A^2)^{1010}P-2A^2
	\end{equation*}
\end{proof}

\section{求矩阵的逆矩阵}
\begin{theorem}
	求矩阵$A=
	\begin{pmatrix}
		0 & 0 & 5 & 2 \\
		0 & 0 & 2 & 1 \\
		1 & -2 & 0 & 0 \\
		1 & 1 & 0 & 0
	\end{pmatrix}$的逆矩阵。
\end{theorem}
\begin{proof}
	考虑副对角线上的分块逆矩阵。
\end{proof}
\begin{theorem}
	已知$A^*=\begin{pmatrix}
		4 & 3 & 0 & 0 \\
		-1 & 0 & 0 & 0 \\
		0 & 0 & 3 & -6 \\
		0 & 0 & -3 & 3
	\end{pmatrix}$,求$A^{-1}$和$A$。
\end{theorem}
\begin{proof}
		$|A^*|=|A|^3$。$A^{-1}=\frac{1}{|A|}A^*$。
\end{proof}
\begin{theorem}
	设$A=
	\begin{pmatrix}
		1 & 0 & 0 \\
		2 & 2 & 0 \\
		3 & 4 & 5
	\end{pmatrix}$,求$(A^*)^{-1}$。
\end{theorem}
\begin{proof}
	$AA^*=|A|E$,于是:
	\begin{equation*}
		\left(\frac{1}{|A|}\right)A^E\qedhere
	\end{equation*}
\end{proof}
\begin{theorem}
	设$A=
	\begin{pmatrix}
		0 & a_1 & 0 & \cdots & 0 \\
		0 & 0 & a_2 & \cdots & 0 \\
		\vdots & \vdots & \vdots & \ddots & \vdots \\
		0 & 0 & 0 & \cdots & a_{n-1} \\
		a_n & 0 & 0 & \cdots & 0
	\end{pmatrix}$,求$A^{-1}$,其中$a_i\ne0$。
\end{theorem}
\begin{proof}
	反序。
\end{proof}
\begin{theorem}
	设$\alpha$为$n$维单位列向量,$E$为$n$阶单位阵,判断$E-\alpha\alpha^T,E+\alpha\alpha^T,E+2\alpha\alpha^T,E-2\alpha\alpha^T$是否可逆。
\end{theorem}
\begin{proof}
	(1)$Ax=0$有非零解$\alpha$,不可逆。其余三个通过特征值判断。
\end{proof}

\subsection{抽象矩阵求逆}
\begin{theorem}
	设方阵$A$满足$A^3-A^2+2A-E=\mathbf{0}$,证明$A$和$E-A$可逆,求逆矩阵。
\end{theorem}
\begin{proof}
	显然:
	\begin{equation*}
		A^3-A^2+2A=E,\;A(A^2-A+2)=E
	\end{equation*} 
	待定系数法求$E-A$的逆矩阵:
	\begin{equation*}
		(E-A)(-A^2+aA+bE)=cE
	\end{equation*}
	将之展开与条件作系数对应。
\end{proof}
\begin{theorem}
	设$A^3=2E,\;B=A^2-2A+2E$,证明$B$可逆并求逆矩阵。
\end{theorem}
\begin{proof}
	显然:
	\begin{equation*}
		B=A^2-2A+A^3=A(A-E)(A+2E)
	\end{equation*}
	证明上式右三个矩阵可逆。$A$是显然的,$A^3-E=E,\;(A-E)(A^2+2A+E)=E,\;A^3+8E=10E,\;(A+2E)(A^2-2A+4E)=10E$。
\end{proof}
\begin{theorem}
	$A^3=\mathbf{0}$,则$E-A,E+A$都可逆。
\end{theorem}
\begin{proof}
	立方差立方和公式。
\end{proof}
\begin{theorem}
	设$A,B,A+B,A^{-1}+B^{-1}$均为$n$阶可逆矩阵,求$(A^{-1}+B^{-1})^{-1}$。
\end{theorem}
\begin{proof}
	显然:
	\begin{equation*}
		A^{-1}+B^{-1}=B^{-1}BA^{-1}+B^{-1}AA^{-1}=B^{-1}(B+A)A^{-1}\qedhere
	\end{equation*}
\end{proof}
\begin{theorem}
	设$n$阶方阵$A,B,C$满足$ABC=E$,证明有$CAB=E$。
\end{theorem}
\begin{proof}
	由逆矩阵的定义直接可得。
\end{proof}
\begin{theorem}
	设$A=\begin{pmatrix}
		1 & 1 & -1 \\
		-1 & 1 & 1 \\
		1 & -1 & 1
	\end{pmatrix}$,且$A^*X(\frac{1}{2}A^*)^*=8A^{-1}X+E$,求$X$。
\end{theorem}
\begin{proof}
	因为$|A|=4$,所以$A$可逆,于是有:
	\begin{gather*}
		A^*=|A|A^{-1}=4A^{-1} \\
		(\frac{1}{2}A^*)^*=(2A^{-1})^*=|2A^{-1}|(2A^{-1})^{-1}=A \\
		4A^{-1}XA=8A{-1}X+E\qedhere
	\end{gather*}
\end{proof}
\begin{theorem}
	设$A^*=
	\begin{pmatrix}
		1 & 0 & 0 & 0 \\
		0 & 1 & 0 & 0 \\
		1 & 0 & 1 & 0 \\
		0 & -3 & 0 & 8
	\end{pmatrix}$,且有$ABA^{-1}=BA^{-1}+3E$,求$B$。
\end{theorem}
\begin{proof}
	两边乘$A$得:
	\begin{equation*}
		AB=B+3A
	\end{equation*}
	两边同乘$A^*$得:
	\begin{equation*}
		A^*AB=A^*B+3AA^*,\;|A|B=A^*B+3|A|E
	\end{equation*}
	由条件,$|A^*|=8$,所以$|A|=2$。
\end{proof}
\begin{theorem}
	设$A,B$为$3$阶矩阵,满足$2A^{-1}B=B-4E$,
	\begin{enumerate}
		\item 证明$A-2E$可逆;
		\item 若$B=
		\begin{pmatrix}
			1 & -2 & 0 \\
			1 & 2 & 0 \\
			0 & 0 & 2
		\end{pmatrix}$,求$A$。
	\end{enumerate}
\end{theorem}
\begin{proof}
	(1)两边同时乘$A$得:
	\begin{gather*}
		2B=AB-4A \\
		2B-AB+4E=\mathbf{0} \\
		(A-2E)(B+bE)=\mathbf{0}
	\end{gather*}
	待定系数法求解。\par
	(2)两边同时乘$A$得:
	\begin{equation*}
		2B=AB-4A,\;2B=A(B-4E)\qedhere
	\end{equation*}
\end{proof}
\begin{theorem}
	设$A=
	\begin{pmatrix}
		1 & 0 & 0 & 0 \\
		-2 & 3 & 0 & 0 \\
		0 & -4 & 5 & 0 \\
		0 & 0 & -6 & 7
	\end{pmatrix}$,$B=(E+A)^{-1}(E-A)$,求$(E+B)^{-1}$。
\end{theorem}
\begin{proof}
	(1)两边左乘$E+A$。\par
	(2)变形:
	\begin{align*}
		E+B&=(E-A)^{-1}(E-A)+(E+A)^{-1}(E-A) \\
		&=(E-A)^{-1}(E+A+E-A)=2(E-A)^{-1}\qedhere
	\end{align*}
\end{proof}
\section{伴随矩阵}
\begin{theorem}
	设$A$是$n$阶方阵,满足$A^m=E$,$m$为正整数。设将$A$中的元素$a_{ij}$用其代数余子式$A_{ij}$代替所得到的矩阵为$B$,证明$B^m=E$。
\end{theorem}
\begin{proof}
	$A^m=AA^{m-1}=E$,所以$A$可逆,于是$A^*=|A|A^{-1}$且$|A|^m=|A^m|=1$。于是:
	\begin{align*}
		B^m&=[(A^*)^T]^m=[(|A|A^{-1})^T]^m \\
		&=[|A|(A^{-1})^T]^m=[(A^{-1})^T]^m \\
		&=[(A^T)^{-1}]^m=[(A^T)^m]^{-1} \\
		&=[(A^m)^T]^{-1}=E\qedhere
	\end{align*}
\end{proof}
\begin{theorem}
	设三阶矩阵$A=(a_{ij})$满足$A^*=A^T$,若$a_{11}=a_{12}=a_{i3}>0$,求$a_{11}$。
\end{theorem}
\begin{proof}
	由$A^*=A^T$可得$A=(A^*)^T$,所以$a_{ij}=A_{ij}$,于是有$|A|=a_{11}A_{11}+a_{12}A_{12}+a_{13}A_{13}=3a_{11}^2>0$。
	\begin{equation*}
		AA^*=|A|E,\;AA^T=|A|E,\;|AA^T|=|A|^3,\;|A|=1\qedhere
	\end{equation*}
\end{proof}
\begin{theorem}
	设$A,B$为$n$阶矩阵,分块矩阵$C=\begin{pmatrix}
		A & \mathbf{0} \\
		\mathbf{0} & B
	\end{pmatrix}$,求$C^*$。
\end{theorem}
\begin{proof}
	
\end{proof}
\begin{theorem}
	设$A$为$n$阶可逆矩,交换$A$的第一行与第二行得到矩阵$B$,则交换$A^*$的第一列和第二列得到$-B^*$。
\end{theorem}
\begin{proof}
	使用初等矩阵。
\end{proof}
\begin{theorem}
	设$3$阶矩阵$A=\begin{pmatrix}
		a & b & b \\
		b & a & b \\
		b & b & a
	\end{pmatrix}$,若$\operatorname{rank}(A^*)=1$,则有$a\ne b,\;a+2b=0$。
\end{theorem}
\begin{proof}
	
\end{proof}

\section{矩阵的秩}
\begin{theorem}
	讨论$n$阶方阵$A=
	\begin{pmatrix}
		a & b & \cdots & b \\
		b & a & \cdots & b \\
		\vdots & \vdots & \ddots & \vdots \\
		b & b & \cdots & a
	\end{pmatrix}$的秩,$n\geqslant2$。
\end{theorem}
\begin{proof}
	第一列加上所有列,第二行到第$n$行依次减去第一行得到:
	\begin{equation*}
		\begin{pmatrix}
			a+(n-1)b & b & \cdots & b \\
			0 & a-b & \cdots & 0 \\
			\vdots & \vdots & \ddots & \vdots \\
			0 & 0 & \cdots & a-b
		\end{pmatrix}
	\end{equation*}
	\begin{enumerate}
		\item $a+(n-1)b\ne0,\;a\ne b,\;\operatorname{rank}(A)=n$;
		\item $a=b\ne0,\;\operatorname{rank}(A)=1$;
		\item $a=b=0,\;\operatorname{rank}(A)=0$;
		\item $a+(n-1)b=0,\;b\ne0,\;\operatorname{rank}(A)=n-1$;
	\end{enumerate}
\end{proof}
\begin{theorem}
	设矩阵$A=
	\begin{pmatrix}
		k & 1 & 1 & 1 \\
		1 & k & 1 & 1 \\
		1 & 1 & k & 1 \\
		1 & 1 & 1 & k
	\end{pmatrix}$,且$\operatorname{rank}(A)=3$,求$k$。
\end{theorem}
\begin{proof}
	令行列式为$0$,讨论求出的$k$是否使得$\operatorname{rank}(A)=3$。
\end{proof}
\begin{theorem}
	设矩阵$A=
	\begin{pmatrix}
		a & -1 & -1 \\
		-1 & a & -1 \\
		-1 & -1 & a
	\end{pmatrix}$与$B=
	\begin{pmatrix}
		1 & 1 & 0 \\
		0 & -1 & 1 \\
		1 & 0 & 1
	\end{pmatrix}$等价,求$a$。
\end{theorem}
\begin{proof}
	等价则秩相同,然后题目就化为了上一题。
\end{proof}
\begin{theorem}
	设$Q=
	\begin{pmatrix}
		1 & 2 & 3 \\
		2 & 4 & t \\
		3 & 6 & 9
	\end{pmatrix}$,$P$为三阶非零矩阵,且满足$PQ=\mathbf{0}$,证明$t\ne6$时$\operatorname{rank}(P)=1$。
\end{theorem}
\begin{proof}
	因为$PQ=\mathbf{0}$,所以$\operatorname{rank}(P)+\operatorname{rank}(Q)\leqslant3$。因为$P\ne\mathbf{0}$,所以$\operatorname{rank}(P)\geqslant1$。$t\ne 6$时$\operatorname{rank}(Q)=2$,于是$\operatorname{rank}(P)=1$。
\end{proof}
\begin{theorem}
	设$A=
	\begin{pmatrix}
		a_1b_1 & a_1b_2 & \cdots & a_1b_n \\
		a_2b_1 & a_2b_2 & \cdots & a_2b_n \\
		\vdots & \vdots & \ddots & \vdots \\
		a_nb_1 & a_nb_2 & \cdots & a_nb_n 
	\end{pmatrix}$,其中$a_ib_j\ne0$,求$\operatorname{rank}(A)$。
\end{theorem}
\begin{proof}
	因为:
	\begin{equation*}
		A=
		\begin{pmatrix}
			a_1 \\
			a_2 \\
			\vdots \\
			a_n
		\end{pmatrix}
		\begin{pmatrix}
			b_1 & b_2 & \cdots & b_n
		\end{pmatrix}
	\end{equation*}
	所以$1\leqslant\operatorname{rank}(A)\leqslant\operatorname{rank}
	\left(\begin{pmatrix}
		b_1 & b_2 & \cdots & b_n
	\end{pmatrix}\right)=1$。
\end{proof}
\begin{theorem}
	设$A$是$m\times n$矩阵,$B$是$n\times m$矩阵,若$AB=E$,证明$\operatorname{rank}(B)=m$。
\end{theorem}
\begin{proof}
	$\operatorname{rank}(AB)=m\leqslant\operatorname{rank}(B)\leqslant m$。
\end{proof}
\begin{theorem}
	设$m\times n$矩阵$A$的秩为$r_1$,$s\times t$矩阵$B$的秩为$r_2$,$C=
	\begin{pmatrix}
		A & \mathbf{0} \\
		\mathbf{0} & B
	\end{pmatrix}$,证明$\operatorname{rank}(C)=r_1+r_2$。
\end{theorem}
\begin{proof}
	用定义证明会更快。
\end{proof}
\begin{theorem}
	设$A$是$m\times n$矩阵,$B$是$n\times t$矩阵,若$AB=\mathbf{0}_{m\times t}$,证明$\operatorname{rank}(A)+\operatorname{rank}(B)\leqslant n$。
\end{theorem}
\begin{theorem}
	设$A$为$m\times n$矩阵,$B$为$n\times t$矩阵,证明:
	\begin{equation*}
		\operatorname{rank}(AB)\geqslant\operatorname{rank}(A)+\operatorname{rank}(B)-n
	\end{equation*}
\end{theorem}
\begin{theorem}
	设$A$是一个$n$阶幂等阵,则$\operatorname{rank}(A)+\operatorname{rank}(A-E)=n$。
\end{theorem}
\begin{proof}
	因为$A(A-E)=\mathbf{0}$,
\end{proof}
\begin{theorem}
	设$A$为$m\times n$阶矩阵,$B$为$n\times m$阶矩阵,若$AB=I$,则$\operatorname{rank}(A)=\operatorname{rank}(B)=m$。
\end{theorem}
\begin{theorem}
	设$\alpha,\beta$为$3$维非零列向量,$A=\alpha\alpha^T+\beta\beta^T$,证明:
	\begin{enumerate}
		\item $\operatorname{rank}(A)\leqslant2$;
		\item 若$\alpha,\beta$线性相关,则$\operatorname{rank}(A)<2$。
	\end{enumerate}
\end{theorem}
\begin{proof}
	(1)由矩阵和的秩公式:
	\begin{equation*}
		\operatorname{rank}(A)=\operatorname{rank}(\alpha\alpha^T+\beta\beta^T)\leqslant\operatorname{rank}(\alpha\alpha^T)+\operatorname{rank}(\beta\beta^T)\leqslant2
	\end{equation*}\par
	(2)设$\beta=k\alpha$,则:
	\begin{equation*}
		\operatorname{rank}(A)=\operatorname{rank}(\alpha\alpha^T)\leqslant\operatorname{rank}(\alpha)<2\qedhere
	\end{equation*}
\end{proof}

\subsection{初等矩阵}