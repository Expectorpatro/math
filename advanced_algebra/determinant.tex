\chapter{行列式}

\begin{theorem}
	求$f(x)$中$x^4$与$x^3$的系数,其中:
	\begin{equation*}
		f(x)=
		\begin{vmatrix}
			x & 2 & 3 & 4 \\
			x & 2x & 3 & 4 \\
			1 & 2 & 3x & 4 \\
			1 & 2 & 3 & 4x
		\end{vmatrix}
	\end{equation*}
\end{theorem}

\begin{theorem}
	设$n\geqslant2$,证明:如果一个$n$阶行列式$D$中的元素为$1$或$-1$,则$D$必为偶数。
\end{theorem}
\begin{proof}
	将$D$按行列式的定义展开一共有$n!$项,因为$D$中的元素为$1$或$-1$,所以展开式中每一项也只能是$1$或$-1$。假设展开式中有$k$项为$-1$,则剩余$n!-k$项为$1$,所以$|D|=-k+(n!-k)=n!-2k$。若$k$为偶数,则$|D|$为偶数;若$k$为奇数,则$|D|$也是偶数。综上,$D$必为偶数。
\end{proof}

\begin{theorem}
	证明元素为$0,1$的三阶行列式$D$的值只能是$0,\pm1,\pm2$。
\end{theorem}
\begin{proof}
	
\end{proof}

\begin{theorem}
	计算行列式:
	\begin{equation*}
		D=
		\begin{vmatrix}
			1 & 1 & 2 & 3 \\
			1 & 2-x^2 & 2 & 3 \\
			2 & 3 & 1 & 5 \\ 
			2 & 3 & 1 & 9-x^2
 		\end{vmatrix}
	\end{equation*}
\end{theorem}
\begin{proof}
	当$2-x^2=1$时,一二两行相同,$|D|=0$,所以$|D|$有因式$(x-1)(x+1)$。当$9-x^2=5$时,一二两行相同,$|D|=0$,所以$|D|$有因式$(x-2)(x+2)$。因为$D$是四阶行列式,所以可设$|D|=k(x-1)(x+1)(x-2)(x+2)$,其中$k$为常数。
\end{proof}

\begin{theorem}
	已知五阶行列式:
	\begin{equation*}
		\begin{vmatrix}
			1 & 2 & 3 & 4 & 5 \\
			2 & 2 & 2 & 1 & 1 \\
			3 & 1 & 2 & 4 & 5 \\
 			1 & 1 & 1 & 2 & 2 \\
			4 & 3 & 1 & 5 & 0
		\end{vmatrix}
		=27
	\end{equation*}
	求$A_{41}+A_{42}+A_{43}$和$A_{44}+A_{45}$。
\end{theorem}
\begin{proof}
	\begin{align*}
		A_{41}+A_{42}+A_{43}+2(A_{44}+A_{45})=27 \\
		2(A_{41}+A_{42}+A_{43})+A_{44}+A_{45}=0
	\end{align*}
	或者更换第四行元素使得新行列式的值等于$A_{41}+A_{42}+A_{43}$和$A_{44}+A_{45}$。
\end{proof}

\begin{theorem}
	设$n$阶行列式:
	\begin{equation*}
		1
	\end{equation*}
\end{theorem}
\begin{proof}
	\begin{equation*}
		A^*=|A|A^{-1}
	\end{equation*}
	分块矩阵
\end{proof}

\begin{theorem}
	已知$A=(a_{ij})\in M_{3}(\mathbb{R})$满足条件:
	\begin{enumerate}
		\item $a_{ij}=A_{ij}$,其中$A_{ij}$是$a_{ij}$的代数余子式
		\item $a_{11}\ne0$。
	\end{enumerate}
	计算行列式$|A|$。
\end{theorem}

\begin{theorem}
	计算$2n$阶行列式:
	\begin{equation*}
		D_{2n}=
		\begin{vmatrix}
			a_n & & & & & & & b_n \\
			& a_{n-1} & & & & & b_{n-1}& \\
			& & \ddots & & & \iddots & & \\ 
			& & & a_1 & b_1 & & & \\ 
			& & & c_1 & d_1 & & & \\ 
			& & c_2 & & & d_2 & & \\ 
			& \iddots & & & & & \ddots & \\ 
			c_n & & & & & & & d_n
		\end{vmatrix}
	\end{equation*}
\end{theorem}
\begin{proof}
	利用Laplace定理,将第$1$行与第$2n$行展开,得到:
	\begin{equation*}
		D_{2n}=
		\begin{vmatrix}
			a_n & b_n \\
			c_n & d_n
		\end{vmatrix}
		(-1)^{1+2n+1+2n}D_{2n-2}
		=(a_nd_n-b_nc_n)D_{2n-2}=\prod_{i=1}^n(a_id_i-b_ic_i)\qedhere
	\end{equation*}
\end{proof}

\begin{theorem}
	计算$n$阶行列式:
	\begin{equation*}
		D_{n}=
		\begin{vmatrix}
			a_1 & b_1 & & & \\
			& a_2 & b_2 & & \\
			& & \ddots & \ddots & \\
			& & & a_{n-1} & b_{n-1} \\
			b_n & & & & a_n
		\end{vmatrix}
	\end{equation*}
\end{theorem}
\begin{proof}
	将该行列式展开:
	\begin{align*}
		D_n
		&=a_1
		\begin{vmatrix}
			a_2 & b_2 & & \\
			& \ddots & \ddots & \\
			& & a_{n-1} & b_{n-1} \\
		 	& & & a_n
		\end{vmatrix}
		+b_n(-1)^{n+1}
		\begin{vmatrix}
			b_1 & & & \\
			a_2 & b_2 & & \\
			& \ddots & \ddots & \\
			& & a_{n-1} & b_{n-1}
		\end{vmatrix} \\
		&=\prod_{i=1}^na_i+(-1)^{n+1}\prod_{i=1}^nb_i\qedhere
	\end{align*}
\end{proof}

\subsubsection{爪形}
\begin{theorem}
	计算$n+1$阶行列式:
	\begin{equation*}
		D_{n+1}=
		\begin{vmatrix}
			a_0 & 1 & 1 & \cdots & 1 \\
			1 & a_1 & 0 & \cdots & 0 \\
			1 & 0 & a_2 & \cdots & 0 \\
			\vdots & \vdots & \vdots & \ddots & \vdots \\
			1 & 0 & 0 & \cdots & a_n
		\end{vmatrix}
	\end{equation*}
	其中$a_1a_2\cdots a_n\ne0$。
\end{theorem}
\begin{proof}
	显然:
	\begin{align*}
		D_{n+1}&=
		\begin{vmatrix}
			a_0-\frac{1}{a_1} & 1 & 1 & \cdots & 1 \\
			0 & a_1 & 0 & \cdots & 0 \\
			1 & 0 & a_2 & \cdots & 0 \\
			\vdots & \vdots & \vdots & \ddots & \vdots \\
			1 & 0 & 0 & \cdots & a_n
		\end{vmatrix} \\
		&=
		\begin{vmatrix}
			a_0-\frac{1}{a_1}-\frac{1}{a_2} & 1 & 1 & \cdots & 1 \\
			0 & a_1 & 0 & \cdots & 0 \\
			0 & 0 & a_2 & \cdots & 0 \\
			\vdots & \vdots & \vdots & \ddots & \vdots \\
			1 & 0 & 0 & \cdots & a_n
		\end{vmatrix} \\
		&=\cdots \\
		&=
		\begin{vmatrix}
			a_0-\sum\limits_{i=1}^{n}\frac{1}{a_i} & 1 & 1 & \cdots & 1 \\
			0 & a_1 & 0 & \cdots & 0 \\
			0 & 0 & a_2 & \cdots & 0 \\
			\vdots & \vdots & \vdots & \ddots & \vdots \\
			0 & 0 & 0 & \cdots & a_n
		\end{vmatrix} \\
		&=\prod_{i=1}^na_i(a_0-\sum\limits_{i=1}^{n}\frac{1}{a_i})\qedhere
	\end{align*}
\end{proof}
\begin{theorem}
	计算$5$阶行列式:
	\begin{equation*}
		D_{5}=
		\begin{vmatrix}
			1-a & a & 0 & 0 & 0 \\
			-1 & 1-a & a & 0 & 0 \\
			0 & -1 & 1-a & a & 0 \\
			0 & 0 & -1 & 1-a & a \\
			0 & 0 & 0 & -1 & 1-a
		\end{vmatrix}
	\end{equation*}
\end{theorem}
\begin{proof}
	显然:
	\begin{align*}
		D_5&=(1-a)
		\begin{vmatrix}
			1-a & a & 0 & 0 \\
			-1 & 1-a & a & 0 \\
			0 & -1 & 1-a & a \\
			0 & 0 & -1 & 1-a
		\end{vmatrix}
		+(-1)^{1+2}a
		\begin{vmatrix}
			-1 & a & 0 & 0 \\
			0 & 1-a & a & 0 \\
			0 & -1 & 1-a & a \\
			0 & 0 & -1 & 1-a
		\end{vmatrix} \\
		&=(1-a)D_4+a
		\begin{vmatrix}
			1-a & a & 0 \\
			-1 & 1-a & a \\
			0 & -1 & 1-a
		\end{vmatrix} \\
		&=(1-a)D_4+aD_3\qedhere
	\end{align*}
\end{proof}
\begin{theorem}
	计算$n$阶行列式:
	\begin{equation*}
		D_n=
		\begin{vmatrix}
			5 & -3 & 0 & 0 & 0 & 0 \\
			-2 & 5 & -3 & 0 & 0 & 0 \\
			0 & -2 & 5 & -3 & 0 & 0 \\
			0 & 0 & \ddots & \ddots & \ddots & 0 \\
			0 & 0 & 0 & -2 & 5 & -3 \\
			0 & 0 & 0 & 0 & -2 & 5
		\end{vmatrix}
	\end{equation*}
\end{theorem}
\begin{proof}
	显然:
	\begin{align*}
		D_n&=5D_{n-1}+(-1)^{1+2}(-2)
		\begin{vmatrix}
			-3 & 0 & 0 & 0 & 0 \\
			-2 & 5 & -3 & 0 & 0 \\
			0 & \ddots & \ddots & \ddots & 0 \\
			0 & 0 & -2 & 5 & -3 \\
			0 & 0 & 0 & -2 & 5
		\end{vmatrix} \\
		&=5D_{n-1}+(-1)^{1+2}(-2)(-3)
		\begin{vmatrix}
			5 & -3 & 0 & 0 \\
			\vdots & \ddots & \ddots & 0 \\
			0 & -2 & 5 & -3 \\
			0 & 0 & -2 & 5
		\end{vmatrix} \\
		&=5D_{n-1}-6D_{n-2}\qedhere
	\end{align*}
\end{proof}
\begin{theorem}
	证明:
	\begin{equation*}
		D_n=
		\begin{vmatrix}
			\cos\alpha & 1 & 0 & 0 & 0 & 0 \\
			1 & 2\cos\alpha & 1 & 0 & 0 & 0 \\
			0 & 1 & 2\cos\alpha & 1 & 0 & 0 \\
			0 & 0 & \ddots & \ddots & \ddots & 0 \\
			0 & 0 & 0 & 1 & 2\cos\alpha & 1 \\
			0 & 0 & 0 & 0 & 1 & 2\cos\alpha
		\end{vmatrix}
	\end{equation*}
\end{theorem}
\begin{proof}
	第$n$行展开,显然:
	\begin{align*}
		D_n&=2\cos\alpha D_{n-1}+1\times(-1)^{n+n-1}
		\begin{vmatrix}
			2\cos\alpha & 1 & 0 & 0 & 0 \\
			1 & 2\cos\alpha & 1 & 0 & 0 \\
			0 & \ddots & \ddots & \ddots & 0 \\
			0 & 0 & 1 & 2\cos\alpha & 0 \\
			0 & 0 & 0 & 1 & 1
		\end{vmatrix}
		\\
		&=2\cos\alpha D_{n-1}+(-1)^{2n-1}\cdot 1\cdot(-1)^{n-1+n-1}D_{n-2} \\
		&=2\cos\alpha D_{n-1}-D_{n-2}
	\end{align*}
	\begin{equation*}
		x^2-2\cos\alpha x+1=0
	\end{equation*}
\end{proof}

\subsubsection{Hessenberg}
\begin{theorem}
	计算$n$阶行列式:
	\begin{equation*}
		D_n=
		\begin{vmatrix}
			x & -1 & 0 & 0 & 0 \\
			0 & x & -1 & 0 & 0 \\
			0 & 0 & \ddots & \ddots & 0 \\
			0 & 0 & \cdots & x & -1 \\
			a_n & a_{n-1} & \cdots & a_2 & x+a_1
		\end{vmatrix}
	\end{equation*}
\end{theorem}
\begin{proof}
	按第一列展开:
	\begin{align*}
		D_n
		&=xD_{n-1}+a_n(-1)^{n+1}(-1)^{n-1} \\
		&=xD_{n-1}+a_n \\
		&=x(xD_{n-2}+a_{n-1})+a_n \\
		&=x^2D_{n-2}+xa_{n-1}+a_n \\
		&=x^2(xD_{n-3}+a_{n-2})+xa_{n-1}+a_n \\
		&=x^3D_{n-3}+x^2a_{n-2}+xa_{n-1}+a_n \\
		&=\cdots \\
		&=x^{n-1}D_1+\sum_{i=2}^{n}a_ix^{n-i} \\
		&=x^{n-1}(x+a_1)+\sum_{i=2}^{n}a_ix^{n-i} \\
		&=x^n+\sum_{i=1}^{n}a_ix^{n-i}\qedhere
	\end{align*}
\end{proof}
\begin{theorem}
	计算$n$阶行列式:
	\begin{equation*}
		D_n=
		\begin{vmatrix}
			2 & 0 & 0 & 0 & 2 \\
			-1 & 2 &  & 0 & 2 \\
			\vdots & \vdots & \ddots & \vdots & \vdots \\
			0 & 0 & \cdots & 2 & 2 \\
			0 & 0 & \cdots & -1 & 2
		\end{vmatrix}
	\end{equation*}
\end{theorem}
\begin{proof}
	按第一行展开得:
	\begin{equation*}
		D_n=2D_{n-1}+(-1)^{n+1}\cdot2\cdot(-1)^{n-1}
	\end{equation*}
\end{proof}
\subsubsection{行和、列和相同}
\begin{theorem}
	计算$n$阶行列式:
	\begin{equation*}
		D_n=
		\begin{vmatrix}
			x_1-m & x_2 & \cdots & x_n \\
			x_1 & x_2-m & \cdots & x_n \\
			\vdots & \vdots & \ddots & \vdots \\
			x_1 & x_2 & \cdots & x_n-m
		\end{vmatrix}
	\end{equation*}
\end{theorem}
\begin{proof}
	将所有列都加到第一列可得:
	\begin{align*}
		D_n
		&=
		\begin{vmatrix}
			\sum\limits_{i=1}^{n}x_i-m & x_2 & \cdots & x_n \\
			\sum\limits_{i=1}^{n}x_i-m & x_2-m & \cdots & x_n \\
			\vdots & \vdots & \ddots & \vdots \\
			\sum\limits_{i=1}^{n}x_i-m & x_2 & \cdots & x_n-m
		\end{vmatrix} \\
		&=
		\begin{vmatrix}
			\sum\limits_{i=1}^{n}x_i-m & x_2 & \cdots & x_n \\
			0 & -m & \cdots & 0 \\
			\vdots & \vdots & \ddots & \vdots \\
			0 & 0 & \cdots & -m
		\end{vmatrix} \\
		&=\left(\sum_{i=1}^{n}x_i-m\right)(-m)^{n-1}\qedhere
	\end{align*}
\end{proof}
\begin{theorem}
	设$x_1,x_2,x_3$时方程$x^3+px+q=0$的三个根,求行列式:
	\begin{equation*}
		\begin{vmatrix}
			x_1 & x_2 & x_3 \\
			x_3 & x_1 & x_2 \\
			x_2 & x_3 & x_1
		\end{vmatrix}
	\end{equation*}
\end{theorem}
\begin{proof}
	显然:
	\begin{equation*}
		x^3+px+q=(x-x_1)(x-x_2)(x-x_3)=x^3-(x_1+x_2+x_3)x^2+\cdots
	\end{equation*}
	所以$x_1+x_2+x_3=0$,把所有行都加到第一行,得到:
	\begin{equation*}
		\begin{vmatrix}
			x_1 & x_2 & x_3 \\
			x_3 & x_1 & x_2 \\
			x_2 & x_3 & x_1
		\end{vmatrix}
		=
		\begin{vmatrix}
			x_1+x_2+x_3 & x_1+x_2+x_3 & x_1+x_2+x_3 \\
			x_3 & x_1 & x_2 \\
			x_2 & x_3 & x_1
		\end{vmatrix}
		=0\qedhere
	\end{equation*}
\end{proof}

\subsubsection{升阶}
\begin{theorem}
	计算$n$阶行列式:
	\begin{equation*}
		D_n=
		\begin{vmatrix}
			a_1+b_1 & a_2 &\cdots & a_n \\
			a_1 & a_2+b_2 & \cdots & a_n \\
			\vdots & \vdots & \ddots & \vdots \\
			a_1 & a_2 & \cdots & a_n+b_n
		\end{vmatrix}
	\end{equation*}
	其中$b_1b_2\cdots b_n\ne0$。
\end{theorem}
\begin{proof}
	加边可得:
	\begin{align*}
		D_n&=
		\begin{vmatrix}
			1 & a_1 & a_2 & \cdots & a_n \\
			0 & a_1+b_1 & a_2 &\cdots & a_n \\
			0 & a_1 & a_2+b_2 & \cdots & a_n \\
			\vdots & \vdots & \vdots & \ddots & \vdots \\
			0 & a_1 & a_2 & \cdots & a_n+b_n
		\end{vmatrix} \\
		&=
		\begin{vmatrix}
			1 & a_1 & a_2 & \cdots & a_n \\
			-1 & b_1 & 0 &\cdots & 0 \\
			-1 & 0 & b_2 & \cdots & 0 \\
			\vdots & \vdots & \vdots & \ddots & \vdots \\
			-1 & 0 & 0 & \cdots & b_n
		\end{vmatrix} \\
		&=
		\begin{vmatrix}
			1+\sum\limits_{i=1}^{n}\frac{a_i}{b_i} & a_1 & a_2 & \cdots & a_n \\
			0 & b_1 & 0 &\cdots & 0 \\
			0 & 0 & b_2 & \cdots & 0 \\
			\vdots & \vdots & \vdots & \ddots & \vdots \\
			0 & 0 & 0 & \cdots & b_n
		\end{vmatrix} \\
		&=\left(1+\sum_{i=1}^{n}\frac{a_i}{b_i}\right)\prod_{i=1}^nb_i\qedhere
	\end{align*}
\end{proof}
\begin{theorem}
	计算$n$阶行列式:
	\begin{equation*}
		D_n=
		\begin{vmatrix}
			1+x_1^2 & x_2x_1 &\cdots & x_nx_1 \\
			x_1x_2 & 1+x_2^2 & \cdots & x_nx_2 \\
			\vdots & \vdots & \ddots & \vdots \\
			x_1x_n & x_2x_n & \cdots & 1+x_n^2
		\end{vmatrix}
	\end{equation*}
\end{theorem}
\begin{proof}
	加边$(1,x_1,x_2,\dots,x_n)$可得:
	\begin{align*}
		D_n&=
		\begin{vmatrix}
			1 & x_1 & x_2 & \cdots & x_n \\
			0 & 1+x_1^2 & x_2x_1 &\cdots & x_nx_1 \\
			0 & x_1x_2 & 1+x_2^2 & \cdots & x_nx_2 \\
			\vdots & \vdots & \vdots & \ddots & \vdots \\
			0 & x_1x_n & x_2x_n & \cdots & 1+x_n^2
		\end{vmatrix} \\
		&=
		\begin{vmatrix}
			1 & x_1 & x_2 & \cdots & x_n \\
			-x_1 & 1 & 0 &\cdots & 0 \\
			-x_2 & 0 & 1 & \cdots & 0 \\
			\vdots & \vdots & \vdots & \ddots & \vdots \\
			-x_n & 0 & 0 & \cdots & 1
		\end{vmatrix} \\
		&=
		\begin{vmatrix}
			1+\sum\limits_{i=1}^{n}x_i^2 & x_1 & x_2 & \cdots & x_n \\
			0 & 1 & 0 &\cdots & 0 \\
			0 & 0 & 1 & \cdots & 0 \\
			\vdots & \vdots & \vdots & \ddots & \vdots \\
			0 & 0 & 0 & \cdots & 1
		\end{vmatrix} \\
		&=1+\sum_{i=1}^{n}x_i^2
	\end{align*}
\end{proof}

\subsubsection{相邻行列元素相差$1$}
\begin{theorem}
	计算$n$阶行列式:
	\begin{align*}
		D_n=
		\begin{vmatrix}
			0 & 1 & 2 & \cdots & n-2 & n-1 \\
			1 & 0 & 1 & \cdots & n-3 & n-2 \\
			2 & 1 & 0 & \cdots & n-4 & n-3 \\
			\vdots & \vdots & \vdots & \ddots & \vdots & \vdots \\
			n-2 & n-3 & n-4 & \cdots & 0 & 1 \\
			n-1 & n-2 & n-3 & \cdots & 1 & 0
		\end{vmatrix}
	\end{align*}
\end{theorem}
\begin{proof}
	第$i$行减去第$i+1$行,化下三角。
\end{proof}
\begin{theorem}
	计算$n$阶行列式:
	\begin{equation*}
		D_n
		=
		\begin{vmatrix}
			1 & 2 & 3 & \cdots & n-1 & n \\
			2 & 3 & 4 & \cdots & n & 1 \\
			3 & 4 & 5 & \cdots & 1 & 2 \\
			\vdots & \vdots & \vdots & \ddots & \vdots & \vdots \\
			n-1 & n & 1 & \cdots & n-3 & n-2 \\
			n & 1 & 2 & \cdots & n-2 & n-1
		\end{vmatrix}
	\end{equation*}
\end{theorem}
\begin{proof}
	第$n$行减去第$n-1$行,依次从下往上。
\end{proof}

\subsubsection{Vandermonde行列式}
\begin{theorem}
	计算行列式:
	\begin{equation*}
		D=
		\begin{vmatrix}
			1 & 1 & 1 & 1 \\
			a & b & c & d \\
			a^2 & b^2 & c^2 & d^2 \\
			a^4 & b^4 & c^4 & d^4
		\end{vmatrix}
	\end{equation*}
\end{theorem}
\begin{proof}
	构造行列式:
	\begin{align*}
		D_1&=
		\begin{vmatrix}
			1 & 1 & 1 & 1 & 1\\
			a & b & c & d & x\\
			a^2 & b^2 & c^2 & d^2 & x^2 \\
			a^3 & b^3 & c^3 & d^3 & x^3 \\
			a^4 & b^4 & c^4 & d^4 & x^4
		\end{vmatrix} \\
		&=A_{15}+xA_{25}+x^2A_{35}+x^3A_{45}+x^4A_{55} \\
		&=(b-a)(c-a)(d-a)(x-a)(c-b)(d-b)(x-b)(d-c)(x-c)(x-d) \\
		&=(x-a)(x-b)(x-c)(x-d)(b-a)(c-a)(d-a)(c-b)(d-b)(d-c)
	\end{align*}
	而$A_{45}=(-1)^{4+5}D=-D$,找$x^3$的系数即可。
\end{proof}

\subsubsection{其它}
\begin{theorem}
	设$\alpha_1,\alpha_2,\alpha_3$都是$3$维列向量,记$A=(\alpha_1,\alpha_2,\alpha_3)$,$B=(\alpha_1+\alpha_2+\alpha_3,\alpha_1+2\alpha_2+4\alpha_3,\alpha_1+3\alpha_2+9\alpha_3)$,已知$|A|=1$,求$|B|$。
\end{theorem}
\begin{proof}
	\begin{gather*}
		B=(\alpha_1,\alpha_2,\alpha_3)
		\begin{pmatrix}
			1 & 1 & 1 \\
			1 & 2 & 4 \\
			1 & 3 & 9
		\end{pmatrix} \\
		|B|=|A||C|\qedhere
	\end{gather*}
\end{proof}
\begin{theorem}
	设$3$阶矩阵$A$的伴随矩阵为$A^*$,且$|A|=\frac{1}{2}$,求$|(3A)^{-1}-2A^*|$。
\end{theorem}
\begin{proof}
	伴随矩阵化为$A^{-1}$。
\end{proof}
\begin{theorem}
	设$A,B$为$n$阶矩阵,$|A|=2,|B|=-3$,求$|A^{-1}B^*-A^*B^{-1}|$。
\end{theorem}
\begin{proof}
	伴随矩阵化逆矩阵。
\end{proof}
\begin{theorem}
	设$A,B$分别为$3$阶矩阵与$5$阶矩阵,$|A|=2,\;|B|=3$,令:
	\begin{equation*}
		C=
		\begin{pmatrix}
			\mathbf{0} & (3A)^* \\
			(2B)^{-1} & \mathbf{0}
		\end{pmatrix}
	\end{equation*}
	求$|C|$。
\end{theorem}
\begin{proof}
	伴随矩阵化逆矩阵。
\end{proof}
\begin{theorem}
	设$A$为三阶矩阵,特征值为$-1,0,1$。令$B=A^3-2A^2+E$,求$|B|$和$|B+E|$。
\end{theorem}
\begin{proof}
	$B$的特征值为$\lambda^3-2\lambda^2+1$,$B+E$的特征值为$\lambda^3-2\lambda^2+2$。
\end{proof}
\begin{theorem}
	设$A$为三阶矩阵,$|A-E|=|A+2E|=|2A+3E|=0$,求$|A^*-3E|$。
\end{theorem}
\begin{proof}
	求出$A^*$与特征值之间的关系,进行转化。求$A$的特征值。
\end{proof}
\begin{theorem}
	设$A,B$为四阶矩阵且二者相似,如果$B^*$的特征值为$1,-1,2,4$,求$|A^*|$。
\end{theorem}
\begin{proof}
	\begin{gather*}
		|B^*|=-8=|B|^3,\;|B|=-2=|A|,\;|A^*|=|A|^3=-8\qedhere
	\end{gather*}
\end{proof}
\begin{theorem}
	设$A$为$n$阶实对称矩阵,$A^2+2A=\mathbf{0}$,$\operatorname{rank}(A)=k$,求$|A+3E|$。
\end{theorem}
\begin{proof}
	由$A^2+2A=\mathbf{0}$可知$A$的特征值为$0$或$-2$。$A$的相似标准形对角线上有$k$个$-2$、$n-2$个$0$。
\end{proof}
\begin{theorem}
	矩阵$A=
	\begin{pmatrix}
		2 & 1 & 0 \\
		1 & 2 & 0 \\
		0 & 0 & 1
	\end{pmatrix}$,矩阵$B$满足$ABA^*=2BA^*+E$,求$|B|$。
\end{theorem}
\begin{proof}
	两边同右乘$A$消去$A^*$,可得:
	\begin{equation*}
		3AB=6B+A,\;3(A-2E)B=A\qedhere
	\end{equation*}
\end{proof}
\begin{theorem}
	$\alpha$是一个单位列向量,$A=E-\alpha\alpha^T$,求$|A|$。
\end{theorem}
\begin{proof}
	$A\alpha=(E-\alpha\alpha^T)\alpha=\alpha-\alpha\alpha^T\alpha=\alpha-\alpha=\mathbf{0}$,$|A|=0$。
\end{proof}
\begin{theorem}
	设$A,B$为$n$阶矩阵,$A^2=E,\;B^2=E,\;|A|+|B|=0$,求$|A+B|$。
\end{theorem}
\begin{proof}
	由条件可知$|A|,|B|$二者中一个为$1$,一个为$-1$。
	\begin{equation*}
		|A+B|=|AE+EB|=|AB^2+A^2B|=|A(B+A)B|=|A||A+B||B|=-|A+B|\qedhere
	\end{equation*}
\end{proof}

\subsubsection{行列式乘法及其应用}
\begin{theorem}
	计算行列式:
	\begin{equation*}
		D=
		\begin{vmatrix}
			a & b & c & d \\
			-b & a & -d  & c \\
			-c & d & a & -b \\
			-d & -c & b & a
		\end{vmatrix}
	\end{equation*}
\end{theorem}
\begin{proof}
	\begin{align*}
		D^2&=
		\begin{vmatrix}
			a & b & c & d \\
			-b & a & -d  & c \\
			-c & d & a & -b \\
			-d & -c & b & a
		\end{vmatrix}
		\begin{vmatrix}
			a & -b & -c & -d \\
			b & a & d  & -c \\
			c & -d & a & b \\
			d & c & -b & a
		\end{vmatrix} \\
		&=
		\begin{vmatrix}
			\operatorname{diag}\{a^2+b^2+c^2+d^2,a^2+b^2+c^2+d^2,a^2+b^2+c^2+d^2,a^2+b^2+c^2+d^2\}
		\end{vmatrix} \\
		&=(a^2+b^2+c^2+d^2)^4
	\end{align*}
	所以$|D|=\pm(a^2+b^2+c^2+d^2)^2$。根据行列式的定义,$|D|$中$a^4$系数为$1$,所以$|D|=(a^2+b^2+c^2+d^2)^2$。
\end{proof}
\begin{theorem}
	证明:
	\begin{equation*}
		D=
		\begin{vmatrix}
			ax+by & ay+bz & az+bx \\
			ay+bz & az+bx & ax+by \\
			az+bx & ax+by & ay+bz
		\end{vmatrix}
		=(a^3+b^3)
		\begin{vmatrix}
			x & y & z \\
			y & z & x \\
			z & x & y
		\end{vmatrix}
	\end{equation*}
\end{theorem}
\begin{proof}
	显然:
	\begin{equation*}
		D=\left|
		\begin{pmatrix}
			a & b & 0 \\
			0 & a & b \\
			b & 0 & a
		\end{pmatrix}
		\begin{pmatrix}
			x & y & z \\
			y & z & x \\
			z & x & y
		\end{pmatrix}
		\right|
		=(a^3+b^3)
		\begin{vmatrix}
			x & y & z \\
			y & z & x \\
			z & x & y
		\end{vmatrix}\qedhere
	\end{equation*}
\end{proof}

\subsubsection{求解行列式方程}
\begin{theorem}
	求方程:
	\begin{equation*}
		\begin{vmatrix}
			1 & 1 & 1 & 1 \\
			1 & 2 & 4 & 8 \\
			1 & -2 & 4 & -8 \\
			1 & x & x^2 & x^3
		\end{vmatrix}=0
	\end{equation*}
	的根。
\end{theorem}
\begin{proof}
	Vandermonde行列式,显然根为$1,2,-2$(三次多项式最多三个根)。
\end{proof}
\begin{theorem}
	解方程:
	\begin{equation*}
		\begin{vmatrix}
			x & a_1 & a_2 & \cdots & a_{n-1} & 1 \\
			a_1 & x & a_2 & \cdots & a_{n-1} & 1 \\
			a_1 & a_2 & x & \cdots & a_{n-1} & 1 \\
			\vdots & \vdots & \vdots & \ddots & \vdots & \vdots \\
			a_1 & a_2 & a_3 & \cdots & x & 1 \\
			a_1 & a_2 & a_3 & \cdots & a_n & 1
		\end{vmatrix}=0
	\end{equation*}
\end{theorem}
\begin{proof}
	观察法:$x=a_1,a_2,\dots,a_n$
\end{proof}
\begin{theorem}
	计算行列式:
	\begin{equation*}
		D_n=
		\begin{vmatrix}
			x & y & \cdots & y & y \\
			z & x & \cdots & y & y \\
			\vdots & \vdots & \ddots & \vdots & \vdots \\
			z & z & \cdots & x & y \\
			z & z & \cdots & z & x
		\end{vmatrix}
	\end{equation*}
\end{theorem}
\begin{proof}
	分类讨论,当$y=z$时化为行和相同的行列式。当$y\ne z$时:
	\begin{align*}
		D_n&=
		\begin{vmatrix}
			x & y & \cdots & y & 0 \\
			z & x & \cdots & y & 0 \\
			\vdots & \vdots & \ddots & \vdots & \vdots \\
			z & z & \cdots & x & 0 \\
			z & z & \cdots & z & x-y 
		\end{vmatrix}
		+
		\begin{vmatrix}
			x & y & \cdots & y & y \\
			z & x & \cdots & y & y \\
			\vdots & \vdots & \ddots & \vdots & \vdots \\
			z & z & \cdots & x & y \\
			z & z & \cdots & z & y
		\end{vmatrix} \\
		&=(x-y)D_{n-1}+
		\begin{vmatrix}
			x-z & y-x & \cdots & 0 & 0 \\
			0 & x-z & \cdots & 0 & 0 \\
			\vdots & \vdots & \ddots & \vdots & \vdots \\
			0 & 0 & \cdots & x-z & 0 \\
			z & z & \cdots & z & y
		\end{vmatrix} \\
		&=(x-y)D_{n-1}+y(x-z)^{n-1}
	\end{align*}
	而由$D_n$的转置可以得到:
	\begin{equation*}
		D_n=(x-z)D_{n-1}+z(x-y)^{n-1}
	\end{equation*}
	于是:
	\begin{equation*}
		\begin{cases}
			D_n-(x-y)D_{n-1}=y(x-z)^{n-1} \\
			D_n-(x-z)D_{n-1}=z(x-y)^{n-1}
		\end{cases}
	\end{equation*}
	由Cramer法则可以解得:
	\begin{equation*}
		D_n=\frac{z(x-y)^n-y(x-z)^n}{z-y}
	\end{equation*}
\end{proof}
\begin{theorem}
	计算行列式:
	\begin{equation*}
		D_n=
		\begin{vmatrix}
			x_1 & \alpha & \cdots & \alpha & \alpha \\
			\beta & x_2 & \cdots & \alpha & \alpha \\
			\vdots & \vdots & \ddots & \vdots & \vdots \\
			\beta & \beta & \cdots & x_{n-1} & \alpha \\
			\beta & \beta & \cdots & \beta & x_n
		\end{vmatrix}
	\end{equation*}
\end{theorem}
\begin{proof}
	当$\alpha\ne\beta$时的情况与上一题可以一样。当$\alpha=\beta$时有:
	\begin{align*}
		D_n&=
		\begin{vmatrix}
			x_1 & \alpha & \cdots & \alpha & \alpha \\
			\alpha & x_2 & \cdots & \alpha & \alpha \\
			\vdots & \vdots & \ddots & \vdots & \vdots \\
			\alpha & \alpha & \cdots & x_{n-1} & \alpha \\
			\alpha & \alpha & \cdots & \alpha & x_n
		\end{vmatrix} \\
		&=
		\begin{vmatrix}
			1 & 1 & 1 & 1 & 1 & 1 \\
			0 & x_1 & \alpha & \cdots & \alpha & \alpha \\
			0 & \alpha & x_2 & \cdots & \alpha & \alpha \\
			\vdots & \vdots & \vdots & \ddots & \vdots & \vdots \\
			0 & \alpha & \alpha & \cdots & x_{n-1} & \alpha \\
			0 & \alpha & \alpha & \cdots & \alpha & x_n
		\end{vmatrix} \\
		&=
		\begin{vmatrix}
			1 & 1 & 1 & \cdots & 1 & 1 \\
			-\alpha & x_1-\alpha & 0 & \cdots & 0 & 0 \\
			-\alpha & 0 & x_2 & \cdots & 0 & 0 \\
			\vdots & \vdots & \vdots & \ddots & \vdots & \vdots \\
			-\alpha & 0 & 0 & \cdots & x_{n-1}-\alpha & 0 \\
			-\alpha & 0 & 0 & \cdots & 0 & x_n-\alpha
		\end{vmatrix}\qedhere
	\end{align*}
\end{proof}
\begin{theorem}
	计算行列式:
	\begin{equation*}
		D_n=
		\begin{vmatrix}
			0 & a_1+a_2 & \cdots & a_1+a_n \\
			a_2+a_1 & 0 & \cdots & a_2+a_n \\
			\vdots & \vdots & \ddots & \vdots \\
			a_n+a_1 & a_n+a_2 & \cdots & 0
		\end{vmatrix}
	\end{equation*}
\end{theorem}
\begin{proof}
	加边:
	\begin{align*}
		D_n&=
		\begin{vmatrix}
			1 & a_1 & a_2 & \cdots & a_n \\
			0 & 0 & a_1+a_2 & \cdots & a_1+a_n \\
			0 & a_2+a_1 & 0 & \cdots & a_2+a_n \\
			\vdots & \vdots & \vdots & \ddots & \vdots \\
			0 & a_n+a_1 & a_n+a_2 & \cdots & 0
		\end{vmatrix} \\
		&=
		\begin{vmatrix}
			1 & a_1 & a_2 & \cdots & a_n \\
			-1 & -a_1 & a_1 & \cdots & a_1 \\
			-1 & a_2 & -a_2 & \cdots & a_2 \\
			\vdots & \vdots & \vdots & \ddots & \vdots \\
			-1 & a_n & a_n & \cdots & -a_n
		\end{vmatrix} \\
		&=
		\begin{vmatrix}
			1 & 0 & 0 & 0 & \cdots & 0 \\
			0 & 1 & a_1 & a_2 & \cdots & a_n \\
			a_1 & -1 & -a_1 & a_1 & \cdots & a_1 \\
			a_2 & -1 & a_2 & -a_2 & \cdots & a_2 \\
			\vdots & \vdots & \vdots & \vdots & \ddots & \vdots \\
			a_n & -1 & a_n & a_n & \cdots & -a_n
		\end{vmatrix} \\
		&=
		\begin{vmatrix}
			1 & 0 & -1 & -1 & \cdots & -1 \\
			0 & 1 & a_1 & a_2 & \cdots & a_n \\
			a_1 & -1 & -2a_1 & 0 & \cdots & 0 \\
			a_2 & -1 & 0 & -2a_2 & \cdots & 0 \\
			\vdots & \vdots & \vdots & \vdots & \ddots & \vdots \\
			a_n & -1 & 0 & 0 & \cdots & -2a_n
		\end{vmatrix} \\
	\end{align*}
	把前两列的第三行到第$n+2$行全部化为0。
\end{proof}
\begin{theorem}
	求行列式:
	\begin{equation*}
		D_n=
		\begin{vmatrix}
			1 & -1 & \cdots & -1 & -1 \\
			1 & 1 & \cdots & -1 & -1 \\
			\vdots & \vdots & \ddots & \vdots & \vdots \\
			1 & 1 & \cdots & 1 & -1 \\
			1 & 1 & \cdots & 1 & 1
		\end{vmatrix}
	\end{equation*}
	的展开式中正项的数目。
\end{theorem}
\begin{proof}
	设正项总数为$x$,则负项总数为$n!-x$。考虑到展开式中每一项不是$1$就是$-1$,所以:
	\begin{equation*}
		D=x-(n!-x)=2x-n!
	\end{equation*}
	由之前的公式求出$D$,代入即可解得$x$。
\end{proof}
\begin{theorem}
	计算行列式:
	\begin{equation*}
		D=
		\begin{vmatrix}
			1-ax_1y_1 & -ax_1y_2 & \cdots & -ax_1y_n \\
			-ax_2y_1 & 1-ax_2y_2 & \cdots & -ax_2y_n \\
			\vdots & \vdots & \ddots & \vdots \\
			-ax_ny_1 & -ax_ny_2 & \cdots & 1-ax_ny_n
		\end{vmatrix}
	\end{equation*}
\end{theorem}
\begin{proof}
	加边:
	\begin{align*}
		D&=
		\begin{vmatrix}
			1 & y_1 & y_2 \cdots & y_n \\
			0 & 1-ax_1y_1 & -ax_1y_2 & \cdots & -ax_1y_n \\
			0 & -ax_2y_1 & 1-ax_2y_2 & \cdots & -ax_2y_n \\
			\vdots & \vdots & \vdots & \ddots & \vdots \\
			0 & -ax_ny_1 & -ax_ny_2 & \cdots & 1-ax_ny_n
		\end{vmatrix}\qedhere
	\end{align*}
\end{proof}