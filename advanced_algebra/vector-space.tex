\chapter{向量空间}

\begin{theorem}
	设$n$维列向量组$\seq{\alpha}{m}(m<n)$线性无关,则$n$维列向量组$\seq{\beta}{m}$线性无关的充分必要条件为矩阵$A=(\seq{\alpha}{m})$与矩阵$B=(\seq{\beta}{m})$等价。
\end{theorem}
\begin{proof}
	同形矩阵等价的充分必要条件为秩相同。
\end{proof}

\begin{theorem}
	设$\seq{\alpha}{t}$是齐次线性方程组$Ax=0$的一个基础解系,向量$\beta$不是方程组$Ax=0$的解,证明$\beta,\beta+\alpha_1,\dots,\beta+\alpha_t$线性无关。
\end{theorem}
\begin{proof}
	若存在一组不全为零的$\seq{a}{t}$使得:
	\begin{equation*}
		a_1\beta+a_2(\beta+\alpha_1)+\cdots+a_t(\beta+\alpha_t)=\mathbf{0}
	\end{equation*}
	则有:
	\begin{equation*}
		a_1\beta+a_2\beta+a_2\alpha_1+\cdots+a_t\beta+a_t\alpha_t=\mathbf{0}
	\end{equation*}
	两边同时乘$A$可得:
	\begin{equation*}
		a_1\beta+a_2\beta+\cdots+a_t\beta=\mathbf{0}
	\end{equation*}
	作差即可得:
	\begin{equation*}
		a_1\alpha_1+a_2\alpha_2+\cdots+a_t\alpha_t=\mathbf{0}
	\end{equation*}
	因为$\seq{\alpha}{t}$为基础解系,线性无关,所以$a_1=a_2=\cdots=a_t=0$,于是$\beta,\beta+\alpha_1,\dots,\beta+\alpha_t$线性无关。
\end{proof}

\begin{theorem}
	设$A,B$是两个非零向量且$AB=\mathbf{0}$,则$A$的列向量组线性相关,$B$的行向量组线性相关。
\end{theorem}
\begin{proof}
	设$A$是$m\times n$矩阵,$B$是$n\times p$矩阵,则:
	\begin{equation*}
		\operatorname{rank}(A)+\operatorname{rank}(B)\leqslant n
	\end{equation*}
	因为$A,B$非零,所以:
	\begin{equation*}
		1\leqslant\operatorname{rank}(A)<n,\;1\leqslant\operatorname{rank}(B)<n
	\end{equation*}
	结果显然。
\end{proof}
\begin{theorem}
	设$n$阶矩阵$A$的伴随矩阵$A^*\ne\mathbf{0}$,若$\seq{\varepsilon}{4}$是$Ax=b$的互不相等的解,则对应的齐次线性方程组$Ax=\mathbf{0}$的基础解系仅含一个非零解向量。
\end{theorem}
\begin{proof}
	因为$A^*\ne\mathbf{0}$,则$\operatorname{rank}(A)\geqslant1$。由题意$\operatorname{rank}(A)=\operatorname{rank}(\overline{A})<n$。由伴随矩阵秩与矩阵秩之间的关系可得$\operatorname{rank}(A^*)=1,\operatorname{rank}(A)=n-1$。
\end{proof}


矩阵形式如何求解基础解系

\begin{theorem}
	求以$\beta_1=(1,-1,1,0)^T,\;\beta_2=(1,1,0,1)^T,\;\beta_3=(2,0,1,1)^T$为解向量的齐次线性方程组。
\end{theorem}
\begin{proof}
	令$B=(\beta_1,\beta_2,\beta_3)$,则有:
	\begin{equation*}
		AB=\mathbf{0},\;B^TA^T=\mathbf{0}
	\end{equation*}
	即求$B^Tx=\mathbf{0}$。$B^T$是已知的,求出的$x$即为$A$的行向量。
\end{proof}

\begin{theorem}
	已知非齐次线性方程组:
	\begin{equation*}
		\begin{cases}
			x_1+x_2+x_3+x_4=-1 \\
			4x_1+3x_2+5x_3-x_4=-1 \\
			ax_1+x_2+3x_3+bx_4=1
		\end{cases}
	\end{equation*}
	有$3$个线性无关的解。
	\begin{enumerate}
		\item 证明方程组系数矩阵$A$的秩为$2$;
		\item 求$a,b$的秩和方程组的通解。
	\end{enumerate}
\end{theorem}
\begin{proof}
	(1)设$\alpha_1,\alpha_2,\alpha_3$是$3$个线性无关的解,于是$\alpha_3-\alpha_1,\alpha_3-\alpha_2$是$Ax=0$线性无关的解,于是$4-\operatorname{rank}(A)\geqslant2$,即$\operatorname{rank}(A)\leqslant2$。$A$有不为$0$的二阶子式,所以$\operatorname{rank}(A)=2$。\par
	(2)化简行阶梯形矩阵。
\end{proof}
\begin{theorem}
	设$A=
	\begin{pmatrix}
		\lambda & 1 & 1 \\
		0 & \lambda-1 & 0 \\
		1 & 1 & \lambda
	\end{pmatrix},\;
	b=
	\begin{pmatrix}
		a \\
		1 \\
		1
	\end{pmatrix}$,已知$Ax=b$存在两个不同的解,
	\begin{enumerate}
		\item 求$\lambda,a$;
		\item 求$Ax=b$的通解
	\end{enumerate}
\end{theorem}
\begin{proof}
	直接化简增广矩阵。
\end{proof}
\begin{theorem}
	设$A=
	\begin{pmatrix}
		1 & a \\
		1 & 0
	\end{pmatrix},\;
	B=
	\begin{pmatrix}
		0 & 1 \\
		1 & b
	\end{pmatrix}$,当$a,b$为多少时,存在矩阵$C$使得$AC-CA=B$,求所有矩阵$C$。
\end{theorem}
\begin{proof}
	设$C=
	\begin{pmatrix}
		x_1 & x_2 \\
		x_3 & x_4
	\end{pmatrix}$,化为线性方程组。
\end{proof}
\begin{theorem}
	设矩阵$A=
	\begin{pmatrix}
		1 & -1 & -1 \\
		2 & a & 1 \\
		-1 & 1 & a
	\end{pmatrix},\;
	B=
	\begin{pmatrix}
		2 & 2 \\
		1 & a \\
		-a-1 & -2
	\end{pmatrix}$,当$a$为何值时,$AX=B$无解、有唯一解、有无穷多解。
\end{theorem}
\begin{proof}
	增广矩阵化简。$A|B$。
\end{proof}
\begin{theorem}
	设$A=(\alpha_1,\alpha_2,\alpha_3,\alpha_4)$是四阶矩阵,若$(1,0,1,0)^T$是$Ax=0$的一个基础解系,则$A^*x=0$的基础解系为$\alpha_2,\alpha_3,\alpha_4$。
\end{theorem}
\begin{proof}
	$\operatorname{rank}(A)=3,|A|=0,\operatorname{rank}(A^*)=1$,所以基础解系有$3$个元素。$AA^*=\mathbf{0}$,$A$的列是解,但是$\alpha_1,\alpha_3$线性相关,所以是$\alpha_2,\alpha_3,\alpha_4$。
\end{proof}
\begin{theorem}
	已知三阶矩阵$A$的第一行为$(a,b,c)$,$a,b,c$不全为$0$,矩阵$B=\begin{pmatrix}
		1 & 2 & 3 \\
		2 & 4 & 6 \\
		3 & 6 & k
	\end{pmatrix}$,$k$为常数且$AB=\mathbf{0}$,求$Ax=\mathbf{0}$的通解。
\end{theorem}
\begin{proof}
	由$AB=\mathbf{0}$可知$\operatorname{rank}(A)+\operatorname{rank}(B)\leqslant3$。因为$A\ne\mathbf{0},B\ne\mathbf{0}$,所以:
	\begin{equation*}
		1\leqslant\operatorname{rank}(A),\operatorname{rank}(B)\leqslant2
	\end{equation*}\par
	若$\operatorname{rank}(A)=2$,则$\operatorname{rank}(B)=1$,$k=9$。\par
	若$\operatorname{rank}(A)=1$,此时$Ax=\mathbf{0}$的同解方程组为$ax_1+bx_2+cx_3=0$,设$a\ne0$,可求出通解。\par 
	\textbf{方法二:讨论$k$是否为$9$}。
\end{proof}

\begin{theorem}
	设$A$是$m\times n$矩阵,$\beta=\seq{b}{n}$,证明:$Ax=\mathbf{0}$的解满足$\beta x=0$的充分必要条件为$\beta$可由$A$的行向量组线性表示。
\end{theorem}

\begin{theorem}
	设$4$元齐次线性方程组一为:
	\begin{equation*}
		\begin{cases}
			x_1+x_2=0 \\
			x_3-x_4=0
		\end{cases}
	\end{equation*}
	齐次线性方程组二的通解为$k_1(0,1,1,0)^T+k_2(-1,2,2,1)^T$。一二是否有非零公共解,若有,求出所有。
\end{theorem}

\begin{theorem}
	已知齐次线性方程组:
	\begin{equation*}
		\begin{cases}
			x_1+2x_2+3x_3=0 \\
			2x_1+3x_2+5x_3=0 \\
			x_1+x_2+ax_3=0
		\end{cases},\quad
		\begin{cases}
			x_1+bx_2+cx_3=0 \\
			2x_1+b^2x_2+(c+1)x_3=0
		\end{cases}
	\end{equation*}
	同解,求$a,b,c$的值。
\end{theorem}
\begin{proof}
	第二个方程组一定有无穷多个解,所以第一个方程组行列式为$0$,解得$a=2$。求得方程组一的解(不带系数)代入方程组二,解得两种情况,要验证方程组二的解也是方程组一的解。
\end{proof}

\begin{theorem}
	设线性方程组:
	\begin{equation*}
		\begin{cases}
			x_1+x_2+x_3=0 \\
			x_1+2x_2+ax_3=0 \\
			x_1+4x_2+a^2x_3=0
		\end{cases}
	\end{equation*}
	与方程$x_1+2x_2+x_3=a-1$有公共解,求$a$与所有公共解。
\end{theorem}
\begin{proof}
	将它们联立,进行增广矩阵的化简。
\end{proof}

\begin{theorem}
	设$n$元线性方程组$Ax=b$,其中:
	\begin{equation*}
		A=
		\begin{pmatrix}
			2a & 1 & \cdots & 0 & 0 \\
			a^2 & 2a & \cdots & 0 & 0 \\
			\vdots & \vdots & \ddots & \vdots & \vdots \\
			0 & 0 & \cdots & 2a & 1 \\
			0 & 0 & \cdots & a^2 & 2a
		\end{pmatrix},\;
		x=
		\begin{pmatrix}
			x_1 \\
			x_2 \\
			\vdots \\
			x_n
		\end{pmatrix},\;
		b=
		\begin{pmatrix}
			1 \\
			0 \\
			\vdots \\
			0
		\end{pmatrix}
	\end{equation*}
	\begin{enumerate}
		\item 证明$|A|=(n+1)a^n$;
		\item $a$为何值时,有唯一解,并求$x_1$;
		\item $a$为何值时,有无穷多解,求通解。
	\end{enumerate}
\end{theorem}
\begin{proof}
	(1)递推得到。(2)Cramer法则求解。
\end{proof}

\begin{theorem}
	设$A=\begin{pmatrix}
		3 & 2 & 2 \\
		2 & 3 & 2 \\
		2 & 2 & 3
	\end{pmatrix}$,$P=
	\begin{pmatrix}
	0 & 1 & 0 \\
	1 & 0 & 1 \\
	0 & 0 & 1
	\end{pmatrix}$,$B=P^{-1}A^*P$,求$B+2E$的特征向量与特征值。
\end{theorem}

\begin{theorem}
	4阶方阵满足$|3E+A|=0,\;AA^T=2E,\;|A|<0$,求$A^*$的一个特征值。
\end{theorem}
\begin{proof}
	$\dfrac{|A|}{\lambda}$
\end{proof}
\begin{theorem}
	$n$阶方阵$A$的各列元素之和都是1,求一个特征值。
\end{theorem}
\begin{proof}
	化为线性方程组有:
	\begin{equation*}
		A^T\begin{pmatrix}
			1 \\
			1 \\
			\cdots \\
			1
		\end{pmatrix}
		=\begin{pmatrix}
			1 \\
			1 \\
			\cdots \\
			1
		\end{pmatrix}
	\end{equation*}
	于是$\lambda=1$是$A^T$的特征值\info{A与A的转置特征值相同}。
\end{proof}

\begin{theorem}
	$A$为$n$阶方阵,$A\ne E$,且$\operatorname{rank}(A+3E)+\operatorname{rank}(A-E)=n$,求一个特征值。
\end{theorem}
\begin{proof}
	因为$A\ne E$,所以$\operatorname{rank}(A-E)>0$,所以$\operatorname{rank}(A+3E)<n,\;|A+3E|=0$,$-3$.
\end{proof}

\begin{theorem}
	设$A$为$3$阶实对称矩阵,$A^2+2A=\mathbf{0}$,$\operatorname{rank}(A)=2$。
	\begin{enumerate}
		\item 求$A$的所有特征值。
		\item $k$为何值时,$A+kE$为正定矩阵。
	\end{enumerate}
\end{theorem}
\begin{proof}
	(1)$\lambda^2+2\lambda=0$,$\lambda$只能为$0$或$-2$。\info{特征值多项式与矩阵多项式的关系}\par
	(2)先证明实对称,正定特征值都大于$0$,于是$k>2$。
\end{proof}

\begin{theorem}
	设$3$阶实对称矩阵$A$的特征值为$\lambda_1=1,\lambda_2=2,\lambda_3=-2$,$\alpha_1=\begin{pmatrix}
		1 \\
		-1 \\
		1
	\end{pmatrix}$是$A$属于$\lambda_1$的一个特征向量,记$B=A^5-4A^3+E$。
	\begin{enumerate}
		\item 验证$\alpha_1$是$B$的特征向量,求$B$所有特征值的特征向量。
		\item 求$B$。
	\end{enumerate}
\end{theorem}
\begin{proof}
	(1)可验证$A$的特征向量都是$B$的特征向量。\par
	(2)
\end{proof}

\begin{theorem}
	设$A$为$2$阶方阵,$\alpha_1,\alpha_2$为线性无关的$2$维列向量,$A\alpha_1=0,\;A\alpha_2=2\alpha_1+\alpha_2$,求$A$的非零特征值。
\end{theorem}
\begin{proof}
	两边加上$2A\alpha_1$可直接得出结论$1$。
\end{proof}

\begin{theorem}
	设矩阵$A=
	\begin{pmatrix}
		4 & 1 & -2 \\
		1 & 2 & a \\
		3 & 1 & -1 
	\end{pmatrix}$的一个特征向量为$\begin{pmatrix}
	1  \\
	1 \\
	2
	\end{pmatrix}$,求$a$。
\end{theorem}
\begin{proof}
	设对应的特征值为$\lambda$,代入求解。
\end{proof}

\begin{theorem}
	设$A=
	\begin{pmatrix}
		2 & 1 & 1 \\
		1 & 2 & 1 \\
		1 & 1 & a 
	\end{pmatrix}$可逆,$\alpha=\begin{pmatrix}
	1 \\
	b \\
	1
	\end{pmatrix}$是$A^*$的一个特征向量,$\lambda$是$\alpha$对应的特征值,求$a,b,\lambda$。
\end{theorem}
\begin{proof}
	通过$AA^*=|A|E$可得$A\alpha=\frac{|A|}{\lambda}\alpha$,代入求解即可得到结果。
\end{proof}

\begin{theorem}
	设$A=
	\begin{pmatrix}
		0 & 0 & 1 \\
		x & 1 & y \\
		1 & 0 & 0 
	\end{pmatrix}$有$3$个线性无关的特征向量,求$x,y$应满足的条件。
\end{theorem}
\begin{proof}
	可求出特征值为$1,-1$,其中$1$是二重根。
\end{proof}

\begin{theorem}
	设$A=
	\begin{pmatrix}
		1 & -3 & 3 \\
		3 & a & 3 \\
		6 & -6 & b
	\end{pmatrix}$有特征值$-2,4$。
	\begin{enumerate}
		\item 求$a,b$;
		\item $A$能否相似于对角矩阵
	\end{enumerate}
\end{theorem}
\begin{proof}
	(1)特征方程求解$a,b$。
\end{proof}

\subsection{由特征值和特征向量求矩阵}
\begin{theorem}
	已知$3$阶实对称矩阵$A$的特征值为$1,-1,0$,其中$1,0$的特征向量分别为$(1,a,1)^T,(a,a+1,1)$,求矩阵$A$。
\end{theorem}
\begin{proof}
	由特征向量的正交性求$a$,再由正交性求另一特征向量,写为线性方程组求基础解系。
\end{proof}
\begin{theorem}
	设$3$阶实对称矩阵$A$的秩为$2$,$\lambda_1=\lambda_2=6$是$A$的$2$重特征值,若$\alpha_1=(1,1,0)^T,\alpha_2=(2,1,1)^T,\lambda_3=(-1,2,-3)^T$都是$A$的属于$6$的特征向量。
	\begin{enumerate}
		\item 求另一特征值与它的特征向量;
		\item 求$A$。
	\end{enumerate}
\end{theorem}
\begin{proof}
	与上题类似,秩为$2$可求得另一特征值为$0$。
\end{proof}
\begin{theorem}
	设$A$为$3$阶矩阵,$\alpha_1,\alpha_2$为$A$的分别属于$-1,1$的特征向量,$A\alpha_3=\alpha_2+\alpha_3$。
	\begin{enumerate}
		\item 证明$\alpha_1,\alpha_2,\alpha_3$线性无关,
		\item 令$P=(\alpha_1,\alpha_2,\alpha_3)$,求$P^{-1}AP$。
	\end{enumerate}
\end{theorem}
\begin{proof}
	设$k_1\alpha_1+k_2\alpha_2+k_3\alpha_3=\mathbf{0}$,则:
	\begin{gather*}
		k_1A\alpha_1+k_2A\alpha_2+k_3A\alpha_3=\mathbf{0} \\
		-k_1\alpha_1+k_2\alpha_2+k_3\alpha_2+k_3\alpha_3=\mathbf{0} \\
		k_1\alpha_1+k_2\alpha_2+k_3\alpha_3-(-k_1\alpha_1+k_2\alpha_2+k_3\alpha_2+k_3\alpha_3)=\mathbf{0}
	\end{gather*}
\end{proof}
\begin{theorem}
	设$\lambda_1,\lambda_2$为矩阵$A$的两个不同的特征值,对应的特征向量分别为$\alpha_1,\alpha_2$,则$\alpha_1$与$A(\alpha_1+\alpha_2)$线性无关的充分必要条件为
\end{theorem}

\begin{theorem}
	设矩阵$A=
	\begin{pmatrix}
		1 & -1 & 1 \\
		x & 4 & y \\
		-3 & -3 & 5
	\end{pmatrix}$,已知$A$有$3$个线性无关的特征向量,$\lambda=2$是$A$的$2$重特征值,求可逆矩阵$P$使得$P^{-1}AP$为为对角矩阵。
\end{theorem}
\begin{proof}
	其它都很简单,注意到$A$可对角化并且$2$是二重特征值,所以几何重数也为$2$,由
	\begin{equation*}
		\dim[\operatorname{Ker}(2E-A)]=2
	\end{equation*}
	和秩零度定理可得出$\operatorname{rank}(2E-A)=1$,$x,y$可一次性求出。
\end{proof}
\begin{theorem}
	已知$3$阶方阵$A$与$3$维列向量$X$使得向量组$X,AX,A^2X$线性无关且满足$A^3X=3AX-2A^2X$。
	\begin{enumerate}
		\item 记$P=(X,AX.A^2X)$,求$3$阶方阵$B$使得$A=PBP^{-1}$;
		\item 计算$|A+E|$。
	\end{enumerate}
\end{theorem}
\begin{proof}
	(1)求解$AP=PB$。(2)相似矩阵同特征值。
\end{proof}
\begin{theorem}
	矩阵$A=
	\begin{pmatrix}
		2 & 2 & 0 \\
		8 & 2 & a \\
		0 & 0 & 6
	\end{pmatrix}$可对角化,求$a$的值与特征向量。
\end{theorem}
\begin{proof}
	可直接求得特征值为$6$和$-2$,其中$6$为二重根,于是$\operatorname{rank}(6E-A)=1$。
\end{proof}

\begin{theorem}
	矩阵$A=
	\begin{pmatrix}
		1 & 2 & -3 \\
		-1 & 4 & -3 \\
		1 & a & 5
	\end{pmatrix}$的特征方程有一个$2$重根,求$a$的值,讨论是否可以对角化。
\end{theorem}

\begin{theorem}
	$n$阶矩阵$A=
	\begin{pmatrix}
		1 & b & \cdots & b \\
		b & 1 & \cdots & b \\
		\vdots & \vdots & \ddots & \vdots \\
		b & b & \cdots & 1
	\end{pmatrix}$
	求$A$的特征值与特征向量。
\end{theorem}

\begin{theorem}
	设$A$为$3$阶矩阵,$\alpha_1,\alpha_2,\alpha_3$为线性无关的$3$维列向量且满足$A\alpha_1=\alpha_1+\alpha_2+\alpha_3$,$A\alpha_2=2\alpha_2+\alpha_3$,$A\alpha_3=2\alpha_2+3\alpha_3$。
	\begin{enumerate}
		\item 求矩阵$B$使得$A(\alpha_1,\alpha_2,\alpha_3)=(\alpha_1,\alpha_2,\alpha_3)B$;
		\item 求矩阵$A$的特征值与特征向量
	\end{enumerate}
\end{theorem}
\begin{proof}
	$A,B$相似
\end{proof}

\begin{theorem}
	设$\alpha,\beta$为$3$维列向量,若$\alpha\beta^T$相似于$
	\begin{pmatrix}
		2 & 0 & 0 \\
		0 & 0 & 0 \\
		0 & 0 & 0
	\end{pmatrix}$,求$\beta^T\alpha$。
\end{theorem}
\begin{proof}
	相似矩阵迹不变
\end{proof}

\begin{theorem}
	设$A$为$3$阶方阵,$P^{-1}AP=\operatorname{diag}\{1,1,2\}$,若$P=(\alpha_1,\alpha_2,\alpha_3)$,$Q=(\alpha_1+\alpha_2,\alpha_2,\alpha_3)$,求$Q^{-1}AQ$。
\end{theorem}
\begin{proof}
	$Q$用$P$表示。
\end{proof}

\begin{theorem}
	矩阵$\begin{pmatrix}
		1 & a & 1 \\
		a & b & a \\
		1 & a & 1
	\end{pmatrix}$与$\operatorname{diag}\{2,b,0\}$相似的充分必要条件为$a=0,b\in\mathbb{R}^{}$
\end{theorem}
\begin{proof}
	$|2E-A|=0$解得$a=0$,代入原矩阵,发现与$b$无关。
\end{proof}

\begin{theorem}
	$A$是奇数阶的正交矩阵,$|A|=1$,证明$A$有特征值$1$。
\end{theorem}
\begin{proof}
	注意到:
	\begin{align*}
		|E-A|=|A^TA-A|=|A^T-E||A|=|A-E|=(-1)^n|E-A|=-|E-A|
	\end{align*}
	所以$|E-A|=0$。
\end{proof}
\begin{theorem}
	设$3$阶实对称矩阵$A$的各行元素之和均为$3$,向量$\alpha_1=(-1,2,-1)^T,\;\alpha_2=(0,-1,1)^T$是线性方程组$Ax=\mathbf{0}$的两个解。
	\begin{enumerate}
		\item 求$A$的特征值与特征向量。
		\item 求正交矩阵$Q$和对角矩阵$\varLambda$使得$Q^TAQ=\varLambda$;
		\item 求$A$和$(A-\dfrac{3}{2}E)^6$。
	\end{enumerate}
\end{theorem}
\begin{proof}
	行和为$3$可求出特征值$3$,线性方程组的两个解可求出$0$是二重特征值,对应特征向量即为$\alpha_1,\alpha_2$。
\end{proof}

\begin{theorem}
	设$\alpha$是$3$维单位向量,求$E-\alpha\alpha^T$的秩。
\end{theorem}
\begin{proof}
	求出$\alpha\alpha^T=
	\begin{pmatrix}
		a^2 & ab & ac \\
		ba & b^2 & bc \\
		ca & cb & c^2
	\end{pmatrix}$,实对称一定可对角化,特征值的和等于迹等于$1$,特征值的积等于行列式等于$0$,求出$0$的特征空间维数为$2$。注意需要证明$E-\alpha\alpha^T$是实对称阵可对角化。
\end{proof}
\begin{theorem}
	证明全一矩阵$A$相似于$B=(\mathbf{0},D)$,其中$D$是从$1$到$n$的列向量。
\end{theorem}

\chapter{二次型}
\begin{theorem}
	设$A,B$都是$n$阶对称矩阵,如果对任意的$x\in\mathbb{R}^{n}$都有$x^TAx=x^TBx$,证明$A=B$。
\end{theorem}
\begin{proof}
	使用标准单位向量验证对角线上的元素相同,再利用两个标准单位向量的和去验证别的元素也相同。
\end{proof}
\begin{theorem}
	设二次型$f(x_1,x_2,x_3)=x_1^2-x_2^2+2ax_1x_3+4x_2x_3$的负惯性指数为$1$,求$a$的范围。
\end{theorem}
\begin{proof}
	(1)特征值\par
	(2)配方$f=(x_1+ax_3)^2-(x_2-2x_3)^2+(4-a^2)x_3^2$,$4-a^2\geqslant0$。
\end{proof}















