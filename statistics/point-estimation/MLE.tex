\section{极大似然估计}

\begin{definition}
	设$(X,\mathscr{A},\mathscr{P})$是可控参数结构,$\mu$为控制测度,$\Theta$是参数空间,$\mathbf{X}$为从总体$F$中抽取的简单样本,$f(\mathbf{X};\theta)$为样本$\mathbf{X}$的概率函数。把$f(\mathbf{X};\theta)$看作$\theta$的函数,称该函数为\gls{LikelihoodFunction},记为$L(\theta;\mathbf{X})$,称$\ell(\theta;\mathbf{X})=\ln L(\theta;\mathbf{X})$为\textbf{对数似然函数}。
\end{definition}
\begin{definition}
	设$(X,\mathscr{A},\mathscr{P})$是可控参数结构,$\mu$为控制测度,$\Theta$是参数空间,$\mathbf{X}$为从总体$F$中抽取的简单样本,$L(\theta;\mathbf{X})$是似然函数。若存在统计量$\delta(\mathbf{X})$使得:
	\begin{equation*}
		L[\delta(\mathbf{X});\mathbf{X}]=\max_{\theta\in\Theta}L(\theta;\mathbf{X})
	\end{equation*}
	或等价地使得:
	\begin{equation*}
		\ell[\delta(\mathbf{X});\mathbf{X}]=\max_{\theta\in\Theta}\ell(\theta;\mathbf{X})
	\end{equation*}
	则称$\delta(\mathbf{X})$是$\theta$的\gls{MLE}。称:
	\begin{equation*}
		\frac{\dif L(\theta;\mathbf{X})}{\dif\theta_i}=0
	\end{equation*}
	为\gls{LikelihoodEquation}。
\end{definition}
\begin{property}\label{prop:MLE}
	设$(X,\mathscr{A},\mathscr{P})$是可控参数结构,$\mu$为控制测度,$\Theta$是参数空间,$\mathbf{X}$为从总体$F$中抽取的简单样本。最大似然估计具有如下性质:
	\begin{enumerate}
		\item 最大似然估计具有不变性,即:若$\theta$的MLE为$\theta^*$,则$\theta$的任一可测函数$g(\theta)$的MLE为$g(\theta^*)$;
		\item 若$\mathscr{P}$为指数族,只要似然方程组的解是自然参数空间的内点,则解必唯一且是MLE;
		\item 若$T(\mathbf{X})$是$\theta$的充分统计量且$\theta$的MLE$\;\delta(\mathbf{X})$唯一存在,则$\delta(\mathbf{X})$必为$T(\mathbf{X})$的函数;
	\end{enumerate}
\end{property}