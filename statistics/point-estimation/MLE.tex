\section{极大似然估计}

\begin{definition}
	设$(X,\mathscr{A},\mathscr{P})$是可控参数结构,$\mu$为控制测度,$\Theta$是参数空间,$\mathbf{X}$为从总体$F$中抽取的简单样本,$f(\mathbf{X},\theta)$为样本$\mathbf{X}$的概率函数。把$f(\mathbf{X},\theta)$看作$\theta$的函数,称该函数为\gls{LikelihoodFunction},记为$L(\theta,\mathbf{X})$,称$\ell(\theta,\mathbf{X})=\ln L(\theta,\mathbf{X})$为\gls{Log-likelihoodFunction}。
\end{definition}
\begin{definition}
	设$(X,\mathscr{A},\mathscr{P})$是可控参数结构,$\mu$为控制测度,$\Theta$是$n$维参数空间,$\mathbf{X}$为从总体$F$中抽取的简单样本,$L(\theta,\mathbf{X})$是似然函数。若存在统计量$\delta(\mathbf{X})$使得:
	\begin{equation*}
		L[\delta(\mathbf{X}),\mathbf{X}]=\max_{\theta\in\Theta}L(\theta,\mathbf{X})
	\end{equation*}
	或等价地使得:
	\begin{equation*}
		\ell[\delta(\mathbf{X}),\mathbf{X}]=\max_{\theta\in\Theta}\ell(\theta,\mathbf{X})
	\end{equation*}
	则称$\delta(\mathbf{X})$是$\theta$的\gls{MLE}。称:
	\begin{equation*}
		\frac{\partial L(\theta,\mathbf{X})}{\partial\theta_i}=0,\; i=1,2,\dots,n
	\end{equation*}
	为\gls{LikelihoodEquation},称:
	\begin{equation*}
		\frac{\partial \ell(\theta,\mathbf{X})}{\partial\theta_i}=0,\; i=1,2,\dots,n
	\end{equation*}
	为\gls{Log-likelihoodEquation}。
\end{definition}
\begin{property}\label{prop:MLE}
	设$(X,\mathscr{A},\mathscr{P})$是可控参数结构,$\mu$为控制测度,$\Theta$是参数空间,$\mathbf{X}$为从总体$F$中抽取的简单样本。最大似然估计具有如下性质:
	\begin{enumerate}
		\item 最大似然估计具有不变性,即:若$\theta$的MLE为$\theta^{\star}$,则$\theta$的任一可测函数$g(\theta)$的MLE为$g(\theta^{\star})$;
		\item 若$\mathscr{P}$为满秩指数族,只要似然方程组的解是自然参数空间的内点,则解必唯一且是MLE;
		\item 若$T$是$\theta$的充分统计量且$\theta$的MLE$\;\delta$唯一存在,则$\delta$必为$T$的函数;
	\end{enumerate}
\end{property}
\begin{proof}
%	(2)设$\mathscr{P}$为$n$维指数族,$\mathbf{X}=(\seq{X}{m})$,则有:
%	\begin{gather*}
%		L(\eta,\mathbf{X})=\prod_{i=1}^{m}C(\eta)\exp\left[\eta^{\top}T(X_i)\right]h(X_i) \\
%		\ell(\eta,\mathbf{X})=\sum_{i=1}^{m}\ln\{C(\eta)\exp[\eta^{\top}T(X_i)]h(X_i)\}=m\ln C(\eta)+\sum_{i=1}^{m}\eta^{\top}T(X_i)+\sum_{i=1}^{m}\ln h(X_i) \\
%		\frac{\partial\ell(\eta,\mathbf{X})}{\partial\eta_j}=m\frac{1}{C(\eta)}\frac{\partial C(\eta)}{\partial\eta_j}+\sum_{i=1}^{m}T_j(X_i)
%	\end{gather*}
	(3)由\cref{theo:FactorizationTheorem}可得:
	\begin{equation*}
		\forall\;\theta\in\Theta,\;\frac{\dif P_{\theta}}{\dif\mu}(x)=g_{\theta}[T(x)]h(x),\;\text{a.e.于}(X,\mathscr{A},\mu)
	\end{equation*}
	于是有:
	\begin{equation*}
		\ell[\theta,\mathbf{X}]=\ln g_{\theta}[T(\mathbf{X})]+\ln h(\mathbf{X})
	\end{equation*}
	上式最大化等价于$g_{\theta}[T(\mathbf{X})]$最大化,即:
	\begin{equation*}
		\delta(\mathbf{X})=\underset{\theta\in\Theta}{\arg\max\ln g_{\theta}[T(\mathbf{X})]}
	\end{equation*}
	由$\delta(\mathbf{X})$的存在性与唯一性,$\delta(\mathbf{X})$必为$T(\mathbf{X})$的函数。
\end{proof}