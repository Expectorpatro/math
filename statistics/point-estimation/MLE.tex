\section{极大似然估计}

\begin{definition}
	设$f(\mathbf{X};\theta)$为样本$\mathbf{X}=(\seq{\mathbf{X}}{n})$的概率函数,$\mathcal{X}$是样本空间。当$\mathbf{X}$固定的时候,把$f(\mathbf{X};\theta)$看作$\theta$的函数,称该函数为\gls{LikelihoodFunction},记为:
	\begin{equation*}
		L(\theta;\mathbf{X})=f(\mathbf{X};\theta),\;\theta\in\Theta,\;\mathbf{X}\in\mathcal{X}
	\end{equation*}
	称$\ell(\theta;\mathbf{X})=\ln L(\theta;\mathbf{X})$为对数似然函数。
\end{definition}
\begin{definition}
	设$\mathbf{X}=(\seq{\mathbf{X}}{n})$是从参数分布族$\mathcal{F}=\{f(x;\theta):\theta\in\Theta\}$中抽取的简单样本,$L(\theta;\mathbf{X})$是似然函数。若存在统计量$\delta(\mathbf{X})$使得:
	\begin{equation*}
		L[\delta(\mathbf{X});\mathbf{X}]=\max_{\theta\in\Theta}L(\theta;\mathbf{X}),\;\forall\;\mathbf{X}\in\mathcal{X}
	\end{equation*}
	或等价地使得:
	\begin{equation*}
		\ell[\delta(\mathbf{X});\mathbf{X}]=\max_{\theta\in\Theta}\ell(\theta;\mathbf{X}),\;\forall\;\mathbf{X}\in\mathcal{X}
	\end{equation*}
	则称$\delta(\mathbf{X})$是$\theta$的\gls{MLE}。称:
	\begin{equation*}
		\frac{\dif L(\theta;\mathbf{X})}{\dif\theta_i}=0
	\end{equation*}
	为\gls{LikelihoodEquation}。
\end{definition}
\begin{property}
	最大似然估计具有如下性质:
	\begin{enumerate}
		\item 最大似然估计具有不变性,即:若$\theta$的MLE为$\theta^*$,则$\theta$的任一可测函数$g(\theta)$的MLE为$g(\theta^*)$;
		\item 若样本分布为指数族,只要似然方程组的解是自然参数空间的内点,则解必唯一且是MLE;
		\item 若$T(\mathbf{X})$是$\theta$的充分统计量且$\theta$的MLE$\;\delta(\mathbf{X})$唯一存在,则$\delta(\mathbf{X})$必为$T(\mathbf{X})$的函数;
	\end{enumerate}
\end{property}