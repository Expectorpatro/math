\section{矩估计}

\begin{definition}
	设$(X,\mathscr{A},\mathscr{P})$是参数统计结构,$\mathbf{X_n}=(\seq{X}{n})$为从总体$F$中抽取的简单样本。将:
	\begin{equation*}
		\mu_{nk}=\frac{1}{n}\sum_{i=1}^{n}X_i^k,\quad
		\nu_{nk}=\frac{1}{n}\sum_{i=1}^{n}(X_i-\mu_{n1})^k
	\end{equation*}
	分别称为样本$k$阶原点矩和样本$k$阶中心矩。
\end{definition}
\begin{theorem}\label{theo:SampleMoment}
	$\nu_{nk}$与原点矩$\mu_{nk}$之间存在如下关系:
	\begin{equation*}
		\nu_{nk}=\sum_{i=0}^{k}\binom{k}{i}\mu_{ni}(-\mu_{n1})^{k-i}
	\end{equation*}
\end{theorem}
\begin{proof}
	由样本中心矩的定义可得:
	\begin{align*}
		\nu_{nk}
		&=\frac{1}{n}\sum_{i=1}^{n}(X_i-\mu_{n1})^k
		=\frac{1}{n}\sum_{i=1}^{n}\sum_{j=0}^{k}\binom{k}{j}X_i^j(-\mu_{n1})^{k-j} \\
		&=\sum_{j=0}^{k}\binom{k}{j}\frac{1}{n}\sum_{i=1}^{n}X_i^j(-\mu_{n1})^{k-j}
		=\sum_{j=0}^{k}\binom{k}{j}\mu_{nj}(-\mu_{n1})^{k-j}\qedhere
	\end{align*}
\end{proof}
\begin{definition}
	设$(X,\mathscr{A},\mathscr{P})$是一个参数统计结构,$\Theta$是参数空间,$\mathbf{X}=(\seq{X}{n})$为从总体中抽取的简单样本,$g(\theta)$是定义在$\Theta$上的函数,它可以表示为总体分布的一些矩的函数,即:
	\begin{equation*}
		g(\theta)=f(\seq{\mu}{s},\seq{\nu}{t})
	\end{equation*}
	将$f$中的总体矩用样本矩代替,得到:
	\begin{equation*}
		\delta(\mathbf{X})=f(\mu_{n1},\mu_{n2},\dots,\mu_{ns},\nu_{n1},\nu_{n2},\dots,\nu_{nt})
	\end{equation*}
	则$\delta(\mathbf{X})$成为$g(\theta)$的一个点估计,称$\delta(\mathbf{X})$为$g(\theta)$的\gls{MomentEstimation},这种求矩估计量的方法称为\gls{MethodOfMoments}。
\end{definition}
\begin{property}\label{prop:MomentEstimation}
	矩估计具有如下性质:
	\begin{enumerate}
		\item 样本$k$阶原点矩$\mu_{nk}$是总体$k$阶原点矩$\mu_k$的无偏估计、强相合估计;
		\item $k=1$时,样本$k$阶中心矩$\nu_{nk}$是总体$k$阶中心矩$\nu_k$的无偏估计,对$k\geqslant2$,$\nu_{nk}$不是$\nu_k$的无偏估计。$\nu_{nk}$是$\nu_k$的强相合估计;
		\item 若待估量$g(\theta)$可以表示为一些总体原点矩的线性组合时,即:
		\begin{equation*}
			g(\theta)=\sum_{i=1}^{n}c_i\mu_{m_i},\;m_i\in\mathbb{N}^+
		\end{equation*}
		则其矩估计:
		\begin{equation*}
			\delta(\mathbf{X})=\sum_{i=1}^{n}c_i\mu_{nm_i},\;m_i\in\mathbb{N}^+
		\end{equation*}
		是$g(\theta)$的无偏估计;
		\item 若待估量$g(\theta)$满足:
		\begin{equation*}
			g(\theta)=f(\seq{\mu}{s},\seq{\nu}{t})
		\end{equation*}
		其中$f$是一个连续函数,则:
		\begin{equation*}
			\delta(\mathbf{X})=f(\mu_{n1},\mu_{n2},\dots,\mu_{ns},\nu_{n1},\nu_{n2},\dots,\nu_{nt})
		\end{equation*}
		是$g(\theta)$的强相合估计;
	\end{enumerate}
\end{property}
\begin{proof}
	(1)由\cref{prop:MeasurableIntegral}(5)可得:
	\begin{equation*}
		\operatorname{E}_{\theta}(\mu_{nk})=\operatorname{E}_{\theta}\left(\frac{1}{n}\sum_{i=1}^{n}X_i^k\right)=\frac{1}{n}\sum_{i=1}^{n}\operatorname{E}_{\theta}(X_i^k)=\frac{1}{n}\sum_{i=1}^{n}\mu_k=\mu_k
	\end{equation*}
	可得无偏性。\par
	由\cref{theo:StrongLawOfLargeNumbers}可得:
	\begin{equation*}
		\mu_{nk}=\frac{1}{n}\sum_{i=1}^{n}X_i^k\overset{\text{a.s.}}{\longrightarrow}\operatorname{E}(X_1^k)=\mu_k
	\end{equation*}\par
	(2)\info{不无偏还未证明}由\cref{theo:Moment}可得:
	\begin{equation*}
		\nu_k=\sum_{i=0}^{k}\binom{k}{i}\mu_i(-\mu_1)^{k-i}=f(\seq{\mu}{k})
	\end{equation*}
	所以$\nu_k$是$\seq{\mu}{k}$的连续函数。由\cref{theo:SampleMoment}、(1)和\cref{theo:ContinuousMappingTheorem}(1)可得:
	\begin{equation*}
		\nu_{nk}=f(\mu_{n1},\mu_{n2},\dots,\mu_{nk})\overset{\text{a.s.}}{\longrightarrow}f(\seq{\mu}{k})=\nu_{k}
	\end{equation*}\par
	(3)由(1)直接可得。\par
	(4)由\cref{theo:ContinuousMappingTheorem}(1)、(1)(2)直接可得\info{需要证明各分量几乎处处收敛则随机向量几乎处处收敛}。
\end{proof}