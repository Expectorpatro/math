\section{常见点估计及其性质}
\begin{theorem}\label{theo:xbar}
	设$(X,\mathscr{A},\mathscr{P})$是统计结构,$\mathbf{X}=(\seq{X}{n})$为从总体$F$中抽取的简单样本,$F$的期望为$\mu$,$\mathbf{Y}=(\seq{Y}{n})$为从总体$G$中抽取的简单样本,$\overline{X}=\dfrac{1}{n}\sum\limits_{i=1}^{n}X_i,\overline{Y}=\dfrac{1}{n}\sum\limits_{i=1}^{n}Y_i$,$a,b\in\mathbb{R}^{}$。有如下等式:
	\begin{gather*}
		\sum_{i=1}^{n}(X_i-a)(Y_i-b)=\sum_{i=1}^{n}(X_i-\overline{X})(Y_i-\overline{Y})+n(\overline{X}-a)(\overline{Y}-b) \\
		\sum_{i=1}^{n}(X_i-\overline{X})^2=\sum_{i=1}^{n}X_i^2-n\overline{X}^2=\sum_{i=1}^{n}(X_i-\mu)^2-n(\overline{X}-\mu)^2
	\end{gather*}
\end{theorem}
\begin{proof}
	注意到:
	\begin{gather*}
		\begin{aligned}
			&\sum_{i=1}^{n}(X_i-a)(Y_i-b)=\sum_{i=1}^{n}(X_i-\overline{X}+\overline{X}-a)(Y_i-\overline{Y}+\overline{Y}-b)  \\
			=&\sum_{i=1}^{n}[(X_i-\overline{X})(Y_i-\overline{Y})+(X_i-\overline{X})(\overline{Y}-b)+(\overline{X}-a)(Y_i-\overline{Y})+(\overline{X}-a)(\overline{Y}-b)] \\
			=&\sum_{i=1}^{n}(X_i-\overline{X})(Y_i-\overline{Y})+(\overline{Y}-b)\sum_{i=1}^{n}(X_i-\overline{X})+(\overline{X}-a)\sum_{i=1}^{n}(Y_i-\overline{Y})+n(\overline{X}-a)(\overline{Y}-b) \\
			=&\sum_{i=1}^{n}(X_i-\overline{X})(Y_i-\overline{Y})+n(\overline{X}-a)(\overline{Y}-b)
		\end{aligned} \\
		\begin{aligned}
			&\sum_{i=1}^{n}(X_i-\overline{X})=\sum_{i=1}^{n}(X_i^2-2X_i\overline{X}+\overline{X}^2)=\sum_{i=1}^{n}X_i^2-2\overline{X}\sum_{i=1}^{n}X_i+n\overline{X}^2 \\
			=&\sum_{i=1}^{n}X_i^2-2n\overline{X}^2+n\overline{X}^2=\sum_{i=1}^{n}X_i^2-n\overline{x}^2
		\end{aligned} \\
		\begin{aligned}
			&\sum_{i=1}^{n}(X_i-\overline{X})^2=\sum_{i=1}^{n}(X_i-\mu+\mu-\overline{X})^2 \\
			=&\sum_{i=1}^{n}(X_i-\mu)^2+2\sum_{i=1}^{n}(X_i-\mu)(\mu-\overline{X})+n(\mu-\overline{X})^2 \\
			=&\sum_{i=1}^{n}(X_i-\mu)^2+2(\mu-\overline{X})(n\overline{X}-n\mu)+n(\mu-\overline{X})^2 \\
			=&\sum_{i=1}^{n}(X_i-\mu)^2-n(\mu-\overline{X})^2
		\end{aligned}\qedhere
	\end{gather*}
\end{proof}
\begin{property}
	设$(X,\mathscr{A},\mathscr{P})$是统计结构,总体$F$的期望和方差分别为$\mu,\sigma^2$,$\mathbf{X}=(\seq{X}{n})$为从总体$F$中抽取的简单样本,$X_i\in L_2(X)$。总体期望和方差的点估计:
	\begin{equation*}
		\overline{X}=\frac{1}{n}\sum_{i=1}^{n}X_i,\quad S^2=\frac{1}{n-1}\sum_{i=1}^{n}(X_i-\overline{X})^2
	\end{equation*}
	具有如下性质:
	\begin{enumerate}
		\item $\operatorname{E}(\overline{X})=\mu,\;\operatorname{Var}(\overline{X})=\dfrac{\sigma^2}{n},\;\operatorname{E}(S^2)=\sigma^2$。若$\nu_4$存在,则:
		\begin{equation*}
			\operatorname{Var}(S^2)=\frac{n(\nu_4-\sigma^4)}{(n-1)^2}+\frac{\nu_4-3\sigma^4}{n(n-1)^2}-\frac{2(\nu_4-2\sigma^4)}{(n-1)^2}
		\end{equation*}
		若$\nu_3$存在,则$\operatorname{Cov}(\overline{X},S^2)=\dfrac{\nu_3}{n}$;
		\item $\overline{X}$和$S^2$分别为期望和方差的强相合估计;
		\item $\overline{X}\overset{a}{\longrightarrow}\operatorname{N}\left(\mu,\dfrac{\sigma^2}{n}\right)$。若$\nu_4$存在,则$S^2\overset{a}{\longrightarrow}\operatorname{N}\left(\sigma^2,\dfrac{\nu_4-\sigma^4}{n}\right)$;
		\item $\operatorname{Corr}(X_i-\overline{X},X_j-\overline{X})=-\dfrac{1}{n-1}(i\ne j),\;S^2=\dfrac{1}{n(n-1)}\sum\limits_{i<j}^{}(X_i-X_j)^2$
	\end{enumerate}
\end{property}
\begin{proof}
	因为$X_i\in L_2(X)$,由\cref{theo:LtLs}可知$\mu<+\infty$,根据\cref{prop:Variance}(1)可得$\sigma^2<+\infty$。\par
	(1)由\cref{prop:MeasurableIntegral}(5)可得:
	\begin{equation*}
		\operatorname{E}(\overline{X})=\frac{1}{n}\sum_{i=1}^{n}\operatorname{E}(X_i)=\mu
	\end{equation*}
	根据\cref{prop:MeasurableIntegral}(5)、\cref{prop:Variance}(3)和\cref{prop:CovMat}(7)可得:
	\begin{equation*}
		\operatorname{Var}(\overline{X})=\frac{1}{n^2}\sum_{i=1}^{n}\operatorname{Var}(X_i)=\frac{n\sigma^2}{n^2}=\frac{\sigma^2}{n}
	\end{equation*}
	由\cref{theo:xbar}可得:
	\begin{equation*}
		S^2=\frac{1}{n-1}\sum_{i=1}^{n}(X_i-\overline{X})^2=\frac{1}{n-1}\left(\sum_{i=1}^{n}X_i^2-n\overline{X}^2\right)
	\end{equation*}
	根据\cref{prop:MeasurableIntegral}(5)和\cref{prop:Variance}(1)($\overline{X}$对应的转换需要根据\cref{prop:MeasurableIntegral}(5)另推)可得:
	\begin{align*}
		&\operatorname{E}(S^2)=\frac{1}{n-1}\sum_{i=1}^{n}\operatorname{E}(X_i^2)-\frac{n}{n-1}\operatorname{E}(\overline{X}^2) \\
		=&\frac{1}{n-1}n\{\operatorname{Var}(X_i)+[\operatorname{E}(X_i)]^2\}-\frac{n}{n-1}\{\operatorname{Var}(\overline{X})+[\operatorname{E}(\overline{X})]^2\} \\
		=&\frac{n}{n-1}(\sigma^2+\mu^2)-\frac{n}{n-1}\left(\frac{\sigma^2}{n}+\mu^2\right)=\frac{n}{n-1}\sigma^2-\frac{n}{n-1}\frac{\sigma^2}{n}=\sigma^2
	\end{align*}
	当$\nu_4$存在时,由\cref{theo:xbar}可得:
	\begin{equation*}
		\sum_{i=1}^{n}(X_i-\overline{X})^2=\sum_{i=1}^{n}(X_i-\mu)^2-n(\overline{X}-\mu)^2
	\end{equation*}
	因为$\nu_4$存在,所以$(X_i-\mu)^2\in L_2(X)$\info{证明$(\overline{X}-\mu)^2\in L_2(X)$,检查协方差公式适用条件以及协方差的良定义},由\cref{prop:MeasurableIntegral}(5)、\cref{prop:Variance}(3)、\cref{prop:CovMat}(7)(5)(3)、\cref{prop:Variance}(1)可得:
	\begin{align*}
		&\operatorname{Var}(S^2)=\operatorname{Var}\left[\frac{1}{n-1}\sum_{i=1}^{n}(X_i-\overline{X})^2\right]=\frac{1}{(n-1)^2}\operatorname{Var}\left[\sum_{i=1}^{n}(X_i-\mu)^2-n(\overline{X}-\mu)^2\right] \\
		=&\frac{1}{(n-1)^2}\left\{\operatorname{Var}\left[\sum_{i=1}^{n}(X_i-\mu)^2\right]+\operatorname{Var}[n(\overline{X}-\mu)^2]-2\operatorname{Cov}\left[\sum_{i=1}^{n}(X_i-\mu)^2,n(\overline{X}-\mu)^2\right]\right\} \\
		=&\frac{1}{(n-1)^2}\left\{\sum_{i=1}^{n}\operatorname{Var}[(X_i-\mu)^2]+n^2\operatorname{Var}[(\overline{X}-\mu)^2]-2n\sum_{i=1}^{n}\operatorname{Cov}[(X_i-\mu)^2,(\overline{X}-\mu)^2]\right\}
	\end{align*}
	由\cref{prop:Variance}(1)可得:
	\begin{equation*}
		\operatorname{Var}[(X_i-\mu)^2]=\operatorname{E}[(X_i-\mu)^4]-\{\operatorname{E}[(X_i-\mu)^2]\}^2=\nu_4-\sigma^4
	\end{equation*}
	由\cref{prop:MeasurableIntegral}(5)、\cref{prop:Variance}(3)、\cref{prop:Independent}(4)、\cref{prop:CovMat}(7)(5)和\info{独立期望}可得:
	\begin{align*}
		&n^2\operatorname{Var}[(\overline{X}-\mu)^2]=n^2\operatorname{Var}\left\{\left[\frac{\sum\limits_{i=1}^{n}(X_i-\mu)}{n}\right]^2\right\}=\frac{1}{n^2}\operatorname{Var}\left\{\left[\sum_{i=1}^{n}(X_i-\mu)\right]^2\right\} \\
		=&\frac{1}{n^2}\operatorname{Var}\left[\sum_{i=1}^{n}(X_i-\mu)^2+2\sum_{i=1}^{n}\sum_{j>i}^{}(X_i-\mu)(X_j-\mu)\right] \\
		=&\frac{1}{n^2}\left\{\sum_{i=1}^{n}\operatorname{Var}[(X_i-\mu)^2]+\operatorname{Var}\left[2\sum_{i=1}^{n}\sum_{j>i}^{}(X_i-\mu)(X_j-\mu)\right]\right. \\
		&\left.+2\operatorname{Cov}\left[\sum_{i=1}^{n}(X_i-\mu)^2,2\sum_{i=1}^{n}\sum_{j>i}^{}(X_i-\mu)(X_j-\mu)\right]\right\} \\
		=&\frac{1}{n^2}\left\{n(\nu_4-\sigma^4)+\operatorname{Cov}\left[2\sum_{i=1}^{n}\sum_{j>i}^{}(X_i-\mu)(X_j-\mu)\right]\right. \\
		&\left.+4\sum_{i=1}^{n}\sum_{j=1}^{n}\sum_{k>j}^{}\operatorname{Cov}\left[(X_i-\mu)^2,(X_j-\mu)(X_k-\mu)\right]\right\} \\
		=&\frac{1}{n^2}\left\{n(\nu_4-\sigma^4)+4\sum_{i=1}^{n}\sum_{j>i}^{}\sum_{k=1}^{n}\sum_{l>k}^{}\operatorname{Cov}\left[(X_i-\mu)(X_j-\mu),(X_k-\mu)(X_l-\mu)\right]\right. \\
		&\left.+4\sum_{i=1}^{n}\sum_{j=1}^{n}\sum_{k>j}^{}\{\operatorname{E}[(X_i-\mu)^2(X_j-\mu)(X_k-\mu)]-\operatorname{E}[(X_i-\mu)^2]\operatorname{E}[(X_j-\mu)(X_k-\mu)]\}\right\} \\
		=&\frac{1}{n^2}\left\{n(\nu_4-\sigma^4)+4\sum_{i=1}^{n}\sum_{j>i}^{}\sum_{k=1}^{n}\sum_{l>k}^{}\{\operatorname{E}[(X_i-\mu)(X_j-\mu)(X_k-\mu)(X_l-\mu)]\right. \\
		&\Big.-\operatorname{E}[(X_i-\mu)(X_j-\mu)]\operatorname{E}[(X_k-\mu)(X_l-\mu)]\}\Big\} \\
		=&\frac{1}{n^2}\left\{n(\nu_4-\sigma^4)+4\sum_{i=1}^{n}\sum_{j>i}^{}\operatorname{E}[(X_i-\mu)^2(X_j-\mu)^2]\right\} \\
		=&\frac{1}{n^2}\left\{n(\nu_4-\sigma^4)+4\sum_{i=1}^{n}\sum_{j>i}^{}\operatorname{E}[(X_i-\mu)^2]\operatorname{E}[(X_j-\mu)^2]\right\}=\frac{1}{n^2}\left[n(\nu_4-\sigma^4)+4\frac{n(n-1)}{2}\sigma^4\right] \\
		=&\frac{1}{n}[(\nu_4-\sigma^4)+2(n-1)\sigma^4]=\frac{1}{n}[\nu_4+(2n-3)\sigma^4]
	\end{align*}
	根据\cref{prop:CovMat}(3)(5)(6)和\info{独立期望}可得:
	\begin{align*}
		&-2n\sum_{i=1}^{n}\operatorname{Cov}[(X_i-\mu)^2,(\overline{X}-\mu)^2] \\
		=&-2n\frac{1}{n^2}\operatorname{Cov}\left[(X_i-\mu)^2,\sum_{j=1}^{n}(X_j-\mu)^2+2\sum_{j=1}^{n}\sum_{k>1}^{}(X_j-\mu)(X_k-\mu)\right] \\
		=&-\frac{2}{n}\sum_{i=1}^{n}\sum_{j=1}^{n}\operatorname{Cov}[(X_i-\mu)^2,(X_j-\mu)^2]-4n\sum_{i=1}^{n}\sum_{j=1}^{n}\sum_{k>j}^{}\operatorname{Cov}[(X_i-\mu)^2,(X_j-\mu)(X_k-\mu)] \\
		=&-\frac{2}{n}\sum_{i=1}^{n}\sum_{j=1}^{n}\{\operatorname{E}[(X_i-\mu)^2(X_j-\mu)^2]-\operatorname{E}[(X_i-\mu)^2]\operatorname{E}[(X_j-\mu)^2]\} \\
		=&-\frac{2}{n}\sum_{i=1}^{n}\sum_{j=1}^{n}\operatorname{E}[(X_i-\mu)^2(X_j-\mu)^2]+\frac{2}{n}\sum_{i=1}^{n}\sum_{j=1}^{n}\operatorname{E}[(X_i-\mu)^2]\operatorname{E}[(X_j-\mu)^2] \\
		=&-\frac{2}{n}[n\nu_4+n(n-1)\sigma^4]+2n^3\sigma^4=-2\nu_4-2(n-1)\sigma^4+2n\sigma^4=-2\nu_4+2\sigma^4
	\end{align*}
	综上:
	\begin{align*}
		\operatorname{Var}(S^2)&=\frac{1}{(n-1)^2}\left\{n(\nu_4-\sigma^4)+\frac{1}{n}[\nu_4+(2n-3)\sigma^4]-2\nu_4+2\sigma^4\right\} \\
		&=\frac{n(\nu_4-\sigma^4)}{(n-1)^2}+\frac{\nu_4-3\sigma^4}{n(n-1)^2}-\frac{2(\nu_4-2\sigma^4)}{(n-1)^2}
	\end{align*}
	当$\nu_3$存在时,由\cref{prop:CovMat}(3)(5)、\cref{prop:MeasurableIntegral}(5)、\cref{prop:Independent}(6)和\info{独立期望}可得:
	\begin{align*}
		&\operatorname{Cov}(\overline{X},S^2)=\operatorname{Cov}\left[\frac{1}{n}\sum_{i=1}^{n}X_i,\frac{1}{n-1}\sum_{i=1}^{n}(X_i-\overline{X})^2\right] \\
		=&\operatorname{Cov}\left\{\frac{1}{n}\sum_{i=1}^{n}X_i,\frac{1}{n-1}\left[\sum_{i=1}^{n}(X_i-\mu)^2-n(\overline{X}-\mu)^2\right]\right\} \\
		=&\frac{1}{n(n-1)}\sum_{i=1}^{n}\sum_{j=1}^{n}\operatorname{Cov}[X_i,(X_j-\mu)^2]-\frac{n}{n(n-1)}\sum_{i=1}^{n}\operatorname{Cov}[X_i,(\overline{X}-\mu)^2] \\
		=&\frac{1}{n(n-1)}\sum_{i=1}^{n}\sum_{j=1}^{n}\{\operatorname{E}[X_i(X_j-\mu)^2]-\operatorname{E}(X_i)\operatorname{E}[(X_j-\mu)^2]\}-\frac{1}{n-1}\sum_{i=1}^{n}\operatorname{Cov}\left[X_i,\left(\frac{1}{n}\sum_{j=1}^{n}X_i-\mu\right)^2\right] \\
		=&\frac{1}{n(n-1)}\sum_{i=1}^{n}\{\operatorname{E}[X_i(X_i-\mu)^2]-\operatorname{E}(X_i)\operatorname{E}[(X_i-\mu)^2]\} \\
		&+\frac{1}{n(n-1)}\sum_{i=1}^{n}\sum_{j\ne i}^{}\{\operatorname{E}[X_i(X_j-\mu)^2]-\operatorname{E}(X_i)\operatorname{E}[(X_j-\mu)^2]\} \\
		&-\frac{1}{n^2(n-1)}\sum_{i=1}^{n}\operatorname{Cov}\left[X_i,\sum_{j=1}^n(X_j-\mu)^2+2\sum_{j=1}^n\sum_{k>j}(X_j-\mu)(X_k-\mu)\right] \\
		=&\frac{1}{n(n-1)}\sum_{i=1}^{n}\operatorname{E}[(X_i-\mu)^3]-\frac{1}{n^2(n-1)}\sum_{i=1}^{n}\sum_{j=1}^n\operatorname{Cov}[X_i,(X_j-\mu)^2] \\
		&-\frac{2}{n^2(n-1)}\sum_{i=1}^{n}\sum_{j=1}^n\sum_{k>j}\operatorname{Cov}[X_i,(X_j-\mu)(X_k-\mu)] \\
		=&\frac{\nu_3}{(n-1)}-\frac{\nu_3}{n(n-1)}-\frac{2}{n^2(n-1)}\sum_{i=1}^{n}\sum_{j=1}^n\sum_{k>j}\{\operatorname{E}[X_i(X_j-\mu)(X_k-\mu)] \\
		&-\operatorname{E}(X_i)\operatorname{E}[(X_j-\mu)(X_k-\mu)]\}\frac{n\nu_3-\nu_3}{n(n-1)}=\frac{\nu_3}{n}
	\end{align*}\par
	(2)由\cref{prop:MomentEstimation}(1)可知$\overline{X}$是$\mu$的强相合估计,由\cref{prop:MomentEstimation}(2)可知$\nu_{n2}$是$\sigma^2$的强相合估计,即$\nu_{n2}\overset{P}{\longrightarrow}\sigma^2$,注意到:
	\begin{equation*}
		S^2=\dfrac{n}{n-1}\nu_{n2},\quad\frac{n}{n-1}\to1
	\end{equation*}
	由\cref{prop:ConvergenceInProbability}(1)(3)可得$S^2\overset{P}{\longrightarrow}\sigma^2$,即$S^2$是$\sigma^2$的强相合估计。\par
	(3)由\cref{theo:CLT}可得:
	\begin{equation*}
		\frac{\sqrt{n}(\overline{X}-\mu)}{\sigma}\overset{d}{\longrightarrow}\operatorname{N}(0,1)
	\end{equation*}
	即:
	\begin{equation*}
		\overline{X}\overset{a}{\longrightarrow}\operatorname{N}\left(\mu,\frac{\sigma^2}{n}\right)
	\end{equation*}
	由\cref{theo:xbar}可得:
	\begin{equation*}
		\frac{1}{n-1}\sum_{i=1}^{n}(X_i-\overline{X})^2=\frac{1}{n-1}\sum_{i=1}^{n}(X_i-\mu)^2-\frac{n}{n-1}(\overline{X}-\mu)^2
	\end{equation*}
	由\cref{prop:Measure}(5)可得:
	\begin{equation*}
		\operatorname{Var}[(X_i-\mu)^2]=\operatorname{E}\{[(X_i-\mu)^2-\sigma^2]^2\}=\operatorname{E}[(X_i-\mu)^4-2\sigma^2(X_i-\mu)^2+\sigma^4]=\nu_4-\sigma^4
	\end{equation*}
	根据\cref{theo:CLT}可得:
	\begin{equation*}
		\sqrt{n}\frac{\dfrac{1}{n}\sum\limits_{i=1}^{n}(X_i-\mu)^2-\sigma^2}{\sqrt{\nu_4-\sigma^4}}\overset{d}{\longrightarrow}\operatorname{N}(0,1)
	\end{equation*}
	由\cref{prop:ConvergenceInProbability}(1)和\cref{theo:Slutsky}即:
	\begin{equation*}
		\sqrt{n}\left[\frac{1}{n}\sum_{i=1}^{n}(X_i-\mu)^2-\sigma^2\right]\overset{d}{\longrightarrow}\operatorname{N}\left(0,\frac{\nu_4-\sigma^4}{n}\right)
	\end{equation*}
	注意到:
	\begin{equation*}
		\sqrt{n}\frac{1}{n-1}\sum_{i=1}^{n}(X_i-\overline{X})^2=\frac{n}{n-1}\frac{1}{n}\sum_{i=1}^{n}(X_i-\mu)^2-\frac{n}{n-1}(\overline{X}-\mu)^2
	\end{equation*}\par
	(4)由\cref{prop:CovMat}(3)(5)(7)可得:
	\begin{align*}
		&\operatorname{Cov}(X_i-\overline{X},X_j-\overline{X})=\frac{1}{n^2}\operatorname{Cov}\left[\sum_{k\ne i}^{}(X_i-X_k),\sum_{l\ne j}^{}(X_j-X_l)\right] \\
		=&\frac{1}{n^2}\sum_{k\ne i}^{}\sum_{l\ne j}^{}[\operatorname{Cov}(X_i,X_j)-\operatorname{Cov}(X_i,X_l)-\operatorname{Cov}(X_k.X_j)+\operatorname{Cov}(X_k,X_l)] \\
		=&\frac{1}{n^2}\sum_{k\ne i}^{}\sum_{l\ne j}^{}[-\operatorname{Cov}(X_i,X_l)-\operatorname{Cov}(X_k,X_j)+\operatorname{Cov}(X_k,X_l)] \\
		=&\frac{1}{n^2}\left[-\sum_{k\ne i}^{}\sum_{l\ne j}^{}\operatorname{Cov}(X_i,X_l)-\sum_{k\ne i}^{}\sum_{l\ne j}^{}\operatorname{Cov}(X_k,X_j)+\sum_{k\ne i}^{}\sum_{l\ne j}^{}\operatorname{Cov}(X_k,X_l)\right] \\
		=&\frac{1}{n^2}\left[-\sum_{k\ne i}^{}\sigma^2-\sum_{k\ne i}^{}(n-1)\operatorname{Cov}(X_k,X_j)+(n-2)\sigma^2\right] \\
		=&\frac{1}{n^2}\left[-(n-1)\sigma^2-(n-1)\sigma^2+(n-2)\sigma^2\right]=-\frac{\sigma^2}{n}
	\end{align*}
	由(1)和\cref{prop:MeasurableIntegral}(5)、\cref{prop:Variance}(3)和\cref{prop:CovMat}(7)有:
	\begin{align*}
		\operatorname{Var}(X_i-\overline{X})=\frac{1}{n^2}\operatorname{Var}\left[(n-1)X_i-\sum_{k\ne i}^{}X_k\right]=\frac{1}{n^2}[(n-1)^2\sigma^2+(n-1)\sigma^2]=\frac{(n-1)\sigma^2}{n}
	\end{align*}
	所以:
	\begin{equation*}
		\operatorname{Corr}(X_i-\overline{X},X_j-\overline{X})=\frac{\operatorname{Cov}(X_i-\overline{X},X_j-\overline{X})}{\sqrt{\operatorname{Var}(X_i-\overline{X})\operatorname{Var}(X_j-\overline{X})}}=\dfrac{-\dfrac{\sigma^2}{n}}{\dfrac{(n-1)\sigma^2}{n}}=\frac{-1}{n-1}
	\end{equation*}
	注意到:
	\begin{align*}
		&\sum_{i<j}^{}(X_i-X_j)^2=\sum_{i=1}^{n-1}\sum_{j>i}^{}(X_i^2-2X_iX_j+X_j^2)=\sum_{i=1}^{n-1}\sum_{j>i}^{}(X_i^2+X_j^2)-2\sum_{i=1}^{n-1}\sum_{j>i}^{}X_iX_j \\
		=&(n-1)\sum_{i=1}^{n}X_i^2-2\sum_{i=1}^{n-1}\sum_{j>i}^{}X_iX_j=n\sum_{i=1}^{n}X_i^2-\left(\sum_{i=1}^{n}X_i\right)^2=n\sum_{i=1}^{n}X_i^2-n^2\overline{X}^2
	\end{align*}
	由\cref{theo:xbar}可得:
	\begin{equation*}
		\frac{1}{n(n-1)}\sum_{i<j}^{}(X_i-X_j)^2=\frac{1}{n-1}\sum_{i=1}^{n}X_i^2-\frac{n}{n-1}\overline{X}^2=\frac{1}{n-1}\left(\sum_{i=1}^{n}X_i^2-n\overline{X}^2\right)=S^2\qedhere
	\end{equation*}
\end{proof}
\begin{definition}
	称上述两个点估计为\gls{SampleMean}和\gls{SampleVariance}。
\end{definition}
\subsubsection{数值运算}
\begin{theorem}
	对于样本方差和样本均值:
	\begin{enumerate}
		\item 记:
		\begin{equation*}
			\overline{x}_n=\frac{1}{n}\sum_{i=1}^{n}x_i,\quad s_n^2=\frac{1}{n-1}\sum_{i=1}^{n}(x_i-\overline{x}_n)^2
		\end{equation*}
		则有:
		\begin{equation*}
			\overline{x}_{n+1}=\overline{x}_n+\frac{1}{n+1}(x_{n+1}-\overline{x}_n),\quad s_{n+1}^2=\frac{n-1}{n}s_n^2+\frac{1}{n}(x_{n+1}-\overline{x}_n)^2
		\end{equation*}
		\item 从同一总体中抽取两个大小分别为$m$和$n$的样本,样本均值分别记为$\overline{x}_1,\overline{x}_2$,样本方差分别为$s_1^2,s_2^2$,将两组样本合并,其均值和样本方差分别为$\overline{x},s^2$,则有:
		\begin{equation*}
			\overline{x}=\frac{m\overline{x}_1+n\overline{x}_2}{m+n},\quad s^2=\frac{(m-1)s_1^2+(n-1)s_2^2}{m+n-1}+\frac{mn(\overline{x}_1-\overline{x}_2)^2}{(m+n)(m+n+1)}
		\end{equation*}
	\end{enumerate}
\end{theorem}
\begin{proof}
	(1)对于均值有:
	\begin{align*}
		\overline{x}_{n+1}&=\frac{1}{n+1}\sum_{i=1}^{n+1}x_i=\frac{n}{n+1}\frac{1}{n}\left(\sum_{i=1}^{n}x_i+x_{n+1}\right) \\
		&=\frac{n}{n+1}\overline{x}_n+\frac{1}{n+1}x_{n+1}=\overline{x}_n+\frac{1}{n+1}(x_{n+1}-\overline{x}_n)
	\end{align*}
	对于方差有:
	\begin{align*}
		s_{n+1}^2&=\frac{1}{n}\sum_{i=1}^{n+1}(x_i-\overline{x}_{n+1})^2=\frac{1}{n}\sum_{i=1}^{n+1}\left[x_i-\overline{x}_n-\frac{1}{n+1}(x_{n+1}-\overline{x}_n)\right]^2 \\
		&=\frac{1}{n}\sum_{i=1}^{n+1}\left[(x_i-\overline{x}_n)^2+\frac{1}{(n+1)^2}(x_{n+1}-\overline{x}_n)^2-\frac{2}{n+1}(x_i-\overline{x}_n)(x_{n+1}-\overline{x}_n)\right] \\
		&=\frac{1}{n}\sum_{i=1}^{n}(x_i-\overline{x}_n)^2+\frac{1}{n}(x_{n+1}-\overline{x}_n)^2+\frac{1}{n(n+1)}(x_{n+1}-\overline{x}_n)^2-\frac{2}{n(n+1)}(x_{n+1}-\overline{x}_n)^2 \\
		&=\frac{n-1}{n}s_n^2+\frac{1}{n}(x_{n+1}-\overline{x}_n)^2-\frac{1}{n(n+1)}(x_{n+1}-\overline{x}_n)^2 \\
		&=\frac{n-1}{n}s_n^2+\frac{1}{n}(x_{n+1}-\overline{x}_n)^2
	\end{align*}\par
	(2)均值的结论是显然的,对于方差有:
	\begin{align*}
		s^2&=\frac{1}{m+n-1}\sum_{i=1}^{m+n}(x_i-\overline{x})^2=\frac{1}{m+n-1}\left[\sum_{i=1}^{m}(x_i-\overline{x})^2+\sum_{i=m+1}^{m+n}(x_i-\overline{x})^2\right] \\
		&=\frac{1}{m+n-1}\left[\sum_{i=1}^{m}(x_i-\overline{x}_1+\overline{x}_1-\overline{x})^2+\sum_{i=m+1}^{m+n}(x_i-\overline{x}_2+\overline{x}_2-\overline{x})^2\right] \\
		&=\frac{1}{m+n-1}\left[(m-1)s_1^2+m(\overline{x}_1-\overline{x})^2+2\sum_{i=1}^{m}(x_i-\overline{x}_1)(\overline{x_1}-\overline{x})\right. \\
		&\quad\left.+(n-1)s_2^2+n(\overline{x}_2-\overline{x})^2+2\sum_{i=m+1}^{m+n}(x_i-\overline{x}_2)(\overline{x_2}-\overline{x})\right] \\
		&=\frac{1}{m+n-1}\left[(m-1)s_1^2+m(\overline{x}_1-\overline{x})^2+(n-1)s_2^2+n(\overline{x}_2-\overline{x})^2\right] \\
		&=\frac{(m-1)s_1^2+(n-1)s_2^2}{m+n-1}+\frac{m(\overline{x}_1-\overline{x})^2+n(\overline{x}_2-\overline{x})^2}{m+n-1}
	\end{align*}
	由于:
	\begin{gather*}
		m(\overline{x}_1-\overline{x})^2=m\left(\overline{x}_1-\frac{m\overline{x}_1+n\overline{x}_2}{m+n}\right)^2=m\frac{n^2(\overline{x}_1-\overline{x}_2)^2}{(m+n)^2} \\
		n(\overline{x}_2-\overline{x})^2=n\left(\overline{x}_2-\frac{m\overline{x}_1+n\overline{x}_2}{m+n}\right)^2=n\frac{m^2(\overline{x}_1-\overline{x}_2)^2}{(m+n)^2}
	\end{gather*}
	所以:
	\begin{align*}
		s^2&=\frac{(m-1)s_1^2+(n-1)s_2^2}{m+n-1}+\frac{m(\overline{x}_1-\overline{x})^2+n(\overline{x}_2-\overline{x})^2}{m+n-1} \\
		&=\frac{(m-1)s_1^2+(n-1)s_2^2}{m+n-1}+\frac{(m+n)mn(\overline{x}_1-\overline{x}_2)^2}{(m+n-1)(m+n)^2} \\
		&=\frac{(m-1)s_1^2+(n-1)s_2^2}{m+n-1}+\frac{mn(\overline{x}_1-\overline{x}_2)^2}{(m+n-1)(m+n)}\qedhere
	\end{align*}
\end{proof}
我高中时热爱生物,高考志愿只填了生物学志愿并且全部不服从调剂,我的本科录取专业是南京农业大学生物学理科基地班,曾任基地班班长,但由于个人实在不擅长背书,反而数理能力较,于是转专业到了数学系的统计学专业,希望换一种方式去做生物学的研究。由于大一在生科院的成绩并不好看,导致我现在的绩点与排名比较拉垮。
我从以下几个方面介绍我自己:
科研。转入数学系之后,我主持一项省级大学生创新训练项目,在此期间以第三作者在期刊Horticulturae上发表论文MLAS: Machine Learning-Based Approach for Predicting Abiotic Stress-Responsive Genes in Chinese Cabbage(导师为第一作者,研究生师姐为第二作者)。我了解机器学习算法,熟练使用scikit-learn包,可以用numpy和pandas手撸一部分模型的代码;对于深度学习,我熟练掌握pytorch框架,了解CNN、MLP、LSTM、GRU、RNN、Transformer、BERT等模型,也了解LLM中的一些分词算法,曾自己从零构建BERT模型,完成数据集构造、模型搭建、Tensorboard与log日志记录、模型评价,并且了解GPU并行运算与单机多卡训练,也会使用梯度累积、混合精度训练等技巧。我也曾学习半监督问题、Transductive Learning、数据集不平衡问题,掌握基本算法。
技术。我的代码能力很强,在编程语言方面我会使用python与R语言(baser与tidyr),LInux命令行的基本使用与bash脚本的编写也没有太大问题,曾跟随农学院黄骥老师学习生物信息学程序设计,获得99分,期末作业的代码也被老师展示在课程群里供同学学习。我也熟练掌握Latex、markdown、Rmarkdown、Quarto、JupyterNotebook、Git等工具,可以很好的将自己的成果进行交互式展示。
数学与统计学专业素养:我对测度论、回归分析、多元统计分析、数理统计掌握较好,并且学完了一部分基于测度论的高等数理统计学,在阅读Lehmann的经典作品“Theory of Point Estimation”、“Testing statistical Hypotheses”以及Jun Shao的“Mathematical Statistics”。我基本学完了王松桂先生的“线性模型引论”,从矩阵广义逆的角度去分析线性模型,不只是本科基础的“应用回归分析”,而对于多元统计,本身这个领域在机器学习中有很多应用,我的了解也比较深入。对于我本科的绝大部分课程,我都是自己一个定理一个定理的推导过来,并且整理为了个人的一本Notes,该Notes我将附在附件中,目前一共800多页,并且尚在逐步更新逐步完善中,我将这份notes公开在Github上,也受到了学院老师向学弟学妹的推荐。
竞赛。我曾获美国大学生数学建模竞赛O奖,受学校与媒体的报道,这可以说明我在利用数学与统计学工具进行建模解决实际问题的能力。
英语。我从大二开始强制自己阅读英文论文,不会的单词查找并记录,目前可以无障碍阅读领域内的文献,并积累下学术词汇400余条。
学术追求与未来打算:我热衷于科研,喜欢钻研东西,我从不局限于课堂知识,喜欢刨根究底将结论进行推广,喜欢那种一点点积累最后突然醒悟的过程。我的未来一定会读博士,有能力的情况下我希望能够从事科研。
我的Github账号为Expectorpatro,上面有我参与的一些项目和自己捣鼓的一些代码,供各位老师参考。
兴趣爱好。目前在努力地健身减肥,比较喜欢辩论这项运动,是生命科学学院辩论队的一员。