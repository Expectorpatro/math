\section{一致最小风险无偏估计}

\begin{definition}
	设$(X,\mathscr{A},\mathscr{P})$是参数结构,$\Theta$是参数空间,$\mathbf{X}$为从总体$F$中抽取的简单样本,$g(\theta)$为待估量,$\operatorname{R}(\theta,d)$为风险函数。若估计量$\delta(\mathbf{X})$对$g(\theta)$的任一估计量$\delta'(\mathbf{X})$有:
	\begin{equation*}
		\forall\;\theta\in\Theta,\;\operatorname{R}[\theta,\delta(\mathbf{X})]\leqslant\operatorname{R}[\theta,\delta'(\mathbf{X})]
	\end{equation*}
	则称$\delta(\mathbf{X})$为$g(\theta)$的\gls{UMRE}。
\end{definition}
\begin{note}
	估计量的一致最小风险估计常不存在。\par
	设$\delta(\mathbf{X})$是$g(\theta)$的一致风险估计。若风险函数$\operatorname{R}(\theta,d)$存在关于$d$的最小值,任取$\theta_0\in\Theta$,我们总能可以取一个有偏好的$\delta'(\mathbf{X})=\arg\min\operatorname{R}(\theta,d)$,那么$\delta(\mathbf{X})$的风险函数在$\theta_0$处的取值也应是最小值。由$\theta_0$的任意性,$\operatorname{R}[\theta,\delta(\mathbf{X})]$需要在整个参数空间$\Theta$上都取到风险函数的最小值,这显然是不太可能存在的。若风险函数不存在关于$d$的最小值,那么一致最小风险估计当然不存在。\par
	考虑到上述情况,我们转向研究在某一估计量族中寻找一致最小风险估计,而不是在所有估计量中去寻找。人们关注最多的便是在无偏估计量族中的情况。
\end{note}
\begin{definition}
	设$(X,\mathscr{A},\mathscr{P})$是参数结构,$\Theta$是参数空间,$\mathbf{X}$为从总体$F$中抽取的简单样本,$g(\theta)$为存在无偏估计量的待估量,$\operatorname{R}(\theta,d)$为风险函数。若估计量$\delta(\mathbf{X})$对$g(\theta)$的任一无偏估计量$\delta'(\mathbf{X})$有:
	\begin{equation*}
		\forall\;\theta\in\Theta,\;\operatorname{R}[\theta,\delta(\mathbf{X})]\leqslant\operatorname{R}[\theta,\delta'(\mathbf{X})]
	\end{equation*}
	则称$\delta(\mathbf{X})$为$g(\theta)$的\gls{UMRUE}。
\end{definition}
\begin{theorem}[Rao-Blackwell Theorem]
	\label{theo:Rao-Blackwell}
	设$(X,\mathscr{A},\mathscr{P})$是参数结构,$\Theta$是参数空间,$\mathbf{X}$为从总体$F$中抽取的简单样本,$g(\theta)$为待估量,$\delta(\mathbf{X})$是$g(\theta)$的估计量,$T$是$\mathscr{P}$的充分统计量,损失函数$L(\theta,d)$是关于$d$的凸函数。若$\delta(\mathbf{X}),L[\theta,\delta(\mathbf{X})]\in L_1(X)$,则:
	\begin{equation*}
		h(T)=\operatorname{E}[\delta(\mathbf{X})|T]
	\end{equation*}
	满足:
	\begin{equation*}
		\forall\;\theta\in\Theta,\;\operatorname{R}[\theta,h(T)]\leqslant\operatorname{R}[\theta,\delta(\mathbf{X})]
	\end{equation*}
	若$L(\theta,d)$关于$d$严格凸,则等号成立当且仅当$\delta(\mathbf{X})=\operatorname{E}[\delta(\mathbf{X})|T]=h(T)\;$a.s.于任意的$P\in\mathscr{P}$。若$\delta(\mathbf{X})$是$g(\theta)$的无偏估计量,则$h(T)$也是$g(\theta)$的无偏估计量。
\end{theorem}
\begin{proof}
	因为$T$是充分统计量,所以$P(\mathbf{X}|T)$与$\theta$无关,$h(T)=\operatorname{E}[\delta(\mathbf{X})|T]$也与$\theta$无关,由条件期望的定义可得$h(T)$可测,于是$h(T)$是一个统计量。因为$\delta(\mathbf{X})\in L_1(X)$,由\cref{prop:ConditionalExpectation}(3)可知$h(T)\in L_1(X)$。\par
	在\cref{ineq:Jensen}中取$\varphi(d)=L(\theta,d)$可得:
	\begin{equation*}
		\forall\;P\in\mathscr{P},\;\varphi[h(T)]=\varphi\{\operatorname{E}[\delta(\mathbf{X})|T]\}\leqslant\operatorname{E}\{\varphi[\delta(\mathbf{X})]|T\}\;\text{a.s.于}P
	\end{equation*}
	当$L(\theta,d)$关于$d$严格凸时等号成立a.s.于任意的$P\in\mathscr{P}$当且仅当$\delta(\mathbf{X})=\operatorname{E}[\delta(\mathbf{X})|T]=h(T)\;$a.s.于任意的$P\in\mathscr{P}$。根据\cref{prop:MeasurableIntegral}(6),对上式两边同时求期望可得:
	\begin{equation*}
		\operatorname{E}\{L[\theta,h(T)]\}\leqslant\operatorname{E}\Big\{\operatorname{E}\{\varphi[\delta(\mathbf{X})]|T\}\Big\}=\operatorname{E}\{\varphi[\delta(\mathbf{X})]\}=\operatorname{E}\{L[\theta,\delta(\mathbf{X})]\}=\operatorname{R}[\theta,\delta(\mathbf{X})]
	\end{equation*}
	若$L(\theta,d)$关于$d$严格凸,由\cref{prop:MeasurableIntegral}(7)可知当$\delta(\mathbf{X})=\operatorname{E}[\delta(\mathbf{X})|T]=h(T)\;$a.s.于任意的$P\in\mathscr{P}$时上式等号成立。当上式等号成立时,因为$L[\theta,\delta(\mathbf{X})]\in L_1$,由\cref{prop:MeasurableIntegral}(5)可得:
	\begin{equation*}
		\operatorname{E}\{L[\theta,h(T)]\}-\operatorname{E}\Big\{\operatorname{E}\{\varphi[\delta(\mathbf{X})]|T\}\Big\}=\operatorname{E}\Big\{\varphi[h(T)]-\operatorname{E}\{\varphi[\delta(\mathbf{X})]|T\}\Big\}=0
	\end{equation*}
	而$\varphi[h(T)]\leqslant\operatorname{E}\{\varphi[\delta(\mathbf{X})]|T\}\;$a.s.于任意的$P\in\mathscr{P}$,根据\cref{prop:MeasurableIntegral}(9)可知$\varphi[h(T)]=\operatorname{E}\{\varphi[\delta(\mathbf{X})]|T\}\;$a.s.于任意的$P\in\mathscr{P}$,即$\delta(\mathbf{X})=\operatorname{E}[\delta(\mathbf{X})|T]=h(T)\;$a.s.于任意的$P\in\mathscr{P}$。于是对任意的$\theta\in\Theta,\;\operatorname{R}[\theta,h(T)]=\operatorname{R}[\theta,\delta(\mathbf{X})]$当且仅当$\delta(\mathbf{X})=\operatorname{E}[\delta(\mathbf{X})|T]=h(T)\;$a.s.于任意的$P\in\mathscr{P}$。\par
	当$\delta(\mathbf{X})$是$g(\theta)$的无偏估计量时,由\cref{prop:ConditionalExpectation}(3)可知$h(T)$也是$g(\theta)$的无偏估计量。
\end{proof}
\begin{corollary}\label{cor:Rao-Blackwell}
	若损失函数$L(\theta,d)$是关于$d$的严格凸函数,则UMRE是充分统计量的函数a.s.于任意的$P\in\mathscr{P}$。
\end{corollary}
\begin{proof}
	设$\delta(\mathbf{X})$是一个UMRE,由\cref{theo:Rao-Blackwell}可知取充分统计量$T$则有:
	\begin{equation*}
		\operatorname{R}\{\operatorname{E}[\delta(\mathbf{X})|T]\}=\operatorname{R}[\delta(\mathbf{X})]
	\end{equation*}
	根据取等条件可知$\delta(\mathbf{X})=\operatorname{E}[\delta(\mathbf{X})|T]=h(T)\;$a.s.于任意的$P\in\mathscr{P}$。
\end{proof}
\begin{definition}
	设$(X,\mathscr{A},\mathscr{P})$是参数结构,$\Theta$是参数空间,$\mathbf{X}$为从总体$F$中抽取的简单样本,$g(\theta)$为待估量,$\delta(\mathbf{X})$是$g(\theta)$的估计量,$T$是$\mathscr{P}$的充分统计量,称:
	\begin{equation*}
		h(T)=\operatorname{E}[\delta(\mathbf{X})|T]
	\end{equation*}
	是$\delta(\mathbf{X})$关于$T$的Rao-Blackwell改进。
\end{definition}
\begin{theorem}[Lehmann-Scheffe Theorem]
	\label{theo:Lehmann-Scheffe}
	设$(X,\mathscr{A},\mathscr{P})$是参数结构,$\Theta$是参数空间,$\mathbf{X}$为从总体$F$中抽取的简单样本,$g(\theta)\in\mathbb{R}^{}$为U可估的待估量,损失函数$L(\theta,d)$是关于$d$的凸函数。若$\mathscr{P}$存在完全充分统计量$S(\mathbf{X})$,则$g(\theta)$的UMRUE存在,任一$g(\theta)$的无偏估计量关于$S(\mathbf{X})$的Rao-Blackwell改进都是UMRUE。若$L(\theta,d)$关于$d$严格凸,则$g(\theta)$的UMRUE在a.s.于任意的$P\in\mathscr{P}$的意义下唯一。
\end{theorem}
\begin{proof}
	任取$g(\theta)$的一个无偏估计量$\delta(\mathbf{X})$,由\cref{theo:Rao-Blackwell}可知$T_1=\operatorname{E}[\delta(\mathbf{X})|S(\mathbf{X})]$仍是$g(\theta)$的一个无偏估计量且风险函数值比$\delta(\mathbf{X})$更小。任取$g(\theta)$的另一无偏估计量$\delta'(\mathbf{X})$,同理可知$T_2=\operatorname{E}[\delta'(\mathbf{X})|S(\mathbf{X})]$仍是$g(\theta)$的一个无偏估计量且风险函数值比$\delta'(\mathbf{X})$更小。因为$g(\theta)\in\mathbb{R}^{}$,所以$\delta(\mathbf{X}),\delta'(\mathbf{X})\in L_1(X)$,于是对任意的$\theta\in\Theta$,由\cref{prop:ConditionalExpectation}(5)(3)、\cref{prop:MeasurableIntegral}(5)可得:
	\begin{align*}
		&\operatorname{E}(T_1-T_2)=\operatorname{E}\{\operatorname{E}[\delta(\mathbf{X})|S(\mathbf{X})]-\operatorname{E}[\delta'(\mathbf{X})|S(\mathbf{X})]\}=\operatorname{E}\{\operatorname{E}[\delta(\mathbf{X})-\delta'(\mathbf{X})|S(\mathbf{X})]\} \\
		=&\operatorname{E}[\delta(\mathbf{X})-\delta'(\mathbf{X})]=\operatorname{E}[\delta(\mathbf{X})]-\operatorname{E}[\delta'(\mathbf{X})]=g(\theta)-g(\theta)=0
	\end{align*}
	根据条件期望的定义可知$T_1-T_2$是关于$S(\mathbf{X})$的可测函数,由完全统计量的定义即可得到$T_1=T_2\;$a.s.于任意的$P\in\mathscr{P}$,$L(\theta,T_1)=L(\theta,T_2)\;$a.s.于任意的$P\in\mathscr{P}$,于是根据\cref{prop:MeasurableIntegral}(7)和\cref{theo:Rao-Blackwell}可得:
	\begin{equation*}
		\operatorname{R}(\delta,T_1)=\operatorname{R}(\delta,T_2)\leqslant\operatorname{R}[\delta,\delta'(\mathbf{X})]
	\end{equation*}
	由$\delta'(\mathbf{X})$的任意性,$T_1$是$g(\theta)$的UMRUE,所以$g(\theta)$的UMRUE存在。由$\delta(\mathbf{X})$的任意性,任一$g(\theta)$的无偏估计量关于$S(\mathbf{X})$的Rao-Blackwell改进都是UMRUE。\par
	对于$g(\theta)$的任意一个UMRUE$\;\delta_1(\mathbf{X})$,它的Rao-Blackwell改进的风险函数值一定等于原本的风险函数值,若$L(\theta,d)$关于$d$严格凸,由\cref{theo:Rao-Blackwell}中的取等条件可知$\delta_1(\mathbf{X})=h[S(\mathbf{X})]\;$a.s.于任意的$P\in\mathscr{P}$,修改其在一个零测集上的数值使得得到的$\delta_1'(\mathbf{X})=h[S(\mathbf{X})]$,由完备性的定义,仿照存在性的证明可得所有经过修改后的UMRUE相等a.s.于任意的$P\in\mathscr{P}$,由\cref{prop:Measure}(3)(次有限可加性)可知原本的UMRUE相等a.s.于任意的$P\in\mathscr{P}$,即UMRUE在a.s.于任意的$P\in\mathscr{P}$的意义下唯一。
\end{proof}
\begin{note}
	上述定理给了寻找UMRUE的第一个方法:如果损失函数$L(\theta,d)$是关于$d$的凸函数,$\mathscr{P}$存在完全充分统计量$S(\mathbf{X})$,则方程组:
	\begin{equation*}
		\forall\;\theta\in\Theta,\;\operatorname{E}\{\delta[S(\mathbf{X})]\}=g(\theta)
	\end{equation*}
	给出的估计量$\delta$即为UMRUE。证明也很简单,满足条件的$\delta[S(\mathbf{X})]$是$g(\theta)$的一个无偏估计,且其关于$S(\mathbf{X})$的Rao-Blackwell改进就是自身。
\end{note}

\subsection{一致最小方差无偏估计}
\begin{definition}
	设$(X,\mathscr{A},\mathscr{P})$是参数结构,$\Theta$是参数空间,$\mathbf{X}$为从总体$F$中抽取的简单样本,$g(\theta)$为存在无偏估计量的待估量。若估计量$\delta(\mathbf{X})$对$g(\theta)$的任一无偏估计量$\delta'(\mathbf{X})$:有\info{链接MSE}:
	\begin{equation*}
		\operatorname{MSE}[\delta(\mathbf{X})]=\operatorname{Var}[\delta(\mathbf{X})]\leqslant\operatorname{MSE}[\delta'(\mathbf{X})]=\operatorname{Var}[\delta'(\mathbf{X})],\;\forall\;\theta\in\Theta
	\end{equation*}
	则称$\delta(\mathbf{X})$为$g(\theta)$的\gls{UMVUE}。
\end{definition}
\begin{note}
	由\cref{prop:ConvexFunction}(1)可知二次函数是严格凸函数,所以前述Rao-Blackwell Theorem和Lehmann-Scheffe Theorem在风险函数为方差时都成立。
\end{note}
%\begin{theorem}[Rao-Blackwell Theorem(MSE)]
%	\label{theo:Rao-BlackwellMSE}
%	设$(X,\mathscr{A},\mathscr{P})$是可控参数结构,$\Theta$是参数空间,$\mathbf{X}$为从总体$F$中抽取的简单样本,$g(\theta)$为U可估的待估量,$\delta(\mathbf{X})$是$g(\theta)$的一个无偏估计量,$T$是$\mathscr{P}$的充分统计量,则:
%	\begin{equation*}
%		h(T)=\operatorname{E}[\delta(\mathbf{X})|T]
%	\end{equation*}
%	是$g(\theta)$的无偏估计,并且有:
%	\begin{equation*}
%		\forall\;\theta\in\Theta,\;\operatorname{Var}[h(T)]\leqslant\operatorname{Var}[\delta(\mathbf{X})]
%	\end{equation*}
%	等号成立当且仅当$\delta(\mathbf{X})=h(T)\;$a.s.于任意的$P\in\mathscr{P}$。
%\end{theorem}
%\begin{proof}
%	因为$T$是充分统计量,所以$P(\mathbf{X}|T)$与$\theta$无关,$h(T)=\operatorname{E}[\delta(\mathbf{X})|T]$也与$\theta$无关,由条件期望的定义可得$h(T)$可测,于是$h(T)$是一个统计量。由\cref{prop:ConditionalExpectation}(3)可得:
%	\begin{equation*}
%		\operatorname{E}[h(T)]=\operatorname{E}\{\operatorname{E}[\delta(\mathbf{X})|T]\}=\operatorname{E}[\delta(\mathbf{X})]=g(\theta)
%	\end{equation*}
%	所以$h(T)$是一个无偏估计量。由\cref{prop:MeasurableIntegral}(5)可得对任意的$\theta\in\Theta$有:
%	\begin{align*}
%		\operatorname{Var}[\delta(\mathbf{X})]&=\operatorname{E}\{[\delta(\mathbf{X})-g(\theta)]^2\}=\operatorname{E}\{[\delta(\mathbf{X})-h(T)+h(T)-g(\theta)]^2\} \\
%		&=\operatorname{E}\{[\delta(\mathbf{X})-h(T)]^2+[h(T)-g(\theta)]^2+2[\delta(\mathbf{X})-h(T)][h(T)-g(\theta)]\} \\
%		&=\operatorname{Var}[h(T)]+\operatorname{E}\{[\delta(\mathbf{X})-h(T)]^2\}+\operatorname{E}\{2[\delta(\mathbf{X})-h(T)][h(T)-g(\theta)]\}
%	\end{align*}
%	由\info{条件期望线性运算}可得:
%	\begin{align*}
%		&\operatorname{E}\{2[\delta(\mathbf{X})-h(T)][h(T)-g(\theta)]\} =\operatorname{E}\Big\{\operatorname{E}\{2[\delta(\mathbf{X})-h(T)][h(T)-g(\theta)]\}|T\Big\} \\
%		=&\operatorname{E}\Big\{2[h(T)-g(\theta)]\operatorname{E}[\delta(\mathbf{X})-h(T)|T]\Big\} =\operatorname{E}\Big\{2[h(T)-g(\theta)]\{\operatorname{E}[\delta(\mathbf{X})|T]-\operatorname{E}[h(T)]\}\Big\} \\
%		=&\operatorname{E}\{2[h(T)-g(\theta)]\cdot0\}=0
%	\end{align*}
%	由\cref{prop:NonnegativeMeasurableIntegral}(2)可得:
%	\begin{equation*}
%		\operatorname{Var}[\delta(\mathbf{X})]=\operatorname{Var}[h(T)]+\operatorname{E}\{[\delta(\mathbf{X})-h(T)]^2\}\geqslant\operatorname{Var}[h(T)]
%	\end{equation*}
%	等号成立当且仅当$\operatorname{E}\{[\delta(\mathbf{X})-h(T)]^2\}=0$,由\cref{prop:NonnegativeMeasurableIntegral}(9)可知当且仅当$\delta(\mathbf{X})=h(T)\;$a.s.于任意的$P\in\mathscr{P}$。
%\end{proof}
%\begin{corollary}\label{cor:Rao-BlackwellMSE}
%	UMVUE是充分统计量的函数a.s.于任意的$P\in\mathscr{P}$。
%\end{corollary}
%\begin{proof}
%	设$\delta(\mathbf{X})$是一个UMVUE,由\cref{theo:Rao-Blackwell}可知取充分统计量$T$则有:
%	\begin{equation*}
%		\operatorname{Var}\{\operatorname{E}[\delta(\mathbf{X})|T]\}=\operatorname{Var}[\delta(\mathbf{X})]
%	\end{equation*}
%	根据取等条件可知$\delta(\mathbf{X})=\operatorname{E}[\delta(\mathbf{X})|T]\;$a.s.于任意的$P\in\mathscr{P}$。
%\end{proof}
\begin{theorem}\label{theo:UMVUE0UnbiasedEstimation}
	设$(X,\mathscr{A},\mathscr{P})$是可控参数结构,$\Theta$是参数空间,$\mathbf{X}$为从总体$F$中抽取的简单样本,$g(\theta)$为U可估的待估量,$\delta(\mathbf{X})$是$g(\theta)$的一个无偏估计量,对任意的$\theta\in\Theta$有$\operatorname{Var}[\delta(\mathbf{X})]<+\infty$。令:
	\begin{equation*}
		A=\{f(\mathbf{X}):\forall\;\theta\in\Theta,\;\operatorname{E}[f(\mathbf{X})]=0\}
	\end{equation*}
	$\delta(\mathbf{X})$是$g(\theta)$的UMVUE的充要条件为:
	\begin{equation*}
		\forall\;\theta\in\Theta,\;\forall\;f(\mathbf{X})\in A,\;\operatorname{Cov}[\delta(\mathbf{X}),f(\mathbf{X})]=\operatorname{E}[\delta(\mathbf{X})\cdot f(\mathbf{X})]=0
	\end{equation*}
\end{theorem}
\begin{proof}
	由\cref{prop:MeasurableIntegral}(5)可得:
	\begin{align*}
		&\operatorname{Cov}[\delta(\mathbf{X}),f(\mathbf{X})]=\operatorname{E}\{[\delta(\mathbf{X})-g(\theta)]f(\mathbf{X})\} \\
		=&\operatorname{E}[\delta(\mathbf{X})\cdot f(\mathbf{X})]-g(\theta)\operatorname{E}[f(\mathbf{X})]=\operatorname{E}[\delta(\mathbf{X})\cdot f(\mathbf{X})]
	\end{align*}\par
	\textbf{(1)充分性:}对任意$g(\theta)$的无偏估计$\delta'(\mathbf{X})$,都存在一个$f(\mathbf{X})\in A$使得$\delta'(\mathbf{X})=\delta(\mathbf{X})+f(\mathbf{X})$。由\cref{prop:Variance}(3)和\cref{prop:NonnegativeMeasurableIntegral}(2)可得:
	\begin{align*}
		\operatorname{Var}[\delta'(\mathbf{X})]&=\operatorname{Var}[\delta(\mathbf{X})+f(\mathbf{X})]=\operatorname{Var}[\delta(\mathbf{X})]+\operatorname{Var}[f(\mathbf{X})]+2\operatorname{Cov}[\delta(\mathbf{X}),f(\mathbf{X})] \\
		&=\operatorname{Var}[\delta(\mathbf{X})]+\operatorname{Var}[f(\mathbf{X})]\geqslant\operatorname{Var}[\delta(\mathbf{X})]
	\end{align*}
	所以$\delta(\mathbf{X})$是$g(\theta)$的UMVUE。\par
	\textbf{(2)必要性:}因为$\delta(\mathbf{X})$是$g(\theta)$的UMVUE,对任意的$f(\mathbf{X})\in A$和任意的$a\in\mathbb{R}^{}$有$\delta(\mathbf{X})+af(\mathbf{X})$是$g(\theta)$的无偏估计量,所以由\cref{prop:Variance}(3)、\cref{prop:CovMat}(3)和\cref{prop:NonnegativeMeasurableIntegral}(10)可得:
	\begin{align*}
		&\operatorname{Var}[\delta(\mathbf{X})]\leqslant\operatorname{Var}[\delta(\mathbf{X})+af(\mathbf{X})]=\operatorname{Var}[\delta(\mathbf{X})]+2\operatorname{Cov}[\delta(\mathbf{X}),af(\mathbf{X})]+\operatorname{Var}[af(\mathbf{X})] \\
		=&\operatorname{Var}[\delta(\mathbf{X})]+2a\operatorname{Cov}[\delta(\mathbf{X}),f(\mathbf{X})]+a^2\operatorname{Var}[f(\mathbf{X})],\;\forall\;\theta\in\Theta
	\end{align*}
	即:
	\begin{equation*}
		2a\operatorname{Cov}[\delta(\mathbf{X}),f(\mathbf{X})]+a^2\operatorname{Var}[f(\mathbf{X})]\geqslant0,\;\forall\;a\in\mathbb{R}^{}
	\end{equation*}
	将上式看作关于$a$的一元二次方程,由判别式可得:
	\begin{equation*}
		4\operatorname{Cov}^2[\delta(\mathbf{X}),f(\mathbf{X})]\leqslant0
	\end{equation*}
	即$\operatorname{Cov}[\delta(\mathbf{X}),f(\mathbf{X})]=0$。由$f(\mathbf{X})$的任意性,必要性成立。
\end{proof}