\chapter{点估计理论}

\begin{definition}
	设$(X,\mathscr{A},\mathscr{P})$是参数结构,$\Theta$是参数空间,$\mathbf{X}=(\seq{X}{n})$为从总体$F$中抽取的简单样本,$g$是定义在$\Theta$上的函数,它是一个\gls{Estimand},一般默认它为实数。若将样本的可测函数$\delta(\mathbf{X})$作为$g(\theta)$的估计,称这种估计方式为\gls{PointEstimation},$\delta(\mathbf{X})$被称为$g(\theta)$的\gls{Estimator}。
\end{definition}
\begin{definition}
	设$(X,\mathscr{A},\mathscr{P})$是参数统计结构,$\Theta$是参数空间,$\mathbf{X}=(\seq{X}{n})$为从总体$F$中抽取的简单样本,$g(\theta)$是定义在$\Theta$上的函数,$\delta(\mathbf{X})$是$g(\theta)$的一个估计量。如果:
	\begin{equation*}
		\forall\;\theta\in\Theta,\;\operatorname{E}_{\theta}[\delta(\mathbf{X})]=g(\theta)
	\end{equation*}
	则称$\delta(\mathbf{X})$为$g(\theta)$的一个\gls{UnbiasedEstimation}。若$g(\theta)$存在无偏估计量,则称$g(\theta)$\gls{UEstimable}。
\end{definition}
\begin{property}\label{prop:0UnbiasedEstimator}
	称满足$\operatorname{E}_{\theta}[\delta(\mathbf{X})]=0$的估计量$\delta(\mathbf{X})$为\textbf{零无偏估计量},它具有如下性质:
	\begin{enumerate}
		\item 设$\delta$是待估量$g(\theta)\in\mathbb{R}^{}$的无偏估计量,所有$g(\theta)$的无偏估计量为$\{\delta-f:\operatorname{E}_{\theta}[f(\mathbf{X})]=0\}$;
		\item 
	\end{enumerate}
\end{property}
\begin{proof}
	(1)估计量$\delta_1$为$g(\theta)$的无偏估计量当且仅当$\operatorname{E}_{\theta}(\delta)=\operatorname{E}_{\theta}(\delta_1)-0$,由\cref{prop:MeasurableIntegral}(5)可得充分性。假设$\delta_2$不可以表示为$\delta$与一个零无偏估计$f$的和,由\cref{prop:MeasurableIntegral}(5)可知$\operatorname{E}_{\theta}(\delta_2-\delta)=\operatorname{E}(\delta_2)-\operatorname{E}(\delta)\ne0$,$\delta_2$不是$g(\theta)$的无偏估计量,必要性得证。
\end{proof}
\begin{definition}
	设$(X,\mathscr{A},\mathscr{P})$是参数结构,$\Theta$是参数空间,$\mathbf{X}=(\seq{X}{n})$为从总体$F$中抽取的简单样本,$g(\theta)$是定义在$\Theta$上的函数,$\delta(\mathbf{X})$是$g(\theta)$的一个估计量。如果:
	\begin{equation*}
		\forall\;\theta\in\Theta,\;\lim_{n\to+\infty}\operatorname{E}_{\theta}[\delta_n(\mathbf{X})]=g(\theta)
	\end{equation*}
	则称$\delta_n(\mathbf{X})$为$g(\theta)$的一个\gls{AsymptoticallyUnbiasedEstimation}。
\end{definition}
\begin{definition}
	设$(X,\mathscr{A},\mathscr{P})$是参数结构,$\Theta$是参数空间,$\mathbf{X}=(\seq{X}{n})$为从总体$F$中抽取的简单样本,$g(\theta)$是定义在$\Theta$上的函数,$\delta(\mathbf{X})$是$g(\theta)$的一个估计量,$\delta_1(\mathbf{X}),\delta_2(\mathbf{X})$是$g(\theta)$的两个不同的无偏估计量。若:
	\begin{equation*}
		\forall\;\theta\in\Theta,\;\operatorname{Var}_{\theta}[\delta_1(\mathbf{X})]\leqslant\operatorname{Var}_{\theta}[\delta_2(\mathbf{X})]
	\end{equation*}
	且至少存在一个$\theta\in\Theta$使得小于号成立,则称估计量$\delta_1(\mathbf{X})$比$\delta_2(\mathbf{X})$\textbf{有效}。
\end{definition}
\begin{definition}
	设$(X,\mathscr{A},\mathscr{P})$是参数结构,$\Theta$是参数空间,$\mathbf{X_n}=(\seq{X}{n})$为从总体$F$中抽取的简单样本,$g(\theta)$是定义在$\Theta$上的函数,$\delta(\mathbf{X})$是$g(\theta)$的一个估计量。若$\delta(\mathbf{X_n})\overset{P}{\longrightarrow}g(\theta)$,则称$\delta(\mathbf{X})$是$g(\theta)$的\gls{WeaklyConsistentEstimation}。若$\delta(\mathbf{X_n})\overset{\text{a.s.}}{\longrightarrow}g(\theta)$,则称$\delta(\mathbf{X})$是$g(\theta)$的\gls{StronglyConsistentEstimation}。若:
	\begin{equation*}
		\lim_{n\to+\infty}\operatorname{E}_{\theta}[|\delta(\mathbf{X_n})-g(\theta)|^r]=0
	\end{equation*}
	则称$\delta(\mathbf{X})$是$g(\theta)$的\textbf{r阶矩相合估计},当$r=2$时称$\delta(\mathbf{X})$是$g(\theta)$的\textbf{均方相合估计}。
\end{definition}

\section{一致最小风险无偏估计}

\begin{definition}
	设$(X,\mathscr{A},\mathscr{P})$是参数结构,$\Theta$是参数空间,$\mathbf{X}$为从总体$F$中抽取的简单样本,$g(\theta)$为待估量,$\operatorname{R}(\theta,d)$为风险函数。若估计量$\delta(\mathbf{X})$对$g(\theta)$的任一估计量$\delta'(\mathbf{X})$有:
	\begin{equation*}
		\forall\;\theta\in\Theta,\;\operatorname{R}[\theta,\delta(\mathbf{X})]\leqslant\operatorname{R}[\theta,\delta'(\mathbf{X})]
	\end{equation*}
	则称$\delta(\mathbf{X})$为$g(\theta)$的\gls{UMRE}。
\end{definition}
\begin{note}
	估计量的一致最小风险估计常不存在。\par
	设$\delta(\mathbf{X})$是$g(\theta)$的一致风险估计。若风险函数$\operatorname{R}(\theta,d)$存在关于$d$的最小值,任取$\theta_0\in\Theta$,我们总能可以取一个有偏好的$\delta'(\mathbf{X})=\arg\min\operatorname{R}(\theta,d)$,那么$\delta(\mathbf{X})$的风险函数在$\theta_0$处的取值也应是最小值。由$\theta_0$的任意性,$\operatorname{R}[\theta,\delta(\mathbf{X})]$需要在整个参数空间$\Theta$上都取到风险函数的最小值,这显然是不太可能存在的。若风险函数不存在关于$d$的最小值,那么一致最小风险估计当然不存在。\par
	考虑到上述情况,我们转向研究在某一估计量族中寻找一致最小风险估计,而不是在所有估计量中去寻找。人们关注最多的便是在无偏估计量族中的情况。
\end{note}
\begin{definition}
	设$(X,\mathscr{A},\mathscr{P})$是参数结构,$\Theta$是参数空间,$\mathbf{X}$为从总体$F$中抽取的简单样本,$g(\theta)$为存在无偏估计量的待估量,$\operatorname{R}(\theta,d)$为风险函数。若估计量$\delta(\mathbf{X})$对$g(\theta)$的任一无偏估计量$\delta'(\mathbf{X})$有:
	\begin{equation*}
		\forall\;\theta\in\Theta,\;\operatorname{R}[\theta,\delta(\mathbf{X})]\leqslant\operatorname{R}[\theta,\delta'(\mathbf{X})]
	\end{equation*}
	则称$\delta(\mathbf{X})$为$g(\theta)$的\gls{UMRUE}。
\end{definition}
\begin{theorem}[Rao-Blackwell Theorem]
	\label{theo:Rao-Blackwell}
	设$(X,\mathscr{A},\mathscr{P})$是参数结构,$\Theta$是参数空间,$\mathbf{X}$为从总体$F$中抽取的简单样本,$g(\theta)$为待估量,$\delta(\mathbf{X})$是$g(\theta)$的估计量,$T$是$\mathscr{P}$的充分统计量,损失函数$L(\theta,d)$是关于$d$的凸函数。若$\delta(\mathbf{X}),L[\theta,\delta(\mathbf{X})]\in L_1(X)$,则:
	\begin{equation*}
		h(T)=\operatorname{E}_{\theta}[\delta(\mathbf{X})|T]
	\end{equation*}
	满足:
	\begin{equation*}
		\forall\;\theta\in\Theta,\;\operatorname{R}[\theta,h(T)]\leqslant\operatorname{R}[\theta,\delta(\mathbf{X})]
	\end{equation*}
	若$L(\theta,d)$关于$d$严格凸,则等号成立当且仅当$\delta(\mathbf{X})=\operatorname{E}_{\theta}[\delta(\mathbf{X})|T]=h(T)\;$a.s.于任意的$P\in\mathscr{P}$。若$\delta(\mathbf{X})$是$g(\theta)$的无偏估计量,则$h(T)$也是$g(\theta)$的无偏估计量。
\end{theorem}
\begin{proof}
	因为$T$是充分统计量,所以$P(\mathbf{X}|T)$与$\theta$无关,$h(T)=\operatorname{E}_{\theta}[\delta(\mathbf{X})|T]$也与$\theta$无关,由条件期望的定义可得$h(T)$可测,于是$h(T)$是一个统计量。因为$\delta(\mathbf{X})\in L_1(X)$,由\cref{prop:ConditionalExpectation}(3)可知$h(T)\in L_1(X)$。\par
	在\cref{ineq:Jensen}中取$\varphi(d)=L(\theta,d)$可得:
	\begin{equation*}
		\forall\;P\in\mathscr{P},\;\varphi[h(T)]=\varphi\{\operatorname{E}_{\theta}[\delta(\mathbf{X})|T]\}\leqslant\operatorname{E}_{\theta}\{\varphi[\delta(\mathbf{X})]|T\}\;\text{a.s.于}P
	\end{equation*}
	当$L(\theta,d)$关于$d$严格凸时等号成立a.s.于任意的$P\in\mathscr{P}$当且仅当$\delta(\mathbf{X})=\operatorname{E}_{\theta}[\delta(\mathbf{X})|T]=h(T)\;$a.s.于任意的$P\in\mathscr{P}$。根据\cref{prop:MeasurableIntegral}(6),对上式两边同时求期望可得:
	\begin{equation*}
		\operatorname{E}_{\theta}\{L[\theta,h(T)]\}\leqslant\operatorname{E}_{\theta}\Big\{\operatorname{E}_{\theta}\{\varphi[\delta(\mathbf{X})]|T\}\Big\}=\operatorname{E}_{\theta}\{\varphi[\delta(\mathbf{X})]\}=\operatorname{E}_{\theta}\{L[\theta,\delta(\mathbf{X})]\}=\operatorname{R}[\theta,\delta(\mathbf{X})]
	\end{equation*}
	若$L(\theta,d)$关于$d$严格凸,由\cref{prop:MeasurableIntegral}(7)可知当$\delta(\mathbf{X})=\operatorname{E}_{\theta}[\delta(\mathbf{X})|T]=h(T)\;$a.s.于任意的$P\in\mathscr{P}$时上式等号成立。当上式等号成立时,因为$L[\theta,\delta(\mathbf{X})]\in L_1$,由\cref{prop:MeasurableIntegral}(5)可得:
	\begin{equation*}
		\operatorname{E}_{\theta}\{L[\theta,h(T)]\}-\operatorname{E}_{\theta}\Big\{\operatorname{E}_{\theta}\{\varphi[\delta(\mathbf{X})]|T\}\Big\}=\operatorname{E}_{\theta}\Big\{\varphi[h(T)]-\operatorname{E}_{\theta}\{\varphi[\delta(\mathbf{X})]|T\}\Big\}=0
	\end{equation*}
	而$\varphi[h(T)]\leqslant\operatorname{E}_{\theta}\{\varphi[\delta(\mathbf{X})]|T\}\;$a.s.于任意的$P\in\mathscr{P}$,根据\cref{prop:MeasurableIntegral}(9)可知$\varphi[h(T)]=\operatorname{E}_{\theta}\{\varphi[\delta(\mathbf{X})]|T\}\;$a.s.于任意的$P\in\mathscr{P}$,即$\delta(\mathbf{X})=\operatorname{E}_{\theta}[\delta(\mathbf{X})|T]=h(T)\;$a.s.于任意的$P\in\mathscr{P}$。于是对任意的$\theta\in\Theta,\;\operatorname{R}[\theta,h(T)]=\operatorname{R}[\theta,\delta(\mathbf{X})]$当且仅当$\delta(\mathbf{X})=\operatorname{E}_{\theta}[\delta(\mathbf{X})|T]=h(T)\;$a.s.于任意的$P\in\mathscr{P}$。\par
	当$\delta(\mathbf{X})$是$g(\theta)$的无偏估计量时,由\cref{prop:ConditionalExpectation}(3)可知$h(T)$也是$g(\theta)$的无偏估计量。
\end{proof}
\begin{corollary}\label{cor:Rao-Blackwell}
	若损失函数$L(\theta,d)$是关于$d$的严格凸函数,则UMRE是充分统计量的函数a.s.于任意的$P\in\mathscr{P}$。
\end{corollary}
\begin{proof}
	设$\delta(\mathbf{X})$是一个UMRE,由\cref{theo:Rao-Blackwell}可知取充分统计量$T$则有:
	\begin{equation*}
		\operatorname{R}\{\operatorname{E}_{\theta}[\delta(\mathbf{X})|T]\}=\operatorname{R}[\delta(\mathbf{X})]
	\end{equation*}
	根据取等条件可知$\delta(\mathbf{X})=\operatorname{E}_{\theta}[\delta(\mathbf{X})|T]=h(T)\;$a.s.于任意的$P\in\mathscr{P}$。
\end{proof}
\begin{definition}
	设$(X,\mathscr{A},\mathscr{P})$是参数结构,$\Theta$是参数空间,$\mathbf{X}$为从总体$F$中抽取的简单样本,$g(\theta)$为待估量,$\delta(\mathbf{X})$是$g(\theta)$的估计量,$T$是$\mathscr{P}$的充分统计量,称:
	\begin{equation*}
		h(T)=\operatorname{E}_{\theta}[\delta(\mathbf{X})|T]
	\end{equation*}
	是$\delta(\mathbf{X})$关于$T$的Rao-Blackwell改进。
\end{definition}
\begin{theorem}[Lehmann-Scheffe Theorem]
	\label{theo:Lehmann-Scheffe}
	设$(X,\mathscr{A},\mathscr{P})$是参数结构,$\Theta$是参数空间,$\mathbf{X}$为从总体$F$中抽取的简单样本,$g(\theta)\in\mathbb{R}^{}$为U可估的待估量,损失函数$L(\theta,d)$是关于$d$的凸函数。若$\mathscr{P}$存在完全充分统计量$S(\mathbf{X})$,则$g(\theta)$的UMRUE存在,任一$g(\theta)$的无偏估计量关于$S(\mathbf{X})$的Rao-Blackwell改进都是UMRUE。若$L(\theta,d)$关于$d$严格凸,则$g(\theta)$的UMRUE在a.s.于任意的$P\in\mathscr{P}$的意义下唯一。
\end{theorem}
\begin{proof}
	任取$g(\theta)$的一个无偏估计量$\delta(\mathbf{X})$,由\cref{theo:Rao-Blackwell}可知$T_1=\operatorname{E}_{\theta}[\delta(\mathbf{X})|S(\mathbf{X})]$仍是$g(\theta)$的一个无偏估计量且风险函数值比$\delta(\mathbf{X})$更小。任取$g(\theta)$的另一无偏估计量$\delta'(\mathbf{X})$,同理可知$T_2=\operatorname{E}_{\theta}[\delta'(\mathbf{X})|S(\mathbf{X})]$仍是$g(\theta)$的一个无偏估计量且风险函数值比$\delta'(\mathbf{X})$更小。因为$g(\theta)\in\mathbb{R}^{}$,所以$\delta(\mathbf{X}),\delta'(\mathbf{X})\in L_1(X)$,于是对任意的$\theta\in\Theta$,由\cref{prop:ConditionalExpectation}(5)(3)、\cref{prop:MeasurableIntegral}(5)可得:
	\begin{align*}
		&\operatorname{E}_{\theta}(T_1-T_2)=\operatorname{E}_{\theta}\{\operatorname{E}_{\theta}[\delta(\mathbf{X})|S(\mathbf{X})]-\operatorname{E}_{\theta}[\delta'(\mathbf{X})|S(\mathbf{X})]\}=\operatorname{E}_{\theta}\{\operatorname{E}_{\theta}[\delta(\mathbf{X})-\delta'(\mathbf{X})|S(\mathbf{X})]\} \\
		=&\operatorname{E}_{\theta}[\delta(\mathbf{X})-\delta'(\mathbf{X})]=\operatorname{E}_{\theta}[\delta(\mathbf{X})]-\operatorname{E}_{\theta}[\delta'(\mathbf{X})]=g(\theta)-g(\theta)=0
	\end{align*}
	根据条件期望的定义可知$T_1-T_2$是关于$S(\mathbf{X})$的可测函数,由完全统计量的定义即可得到$T_1=T_2\;$a.s.于任意的$P\in\mathscr{P}$,$L(\theta,T_1)=L(\theta,T_2)\;$a.s.于任意的$P\in\mathscr{P}$,于是根据\cref{prop:MeasurableIntegral}(7)和\cref{theo:Rao-Blackwell}可得:
	\begin{equation*}
		\operatorname{R}(\delta,T_1)=\operatorname{R}(\delta,T_2)\leqslant\operatorname{R}[\delta,\delta'(\mathbf{X})]
	\end{equation*}
	由$\delta'(\mathbf{X})$的任意性,$T_1$是$g(\theta)$的UMRUE,所以$g(\theta)$的UMRUE存在。由$\delta(\mathbf{X})$的任意性,任一$g(\theta)$的无偏估计量关于$S(\mathbf{X})$的Rao-Blackwell改进都是UMRUE。\par
	对于$g(\theta)$的任意一个UMRUE$\;\delta_1(\mathbf{X})$,它的Rao-Blackwell改进的风险函数值一定等于原本的风险函数值,若$L(\theta,d)$关于$d$严格凸,由\cref{theo:Rao-Blackwell}中的取等条件可知$\delta_1(\mathbf{X})=h[S(\mathbf{X})]\;$a.s.于任意的$P\in\mathscr{P}$,修改其在一个零测集上的数值使得得到的$\delta_1'(\mathbf{X})=h[S(\mathbf{X})]$,由完备性的定义,仿照存在性的证明可得所有经过修改后的UMRUE相等a.s.于任意的$P\in\mathscr{P}$,由\cref{prop:Measure}(3)(次有限可加性)可知原本的UMRUE相等a.s.于任意的$P\in\mathscr{P}$,即UMRUE在a.s.于任意的$P\in\mathscr{P}$的意义下唯一。
\end{proof}
\begin{note}
	上述定理给了寻找UMRUE的第一个方法:如果损失函数$L(\theta,d)$是关于$d$的凸函数,$\mathscr{P}$存在完全充分统计量$S(\mathbf{X})$,则方程组:
	\begin{equation*}
		\forall\;\theta\in\Theta,\;\operatorname{E}_{\theta}\{\delta[S(\mathbf{X})]\}=g(\theta)
	\end{equation*}
	给出的估计量$\delta$即为UMRUE。证明也很简单,满足条件的$\delta[S(\mathbf{X})]$是$g(\theta)$的一个无偏估计,且其关于$S(\mathbf{X})$的Rao-Blackwell改进就是自身。
\end{note}

\subsection{一致最小方差无偏估计}
\begin{definition}
	设$(X,\mathscr{A},\mathscr{P})$是参数结构,$\Theta$是参数空间,$\mathbf{X}$为从总体$F$中抽取的简单样本,$g(\theta)$为存在无偏估计量的待估量。若估计量$\delta(\mathbf{X})$对$g(\theta)$的任一无偏估计量$\delta'(\mathbf{X})$:有\info{链接MSE}:
	\begin{equation*}
		\operatorname{MSE}_{\theta}[\delta(\mathbf{X})]=\operatorname{Var}_{\theta}[\delta(\mathbf{X})]\leqslant\operatorname{MSE}_{\theta}[\delta'(\mathbf{X})]=\operatorname{Var}_{\theta}[\delta'(\mathbf{X})],\;\forall\;\theta\in\Theta
	\end{equation*}
	则称$\delta(\mathbf{X})$为$g(\theta)$的\gls{UMVUE}。
\end{definition}
\begin{note}
	由\cref{prop:ConvexFunction}(1)可知二次函数是严格凸函数,所以前述Rao-Blackwell Theorem和Lehmann-Scheffe Theorem在风险函数为方差时都成立。
\end{note}
%\begin{theorem}[Rao-Blackwell Theorem(MSE)]
%	\label{theo:Rao-BlackwellMSE}
%	设$(X,\mathscr{A},\mathscr{P})$是可控参数结构,$\Theta$是参数空间,$\mathbf{X}$为从总体$F$中抽取的简单样本,$g(\theta)$为U可估的待估量,$\delta(\mathbf{X})$是$g(\theta)$的一个无偏估计量,$T$是$\mathscr{P}$的充分统计量,则:
%	\begin{equation*}
%		h(T)=\operatorname{E}[\delta(\mathbf{X})|T]
%	\end{equation*}
%	是$g(\theta)$的无偏估计,并且有:
%	\begin{equation*}
%		\forall\;\theta\in\Theta,\;\operatorname{Var}[h(T)]\leqslant\operatorname{Var}[\delta(\mathbf{X})]
%	\end{equation*}
%	等号成立当且仅当$\delta(\mathbf{X})=h(T)\;$a.s.于任意的$P\in\mathscr{P}$。
%\end{theorem}
%\begin{proof}
%	因为$T$是充分统计量,所以$P(\mathbf{X}|T)$与$\theta$无关,$h(T)=\operatorname{E}[\delta(\mathbf{X})|T]$也与$\theta$无关,由条件期望的定义可得$h(T)$可测,于是$h(T)$是一个统计量。由\cref{prop:ConditionalExpectation}(3)可得:
%	\begin{equation*}
%		\operatorname{E}[h(T)]=\operatorname{E}\{\operatorname{E}[\delta(\mathbf{X})|T]\}=\operatorname{E}[\delta(\mathbf{X})]=g(\theta)
%	\end{equation*}
%	所以$h(T)$是一个无偏估计量。由\cref{prop:MeasurableIntegral}(5)可得对任意的$\theta\in\Theta$有:
%	\begin{align*}
%		\operatorname{Var}[\delta(\mathbf{X})]&=\operatorname{E}\{[\delta(\mathbf{X})-g(\theta)]^2\}=\operatorname{E}\{[\delta(\mathbf{X})-h(T)+h(T)-g(\theta)]^2\} \\
%		&=\operatorname{E}\{[\delta(\mathbf{X})-h(T)]^2+[h(T)-g(\theta)]^2+2[\delta(\mathbf{X})-h(T)][h(T)-g(\theta)]\} \\
%		&=\operatorname{Var}[h(T)]+\operatorname{E}\{[\delta(\mathbf{X})-h(T)]^2\}+\operatorname{E}\{2[\delta(\mathbf{X})-h(T)][h(T)-g(\theta)]\}
%	\end{align*}
%	由\info{条件期望线性运算}可得:
%	\begin{align*}
%		&\operatorname{E}\{2[\delta(\mathbf{X})-h(T)][h(T)-g(\theta)]\} =\operatorname{E}\Big\{\operatorname{E}\{2[\delta(\mathbf{X})-h(T)][h(T)-g(\theta)]\}|T\Big\} \\
%		=&\operatorname{E}\Big\{2[h(T)-g(\theta)]\operatorname{E}[\delta(\mathbf{X})-h(T)|T]\Big\} =\operatorname{E}\Big\{2[h(T)-g(\theta)]\{\operatorname{E}[\delta(\mathbf{X})|T]-\operatorname{E}[h(T)]\}\Big\} \\
%		=&\operatorname{E}\{2[h(T)-g(\theta)]\cdot0\}=0
%	\end{align*}
%	由\cref{prop:NonnegativeMeasurableIntegral}(2)可得:
%	\begin{equation*}
%		\operatorname{Var}[\delta(\mathbf{X})]=\operatorname{Var}[h(T)]+\operatorname{E}\{[\delta(\mathbf{X})-h(T)]^2\}\geqslant\operatorname{Var}[h(T)]
%	\end{equation*}
%	等号成立当且仅当$\operatorname{E}\{[\delta(\mathbf{X})-h(T)]^2\}=0$,由\cref{prop:NonnegativeMeasurableIntegral}(9)可知当且仅当$\delta(\mathbf{X})=h(T)\;$a.s.于任意的$P\in\mathscr{P}$。
%\end{proof}
%\begin{corollary}\label{cor:Rao-BlackwellMSE}
%	UMVUE是充分统计量的函数a.s.于任意的$P\in\mathscr{P}$。
%\end{corollary}
%\begin{proof}
%	设$\delta(\mathbf{X})$是一个UMVUE,由\cref{theo:Rao-Blackwell}可知取充分统计量$T$则有:
%	\begin{equation*}
%		\operatorname{Var}\{\operatorname{E}[\delta(\mathbf{X})|T]\}=\operatorname{Var}[\delta(\mathbf{X})]
%	\end{equation*}
%	根据取等条件可知$\delta(\mathbf{X})=\operatorname{E}[\delta(\mathbf{X})|T]\;$a.s.于任意的$P\in\mathscr{P}$。
%\end{proof}
\begin{theorem}\label{theo:UMVUE0UnbiasedEstimation}
	设$(X,\mathscr{A},\mathscr{P})$是参数结构,$\Theta$是参数空间,$\mathbf{X}$为从总体$F$中抽取的简单样本,$g(\theta)$为U可估的待估量,$\delta(\mathbf{X})$是$g(\theta)$的一个无偏估计量,对任意的$\theta\in\Theta$有$\operatorname{Var}_{\theta}[\delta(\mathbf{X})]<+\infty$。令:
	\begin{equation*}
		A=\{f(\mathbf{X}):\forall\;\theta\in\Theta,\;\operatorname{E}_{\theta}[f(\mathbf{X})]=0\}
	\end{equation*}
	$\delta(\mathbf{X})$是$g(\theta)$的UMVUE的充要条件为:
	\begin{equation*}
		\forall\;\theta\in\Theta,\;\forall\;f(\mathbf{X})\in A,\;\operatorname{Cov}_{\theta}[\delta(\mathbf{X}),f(\mathbf{X})]=\operatorname{E}_{\theta}[\delta(\mathbf{X})\cdot f(\mathbf{X})]=0
	\end{equation*}
\end{theorem}
\begin{proof}
	由\cref{prop:MeasurableIntegral}(5)可得:
	\begin{align*}
		&\operatorname{Cov}_{\theta}[\delta(\mathbf{X}),f(\mathbf{X})]=\operatorname{E}_{\theta}\{[\delta(\mathbf{X})-g(\theta)]f(\mathbf{X})\} \\
		=&\operatorname{E}_{\theta}[\delta(\mathbf{X})\cdot f(\mathbf{X})]-g(\theta)\operatorname{E}_{\theta}[f(\mathbf{X})]=\operatorname{E}_{\theta}[\delta(\mathbf{X})\cdot f(\mathbf{X})]
	\end{align*}\par
	\textbf{(1)充分性:}对任意$g(\theta)$的无偏估计$\delta'(\mathbf{X})$,都存在一个$f(\mathbf{X})\in A$使得$\delta'(\mathbf{X})=\delta(\mathbf{X})+f(\mathbf{X})$。由\cref{prop:Variance}(3)和\cref{prop:NonnegativeMeasurableIntegral}(2)可得:
	\begin{align*}
		\operatorname{Var}_{\theta}[\delta'(\mathbf{X})]&=\operatorname{Var}_{\theta}[\delta(\mathbf{X})+f(\mathbf{X})]=\operatorname{Var}_{\theta}[\delta(\mathbf{X})]+\operatorname{Var}_{\theta}[f(\mathbf{X})]+2\operatorname{Cov}_{\theta}[\delta(\mathbf{X}),f(\mathbf{X})] \\
		&=\operatorname{Var}_{\theta}[\delta(\mathbf{X})]+\operatorname{Var}_{\theta}[f(\mathbf{X})]\geqslant\operatorname{Var}_{\theta}[\delta(\mathbf{X})]
	\end{align*}
	所以$\delta(\mathbf{X})$是$g(\theta)$的UMVUE。\par
	\textbf{(2)必要性:}因为$\delta(\mathbf{X})$是$g(\theta)$的UMVUE,对任意的$f(\mathbf{X})\in A$和任意的$a\in\mathbb{R}^{}$有$\delta(\mathbf{X})+af(\mathbf{X})$是$g(\theta)$的无偏估计量,所以由\cref{prop:Variance}(3)、\cref{prop:CovMat}(3)和\cref{prop:NonnegativeMeasurableIntegral}(10)可得:
	\begin{align*}
		&\operatorname{Var}_{\theta}[\delta(\mathbf{X})]\leqslant\operatorname{Var}_{\theta}[\delta(\mathbf{X})+af(\mathbf{X})]=\operatorname{Var}_{\theta}[\delta(\mathbf{X})]+2\operatorname{Cov}_{\theta}[\delta(\mathbf{X}),af(\mathbf{X})]+\operatorname{Var}_{\theta}[af(\mathbf{X})] \\
		=&\operatorname{Var}_{\theta}[\delta(\mathbf{X})]+2a\operatorname{Cov}_{\theta}[\delta(\mathbf{X}),f(\mathbf{X})]+a^2\operatorname{Var}_{\theta}[f(\mathbf{X})],\;\forall\;\theta\in\Theta
	\end{align*}
	即:
	\begin{equation*}
		2a\operatorname{Cov}_{\theta}[\delta(\mathbf{X}),f(\mathbf{X})]+a^2\operatorname{Var}_{\theta}[f(\mathbf{X})]\geqslant0,\;\forall\;a\in\mathbb{R}^{}
	\end{equation*}
	将上式看作关于$a$的一元二次方程,由判别式可得:
	\begin{equation*}
		4\operatorname{Cov}_{\theta}^2[\delta(\mathbf{X}),f(\mathbf{X})]\leqslant0
	\end{equation*}
	即$\operatorname{Cov}_{\theta}[\delta(\mathbf{X}),f(\mathbf{X})]=0$。由$f(\mathbf{X})$的任意性,必要性成立。
\end{proof}
\begin{corollary}\label{cor:UMVUE0UnbiasedEstimation}
		设$(X,\mathscr{A},\mathscr{P})$是参数结构,$\Theta$是参数空间,$\mathbf{X}$为从总体$F$中抽取的简单样本,$g(\theta)$为U可估的待估量,$T$是$\mathscr{P}$的充分统计量,$\delta(T)$是$g(\theta)$的一个无偏估计量,对任意的$\theta\in\Theta$有$\operatorname{Var}_{\theta}[\delta(T)]<+\infty$。令:
	\begin{equation*}
		A=\{f(T):\forall\;\theta\in\Theta,\;\operatorname{E}_{\theta}[f(T)]=0\}
	\end{equation*}
	$\delta(T)$是$g(\theta)$的UMVUE的充要条件为:
	\begin{equation*}
		\forall\;\theta\in\Theta,\;\forall\;f(T)\in A,\;\operatorname{Cov}_{\theta}[\delta(T),f(T)]=\operatorname{E}_{\theta}[\delta(T)\cdot f(T)]=0
	\end{equation*}
\end{corollary}
\subsubsection{信息不等式}
\begin{definition}
	设$(X,\mathscr{A},\mathscr{P})$是参数结构,$\Theta$是参数空间,$T$是$(X,\mathscr{A})$到$(Y,\mathscr{B})$上的统计量。若
\end{definition}
\begin{note}
	总结一下求UMVUE的方法:
	\begin{enumerate}
		\item 求分布族的充分统计量,求充分统计量的期望从而得到一个充分统计量的函数使得它是待估量的无偏估计;求所有零无偏估计,使得协方差为$0$;
		\item 求分布族的完全充分统计量,求待估量的一个无偏估计,求该无偏估计关于完全充分统计量的Rao-Blackwell改进。
	\end{enumerate}
\end{note}
\section{矩估计}

\subsection{矩法}
\begin{definition}
	设$(X,\mathscr{A},\mathscr{P})$是一个参数统计结构,$\mathbf{X_n}=(\seq{X}{n})$为从总体$F$中抽取的简单样本。将:
	\begin{equation*}
		\mu_{nk}=\frac{1}{n}\sum_{i=1}^{n}X_i^k,\quad
		\nu_{nk}=\frac{1}{n}\sum_{i=1}^{n}(X_i-\mu_{n1})^k
	\end{equation*}
	分别称为样本$k$阶原点矩和样本$k$阶中心矩。
\end{definition}
\begin{theorem}\label{theo:SampleMoment}
	$\nu_{nk}$与原点矩$\mu_{nk}$之间存在如下关系:
	\begin{equation*}
		\nu_{nk}=\sum_{i=0}^{k}\binom{k}{i}\mu_{ni}(-\mu_{n1})^{k-i}
	\end{equation*}
\end{theorem}
\begin{proof}
	由样本中心矩的定义可得:
	\begin{align*}
		\nu_{nk}
		&=\frac{1}{n}\sum_{i=1}^{n}(X_i-\mu_{n1})^k
		=\frac{1}{n}\sum_{i=1}^{n}\sum_{j=0}^{k}\binom{k}{j}X_i^j(-\mu_{n1})^{k-j} \\
		&=\sum_{j=0}^{k}\binom{k}{j}\frac{1}{n}\sum_{i=1}^{n}X_i^j(-\mu_{n1})^{k-j}
		=\sum_{j=0}^{k}\binom{k}{j}\mu_{nj}(-\mu_{n1})^{k-j}\qedhere
	\end{align*}
\end{proof}
\begin{definition}
	设$(X,\mathscr{A},\mathscr{P})$是一个参数统计结构,$\Theta$是参数空间,$\mathbf{X_n}=(\seq{X}{n})$为从总体中抽取的简单样本,$g(\theta)$是定义在$\Theta$上的函数,$\delta(\mathbf{X})$是$g(\theta)$的一个估计量。它可以表示为总体分布的一些矩的函数,即:
	\begin{equation*}
		g(\theta)=f(\seq{\mu}{s},\seq{\nu}{t})
	\end{equation*}
	设$\mathbf{X}=(\seq{X}{n})$是从上述总体分布族中抽取的简单样本,将$f$中的总体矩用样本矩代替,得到:
	\begin{equation*}
		\delta(\mathbf{X})=f(\mu_{n1},\mu_{n2},\dots,\mu_{ns},\nu_{n1},\nu_{n2},\dots,\nu_{nt})
	\end{equation*}
	则$\delta(\mathbf{X})$成为$g(\theta)$的一个点估计,称$\delta(\mathbf{X})$为$g(\theta)$的\gls{MomentEstimation},这种求矩估计量的方法称为\gls{MethodOfMoments}。
\end{definition}

\subsection{矩估计的性质}
\begin{property}
	矩估计具有如下性质:
	\begin{enumerate}
		\item 样本$k$阶原点矩$\mu_{nk}$是总体$k$阶原点矩$\mu_k$的无偏估计、强相合估计;
		\item $k=1$时,样本$k$阶中心矩$\nu_{nk}$是总体$k$阶中心矩$\nu_k$的无偏估计,对$k\geqslant2$,$\nu_{nk}$不是$\nu_k$的无偏估计。$\nu_{nk}$是$\nu_k$的强相合估计;
		\item 若待估量$g(\theta)$可以表示为一些总体原点矩的线性组合时,即:
		\begin{equation*}
			g(\theta)=\sum_{i=1}^{n}c_i\mu_{m_i},\;m_i\in\mathbb{N}^+
		\end{equation*}
		则其矩估计:
		\begin{equation*}
			\delta(\mathbf{X})=\sum_{i=1}^{n}c_i\mu_{nm_i},\;m_i\in\mathbb{N}^+
		\end{equation*}
		是$g(\theta)$的无偏估计;
		\item 若待估量$g(\theta)$满足:
		\begin{equation*}
			g(\theta)=f(\seq{\mu}{s},\seq{\nu}{t})
		\end{equation*}
		其中$f$是一个连续函数,则:
		\begin{equation*}
			\delta(\mathbf{X})=f(\mu_{n1},\mu_{n2},\dots,\mu_{ns},\nu_{n1},\nu_{n2},\dots,\nu_{nt})
		\end{equation*}
		是$g(\theta)$的强相合估计;
	\end{enumerate}
\end{property}
\begin{proof}
	(1)由:
	\begin{equation*}
		\operatorname{E}(\mu_{nk})=\operatorname{E}\left[\frac{1}{n}\sum_{i=1}^{n}X_i^k\right]=\frac{1}{n}\sum_{i=1}^{n}\operatorname{E}(X_i^k)=\frac{1}{n}\sum_{i=1}^{n}\mu_k=\mu_k
	\end{equation*}
	可得无偏性。\par
	由\cref{theo:StrongLawOfLargeNumbers}可得:
	\begin{equation*}
		\mu_{nk}=\frac{1}{n}\sum_{i=1}^{n}X_i^k\overset{\text{a.e.}}{\longrightarrow}\frac{1}{n}\sum_{i=1}^{n}E(X_i^k)=\mu_k
	\end{equation*}\par
	(2)\info{不无偏还未证明}由\cref{theo:Moment}可得:
	\begin{equation*}
		\nu_k=\sum_{i=0}^{k}\binom{k}{i}\mu_i(-\mu_1)^{k-i}=f(\seq{\mu}{k})
	\end{equation*}
	所以$\nu_k$是$\seq{\mu}{k}$的连续函数。由\cref{theo:ContinuousMappingTheorem}(1)、(1)、\cref{theo:SampleMoment}可得:
	\begin{equation*}
		\nu_{nk}=f(\mu_{n1},\mu_{n2},\dots,\mu_{nk})\overset{\text{a.e.}}{\longrightarrow}f(\seq{\mu}{k})=\nu_{k}
	\end{equation*}\par
	(3)由(1)直接可得。\par
	(4)由\cref{theo:ContinuousMappingTheorem}(1)、(1)(2)直接可得。
\end{proof}
\section{极大似然估计}

\begin{definition}
	设$(X,\mathscr{A},\mathscr{P})$是可控参数结构,$\mu$为控制测度,$\Theta$是参数空间,$\mathbf{X}$为从总体$F$中抽取的简单样本,$f(\mathbf{X};\theta)$为样本$\mathbf{X}$的概率函数。把$f(\mathbf{X};\theta)$看作$\theta$的函数,称该函数为\gls{LikelihoodFunction},记为$L(\theta;\mathbf{X})$,称$\ell(\theta;\mathbf{X})=\ln L(\theta;\mathbf{X})$为\textbf{对数似然函数}。
\end{definition}
\begin{definition}
	设$(X,\mathscr{A},\mathscr{P})$是可控参数结构,$\mu$为控制测度,$\Theta$是参数空间,$\mathbf{X}$为从总体$F$中抽取的简单样本,$L(\theta;\mathbf{X})$是似然函数。若存在统计量$\delta(\mathbf{X})$使得:
	\begin{equation*}
		L[\delta(\mathbf{X});\mathbf{X}]=\max_{\theta\in\Theta}L(\theta;\mathbf{X})
	\end{equation*}
	或等价地使得:
	\begin{equation*}
		\ell[\delta(\mathbf{X});\mathbf{X}]=\max_{\theta\in\Theta}\ell(\theta;\mathbf{X})
	\end{equation*}
	则称$\delta(\mathbf{X})$是$\theta$的\gls{MLE}。称:
	\begin{equation*}
		\frac{\dif L(\theta;\mathbf{X})}{\dif\theta_i}=0
	\end{equation*}
	为\gls{LikelihoodEquation}。
\end{definition}
\begin{property}\label{prop:MLE}
	设$(X,\mathscr{A},\mathscr{P})$是可控参数结构,$\mu$为控制测度,$\Theta$是参数空间,$\mathbf{X}$为从总体$F$中抽取的简单样本。最大似然估计具有如下性质:
	\begin{enumerate}
		\item 最大似然估计具有不变性,即:若$\theta$的MLE为$\theta^*$,则$\theta$的任一可测函数$g(\theta)$的MLE为$g(\theta^*)$;
		\item 若$\mathscr{P}$为指数族,只要似然方程组的解是自然参数空间的内点,则解必唯一且是MLE;
		\item 若$T(\mathbf{X})$是$\theta$的充分统计量且$\theta$的MLE$\;\delta(\mathbf{X})$唯一存在,则$\delta(\mathbf{X})$必为$T(\mathbf{X})$的函数;
	\end{enumerate}
\end{property}
\section{区间估计}

\begin{definition}
	设$(X,\mathscr{A},\mathscr{P})$是参数结构,$\Theta$是参数空间,$\mathbf{X}$为从总体$F$中抽取的简单样本,$g_1(\theta)\in\mathbb{R}^{n},g_2(\theta)\in\mathbb{R}^{}$是待估量,$\delta(\mathbf{X}),\delta_1(\mathbf{X}),\delta_2(\mathbf{X})$是统计量且满足对任意的样本有$\delta_1(\mathbf{X})\leqslant\delta_2(\mathbf{X})$,$\alpha\in(0,1)$为给定值。若统计量$\delta(\mathbf{X})$满足:
	\begin{equation*}
		\forall\;\theta\in\Theta,\;P_{\theta}(\{g_1(\theta)\in \delta(\mathbf{X})\})\geqslant 1-\alpha
	\end{equation*}
	则称$\delta(\mathbf{X})$为$g_1(\theta)$\gls{ConfidenceLevel}为$1-\alpha$的\gls{ConfidenceRegion},若$n=1$则称$\delta(\mathbf{X})$为$g_1(\theta)$置信水平为$1-\alpha$的\gls{CI},上式取等号时称$\delta(\mathbf{X})$为$g_1(\theta)$置信水平为$1-\alpha$的\textbf{同等}置信域,称:
	\begin{equation*}
		\inf_{\theta\in\Theta}P_{\theta}(\{g_1(\theta)\in\delta(\mathbf{X})\})
	\end{equation*}
	为其\gls{ConfidenceCoefficient}。若$\delta_1(\mathbf{X})$满足:
	\begin{equation*}
		\forall\;\theta\in\Theta,\;P_{\theta}(\{\delta_1(\mathbf{X})\leqslant g_2(\theta)\})\geqslant 1-\alpha
	\end{equation*}	
	则称$\delta_1(\mathbf{X})$是$g_2(\theta)$置信水平为$1-\alpha$的\gls{LowerConfidenceLimit},上式取等号时称$\delta_1(\mathbf{X})$为$g_2(\theta)$置信水平为$1-\alpha$的\textbf{同等}置信下限,称:
	\begin{equation*}
		\inf_{\theta\in\Theta}P_{\theta}(\{\delta_1(\mathbf{X})\leqslant g_2(\theta)\})
	\end{equation*}
	为其置信系数。若$\delta_2(\mathbf{X})$满足:
	\begin{equation*}
		\forall\;\theta\in\Theta,\;P_{\theta}(\{g_2(\theta)\leqslant\delta_2(\mathbf{X})\})\geqslant 1-\alpha
	\end{equation*}	
	则称$\delta_2(\mathbf{X})$是$g_2(\theta)$置信水平为$1-\alpha$的\gls{UpperConfidenceLimit},上式取等号时称$\delta_2(\mathbf{X})$为$g_2(\theta)$置信水平为$1-\alpha$的\textbf{同等}置信上限,称:
	\begin{equation*}
		\inf_{\theta\in\Theta}P_{\theta}(\{g_2(\theta)\leqslant\delta_2(\mathbf{X})\})
	\end{equation*}
	为其置信系数。
\end{definition}
\begin{note}
	什么样的置信域是好的?我们当然希望置信域能够包含待估量的真实值(即提高置信水平或置信系数),并且包含的非真实值尽可能的少,但这两点往往是冲突的:置信域越大,更有可能包含真实值,但精度就会降低;置信域小则包含真实值的概率也小。Nyeman提出了一个区间估计的标准:在保持置信水平尽可能大的前提下,使得置信域尽可能的小,即可靠度优先。
\end{note}
\begin{note}[枢轴量法]
	枢轴量法是一个常见的构造区间估计的方法,下面给出其实现的具体步骤:
	\begin{enumerate}
		\item 找到一个待估量$g(\theta)$的良好的点估计$\delta(\mathbf{X})$;
		\item 求出随机变量$f(g,\delta)$的分布,要求$f$的分布与$g(\theta)$的取值无关,称$f(g,\delta)$为\gls{PivotalQuantity};
		\item 根据$f(g,\delta)$的分位点给出给定置信水平的区间估计或置信上(下)限。
	\end{enumerate}
	考虑置信区间的情况,若$f(g,\delta)$的分布是单峰对称分布(如一元正态分布),很容易找出给定置信水平的最小置信区间($g$在$f$的分子位置时即取$f(g,\delta)$分布的$\frac{\alpha}{2}$与$1-\frac{\alpha}{2}$分位点),但很多情况下我们需要数值方法来求解最小置信区间,为了避免这些麻烦,我们在多数情况下会构造\textbf{等尾}置信区间,即依旧取$f(g,\delta)$分布的$\frac{\alpha}{2}$与$1-\frac{\alpha}{2}$分位点作为区间估计的两端。
\end{note}
\begin{theorem}
	设$(X,\mathscr{A},\mathscr{P})$是参数结构,$\mathbf{X}=(\seq{X}{m})$为从总体$F$中抽取的简单样本,$\mathbf{Y}=(\seq{Y}{n})$为从总体$G$中抽取的简单样本,$\overline{X},\overline{Y}$为样本均值,$S_X^2,S_Y^2$为样本方差,$g(\theta)$为待估量,其区间估计有如下结论:
	\begin{enumerate}
		\item 若置信水平为$1-\alpha_1$的$g(\theta)$的置信上限$\delta_1(\mathbf{X})$和置信水平为$1-\alpha_2$的$g(\theta)$的置信上限$\delta_2(\mathbf{X})$满足对任意的样本有$\delta_1(\mathbf{X})\leqslant\delta_2(\mathbf{X})$,则$[\delta_1(\mathbf{X}),\delta_2(\mathbf{X})]$是$g(\theta)$置信水平为$1-\alpha_1-\alpha_2$的置信区间;
		\item 若总体$F$服从$\operatorname{N}(\mu,\sigma^2)$,在$\sigma^2$已知时$g(\theta)=\mu$置信水平为$1-\alpha$的区间估计为:
		\begin{equation*}
			\left[\overline{X}-u_{1-\frac{\alpha}{2}}\sqrt{\dfrac{\sigma^2}{m}},\overline{X}+u_{1-\frac{\alpha}{2}}\sqrt{\dfrac{\sigma^2}{m}}\right]
		\end{equation*}
		$\sigma^2$未知时$g(\theta)=\mu$置信水平为$1-\alpha$的区间估计为:
		\begin{equation*}
			\left[\overline{X}-\operatorname{t}_{m-1}\left(1-\frac{\alpha}{2}\right)\sqrt{\dfrac{S_X^2}{m}},\overline{X}+\operatorname{t}_{m-1}\left(1-\frac{\alpha}{2}\right)\sqrt{\dfrac{S_X^2}{m}}\right]
		\end{equation*}
		\item 若总体$F$服从$\operatorname{N}(\mu,\sigma^2)$,在$\mu$已知时$g(\theta)=\sigma^2$置信水平为$1-\alpha$的区间估计为:
		\begin{equation*}
			\left[\frac{1}{\chi_{m}^2\left(1-\frac{\alpha}{2}\right)}\sum_{i=1}^{m}(X_i-\mu)^2,\frac{1}{\chi_{m}^2\left(\frac{\alpha}{2}\right)}\sum_{i=1}^{m}(X_i-\mu)^2\right]
		\end{equation*}
		$\mu$未知时$g(\theta)=\sigma^2$置信水平为$1-\alpha$的区间估计为:
		\begin{equation*}
			\left[\frac{(m-1)S_X^2}{\chi_{m-1}^2\left(1-\frac{\alpha}{2}\right)},\frac{(m-1)S_X^2}{\chi_{m-1}^2\left(\frac{\alpha}{2}\right)}\right]
		\end{equation*}
		\item 若总体$F$服从$\operatorname{N}(\mu,\sigma_1^2)$,$G$服从$\operatorname{N}(\nu,\sigma_2^2)$,$\sigma_1^2$和$\sigma_2^2$已知时$g(\theta)=\mu-\nu$置信水平为$1-\alpha$的区间估计为:
		\begin{equation*}
			\left[\overline{X}-\overline{Y}-u_{1-\frac{\alpha}{2}}\sqrt{\dfrac{\sigma_1^2}{m}+\dfrac{\sigma_2^2}{n}},\overline{X}-\overline{Y}+u_{1-\frac{\alpha}{2}}\sqrt{\dfrac{\sigma_1^2}{m}+\dfrac{\sigma_2^2}{n}}\right]
		\end{equation*}
		$\sigma_1^2=c\sigma_2^2(c>0)$时$g(\theta)=\mu-\nu$置信水平为$1-\alpha$的区间估计为:
		\begin{equation*}
			\left[\overline{X}-\overline{Y}-\operatorname{t}_{m+n-2}\left(1-\frac{\alpha}{2}\right)S_w\sqrt{\dfrac{mc+n}{mn}},\overline{X}-\overline{Y}+\operatorname{t}_{m+n-2}\left(1-\frac{\alpha}{2}\right)S_w\sqrt{\dfrac{mc+n}{mn}}\right]
		\end{equation*}
		其中:
		\begin{equation*}
			S_w^2=\frac{(m-1)S_X^2+(n-1)S_Y^2/c}{m+n-2}
		\end{equation*}
		\item 若总体$F$服从$\operatorname{N}(\mu,\sigma_1^2)$,$G$服从$\operatorname{N}(\nu,\sigma_2^2)$,在$\mu,\nu$已知时$g(\theta)=\dfrac{\sigma_1^2}{\sigma_2^2}$置信水平为$1-\alpha$的区间估计为:
		\begin{equation*}
			\left[\frac{n\sum\limits_{i=1}^{m}(X_i-\mu)^2}{m\sum\limits_{i=1}^{n}(Y_i-\nu)^2\operatorname{F}_{m,n}\left(1-\frac{\alpha}{2}\right)},\frac{n\sum\limits_{i=1}^{m}(X_i-\mu)^2}{m\sum\limits_{i=1}^{n}(Y_i-\nu)^2\operatorname{F}_{m,n}\left(\frac{\alpha}{2}\right)}\right]
		\end{equation*}
		$\mu,\nu$未知时$g(\theta)=\dfrac{\sigma_1^2}{\sigma_2^2}$置信水平为$1-\alpha$的区间估计为:
		\begin{equation*}
			\left[\frac{S_X^2}{S_Y^2\operatorname{F}_{m-1,n-1}\left(1-\frac{\alpha}{2}\right)},\frac{S_X^2}{S_Y^2\operatorname{F}_{m-1,n-1}\left(\frac{\alpha}{2}\right)}\right]
		\end{equation*}
	\end{enumerate}
\end{theorem}
\begin{proof}
	(1)由\cref{prop:Measure}(3)(次有限可加性)可得:
	\begin{align*}
		P(\{g(\theta)\leqslant\delta_1(\mathbf{X})\}\cup\{g(\theta)\geqslant\delta_2(\mathbf{X})\})&\leqslant P(\{g(\theta)\leqslant\delta_1(\mathbf{X})\})+P(\{g(\theta)\geqslant\delta_2(\mathbf{X})\}) \\
		&\leqslant\alpha_1+\alpha_2,\;\forall\;P\in\mathscr{P}
	\end{align*}
	根据\cref{prop:Measure}(2)可得:
	\begin{equation*}
		\forall\;P\in\mathscr{P},\;P\left(\{\delta_1(\mathbf{X})\leqslant g(\theta)\leqslant\delta_2(\mathbf{X})\}\right)\geqslant1-\alpha_1-\alpha_2
	\end{equation*}\par
	(2)\textbf{已知$\sigma^2$:}由\cref{theo:SamplingDist1}(1)和\cref{prop:MultiNormal}(2)可知:
	\begin{equation*}
		f(\mu,\bar{X})=\frac{\sqrt{m}(\overline{X}-\mu)}{\sigma}\sim\operatorname{N}(0,1)
	\end{equation*}
	根据(1)可知可取:
	\begin{equation*}
		u_{\frac{\alpha}{2}}\leqslant f(\mu,\bar{X})\leqslant u_{1-\frac{\alpha}{2}}
	\end{equation*}
	即:
	\begin{equation*}
		\left[\overline{X}-u_{1-\frac{\alpha}{2}}\sqrt{\dfrac{\sigma^2}{m}},\overline{X}-u_{1-\frac{\alpha}{2}}\sqrt{\dfrac{\sigma^2}{m}}\right]
	\end{equation*}\par
	\textbf{未知$\sigma^2$:}由\cref{theo:SamplingDist1}(4)可知:
	\begin{equation*}
		f(\mu,\bar{X})=\frac{\sqrt{m}(\overline{X}-\mu)}{S_X}\sim\operatorname{t}_{m-1}
	\end{equation*}
	根据(1)可知可取:
	\begin{equation*}
		\operatorname{t}_{m-1}\left(\frac{\alpha}{2}\right)\leqslant f(\mu,\bar{X})\leqslant\operatorname{t}_{m-1}\left(1-\frac{\alpha}{2}\right)
	\end{equation*}
	即:
	\begin{equation*}
		\left[\overline{X}-\operatorname{t}_{m-1}\left(1-\frac{\alpha}{2}\right)\sqrt{\dfrac{S_X^2}{m}},\overline{X}+\operatorname{t}_{m-1}\left(1-\frac{\alpha}{2}\right)\sqrt{\dfrac{S_X^2}{m}}\right]
	\end{equation*}\par
	(3)\textbf{已知$\mu$:}由\cref{prop:MultiNormal}(2)和$\chi^2$分布的定义可知:
	\begin{equation*}
		f\left(\sigma^2,\frac{1}{m}\sum_{i=1}^{m}(X_i-\mu)^2\right)=\frac{1}{\sigma^2}\sum_{i=1}^{m}(X_i-\mu)^2\sim\chi_{m}^2
	\end{equation*}
	根据(1)可知可取:
	\begin{equation*}
		\chi_{m}^2\left(\frac{\alpha}{2}\right)\leqslant f\left(\sigma^2,\frac{1}{m}\sum_{i=1}^{m}(X_i-\mu)^2\right)\leqslant\chi_{m}^2\left(1-\frac{\alpha}{2}\right)
	\end{equation*}
	即:
	\begin{equation*}
		\left[\frac{1}{\chi_{m-1}^2\left(1-\frac{\alpha}{2}\right)}\sum_{i=1}^{m}(X_i-\mu)^2,\frac{1}{\chi_{m-1}^2\left(\frac{\alpha}{2}\right)}\sum_{i=1}^{m}(X_i-\mu)^2\right]
	\end{equation*}\par
	\textbf{未知$\mu$:}由\cref{theo:SamplingDist1}(2)可知:
	\begin{equation*}
		f\left(\sigma^2,\frac{S_X^2}{m-1}\right)=\frac{(m-1)S_X^2}{\sigma^2}\sim\chi_{m-1}^2
	\end{equation*}
	根据(1)可知可取:
	\begin{equation*}
		\chi_{m-1}^2\left(\frac{\alpha}{2}\right)\leqslant f\left(\sigma^2,\frac{S_X^2}{m-1}\right)\leqslant\chi_{m-1}^2\left(1-\frac{\alpha}{2}\right)
	\end{equation*}
	即:
	\begin{equation*}
		\left[\frac{(m-1)S_X^2}{\chi_{m-1}^2\left(1-\frac{\alpha}{2}\right)},\frac{(m-1)S_X^2}{\chi_{m-1}^2\left(\frac{\alpha}{2}\right)}\right]
	\end{equation*}\par
	(4)\textbf{已知$\sigma_1^2,\sigma_2^2$:}由(1)(得到$\overline{X}$和$\overline{Y}$的分布)、$\seq{X}{m}$与$\seq{Y}{n}$相互独立(由\cref{prop:MultiNormal}(6)得到二维随机向量$(\overline{X},\overline{Y})^T$的分布)和\cref{prop:MultiNormal}(2)(对$(\overline{X},\overline{Y})^T$用二维行向量$(1,-1)$做线性变换,再做标准化\info{标准化})可得:
	\begin{equation*}
		f(\mu-\nu,\overline{X}-\overline{Y})=\frac{\overline{X}-\overline{Y}-(\mu-\nu)}{\sqrt{\dfrac{\sigma_1^2}{m}+\dfrac{\sigma_2^2}{n}}}\sim\operatorname{N}\left(0,1\right)
	\end{equation*}
	根据(1)可知可取:
	\begin{equation*}
		u_{\frac{\alpha}{2}}\leqslant f(\mu-\nu,\overline{X}-\overline{Y})\leqslant u_{1-\frac{\alpha}{2}}
	\end{equation*}
	即:
	\begin{equation*}
		\left[\overline{X}-\overline{Y}-u_{1-\frac{\alpha}{2}}\sqrt{\dfrac{\sigma_1^2}{m}+\dfrac{\sigma_2^2}{n}},\overline{X}-\overline{Y}+u_{1-\frac{\alpha}{2}}\sqrt{\dfrac{\sigma_1^2}{m}+\dfrac{\sigma_2^2}{n}}\right]
	\end{equation*}\par
	\textbf{$\sigma_1^2=\sigma_2^2$:}类似\cref{theo:SamplingDist1}(5)可得:
	\begin{equation*}
		f(\mu-\nu,\overline{X}-\overline{Y})=\dfrac{\overline{X}-\overline{Y}-(\mu-\nu)}{\sqrt{(m-1)S_X^2+(n-1)S_Y^2/c}}\sqrt{\dfrac{mn(m+n-2)}{mc+n}}\sim \operatorname{t}_{m+n-2}
	\end{equation*}
	根据(1)可知可取:
	\begin{equation*}
		\operatorname{t}_{m+n-2}\left(\frac{\alpha}{2}\right)\leqslant f(\mu-\nu,\overline{X}-\overline{Y})\leqslant \operatorname{t}_{m+n-2}\left(1-\frac{\alpha}{2}\right)
	\end{equation*}
	记:
	\begin{equation*}
		S_w^2=\frac{(m-1)S_X^2+(n-1)S_Y^2/c}{m+n-2}
	\end{equation*}
	于是可得:
	\begin{equation*}
		\left[\overline{X}-\overline{Y}-\operatorname{t}_{m+n-2}\left(1-\frac{\alpha}{2}\right)S_w\sqrt{\dfrac{mc+n}{mn}},\overline{X}-\overline{Y}+\operatorname{t}_{m+n-2}\left(1-\frac{\alpha}{2}\right)S_w\sqrt{\dfrac{mc+n}{mn}}\right]
	\end{equation*}\par
	(5)\textbf{已知$\mu,\nu$:}由\cref{prop:MultiNormal}(2)和$\chi^2$分布的定义可知:
	\begin{equation*}
		\delta_1(\mathbf{X})=\frac{1}{\sigma_1^2}\sum_{i=1}^{m}(X_i-\mu)^2\sim\chi_{m}^2,\quad\delta_2(\mathbf{Y})=\frac{1}{\sigma_2^2}\sum_{i=1}^{n}(Y_i-\nu)^2\sim\chi_{n}^2
	\end{equation*}
	因为$\seq{X}{m}$和$\seq{Y}{n}$相互独立,所以$\delta_1(\mathbf{X})$与$\delta_2(\mathbf{Y})$也相互独立,于是有:
	\begin{equation*}
		f\left(\frac{\sigma_1^2}{\sigma_2^2},\frac{S_X^2(n-1)}{S_Y^2(m-1)}\right)=\frac{n\sum\limits_{i=1}^{m}(X_i-\mu)^2\sigma_2^2}{m\sum\limits_{i=1}^{n}(Y_i-\nu)^2\sigma_1^2}\sim\operatorname{F}_{m,n}
	\end{equation*}
	根据(1)可知可取:
	\begin{equation*}
		\operatorname{F}_{m,n}\left(\frac{\alpha}{2}\right)\leqslant f\left(\frac{\sigma_1^2}{\sigma_2^2},\frac{S_X^2(n-1)}{S_Y^2(m-1)}\right)\leqslant\operatorname{F}_{m,n}\left(1-\frac{\alpha}{2}\right)
	\end{equation*}
	即:
	\begin{equation*}
		\left[\frac{n\sum\limits_{i=1}^{m}(X_i-\mu)^2}{m\sum\limits_{i=1}^{n}(Y_i-\nu)^2\operatorname{F}_{m,n}\left(1-\frac{\alpha}{2}\right)},\frac{n\sum\limits_{i=1}^{m}(X_i-\mu)^2}{m\sum\limits_{i=1}^{n}(Y_i-\nu)^2\operatorname{F}_{m,n}\left(\frac{\alpha}{2}\right)}\right]
	\end{equation*}\par
	\textbf{未知$\mu,\nu$:}由\cref{theo:SamplingDist1}(6)可得:
	\begin{equation*}
		f\left(\frac{\sigma_1^2}{\sigma_2^2},\frac{S_X^2(n-1)}{S_Y^2(m-1)}\right)=\frac{S_X^2\sigma_2^2}{S_Y^2\sigma_1^2}\sim\operatorname{F}_{m-1,n-1}
	\end{equation*}
	根据(1)可知可取:
	\begin{equation*}
		\operatorname{F}_{m-1.n-1}\left(\frac{\alpha}{2}\right)\leqslant f\left(\frac{\sigma_1^2}{\sigma_2^2},\frac{S_X^2(n-1)}{S_Y^2(m-1)}\right)\leqslant\operatorname{F}_{m-1,n-1}\left(1-\frac{\alpha}{2}\right)
	\end{equation*}
	即:
	\begin{equation*}
		\left[\frac{S_X^2}{S_Y^2\operatorname{F}_{m-1,n-1}\left(1-\frac{\alpha}{2}\right)},\frac{S_X^2}{S_Y^2\operatorname{F}_{m-1,n-1}\left(\frac{\alpha}{2}\right)}\right]\qedhere
	\end{equation*}
\end{proof}
\subsubsection{样本量问题}
\begin{definition}
	称置信区间的半径为\gls{MOE}。
\end{definition}
\begin{note}
	可以通过控制误差幅度来计算样本量,比如我们要控制误差幅度在$a$以内,列出不等式可以去计算满足要求的样本量范围,一般来讲误差幅度越小所需的样本量就越大,由此可得到最小样本量。当方程难以求解时,可以采用Monte Carlo算法计算出大量样本量与其对应的MOE值,从而选择出合适的样本量。
\end{note}

\begin{theorem}
	对于样本方差和样本均值:
	\begin{enumerate}
		\item 记:
		\begin{equation*}
			\overline{x}_n=\frac{1}{n}\sum_{i=1}^{n}x_i,\quad s_n^2=\frac{1}{n-1}\sum_{i=1}^{n}(x_i-\overline{x}_n)^2
		\end{equation*}
		则有:
		\begin{equation*}
			\overline{x}_{n+1}=\overline{x}_n+\frac{1}{n+1}(x_{n+1}-\overline{x}_n),\quad s_{n+1}^2=\frac{n-1}{n}s_n^2+\frac{1}{n}(x_{n+1}-\overline{x}_n)^2
		\end{equation*}
		\item 从同一总体中抽取两个大小分别为$m$和$n$的样本,样本均值分别记为$\overline{x}_1,\overline{x}_2$,样本方差分别为$s_1^2,s_2^2$,将两组样本合并,其均值和样本方差分别为$\overline{x},s^2$,则有:
		\begin{equation*}
			\overline{x}=\frac{m\overline{x}_1+n\overline{x}_2}{m+n},\quad s^2=\frac{(m-1)s_1^2+(n-1)s_2^2}{m+n-1}+\frac{mn(\overline{x}_1-\overline{x}_2)^2}{(m+n)(m+n+1)}
		\end{equation*}
	\end{enumerate}
\end{theorem}
\begin{proof}
	(1)对于均值有:
	\begin{align*}
		\overline{x}_{n+1}&=\frac{1}{n+1}\sum_{i=1}^{n+1}x_i=\frac{n}{n+1}\frac{1}{n}\left(\sum_{i=1}^{n}x_i+x_{n+1}\right) \\
		&=\frac{n}{n+1}\overline{x}_n+\frac{1}{n+1}x_{n+1}=\overline{x}_n+\frac{1}{n+1}(x_{n+1}-\overline{x}_n)
	\end{align*}
	对于方差有:
	\begin{align*}
		s_{n+1}^2&=\frac{1}{n}\sum_{i=1}^{n+1}(x_i-\overline{x}_{n+1})^2=\frac{1}{n}\sum_{i=1}^{n+1}\left[x_i-\overline{x}_n-\frac{1}{n+1}(x_{n+1}-\overline{x}_n)\right]^2 \\
		&=\frac{1}{n}\sum_{i=1}^{n+1}\left[(x_i-\overline{x}_n)^2+\frac{1}{(n+1)^2}(x_{n+1}-\overline{x}_n)^2-\frac{2}{n+1}(x_i-\overline{x}_n)(x_{n+1}-\overline{x}_n)\right] \\
		&=\frac{1}{n}\sum_{i=1}^{n}(x_i-\overline{x}_n)^2+\frac{1}{n}(x_{n+1}-\overline{x}_n)^2+\frac{1}{n(n+1)}(x_{n+1}-\overline{x}_n)^2-\frac{2}{n(n+1)}(x_{n+1}-\overline{x}_n)^2 \\
		&=\frac{n-1}{n}s_n^2+\frac{1}{n}(x_{n+1}-\overline{x}_n)^2-\frac{1}{n(n+1)}(x_{n+1}-\overline{x}_n)^2 \\
		&=\frac{n-1}{n}s_n^2+\frac{1}{n}(x_{n+1}-\overline{x}_n)^2
	\end{align*}\par
	(2)均值的结论是显然的,对于方差有:
	\begin{align*}
		s^2&=\frac{1}{m+n-1}\sum_{i=1}^{m+n}(x_i-\overline{x})^2=\frac{1}{m+n-1}\left[\sum_{i=1}^{m}(x_i-\overline{x})^2+\sum_{i=m+1}^{m+n}(x_i-\overline{x})^2\right] \\
		&=\frac{1}{m+n-1}\left[\sum_{i=1}^{m}(x_i-\overline{x}_1+\overline{x}_1-\overline{x})^2+\sum_{i=m+1}^{m+n}(x_i-\overline{x}_2+\overline{x}_2-\overline{x})^2\right] \\
		&=\frac{1}{m+n-1}\left[(m-1)s_1^2+m(\overline{x}_1-\overline{x})^2+2\sum_{i=1}^{m}(x_i-\overline{x}_1)(\overline{x_1}-\overline{x})\right. \\
		&\quad\left.+(n-1)s_2^2+n(\overline{x}_2-\overline{x})^2+2\sum_{i=m+1}^{m+n}(x_i-\overline{x}_2)(\overline{x_2}-\overline{x})\right] \\
		&=\frac{1}{m+n-1}\left[(m-1)s_1^2+m(\overline{x}_1-\overline{x})^2+(n-1)s_2^2+n(\overline{x}_2-\overline{x})^2\right] \\
		&=\frac{(m-1)s_1^2+(n-1)s_2^2}{m+n-1}+\frac{m(\overline{x}_1-\overline{x})^2+n(\overline{x}_2-\overline{x})^2}{m+n-1}
	\end{align*}
	由于:
	\begin{gather*}
		m(\overline{x}_1-\overline{x})^2=m\left(\overline{x}_1-\frac{m\overline{x}_1+n\overline{x}_2}{m+n}\right)^2=m\frac{n^2(\overline{x}_1-\overline{x}_2)^2}{(m+n)^2} \\
		n(\overline{x}_2-\overline{x})^2=n\left(\overline{x}_2-\frac{m\overline{x}_1+n\overline{x}_2}{m+n}\right)^2=n\frac{m^2(\overline{x}_1-\overline{x}_2)^2}{(m+n)^2}
	\end{gather*}
	所以:
	\begin{align*}
		s^2&=\frac{(m-1)s_1^2+(n-1)s_2^2}{m+n-1}+\frac{m(\overline{x}_1-\overline{x})^2+n(\overline{x}_2-\overline{x})^2}{m+n-1} \\
		&=\frac{(m-1)s_1^2+(n-1)s_2^2}{m+n-1}+\frac{(m+n)mn(\overline{x}_1-\overline{x}_2)^2}{(m+n-1)(m+n)^2} \\
		&=\frac{(m-1)s_1^2+(n-1)s_2^2}{m+n-1}+\frac{mn(\overline{x}_1-\overline{x}_2)^2}{(m+n-1)(m+n)}\qedhere
	\end{align*}
\end{proof}

