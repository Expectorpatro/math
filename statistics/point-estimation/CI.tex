\section{区间估计}

\begin{definition}
	设$(X,\mathscr{A},\mathscr{P})$是参数结构,$\Theta$是参数空间,$\mathbf{X}$为从总体$F$中抽取的简单样本,$g_1(\theta)\in\mathbb{R}^{n},g_2(\theta)\in\mathbb{R}^{}$是待估量,$\delta(\mathbf{X}),\delta_1(\mathbf{X}),\delta_2(\mathbf{X})$是统计量且满足对任意的样本有$\delta_1(\mathbf{X})\leqslant\delta_2(\mathbf{X})$,$\alpha\in(0,1)$为给定值。若统计量$\delta(\mathbf{X})$满足:
	\begin{equation*}
		\forall\;\theta\in\Theta,\;P_{\theta}(\{g_1(\theta)\in \delta(\mathbf{X})\})\geqslant 1-\alpha
	\end{equation*}
	则称$\delta(\mathbf{X})$为$g_1(\theta)$\gls{ConfidenceLevel}为$1-\alpha$的\gls{ConfidenceRegion},若$n=1$则称$\delta(\mathbf{X})$为$g_1(\theta)$置信水平为$1-\alpha$的\gls{CI},上式取等号时称$\delta(\mathbf{X})$为$g_1(\theta)$置信水平为$1-\alpha$的\textbf{同等}置信域,称:
	\begin{equation*}
		\inf_{\theta\in\Theta}P_{\theta}(\{g_1(\theta)\in\delta(\mathbf{X})\})
	\end{equation*}
	为其\gls{ConfidenceCoefficient}。若$\delta_1(\mathbf{X})$满足:
	\begin{equation*}
		\forall\;\theta\in\Theta,\;P_{\theta}(\{\delta_1(\mathbf{X})\leqslant g_2(\theta)\})\geqslant 1-\alpha
	\end{equation*}	
	则称$\delta_1(\mathbf{X})$是$g_2(\theta)$置信水平为$1-\alpha$的\gls{LowerConfidenceLimit},上式取等号时称$\delta_1(\mathbf{X})$为$g_2(\theta)$置信水平为$1-\alpha$的\textbf{同等}置信下限,称:
	\begin{equation*}
		\inf_{\theta\in\Theta}P_{\theta}(\{\delta_1(\mathbf{X})\leqslant g_2(\theta)\})
	\end{equation*}
	为其置信系数。若$\delta_2(\mathbf{X})$满足:
	\begin{equation*}
		\forall\;\theta\in\Theta,\;P_{\theta}(\{g_2(\theta)\leqslant\delta_2(\mathbf{X})\})\geqslant 1-\alpha
	\end{equation*}	
	则称$\delta_2(\mathbf{X})$是$g_2(\theta)$置信水平为$1-\alpha$的\gls{UpperConfidenceLimit},上式取等号时称$\delta_2(\mathbf{X})$为$g_2(\theta)$置信水平为$1-\alpha$的\textbf{同等}置信上限,称:
	\begin{equation*}
		\inf_{\theta\in\Theta}P_{\theta}(\{g_2(\theta)\leqslant\delta_2(\mathbf{X})\})
	\end{equation*}
	为其置信系数。
\end{definition}
\begin{note}
	什么样的置信域是好的?我们当然希望置信域能够包含待估量的真实值(即提高置信水平或置信系数),并且包含的非真实值尽可能的少,但这两点往往是冲突的:置信域越大,更有可能包含真实值,但精度就会降低;置信域小则包含真实值的概率也小。Nyeman提出了一个区间估计的标准:在保持置信水平尽可能大的前提下,使得置信域尽可能的小,即可靠度优先。
\end{note}
\begin{note}[枢轴量法]
	枢轴量法是一个常见的构造区间估计的方法,下面给出其实现的具体步骤:
	\begin{enumerate}
		\item 找到一个待估量$g(\theta)$的良好的点估计$\delta(\mathbf{X})$;
		\item 求出随机变量$f(g,\delta)$的分布,要求$f$的分布与$g(\theta)$的取值无关,称$f(g,\delta)$为\gls{PivotalQuantity};
		\item 根据$f(g,\delta)$的分位点给出给定置信水平的区间估计或置信上(下)限。
	\end{enumerate}
	考虑置信区间的情况,若$f(g,\delta)$的分布是单峰对称分布(如一元正态分布),很容易找出给定置信水平的最小置信区间($g$在$f$的分子位置时即取$f(g,\delta)$分布的$\frac{\alpha}{2}$与$1-\frac{\alpha}{2}$分位点),但很多情况下我们需要数值方法来求解最小置信区间,为了避免这些麻烦,我们在多数情况下会构造\textbf{等尾}置信区间,即依旧取$f(g,\delta)$分布的$\frac{\alpha}{2}$与$1-\frac{\alpha}{2}$分位点作为区间估计的两端。
\end{note}
\begin{theorem}
	设$(X,\mathscr{A},\mathscr{P})$是参数结构,$\mathbf{X}=(\seq{X}{m})$为从总体$F$中抽取的简单样本,$\mathbf{Y}=(\seq{Y}{n})$为从总体$G$中抽取的简单样本,$\overline{X},\overline{Y}$为样本均值,$S_X^2,S_Y^2$为样本方差,$g(\theta)$为待估量,其区间估计有如下结论:
	\begin{enumerate}
		\item 若置信水平为$1-\alpha_1$的$g(\theta)$的置信上限$\delta_1(\mathbf{X})$和置信水平为$1-\alpha_2$的$g(\theta)$的置信上限$\delta_2(\mathbf{X})$满足对任意的样本有$\delta_1(\mathbf{X})\leqslant\delta_2(\mathbf{X})$,则$[\delta_1(\mathbf{X}),\delta_2(\mathbf{X})]$是$g(\theta)$置信水平为$1-\alpha_1-\alpha_2$的置信区间;
		\item 若总体$F$服从$\operatorname{N}(\mu,\sigma^2)$,在$\sigma^2$已知时$g(\theta)=\mu$置信水平为$1-\alpha$的区间估计为:
		\begin{equation*}
			\left[\overline{X}-u_{1-\frac{\alpha}{2}}\sqrt{\dfrac{\sigma^2}{m}},\overline{X}+u_{1-\frac{\alpha}{2}}\sqrt{\dfrac{\sigma^2}{m}}\right]
		\end{equation*}
		$\sigma^2$未知时$g(\theta)=\mu$置信水平为$1-\alpha$的区间估计为:
		\begin{equation*}
			\left[\overline{X}-\operatorname{t}_{m-1}\left(1-\frac{\alpha}{2}\right)\sqrt{\dfrac{S_X^2}{m}},\overline{X}+\operatorname{t}_{m-1}\left(1-\frac{\alpha}{2}\right)\sqrt{\dfrac{S_X^2}{m}}\right]
		\end{equation*}
		\item 若总体$F$服从$\operatorname{N}(\mu,\sigma^2)$,在$\mu$已知时$g(\theta)=\sigma^2$置信水平为$1-\alpha$的区间估计为:
		\begin{equation*}
			\left[\frac{1}{\chi_{m}^2\left(1-\frac{\alpha}{2}\right)}\sum_{i=1}^{m}(X_i-\mu)^2,\frac{1}{\chi_{m}^2\left(\frac{\alpha}{2}\right)}\sum_{i=1}^{m}(X_i-\mu)^2\right]
		\end{equation*}
		$\mu$未知时$g(\theta)=\sigma^2$置信水平为$1-\alpha$的区间估计为:
		\begin{equation*}
			\left[\frac{(m-1)S_X^2}{\chi_{m-1}^2\left(1-\frac{\alpha}{2}\right)},\frac{(m-1)S_X^2}{\chi_{m-1}^2\left(\frac{\alpha}{2}\right)}\right]
		\end{equation*}
		\item 若总体$F$服从$\operatorname{N}(\mu,\sigma_1^2)$,$G$服从$\operatorname{N}(\nu,\sigma_2^2)$,$\sigma_1^2$和$\sigma_2^2$已知时$g(\theta)=\mu-\nu$置信水平为$1-\alpha$的区间估计为:
		\begin{equation*}
			\left[\overline{X}-\overline{Y}-u_{1-\frac{\alpha}{2}}\sqrt{\dfrac{\sigma_1^2}{m}+\dfrac{\sigma_2^2}{n}},\overline{X}-\overline{Y}+u_{1-\frac{\alpha}{2}}\sqrt{\dfrac{\sigma_1^2}{m}+\dfrac{\sigma_2^2}{n}}\right]
		\end{equation*}
		$\sigma_1^2=c\sigma_2^2(c>0)$时$g(\theta)=\mu-\nu$置信水平为$1-\alpha$的区间估计为:
		\begin{equation*}
			\left[\overline{X}-\overline{Y}-\operatorname{t}_{m+n-2}\left(1-\frac{\alpha}{2}\right)S_w\sqrt{\dfrac{mc+n}{mn}},\overline{X}-\overline{Y}+\operatorname{t}_{m+n-2}\left(1-\frac{\alpha}{2}\right)S_w\sqrt{\dfrac{mc+n}{mn}}\right]
		\end{equation*}
		其中:
		\begin{equation*}
			S_w^2=\frac{(m-1)S_X^2+(n-1)S_Y^2/c}{m+n-2}
		\end{equation*}
		\item 若总体$F$服从$\operatorname{N}(\mu,\sigma_1^2)$,$G$服从$\operatorname{N}(\nu,\sigma_2^2)$,在$\mu,\nu$已知时$g(\theta)=\dfrac{\sigma_1^2}{\sigma_2^2}$置信水平为$1-\alpha$的区间估计为:
		\begin{equation*}
			\left[\frac{n\sum\limits_{i=1}^{m}(X_i-\mu)^2}{m\sum\limits_{i=1}^{n}(Y_i-\nu)^2\operatorname{F}_{m,n}\left(1-\frac{\alpha}{2}\right)},\frac{n\sum\limits_{i=1}^{m}(X_i-\mu)^2}{m\sum\limits_{i=1}^{n}(Y_i-\nu)^2\operatorname{F}_{m,n}\left(\frac{\alpha}{2}\right)}\right]
		\end{equation*}
		$\mu,\nu$未知时$g(\theta)=\dfrac{\sigma_1^2}{\sigma_2^2}$置信水平为$1-\alpha$的区间估计为:
		\begin{equation*}
			\left[\frac{S_X^2}{S_Y^2\operatorname{F}_{m-1,n-1}\left(1-\frac{\alpha}{2}\right)},\frac{S_X^2}{S_Y^2\operatorname{F}_{m-1,n-1}\left(\frac{\alpha}{2}\right)}\right]
		\end{equation*}
	\end{enumerate}
\end{theorem}
\begin{proof}
	(1)由\cref{prop:Measure}(3)(次有限可加性)可得:
	\begin{align*}
		P(\{g(\theta)\leqslant\delta_1(\mathbf{X})\}\cup\{g(\theta)\geqslant\delta_2(\mathbf{X})\})&\leqslant P(\{g(\theta)\leqslant\delta_1(\mathbf{X})\})+P(\{g(\theta)\geqslant\delta_2(\mathbf{X})\}) \\
		&\leqslant\alpha_1+\alpha_2,\;\forall\;P\in\mathscr{P}
	\end{align*}
	根据\cref{prop:Measure}(2)可得:
	\begin{equation*}
		\forall\;P\in\mathscr{P},\;P\left(\{\delta_1(\mathbf{X})\leqslant g(\theta)\leqslant\delta_2(\mathbf{X})\}\right)\geqslant1-\alpha_1-\alpha_2
	\end{equation*}\par
	(2)\textbf{已知$\sigma^2$:}由\cref{theo:SamplingDist1}(1)和\cref{prop:MultiNormal}(2)可知:
	\begin{equation*}
		f(\mu,\overline{X})=\frac{\sqrt{m}(\overline{X}-\mu)}{\sigma}\sim\operatorname{N}(0,1)
	\end{equation*}
	根据(1)可知可取:
	\begin{equation*}
		u_{\frac{\alpha}{2}}\leqslant f(\mu,\overline{X})\leqslant u_{1-\frac{\alpha}{2}}
	\end{equation*}
	即:
	\begin{equation*}
		\left[\overline{X}-u_{1-\frac{\alpha}{2}}\sqrt{\dfrac{\sigma^2}{m}},\overline{X}-u_{1-\frac{\alpha}{2}}\sqrt{\dfrac{\sigma^2}{m}}\right]
	\end{equation*}\par
	\textbf{未知$\sigma^2$:}由\cref{theo:SamplingDist1}(4)可知:
	\begin{equation*}
		f(\mu,\overline{X})=\frac{\sqrt{m}(\overline{X}-\mu)}{S_X}\sim\operatorname{t}_{m-1}
	\end{equation*}
	根据(1)可知可取:
	\begin{equation*}
		\operatorname{t}_{m-1}\left(\frac{\alpha}{2}\right)\leqslant f(\mu,\overline{X})\leqslant\operatorname{t}_{m-1}\left(1-\frac{\alpha}{2}\right)
	\end{equation*}
	即:
	\begin{equation*}
		\left[\overline{X}-\operatorname{t}_{m-1}\left(1-\frac{\alpha}{2}\right)\sqrt{\dfrac{S_X^2}{m}},\overline{X}+\operatorname{t}_{m-1}\left(1-\frac{\alpha}{2}\right)\sqrt{\dfrac{S_X^2}{m}}\right]
	\end{equation*}\par
	(3)\textbf{已知$\mu$:}由\cref{prop:MultiNormal}(2)和$\chi^2$分布的定义可知:
	\begin{equation*}
		f\left(\sigma^2,\frac{1}{m}\sum_{i=1}^{m}(X_i-\mu)^2\right)=\frac{1}{\sigma^2}\sum_{i=1}^{m}(X_i-\mu)^2\sim\chi_{m}^2
	\end{equation*}
	根据(1)可知可取:
	\begin{equation*}
		\chi_{m}^2\left(\frac{\alpha}{2}\right)\leqslant f\left(\sigma^2,\frac{1}{m}\sum_{i=1}^{m}(X_i-\mu)^2\right)\leqslant\chi_{m}^2\left(1-\frac{\alpha}{2}\right)
	\end{equation*}
	即:
	\begin{equation*}
		\left[\frac{1}{\chi_{m-1}^2\left(1-\frac{\alpha}{2}\right)}\sum_{i=1}^{m}(X_i-\mu)^2,\frac{1}{\chi_{m-1}^2\left(\frac{\alpha}{2}\right)}\sum_{i=1}^{m}(X_i-\mu)^2\right]
	\end{equation*}\par
	\textbf{未知$\mu$:}由\cref{theo:SamplingDist1}(2)可知:
	\begin{equation*}
		f\left(\sigma^2,\frac{S_X^2}{m-1}\right)=\frac{(m-1)S_X^2}{\sigma^2}\sim\chi_{m-1}^2
	\end{equation*}
	根据(1)可知可取:
	\begin{equation*}
		\chi_{m-1}^2\left(\frac{\alpha}{2}\right)\leqslant f\left(\sigma^2,\frac{S_X^2}{m-1}\right)\leqslant\chi_{m-1}^2\left(1-\frac{\alpha}{2}\right)
	\end{equation*}
	即:
	\begin{equation*}
		\left[\frac{(m-1)S_X^2}{\chi_{m-1}^2\left(1-\frac{\alpha}{2}\right)},\frac{(m-1)S_X^2}{\chi_{m-1}^2\left(\frac{\alpha}{2}\right)}\right]
	\end{equation*}\par
	(4)\textbf{已知$\sigma_1^2,\sigma_2^2$:}由(1)(得到$\overline{X}$和$\overline{Y}$的分布)、$\seq{X}{m}$与$\seq{Y}{n}$相互独立(由\cref{prop:MultiNormal}(6)得到二维随机向量$(\overline{X},\overline{Y})^T$的分布)和\cref{prop:MultiNormal}(2)(对$(\overline{X},\overline{Y})^T$用二维行向量$(1,-1)$做线性变换,再做标准化\info{标准化})可得:
	\begin{equation*}
		f(\mu-\nu,\overline{X}-\overline{Y})=\frac{\overline{X}-\overline{Y}-(\mu-\nu)}{\sqrt{\dfrac{\sigma_1^2}{m}+\dfrac{\sigma_2^2}{n}}}\sim\operatorname{N}\left(0,1\right)
	\end{equation*}
	根据(1)可知可取:
	\begin{equation*}
		u_{\frac{\alpha}{2}}\leqslant f(\mu-\nu,\overline{X}-\overline{Y})\leqslant u_{1-\frac{\alpha}{2}}
	\end{equation*}
	即:
	\begin{equation*}
		\left[\overline{X}-\overline{Y}-u_{1-\frac{\alpha}{2}}\sqrt{\dfrac{\sigma_1^2}{m}+\dfrac{\sigma_2^2}{n}},\overline{X}-\overline{Y}+u_{1-\frac{\alpha}{2}}\sqrt{\dfrac{\sigma_1^2}{m}+\dfrac{\sigma_2^2}{n}}\right]
	\end{equation*}\par
	\textbf{$\sigma_1^2=\sigma_2^2$:}类似\cref{theo:SamplingDist1}(5)可得:
	\begin{equation*}
		f(\mu-\nu,\overline{X}-\overline{Y})=\dfrac{\overline{X}-\overline{Y}-(\mu-\nu)}{\sqrt{(m-1)S_X^2+(n-1)S_Y^2/c}}\sqrt{\dfrac{mn(m+n-2)}{mc+n}}\sim \operatorname{t}_{m+n-2}
	\end{equation*}
	根据(1)可知可取:
	\begin{equation*}
		\operatorname{t}_{m+n-2}\left(\frac{\alpha}{2}\right)\leqslant f(\mu-\nu,\overline{X}-\overline{Y})\leqslant \operatorname{t}_{m+n-2}\left(1-\frac{\alpha}{2}\right)
	\end{equation*}
	记:
	\begin{equation*}
		S_w^2=\frac{(m-1)S_X^2+(n-1)S_Y^2/c}{m+n-2}
	\end{equation*}
	于是可得:
	\begin{equation*}
		\left[\overline{X}-\overline{Y}-\operatorname{t}_{m+n-2}\left(1-\frac{\alpha}{2}\right)S_w\sqrt{\dfrac{mc+n}{mn}},\overline{X}-\overline{Y}+\operatorname{t}_{m+n-2}\left(1-\frac{\alpha}{2}\right)S_w\sqrt{\dfrac{mc+n}{mn}}\right]
	\end{equation*}\par
	(5)\textbf{已知$\mu,\nu$:}由\cref{prop:MultiNormal}(2)和$\chi^2$分布的定义可知:
	\begin{equation*}
		\delta_1(\mathbf{X})=\frac{1}{\sigma_1^2}\sum_{i=1}^{m}(X_i-\mu)^2\sim\chi_{m}^2,\quad\delta_2(\mathbf{Y})=\frac{1}{\sigma_2^2}\sum_{i=1}^{n}(Y_i-\nu)^2\sim\chi_{n}^2
	\end{equation*}
	因为$\seq{X}{m}$和$\seq{Y}{n}$相互独立,所以$\delta_1(\mathbf{X})$与$\delta_2(\mathbf{Y})$也相互独立,于是有:
	\begin{equation*}
		f\left(\frac{\sigma_1^2}{\sigma_2^2},\frac{S_X^2(n-1)}{S_Y^2(m-1)}\right)=\frac{n\sum\limits_{i=1}^{m}(X_i-\mu)^2\sigma_2^2}{m\sum\limits_{i=1}^{n}(Y_i-\nu)^2\sigma_1^2}\sim\operatorname{F}_{m,n}
	\end{equation*}
	根据(1)可知可取:
	\begin{equation*}
		\operatorname{F}_{m,n}\left(\frac{\alpha}{2}\right)\leqslant f\left(\frac{\sigma_1^2}{\sigma_2^2},\frac{S_X^2(n-1)}{S_Y^2(m-1)}\right)\leqslant\operatorname{F}_{m,n}\left(1-\frac{\alpha}{2}\right)
	\end{equation*}
	即:
	\begin{equation*}
		\left[\frac{n\sum\limits_{i=1}^{m}(X_i-\mu)^2}{m\sum\limits_{i=1}^{n}(Y_i-\nu)^2\operatorname{F}_{m,n}\left(1-\frac{\alpha}{2}\right)},\frac{n\sum\limits_{i=1}^{m}(X_i-\mu)^2}{m\sum\limits_{i=1}^{n}(Y_i-\nu)^2\operatorname{F}_{m,n}\left(\frac{\alpha}{2}\right)}\right]
	\end{equation*}\par
	\textbf{未知$\mu,\nu$:}由\cref{theo:SamplingDist1}(6)可得:
	\begin{equation*}
		f\left(\frac{\sigma_1^2}{\sigma_2^2},\frac{S_X^2(n-1)}{S_Y^2(m-1)}\right)=\frac{S_X^2\sigma_2^2}{S_Y^2\sigma_1^2}\sim\operatorname{F}_{m-1,n-1}
	\end{equation*}
	根据(1)可知可取:
	\begin{equation*}
		\operatorname{F}_{m-1.n-1}\left(\frac{\alpha}{2}\right)\leqslant f\left(\frac{\sigma_1^2}{\sigma_2^2},\frac{S_X^2(n-1)}{S_Y^2(m-1)}\right)\leqslant\operatorname{F}_{m-1,n-1}\left(1-\frac{\alpha}{2}\right)
	\end{equation*}
	即:
	\begin{equation*}
		\left[\frac{S_X^2}{S_Y^2\operatorname{F}_{m-1,n-1}\left(1-\frac{\alpha}{2}\right)},\frac{S_X^2}{S_Y^2\operatorname{F}_{m-1,n-1}\left(\frac{\alpha}{2}\right)}\right]\qedhere
	\end{equation*}
\end{proof}
\subsubsection{样本量问题}
\begin{definition}
	称置信区间的半径为\gls{MOE}。
\end{definition}
\begin{note}
	可以通过控制误差幅度来计算样本量,比如我们要控制误差幅度在$a$以内,列出不等式可以去计算满足要求的样本量范围,一般来讲误差幅度越小所需的样本量就越大,由此可得到最小样本量。当方程难以求解时,可以采用Monte Carlo算法计算出大量样本量与其对应的MOE值,从而选择出合适的样本量。
\end{note}