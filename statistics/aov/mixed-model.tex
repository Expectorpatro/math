\section{混合模型下的两因子方差分析}
仅讨论等重复情形。
\begin{table}[H] 
	\centering
	\begin{tabularx}{\textwidth}
		{c|>{\centering\arraybackslash}X>{\centering\arraybackslash}Xc>{\centering\arraybackslash}X}
		\hline
		\diagbox{因子$A$}{因子$B$} & $B_1$ & $B_2$ & $\cdots$ & $B_b$ \\ \hline
		$A_1$ & 
		$y_{111}, y_{112}, \dots, y_{11m}$ & 
		$y_{121}, y_{122}, \dots, y_{12m}$ & 
		$\cdots$ & 
		$y_{1b1}, y_{1b2}, \dots, y_{1bm}$ \\ 
		$A_2$ & 
		$y_{211}, y_{212}, \dots, y_{21m}$ & 
		$y_{221}, y_{222}, \dots, y_{22m}$ & 
		$\cdots$ & 
		$y_{2b1}, y_{2b2}, \dots, y_{2bm}$ \\
		$\vdots$ & 
		$\vdots$ & 
		$\vdots$ & 
		& 
		$\vdots$ \\
		$A_a$ & 
		$y_{a11}, y_{a12}, \dots, y_{a1m}$ & 
		$y_{a21}, y_{a22}, \dots, y_{a2m}$ & 
		$\cdots$ & 
		$y_{ab1}, y_{ab2}, \dots, y_{abm}$ \\ 
		\hline
	\end{tabularx}
	\caption{等重复两因子试验数据表}
\end{table}
其中$y_{ijk}$表示在因子A的第$i$个水平$A_i$和因子B的第$j$个水平$B_j$下第$k$次重复试验的观察值。设因子A是固定地,因子B是随机的。

\subsection{统计模型}
此时的统计模型为:
\begin{equation*}\label{model:mixed-two-way-anova}
	\begin{cases}
		y_{ijk}=\mu+\tau_i+\beta_j+(\tau\beta)_{ij}+\varepsilon_{ijk} \\
		\text{诸}\varepsilon_{ijk}\quad\mathrm{i.i.d.~}N(0,\sigma^2) \\
		\text{诸}\beta_j\quad\mathrm{i.i.d.~}N(0,\sigma_\beta^2) \\
		\text{诸}(\tau\beta)_{ij}\quad\mathrm{i.i.d.~}N(0,\frac{a-1}{a}\sigma_{\tau\beta}^2) \\
		\sum\limits_{i=1}^a\tau_i=0,\quad\sum\limits_{i=1}^a(\tau\beta)_{ij}=0 \\
		\text{诸}(\tau\beta)_{ij}\text{彼此之间是不相关的} \\
		i=1,2,\dots,a,\;j=1,2,\dots,b,\;k=1,2,\dots,m
	\end{cases}
\end{equation*}
这里$(\tau\beta)_{ij}$的方差写成这样是为了让底下的计算更加简洁。

\subsection{统计假设}
此时需要检验如下三个零假设:
\begin{equation*}
	\begin{cases}
		H_{01}:\tau_1=\tau_2=\cdots=\tau_a=0, \\
		H_{02}:\sigma_\beta^2=0 \\
		H_{03}:\sigma_{\tau\beta}^2=0
	\end{cases}
\end{equation*}

\subsection{方差分析}
\subsubsection{偏差平方和的分解}
公式仍然成立:
\begin{equation*}
	SST=SSA+SSB+SSAB+SSe
\end{equation*}
\subsubsection{各平方和的期望}
这里直接列出各平方和的期望。
\begin{gather*}
	E(MSA)=\sigma^2+m\sigma_{\tau\beta}^2+\frac{bm\sum\limits_{i=1}^a\tau_i^2}{a-1} \\
	E(MSB)=\sigma^2+am\sigma_\beta^2 \\
	E(MSAB)=\sigma^2+m\sigma_{\tau\beta}^2 \\
	E(MSe)=\sigma^2
\end{gather*}
\subsubsection{统计量及其分布}
统计量及其分布如下:
\begin{gather*}
	F_A=\frac{MSA}{MSAB}\sim F(a-1,\;(a-1)(b-1)) \\
	F_B=\frac{MSB}{MSe}\sim F(b-1,\;ab(m-1)) \\
	F_{AB}=\frac{MSAB}{MSe}\sim F((a-1)(b-1),\;ab(m-1))
\end{gather*}
\subsubsection{拒绝域}
显著性水平为$\alpha$时的拒绝域为:
\begin{gather*}
	F_A>F_{1-\alpha}(a-1,\;(a-1)(b-1)) \\
	F_B>F_{1-\alpha}(b-1,\;ab(m-1)) \\
	F_{AB}>F_{1-\alpha}((a-1)(b-1),\;ab(m-1))
\end{gather*}
\subsection{方差分析表}
\begin{table}[H]
	\centering
	\begin{tabularx}{\textwidth}
		{>{\centering\arraybackslash}c|*{5}{>{\centering\arraybackslash}X}}
		\toprule
		来源   &平方和&自由度&均方和             &F值  \\ 
		\midrule
		因子A(固定)&SSA&$f_A=a-1$ &$\frac{SSA}{a-1}$ &$F=\frac{MSA}{MSAB}$\\
		因子B(随机)&SSB&$f_B=b-1$ &$\frac{SSB}{b-1}$ &$F=\frac{MSB}{MSe}$\\
		交互作用AB &SSAB &$f_{AB}=(a-1)(b-1)$ &$\frac{SSAB}{(a-1)(b-1)}$ &$F=\frac{MSAB}{MSe}$ \\
		误差   &SSe  &$f_e=ab(m-1)$ &$\frac{SSe}{ab(m-1)}$ & \\
		总     &SST  &$f_T=abm-1$ &                  & \\
		\bottomrule
	\end{tabularx}
	\caption{等重复、有交互作用的混合模型下两因子试验方差分析表}
\end{table}
平方和公式可按下列公式计算:
\begin{equation*}
	\begin{cases}
		SST=\sum\limits_{i=1}^a\sum\limits_{j=1}^b\sum\limits_{k=1}^my_{ijk}^2-\frac{y_{...}^2}{abm} \\
		SSA=\sum\limits_{i=1}^a\frac{y_{i..}^2}{bm}-\frac{y_{...}^2}{abm} \\
		SSB=\sum\limits_{j=1}^b\frac{y_{.j.}^2}{am}-\frac{y_{...}^2}{abm} \\
		SSAB=\sum\limits_{i=1}^a\sum\limits_{j=1}^b\frac{y_{ij.}^2}{m}-\frac{y_{...}^2}{abm}-SSA-SSB \\
		SSe=SST-SSA-SSB-SSAB
	\end{cases}
\end{equation*}

\subsection{多重比较比较问题}
如果固定因子A显著,则可使用Duncan多重比较法,但此时需要注意拒绝域应修改为如下形式:
\begin{equation*}
	A:R_p>r_{1-\alpha}(p,f)\sqrt{\frac{MSAB}{bm}}
\end{equation*}

\subsection{参数估计}
\subsubsection{点估计}
下给出混合模型参数的点估计:
\begin{gather*}
	\hat{\mu}=\bar{y}_{...},\quad\hat{\tau}_i=\bar{y}_{i..}-\bar{y}_{...},\quad i=1,2,\dots,a \\
	\hat{\sigma^2}=MSe,\quad\hat{\sigma^2}_\beta=\frac{MSB-MSe}{am},\quad\hat{\sigma^2}_{\tau\beta}=\frac{MSAB-MSe}{m}
\end{gather*}