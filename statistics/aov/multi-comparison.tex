\section{多重假设检验}
当方差分析拒绝零假设,认为因子的不同水平对实验结果的影响有显著差异时,随之而来的问题就是到底哪几个因子之间是有显著差异的,此时就需要进行比较。但是如果对于每一对水平都使用通常的$t$检验的话,将大大提高整个检验问题犯第一类错误的概率(见\cref{reason for multi-comparison})。本节先介绍一般的对比,再介绍等重复情况下的Duncan多重比较法与一般情形下的Scheffe多重比较法。

\subsection{对比}
\subsubsection{对比的定义}
\begin{definition}
	对比是指因子诸效应的一个线性组合:
	\begin{equation*}
		\begin{cases}
			c=\sum\limits_{i=1}^ac_i\tau_i,\\
			s.t.\quad\sum\limits_{i=1}^ac_i=0
		\end{cases}
	\end{equation*}
	因为$\mu_i=\mu+\tau_i$,所以对比也可表示为:
	\begin{equation*}
		\begin{cases}
			c=\sum\limits_{i=1}^ac_i\mu_i,\\
			s.t.\quad\sum\limits_{i=1}^ac_i=0
		\end{cases}
	\end{equation*}
	全体对比构成一个$a-1$维线性空间:
	\begin{equation*}
		\{\mathbf{c}=(c_1,c_2,\dots,c_a)':\sum_{i=1}^ac_i=0\}
	\end{equation*}
\end{definition}
\subsubsection{目的}
我们此时希望检验假设:
\begin{equation*}
	H_0:\sum_{i=1}^ac_i\mu_i=0,\quad s.t.\sum_{i=1}^ac_i=0
\end{equation*}
\subsubsection{假设的检验方法}
由固定效应下的单因子方差分析的统计模型\cref{model:fixed-effect-one-way-anova},可知:
\begin{equation*}
	\sum_{i=1}^ac_i\bar{y}_{i.}\sim N(\sum_{i=1}^ac_i\mu_i,\;\sum_{i=1}^a\frac{c_i^2\sigma^2}{n_i})
\end{equation*}
定义对比的平方和为:
\begin{equation*}
	SSc=\frac{\left(\sum\limits_{i=1}^ac_i\bar{y}_{i.}\right)^2}{\sum\limits_{i=1}^a\dfrac{c_i^2}{n_i}}
\end{equation*}
当$H_0$成立时,$\dfrac{SSc}{\sigma^2}$是一个标准正态变量的平方,也就是说它服从$\chi^2(1)$。\info{记得证明独立}
所以,当$H_0$成立的时候,统计量
\begin{equation*}
	F=\frac{SSc}{MSe}\sim\text{F}(1,\;n-a)
\end{equation*}
MSe是$\sigma^2$的无偏估计,而当$H_0$成立时,SSc也是$\sigma^2$的无偏估计。如果F值很大,则有理由怀疑零假设(从$\left(\sum\limits_{i=1}^ac_i\bar{y}_{i.}\right)^2$较大这一点怀疑它的期望不为$0$)。所以$H_0$的拒绝域是右向单尾的,显著性水平为$\alpha$时的拒绝域为:
\begin{equation*}
	F=\frac{SSc}{MSe}>F_{1-\alpha}(1,\;n-a)
\end{equation*}

\subsection{正交对比}
仅讨论等重复情况下对比系数向量标准化(即模长为$1$)的正交对比。
\subsubsection{正交对比的定义}
对比有一种特殊情况,即正交对比:
\begin{definition}
	$c=\sum\limits_{i=1}^ac_i\mu_i$和$d=\sum\limits_{i=1}^ad_i\mu_i$是两个对比。若:
	\begin{equation*}
		\sum_{i=1}^ac_id_i=0
	\end{equation*}
	则称这两个对比为正交对比。
\end{definition}
\subsubsection{目的}
由对比空间的维数可知,线性无关的正交对比组最多有$a-1$个对比。而正交对比即是为了检验任意$a-1$个对比的值是否为$0$:
\begin{gather*}
	c^1=\sum_{i=1}^ac_{1,i}\mu_i \\
	c^2=\sum_{i=1}^ac_{2,i}\mu_i \\
	\cdots\cdots \\
	c^{(a-1)}=\sum_{i=1}^ac_{a-1,i}\mu_i
\end{gather*}
\subsubsection{假设的检验方法}
只需注意到此时:
\begin{equation*}
	SScj=\frac{\left(\sum\limits_{i=1}^ac_{j,i}\bar{y}_{i.}\right)^2}{\sum\limits_{i=1}^a\dfrac{c_{j,i}^2}{m}}=m\left(\sum\limits_{i=1}^ac_{j,i}\bar{y}_{i.}\right)^2
\end{equation*}
剩余步骤与对比一样。
\subsubsection{正交对比与SSA的关系}
上述对比的一个无偏估计为:
\begin{gather*}
	\hat{c}^1=\sum_{i=1}^ac_{1,i}\bar{y}_{i.} \\
	\hat{c}^2=\sum_{i=1}^ac_{2,i}\bar{y}_{i.} \\
	\cdots\cdots \\
	\hat{c}^{(a-1)}=\sum_{i=1}^ac_{a-1,i}\bar{y}_{i.}
\end{gather*}
令:
\begin{equation*}
	\hat{c}^{a}=\sum_{i=1}^a\frac{1}{\sqrt{a}}\bar{y}_{i.}
\end{equation*}
需要注意这并不是一个对比。\par
注意到:
\begin{equation*}
	(\hat{c}^1,\hat{c}^2,\dots,\hat{c}^a)'=A(\bar{y}_{1.},\bar{y}_{2.},\dots,\bar{y}_{a.})'
\end{equation*}
这之中的矩阵$A$是一个正交矩阵。所以:
\begin{equation*}
	(\hat{c}^1)^2+(\hat{c}^2)^2+\cdots+(\hat{c}^a)^2=\sum_{i=1}^a\bar{y}_{i.}^2
\end{equation*}
于是:
\begin{align*}
	(\hat{c}^1)^2+(\hat{c}^2)^2+\cdots+(\hat{c}^{a-1})^2
	&=\sum_{i=1}^a\bar{y}_{i.}^2-\frac{1}{a}\left(\sum\limits_{i=1}^a\bar{y}_{i.}\right)^2 \\
	&=\sum_{i=1}^a\left(\frac{y_{i.}}{m}\right)^2-\frac{1}{a}\left(\frac{\sum_{i=1}^am\bar{y}_{i.}}{m}\right)^2 \\
	&=\sum_{i=1}^a\frac{y_{i.}^2}{m^2}-\frac{y_{..}^2}{am^2} \\
	&=\frac{1}{m}SSA
\end{align*}
再由:
\begin{equation*}
	SScj=\frac{\left(\sum\limits_{i=1}^ac_{j,i}\bar{y}_{i.}\right)^2}{\sum\limits_{i=1}^a\dfrac{c_{j,i}^2}{m}}=\frac{(\hat{c}^j)^2}{\frac{1}{m}}=m(\hat{c}^j)^2
\end{equation*}
所以:
\begin{equation*}
	SSc1+SSc2+\cdots+SSc(a-1)=SSA
\end{equation*}

\subsection{Duncan多重比较法}
在很多问题中,我们往往不知道要如何构造适当的对比,也有可能检验$a-1$个以上的比较,此时对比的相关方法就无法使用了。接下来介绍这种情况下的一种解决方案,即Duncan多重比较法,它只适用于等重复情况。
\subsubsection{目的}
检验$H_0:\mu_i=\mu_j,\;\forall\;i\ne j$。
\subsubsection{$p$级极差的定义}
\begin{definition}
	将$a$个水平下观察值的平均值$\bar{y}_{1.},\bar{y}_{2.},\dots,\bar{y}_{a.}$从小到大排序。如果其中任意两个数在排序后中间还有$p-2,\;p\geqslant2$个数,那么这两个数的差称为$p$级极差,记为$R_p$。
\end{definition}
\subsubsection{统计量及其分布}
设$f$为SSe的自由度,$m$为重复次数,则Duncan多重比较法的统计量为:
\begin{equation*}
	r(p,f)=\frac{R_p}{\sqrt{\dfrac{MSe}{m}}}
\end{equation*}
在$\mu_1=\mu_2=\cdots=\mu_a$即$\tau_1=\tau_2=\cdots=\tau_a$的情况下,$r(p,f)$与$\mu,\;\sigma^2$无关。
\begin{proof}
	将$r(p,f)$分子分母同除$\sigma$可得:
	\begin{equation*}
		r(p.f)=\frac{\dfrac{R_p}{\sigma/\sqrt{m}}}{\sqrt{\dfrac{MSe}{\sigma^2}}}
	\end{equation*}
	注意到分母:
	\begin{equation*}
		\frac{MSe}{\sigma^2}=\frac{1}{f}\frac{SSe}{\sigma^2}
	\end{equation*}
	是一个服从$\chi^2(f)$分布变量的$\dfrac{1}{f}$倍,其分布与$\mu,\;\sigma^2$无关。\par
	在零假设成立的情况下:
	\begin{equation*}
		\frac{\bar{y}_{i.}-\mu}{\sigma/\sqrt{m}}\sim N(0,\;1)
	\end{equation*}
	这个分布与$\mu,\;\sigma^2$无关,所以分子:
	\begin{equation*}
		\frac{R_p}{\sigma/\sqrt{m}}=\max(\frac{\bar{y}_{1.}-\mu}{\sigma/\sqrt{m}},\dots,\frac{\bar{y}_{a.}-\mu}{\sigma/\sqrt{m}})-\min(\frac{\bar{y}_{1.}-\mu}{\sigma/\sqrt{m}},\dots,\frac{\bar{y}_{a.}-\mu}{\sigma/\sqrt{m}})
	\end{equation*}
	也与$\mu,\;\sigma^2$无关。\par
	综上,此时$r(p,f)$与$\mu,\;\sigma^2$无关。
\end{proof}
\subsubsection{检验原理}
当$\mu_i=\mu_j$不成立时,对应的$R_p$会较大。因此当$r(p,f)$较大时,有理由怀疑零假设。所以$H_0$的拒绝域是右向单尾的,显著性水平为$\alpha$时的拒绝域为:
\begin{equation*}
	r(p,f)>r_{1-\alpha}(p,f)
\end{equation*}
即:
\begin{equation*}
	R_p>r_{1-\alpha}(p,f)\sqrt{\frac{MSe}{m}}
\end{equation*}
\subsubsection{$r(p,f)$分布的Monte Carlo模拟}
\begin{algorithm}
	\caption{Duncan 多重比较法统计量分布的蒙特卡洛模拟}
	\begin{algorithmic}[1]
		\State \textbf{Input:} $m$, $a$, $p$, $f$, $N$ \Comment{组内重复次数、组数、极差的级数、SSe的自由度、模拟次数}
		\State \textbf{Output:} $r(p,f)$的模拟分布
		
		\State 初始化模拟值存储向量:$List\gets\emptyset$
		\For{$i \gets 1$ to $N$}
		\State 生成$a$个随机数$x_i\sim N(0,1),\;i=1,2,\dots,a$
		\State 计算$\frac{R_p}{\sigma/\sqrt{m}}$:
		\begin{equation*}
			\frac{R_p}{\sigma/\sqrt{m}}=\max\limits_{i=1,2,\dots,a}\{x_i\}-\min\limits_{i=1,2,\dots,a}\{x_i\}
		\end{equation*}
		\State 从$\chi^2(f)$中产生一个样本记为$\chi^2$
		\State 计算$r(p,f)$:
		\begin{equation*}
			r(p,f)=\frac{\dfrac{R_p}{\sigma/\sqrt{m}}}{\sqrt{\chi^2/f}}
		\end{equation*}
		\State 将 $r(p,f)$加入$List$
		\EndFor
		\State 返回$List$
	\end{algorithmic}
\end{algorithm}
\subsubsection{检验步骤}
将$\bar{y}_{1.},\bar{y}_{2.},\dots,\bar{y}_{a.}$从小到大排序为$\bar{y}^1,\bar{y}^2,\dots,\bar{y}^a$。令:
\begin{equation*}
	r_{1-\alpha}(p,f)\sqrt{\frac{MSe}{m}}=R_p^*
\end{equation*}
按以下顺序进行比较:
\begin{gather*}
	\bar{y}^a-\bar{y}^1\text{与$R_a^*$进行比较} \\
	\bar{y}^a-\bar{y}^2\text{与$R_{a-1}^*$进行比较} \\
	\cdots\cdots \\
	\bar{y}^a-\bar{y}^{a-1}\text{与$R_2^*$进行比较} \\
	\bar{y}^{a-1}-\bar{y}^1\text{与$R_{a-1}^*$进行比较} \\
	\bar{y}^{a-1}-\bar{y}^2\text{与$R_{a-2}^*$进行比较} \\
	\cdots\cdots
\end{gather*}
直到全部$\binom{a}{2}$对水平均值比较完为止。

\subsection{Scheffe多重比较法}
Scheffe证明了,在固定效应下的单因子方差分析统计模型下(即\cref{model:fixed-effect-one-way-anova}),对显著性水平$\alpha$,一切对比$c$的值同时满足不等式:
\begin{equation*}
	|\hat{c}-c|=\left|\sum_{i=1}^ac_i\bar{y}_{i.}-c\right|\leqslant\sqrt{(a-1)F_{1-\alpha}(a-1,\;n-a)MSe\sum\limits_{i=1}^a\dfrac{c_i^2}{n_i}}
\end{equation*}
的概率等于$1-\alpha$。也就是说,$H_0:\sum\limits_{i=1}^ac_i\mu_i=0$的拒绝域都是:
\begin{equation*}
	|\hat{c}|>\sqrt{(a-1)F_{1-\alpha}(a-1,\;n-a)MSe\sum\limits_{i=1}^a\dfrac{c_i^2}{n_i}}
\end{equation*}