\section{多重假设检验}

\subsection{正交对比}
仅讨论等重复情况下对比系数向量标准化(即模长为$1$)的正交对比。
\subsubsection{正交对比的定义}
对比有一种特殊情况,即正交对比:
\begin{definition}
	$c=\sum\limits_{i=1}^ac_i\mu_i$和$d=\sum\limits_{i=1}^ad_i\mu_i$是两个对比。若:
	\begin{equation*}
		\sum_{i=1}^ac_id_i=0
	\end{equation*}
	则称这两个对比为正交对比。
\end{definition}
\subsubsection{目的}
由对比空间的维数可知,线性无关的正交对比组最多有$a-1$个对比。而正交对比即是为了检验任意$a-1$个对比的值是否为$0$:
\begin{gather*}
	c^1=\sum_{i=1}^ac_{1,i}\mu_i \\
	c^2=\sum_{i=1}^ac_{2,i}\mu_i \\
	\cdots\cdots \\
	c^{(a-1)}=\sum_{i=1}^ac_{a-1,i}\mu_i
\end{gather*}
\subsubsection{假设的检验方法}
只需注意到此时:
\begin{equation*}
	SScj=\frac{\left(\sum\limits_{i=1}^ac_{j,i}\bar{y}_{i.}\right)^2}{\sum\limits_{i=1}^a\dfrac{c_{j,i}^2}{m}}=m\left(\sum\limits_{i=1}^ac_{j,i}\bar{y}_{i.}\right)^2
\end{equation*}
剩余步骤与对比一样。
\subsubsection{正交对比与SSA的关系}
上述对比的一个无偏估计为:
\begin{gather*}
	\hat{c}^1=\sum_{i=1}^ac_{1,i}\bar{y}_{i.} \\
	\hat{c}^2=\sum_{i=1}^ac_{2,i}\bar{y}_{i.} \\
	\cdots\cdots \\
	\hat{c}^{(a-1)}=\sum_{i=1}^ac_{a-1,i}\bar{y}_{i.}
\end{gather*}
令:
\begin{equation*}
	\hat{c}^{a}=\sum_{i=1}^a\frac{1}{\sqrt{a}}\bar{y}_{i.}
\end{equation*}
需要注意这并不是一个对比。\par
注意到:
\begin{equation*}
	(\hat{c}^1,\hat{c}^2,\dots,\hat{c}^a)'=A(\bar{y}_{1.},\bar{y}_{2.},\dots,\bar{y}_{a.})'
\end{equation*}
这之中的矩阵$A$是一个正交矩阵。所以:
\begin{equation*}
	(\hat{c}^1)^2+(\hat{c}^2)^2+\cdots+(\hat{c}^a)^2=\sum_{i=1}^a\bar{y}_{i.}^2
\end{equation*}
于是:
\begin{align*}
	(\hat{c}^1)^2+(\hat{c}^2)^2+\cdots+(\hat{c}^{a-1})^2
	&=\sum_{i=1}^a\bar{y}_{i.}^2-\frac{1}{a}\left(\sum\limits_{i=1}^a\bar{y}_{i.}\right)^2 \\
	&=\sum_{i=1}^a\left(\frac{y_{i.}}{m}\right)^2-\frac{1}{a}\left(\frac{\sum_{i=1}^am\bar{y}_{i.}}{m}\right)^2 \\
	&=\sum_{i=1}^a\frac{y_{i.}^2}{m^2}-\frac{y_{..}^2}{am^2} \\
	&=\frac{1}{m}SSA
\end{align*}
再由:
\begin{equation*}
	SScj=\frac{\left(\sum\limits_{i=1}^ac_{j,i}\bar{y}_{i.}\right)^2}{\sum\limits_{i=1}^a\dfrac{c_{j,i}^2}{m}}=\frac{(\hat{c}^j)^2}{\frac{1}{m}}=m(\hat{c}^j)^2
\end{equation*}
所以:
\begin{equation*}
	SSc1+SSc2+\cdots+SSc(a-1)=SSA
\end{equation*}
