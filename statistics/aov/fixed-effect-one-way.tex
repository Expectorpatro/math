\section{固定效应下的单因子方差分析}
\begin{table}[H]
	\centering
	\begin{tabularx}{\textwidth}
		{>{\centering\arraybackslash}c|*{4}{>{\centering\arraybackslash}X}}
		\hline
		水平   & \multicolumn{4}{c}{观测值} \\ 
		\hline
		$A_1$    & $y_{11}$ & $y_{12}$  & $\cdots$  & $y_{1n_1}$ \\
		$A_2$    & $y_{21}$ & $y_{22}$  & $\cdots$  & $y_{2n_2}$ \\
		$\vdots$ & $\vdots$ & $\vdots$  &           & $\vdots$   \\
		$A_a$    & $y_{a1}$ & $y_{a2}$  & $\cdots$  & $y_{an_a}$ 
		\\
		\hline
	\end{tabularx}
	\caption{固定效应下的单因子试验数据}
\end{table}
其中$y_{ij}$表示在第$i$个水平$A_i$下第$j$次重复试验的观察值。记$n=\sum\limits_{i=1}^{a}n_i$。

\subsection{统计模型}
假设一个数据$y$由两部分组成:
\begin{enumerate}
	\item 因子的影响部分$\mu$,随因子水平的变化而变化。
	\item 试验的随机误差$\varepsilon$,假设所有随机误差来自同一个正态总体$N(0,\sigma^2)$。
\end{enumerate}
则统计模型可写作:
\begin{equation*}
	\begin{cases}
		y_{ij}=\mu_i+\varepsilon_{ij} \\
		\text{诸}\varepsilon_{ij}\quad\mathrm{i.i.d.~}N(0,\sigma^2)
	\end{cases}
	\qquad i=1,2,\dots,a,\;j=1,2,\dots,n_i
\end{equation*}
还可将统计模型写成意义更清晰的形式,记:
\begin{equation*}
	\mu=\frac{1}{n}\sum_{i=1}^an_i\mu_i,\quad\tau_i=\mu_i-\mu,\;i=1,2,\dots,a
\end{equation*}
称$\mu$为一般平均(这里表示因子A的这$a$个水平对数据的一般影响),$\tau_i$为因子A第$i$个水平$A_i$的效应。那么统计模型即可改写为:
\begin{equation*}\label{model:fixed-effect-one-way-anova}
	\begin{cases}
		y_{ij}=\mu+\tau_i+\varepsilon_{ij} \\
		\text{诸}\varepsilon_{ij}\quad\mathrm{i.i.d.~}N(0,\sigma^2) \\
		s.t.\quad\sum\limits_{i=1}^an_i\tau_i=0
	\end{cases}
	\qquad i=1,2,\dots,a,\;j=1,2,\dots,n_i
\end{equation*}

\subsection{统计假设}\label{reason for multi-comparison}
由上述统计模型,方差分析即需要判断$\tau_1,\tau_2,\dots,\tau_a$是否相同。如果我们对这$a$个参数两两进行比较,一共需要检验$\binom{a}{2}$个不同的假设。对于其中一个假设,如果控制犯第一类错误的概率是$\alpha=0.05$并且这些检验是相互独立的,则错误地拒绝这$\binom{a}{2}$个假设中至少一个假设的概率为$1-(0.95)^\binom{a}{2}$。当$a=5$时,这个概率就已经达到了$40\%$。这种方式会大大提高犯第一类错误的概率。\par
为了控制犯第一类错误的概率,我们应该直接检验如下假设:
\begin{equation*}
	\begin{cases}
		H_0:\tau_1=\tau_2=\cdots=\tau_a=0, \\
		H_1:\tau_i\ne 0\quad\text{至少对一个$i$不成立}
	\end{cases}
\end{equation*}

\subsection{偏差平方和的分解}
记:
\begin{equation*}
	y_{..}=\sum_{i=1}^a\sum_{j=1}^{n_i}y_{ij},\quad\bar{y}_{..}=\frac{y_{..}}{n},\quad y_{i.}=\sum_{j=1}^{n_i}y_{ij},\quad \bar{y}_{i.}=\frac{y_{i.}}{n_i},\;i=1,2,\dots,a
\end{equation*}
全部数据之间的差异可用下述总偏差平方和表示:
\begin{equation*}
	SST=\sum_{i=1}^a\sum_{j=1}^{n_i}(y_{ij}-\bar{y}_{..})^2
\end{equation*}
引起数据$y_{ij}$之间差异的原因有两点:
\begin{enumerate}
	\item 因子A的$a$个水平对试验结果的影响不同。
	\item 试验具有误差。
\end{enumerate}
为了区分并比较这两个原因对数据的影响,需要对SST进行分解:
\begin{align*}
	SST
	&=\sum_{i=1}^a\sum_{j=1}^{n_i}\left[(y_{ij}-\bar{y}_{i.})+(\bar{y}_{i.}-\bar{y}_{..})\right]^2 \\
	&=\sum_{i=1}^a\sum_{j=1}^{n_i}(y_{ij}-\bar{y}_{i.})^2+\sum_{i=1}^a\sum_{j=1}^{n_i}(\bar{y}_{i.}-\bar{y}_{..})^2+\sum_{i=1}^a\sum_{j=1}^{n_i}2(y_{ij}-\bar{y}_{i.})(\bar{y}_{i.}-\bar{y}_{..}) \\
	&=\sum_{i=1}^a\sum_{j=1}^{n_i}(y_{ij}-\bar{y}_{i.})^2+\sum_{i=1}^an_i(\bar{y}_{i.}-\bar{y}_{..})^2+2\sum_{i=1}^a(\bar{y}_{i.}-\bar{y}_{..})\sum_{j=1}^{n_i}(y_{ij}-\bar{y}_{i.}) \\	&=\sum_{i=1}^a\sum_{j=1}^{n_i}(y_{ij}-\bar{y}_{i.})^2+\sum_{i=1}^an_i(\bar{y}_{i.}-\bar{y}_{..})^2
\end{align*}
\subsubsection{SSe}
记:
\begin{equation*}
	\varepsilon_{..}=\sum_{i=1}^a\sum_{j=1}^{n_i}\varepsilon_{ij},\quad\bar{\varepsilon}_{..}=\frac{\varepsilon_{..}}{n},\quad\varepsilon_{i.}=\sum_{j=1}^{n_i}\varepsilon_{ij},\quad\bar{\varepsilon}_{i.}=\frac{1}{n_i}\sum_{j=1}^{n_i}\varepsilon_{ij}
\end{equation*}
因此在上偏差平方和的分解中,第一项可写作:
\begin{equation*}
	\sum_{i=1}^a\sum_{j=1}^{n_i}(y_{ij}-\bar{y}_{i.})^2=\sum_{i=1}^a\sum_{j=1}^{n_i}(\varepsilon_{ij}-\bar{\varepsilon}_{i.})^2
\end{equation*}
因为该项完全是由误差引起的,所以称之为误差平方和或组内差异,记作SSe。\par
\subsubsection{SSA}
因为:
\begin{align*}
	\bar{y}_{i.}-\bar{y}_{..}
	&=\mu+\tau_i+\bar{\varepsilon}_{i.}-\frac{1}{n}\sum_{k=1}^a\sum_{j=1}^{n_k}(\mu+\tau_k+\varepsilon_{kj}) \\
	&=\mu+\tau_i+\bar{\varepsilon}_{i.}-\frac{1}{n}\left(n\mu+\sum_{k=1}^an_k\tau_k+\varepsilon_{..}\right) \\
	&=\mu+\tau_i+\bar{\varepsilon}_{i.}-\frac{1}{n}\left(n\mu+\varepsilon_{..}\right) \\
	&=\tau_i+\bar{\varepsilon}_{i.}-\bar{\varepsilon}_{..}
\end{align*}
所以:
\begin{equation*}
	\sum_{i=1}^an_i(\bar{y}_{i.}-\bar{y}_{..})^2=\sum_{i=1}^an_i(\tau_i+\bar{\varepsilon}_{i.}-\bar{\varepsilon}_{..})^2
\end{equation*}
其中的随机误差都是平均过的,期望值不变而且方差缩小了。所以这个平方和虽然受到水平变动和随机误差两方面的影响,但是当因子A的不同水平对试验结果有显著差异的时候,它主要受到因子A水平变动的影响。所以称该平方和为因子的平方和或组间差异,记作SSA。\par
\subsubsection{总偏差平方和分解公式}
综上,总偏差平方和有如下分解公式:
\begin{equation*}
	SST=SSA+SSe
\end{equation*}

\subsection{检验统计量}
\subsubsection{关于$\varepsilon$的一些结论}
下给出关于$\varepsilon$的一些结论:
\begin{gather*}
	\bar{\varepsilon}_{i.}\sim N(0,\frac{\sigma^2}{n_i}),\quad\bar{\varepsilon}_{..}\sim N(0,\frac{\sigma^2}{n}) \\
	E(\varepsilon_{ij}^2)=\sigma^2,\quad
	E(\bar{\varepsilon}_{i.}^2)=\frac{\sigma^2}{n_i},\quad
	E(\bar{\varepsilon}_{..}^2)=\frac{\sigma^2}{n},\quad
	E(\bar{\varepsilon}_{i.}\bar{\varepsilon}_{..})=\frac{\sigma^2}{n},\quad
	E(\varepsilon_{ij}\bar{\varepsilon}_{i.})=\frac{\sigma^2}{n_i}
\end{gather*}
\begin{proof}
	正态分布的两个结论可直接由独立正态随机变量的线性运算求得。
	\begin{align*}
		E(\bar{\varepsilon}_{ij}^2)&=Var(\bar{\varepsilon}_{ij})+[E(\bar{\varepsilon}_{ij})]^2=\sigma^2 \\
		E(\bar{\varepsilon}_{i.}^2)&=Var(\bar{\varepsilon}_{i.})+[E(\bar{\varepsilon}_{i.})]^2=\frac{\sigma^2}{n_i} \\
		E(\bar{\varepsilon}_{..}^2)&=Var(\bar{\varepsilon}_{..})+[E(\bar{\varepsilon}_{..})]^2=\frac{\sigma^2}{n}
	\end{align*}
	下求$E(\bar{\varepsilon}_{i.}\bar{\varepsilon}_{..})$:
	\begin{align*}
		E(\bar{\varepsilon}_{i.}\bar{\varepsilon}_{..})
		&=E\left[\left(\frac{1}{n_i}\sum_{j=1}^{n_i}\varepsilon_{ij}\right)\left(\frac{1}{n}\sum_{k=1}^a\sum_{j=1}^{n_i}\varepsilon_{kj}\right)\right] \\
		&=\frac{1}{n_in}E\left[\left(\sum_{j=1}^{n_i}\varepsilon_{ij}\right)^2+\left(\sum_{j=1}^{n_i}\varepsilon_{ij}\right)\left(\sum_{k\ne i}\sum_{j=1}^{n_k}\varepsilon_{kj}\right)\right] \\
		&=\frac{1}{n_in}E\left[\left(\sum_{j=1}^{n_i}\varepsilon_{ij}\right)^2\right]+\frac{1}{n_in}E\left[\left(\sum_{j=1}^{n_i}\varepsilon_{ij}\right)\left(\sum_{k\ne i}\sum_{j=1}^{n_k}\varepsilon_{kj}\right)\right]
	\end{align*}
	因为$\varepsilon_{ij}$彼此独立,所以上式中第二项为$0$:
	\begin{equation*}
		E\left[\left(\sum_{j=1}^{n_i}\varepsilon_{ij}\right)\left(\sum_{k\ne i}\sum_{j=1}^{n_k}\varepsilon_{kj}\right)\right]=E\left(\sum_{j=1}^{n_i}\varepsilon_{ij}\right)E\left(\sum_{k\ne i}\sum_{j=1}^{n_k}\varepsilon_{kj}\right)=0
	\end{equation*}
	所以:
	\begin{align*}
		E(\bar{\varepsilon}_{i.}\bar{\varepsilon}_{..})
		&=\frac{1}{n_in}E\left[\left(\sum_{j=1}^{n_i}\varepsilon_{ij}\right)^2\right] \\
		&=\frac{1}{n_in}E\left(\sum_{j=1}^{n_i}\varepsilon_{ij}^2+\sum_{k\ne j}\varepsilon_{ij}\varepsilon_{ik}\right) \\
		&=\frac{1}{n_in}\left[\sum_{j=1}^{n_i}E(\varepsilon_{ij}^2)+\sum_{k\ne j}E(\varepsilon_{ij}\varepsilon_{ik})\right] \\
		&=\frac{1}{n_in}\left[n_i\sigma^2+\sum_{k\ne j}E(\varepsilon_{ij})E(\varepsilon_{ik})\right] \\
		&=\frac{1}{n}\sigma^2
	\end{align*}
	下求$E(\varepsilon_{ij}\bar{\varepsilon}_{i.})$:
	\begin{align*}
		E(\varepsilon_{ij}\bar{\varepsilon}_{i.})
		&=E\left(\varepsilon_{ij}\frac{1}{n_i}\sum_{k=1}^{n_i}\varepsilon_{ik}\right) \\
		&=E\left(\frac{1}{n_i}\varepsilon_{ij}^2+\frac{1}{n_i}\sum_{k\ne j}\varepsilon_{ij}\varepsilon_{ik}\right) \\
		&=\frac{1}{n_i}\left[E(\varepsilon_{ij}^2)+\sum_{k\ne j}E(\varepsilon_{ij}\varepsilon_{ik})\right] \\
		&=\frac{1}{n_i}\left[\sigma^2+\sum_{k\ne j}E(\varepsilon_{ij})E(\varepsilon_{ik})\right] \\
		&=\frac{\sigma^2}{n_i}\qedhere
	\end{align*}
\end{proof}
\subsubsection{SSA的期望}
下求SSA的期望:
\begin{align*}
	E(SSA)
	&=E\left[\sum_{i=1}^an_i(\tau_i+\bar{\varepsilon}_{i.}-\bar{\varepsilon}_{..})^2\right] \\
	&=E\left[\sum_{i=1}^an_i(\tau_i^2+\bar{\varepsilon}_{i.}^2+\bar{\varepsilon}_{..}^2+2\tau_i\bar{\varepsilon}_{i.}-2\tau_i\bar{\varepsilon}_{..}-2\bar{\varepsilon}_{i.}\bar{\varepsilon}_{..})\right] \\
	&=\sum_{i=1}^an_i\tau_i^2+\sum_{i=1}^an_iE(\bar{\varepsilon}_{i.}^2)+\sum_{i=1}^an_iE(\bar{\varepsilon}_{..}^2)-2\sum_{i=1}^an_iE(\bar{\varepsilon}_{i.}\bar{\varepsilon}_{..}) \\
	&=\sum_{i=1}^an_i\tau_i^2+\sum_{i=1}^an_i\frac{\sigma^2}{n_i}+\sum_{i=1}^an_i\frac{\sigma^2}{n}-2\sum_{i=1}^an_i\frac{\sigma^2}{n} \\
	&=\sum_{i=1}^an_i\tau_i^2+(a-1)\sigma^2
\end{align*}
\subsubsection{SSe的期望}
下求SSe的期望:
\begin{align*}
	E(SSe)
	&=E\left[\sum_{i=1}^a\sum_{j=1}^{n_i}(\varepsilon_{ij}-\bar{\varepsilon}_{i.})^2\right] \\
	&=\sum_{i=1}^a\sum_{j=1}^{n_i}E[(\varepsilon_{ij}-\bar{\varepsilon}_{i.})^2] \\
	&=\sum_{i=1}^a\sum_{j=1}^{n_i}E(\varepsilon_{ij}^2+\bar{\varepsilon}_{i.}^2-2\varepsilon_{ij}\bar{\varepsilon}_{i.}) \\
	&=\sum_{i=1}^a\sum_{j=1}^{n_i}\left(\sigma^2+\frac{\sigma^2}{n_i}-\frac{2\sigma^2}{n_i}\right) \\
	&=(n-a)\sigma^2
\end{align*}
\subsubsection{构建统计量}
称$\frac{SSA}{a-1}$为因子A的均方和,记为MSA;称$\frac{SSe}{n-a}$为误差均方和,记为MSe。\par
由前述,MSe是$\sigma^2$的无偏估计,而当零假设成立时,MSA也是$\sigma^2$的一个无偏估计。如果二者比值很大,即MSA比MSe大很多($\sum\limits_{i=1}^a\tau_i^2$很大),我们就有理由怀疑零假设。由此构建统计量:
\begin{equation*}
	F=\frac{MSA}{MSe}=\frac{\frac{SSA}{a-1}}{\frac{SSe}{n-a}}
\end{equation*}
在该统计量的情况下,$H_0$的拒绝域是右向单尾的。下求该统计量的分布。

\subsection{统计量的分布}
上述统计量服从如下分布:
\begin{equation*}
	F\sim F(a-1,\;n-a)
\end{equation*}
所以$H_0$在显著性水平为$\alpha$时的拒绝域为:
\begin{equation*}
	F>F_{1-\alpha}(a-1,\;n-a)
\end{equation*}

\subsection{方差分析表}
\begin{table}[H]
	\centering
	\begin{tabularx}{\textwidth}
		{>{\centering\arraybackslash}c|*{5}{>{\centering\arraybackslash}X}}
		\toprule
		来源   &平方和&自由度&均方和             &F值  \\ 
		\midrule
		因子A&SSA&$f_A=a-1$ &$\frac{SSA}{a-1}$ &$F=\frac{MSA}{MSe}$\\
		误差   &SSe  &$f_e=n-a$ &$\frac{SSe}{n-a}$ & \\
		总     &SST  &$f_T=n-1$ &                  & \\
		\bottomrule
	\end{tabularx}
	\caption{固定效应下单因子试验方差分析表}
\end{table}
平方和公式可按下列公式计算:
\begin{equation*}
	\begin{cases}
		SST=\sum\limits_{i=1}^a\sum\limits_{j=1}^{n_i}y_{ij}^2-\frac{y_{..}^2}{n} \\
		SSA=\sum\limits_{i=1}^a\frac{y_{i.}^2}{n_i}-\frac{y_{..}^2}{n} \\
		SSe=SST-SSA
	\end{cases}
\end{equation*}

\subsection{参数估计}
固定效应下的单因素方差分析有三类参数:$\mu$,诸$\tau_i$和$\sigma^2$。下讨论这三类参数的点估计与区间估计问题。
\subsubsection{点估计}
参数的点估计如下:
\begin{gather*}
	\hat{\sigma^2}=MSe \\
	\hat{\mu}=\bar{y}_{..} \\
	\hat{\tau}_i=\bar{y}_{i.}-\bar{y}_{..},\;i=1,2,\dots,a
\end{gather*}
其中$\hat{\mu},\;\hat{\tau}_i$是使用最小二乘估计得到的。
\begin{proof}
	分别用$\hat{\mu},\;\hat{\tau}_i$表示$\mu$与诸$\tau_i$的估计,用$\hat{y}_{ij}=\hat{\mu}+\hat{\tau}_i$表示$y_{ij}$的估计,$i=1,2,\dots,a,\;j=1,2,\dots,n_i$。损失函数为:
	\begin{equation*}
		L=\sum_{i=1}^a\sum_{j=1}^{n_i}(y_{ij}-\hat{y}_{ij})^2=\sum_{i=1}^a\sum_{j=1}^{n_i}(y_{ij}-\hat{\mu}-\hat{\tau}_i)^2
	\end{equation*}
	最小二乘解需要满足:
	\begin{equation*}
		\begin{cases}
			\vspace{2ex}
			\dfrac{\partial L}{\partial\hat{\mu}}=0, \\
			\vspace{2ex}
			\dfrac{\partial L}{\partial\hat{\tau}_i}=0,\quad
			i=1,2,\dots,a \\
			\vspace{2ex}
			\sum\limits_{i=1}^an_i\hat{\tau}_i=0
		\end{cases}
	\end{equation*}
	解得:
	\begin{equation*}
		\begin{cases}
			\hat{\mu}=\bar{y}_{..} \\
			\hat{\tau}_i=\bar{y}_{i.}-\bar{y}_{..},\;i=1,2,\dots,a
		\end{cases}
		\qedhere
	\end{equation*}
\end{proof}
\subsubsection{区间估计}
\paragraph{$\mu_i=\mu+\tau_i$}
\begin{equation*}
	\bar{y}_{i.}\pm t_{1-\alpha}(n-a)\sqrt{\frac{MSe}{n_i}}
\end{equation*}
\paragraph{$\tau_i$}
\begin{equation*}
	\bar{y}_{i.}-\bar{y}_{..}\pm t_{1-\alpha}(n-a)\sqrt{MSe\left(\frac{1}{n_i}-\frac{1}{n}\right)}
\end{equation*}
\paragraph{$\mu_i-\mu_j=\tau_i-\tau_j$}
\begin{equation*}
	\bar{y}_{i.}-\bar{y}_{j.}\pm t_{1-\alpha}(n-a)\sqrt{MSe\left(\frac{1}{n_i}-\frac{1}{n_j}\right)}
\end{equation*}