\section{可加效应下的两因子方差分析}
可加效应模型一般不必进行重复实验,每个水平组合下只做一次实验就够了。


\subsection{统计假设}
可加效应下的两因子方差分析需要检验如下两个零假设:
\begin{equation*}
	\begin{cases}
		H_{01}:\tau_1=\tau_2=\cdots=\tau_a=0, \\
		H_{02}:\beta_1=\beta_2=\cdots=\beta_b=0
	\end{cases}
\end{equation*}

\subsection{偏差平方和的分解}
记:
\begin{gather*}
	y_{..}=\sum_{i=1}^a\sum_{j=1}^by_{ij},\quad
	\bar{y}_{..}=\frac{y_{..}}{ab} \\
	y_{i.}=\sum_{j=1}^by_{ij},\quad
	\bar{y}_{i.}=\frac{y_{i.}}{b},\quad i=1,2,\dots,a \\
	y_{.j}=\sum_{i=1}^ay_{ij},\quad
	\bar{y}_{.j}=\frac{y_{.j}}{a},\quad j=1,2,\dots,b 
\end{gather*}
全部数据之间的差异可用下述总偏差平方和表示:
\begin{equation*}
	SST=\sum_{i=1}^a\sum_{j=1}^b(y_{ij}-\bar{y}_{..})^2
\end{equation*}
引起数据$y_{ij}$之间差异的原因有三点:
\begin{enumerate}
	\item 因子A的$a$个水平对试验结果的影响不同。
	\item 因子B的$b$个水平对试验结果的影响不同。
	\item 试验具有误差。
\end{enumerate}
为了区分并比较这三个原因对数据的影响,需要对SST进行分解:
\begin{align*}
	SST
	&=\sum_{i=1}^a\sum_{j=1}^b(y_{ij}-\bar{y}_{..})^2 \\
	&=\sum_{i=1}^a\sum_{j=1}^b\left[(\bar{y}_{i.}-\bar{y}_{..})+(\bar{y}_{.j}-\bar{y}_{..})+(y_{ij}-\bar{y}_{i.}-\bar{y}_{.j}+\bar{y}_{..})\right]^2 \\
	&=\sum_{i=1}^a\sum_{j=1}^b(\bar{y}_{i.}-\bar{y}_{..})^2+\sum_{i=1}^a\sum_{j=1}^b(\bar{y}_{.j}-\bar{y}_{..})^2+\sum_{i=1}^a\sum_{j=1}^b(y_{ij}-\bar{y}_{i.}-\bar{y}_{.j}+\bar{y}_{..})^2 \\
	&\quad+2\sum_{i=1}^a\sum_{j=1}^b(\bar{y}_{i.}-\bar{y}_{..})(\bar{y}_{.j}-\bar{y}_{..}) \\
	&\quad+2\sum_{i=1}^a\sum_{j=1}^b(\bar{y}_{i.}-\bar{y}_{..})(y_{ij}-\bar{y}_{i.}-\bar{y}_{.j}+\bar{y}_{..}) \\
	&\quad+2\sum_{i=1}^a\sum_{j=1}^b(\bar{y}_{.j}-\bar{y}_{..})(y_{ij}-\bar{y}_{i.}-\bar{y}_{.j}+\bar{y}_{..})
\end{align*}
由定义可发现上式后三项为$0$(第三项需要注意交换求和顺序)。
\subsubsection{SSe}
记:
\begin{gather*}
	SSe=\sum_{i=1}^a\sum_{j=1}^b(y_{ij}-\bar{y}_{i.}-\bar{y}_{.j}+\bar{y}_{..})^2 \\
	\varepsilon_{..}=\sum_{i=1}^a\sum_{j=1}^b\varepsilon_{ij},\quad
	\bar{\varepsilon}_{..}=\frac{\varepsilon_{..}}{ab} \\
	\varepsilon_{i.}=\sum_{j=1}^b\varepsilon_{ij},\quad
	\bar{\varepsilon}_{i.}=\frac{1}{b}\sum_{j=1}^b\varepsilon_{ij},\quad i=1,2,\dots,a \\
	\varepsilon_{.j}=\sum_{i=1}^a\varepsilon_{ij},\quad
	\bar{\varepsilon}_{.j}=\frac{1}{a}\sum_{i=1}^a\varepsilon_{ij},\quad j=1,2,\dots,b \\
	\bar{\tau}=\frac{1}{a}\sum_{i=1}^a\tau_i=0,\quad
	\bar{\beta}=\frac{1}{b}\sum_{j=1}^b\beta_j=0
\end{gather*}
则有:
\begin{align*}
	SSe
	&=\sum_{i=1}^a\sum_{j=1}^b(y_{ij}-\bar{y}_{i.}-\bar{y}_{.j}+\bar{y}_{..})^2 \\
	&=\sum_{i=1}^a\sum_{j=1}^b(\mu+\tau_i+\beta_j+\varepsilon_{ij}-\mu-\tau_i-\bar{\beta}-\bar{\varepsilon}_{i.}-\mu-\bar{\tau}-\beta_j-\bar{\varepsilon}_{.j}+\mu+\bar{\tau}+\bar{\beta}+\bar{\varepsilon}_{..})^2 \\
	&=\sum_{i=1}^a\sum_{j=1}^b(\varepsilon_{ij}-\bar{\varepsilon}_{i.}-\bar{\varepsilon}_{.j}+\bar{\varepsilon}_{..})^2
\end{align*}
因为该项完全是由误差引起的,所以称之为误差平方和。
\subsubsection{SSA}
记:
\begin{equation*}
	SSA=\sum_{i=1}^a\sum_{j=1}^b(\bar{y}_{i.}-\bar{y}_{..})^2=b\sum_{i=1}^a(\bar{y}_{i.}-\bar{y}_{..})^2
\end{equation*}
可以发现:
\begin{align*}
	SSA&=b\sum_{i=1}^a(\bar{y}_{i.}-\bar{y}_{..})^2 \\
	&=b\sum_{i=1}^a(\mu+\tau_i+\bar{\beta}+\bar{\varepsilon}_{i.}-\mu-\bar{\tau}-\bar{\beta}-\bar{\varepsilon}_{..})^2 \\
	&=b\sum_{i=1}^a(\tau_i+\bar{\varepsilon}_{i.}-\bar{\varepsilon}_{..})^2
\end{align*}
其中的随机误差都是平均过的,期望值不变而且方差缩小了。所以这个平方和虽然受到水平变动和随机误差两方面的影响,但是当因子A的不同水平对试验结果有显著差异的时候,它主要受到因子A水平变动的影响。所以称该平方和为因子A的平方和。

\subsubsection{SSB}
记:
\begin{equation*}
	SSB=\sum_{i=1}^a\sum_{j=1}^b(\bar{y}_{.j}-\bar{y}_{..})^2=a\sum_{j=1}^b(\bar{y}_{.j}-\bar{y}_{..})^2
\end{equation*}
可以发现:
\begin{align*}
	SSB&=a\sum_{j=1}^b(\bar{y}_{.j}-\bar{y}_{..})^2 \\
	&=a\sum_{j=1}^b(\mu+\bar{\tau}+\beta_j+\bar{\varepsilon}_{.j}-\mu-\bar{\tau}-\bar{\beta}-\bar{\varepsilon}_{..})^2 \\
	&=a\sum_{j=1}^b(\beta_j+\bar{\varepsilon}_{.j}-\bar{\varepsilon}_{..})^2
\end{align*}
其中的随机误差都是平均过的,期望值不变而且方差缩小了。所以这个平方和虽然受到水平变动和随机误差两方面的影响,但是当因子B的不同水平对试验结果有显著差异的时候,它主要受到因子B水平变动的影响。所以称该平方和为因子B的平方和。\par
\subsubsection{总偏差平方和分解公式}
综上,总偏差平方和有如下分解公式:
\begin{equation*}
	SST=SSA+SSB+SSe
\end{equation*}

\subsection{检验统计量}
\subsubsection{关于$\varepsilon$的一些结论}
下给出关于$\varepsilon$的一些结论:
\begin{gather*}
	\bar{\varepsilon}_{i.}\sim N(0,\frac{\sigma^2}{b}),\quad\bar{\varepsilon}_{.j}\sim
	N(0,\frac{\sigma^2}{a}),\quad\bar{\varepsilon}_{..}\sim N(0,\frac{\sigma^2}{ab}) \\
	E(\varepsilon_{ij}^2)=\sigma^2,\quad
	E(\bar{\varepsilon}_{i.}^2)=\frac{\sigma^2}{b},\quad
	E(\bar{\varepsilon}_{.j}^2)=\frac{\sigma^2}{a},\quad
	E(\bar{\varepsilon}_{..}^2)=\frac{\sigma^2}{ab} \\
	E(\bar{\varepsilon}_{i.}\bar{\varepsilon}_{..})=\frac{\sigma^2}{ab},\quad
	E(\varepsilon_{.j}\bar{\varepsilon}_{..})=\frac{\sigma^2}{ab}
\end{gather*}
\begin{proof}
	正态分布的三个结论可直接由独立正态随机变量的线性运算求得。
	\begin{align*}
		E(\bar{\varepsilon}_{ij}^2)&=Var(\bar{\varepsilon}_{ij})+[E(\bar{\varepsilon}_{ij})]^2=\sigma^2 \\
		E(\bar{\varepsilon}_{i.}^2)&=Var(\bar{\varepsilon}_{i.})+[E(\bar{\varepsilon}_{i.})]^2=\frac{\sigma^2}{b} \\
		E(\bar{\varepsilon}_{.j}^2)&=Var(\bar{\varepsilon}_{.j})+[E(\bar{\varepsilon}_{.j})]^2=\frac{\sigma^2}{a} \\
		E(\bar{\varepsilon}_{..}^2)&=Var(\bar{\varepsilon}_{..})+[E(\bar{\varepsilon}_{..})]^2=\frac{\sigma^2}{ab}
	\end{align*}
	下求$E(\bar{\varepsilon}_{i.}\bar{\varepsilon}_{..})$:
	\begin{align*}
		E(\bar{\varepsilon}_{i.}\bar{\varepsilon}_{..})
		&=E\left[\left(\frac{1}{b}\sum_{j=1}^b\varepsilon_{ij}\right)\left(\frac{1}{ab}\sum_{i=1}^a\sum_{j=1}^b\varepsilon_{ij}\right)\right] \\
		&=\frac{1}{ab^2}E\left(\sum_{j=1}^b\varepsilon_{ij}^2\right) \\
		&=\frac{1}{ab^2}\sum_{j=1}^bE(\varepsilon_{ij}^2) \\
		&=\frac{\sigma^2}{ab}
	\end{align*}
	下求$E(\varepsilon_{.j}\bar{\varepsilon}_{..})$:
	\begin{align*}
		E(\varepsilon_{.j}\bar{\varepsilon}_{..})
		&=E\left[\left(\frac{1}{a}\sum_{i=1}^a\varepsilon_{ij}\right)\left(\frac{1}{ab}\sum_{i=1}^a\sum_{j=1}^b\varepsilon_{ij}\right)\right] \\
		&=\frac{1}{a^2b}E\left(\sum_{i=1}^a\varepsilon_{ij}^2\right) \\
		&=\frac{1}{ab^2}\sum_{i=1}^aE(\varepsilon_{ij}^2) \\
		&=\frac{\sigma^2}{ab}\qedhere
	\end{align*}
\end{proof}
\subsubsection{SSA的期望}
下求SSA的期望:
\begin{align*}
	E(SSA)
	&=E\left[b\sum_{i=1}^a(\tau_i+\bar{\varepsilon}_{i.}-\bar{\varepsilon}_{..})^2\right] \\
	&=E\left[b\sum_{i=1}^a(\tau_i^2+\bar{\varepsilon}_{i.}^2+\bar{\varepsilon}_{..}^2+2\tau_i\bar{\varepsilon}_{i.}-2\tau_i\bar{\varepsilon}_{..}-2\bar{\varepsilon}_{i.}\bar{\varepsilon}_{..})\right] \\
	&=b\sum_{i=1}^a\tau_i^2+b\sum_{i=1}^aE(\bar{\varepsilon}_{i.}^2)+b\sum_{i=1}^aE(\bar{\varepsilon}_{..}^2)-\frac{2b}{ab^2}\sum_{i=1}^a\sum_{j=1}^bE(\varepsilon_{ij}^2) \\
	&=b\sum_{i=1}^a\tau_i^2+ba\frac{\sigma^2}{b}+ab\frac{\sigma^2}{ab}-\frac{2b}{ab^2}ab\sigma^2 \\
	&=b\sum_{i=1}^a\tau_i^2+(a-1)\sigma^2
\end{align*}
\subsubsection{SSB的期望}
下求SSB的期望:
\begin{align*}
	E(SSB)
	&=E\left[a\sum_{j=1}^b(\beta_j+\bar{\varepsilon}_{.j}-\bar{\varepsilon}_{..})^2\right] \\
	&=E\left[a\sum_{j=1}^b(\beta_j^2+\bar{\varepsilon}_{.j}^2+\bar{\varepsilon}_{..}^2+2\beta_j\bar{\varepsilon}_{.j}-2\beta_j\bar{\varepsilon}_{..}-2\bar{\varepsilon}_{.j}\bar{\varepsilon}_{..})\right] \\
	&=a\sum_{j=1}^b\beta_j^2+a\sum_{j=1}^bE(\bar{\varepsilon}_{.j}^2)+a\sum_{j=1}^bE(\bar{\varepsilon}_{..}^2)-\frac{2a}{a^2b}\sum_{i=1}^a\sum_{j=1}^bE(\varepsilon_{ij}^2) \\
	&=a\sum_{j=1}^b\beta_j^2+ab\frac{\sigma^2}{a}+ab\frac{\sigma^2}{ab}-\frac{2a}{a^2b}ab\sigma^2 \\
	&=a\sum_{j=1}^b\beta_j^2+(b-1)\sigma^2
\end{align*}
\subsubsection{SSe的期望}
下求SSe的期望(把均值展开,利用独立性就可以得到结果):
\begin{align*}
	E(SSe)
	&=E\left[\sum_{i=1}^a\sum_{j=1}^b(\varepsilon_{ij}-\bar{\varepsilon}_{i.}-\bar{\varepsilon}_{.j}+\bar{\varepsilon}_{..})^2\right] \\
	&=E\left[\sum_{i=1}^a\sum_{j=1}^b(\varepsilon_{ij}^2+\bar{\varepsilon}_{i.}^2+\bar{\varepsilon}_{.j}^2+\bar{\varepsilon}_{..}^2
	-2\varepsilon_{ij}\bar{\varepsilon}_{i.}-2\varepsilon_{ij}\bar{\varepsilon}_{.j}+2\varepsilon_{ij}\bar{\varepsilon}_{..}
	+2\bar{\varepsilon}_{i.}\bar{\varepsilon}_{.j}-2\bar{\varepsilon}_{i.}\bar{\varepsilon}_{..}-2\bar{\varepsilon}_{.j}\bar{\varepsilon}_{..})\right] \\
	&=(a-1)(b-1)\sigma^2
\end{align*}
\subsubsection{构建统计量}
称$\frac{SSA}{a-1}$为因子A的均方和,记为MSA;称$\frac{SSB}{b-1}$为因子B的均方和,记为MSB;称$\frac{SSe}{(a-1)(b-1)}$为误差均方和,记为MSe。\par
由前述,MSe是$\sigma^2$的无偏估计,而当零假设成立时,MSA和MSB也是$\sigma^2$的一个无偏估计。如果MSA、MSB与MSe比值很大,即MSA、MSB比MSe大很多($b\sum\limits_{i=1}^a\tau_i^2,\;a\sum\limits_{j=1}^b\beta_j^2$很大),我们就有理由怀疑零假设。。由此构建统计量:
\begin{gather*}
	F_A=\frac{MSA}{MSe}=\frac{\frac{SSA}{a-1}}{\frac{SSe}{(a-1)(b-1)}} \\
	F_B=\frac{MSB}{MSe}=\frac{\frac{SSB}{b-1}}{\frac{SSe}{(a-1)(b-1)}} 	
\end{gather*}
在这些统计量的情况下,$H_{01},\;H_{02}$的拒绝域是右向单尾的。下求统计量的分布。

\subsection{统计量的分布}
上述统计量服从如下分布:
\begin{gather*}
	F_A\sim F(a-1,\;(a-1)(b-1)) \\
	F_B\sim F(b-1,\;(a-1)(b-1))
\end{gather*}
所以$H_{01},\;H_{02}$在显著性水平为$\alpha$时的拒绝域为:
\begin{gather*}
	F_A>F_{1-\alpha}(a-1,\;(a-1)(b-1)) \\
	F_B>F_{1-\alpha}(b-1,\;(a-1)(b-1))
\end{gather*}

\subsection{方差分析表}
\begin{table}[H]
	\centering
	\begin{tabularx}{\textwidth}
		{>{\centering\arraybackslash}c|*{5}{>{\centering\arraybackslash}X}}
		\toprule
		来源   &平方和&自由度&均方和             &F值  \\ 
		\midrule
		因子A&SSA&$f_A=a-1$ &$\frac{SSA}{a-1}$ &$F=\frac{MSA}{MSe}$\\
		因子B&SSB&$f_B=b-1$ &$\frac{SSB}{b-1}$ &$F=\frac{MSB}{MSe}$\\
		误差   &SSe  &$f_e=(a-1)(b-1)$ &$\frac{SSe}{(a-1)(b-1)}$ & \\
		总     &SST  &$f_T=ab-1$ &                  & \\
		\bottomrule
	\end{tabularx}
	\caption{无重复可加效应下两因子试验方差分析表}
\end{table}
平方和公式可按下列公式计算:
\begin{equation*}
	\begin{cases}
		SST=\sum\limits_{i=1}^a\sum\limits_{j=1}^by_{ij}^2-\frac{y_{..}^2}{ab} \\
		SSA=\sum\limits_{i=1}^a\frac{y_{i.}^2}{b}-\frac{y_{..}^2}{ab} \\
		SSB=\sum\limits_{j=1}^b\frac{y_{.j}^2}{a}-\frac{y_{..}^2}{ab} \\
		SSe=SST-SSA-SSB
	\end{cases}
\end{equation*}

\subsection{参数估计}
可加效应下的两因子方差分析有四类参数:$\mu$,诸$\tau_i$,诸$\beta_j$和$\sigma^2$。下讨论这四类参数的点估计问题。
\subsubsection{点估计}
参数的点估计如下:
\begin{gather*}
	\hat{\sigma^2}=MSe \\
	\hat{\mu}=\bar{y}_{..} \\
	\hat{\tau}_i=\bar{y}_{i.}-\bar{y}_{..},\;i=1,2,\dots,a \\
	\hat{\beta}_j=\bar{y}_{.j}-\bar{y}_{..},\;j=1,2,\dots,b
\end{gather*}
其中$\hat{\mu},\;\hat{\tau}_i,\;\hat{\beta}_j$是使用最小二乘估计得到的。
\begin{proof}
	分别用$\hat{\mu},\;\hat{\tau}_i,\;\hat{\beta}_j$表示$\mu$,诸$\tau_i$和诸$\beta_j$的估计,用$\hat{y}_{ij}=\hat{\mu}+\hat{\tau}_i+\hat{\beta}_j$表示$y_{ij}$的估计,$i=1,2,\dots,a,\;j=1,2,\dots,b$。损失函数为:
	\begin{equation*}
		L=\sum_{i=1}^a\sum_{j=1}^b(y_{ij}-\hat{y}_{ij})^2=\sum_{i=1}^a\sum_{j=1}^b(y_{ij}-\hat{\mu}-\hat{\tau}_i-\hat{\beta}_j)^2
	\end{equation*}
	最小二乘解需要满足:
	\begin{equation*}
		\begin{cases}
			\vspace{2ex}
			\dfrac{\partial L}{\partial\hat{\mu}}=0, \\
			\vspace{2ex}
			\dfrac{\partial L}{\partial\hat{\tau}_i}=0,\quad 
			i=1,2,\dots,a \\
			\vspace{2ex}
			\dfrac{\partial L}{\partial\hat{\beta}_j}=0,\quad
			j=1,2,\dots,b \\
			\vspace{2ex}
			\sum\limits_{i=1}^a\hat{\tau}_i=0 \\
			\vspace{2ex}
			\sum\limits_{j=1}^b\hat{\beta}_j=0
		\end{cases}
	\end{equation*}
	上式即可解出结果。
\end{proof}

\subsection{多重比较问题}
如果此时某因子显著,则需要对它的各水平均值采用Duncan多重比较法去判断哪些水平之间存在显著差异。此时的水平均值即为在另一因子各水平下的均值。

\subsection{等重复试验情形}
\subsubsection{等重复试验下的方差分析表}
\begin{table}[H]
	\centering
	\begin{tabularx}{\textwidth}
		{>{\centering\arraybackslash}c|*{5}{>{\centering\arraybackslash}X}}
		\toprule
		来源   &平方和&自由度&均方和             &F值  \\ 
		\midrule
		因子A&SSA&$f_A=a-1$ &$\frac{SSA}{a-1}$ &$F=\frac{MSA}{MSe}$\\
		因子B&SSB&$f_B=b-1$ &$\frac{SSB}{b-1}$ &$F=\frac{MSB}{MSe}$\\
		误差   &SSe  &$f_e=abm-a-b+1$ &$\frac{SSe}{f_e}$ & \\
		总     &SST  &$f_T=abm-1$ &                  & \\
		\bottomrule
	\end{tabularx}
	\caption{可加效应下两因子等重复试验方差分析表}
\end{table}
其中$m$为重复实验次数。平方和公式可按下列公式计算:
\begin{equation*}
	\begin{cases}
		SST=\sum\limits_{i=1}^a\sum\limits_{j=1}^b\sum\limits_{k=1}^my_{ijk}^2-\frac{y_{...}^2}{abm} \\
		SSA=\sum\limits_{i=1}^a\frac{y_{i..}^2}{bm}-\frac{y_{...}^2}{abm} \\
		SSB=\sum\limits_{j=1}^b\frac{y_{.j.}^2}{am}-\frac{y_{..}^2}{abm} \\
		SSe=SST-SSA-SSB
	\end{cases}
\end{equation*}
其中:
\begin{gather*}
	y_{...}=\sum_{i=1}^a\sum_{j=1}^b\sum_{k=1}^my_{ijk} \\
	y_{i..}=\sum_{j=1}^b\sum_{k=1}^my_{ijk},\quad i=1,2,\dots,a \\
	y_{.j.}=\sum_{i=1}^a\sum_{k=1}^my_{ijk},\quad j=1,2,\dots,b
\end{gather*}
\subsubsection{参数估计}
参数的点估计如下:
\begin{gather*}
	\hat{\mu}=\bar{y}_{...} \\
	\hat{\tau}_i=\bar{y}_{i..}-\bar{y}_{...},\;i=1,2,\dots,a \\
	\hat{\beta}_j=\bar{y}_{.j.}-\bar{y}_{...},\;j=1,2,\dots,b
\end{gather*}
\subsubsection{多重比较问题}
此时的Duncan多重比较过程与无重复试验的情况完全一样,只是需要注意:
\begin{gather*}
	A:R_p>r_{1-\alpha}(p,f)\sqrt{\frac{MSe}{bm}} \\
	B:R_p>r_{1-\alpha}(p,f)\sqrt{\frac{MSe}{am}}
\end{gather*}














