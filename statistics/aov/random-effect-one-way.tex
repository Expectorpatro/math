\section{随机效应下的单因子方差分析}
\begin{table}[ht]
	\centering
	\begin{tabularx}{\textwidth}
		{>{\centering\arraybackslash}c|*{4}{>{\centering\arraybackslash}X}}
		\hline
		水平   & \multicolumn{4}{c}{观测值} \\ 
		\hline
		$A_1$    & $y_{11}$ & $y_{12}$  & $\cdots$  & $y_{1n_1}$ \\
		$A_2$    & $y_{21}$ & $y_{22}$  & $\cdots$  & $y_{2n_2}$ \\
		$\vdots$ & $\vdots$ & $\vdots$  &           & $\vdots$   \\
		$A_a$    & $y_{a1}$ & $y_{a2}$  & $\cdots$  & $y_{an_a}$ 
		\\
		\hline
	\end{tabularx}
	\caption{随机效应下的单因子试验数据}
\end{table}
其中$y_{ij}$表示在第$i$个水平$A_i$下第$j$次重复试验的观察值。记$n=\sum\limits_{i=1}^{a}n_i$。

\subsection{统计模型}
随机效应下的单因子方差分析统计模型为:
\begin{equation*}\label{model:random-effect-one-way-anova}
	\begin{cases}
		y_{ij}=\mu+\tau_i+\varepsilon_{ij} \\
		\text{诸}\tau_i\quad\mathrm{i.i.d.~}N(0,\sigma_\tau^2) \\
		\text{诸}\varepsilon_{ij}\quad\mathrm{i.i.d.~}N(0,\sigma^2) \\
		\text{诸}\varepsilon_{ij}\text{、诸}\tau_i\text{相互独立}
	\end{cases}
	\qquad i=1,2,\cdots,a,\;j=1,2,\cdots,n_i
\end{equation*}
称$\mu$为一般平均(这里表示因子对数据的一般影响),$\tau_i$为因子A的第$i$个水品的随机效应。

\subsection{统计假设}
随机效应下单因素方差分析检验的统计假设为:
\begin{equation*}
	\begin{cases}
		H_0:\sigma_\tau^2=0 \\
		H_1:\sigma_\tau^2>0
	\end{cases}
\end{equation*}
需要注意的是,此时拒绝零假设意味着因子的全部水平(不论是否参与过试验)之间有显著差异。

\subsection{方差分析}
\subsubsection{偏差平方和的分解}
由于与固定效应情况下三个平方和的定义完全相同,所以随机效应下偏差平方和分解公式与之前一模一样。
\begin{gather*}
	SST=SSA+SSe \\
	SSA=\sum_{i=1}^an_i(\bar{y}_{i.}-\bar{y}_{..})^2 \\
	SSe=\sum_{i=1}^a\sum_{j=1}^{n_i}(y_{ij}-\bar{y}_{i.})^2=\sum_{i=1}^a\sum_{j=1}^{n_i}(\varepsilon_{ij}-\bar{\varepsilon}_{i.})^2
\end{gather*}
\subsubsection{各平方和的期望}
先计算SSA与SSe的期望。
\begin{equation*}
	E(SSe)=(n-a)\sigma^2,\;E(SSA)=\left(n-\frac{\sum\limits_{i=1}^an_i^2}{n}\right)\sigma_\tau^2+(a-1)\sigma^2
\end{equation*}
\begin{proof}
	(1)因为数据结构的形式完全一样,所以SSe的期望与固定效应情形下的一模一样。\par
	(2)注意到$\tau_i$的形式发生变化,$E(SSA)$需要重新求解。
	\begin{align*}
		\bar{y}_{i.}-\bar{y}_{..}
		&=\mu+\tau_i+\bar{\varepsilon}_{i.}-\frac{1}{n}\sum_{k=1}^a\sum_{j=1}^{n_k}(\mu+\tau_k+\varepsilon_{kj}) \\
		&=\mu+\tau_i+\bar{\varepsilon}_{i.}-\frac{1}{n}\left(n\mu+\sum_{k=1}^an_k\tau_k+\varepsilon_{..}\right) \\
		&=\tau_i-\frac{\sum\limits_{k=1}^an_k\tau_k}{n}+\bar{\varepsilon}_{i.}-\bar{\varepsilon}_{..}
	\end{align*}
	\begin{align*}
		E(SSA)
		&=E\left[\sum_{i=1}^an_i\left(\tau_i-\frac{\sum\limits_{k=1}^an_k\tau_k}{n}+\bar{\varepsilon}_{i.}-\bar{\varepsilon}_{..}\right)^2\right] \\
		&=E\left\{\sum_{i=1}^an_i\left[\tau_i^2+\left(\frac{\sum\limits_{k=1}^an_k\tau_k}{n}\right)^2+\bar{\varepsilon}_{i.}^2+\bar{\varepsilon}_{..}^2-2\tau_i\frac{\sum\limits_{k=1}^an_k\tau_k}{n}+2\tau_i\bar{\varepsilon}_{i.}-2\tau_i\bar{\varepsilon}_{..}-2\bar{\varepsilon}_{i.}\bar{\varepsilon}_{..}\right]\right\} \\
		&=\sum_{i=1}^an_iE(\tau_i^2)+\sum_{i=1}^an_iE(\bar{\varepsilon}_{i.}^2)+\sum_{i=1}^an_iE(\bar{\varepsilon}_{..}^2)-2\sum_{i=1}^an_iE(\bar{\varepsilon}_{i.}\bar{\varepsilon}_{..})-\frac{\sum\limits_{i=1}^an_i^2E(\tau_i^2)}{n} \\
		&=\sum_{i=1}^an_i\sigma_\tau^2+\sum_{i=1}^an_i\frac{\sigma^2}{n_i}+\sum_{i=1}^an_i\frac{\sigma^2}{n}-2\sum_{i=1}^an_i\frac{\sigma^2}{n}-\frac{\sum\limits_{i=1}^an_i^2\sigma_\tau^2}{n} \\
		&=\left(n-\frac{\sum_{i=1}^an_i^2}{n}\right)\sigma_\tau^2+(a-1)\sigma^2\qedhere
	\end{align*}
\end{proof}
\subsubsection{统计量及其分布}
称$\frac{SSA}{a-1}$为因子A的均方和,记为MSA;称$\frac{SSe}{n-a}$为误差均方和,记为MSe。\par
由前述,MSe是$\sigma^2$的无偏估计,而当零假设成立时,MSA也是$\sigma^2$的一个无偏估计。如果二者比值很大,即MSA比MSe大很多($\left(\frac{n^2-\sum_{i=1}^an_i^2}{n(a-1)}\right)\sigma_\tau^2$很大),我们就有理由怀疑零假设。由此构建统计量:
\begin{equation*}
	F=\frac{MSA}{MSe}=\frac{\frac{SSA}{a-1}}{\frac{SSe}{n-a}}
\end{equation*}
在该统计量的情况下,$H_0$的拒绝域是右向单尾的。\par
由于在假设$H_0$成立时,随机效应模型与固定效应模型的观察值$y_{ij}$的数据结构的形式完全一样,所以在随机效应模型中仍然有:
\begin{equation*}
	F\sim F(a-1,\;n-a)
\end{equation*}
\subsubsection{拒绝域}
综上所述,显著性水平为$\alpha$时的拒绝域为:
\begin{equation*}
	F>F_{1-\alpha}(a-1,\;n-a)
\end{equation*}

\subsection{方差分析表}
\begin{table}[H]
	\centering
	\begin{tabularx}{\textwidth}
		{>{\centering\arraybackslash}c|*{5}{>{\centering\arraybackslash}X}}
		\toprule
		来源   &平方和&自由度&均方和             &F值  \\ 
		\midrule
		因子A&SSA&$f_A=a-1$ &$\frac{SSA}{a-1}$ &$F=\frac{MSA}{MSe}$\\
		误差   &SSe  &$f_e=n-a$ &$\frac{SSe}{n-a}$ & \\
		总     &SST  &$f_T=n-1$ &                  & \\
		\bottomrule
	\end{tabularx}
	\caption{随机效应下单因子试验方差分析表}
\end{table}
平方和公式可按下列公式计算:
\begin{equation*}
	\begin{cases}
		SST=\sum\limits_{i=1}^a\sum\limits_{j=1}^{n_i}y_{ij}^2-\frac{y_{..}^2}{n} \\
		SSA=\sum\limits_{i=1}^a\frac{y_{i.}^2}{n_i}-\frac{y_{..}^2}{n} \\
		SSe=SST-SSA
	\end{cases}
\end{equation*}

\subsection{参数估计}
我们此时关心方差分量的估计。
\subsubsection{点估计}
由SSA、SSe期望的计算,可给出各方差分量的无偏点估计如下:
\begin{gather*}
	\hat{\sigma^2}=MSe \\
	\hat{\sigma^2}_\tau=\frac{n(a-1)(MSA-MSe)}{n^2-\sum\limits_{i=1}^an_i^2}
\end{gather*}
这里需要注意的是,$\hat{\sigma^2}_\tau$有可能小于$0$,这是由估计方法决定的。
\subsubsection{区间估计}
考虑$\sigma^2$的区间估计。因为:
\begin{equation*}
	\frac{SSe}{\sigma^2}\sim\chi^2(n-a)
\end{equation*}
所以显著性水平为$\alpha$时,$\sigma^2$的置信区间为:
\begin{equation*}
	\left(\frac{SSe}{\chi_{\alpha/2}^2(n-a)},\;\frac{SSe}{\chi_{1-\alpha/2}^2(n-a)}\right)
\end{equation*}
