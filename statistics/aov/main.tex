\chapter{方差分析}

这章讨论方差分析问题。方差分析的目的是判断因子的水平不同是否会对试验结果造成显著差异。
\subsubsection{按因子数对方差分析进行分类}
可按试验的因子数对方差分析进行分类:
\begin{enumerate}
	\item 单因子方差分析:讨论单个因子的水平变动对试验结果造成的影响。
	\item 多因子方差分析:讨论多个因子的水平变动对试验结果的影响,也讨论多个因子水平的组合对试验结果造成的额外影响。
\end{enumerate}
\subsubsection{按因子水平的选取方式对方差分析进行分类}
可按因子水平的选取方式对方差分析进行分类:
\begin{enumerate}
	\item 固定效应模型:试验中选择的因子的水平是在试验前由试验人按主观意图指定好的,只希望得到适用于这些水平的结论。在多因子方差分析中可进一步分类:
	\begin{itemize}
		\item 可加效应模型:仅讨论各因子对试验结果影响的大小。
		\item 交互效应模型:不仅讨论各因子对试验结果影响的大小,也讨论因子的水平组合对试验结果造成的额外影响。
	\end{itemize}
	\item 随机效应模型:试验中选择的因子的水平是从全部可能的水平中随机选择的一个样本,希望得到适用于全部水平(无论是否参与试验)的结论。
	\item 混合模型:仅存在于多因子方差分析,部分因子的水平在试验前由试验人按主观意图指定,另一部分因子的水平是随机选定的。
\end{enumerate}

\section{多重假设检验}
当方差分析拒绝零假设,认为因子的不同水平对实验结果的影响有显著差异时,随之而来的问题就是到底哪几个因子之间是有显著差异的,此时就需要进行比较。但是如果对于每一对水平都使用通常的$t$检验的话,将大大提高整个检验问题犯第一类错误的概率(见\cref{reason for multi-comparison})。本节先介绍一般的对比,再介绍等重复情况下的Duncan多重比较法与一般情形下的Scheffe多重比较法。

\subsection{对比}
\subsubsection{对比的定义}
\begin{definition}
	对比是指因子诸效应的一个线性组合:
	\begin{equation*}
		\begin{cases}
			c=\sum\limits_{i=1}^ac_i\tau_i,\\
			s.t.\quad\sum\limits_{i=1}^ac_i=0
		\end{cases}
	\end{equation*}
	因为$\mu_i=\mu+\tau_i$,所以对比也可表示为:
	\begin{equation*}
		\begin{cases}
			c=\sum\limits_{i=1}^ac_i\mu_i,\\
			s.t.\quad\sum\limits_{i=1}^ac_i=0
		\end{cases}
	\end{equation*}
	全体对比构成一个$a-1$维线性空间:
	\begin{equation*}
		\{\mathbf{c}=(c_1,c_2,\dots,c_a)':\sum_{i=1}^ac_i=0\}
	\end{equation*}
\end{definition}
\subsubsection{目的}
我们此时希望检验假设:
\begin{equation*}
	H_0:\sum_{i=1}^ac_i\mu_i=0,\quad s.t.\sum_{i=1}^ac_i=0
\end{equation*}
\subsubsection{假设的检验方法}
由固定效应下的单因子方差分析的统计模型\cref{model:fixed-effect-one-way-anova},可知:
\begin{equation*}
	\sum_{i=1}^ac_i\bar{y}_{i.}\sim N(\sum_{i=1}^ac_i\mu_i,\;\sum_{i=1}^a\frac{c_i^2\sigma^2}{n_i})
\end{equation*}
定义对比的平方和为:
\begin{equation*}
	SSc=\frac{\left(\sum\limits_{i=1}^ac_i\bar{y}_{i.}\right)^2}{\sum\limits_{i=1}^a\dfrac{c_i^2}{n_i}}
\end{equation*}
当$H_0$成立时,$\dfrac{SSc}{\sigma^2}$是一个标准正态变量的平方,也就是说它服从$\chi^2(1)$。\info{记得证明独立}
所以,当$H_0$成立的时候,统计量
\begin{equation*}
	F=\frac{SSc}{MSe}\sim\text{F}(1,\;n-a)
\end{equation*}
MSe是$\sigma^2$的无偏估计,而当$H_0$成立时,SSc也是$\sigma^2$的无偏估计。如果F值很大,则有理由怀疑零假设(从$\left(\sum\limits_{i=1}^ac_i\bar{y}_{i.}\right)^2$较大这一点怀疑它的期望不为$0$)。所以$H_0$的拒绝域是右向单尾的,显著性水平为$\alpha$时的拒绝域为:
\begin{equation*}
	F=\frac{SSc}{MSe}>F_{1-\alpha}(1,\;n-a)
\end{equation*}

\subsection{正交对比}
仅讨论等重复情况下对比系数向量标准化(即模长为$1$)的正交对比。
\subsubsection{正交对比的定义}
对比有一种特殊情况,即正交对比:
\begin{definition}
	$c=\sum\limits_{i=1}^ac_i\mu_i$和$d=\sum\limits_{i=1}^ad_i\mu_i$是两个对比。若:
	\begin{equation*}
		\sum_{i=1}^ac_id_i=0
	\end{equation*}
	则称这两个对比为正交对比。
\end{definition}
\subsubsection{目的}
由对比空间的维数可知,线性无关的正交对比组最多有$a-1$个对比。而正交对比即是为了检验任意$a-1$个对比的值是否为$0$:
\begin{gather*}
	c^1=\sum_{i=1}^ac_{1,i}\mu_i \\
	c^2=\sum_{i=1}^ac_{2,i}\mu_i \\
	\cdots\cdots \\
	c^{(a-1)}=\sum_{i=1}^ac_{a-1,i}\mu_i
\end{gather*}
\subsubsection{假设的检验方法}
只需注意到此时:
\begin{equation*}
	SScj=\frac{\left(\sum\limits_{i=1}^ac_{j,i}\bar{y}_{i.}\right)^2}{\sum\limits_{i=1}^a\dfrac{c_{j,i}^2}{m}}=m\left(\sum\limits_{i=1}^ac_{j,i}\bar{y}_{i.}\right)^2
\end{equation*}
剩余步骤与对比一样。
\subsubsection{正交对比与SSA的关系}
上述对比的一个无偏估计为:
\begin{gather*}
	\hat{c}^1=\sum_{i=1}^ac_{1,i}\bar{y}_{i.} \\
	\hat{c}^2=\sum_{i=1}^ac_{2,i}\bar{y}_{i.} \\
	\cdots\cdots \\
	\hat{c}^{(a-1)}=\sum_{i=1}^ac_{a-1,i}\bar{y}_{i.}
\end{gather*}
令:
\begin{equation*}
	\hat{c}^{a}=\sum_{i=1}^a\frac{1}{\sqrt{a}}\bar{y}_{i.}
\end{equation*}
需要注意这并不是一个对比。\par
注意到:
\begin{equation*}
	(\hat{c}^1,\hat{c}^2,\dots,\hat{c}^a)'=A(\bar{y}_{1.},\bar{y}_{2.},\dots,\bar{y}_{a.})'
\end{equation*}
这之中的矩阵$A$是一个正交矩阵。所以:
\begin{equation*}
	(\hat{c}^1)^2+(\hat{c}^2)^2+\cdots+(\hat{c}^a)^2=\sum_{i=1}^a\bar{y}_{i.}^2
\end{equation*}
于是:
\begin{align*}
	(\hat{c}^1)^2+(\hat{c}^2)^2+\cdots+(\hat{c}^{a-1})^2
	&=\sum_{i=1}^a\bar{y}_{i.}^2-\frac{1}{a}\left(\sum\limits_{i=1}^a\bar{y}_{i.}\right)^2 \\
	&=\sum_{i=1}^a\left(\frac{y_{i.}}{m}\right)^2-\frac{1}{a}\left(\frac{\sum_{i=1}^am\bar{y}_{i.}}{m}\right)^2 \\
	&=\sum_{i=1}^a\frac{y_{i.}^2}{m^2}-\frac{y_{..}^2}{am^2} \\
	&=\frac{1}{m}SSA
\end{align*}
再由:
\begin{equation*}
	SScj=\frac{\left(\sum\limits_{i=1}^ac_{j,i}\bar{y}_{i.}\right)^2}{\sum\limits_{i=1}^a\dfrac{c_{j,i}^2}{m}}=\frac{(\hat{c}^j)^2}{\frac{1}{m}}=m(\hat{c}^j)^2
\end{equation*}
所以:
\begin{equation*}
	SSc1+SSc2+\cdots+SSc(a-1)=SSA
\end{equation*}

\subsection{Duncan多重比较法}
在很多问题中,我们往往不知道要如何构造适当的对比,也有可能检验$a-1$个以上的比较,此时对比的相关方法就无法使用了。接下来介绍这种情况下的一种解决方案,即Duncan多重比较法,它只适用于等重复情况。
\subsubsection{目的}
检验$H_0:\mu_i=\mu_j,\;\forall\;i\ne j$。
\subsubsection{$p$级极差的定义}
\begin{definition}
	将$a$个水平下观察值的平均值$\bar{y}_{1.},\bar{y}_{2.},\dots,\bar{y}_{a.}$从小到大排序。如果其中任意两个数在排序后中间还有$p-2,\;p\geqslant2$个数,那么这两个数的差称为$p$级极差,记为$R_p$。
\end{definition}
\subsubsection{统计量及其分布}
设$f$为SSe的自由度,$m$为重复次数,则Duncan多重比较法的统计量为:
\begin{equation*}
	r(p,f)=\frac{R_p}{\sqrt{\dfrac{MSe}{m}}}
\end{equation*}
在$\mu_1=\mu_2=\cdots=\mu_a$即$\tau_1=\tau_2=\cdots=\tau_a$的情况下,$r(p,f)$与$\mu,\;\sigma^2$无关。
\begin{proof}
	将$r(p,f)$分子分母同除$\sigma$可得:
	\begin{equation*}
		r(p.f)=\frac{\dfrac{R_p}{\sigma/\sqrt{m}}}{\sqrt{\dfrac{MSe}{\sigma^2}}}
	\end{equation*}
	注意到分母:
	\begin{equation*}
		\frac{MSe}{\sigma^2}=\frac{1}{f}\frac{SSe}{\sigma^2}
	\end{equation*}
	是一个服从$\chi^2(f)$分布变量的$\dfrac{1}{f}$倍,其分布与$\mu,\;\sigma^2$无关。\par
	在零假设成立的情况下:
	\begin{equation*}
		\frac{\bar{y}_{i.}-\mu}{\sigma/\sqrt{m}}\sim N(0,\;1)
	\end{equation*}
	这个分布与$\mu,\;\sigma^2$无关,所以分子:
	\begin{equation*}
		\frac{R_p}{\sigma/\sqrt{m}}=\max(\frac{\bar{y}_{1.}-\mu}{\sigma/\sqrt{m}},\dots,\frac{\bar{y}_{a.}-\mu}{\sigma/\sqrt{m}})-\min(\frac{\bar{y}_{1.}-\mu}{\sigma/\sqrt{m}},\dots,\frac{\bar{y}_{a.}-\mu}{\sigma/\sqrt{m}})
	\end{equation*}
	也与$\mu,\;\sigma^2$无关。\par
	综上,此时$r(p,f)$与$\mu,\;\sigma^2$无关。
\end{proof}
\subsubsection{检验原理}
当$\mu_i=\mu_j$不成立时,对应的$R_p$会较大。因此当$r(p,f)$较大时,有理由怀疑零假设。所以$H_0$的拒绝域是右向单尾的,显著性水平为$\alpha$时的拒绝域为:
\begin{equation*}
	r(p,f)>r_{1-\alpha}(p,f)
\end{equation*}
即:
\begin{equation*}
	R_p>r_{1-\alpha}(p,f)\sqrt{\frac{MSe}{m}}
\end{equation*}
\subsubsection{$r(p,f)$分布的Monte Carlo模拟}
\begin{algorithm}
	\caption{Duncan 多重比较法统计量分布的蒙特卡洛模拟}
	\begin{algorithmic}[1]
		\State \textbf{Input:} $m$, $a$, $p$, $f$, $N$ \Comment{组内重复次数、组数、极差的级数、SSe的自由度、模拟次数}
		\State \textbf{Output:} $r(p,f)$的模拟分布
		
		\State 初始化模拟值存储向量:$List\gets\emptyset$
		\For{$i \gets 1$ to $N$}
		\State 生成$a$个随机数$x_i\sim N(0,1),\;i=1,2,\dots,a$
		\State 计算$\frac{R_p}{\sigma/\sqrt{m}}$:
		\begin{equation*}
			\frac{R_p}{\sigma/\sqrt{m}}=\max\limits_{i=1,2,\dots,a}\{x_i\}-\min\limits_{i=1,2,\dots,a}\{x_i\}
		\end{equation*}
		\State 从$\chi^2(f)$中产生一个样本记为$\chi^2$
		\State 计算$r(p,f)$:
		\begin{equation*}
			r(p,f)=\frac{\dfrac{R_p}{\sigma/\sqrt{m}}}{\sqrt{\chi^2/f}}
		\end{equation*}
		\State 将 $r(p,f)$加入$List$
		\EndFor
		\State 返回$List$
	\end{algorithmic}
\end{algorithm}
\subsubsection{检验步骤}
将$\bar{y}_{1.},\bar{y}_{2.},\dots,\bar{y}_{a.}$从小到大排序为$\bar{y}^1,\bar{y}^2,\dots,\bar{y}^a$。令:
\begin{equation*}
	r_{1-\alpha}(p,f)\sqrt{\frac{MSe}{m}}=R_p^*
\end{equation*}
按以下顺序进行比较:
\begin{gather*}
	\bar{y}^a-\bar{y}^1\text{与$R_a^*$进行比较} \\
	\bar{y}^a-\bar{y}^2\text{与$R_{a-1}^*$进行比较} \\
	\cdots\cdots \\
	\bar{y}^a-\bar{y}^{a-1}\text{与$R_2^*$进行比较} \\
	\bar{y}^{a-1}-\bar{y}^1\text{与$R_{a-1}^*$进行比较} \\
	\bar{y}^{a-1}-\bar{y}^2\text{与$R_{a-2}^*$进行比较} \\
	\cdots\cdots
\end{gather*}
直到全部$\binom{a}{2}$对水平均值比较完为止。

\subsection{Scheffe多重比较法}
Scheffe证明了,在固定效应下的单因子方差分析统计模型下(即\cref{model:fixed-effect-one-way-anova}),对显著性水平$\alpha$,一切对比$c$的值同时满足不等式:
\begin{equation*}
	|\hat{c}-c|=\left|\sum_{i=1}^ac_i\bar{y}_{i.}-c\right|\leqslant\sqrt{(a-1)F_{1-\alpha}(a-1,\;n-a)MSe\sum\limits_{i=1}^a\dfrac{c_i^2}{n_i}}
\end{equation*}
的概率等于$1-\alpha$。也就是说,$H_0:\sum\limits_{i=1}^ac_i\mu_i=0$的拒绝域都是:
\begin{equation*}
	|\hat{c}|>\sqrt{(a-1)F_{1-\alpha}(a-1,\;n-a)MSe\sum\limits_{i=1}^a\dfrac{c_i^2}{n_i}}
\end{equation*}
\section{随机效应下的单因子方差分析}
\begin{table}[ht]
	\centering
	\begin{tabularx}{\textwidth}
		{>{\centering\arraybackslash}c|*{4}{>{\centering\arraybackslash}X}}
		\hline
		水平   & \multicolumn{4}{c}{观测值} \\ 
		\hline
		$A_1$    & $y_{11}$ & $y_{12}$  & $\cdots$  & $y_{1n_1}$ \\
		$A_2$    & $y_{21}$ & $y_{22}$  & $\cdots$  & $y_{2n_2}$ \\
		$\vdots$ & $\vdots$ & $\vdots$  &           & $\vdots$   \\
		$A_a$    & $y_{a1}$ & $y_{a2}$  & $\cdots$  & $y_{an_a}$ 
		\\
		\hline
	\end{tabularx}
	\caption{随机效应下的单因子试验数据}
\end{table}
其中$y_{ij}$表示在第$i$个水平$A_i$下第$j$次重复试验的观察值。记$n=\sum\limits_{i=1}^{a}n_i$。

\subsection{统计模型}
随机效应下的单因子方差分析统计模型为:
\begin{equation*}\label{model:random-effect-one-way-anova}
	\begin{cases}
		y_{ij}=\mu+\tau_i+\varepsilon_{ij} \\
		\text{诸}\tau_i\quad\mathrm{i.i.d.~}N(0,\sigma_\tau^2) \\
		\text{诸}\varepsilon_{ij}\quad\mathrm{i.i.d.~}N(0,\sigma^2) \\
		\text{诸}\varepsilon_{ij}\text{、诸}\tau_i\text{相互独立}
	\end{cases}
	\qquad i=1,2,\cdots,a,\;j=1,2,\cdots,n_i
\end{equation*}
称$\mu$为一般平均(这里表示因子对数据的一般影响),$\tau_i$为因子A的第$i$个水品的随机效应。

\subsection{统计假设}
随机效应下单因素方差分析检验的统计假设为:
\begin{equation*}
	\begin{cases}
		H_0:\sigma_\tau^2=0 \\
		H_1:\sigma_\tau^2>0
	\end{cases}
\end{equation*}
需要注意的是,此时拒绝零假设意味着因子的全部水平(不论是否参与过试验)之间有显著差异。

\subsection{方差分析}
\subsubsection{偏差平方和的分解}
由于与固定效应情况下三个平方和的定义完全相同,所以随机效应下偏差平方和分解公式与之前一模一样。
\begin{gather*}
	SST=SSA+SSe \\
	SSA=\sum_{i=1}^an_i(\bar{y}_{i.}-\bar{y}_{..})^2 \\
	SSe=\sum_{i=1}^a\sum_{j=1}^{n_i}(y_{ij}-\bar{y}_{i.})^2=\sum_{i=1}^a\sum_{j=1}^{n_i}(\varepsilon_{ij}-\bar{\varepsilon}_{i.})^2
\end{gather*}
\subsubsection{各平方和的期望}
先计算SSA与SSe的期望。
\begin{equation*}
	E(SSe)=(n-a)\sigma^2,\;E(SSA)=\left(n-\frac{\sum\limits_{i=1}^an_i^2}{n}\right)\sigma_\tau^2+(a-1)\sigma^2
\end{equation*}
\begin{proof}
	(1)因为数据结构的形式完全一样,所以SSe的期望与固定效应情形下的一模一样。\par
	(2)注意到$\tau_i$的形式发生变化,$E(SSA)$需要重新求解。
	\begin{align*}
		\bar{y}_{i.}-\bar{y}_{..}
		&=\mu+\tau_i+\bar{\varepsilon}_{i.}-\frac{1}{n}\sum_{k=1}^a\sum_{j=1}^{n_k}(\mu+\tau_k+\varepsilon_{kj}) \\
		&=\mu+\tau_i+\bar{\varepsilon}_{i.}-\frac{1}{n}\left(n\mu+\sum_{k=1}^an_k\tau_k+\varepsilon_{..}\right) \\
		&=\tau_i-\frac{\sum\limits_{k=1}^an_k\tau_k}{n}+\bar{\varepsilon}_{i.}-\bar{\varepsilon}_{..}
	\end{align*}
	\begin{align*}
		E(SSA)
		&=E\left[\sum_{i=1}^an_i\left(\tau_i-\frac{\sum\limits_{k=1}^an_k\tau_k}{n}+\bar{\varepsilon}_{i.}-\bar{\varepsilon}_{..}\right)^2\right] \\
		&=E\left\{\sum_{i=1}^an_i\left[\tau_i^2+\left(\frac{\sum\limits_{k=1}^an_k\tau_k}{n}\right)^2+\bar{\varepsilon}_{i.}^2+\bar{\varepsilon}_{..}^2-2\tau_i\frac{\sum\limits_{k=1}^an_k\tau_k}{n}+2\tau_i\bar{\varepsilon}_{i.}-2\tau_i\bar{\varepsilon}_{..}-2\bar{\varepsilon}_{i.}\bar{\varepsilon}_{..}\right]\right\} \\
		&=\sum_{i=1}^an_iE(\tau_i^2)+\sum_{i=1}^an_iE(\bar{\varepsilon}_{i.}^2)+\sum_{i=1}^an_iE(\bar{\varepsilon}_{..}^2)-2\sum_{i=1}^an_iE(\bar{\varepsilon}_{i.}\bar{\varepsilon}_{..})-\frac{\sum\limits_{i=1}^an_i^2E(\tau_i^2)}{n} \\
		&=\sum_{i=1}^an_i\sigma_\tau^2+\sum_{i=1}^an_i\frac{\sigma^2}{n_i}+\sum_{i=1}^an_i\frac{\sigma^2}{n}-2\sum_{i=1}^an_i\frac{\sigma^2}{n}-\frac{\sum\limits_{i=1}^an_i^2\sigma_\tau^2}{n} \\
		&=\left(n-\frac{\sum_{i=1}^an_i^2}{n}\right)\sigma_\tau^2+(a-1)\sigma^2\qedhere
	\end{align*}
\end{proof}
\subsubsection{统计量及其分布}
称$\frac{SSA}{a-1}$为因子A的均方和,记为MSA;称$\frac{SSe}{n-a}$为误差均方和,记为MSe。\par
由前述,MSe是$\sigma^2$的无偏估计,而当零假设成立时,MSA也是$\sigma^2$的一个无偏估计。如果二者比值很大,即MSA比MSe大很多($\left(\frac{n^2-\sum_{i=1}^an_i^2}{n(a-1)}\right)\sigma_\tau^2$很大),我们就有理由怀疑零假设。由此构建统计量:
\begin{equation*}
	F=\frac{MSA}{MSe}=\frac{\frac{SSA}{a-1}}{\frac{SSe}{n-a}}
\end{equation*}
在该统计量的情况下,$H_0$的拒绝域是右向单尾的。\par
由于在假设$H_0$成立时,随机效应模型与固定效应模型的观察值$y_{ij}$的数据结构的形式完全一样,所以在随机效应模型中仍然有:
\begin{equation*}
	F\sim F(a-1,\;n-a)
\end{equation*}
\subsubsection{拒绝域}
综上所述,显著性水平为$\alpha$时的拒绝域为:
\begin{equation*}
	F>F_{1-\alpha}(a-1,\;n-a)
\end{equation*}

\subsection{方差分析表}
\begin{table}[H]
	\centering
	\begin{tabularx}{\textwidth}
		{>{\centering\arraybackslash}c|*{5}{>{\centering\arraybackslash}X}}
		\toprule
		来源   &平方和&自由度&均方和             &F值  \\ 
		\midrule
		因子A&SSA&$f_A=a-1$ &$\frac{SSA}{a-1}$ &$F=\frac{MSA}{MSe}$\\
		误差   &SSe  &$f_e=n-a$ &$\frac{SSe}{n-a}$ & \\
		总     &SST  &$f_T=n-1$ &                  & \\
		\bottomrule
	\end{tabularx}
	\caption{随机效应下单因子试验方差分析表}
\end{table}
平方和公式可按下列公式计算:
\begin{equation*}
	\begin{cases}
		SST=\sum\limits_{i=1}^a\sum\limits_{j=1}^{n_i}y_{ij}^2-\frac{y_{..}^2}{n} \\
		SSA=\sum\limits_{i=1}^a\frac{y_{i.}^2}{n_i}-\frac{y_{..}^2}{n} \\
		SSe=SST-SSA
	\end{cases}
\end{equation*}

\subsection{参数估计}
我们此时关心方差分量的估计。
\subsubsection{点估计}
由SSA、SSe期望的计算,可给出各方差分量的无偏点估计如下:
\begin{gather*}
	\hat{\sigma^2}=MSe \\
	\hat{\sigma^2}_\tau=\frac{n(a-1)(MSA-MSe)}{n^2-\sum\limits_{i=1}^an_i^2}
\end{gather*}
这里需要注意的是,$\hat{\sigma^2}_\tau$有可能小于$0$,这是由估计方法决定的。
\subsubsection{区间估计}
考虑$\sigma^2$的区间估计。因为:
\begin{equation*}
	\frac{SSe}{\sigma^2}\sim\chi^2(n-a)
\end{equation*}
所以显著性水平为$\alpha$时,$\sigma^2$的置信区间为:
\begin{equation*}
	\left(\frac{SSe}{\chi_{\alpha/2}^2(n-a)},\;\frac{SSe}{\chi_{1-\alpha/2}^2(n-a)}\right)
\end{equation*}

\section{可加效应下的两因子方差分析}
可加效应模型一般不必进行重复实验,每个水平组合下只做一次实验就够了。
\begin{table}[H] 
	\centering
	\begin{tabularx}{\textwidth}
		{c|>{\centering\arraybackslash}X>{\centering\arraybackslash}Xc>{\centering\arraybackslash}X}
		\hline
		\diagbox{因子A}{因子B} & $B_1$ & $B_2$ & $\cdots$ & $B_b$ \\ \hline
		$A_1$ & $y_{11}$ & $y_{12}$ & $\cdots$ & $y_{1b}$ \\ 
		$A_2$ & $y_{21}$ & $y_{22}$ & $\cdots$ & $y_{2b}$ \\
		$\vdots$ & $\vdots$ & $\vdots$ & & $\vdots$ \\
		$A_a$ & $y_{a1}$ & $y_{a2}$ & $\cdots$ & $y_{ab}$ \\ 
		\hline
	\end{tabularx}
	\caption{无重复两因子试验数据表}
\end{table}
其中$y_{ij}$表示在因子A的第$i$个水平$A_i$和因子B的第$j$个水平$B_j$下试验的观察值。记$n=ab$。

\subsection{统计模型}
假设一个数据$y$由两部分组成:
\begin{enumerate}
	\item 因子组合$(A_i,B_j)$的影响部分$\mu_{ij}$,随因子水平组合的变化而变化。
	\item 试验的随机误差$\varepsilon$,假设所有随机误差来自同一个正态总体$ N(0,\sigma^2)$。
\end{enumerate}
则统计模型可写作:
\begin{equation*}
	\begin{cases}
		y_{ij}=\mu_{ij}+\varepsilon_{ij} \\
		\text{诸}\varepsilon_{ij}\quad\mathrm{i.i.d.~}N(0,\sigma^2) \\
		i=1,2,\dots,a,\;j=1,2,\dots,b
	\end{cases}
\end{equation*}
还可将统计模型写成意义更清晰的形式,记:
\begin{gather*}
	\mu=\frac{1}{ab}\sum_{i=1}^a\sum_{j=1}^b\mu_{ij} \\
	\bar{\mu}_{i.}=\sum_{j=1}^b\mu_{ij},\quad\tau_i=\bar{\mu}_{i.}-\mu,\quad i=1,2,\dots,a \\
	\bar{\mu}_{.j}=\sum_{i=1}^a\mu_{ij},\quad\beta_j=\bar{\mu}_{.j}-\mu,\quad j=1,2,\dots,b \\
\end{gather*}
称$\mu$为一般平均,表示$ab$个总体的均值的平均值。称$\tau_i$为因子A第$i$个水平$A_i$的主效应。称$\beta_j$为因子B第$j$个水平$B_j$的主效应。那么统计模型即可改写为:
\begin{equation*}\label{model:additive-effect-two-way-anova}
	\begin{cases}
		y_{ij}=\mu+\tau_i+\beta_j+\varepsilon_{ij} \\
		\text{诸}\varepsilon_{ij}\quad\mathrm{i.i.d.~}N(0,\sigma^2) \\
		s.t.\quad\sum\limits_{i=1}^a\tau_i=0,\quad\sum\limits_{j=1}^b\beta_j=0 \\
		i=1,2,\dots,a,\;j=1,2,\dots,b
	\end{cases}
\end{equation*}

\subsection{统计假设}
可加效应下的两因子方差分析需要检验如下两个零假设:
\begin{equation*}
	\begin{cases}
		H_{01}:\tau_1=\tau_2=\cdots=\tau_a=0, \\
		H_{02}:\beta_1=\beta_2=\cdots=\beta_b=0
	\end{cases}
\end{equation*}

\subsection{偏差平方和的分解}
记:
\begin{gather*}
	y_{..}=\sum_{i=1}^a\sum_{j=1}^by_{ij},\quad
	\bar{y}_{..}=\frac{y_{..}}{ab} \\
	y_{i.}=\sum_{j=1}^by_{ij},\quad
	\bar{y}_{i.}=\frac{y_{i.}}{b},\quad i=1,2,\dots,a \\
	y_{.j}=\sum_{i=1}^ay_{ij},\quad
	\bar{y}_{.j}=\frac{y_{.j}}{a},\quad j=1,2,\dots,b 
\end{gather*}
全部数据之间的差异可用下述总偏差平方和表示:
\begin{equation*}
	SST=\sum_{i=1}^a\sum_{j=1}^b(y_{ij}-\bar{y}_{..})^2
\end{equation*}
引起数据$y_{ij}$之间差异的原因有三点:
\begin{enumerate}
	\item 因子A的$a$个水平对试验结果的影响不同。
	\item 因子B的$b$个水平对试验结果的影响不同。
	\item 试验具有误差。
\end{enumerate}
为了区分并比较这三个原因对数据的影响,需要对SST进行分解:
\begin{align*}
	SST
	&=\sum_{i=1}^a\sum_{j=1}^b(y_{ij}-\bar{y}_{..})^2 \\
	&=\sum_{i=1}^a\sum_{j=1}^b\left[(\bar{y}_{i.}-\bar{y}_{..})+(\bar{y}_{.j}-\bar{y}_{..})+(y_{ij}-\bar{y}_{i.}-\bar{y}_{.j}+\bar{y}_{..})\right]^2 \\
	&=\sum_{i=1}^a\sum_{j=1}^b(\bar{y}_{i.}-\bar{y}_{..})^2+\sum_{i=1}^a\sum_{j=1}^b(\bar{y}_{.j}-\bar{y}_{..})^2+\sum_{i=1}^a\sum_{j=1}^b(y_{ij}-\bar{y}_{i.}-\bar{y}_{.j}+\bar{y}_{..})^2 \\
	&\quad+2\sum_{i=1}^a\sum_{j=1}^b(\bar{y}_{i.}-\bar{y}_{..})(\bar{y}_{.j}-\bar{y}_{..}) \\
	&\quad+2\sum_{i=1}^a\sum_{j=1}^b(\bar{y}_{i.}-\bar{y}_{..})(y_{ij}-\bar{y}_{i.}-\bar{y}_{.j}+\bar{y}_{..}) \\
	&\quad+2\sum_{i=1}^a\sum_{j=1}^b(\bar{y}_{.j}-\bar{y}_{..})(y_{ij}-\bar{y}_{i.}-\bar{y}_{.j}+\bar{y}_{..})
\end{align*}
由定义可发现上式后三项为$0$(第三项需要注意交换求和顺序)。
\subsubsection{SSe}
记:
\begin{gather*}
	SSe=\sum_{i=1}^a\sum_{j=1}^b(y_{ij}-\bar{y}_{i.}-\bar{y}_{.j}+\bar{y}_{..})^2 \\
	\varepsilon_{..}=\sum_{i=1}^a\sum_{j=1}^b\varepsilon_{ij},\quad
	\bar{\varepsilon}_{..}=\frac{\varepsilon_{..}}{ab} \\
	\varepsilon_{i.}=\sum_{j=1}^b\varepsilon_{ij},\quad
	\bar{\varepsilon}_{i.}=\frac{1}{b}\sum_{j=1}^b\varepsilon_{ij},\quad i=1,2,\dots,a \\
	\varepsilon_{.j}=\sum_{i=1}^a\varepsilon_{ij},\quad
	\bar{\varepsilon}_{.j}=\frac{1}{a}\sum_{i=1}^a\varepsilon_{ij},\quad j=1,2,\dots,b \\
	\bar{\tau}=\frac{1}{a}\sum_{i=1}^a\tau_i=0,\quad
	\bar{\beta}=\frac{1}{b}\sum_{j=1}^b\beta_j=0
\end{gather*}
则有:
\begin{align*}
	SSe
	&=\sum_{i=1}^a\sum_{j=1}^b(y_{ij}-\bar{y}_{i.}-\bar{y}_{.j}+\bar{y}_{..})^2 \\
	&=\sum_{i=1}^a\sum_{j=1}^b(\mu+\tau_i+\beta_j+\varepsilon_{ij}-\mu-\tau_i-\bar{\beta}-\bar{\varepsilon}_{i.}-\mu-\bar{\tau}-\beta_j-\bar{\varepsilon}_{.j}+\mu+\bar{\tau}+\bar{\beta}+\bar{\varepsilon}_{..})^2 \\
	&=\sum_{i=1}^a\sum_{j=1}^b(\varepsilon_{ij}-\bar{\varepsilon}_{i.}-\bar{\varepsilon}_{.j}+\bar{\varepsilon}_{..})^2
\end{align*}
因为该项完全是由误差引起的,所以称之为误差平方和。
\subsubsection{SSA}
记:
\begin{equation*}
	SSA=\sum_{i=1}^a\sum_{j=1}^b(\bar{y}_{i.}-\bar{y}_{..})^2=b\sum_{i=1}^a(\bar{y}_{i.}-\bar{y}_{..})^2
\end{equation*}
可以发现:
\begin{align*}
	SSA&=b\sum_{i=1}^a(\bar{y}_{i.}-\bar{y}_{..})^2 \\
	&=b\sum_{i=1}^a(\mu+\tau_i+\bar{\beta}+\bar{\varepsilon}_{i.}-\mu-\bar{\tau}-\bar{\beta}-\bar{\varepsilon}_{..})^2 \\
	&=b\sum_{i=1}^a(\tau_i+\bar{\varepsilon}_{i.}-\bar{\varepsilon}_{..})^2
\end{align*}
其中的随机误差都是平均过的,期望值不变而且方差缩小了。所以这个平方和虽然受到水平变动和随机误差两方面的影响,但是当因子A的不同水平对试验结果有显著差异的时候,它主要受到因子A水平变动的影响。所以称该平方和为因子A的平方和。

\subsubsection{SSB}
记:
\begin{equation*}
	SSB=\sum_{i=1}^a\sum_{j=1}^b(\bar{y}_{.j}-\bar{y}_{..})^2=a\sum_{j=1}^b(\bar{y}_{.j}-\bar{y}_{..})^2
\end{equation*}
可以发现:
\begin{align*}
	SSB&=a\sum_{j=1}^b(\bar{y}_{.j}-\bar{y}_{..})^2 \\
	&=a\sum_{j=1}^b(\mu+\bar{\tau}+\beta_j+\bar{\varepsilon}_{.j}-\mu-\bar{\tau}-\bar{\beta}-\bar{\varepsilon}_{..})^2 \\
	&=a\sum_{j=1}^b(\beta_j+\bar{\varepsilon}_{.j}-\bar{\varepsilon}_{..})^2
\end{align*}
其中的随机误差都是平均过的,期望值不变而且方差缩小了。所以这个平方和虽然受到水平变动和随机误差两方面的影响,但是当因子B的不同水平对试验结果有显著差异的时候,它主要受到因子B水平变动的影响。所以称该平方和为因子B的平方和。\par
\subsubsection{总偏差平方和分解公式}
综上,总偏差平方和有如下分解公式:
\begin{equation*}
	SST=SSA+SSB+SSe
\end{equation*}

\subsection{检验统计量}
\subsubsection{关于$\varepsilon$的一些结论}
下给出关于$\varepsilon$的一些结论:
\begin{gather*}
	\bar{\varepsilon}_{i.}\sim N(0,\frac{\sigma^2}{b}),\quad\bar{\varepsilon}_{.j}\sim
	N(0,\frac{\sigma^2}{a}),\quad\bar{\varepsilon}_{..}\sim N(0,\frac{\sigma^2}{ab}) \\
	E(\varepsilon_{ij}^2)=\sigma^2,\quad
	E(\bar{\varepsilon}_{i.}^2)=\frac{\sigma^2}{b},\quad
	E(\bar{\varepsilon}_{.j}^2)=\frac{\sigma^2}{a},\quad
	E(\bar{\varepsilon}_{..}^2)=\frac{\sigma^2}{ab} \\
	E(\bar{\varepsilon}_{i.}\bar{\varepsilon}_{..})=\frac{\sigma^2}{ab},\quad
	E(\varepsilon_{.j}\bar{\varepsilon}_{..})=\frac{\sigma^2}{ab}
\end{gather*}
\begin{proof}
	正态分布的三个结论可直接由独立正态随机变量的线性运算求得。
	\begin{align*}
		E(\bar{\varepsilon}_{ij}^2)&=Var(\bar{\varepsilon}_{ij})+[E(\bar{\varepsilon}_{ij})]^2=\sigma^2 \\
		E(\bar{\varepsilon}_{i.}^2)&=Var(\bar{\varepsilon}_{i.})+[E(\bar{\varepsilon}_{i.})]^2=\frac{\sigma^2}{b} \\
		E(\bar{\varepsilon}_{.j}^2)&=Var(\bar{\varepsilon}_{.j})+[E(\bar{\varepsilon}_{.j})]^2=\frac{\sigma^2}{a} \\
		E(\bar{\varepsilon}_{..}^2)&=Var(\bar{\varepsilon}_{..})+[E(\bar{\varepsilon}_{..})]^2=\frac{\sigma^2}{ab}
	\end{align*}
	下求$E(\bar{\varepsilon}_{i.}\bar{\varepsilon}_{..})$:
	\begin{align*}
		E(\bar{\varepsilon}_{i.}\bar{\varepsilon}_{..})
		&=E\left[\left(\frac{1}{b}\sum_{j=1}^b\varepsilon_{ij}\right)\left(\frac{1}{ab}\sum_{i=1}^a\sum_{j=1}^b\varepsilon_{ij}\right)\right] \\
		&=\frac{1}{ab^2}E\left(\sum_{j=1}^b\varepsilon_{ij}^2\right) \\
		&=\frac{1}{ab^2}\sum_{j=1}^bE(\varepsilon_{ij}^2) \\
		&=\frac{\sigma^2}{ab}
	\end{align*}
	下求$E(\varepsilon_{.j}\bar{\varepsilon}_{..})$:
	\begin{align*}
		E(\varepsilon_{.j}\bar{\varepsilon}_{..})
		&=E\left[\left(\frac{1}{a}\sum_{i=1}^a\varepsilon_{ij}\right)\left(\frac{1}{ab}\sum_{i=1}^a\sum_{j=1}^b\varepsilon_{ij}\right)\right] \\
		&=\frac{1}{a^2b}E\left(\sum_{i=1}^a\varepsilon_{ij}^2\right) \\
		&=\frac{1}{ab^2}\sum_{i=1}^aE(\varepsilon_{ij}^2) \\
		&=\frac{\sigma^2}{ab}\qedhere
	\end{align*}
\end{proof}
\subsubsection{SSA的期望}
下求SSA的期望:
\begin{align*}
	E(SSA)
	&=E\left[b\sum_{i=1}^a(\tau_i+\bar{\varepsilon}_{i.}-\bar{\varepsilon}_{..})^2\right] \\
	&=E\left[b\sum_{i=1}^a(\tau_i^2+\bar{\varepsilon}_{i.}^2+\bar{\varepsilon}_{..}^2+2\tau_i\bar{\varepsilon}_{i.}-2\tau_i\bar{\varepsilon}_{..}-2\bar{\varepsilon}_{i.}\bar{\varepsilon}_{..})\right] \\
	&=b\sum_{i=1}^a\tau_i^2+b\sum_{i=1}^aE(\bar{\varepsilon}_{i.}^2)+b\sum_{i=1}^aE(\bar{\varepsilon}_{..}^2)-\frac{2b}{ab^2}\sum_{i=1}^a\sum_{j=1}^bE(\varepsilon_{ij}^2) \\
	&=b\sum_{i=1}^a\tau_i^2+ba\frac{\sigma^2}{b}+ab\frac{\sigma^2}{ab}-\frac{2b}{ab^2}ab\sigma^2 \\
	&=b\sum_{i=1}^a\tau_i^2+(a-1)\sigma^2
\end{align*}
\subsubsection{SSB的期望}
下求SSB的期望:
\begin{align*}
	E(SSB)
	&=E\left[a\sum_{j=1}^b(\beta_j+\bar{\varepsilon}_{.j}-\bar{\varepsilon}_{..})^2\right] \\
	&=E\left[a\sum_{j=1}^b(\beta_j^2+\bar{\varepsilon}_{.j}^2+\bar{\varepsilon}_{..}^2+2\beta_j\bar{\varepsilon}_{.j}-2\beta_j\bar{\varepsilon}_{..}-2\bar{\varepsilon}_{.j}\bar{\varepsilon}_{..})\right] \\
	&=a\sum_{j=1}^b\beta_j^2+a\sum_{j=1}^bE(\bar{\varepsilon}_{.j}^2)+a\sum_{j=1}^bE(\bar{\varepsilon}_{..}^2)-\frac{2a}{a^2b}\sum_{i=1}^a\sum_{j=1}^bE(\varepsilon_{ij}^2) \\
	&=a\sum_{j=1}^b\beta_j^2+ab\frac{\sigma^2}{a}+ab\frac{\sigma^2}{ab}-\frac{2a}{a^2b}ab\sigma^2 \\
	&=a\sum_{j=1}^b\beta_j^2+(b-1)\sigma^2
\end{align*}
\subsubsection{SSe的期望}
下求SSe的期望(把均值展开,利用独立性就可以得到结果):
\begin{align*}
	E(SSe)
	&=E\left[\sum_{i=1}^a\sum_{j=1}^b(\varepsilon_{ij}-\bar{\varepsilon}_{i.}-\bar{\varepsilon}_{.j}+\bar{\varepsilon}_{..})^2\right] \\
	&=E\left[\sum_{i=1}^a\sum_{j=1}^b(\varepsilon_{ij}^2+\bar{\varepsilon}_{i.}^2+\bar{\varepsilon}_{.j}^2+\bar{\varepsilon}_{..}^2
	-2\varepsilon_{ij}\bar{\varepsilon}_{i.}-2\varepsilon_{ij}\bar{\varepsilon}_{.j}+2\varepsilon_{ij}\bar{\varepsilon}_{..}
	+2\bar{\varepsilon}_{i.}\bar{\varepsilon}_{.j}-2\bar{\varepsilon}_{i.}\bar{\varepsilon}_{..}-2\bar{\varepsilon}_{.j}\bar{\varepsilon}_{..})\right] \\
	&=(a-1)(b-1)\sigma^2
\end{align*}
\subsubsection{构建统计量}
称$\frac{SSA}{a-1}$为因子A的均方和,记为MSA;称$\frac{SSB}{b-1}$为因子B的均方和,记为MSB;称$\frac{SSe}{(a-1)(b-1)}$为误差均方和,记为MSe。\par
由前述,MSe是$\sigma^2$的无偏估计,而当零假设成立时,MSA和MSB也是$\sigma^2$的一个无偏估计。如果MSA、MSB与MSe比值很大,即MSA、MSB比MSe大很多($b\sum\limits_{i=1}^a\tau_i^2,\;a\sum\limits_{j=1}^b\beta_j^2$很大),我们就有理由怀疑零假设。。由此构建统计量:
\begin{gather*}
	F_A=\frac{MSA}{MSe}=\frac{\frac{SSA}{a-1}}{\frac{SSe}{(a-1)(b-1)}} \\
	F_B=\frac{MSB}{MSe}=\frac{\frac{SSB}{b-1}}{\frac{SSe}{(a-1)(b-1)}} 	
\end{gather*}
在这些统计量的情况下,$H_{01},\;H_{02}$的拒绝域是右向单尾的。下求统计量的分布。

\subsection{统计量的分布}
上述统计量服从如下分布:
\begin{gather*}
	F_A\sim F(a-1,\;(a-1)(b-1)) \\
	F_B\sim F(b-1,\;(a-1)(b-1))
\end{gather*}
所以$H_{01},\;H_{02}$在显著性水平为$\alpha$时的拒绝域为:
\begin{gather*}
	F_A>F_{1-\alpha}(a-1,\;(a-1)(b-1)) \\
	F_B>F_{1-\alpha}(b-1,\;(a-1)(b-1))
\end{gather*}

\subsection{方差分析表}
\begin{table}[H]
	\centering
	\begin{tabularx}{\textwidth}
		{>{\centering\arraybackslash}c|*{5}{>{\centering\arraybackslash}X}}
		\toprule
		来源   &平方和&自由度&均方和             &F值  \\ 
		\midrule
		因子A&SSA&$f_A=a-1$ &$\frac{SSA}{a-1}$ &$F=\frac{MSA}{MSe}$\\
		因子B&SSB&$f_B=b-1$ &$\frac{SSB}{b-1}$ &$F=\frac{MSB}{MSe}$\\
		误差   &SSe  &$f_e=(a-1)(b-1)$ &$\frac{SSe}{(a-1)(b-1)}$ & \\
		总     &SST  &$f_T=ab-1$ &                  & \\
		\bottomrule
	\end{tabularx}
	\caption{无重复可加效应下两因子试验方差分析表}
\end{table}
平方和公式可按下列公式计算:
\begin{equation*}
	\begin{cases}
		SST=\sum\limits_{i=1}^a\sum\limits_{j=1}^by_{ij}^2-\frac{y_{..}^2}{ab} \\
		SSA=\sum\limits_{i=1}^a\frac{y_{i.}^2}{b}-\frac{y_{..}^2}{ab} \\
		SSB=\sum\limits_{j=1}^b\frac{y_{.j}^2}{a}-\frac{y_{..}^2}{ab} \\
		SSe=SST-SSA-SSB
	\end{cases}
\end{equation*}

\subsection{参数估计}
可加效应下的两因子方差分析有四类参数:$\mu$,诸$\tau_i$,诸$\beta_j$和$\sigma^2$。下讨论这四类参数的点估计问题。
\subsubsection{点估计}
参数的点估计如下:
\begin{gather*}
	\hat{\sigma^2}=MSe \\
	\hat{\mu}=\bar{y}_{..} \\
	\hat{\tau}_i=\bar{y}_{i.}-\bar{y}_{..},\;i=1,2,\dots,a \\
	\hat{\beta}_j=\bar{y}_{.j}-\bar{y}_{..},\;j=1,2,\dots,b
\end{gather*}
其中$\hat{\mu},\;\hat{\tau}_i,\;\hat{\beta}_j$是使用最小二乘估计得到的。
\begin{proof}
	分别用$\hat{\mu},\;\hat{\tau}_i,\;\hat{\beta}_j$表示$\mu$,诸$\tau_i$和诸$\beta_j$的估计,用$\hat{y}_{ij}=\hat{\mu}+\hat{\tau}_i+\hat{\beta}_j$表示$y_{ij}$的估计,$i=1,2,\dots,a,\;j=1,2,\dots,b$。损失函数为:
	\begin{equation*}
		L=\sum_{i=1}^a\sum_{j=1}^b(y_{ij}-\hat{y}_{ij})^2=\sum_{i=1}^a\sum_{j=1}^b(y_{ij}-\hat{\mu}-\hat{\tau}_i-\hat{\beta}_j)^2
	\end{equation*}
	最小二乘解需要满足:
	\begin{equation*}
		\begin{cases}
			\vspace{2ex}
			\dfrac{\partial L}{\partial\hat{\mu}}=0, \\
			\vspace{2ex}
			\dfrac{\partial L}{\partial\hat{\tau}_i}=0,\quad 
			i=1,2,\dots,a \\
			\vspace{2ex}
			\dfrac{\partial L}{\partial\hat{\beta}_j}=0,\quad
			j=1,2,\dots,b \\
			\vspace{2ex}
			\sum\limits_{i=1}^a\hat{\tau}_i=0 \\
			\vspace{2ex}
			\sum\limits_{j=1}^b\hat{\beta}_j=0
		\end{cases}
	\end{equation*}
	上式即可解出结果。
\end{proof}

\subsection{多重比较问题}
如果此时某因子显著,则需要对它的各水平均值采用Duncan多重比较法去判断哪些水平之间存在显著差异。此时的水平均值即为在另一因子各水平下的均值。

\subsection{等重复试验情形}
\subsubsection{等重复试验下的方差分析表}
\begin{table}[H]
	\centering
	\begin{tabularx}{\textwidth}
		{>{\centering\arraybackslash}c|*{5}{>{\centering\arraybackslash}X}}
		\toprule
		来源   &平方和&自由度&均方和             &F值  \\ 
		\midrule
		因子A&SSA&$f_A=a-1$ &$\frac{SSA}{a-1}$ &$F=\frac{MSA}{MSe}$\\
		因子B&SSB&$f_B=b-1$ &$\frac{SSB}{b-1}$ &$F=\frac{MSB}{MSe}$\\
		误差   &SSe  &$f_e=abm-a-b+1$ &$\frac{SSe}{f_e}$ & \\
		总     &SST  &$f_T=abm-1$ &                  & \\
		\bottomrule
	\end{tabularx}
	\caption{可加效应下两因子等重复试验方差分析表}
\end{table}
其中$m$为重复实验次数。平方和公式可按下列公式计算:
\begin{equation*}
	\begin{cases}
		SST=\sum\limits_{i=1}^a\sum\limits_{j=1}^b\sum\limits_{k=1}^my_{ijk}^2-\frac{y_{...}^2}{abm} \\
		SSA=\sum\limits_{i=1}^a\frac{y_{i..}^2}{bm}-\frac{y_{...}^2}{abm} \\
		SSB=\sum\limits_{j=1}^b\frac{y_{.j.}^2}{am}-\frac{y_{..}^2}{abm} \\
		SSe=SST-SSA-SSB
	\end{cases}
\end{equation*}
其中:
\begin{gather*}
	y_{...}=\sum_{i=1}^a\sum_{j=1}^b\sum_{k=1}^my_{ijk} \\
	y_{i..}=\sum_{j=1}^b\sum_{k=1}^my_{ijk},\quad i=1,2,\dots,a \\
	y_{.j.}=\sum_{i=1}^a\sum_{k=1}^my_{ijk},\quad j=1,2,\dots,b
\end{gather*}
\subsubsection{参数估计}
参数的点估计如下:
\begin{gather*}
	\hat{\mu}=\bar{y}_{...} \\
	\hat{\tau}_i=\bar{y}_{i..}-\bar{y}_{...},\;i=1,2,\dots,a \\
	\hat{\beta}_j=\bar{y}_{.j.}-\bar{y}_{...},\;j=1,2,\dots,b
\end{gather*}
\subsubsection{多重比较问题}
此时的Duncan多重比较过程与无重复试验的情况完全一样,只是需要注意:
\begin{gather*}
	A:R_p>r_{1-\alpha}(p,f)\sqrt{\frac{MSe}{bm}} \\
	B:R_p>r_{1-\alpha}(p,f)\sqrt{\frac{MSe}{am}}
\end{gather*}















\section{交互效应下的两因子方差分析}
仅讨论等重复情形。
\begin{table}[H] 
	\centering
	\begin{tabularx}{\textwidth}
		{c|>{\centering\arraybackslash}X>{\centering\arraybackslash}Xc>{\centering\arraybackslash}X}
		\hline
		\diagbox{因子$A$}{因子$B$} & $B_1$ & $B_2$ & $\cdots$ & $B_b$ \\ \hline
		$A_1$ & 
		$y_{111}, y_{112}, \dots, y_{11m}$ & 
		$y_{121}, y_{122}, \dots, y_{12m}$ & 
		$\cdots$ & 
		$y_{1b1}, y_{1b2}, \dots, y_{1bm}$ \\ 
		$A_2$ & 
		$y_{211}, y_{212}, \dots, y_{21m}$ & 
		$y_{221}, y_{222}, \dots, y_{22m}$ & 
		$\cdots$ & 
		$y_{2b1}, y_{2b2}, \dots, y_{2bm}$ \\
		$\vdots$ & 
		$\vdots$ & 
		$\vdots$ & 
		& 
		$\vdots$ \\
		$A_a$ & 
		$y_{a11}, y_{a12}, \dots, y_{a1m}$ & 
		$y_{a21}, y_{a22}, \dots, y_{a2m}$ & 
		$\cdots$ & 
		$y_{ab1}, y_{ab2}, \dots, y_{abm}$ \\ 
		\hline
	\end{tabularx}
	\caption{等重复两因子试验数据表}
\end{table}
其中$y_{ijk}$表示在因子A的第$i$个水平$A_i$和因子B的第$j$个水平$B_j$下第$k$次重复试验的观察值。

\subsection{统计模型}
由可加效应下的两因子方差分析模型(见\cref{model:additive-effect-two-way-anova}),如果:
\begin{equation*}
	\mu_{ij}\ne \mu+\tau_i+\beta_j
\end{equation*}
则记:
\begin{equation*}
	(\tau\beta)_{ij}=\mu_{ij}-(\mu+\tau_i+\beta_j),\quad i=1,2,\dots,a,\;j=1,2,\dots,b
\end{equation*}
称$(\tau\beta)_{ij}$为因子A的水平$A_i$和因子B的水平$B_j$的交互效应。它表示两个因子的主效应之外,由于水平搭配而引起的新的效应。所有交互效应的全体称为交互作用,记为AB或A$\times$B。交互效应应满足如下条件:
\begin{gather*}
	\sum_{j=1}^b(\tau\beta)_{ij}=0,\quad i=1,2,\dots,a \\
	\sum_{i=1}^a(\tau\beta)_{ij}=0,\quad j=1,2,\dots,b
\end{gather*}
此时的统计模型即为:
\begin{equation*}\label{model:interaction-effect-two-way-anova}
	\begin{cases}
		y_{ijk}=\mu+\tau_i+\beta_j+(\tau\beta)_{ij}+\varepsilon_{ijk} \\
		\text{诸}\varepsilon_{ijk}\quad\mathrm{i.i.d.~}N(0,\sigma^2) \\
		s.t.\quad\sum\limits_{i=1}^a\tau_i=0,\quad\sum\limits_{j=1}^b\beta_j=0 \\
		\qquad\;\;\sum\limits_{j=1}^b(\tau\beta)_{ij}=0,\quad 	\sum\limits_{i=1}^a(\tau\beta)_{ij}=0 \\
		i=1,2,\dots,a,\;j=1,2,\dots,b,\;k=1,2,\dots,m
	\end{cases}
\end{equation*}

\subsection{统计假设}
交互效应下的两因子方差分析需要检验如下三个零假设:
\begin{equation*}
	\begin{cases}
		H_{01}:\tau_1=\tau_2=\cdots=\tau_a=0, \\
		H_{02}:\beta_1=\beta_2=\cdots=\beta_b=0 \\
		H_{03}:(\tau\beta)_{ij}=0,\;\forall\;i,j
	\end{cases}
\end{equation*}

\subsection{偏差平方和的分解}
记:
\begin{gather*}
	y_{...}=\sum_{i=1}^a\sum_{j=1}^b\sum_{k=1}^my_{ijk},\quad
	\bar{y}_{...}=\frac{y_{...}}{abm} \\
	y_{i..}=\sum_{j=1}^b\sum_{k=1}^my_{ijk},\quad
	\bar{y}_{i..}=\frac{y_{i..}}{bm},\quad i=1,2,\dots,a \\
	y_{.j.}=\sum_{i=1}^a\sum_{k=1}^my_{ijk},\quad
	\bar{y}_{.j.}=\frac{y_{.j.}}{am},\quad j=1,2,\dots,b \\
	y_{ij.}=\sum_{k=1}^my_{ijk},\quad
	\bar{y}_{ij.}=\frac{y_{ij.}}{m}
\end{gather*}
全部数据之间的差异可用下述总偏差平方和表示:
\begin{equation*}
	SST=\sum_{i=1}^a\sum_{j=1}^b(y_{ij}-\bar{y}_{..})^2
\end{equation*}
引起数据$y_{ij}$之间差异的原因有四点:
\begin{enumerate}
	\item 因子A的$a$个水平对试验结果的影响不同。
	\item 因子B的$b$个水平对试验结果的影响不同。
	\item 因子A和因子B的交互作用。
	\item 试验具有误差。
\end{enumerate}
为了区分并比较这四个原因对数据的影响,需要对SST进行分解:
\begin{align*}
	SST
	&=\sum_{i=1}^a\sum_{j=1}^b\sum_{k=1}^m(y_{ijk}-\bar{y}_{...})^2 \\
	&=\sum_{i=1}^a\sum_{j=1}^b\sum_{k=1}^m\left[(\bar{y}_{i..}-\bar{y}_{...})+(\bar{y}_{.j.}-\bar{y}_{...})+(\bar{y}_{ij.}-\bar{y}_{i..}-\bar{y}_{.j.}+\bar{y}_{...})+(y_{ijk}-\bar{y}_{ij.})\right]^2 \\
	&=bm\sum_{i=1}^a(\bar{y}_{i..}-\bar{y}_{...})^2+am\sum_{j=1}^b(\bar{y}_{.j.}-\bar{y}_{...})^2+m\sum_{i=1}^a\sum_{j=1}^b(\bar{y}_{ij.}-\bar{y}_{i..}-\bar{y}_{.j.}+\bar{y}_{...})^2+\sum_{i=1}^a\sum_{j=1}^b\sum_{k=1}^m(y_{ijk}-\bar{y}_{ij.})^2
\end{align*}
仿照单因子方差分析、可加效应下的两因子方差分析的做法,仍可以分别定义:
\begin{gather*}
	SSA=bm\sum_{i=1}^a(\bar{y}_{i..}-\bar{y}_{...})^2=bm\sum_{i=1}^a(\tau_i+\bar{\varepsilon}_{i..}-\bar{\varepsilon}_{...})^2 \\
	SSB=am\sum_{j=1}^b(\bar{y}_{.j.}-\bar{y}_{...})^2=am\sum_{j=1}^b(\beta_j+\bar{\varepsilon}_{.j.}-\bar{\varepsilon}_{...})^2 \\
	SSAB=m\sum_{i=1}^a\sum_{j=1}^b(\bar{y}_{ij.}-\bar{y}_{i..}-\bar{y}_{.j.}+\bar{y}_{...})^2=m\sum_{i=1}^a\sum_{j=1}^b\left[(\tau\beta)_{ij}+\bar{\varepsilon}_{ij.}-\bar{\varepsilon}_{i..}-\bar{\varepsilon}_{.j.}+3\bar{\varepsilon}_{...}\right]^2 \\
	SSe=\sum_{i=1}^a\sum_{j=1}^b\sum_{k=1}^m(y_{ijk}-\bar{y}_{ij.})^2=\sum_{i=1}^a\sum_{j=1}^b\sum_{k=1}^m(\varepsilon_{ijk}-\bar{\varepsilon}_{ij.})^2
\end{gather*}
分别称如上公式为:因子A的偏差平方和、因子B的偏差平方和、因子A与因子B的交互作用的偏差平方和、误差平方和。
\subsubsection{总偏差平方和分解公式}
综上,总偏差平方和有如下分解公式:
\begin{equation*}
	SST=SSA+SSB++SSAB+SSe
\end{equation*}

\subsection{检验统计量}
仿照单因子方差分析、可加效应下的两因子方差分析的做法,仍可以求得有关$\varepsilon$的一系列结论,以及各偏差平方和的期望。这里直接列出结论:
\begin{gather*}
	E(SSA)=(a-1)\sigma^2+bm\sum_{i=1}^a\tau_i^2 \\
	E(SSB)=(b-1)\sigma^2+am\sum_{j=1}^b\beta_j^2 \\
	E(SSAB)=(a-1)(b-1)\sigma^2+m\sum_{i=1}^a\sum_{j=1}^b(\tau\beta)_{ij}^2 \\
	E(SSe)=ab(m-1)\sigma^2
\end{gather*}
称$\frac{SSA}{a-1}$为因子A的均方和,记为MSA;称$\frac{SSB}{b-1}$为因子B的均方和,记为MSB;称$\frac{SSAB}{(a-1)(b-1)}$为因子A与因子B的交互作用的均方和,记为MSAB;称$\frac{SSe}{ab(m-1)}$为误差均方和,记为MSe。\par
由前述,MSe是$\sigma^2$的无偏估计,而当零假设成立时,MSA、MSB、MSAB也是$\sigma^2$的一个无偏估计。如果MSA、MSB、MSAB与MSe比值很大,即MSA、MSB、MSAB比MSe大很多($b\sum\limits_{i=1}^a\tau_i^2,\;a\sum\limits_{j=1}^b\beta_j^2,\;m\sum\limits_{i=1}^a\sum\limits_{j=1}^b(\tau\beta)_{ij}^2$很大),我们就有理由怀疑零假设。由此构建统计量:
\begin{gather*}
	F_A=\frac{MSA}{MSe}=\frac{\frac{SSA}{a-1}}{\frac{SSe}{ab(m-1)}} \\
	F_B=\frac{MSB}{MSe}=\frac{\frac{SSB}{b-1}}{\frac{SSe}{ab(m-1)}} \\
	F_{AB}=\frac{MSAB}{MSe}=\frac{\frac{SSAB}{(a-1)(b-1)}}{\frac{SSe}{ab(m-1)}}
\end{gather*}
在这些统计量的情况下,$H_{01},\;H_{02},\;H_{03}$的拒绝域是右向单尾的。下求统计量的分布。

\subsection{统计量的分布}
上述统计量服从如下分布:
\begin{gather*}
	F_A\sim F(a-1,\;ab(m-1)) \\
	F_B\sim F(b-1,\;ab(m-1)) \\
	F_{AB}\sim F((a-1)(b-1),\;ab(m-1))
\end{gather*}
所以$H_{01},\;H_{02}$在显著性水平为$\alpha$时的拒绝域为:
\begin{gather*}
	F_A>F_{1-\alpha}(a-1,\;ab(m-1)) \\
	F_B>F_{1-\alpha}(b-1,\;ab(m-1)) \\
	F_{AB}>F_{1-\alpha}((a-1)(b-1),\;ab(m-1))
\end{gather*}

\subsection{方差分析表}
\begin{table}[H]
	\centering
	\begin{tabularx}{\textwidth}
		{>{\centering\arraybackslash}c|*{5}{>{\centering\arraybackslash}X}}
		\toprule
		来源   &平方和&自由度&均方和             &F值  \\ 
		\midrule
		因子A&SSA&$f_A=a-1$ &$\frac{SSA}{a-1}$ &$F=\frac{MSA}{MSe}$\\
		因子B&SSB&$f_B=b-1$ &$\frac{SSB}{b-1}$ &$F=\frac{MSB}{MSe}$\\
		交互作用AB &SSAB &$f_{AB}=(a-1)(b-1)$ &$\frac{SSAB}{(a-1)(b-1)}$ &$F=\frac{MSAB}{MSe}$ \\
		误差   &SSe  &$f_e=ab(m-1)$ &$\frac{SSe}{ab(m-1)}$ & \\
		总     &SST  &$f_T=abm-1$ &                  & \\
		\bottomrule
	\end{tabularx}
	\caption{等重复交互效应下两因子试验方差分析表}
\end{table}
平方和公式可按下列公式计算:
\begin{equation*}
	\begin{cases}
		SST=\sum\limits_{i=1}^a\sum\limits_{j=1}^b\sum\limits_{k=1}^my_{ijk}^2-\frac{y_{...}^2}{abm} \\
		SSA=\sum\limits_{i=1}^a\frac{y_{i..}^2}{bm}-\frac{y_{...}^2}{abm} \\
		SSB=\sum\limits_{j=1}^b\frac{y_{.j.}^2}{am}-\frac{y_{...}^2}{abm} \\
		SSAB=\sum\limits_{i=1}^a\sum\limits_{j=1}^b\frac{y_{ij.}^2}{m}-\frac{y_{...}^2}{abm}-SSA-SSB \\
		SSe=SST-SSA-SSB-SSAB
	\end{cases}
\end{equation*}

\subsection{参数估计}
交互效应下的两因子方差分析有五类参数:$\mu$,诸$\tau_i$,诸$\beta_j$,诸$(\tau\beta)_{ij}$和$\sigma^2$。下讨论这五类参数的点估计问题。
\subsubsection{点估计}
参数的点估计如下:
\begin{gather*}
	\hat{\sigma^2}=MSe \\
	\hat{\mu}=\bar{y}_{...} \\
	\hat{\tau}_i=\bar{y}_{i..}-\bar{y}_{...},\quad
	\hat{\beta}_j=\bar{y}_{.j.}-\bar{y}_{...} \\
	\widehat{(\tau\beta)}_{ij}=\bar{y}_{ij.}-\bar{y}_{i..}-\bar{y}_{.j.}+\bar{y}_{...} \\
	i=1,2,\dots,a,\;j=1,2,\dots,b
\end{gather*}
证明过程与之前类似。


\section{随机效应下的两因子方差分析}
仅讨论等重复、有交互作用情形。
\begin{table}[H] 
	\centering
	\begin{tabularx}{\textwidth}
		{c|>{\centering\arraybackslash}X>{\centering\arraybackslash}Xc>{\centering\arraybackslash}X}
		\hline
		\diagbox{因子$A$}{因子$B$} & $B_1$ & $B_2$ & $\cdots$ & $B_b$ \\ \hline
		$A_1$ & 
		$y_{111}, y_{112}, \dots, y_{11m}$ & 
		$y_{121}, y_{122}, \dots, y_{12m}$ & 
		$\cdots$ & 
		$y_{1b1}, y_{1b2}, \dots, y_{1bm}$ \\ 
		$A_2$ & 
		$y_{211}, y_{212}, \dots, y_{21m}$ & 
		$y_{221}, y_{222}, \dots, y_{22m}$ & 
		$\cdots$ & 
		$y_{2b1}, y_{2b2}, \dots, y_{2bm}$ \\
		$\vdots$ & 
		$\vdots$ & 
		$\vdots$ & 
		& 
		$\vdots$ \\
		$A_a$ & 
		$y_{a11}, y_{a12}, \dots, y_{a1m}$ & 
		$y_{a21}, y_{a22}, \dots, y_{a2m}$ & 
		$\cdots$ & 
		$y_{ab1}, y_{ab2}, \dots, y_{abm}$ \\ 
		\hline
	\end{tabularx}
	\caption{等重复两因子试验数据表}
\end{table}
其中$y_{ijk}$表示在因子A的第$i$个水平$A_i$和因子B的第$j$个水平$B_j$下第$k$次重复试验的观察值。

\subsection{统计模型}
此时的统计模型为:
\begin{equation*}\label{model:random-effect-two-way-anova}
	\begin{cases}
		y_{ijk}=\mu+\tau_i+\beta_j+(\tau\beta)_{ij}+\varepsilon_{ijk} \\
		\text{诸}\varepsilon_{ijk}\quad\mathrm{i.i.d.~}N(0,\sigma^2) \\
		\text{诸}\tau_i\quad\mathrm{i.i.d.~}N(0,\sigma_\tau^2) \\
		\text{诸}\beta_j\quad\mathrm{i.i.d.~}N(0,\sigma_\beta^2) \\
		\text{诸}(\tau\beta)_{ij}\quad\mathrm{i.i.d.~}N(0,\sigma_{\tau\beta}^2) \\
		\text{诸}\varepsilon_{ijk}\text{、诸}\tau_i\text{、诸}\beta_j\text{、诸}(\tau\beta)_{ij}\text{相互独立} \\
		i=1,2,\dots,a,\;j=1,2,\dots,b,\;k=1,2,\dots,m
	\end{cases}
\end{equation*}

\subsection{统计假设}
此时需要检验如下三个零假设:
\begin{equation*}
	\begin{cases}
		H_{01}:\sigma_\tau^2=0, \\
		H_{02}:\sigma_\beta^2=0 \\
		H_{03}:\sigma_{\tau\beta}^2=0
	\end{cases}
\end{equation*}

\subsection{方差分析}
\subsubsection{偏差平方和的分解}
由于与交互效应情况下五个平方和的定义完全相同,所以随机效应下偏差平方和分解公式与之前一模一样。
\begin{gather*}
	SST=SSA+SSB+SSAB+SSe \\
	SSA=bm\sum_{i=1}^a(\bar{y}_{i..}-\bar{y}_{...})^2=bm\sum_{i=1}^a(\tau_i+\bar{\varepsilon}_{i..}-\bar{\varepsilon}_{...})^2 \\
	SSB=am\sum_{j=1}^b(\bar{y}_{.j.}-\bar{y}_{...})^2=am\sum_{j=1}^b(\beta_j+\bar{\varepsilon}_{.j.}-\bar{\varepsilon}_{...})^2 \\
	SSAB=m\sum_{i=1}^a\sum_{j=1}^b(\bar{y}_{ij.}-\bar{y}_{i..}-\bar{y}_{.j.}+\bar{y}_{...})^2=m\sum_{i=1}^a\sum_{j=1}^b\left[(\tau\beta)_{ij}+\bar{\varepsilon}_{ij.}-\bar{\varepsilon}_{i..}-\bar{\varepsilon}_{.j.}+3\bar{\varepsilon}_{...}\right]^2 \\
	SSe=\sum_{i=1}^a\sum_{j=1}^b\sum_{k=1}^m(y_{ijk}-\bar{y}_{ij.})^2=\sum_{i=1}^a\sum_{j=1}^b\sum_{k=1}^m(\varepsilon_{ijk}-\bar{\varepsilon}_{ij.})^2
\end{gather*}
\subsubsection{各平方和的期望}
这里直接列出各平方和的期望,证明是容易的。
\begin{gather*}
	E(SSA)=(a-1)\sigma^2+m(a-1)\sigma_{\tau\beta}^2+(a-1)bm\sigma_\tau^2  \\
	E(SSB)=(b-1)\sigma^2+m(b-1)\sigma_{\tau\beta}^2+(b-1)am\sigma_{\beta}^2 \\
	E(SSAB)=(a-1)(b-1)\sigma^2+m(a-1)(b-1)\sigma_{\tau\beta}^2 \\
	E(SSe)=ab(m-1)\sigma^2
\end{gather*}
\subsubsection{统计量及其分布}
称$\frac{SSA}{a-1}$为因子A的均方和,记为MSA;称$\frac{SSB}{b-1}$为因子B的均方和,记为MSB;称$\frac{SSAB}{(a-1)(b-1)}$为因子A与因子B的交互作用的均方和,记为MSAB;称$\frac{SSe}{ab(m-1)}$为误差均方和,记为MSe。\par
由前述,MSe是$\sigma^2$的无偏估计。在$H_{01}$成立时,MSA是$\sigma^2+m\sigma_{\tau\beta}^2$的无偏估计;在$H_{02}$成立时,MSA是$\sigma^2+m\sigma_{\tau\beta}^2$的无偏估计;在$H_{03}$成立时,MSAB是$\sigma^2$的无偏估计。如果MSA与MSAB比值很大,则有理由怀疑$H_01$;如果MSB与MSAB比值很大,则有理由怀疑$H_02$;如果MSAB与MSe比值很大,则有理由怀疑$H_03$。由此构建统计量:
\begin{gather*}
	F_A=\frac{MSA}{MSAB}=\frac{\frac{SSA}{a-1}}{\frac{SSAB}{(a-1)(b-1)}} \\
	F_B=\frac{MSB}{MSAB}=\frac{\frac{SSB}{b-1}}{\frac{SSAB}{(a-1)(b-1)}} \\
	F_{AB}=\frac{MSAB}{MSe}=\frac{\frac{SSAB}{(a-1)(b-1)}}{\frac{SSe}{ab(m-1)}}
\end{gather*}
在这些统计量的情况下,$H_{01},\;H_{02},\;H_{03}$的拒绝域是右向单尾的。\par
由于在假设$H_{01},\;H_{02},\;H_{03}$成立时,随机效应模型与可加效应模型的观察值$y_{ijk}$的数据结构的形式完全一样,所以在随机效应模型中仍然有:
\begin{gather*}
	F_A\sim F(a-1,\;(a-1)(b-1)) \\
	F_B\sim F(b-1,\;(a-1)(b-1)) \\
	F_{AB}\sim F((a-1)(b-1),\;ab(m-1))
\end{gather*}
\subsubsection{拒绝域}
综上所述,显著性水平为$\alpha$时的拒绝域为:
\begin{gather*}
	F_A>F_{1-\alpha}(a-1,\;(a-1)(b-1)) \\
	F_B>F_{1-\alpha}(b-1,\;(a-1)(b-1)) \\
	F_{AB}>F_{1-\alpha}((a-1)(b-1),\;ab(m-1))
\end{gather*}

\subsection{方差分析表}
\begin{table}[H]
	\centering
	\begin{tabularx}{\textwidth}
		{>{\centering\arraybackslash}c|*{5}{>{\centering\arraybackslash}X}}
		\toprule
		来源   &平方和&自由度&均方和             &F值  \\ 
		\midrule
		因子A&SSA&$f_A=a-1$ &$\frac{SSA}{a-1}$ &$F=\frac{MSA}{MSAB}$\\
		因子B&SSB&$f_B=b-1$ &$\frac{SSB}{b-1}$ &$F=\frac{MSB}{MSAB}$\\
		交互作用AB &SSAB &$f_{AB}=(a-1)(b-1)$ &$\frac{SSAB}{(a-1)(b-1)}$ &$F=\frac{MSAB}{MSe}$ \\
		误差   &SSe  &$f_e=ab(m-1)$ &$\frac{SSe}{ab(m-1)}$ & \\
		总     &SST  &$f_T=abm-1$ &                  & \\
		\bottomrule
	\end{tabularx}
	\caption{等重复、有交互作用的随机效应下两因子试验方差分析表}
\end{table}
平方和公式可按下列公式计算:
\begin{equation*}
	\begin{cases}
		SST=\sum\limits_{i=1}^a\sum\limits_{j=1}^b\sum\limits_{k=1}^my_{ijk}^2-\frac{y_{...}^2}{abm} \\
		SSA=\sum\limits_{i=1}^a\frac{y_{i..}^2}{bm}-\frac{y_{...}^2}{abm} \\
		SSB=\sum\limits_{j=1}^b\frac{y_{.j.}^2}{am}-\frac{y_{...}^2}{abm} \\
		SSAB=\sum\limits_{i=1}^a\sum\limits_{j=1}^b\frac{y_{ij.}^2}{m}-\frac{y_{...}^2}{abm}-SSA-SSB \\
		SSe=SST-SSA-SSB-SSAB
	\end{cases}
\end{equation*}

\subsection{参数估计}
我们此时关心方差分量的估计。
\subsubsection{点估计}
由SSA、SSB、SSAB、SSe期望的计算,可给出各方差分量的无偏点估计如下:
\begin{gather*}
	\hat{\sigma^2}=MSe \\
	\hat{\sigma^2}_\tau=\frac{MSA-MSAB}{bm} \\
	\hat{\sigma^2}_\beta=\frac{MSB-MSAB}{am} \\
	\hat{\sigma^2}_{\tau\beta}=\frac{MSAB-MSe}{m}
\end{gather*}
\section{混合模型下的两因子方差分析}
仅讨论等重复情形。
\begin{table}[H] 
	\centering
	\begin{tabularx}{\textwidth}
		{c|>{\centering\arraybackslash}X>{\centering\arraybackslash}Xc>{\centering\arraybackslash}X}
		\hline
		\diagbox{因子$A$}{因子$B$} & $B_1$ & $B_2$ & $\cdots$ & $B_b$ \\ \hline
		$A_1$ & 
		$y_{111}, y_{112}, \dots, y_{11m}$ & 
		$y_{121}, y_{122}, \dots, y_{12m}$ & 
		$\cdots$ & 
		$y_{1b1}, y_{1b2}, \dots, y_{1bm}$ \\ 
		$A_2$ & 
		$y_{211}, y_{212}, \dots, y_{21m}$ & 
		$y_{221}, y_{222}, \dots, y_{22m}$ & 
		$\cdots$ & 
		$y_{2b1}, y_{2b2}, \dots, y_{2bm}$ \\
		$\vdots$ & 
		$\vdots$ & 
		$\vdots$ & 
		& 
		$\vdots$ \\
		$A_a$ & 
		$y_{a11}, y_{a12}, \dots, y_{a1m}$ & 
		$y_{a21}, y_{a22}, \dots, y_{a2m}$ & 
		$\cdots$ & 
		$y_{ab1}, y_{ab2}, \dots, y_{abm}$ \\ 
		\hline
	\end{tabularx}
	\caption{等重复两因子试验数据表}
\end{table}
其中$y_{ijk}$表示在因子A的第$i$个水平$A_i$和因子B的第$j$个水平$B_j$下第$k$次重复试验的观察值。设因子A是固定地,因子B是随机的。

\subsection{统计模型}
此时的统计模型为:
\begin{equation*}\label{model:mixed-two-way-anova}
	\begin{cases}
		y_{ijk}=\mu+\tau_i+\beta_j+(\tau\beta)_{ij}+\varepsilon_{ijk} \\
		\text{诸}\varepsilon_{ijk}\quad\mathrm{i.i.d.~}N(0,\sigma^2) \\
		\text{诸}\beta_j\quad\mathrm{i.i.d.~}N(0,\sigma_\beta^2) \\
		\text{诸}(\tau\beta)_{ij}\quad\mathrm{i.i.d.~}N(0,\frac{a-1}{a}\sigma_{\tau\beta}^2) \\
		\sum\limits_{i=1}^a\tau_i=0,\quad\sum\limits_{i=1}^a(\tau\beta)_{ij}=0 \\
		\text{诸}(\tau\beta)_{ij}\text{彼此之间是不相关的} \\
		i=1,2,\dots,a,\;j=1,2,\dots,b,\;k=1,2,\dots,m
	\end{cases}
\end{equation*}
这里$(\tau\beta)_{ij}$的方差写成这样是为了让底下的计算更加简洁。

\subsection{统计假设}
此时需要检验如下三个零假设:
\begin{equation*}
	\begin{cases}
		H_{01}:\tau_1=\tau_2=\cdots=\tau_a=0, \\
		H_{02}:\sigma_\beta^2=0 \\
		H_{03}:\sigma_{\tau\beta}^2=0
	\end{cases}
\end{equation*}

\subsection{方差分析}
\subsubsection{偏差平方和的分解}
公式仍然成立:
\begin{equation*}
	SST=SSA+SSB+SSAB+SSe
\end{equation*}
\subsubsection{各平方和的期望}
这里直接列出各平方和的期望。
\begin{gather*}
	E(MSA)=\sigma^2+m\sigma_{\tau\beta}^2+\frac{bm\sum\limits_{i=1}^a\tau_i^2}{a-1} \\
	E(MSB)=\sigma^2+am\sigma_\beta^2 \\
	E(MSAB)=\sigma^2+m\sigma_{\tau\beta}^2 \\
	E(MSe)=\sigma^2
\end{gather*}
\subsubsection{统计量及其分布}
统计量及其分布如下:
\begin{gather*}
	F_A=\frac{MSA}{MSAB}\sim F(a-1,\;(a-1)(b-1)) \\
	F_B=\frac{MSB}{MSe}\sim F(b-1,\;ab(m-1)) \\
	F_{AB}=\frac{MSAB}{MSe}\sim F((a-1)(b-1),\;ab(m-1))
\end{gather*}
\subsubsection{拒绝域}
显著性水平为$\alpha$时的拒绝域为:
\begin{gather*}
	F_A>F_{1-\alpha}(a-1,\;(a-1)(b-1)) \\
	F_B>F_{1-\alpha}(b-1,\;ab(m-1)) \\
	F_{AB}>F_{1-\alpha}((a-1)(b-1),\;ab(m-1))
\end{gather*}
\subsection{方差分析表}
\begin{table}[H]
	\centering
	\begin{tabularx}{\textwidth}
		{>{\centering\arraybackslash}c|*{5}{>{\centering\arraybackslash}X}}
		\toprule
		来源   &平方和&自由度&均方和             &F值  \\ 
		\midrule
		因子A(固定)&SSA&$f_A=a-1$ &$\frac{SSA}{a-1}$ &$F=\frac{MSA}{MSAB}$\\
		因子B(随机)&SSB&$f_B=b-1$ &$\frac{SSB}{b-1}$ &$F=\frac{MSB}{MSe}$\\
		交互作用AB &SSAB &$f_{AB}=(a-1)(b-1)$ &$\frac{SSAB}{(a-1)(b-1)}$ &$F=\frac{MSAB}{MSe}$ \\
		误差   &SSe  &$f_e=ab(m-1)$ &$\frac{SSe}{ab(m-1)}$ & \\
		总     &SST  &$f_T=abm-1$ &                  & \\
		\bottomrule
	\end{tabularx}
	\caption{等重复、有交互作用的混合模型下两因子试验方差分析表}
\end{table}
平方和公式可按下列公式计算:
\begin{equation*}
	\begin{cases}
		SST=\sum\limits_{i=1}^a\sum\limits_{j=1}^b\sum\limits_{k=1}^my_{ijk}^2-\frac{y_{...}^2}{abm} \\
		SSA=\sum\limits_{i=1}^a\frac{y_{i..}^2}{bm}-\frac{y_{...}^2}{abm} \\
		SSB=\sum\limits_{j=1}^b\frac{y_{.j.}^2}{am}-\frac{y_{...}^2}{abm} \\
		SSAB=\sum\limits_{i=1}^a\sum\limits_{j=1}^b\frac{y_{ij.}^2}{m}-\frac{y_{...}^2}{abm}-SSA-SSB \\
		SSe=SST-SSA-SSB-SSAB
	\end{cases}
\end{equation*}

\subsection{多重比较比较问题}
如果固定因子A显著,则可使用Duncan多重比较法,但此时需要注意拒绝域应修改为如下形式:
\begin{equation*}
	A:R_p>r_{1-\alpha}(p,f)\sqrt{\frac{MSAB}{bm}}
\end{equation*}

\subsection{参数估计}
\subsubsection{点估计}
下给出混合模型参数的点估计:
\begin{gather*}
	\hat{\mu}=\bar{y}_{...},\quad\hat{\tau}_i=\bar{y}_{i..}-\bar{y}_{...},\quad i=1,2,\dots,a \\
	\hat{\sigma^2}=MSe,\quad\hat{\sigma^2}_\beta=\frac{MSB-MSe}{am},\quad\hat{\sigma^2}_{\tau\beta}=\frac{MSAB-MSe}{m}
\end{gather*}
\chapter{部分实施问题}
前几节介绍的方法都要求每个水平组合至少做一次试验,如果考虑因子间的交互作用,对每个水平组合还要做重复试验。这就导致当因子数较多或者因子水平书较多时,试验总次数会非常多。这就提出了一个问题:如何做到只对全部水平组合的一部分做试验,也能比较因子的各水平对试验结果是否有显著差异、不同因子间是否有交互作用?这就是全因子试验的部分实施问题。

\section{正交拉丁方设计}
本节介绍一种在无交互效应、各因子水平数相等情况下的部分实施问题的解决方案。设每个因子有$n$个水平,一共有$k$个因子。
\subsection*{原理}
要使得在对任一因子的效应作比较时能够消除其它因子水平变动对数据的影响,就需要保证:在任一因子的任一水平下,其它因子的每个水平都重复相同次数。此时称这些因子彼此之间正交。
\begin{definition}
	由$p$个不同符号排成的$p$阶方阵中,如果每行的$p$个元素不同,每列的$p$个元素也不同,则称这个方阵为一个$p$阶拉丁方。如果两个$p$阶拉丁方重叠时,第一个拉丁方中的任一元素与第二个拉丁方中的每个元素都相遇且只相遇一次,则称这一对拉丁方相互正交。
\end{definition}
下面的第一个方阵就是一个$3$阶拉丁方,第二个矩阵表示一对正交拉丁方重叠。
\begin{equation*}
	\begin{aligned}
		&\begin{pmatrix}
			A & B & C \\
			B & C & A \\
			C & A & B
		\end{pmatrix}
		\quad &\quad
		&\begin{pmatrix}
			A\alpha & B\beta & C\gamma \\
			B\gamma & C\alpha & A\beta \\
			C\beta & A\gamma & B\alpha
		\end{pmatrix}
	\end{aligned}
\end{equation*}
我们可以发现,拉丁方中的元素作为某一个因子的不同水平时,互相正交的拉丁方满足因子间正交。由此产生了正交拉丁方设计。
\begin{table}[H]
	\centering
	\begin{tabularx}{\textwidth}{c|>{\centering\arraybackslash}X>{\centering\arraybackslash}X>{\centering\arraybackslash}X>{\centering\arraybackslash}X}
		\hline
		\diagbox{行因子}{列因子} & $C_1$ & $C_2$ & $\cdots$ & $C_n$ \\
		\hline
		$R_1$ & $\cdots$ & $\cdots$ & $\cdots$ & $\cdots$ \\
		$R_2$ & $\cdots$ & $\cdots$ & $\cdots$ & $\cdots$ \\
		$\vdots$ & $\vdots$ & $\vdots$ & $\vdots$ & $\vdots$ \\
		$R_n$ & $\cdots$ & $\cdots$ & $\cdots$ & $\cdots$ \\
		\hline
	\end{tabularx}
	\caption{$k$因子各$n$水平正交拉丁方设计表}
\end{table}
上表中$\cdots$部分表示$k-2$个互相正交的拉丁方重叠后矩阵位置上的对应元素。因为拉丁方之间是正交的,所以不同的拉丁因子之间是正交的,而每个拉丁因子与行因子、列因子也是正交的,列因子与行因子显然正交,所以该试验方案可以被用作$k$因子各$n$水平的研究,其中需要做$n^2$次试验,每一次试验使用到的因子水平即为上表每一个单元中的$k-2$个拉丁方因子水平与其行列因子水平的组合。\par
三因子正交拉丁方设计又称拉丁方设计,四因子正交拉丁方设计又称希腊-拉丁方设计,涉及到四个以上因子的正交拉丁方设计称为超方设计。下对拉丁方设计与希腊-拉丁方设计做详细介绍。

\subsection{拉丁方设计}
拉丁方设计命名的由来是因为其中拉丁方的元素用拉丁字母来表示。
\begin{table}[H]
	\centering
	\begin{tabularx}{\textwidth}{c|>{\centering\arraybackslash}X>{\centering\arraybackslash}X>{\centering\arraybackslash}X}
		\hline
		\diagbox{\text{行因子}}{\text{列因子}} & $C_1$ & $C_2$ & $C_3$ \\ 
		\hline
		$R_1$ & $A$ & $B$ & $C$ \\ 
		$R_2$ & $B$ & $C$ & $A$ \\ 
		$R_3$ & $C$ & $A$ & $B$ \\ 
		\hline
	\end{tabularx}
	\caption{拉丁方设计表}
\end{table}
\subsubsection{统计模型}
\begin{equation*}
	\begin{cases}
		y_{ijk}=\mu+\alpha_i+\tau_j+\beta_k+\varepsilon_{ijk} \\
		\text{诸}\varepsilon_{ijk}\quad\mathrm{i.i.d.~}N(0,\sigma^2) \\
		s.t.\quad\sum\limits_{i=1}^p\alpha_i=0,\quad\sum\limits_{j=1}^p\tau_j=0,\quad\sum\limits_{k=1}^p\beta_k=0 \\
		i=1,2,\dots,p,\;j=1,2,\dots,p,\;k=1,2,\dots,p
	\end{cases}
\end{equation*}
其中$y_{ijk}$是在行因子第$i$个水平、列因子第$k$个水平和拉丁因子第$j$个水平下试验的观察值。$\mu$为一般平均,$\alpha_i$是行因子第$i$个水平的效应,$\tau_j$是拉丁因子第$j$个水平的效应,$\beta_k$是列因子第$k$个水平的效应。需要注意的是,因为拉丁方设计的缘故,三个下标之间不是独立的。
\subsubsection{方差分析}
\begin{equation*}
	SST=SS_{\text{拉丁}}+SS_{\text{行}}+SS_{\text{列}}+SSe
\end{equation*}
\begin{table}[H] 
	\centering
	\begin{tabularx}{\textwidth}{c|>{\centering\arraybackslash}X>{\centering\arraybackslash}X>{\centering\arraybackslash}X>{\centering\arraybackslash}X}
		\toprule
		来源   & 平方和 & 自由度 & 均方和 & $F$ 值 \\ 
		\midrule
		拉丁因子 & $SS_\text{拉丁}$ & $p-1$ & $MS_\text{拉丁}= \frac{SS_\text{拉丁}}{p-1}$ & $F = \frac{MS_\text{拉丁}}{MS_e}$ \\ 
		行因子   & $SS_\text{行}$ & $p-1$ & $MS_\text{行} = \frac{SS_\text{行}}{p-1}$ & $F = \frac{MS_\text{行}}{MS_e}$ \\ 
		列因子   & $SS_\text{列}$ & $p-1$ & $MS_\text{列}=\frac{SS_\text{列}}{p-1}$ & $F = \frac{MS_\text{列}}{MS_e}$ \\ 
		误差     & $SS_e$ & $(p-2)(p-1)$ & $MS_e = \frac{SS_e}{(p-2)(p-1)}$ & \\ 
		总和     & $SS_T$ & $p^2-1$ & & \\ 
		\bottomrule
	\end{tabularx}
	\caption{拉丁方设计方差分析表}
\end{table}
其中:
\begin{equation*}
	\begin{cases}
		SST=\sum\limits_{i=1}^p\sum\limits_{j=1}^p\sum\limits_{k=1}^py_{ijk}^2-\frac{y_{...}^2}{p^2} \\
		SS_\text{行}=\sum\limits_{i=1}^p\frac{y_{i..}^2}{p}-\frac{y_{...}^2}{p^2} \\
		SS_\text{列}=\sum\limits_{k=1}^p\frac{y_{..k}^2}{p}-\frac{y_{...}^2}{p^2} \\
		SS_\text{拉丁}=\sum\limits_{j=1}^p\frac{y_{.j.}^2}{p}-\frac{y_{...}^2}{p^2} \\
		SSe=SST-SS_\text{行}-SS_\text{列}-SS_\text{拉丁}
	\end{cases}
\end{equation*}
\subsubsection{多重比较问题}
此时的Duncan多重比较过程需要注意:
\begin{equation*}
	R_p>r_{1-\alpha}(p,f)\sqrt{\frac{MSe}{p}}
\end{equation*}

\subsection{希腊拉丁方设计}
希腊-拉丁方设计命名的由来是因为其中拉丁方的元素分别用拉丁字母和希腊字母来表示。
\begin{table}[H]
	\centering
	\begin{tabularx}{\textwidth}{c|>{\centering\arraybackslash}X>{\centering\arraybackslash}X>{\centering\arraybackslash}X}
		\hline
		\diagbox{\text{行因子}}{\text{列因子}} & $C_1$ & $C_2$ & $C_3$ \\  
		\hline
		$R_1$ & $A\alpha$ & $B\beta$ & $C\gamma$ \\ 
		$R_2$ & $B\gamma$ & $C\alpha$ & $A\beta$ \\ 
		$R_3$ & $C\beta$ & $A\gamma$ & $B\alpha$ \\ 
		\hline
	\end{tabularx}
	\caption{希腊-拉丁方设计表}
\end{table}
\subsubsection{统计模型}
\begin{equation*}
	\begin{cases}
		y_{ijkl}=\mu+\alpha_i+\tau_j+\phi_k+\beta_l+\varepsilon_{ijk} \\
		\text{诸}\varepsilon_{ijkl}\quad\mathrm{i.i.d.~}N(0,\sigma^2) \\
		s.t.\quad\sum\limits_{i=1}^p\alpha_i=0,\quad\sum\limits_{j=1}^p\tau_j=0,\quad\sum\limits_{k=1}^p\pi_k=0,\quad\sum\limits_{l=1}^p\beta_l=0 \\
		i=1,2,\dots,p,\;j=1,2,\dots,p,\;k=1,2,\dots,p,\;l=1,2,\dots,p
	\end{cases}
\end{equation*}
其中$y_{ijkl}$是在行因子第$i$个水平、列因子第$l$个水平、拉丁因子第$j$个水平和希腊因子第$k$个水平下试验的观察值。$\mu$为一般平均,$\alpha_i$是行因子第$i$个水平的效应,$\tau_j$是拉丁因子第$j$个水平的效应,$\phi_k$是希腊因子第$k$个水平的效应,$\beta_l$是列因子第$l$个水平的效应。需要注意的是,因为希腊-拉丁方设计的缘故,四个下标之间不是独立的。
\subsubsection{方差分析}
\begin{equation*}
	SST=SS_{\text{拉丁}}+SS_{\text{希腊}}+SS_{\text{行}}+SS_{\text{列}}+SSe
\end{equation*}
\begin{table}[H] 
	\centering
	\begin{tabularx}{\textwidth}{c|>{\centering\arraybackslash}X>{\centering\arraybackslash}X>{\centering\arraybackslash}X>{\centering\arraybackslash}X}
		\toprule
		来源   & 平方和 & 自由度 & 均方和 & $F$ 值 \\ 
		\midrule
		拉丁因子 & $SS_\text{拉丁}$ & $p-1$ & $MS_\text{拉丁}= \frac{SS_\text{拉丁}}{p-1}$ & $F = \frac{MS_\text{拉丁}}{MS_e}$ \\ 
		希腊因子 & $SS_\text{希腊}$ & $p-1$ & $MS_\text{希腊}= \frac{SS_\text{希腊}}{p-1}$ & $F = \frac{MS_\text{希腊}}{MS_e}$ \\ 
		行因子   & $SS_\text{行}$ & $p-1$ & $MS_\text{行} = \frac{SS_\text{行}}{p-1}$ & $F = \frac{MS_\text{行}}{MS_e}$ \\ 
		列因子   & $SS_\text{列}$ & $p-1$ & $MS_\text{列}=\frac{SS_\text{列}}{p-1}$ & $F = \frac{MS_\text{列}}{MS_e}$ \\ 
		误差     & $SS_e$ & $(p-3)(p-1)$ & $MS_e = \frac{SS_e}{(p-3)(p-1)}$ & \\ 
		总和     & $SS_T$ & $p^2-1$ & & \\ 
		\bottomrule
	\end{tabularx}
	\caption{拉丁方设计方差分析表}
\end{table}
其中:
\begin{equation*}
	\begin{cases}
		SST=\sum\limits_{i=1}^p\sum\limits_{j=1}^p\sum\limits_{k=1}^p\sum\limits_{l=1}^py_{ijkl}^2-\frac{y_{....}^2}{p^2} \\
		SS_\text{行}=\sum\limits_{i=1}^p\frac{y_{i...}^2}{p}-\frac{y_{....}^2}{p^2} \\
		SS_\text{列}=\sum\limits_{k=1}^p\frac{y_{...l}^2}{p}-\frac{y_{....}^2}{p^2} \\
		SS_\text{拉丁}=\sum\limits_{j=1}^p\frac{y_{.j..}^2}{p}-\frac{y_{....}^2}{p^2} \\
		SS_\text{希腊}=\sum\limits_{k=1}^p\frac{y_{..k.}^2}{p}-\frac{y_{....}^2}{p^2} \\
		SSe=SST-SS_\text{行}-SS_\text{列}-SS_\text{拉丁}-SS_\text{希腊}
	\end{cases}
\end{equation*}
\subsubsection{多重比较问题}
此时的Duncan多重比较过程需要注意:
\begin{equation*}
	R_p>r_{1-\alpha}(p,f)\sqrt{\frac{MSe}{p}}
\end{equation*}

\section{正交表设计}
设一个试验问题有$k$个因子,每个因子有$n$个水平(分别称为$0$水平,$1$水平,……,$n-1$水平),全部水平组合有$n^k$个。本节简要讨论当$n$为素数时,用$n$水平正交表实现它的部分实施的方法。
\begin{theorem}
	当$n$为素数时,存在$n-1$个相互正交的$n$阶拉丁方。
\end{theorem}
由上述组合数学中的定理,$n^k$设计中任意两个因子的交互作用,例如$AB$,都快可以被分解为$n-1$个分量,即$AB$分量、$A^2B$分量,……,$A^{n-1}B$分量,两因子交互作用的自由度为$(n-1)^2$,每个分量的自由度为$n-1$
\begin{definition}
	如果一个矩阵满足下述条件:
	\begin{enumerate}
		\item 任意一列中不同数字的重复数相等。
		\item 任意两列同行数字构成若干数对,每个数对的重复数也相等。
	\end{enumerate}
	则称其为一个正交表,记为$L_r(n^c)$,其中$L$为正交表符号,$r$表示正交表行数,$c$表示正交表列数,$n$表示正交表中不同数字的个数。
\end{definition}
任意两列的交互作用列是表中另外一列,列名相乘时用指数法则模2取余。交互效应列的数字由主效应列相乘得到。
同行主效应列构成的数组代表一个试验点,也代表着因子的主效应的估计量的对比的代数符号。
