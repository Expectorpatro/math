\section{随机效应下的两因子方差分析}
仅讨论等重复、有交互作用情形。
\begin{table}[H] 
	\centering
	\begin{tabularx}{\textwidth}
		{c|>{\centering\arraybackslash}X>{\centering\arraybackslash}Xc>{\centering\arraybackslash}X}
		\hline
		\diagbox{因子$A$}{因子$B$} & $B_1$ & $B_2$ & $\cdots$ & $B_b$ \\ \hline
		$A_1$ & 
		$y_{111}, y_{112}, \dots, y_{11m}$ & 
		$y_{121}, y_{122}, \dots, y_{12m}$ & 
		$\cdots$ & 
		$y_{1b1}, y_{1b2}, \dots, y_{1bm}$ \\ 
		$A_2$ & 
		$y_{211}, y_{212}, \dots, y_{21m}$ & 
		$y_{221}, y_{222}, \dots, y_{22m}$ & 
		$\cdots$ & 
		$y_{2b1}, y_{2b2}, \dots, y_{2bm}$ \\
		$\vdots$ & 
		$\vdots$ & 
		$\vdots$ & 
		& 
		$\vdots$ \\
		$A_a$ & 
		$y_{a11}, y_{a12}, \dots, y_{a1m}$ & 
		$y_{a21}, y_{a22}, \dots, y_{a2m}$ & 
		$\cdots$ & 
		$y_{ab1}, y_{ab2}, \dots, y_{abm}$ \\ 
		\hline
	\end{tabularx}
	\caption{等重复两因子试验数据表}
\end{table}
其中$y_{ijk}$表示在因子A的第$i$个水平$A_i$和因子B的第$j$个水平$B_j$下第$k$次重复试验的观察值。

\subsection{统计模型}
此时的统计模型为:
\begin{equation*}\label{model:random-effect-two-way-anova}
	\begin{cases}
		y_{ijk}=\mu+\tau_i+\beta_j+(\tau\beta)_{ij}+\varepsilon_{ijk} \\
		\text{诸}\varepsilon_{ijk}\quad\mathrm{i.i.d.~}N(0,\sigma^2) \\
		\text{诸}\tau_i\quad\mathrm{i.i.d.~}N(0,\sigma_\tau^2) \\
		\text{诸}\beta_j\quad\mathrm{i.i.d.~}N(0,\sigma_\beta^2) \\
		\text{诸}(\tau\beta)_{ij}\quad\mathrm{i.i.d.~}N(0,\sigma_{\tau\beta}^2) \\
		\text{诸}\varepsilon_{ijk}\text{、诸}\tau_i\text{、诸}\beta_j\text{、诸}(\tau\beta)_{ij}\text{相互独立} \\
		i=1,2,\dots,a,\;j=1,2,\dots,b,\;k=1,2,\dots,m
	\end{cases}
\end{equation*}

\subsection{统计假设}
此时需要检验如下三个零假设:
\begin{equation*}
	\begin{cases}
		H_{01}:\sigma_\tau^2=0, \\
		H_{02}:\sigma_\beta^2=0 \\
		H_{03}:\sigma_{\tau\beta}^2=0
	\end{cases}
\end{equation*}

\subsection{方差分析}
\subsubsection{偏差平方和的分解}
由于与交互效应情况下五个平方和的定义完全相同,所以随机效应下偏差平方和分解公式与之前一模一样。
\begin{gather*}
	SST=SSA+SSB+SSAB+SSe \\
	SSA=bm\sum_{i=1}^a(\bar{y}_{i..}-\bar{y}_{...})^2=bm\sum_{i=1}^a(\tau_i+\bar{\varepsilon}_{i..}-\bar{\varepsilon}_{...})^2 \\
	SSB=am\sum_{j=1}^b(\bar{y}_{.j.}-\bar{y}_{...})^2=am\sum_{j=1}^b(\beta_j+\bar{\varepsilon}_{.j.}-\bar{\varepsilon}_{...})^2 \\
	SSAB=m\sum_{i=1}^a\sum_{j=1}^b(\bar{y}_{ij.}-\bar{y}_{i..}-\bar{y}_{.j.}+\bar{y}_{...})^2=m\sum_{i=1}^a\sum_{j=1}^b\left[(\tau\beta)_{ij}+\bar{\varepsilon}_{ij.}-\bar{\varepsilon}_{i..}-\bar{\varepsilon}_{.j.}+3\bar{\varepsilon}_{...}\right]^2 \\
	SSe=\sum_{i=1}^a\sum_{j=1}^b\sum_{k=1}^m(y_{ijk}-\bar{y}_{ij.})^2=\sum_{i=1}^a\sum_{j=1}^b\sum_{k=1}^m(\varepsilon_{ijk}-\bar{\varepsilon}_{ij.})^2
\end{gather*}
\subsubsection{各平方和的期望}
这里直接列出各平方和的期望,证明是容易的。
\begin{gather*}
	E(SSA)=(a-1)\sigma^2+m(a-1)\sigma_{\tau\beta}^2+(a-1)bm\sigma_\tau^2  \\
	E(SSB)=(b-1)\sigma^2+m(b-1)\sigma_{\tau\beta}^2+(b-1)am\sigma_{\beta}^2 \\
	E(SSAB)=(a-1)(b-1)\sigma^2+m(a-1)(b-1)\sigma_{\tau\beta}^2 \\
	E(SSe)=ab(m-1)\sigma^2
\end{gather*}
\subsubsection{统计量及其分布}
称$\frac{SSA}{a-1}$为因子A的均方和,记为MSA;称$\frac{SSB}{b-1}$为因子B的均方和,记为MSB;称$\frac{SSAB}{(a-1)(b-1)}$为因子A与因子B的交互作用的均方和,记为MSAB;称$\frac{SSe}{ab(m-1)}$为误差均方和,记为MSe。\par
由前述,MSe是$\sigma^2$的无偏估计。在$H_{01}$成立时,MSA是$\sigma^2+m\sigma_{\tau\beta}^2$的无偏估计;在$H_{02}$成立时,MSA是$\sigma^2+m\sigma_{\tau\beta}^2$的无偏估计;在$H_{03}$成立时,MSAB是$\sigma^2$的无偏估计。如果MSA与MSAB比值很大,则有理由怀疑$H_01$;如果MSB与MSAB比值很大,则有理由怀疑$H_02$;如果MSAB与MSe比值很大,则有理由怀疑$H_03$。由此构建统计量:
\begin{gather*}
	F_A=\frac{MSA}{MSAB}=\frac{\frac{SSA}{a-1}}{\frac{SSAB}{(a-1)(b-1)}} \\
	F_B=\frac{MSB}{MSAB}=\frac{\frac{SSB}{b-1}}{\frac{SSAB}{(a-1)(b-1)}} \\
	F_{AB}=\frac{MSAB}{MSe}=\frac{\frac{SSAB}{(a-1)(b-1)}}{\frac{SSe}{ab(m-1)}}
\end{gather*}
在这些统计量的情况下,$H_{01},\;H_{02},\;H_{03}$的拒绝域是右向单尾的。\par
由于在假设$H_{01},\;H_{02},\;H_{03}$成立时,随机效应模型与可加效应模型的观察值$y_{ijk}$的数据结构的形式完全一样,所以在随机效应模型中仍然有:
\begin{gather*}
	F_A\sim F(a-1,\;(a-1)(b-1)) \\
	F_B\sim F(b-1,\;(a-1)(b-1)) \\
	F_{AB}\sim F((a-1)(b-1),\;ab(m-1))
\end{gather*}
\subsubsection{拒绝域}
综上所述,显著性水平为$\alpha$时的拒绝域为:
\begin{gather*}
	F_A>F_{1-\alpha}(a-1,\;(a-1)(b-1)) \\
	F_B>F_{1-\alpha}(b-1,\;(a-1)(b-1)) \\
	F_{AB}>F_{1-\alpha}((a-1)(b-1),\;ab(m-1))
\end{gather*}

\subsection{方差分析表}
\begin{table}[H]
	\centering
	\begin{tabularx}{\textwidth}
		{>{\centering\arraybackslash}c|*{5}{>{\centering\arraybackslash}X}}
		\toprule
		来源   &平方和&自由度&均方和             &F值  \\ 
		\midrule
		因子A&SSA&$f_A=a-1$ &$\frac{SSA}{a-1}$ &$F=\frac{MSA}{MSAB}$\\
		因子B&SSB&$f_B=b-1$ &$\frac{SSB}{b-1}$ &$F=\frac{MSB}{MSAB}$\\
		交互作用AB &SSAB &$f_{AB}=(a-1)(b-1)$ &$\frac{SSAB}{(a-1)(b-1)}$ &$F=\frac{MSAB}{MSe}$ \\
		误差   &SSe  &$f_e=ab(m-1)$ &$\frac{SSe}{ab(m-1)}$ & \\
		总     &SST  &$f_T=abm-1$ &                  & \\
		\bottomrule
	\end{tabularx}
	\caption{等重复、有交互作用的随机效应下两因子试验方差分析表}
\end{table}
平方和公式可按下列公式计算:
\begin{equation*}
	\begin{cases}
		SST=\sum\limits_{i=1}^a\sum\limits_{j=1}^b\sum\limits_{k=1}^my_{ijk}^2-\frac{y_{...}^2}{abm} \\
		SSA=\sum\limits_{i=1}^a\frac{y_{i..}^2}{bm}-\frac{y_{...}^2}{abm} \\
		SSB=\sum\limits_{j=1}^b\frac{y_{.j.}^2}{am}-\frac{y_{...}^2}{abm} \\
		SSAB=\sum\limits_{i=1}^a\sum\limits_{j=1}^b\frac{y_{ij.}^2}{m}-\frac{y_{...}^2}{abm}-SSA-SSB \\
		SSe=SST-SSA-SSB-SSAB
	\end{cases}
\end{equation*}

\subsection{参数估计}
我们此时关心方差分量的估计。
\subsubsection{点估计}
由SSA、SSB、SSAB、SSe期望的计算,可给出各方差分量的无偏点估计如下:
\begin{gather*}
	\hat{\sigma^2}=MSe \\
	\hat{\sigma^2}_\tau=\frac{MSA-MSAB}{bm} \\
	\hat{\sigma^2}_\beta=\frac{MSB-MSAB}{am} \\
	\hat{\sigma^2}_{\tau\beta}=\frac{MSAB-MSe}{m}
\end{gather*}