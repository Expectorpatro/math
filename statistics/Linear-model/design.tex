\section{最优设计}
在之前的讨论中,我们都是在设计阵$X$已知的情况下去对模型的统计性质进行讨论,但理论表明我们可以选择试验点(即选择$(X,y)$,$(X,y)$的一行被称之为一个试验点)来让模型进一步获得某些优良性质,这种方法被称之为\textbf{最优设计}。下面给出设计的模型。

\begin{definition}\label{model:design}
	设$\mathcal{X}\in\mathbb{R}^{p}$是\textbf{试验点集},试验时只能在$\mathcal{X}$内选取试验点,试验点记为$z$。假设:
	\begin{enumerate}
		\item $\mathcal{X}$是一个紧集,由\info{$\mathbb{R}^{n}$上的紧集等价于有界闭集}可知该条件等价于$\mathcal{X}$是一个有界闭集;
		\item $f_1(x),f_2(x),\dots,f_p(x)$是$\mathcal{X}$上已知的$p$个线性无关的连续函数,记:
		\begin{equation*}
			f(x)=[f_1(x),f_2(x),\dots,f_p(x)]^T
		\end{equation*}
		\item 自变量$x$与响应变量$y$满足:
		\begin{equation*}
			y(x)=f^T(x)\beta+\varepsilon,\quad\varepsilon\sim\operatorname{N}(0,\sigma^2)
		\end{equation*}
	\end{enumerate}
\end{definition}
\subsection{信息矩阵}
\begin{definition}
	对于\cref{model:design},称试验点集$\mathcal{X}$上的任一概率分布$\mu$为一个\textbf{设计},定义其\textbf{信息矩阵}为:
	\begin{equation*}
		M(\mu)=\int_{\mathcal{X}}f(x)f^T(x)\dif\mu
	\end{equation*}
	若$\mu$是一个取值个数有限的离散型概率分布,$n\in\mathbb{N}^+$为$\mu$的取值数,则记:
	\begin{equation*}
		\mu=
		\begin{pmatrix}
			z_1 & z_2 & \cdots & z_n \\
			p_1 & p_2 & \cdots & p_n
		\end{pmatrix}
	\end{equation*}
	称$\seq{z}{n}$为设计$\mu$的\textbf{谱点}。
\end{definition}
\begin{property}
	对于\cref{model:design},信息矩阵具有下述性质:
	\begin{enumerate}
		\item 任一设计$\mu$的信息矩阵$M(\mu)$都是半正定的;
		\item 若设计:
		\begin{equation*}
			\mu=
			\begin{pmatrix}
				z_1 & z_2 & \cdots & z_n \\
				p_1 & p_2 & \cdots & p_n
			\end{pmatrix}
		\end{equation*}
		的谱点数$n<p$,则$M(\mu)$是退化矩阵;
		\item $M(\Xi)$是一个闭凸集;
		\item 对于任意设计$\mu$,存在离散设计:
		\begin{equation*}
			\mu_0=
			\begin{pmatrix}
				z_1 & z_2 & \cdots & z_n \\
				p_1 & p_2 & \cdots & p_n
			\end{pmatrix}
		\end{equation*}
		使得$M(\mu)=M(\mu_0)$,其中:
		\begin{equation*}
			n\leqslant\frac{p(p+1)}{2}+1
		\end{equation*}
	\end{enumerate}
\end{property}
\begin{proof}
	(1)任取设计$\mu$,由信息矩阵的定义,$M(\mu)$是对称矩阵。由\cref{prop:NonnegativeMeasurableIntegral}(2)(10)可知对任意的$z\in\mathbb{R}^{p}$有:
	\begin{equation*}
		z^TM(\mu)z=z^T\int_{\mathcal{X}}f(x)f^T(x)\dif\mu z=\int_{\mathcal{X}}z^Tf(x)f^T(x)z\dif\mu=\int_{\mathcal{X}}[f^T(x)z]^2\dif\mu\geqslant0
	\end{equation*}
	所以$M(\mu)$是半正定矩阵。由$\mu$的任意性即可得出结论。\par
	(2)此时有:
	\begin{equation*}
		M(\mu)=\sum_{i=1}^{n}p_if(z_i)f^T(z_i)=
		\begin{pmatrix}
			p_1 & 0 & \cdots & 0 \\
			0 & p_2 & \cdots & 0 \\
			\vdots & \vdots & \ddots & \vdots \\
			0 & 0 & \cdots & p_n 
		\end{pmatrix}
		[f(z_1),f(z_2),\dots,f(z_n)]
		\begin{pmatrix}
			f^T(z_1) \\
			f^T(z_2) \\
			\vdots \\
			f^T(z_n)
		\end{pmatrix}
	\end{equation*}
	因为$n<p$,所以$\operatorname{rank}\{[f(z_1),f(z_2),\dots,f(z_n)]\}\leqslant n$,而由\cref{prop:MatrixRank}(9)可得:
	\begin{equation*}
		\operatorname{rank}[M(\mu)]\leqslant\operatorname{rank}\{[f(z_1),f(z_2),\dots,f(z_n)]\}\leqslant n<p
	\end{equation*}
	所以$M(\mu)$是退化矩阵。\par
	(3)对任意的$\mu_1,\mu_2\in\Xi$,若$\alpha\in[0,1]$,则由\cref{prop:MeasurableIntegral}(5)可得:
	\begin{align*}
		\alpha M(\mu_1)+(1-\alpha)M(\mu_2)&=\alpha\int_{\mathcal{X}}f(x)f^T(x)\dif\mu_1+(1-\alpha)\int_{\mathcal{X}}f(x)f^T(x)\dif\mu_2 \\
		&=\int_{\mathcal{X}}\alpha f(x)f^T(x)\dif\mu_1+\int_{\mathcal{X}}(1-\alpha)f(x)f^T(x)\dif\mu_2 \\
		&=\int_{\mathcal{X}}f(x)f^T(x)\dif(\alpha\mu_1)+\int_{\mathcal{X}}f(x)f^T(x)\dif[(1-\alpha)\mu_2] \\
		&=\int_{\mathcal{X}}f(x)f^T(x)\dif[\alpha\mu_1+(1-\alpha)\mu_2] \\
		&=M[\alpha\mu_1+(1-\alpha)\mu_2]
	\end{align*}
	因为$\alpha\mu_1+(1-\alpha)\mu_2\in\Xi$,所以$M[\alpha\mu_1+(1-\alpha)\mu_2]\in M(\Phi)$,于是$M(\Phi)$是一个凸集。\info{仍需证明是一个闭集}\par
\end{proof}
\subsection{优良性准则}
\begin{definition}
	对于\cref{model:design},设$\Xi=\{\mu:\mu\text{是$\mathcal{X}$上的概率分布}\},\;M(\Xi)=\{M(\mu):\mu\in\Xi\}$,$\Phi$是定义在$M(\Xi)$上的一个\textbf{准则函数}。若设计$\mu^*\in\Xi$满足:
	\begin{equation*}
		\Phi[M(\mu^*)]=\inf_{\mu\in\Xi}\Phi[M(\mu)]
	\end{equation*}
	则称设计$\mu^*$为$\Phi$最优设计。
\end{definition}
\subsubsection{D最优性}
\begin{definition}
	对于\cref{model:design},设定义在$M(\Xi)$上的函数$\Phi[M(\mu)]=-\ln[\det M(\mu)]$。若设计$\mu^*$满足:
	\begin{equation*}
		\Phi[M(\mu^*)]=\inf_{\mu\in\Xi}-\ln[\det M(\mu)]
	\end{equation*}
	则称$\mu^*$为D最优设计。
\end{definition}
\begin{note}
	可以证明在参数$\beta$下\cref{theo:NormalLinearModelConfidenceEllipsoid}中的置信椭球体积为:
	\begin{equation*}
		V(\mu)=\frac{(p+2)^{k/2}\pi^{k/2}}{\Gamma\left(\frac{k}{2}+1\right)\sqrt{\det M(\mu)}}
	\end{equation*}
	所以D最优设计的意义即为使得置信椭球的体积最小。
\end{note}
\subsubsection{G最优性}
\begin{definition}
	对于\cref{model:design},设定义在$M(\Xi)$上的函数:
	\begin{equation*}
		\Phi[M(\mu)]=
		\begin{cases}
			\max\limits_{x\in\mathcal{X}}d(x,\mu)=\max\limits_{x\in\mathcal{X}}f(x)^TM^{-1}(\mu)f(x),&\det M(\mu)\ne0 \\
			+\infty,&\det M(\mu)=0
		\end{cases}
	\end{equation*}
	若设计$\mu^*$满足:
	\begin{equation*}
		\Phi[M(\mu^*)]=\inf_{\mu\in\Xi}\Phi[M(\mu)]
	\end{equation*}
	则称$\mu^*$为G最优设计。
\end{definition}
\begin{note}
	当利用\cref{model:design}进行预测时,在信息矩阵$M(\mu)$非退化的情况下,由\cref{prop:NormalLinearModel}(2)可知点$x$处预测值$\hat{y}(x)$的方差等于:
	\begin{equation*}
		\operatorname{Var}[\hat{y}(x)]=\sigma^2f^T(x)M^{-1}(\mu)f(x)=d(x,\mu)
	\end{equation*}
	所以$G$最优设计的意义即为使得在试验点集中方差最大的点的方差最小。
\end{note}
\subsubsection{D最优与G最优的等价性}
\begin{lemma}
	对于\cref{model:design},有如下结论:
	\begin{enumerate}
		\item 设$\mu\in\Phi$且$\det M(\mu)\ne0$,则:
		\begin{equation*}
			\operatorname{E}[d(x,\mu)]=\int_{\mathcal{X}}d(x,\mu)\dif\mu=p,\quad\max_{x\in\mathcal{X}}d(x,\mu)\geqslant p
		\end{equation*}
		\item $\ln[\det M(\mu)]$作为$M(\Phi)$上的函数是一个凹函数;
		\item 设设计$\mu_1,\mu_2$的信息矩阵分别为$M(\mu_1),M(\mu_2)$,$0<\alpha<1$。令$\mu=\alpha\mu_1+(1-\alpha)\mu_2$,则:
		\begin{equation*}
			\frac{\dif\ln[\det M(\mu)]}{\dif\alpha}=\operatorname{tr}\{M^{-1}(\mu)[M(\mu_1)-M(\mu_2)]\}
		\end{equation*}
		\item 设$\mu$是任一设计,$\mu_{x_0}$表示只有一个谱点$x_0$的设计,$0<\alpha<1$,则对设计$\mu_0=\alpha\mu_{x_0}+(1-\alpha)\mu$有:
		\begin{equation*}
			\det M(\mu_0)=(1-\alpha)^p\left[1+\frac{\alpha}{1-\alpha}d(x_0,\mu)\right]\det M(\mu)
		\end{equation*}
	\end{enumerate}
\end{lemma}
\begin{theorem}
	对于\cref{model:design},下述三个结论等价:
	\begin{enumerate}
		\item $\mu$是D最优设计;
		\item $\mu$是G最优设计;
		\item $\mu^*$满足$\max\limits_{x\in\mathcal{X}}d(x,\mu^*)=p$;
	\end{enumerate}
	分别满足上述条件(1)、(2)、(3)的三个信息矩阵类彼此相同,满足上述条件(1)、(2)、(3)的设计的系数和为$1$的线性组合也满足条件(1)、(2)、(3)。
\end{theorem}