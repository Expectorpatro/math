\section{数值计算}

本节介绍线性模型的实际数值计算问题,即在2025年的今天,我们在计算机中到底使用着怎样的方法去求解\cref{model:LinearModel}中的参数。

\subsection{QR分解}
对于\cref{model:LinearModel},其最小二乘解满足:
\begin{equation*}
	\hat{\beta}=\arg\min_{\beta}||y-X\beta||^2
\end{equation*}\par
\textbf{$X$可逆时:}对$X$作QR分解:
\begin{equation*}
	||y-X\beta||^2=||y-QR\beta||^2=[Q^T(y-QR\beta)]^TQ^T(y-QR\beta)=||Q^Ty-R\beta||
\end{equation*}
因为$X$可逆,所以此时$R$是可逆的,由\info{可逆矩阵满秩}和\cref{theo:SolutionOfSLE2}可知存在$\beta\in\mathbb{R}^{p}$使得$R\beta=Q^Ty$。因为$R$是一个上三角矩阵,所以通过\info{Gaussian迭代法}可方便地计算得到$\beta$。\par
\textbf{$X$不可逆时:}此时$R$必然可以呈现为前$\operatorname{rank}(X)$行构成上三角矩阵后$n-\operatorname{rank}(X)$行为零向量,即对角元素为$0$,所以此时$R$可以分解为:
\begin{equation*}
	R=
	\begin{pmatrix}
		R_1 & R_2 \\
		\mathbf{0} & \mathbf{0}
	\end{pmatrix}
\end{equation*}
这样的形式,其中$R_1\in M_{n-\operatorname{rank}(X)}(\mathbb{R}^{})$是上三角矩阵,只需对此时的$Q$矩阵也进行分块就可以将问题转化为可逆时的情况。我们得想办法让$R$矩阵呈现出上面的样子,由\cref{theo:Householder}和\cref{theo:QR}中的内容,只需通过矩阵$P$作变换$XP$($P$调换$X$中列向量之间的顺序)使得$XP$的第一列为$X$的列范数最大的列、$XP$的第二列为$X$去掉第一行后列范数最大的列……即可,这是由于我们选取的Householder变换让向量变为第一个元素为该向量的范数其它分量都为$0$的列向量,且Householder变换不改变后面向量的长度,这样一来就可以使得$XP$进行$QR$分解后得到的$R$的主对角元依次减小,主对角线为$0$的列对应的参数也无需再进行估计。\par
\begin{note}
	其实在实际情况中$X$都是不可逆的,$n$总是大于$p$的。
\end{note}

\subsection{SVD分解}