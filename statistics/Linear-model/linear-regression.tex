\section{线性回归模型}
\begin{definition}\label{model:LinearRegressionModel}
	设因变量$Y$和自变量$\seq{X}{p-1}$满足:
	\begin{equation*}
		Y=\beta_0+\beta_1X_1+\beta_2X_2+\cdots+\beta_{p-1}X_{p-1}+e
	\end{equation*}
	若对因变量$Y$和自变量$\seq{X}{p-1}$进行了$n$次观察,得到$n$组数据,它们满足:
	\begin{equation*}
		y_i=\beta_0+\beta_1x_{i1}+\beta_2x_{i2}+\cdots+\beta_{p-1}x_{ip-1}+\varepsilon_i,\quad i=1,2,\dots,n
	\end{equation*}
	记:
	\begin{equation*}
		y=
		\begin{pmatrix}
			y_1 \\
			y_2 \\
			\vdots \\
			y_n
		\end{pmatrix},\quad
		X=
		\begin{pmatrix}
			1 & x_{11} & \cdots & x_{1p-1} \\
			1 & x_{21} & \cdots & x_{2p-1} \\
			\vdots & \vdots & \ddots & \vdots \\
			1 & x_{n1} & \cdots & x_{np-1} \\
		\end{pmatrix},\quad
		\beta=
		\begin{pmatrix}
			\beta_0 \\
			\beta_1 \\
			\vdots \\
			\beta_{p-1}
		\end{pmatrix},\quad
		\varepsilon=
		\begin{pmatrix}
			\varepsilon_1 \\
			\varepsilon_2 \\
			\vdots \\
			\varepsilon_n
		\end{pmatrix}
	\end{equation*}
	且假设$\operatorname{rank}(X)=p,\;\operatorname{E}(\varepsilon)=\mathbf{0},\;\operatorname{Cov}(\varepsilon)=\sigma^2I_n$,则得到\gls{LinearRegressionModel}:
	\begin{equation*}
		y=X\beta+\varepsilon,\quad\operatorname{E}(\varepsilon)=\mathbf{0},\quad\operatorname{Cov}(\varepsilon)=\sigma^2I_n
	\end{equation*}
	称$\beta_0$为常数项,$\beta_I=(\seq{\beta}{p-1})^T$为回归系数。若记$X=(\mathbf{1}_n,\tilde{X})$,其中$\mathbf{1}_n$为由$n$个$1$构成的列向量,则线性回归模型可被改写为:
	\begin{equation*}
		y=\mathbf{1}_n\beta_0+\tilde{X}\beta_I+\varepsilon,\quad\operatorname{E}(\varepsilon)=\mathbf{0},\quad\operatorname{Cov}(\varepsilon)=\sigma^2I_n
	\end{equation*}
\end{definition}
\begin{derivation}
	由\cref{theo:AATPositiveSemidefinite}可知$X^TX$是一个半正定阵。若存在不为零向量的$\alpha$使得$\alpha^TX^TX\alpha=0$,则有$||X\alpha||=0$,即$X\alpha=\mathbf{0}$。而$\operatorname{rank}(X)=p$,所以$\alpha=\mathbf{0}$,矛盾,于是$X^TX$是一个正定阵,它是可逆的\info{正定阵可逆}。
\end{derivation}
\begin{definition}\label{model:NormalLinearRegressionModel}
	对于\cref{model:LinearRegressionModel},若$\varepsilon\sim\operatorname{N}_n(\mathbf{0},\sigma^2I_n)$,则称此时的线性回归模型为\textbf{正态线性回归模型}。
\end{definition}
\subsubsection{中心化与标准化处理}
\begin{definition}
	对于\cref{model:LinearRegressionModel},记:
	\begin{equation*}
		\bar{x}_j=\frac{1}{n}\sum_{i=1}^{n}x_{ij},\quad s_j^2=\sum_{i=1}^{n}(x_{ij}-\bar{x}_j)^2
	\end{equation*}\par
	称:
	\begin{gather*}
		y_i=\gamma_0+\beta_1(x_{i1}-\bar{x}_1)+\beta_2(x_{i2}-\bar{x}_2)+\cdots+\beta_{p-1}(x_{ip-1}-\bar{x}_{p-1})+\varepsilon_i \\
		y=\gamma_0\mathbf{1}_n+\tilde{X}_c\beta_I+\varepsilon,\quad\operatorname{E}(\varepsilon)=\mathbf{0},\quad\operatorname{Cov}(\varepsilon)=\sigma^2I_n
	\end{gather*}
	为\textbf{中心化线性回归模型},其中$\gamma_0=\beta_0+\bar{x}^T\beta_I,\;\bar{x}=(\seq{\bar{x}}{p-1})^T$,$\tilde{X}_c=\left(I_n-\dfrac{1}{n}\mathbf{1}_n\mathbf{1}_n^T\right)\tilde{X}$被称为\textbf{中心化设计阵},第二行是第一行的矩阵表示。\par
	称:
	\begin{gather*}
		y_i=\alpha_0+\beta_1^0\frac{x_{i1}-\bar{x}_1}{s_1}+\beta_2^0\frac{x_{i2}-\bar{x}_2}{s_2}+\cdots+\beta^0_{p-1}\frac{x_{ip-1}-\bar{x}_{p-1}}{s_{p-1}}+\varepsilon_i \\
		y=\alpha_0\mathbf{1}_n+Z\beta_S+\varepsilon,\quad\operatorname{E}(\varepsilon)=\mathbf{0},\quad\operatorname{Cov}(\varepsilon)=\sigma^2I_n
	\end{gather*}
	为\textbf{标准化线性回归模型},其中$\alpha_0=\beta_0,\beta_i^0=s_i\beta_i$,$Z$为经过中心化和标准化的设计阵,第二行是第一行的矩阵表示。
\end{definition}
\begin{theorem}\label{theo:CenterStandardLinearModel}
	对于中心化和标准化后的线性回归模型,设$\hat{\beta}=(\seq{\hat{\beta}}{p-1})$为正则方程的解,则有以下结论:
	\begin{enumerate}
		\item $\mathbf{1}_n^T\tilde{X_c}=\mathbf{0}$;
		\item $\hat{\gamma}_0=\bar{y}=\hat{\beta}_0+\bar{x}^T\hat{\beta}_I$;
		\item $\mathbf{1}_n^TZ=\mathbf{0}$;
		\item $R=Z^TZ$,其中$R$为回归自变量之间的相关系数矩阵;
		\item $\hat{\alpha}_0=\bar{y}$;
		\item $\hat{\beta}_i^0=s_i\hat{\beta}_i$;
		\item 三种模型的经验回归方程分别为:
		\begin{gather*}
			\text{一般模型:}\hat{Y}=\hat{\beta}_0+\hat{\beta}_1X_1+\hat{\beta}_2X_2+\cdots+\hat{\beta}_{p-1}X_{p-1} \\
			\text{中心化模型:}\hat{Y}=\hat{\gamma}_0+\hat{\beta}_1(X_1-\bar{x}_1)+\hat{\beta}_2(X_2-\bar{x}_2)+\cdots+\hat{\beta}_{p-1}(X_{p-1}-\bar{x}_{p-1}) \\
			\text{标准化模型:}\hat{Y}=\hat{\alpha}_0+\hat{\beta}_1^0\frac{X_1-\bar{x}_1}{s_1}+\hat{\beta}_2^0\frac{X_2-\bar{x}_2}{s_2}+\cdots+\hat{\beta}_{p-1}^0\frac{X_{p-1}-\bar{x}_{p-1}}{s_{p-1}}
		\end{gather*}
	\end{enumerate}
\end{theorem}
\begin{proof}
	直接代入正则方程观察即可。
\end{proof}

\subsection{假设检验}
\begin{definition}
	对于\cref{model:NormalLinearRegressionModel},称假设检验:
	\begin{equation*}
		H_0:\beta_1=\beta_2=\cdots=\beta_{p-1}=0,\quad H_1:\text{至少有一个回归系数不为}0
	\end{equation*}
	为\textbf{回归方程的显著性检验}。称:
	\begin{equation*}
		H_0:\beta_i=0,\quad H_1:\beta_i\ne0
	\end{equation*}
	为\textbf{回归系数$\beta_i$的显著性检验}。
\end{definition}
\begin{theorem}
	对于\cref{model:NormalLinearRegressionModel},回归方程的显著性检验的统计量和拒绝域为:
	\begin{equation*}
		F=\frac{\hat{\beta}_I^T\tilde{X}_c^Ty/(p-1)}{\operatorname{SSE}/(n-p)},\quad \{F:F>F_{p-1,n-p}(\alpha)\}
	\end{equation*}
	回归系数$\beta_i$的显著性检验的统计量和拒绝域为:
	\begin{equation*}
		t=\frac{\hat{\beta}_i}{\sqrt{c_{ii}}\hat{\sigma}},\quad\{t:|t|>t_{n-p}(\alpha/2)\}
	\end{equation*}
\end{theorem}
\begin{proof}
	(1)可以发现此时的假设检验即为在\cref{theo:NormalLinearModelHypothesisTesting}中取$A=(\mathbf{0},I_{p-1}),\;b=\mathbf{0}$时的情况。原假设下模型变为:
	\begin{equation*}
		y_i=\beta_0+\varepsilon_i,\quad i=1,2,\dots,n
	\end{equation*}
	此时设计阵为$\mathbf{1}_n$,正则方程为$n\hat{\beta}_0=n\bar{y}$,$\hat{\beta}=\bar{y}$,由\cref{theo:SSESSEACalculate}可得$\operatorname{SSE}_A=y^Ty-\bar{y}\mathbf{1}_n^Ty=y^Ty-\bar{y}n\bar{y}=y^Ty-n\bar{y}^2$。备择假设下根据\cref{theo:CenterStandardLinearModel}(2)可得:
	\begin{equation*}
		\operatorname{SSE}=y^Ty-\hat{\beta}^TX^Ty=y^Ty-\hat{\gamma}_0n\bar{y}-\hat{\beta}_I^T\tilde{X}_c^Ty=y^Ty-n\bar{y}^2-\hat{\beta}_I^T\tilde{X}_c^Ty
	\end{equation*}
	于是有:
	\begin{equation*}
		\operatorname{SSE}_A-\operatorname{SSE}=y^Ty-n\bar{y}^2-(y^Ty-n\bar{y}^2-\hat{\beta}_I^T\tilde{X}_c^Ty)=\hat{\beta}_I^T\tilde{X}_c^Ty
	\end{equation*}
	由\cref{theo:NormalLinearModelHypothesisTesting}(5)即可得出结论。\par
	(2)仿照\cref{prop:NormalLinearModel}(2)的证明过程可知:
	\begin{equation*}
		\hat{\beta}\sim\operatorname{N}_p[\beta,\sigma^2(X^TX)^{-1}]
	\end{equation*}
	记$C=(X^TX)^{-1}=(c_{ij})$,由\cref{prop:MultiNormal}(3)可得:
	\begin{equation*}
		\hat{\beta}_i\sim\operatorname{N}(\beta_i,\sigma^2c_{ii})
	\end{equation*}
	所以当$H_0$成立时有:
	\begin{equation*}
		\frac{\hat{\beta}_i}{\sigma\sqrt{c_{ii}}}\sim\operatorname{N}(0,1)
	\end{equation*}
	由\cref{prop:NormalLinearModel}(2)可知:
	\begin{equation*}
		\frac{(n-p)\hat{\sigma}^2}{\sigma^2}\sim\chi_{n-r}^2
	\end{equation*}
	所以:
	\begin{equation*}
		\frac{\hat{\beta}_i}{\sigma\sqrt{c_{ii}}}\Big/\sqrt{\frac{(n-p)\hat{\sigma}^2}{\sigma^2(n-p)}}=\frac{\hat{\beta}_i}{\sqrt{c_{ii}}\hat{\sigma}}\sim t_{n-p}\qedhere
	\end{equation*}
\end{proof}

\subsection{模型选择}
\subsubsection{变量选择}
\begin{theorem}
	当全模型正确时:
	\begin{enumerate}
		\item 剔除一部分自变量后,可使得剩余的那部分自变量的回归系数的LSE的方差减小,但此时的估计一般为有偏估计。若被剔除的自变量对因变量影响较小,则可使得剩余的那部分自变量的回归系数的LSE的MSE减小;
		\item 若用选模型作预测,预测一般是有偏的,但预测偏差的方差减小。若被剔除的自变量对因变量影响较小,则可使得预测的MSE减小。
	\end{enumerate}
\end{theorem}
\subsubsection{RMSq准则}
\subsubsection{Cp准则}
\subsubsection{AIC准则}

\subsection{回归诊断}
\subsubsection{残差分析}
\begin{definition}
	当模型为\cref{model:NormalLinearRegressionModel}时,将:
	\begin{equation*}
		r_i=\frac{\hat{e}_i}{\sqrt{\hat{\sigma}^2(1-p_{ii})}}
	\end{equation*}
	称为\textbf{学生化残差}。由\cref{prop:ehat}(1)可知它相当于是对$\hat{e}$进行标准化然后用$\sigma^2$的估计$\hat{\sigma}^2$进行替换后得到的值。
\end{definition}
\begin{note}
	应用上可以近似地认为$r_i$相互独立且服从$N(0,1)$。若将$\hat{e}$与$\hat{y}$画作二维散点图,以$\hat{y}$为横坐标、$r_i$为纵坐标,点的纵坐标应大致分布于$[-2,2]$,且由\cref{prop:ehat}(3)和\cref{prop:MultiNormal}(2)(8)可得这些点不应呈现出任何趋势($\hat{y}$与$\hat{e}_i$独立)。若不满足上述现象,则此时模型有问题,可能是漏掉了重要的回归自变量,也可能是数据不满足\cref{model:NormalLinearRegressionModel}中的假设。
\end{note}
\begin{definition}
	称以某种残差为纵坐标、其它量为横坐标的散点图为\textbf{残差图}。
\end{definition}
\subsubsection{影响分析}
影响分析即为探查对估计或预测有较大影响的数据。
\begin{definition}
	定义第$i$个样本\textbf{Cook统计量}为:
	\begin{equation*}
		D_1=\frac{(\hat{\beta}-\hat{\beta}_{(i)})^TX^TX(\hat{\beta}-\hat{\beta}_{(i)})}{p\hat{\sigma}^2},\quad i=1,2,\dots,n
	\end{equation*}
	其中$\hat{\beta}_{(i)}$表示删除第$i$个样本后拟合得到的回归系数。
\end{definition}
\begin{theorem}
	$D_i$有如下计算公式:
	\begin{equation*}
		D_i=\frac{1}{p}\left(\frac{p_{ii}}{1-p_{ii}}\right)r_i^2,\quad i=1,2,\dots,n
	\end{equation*}
	其中$r_i$时学生化残差,$p_{ii}$时矩阵$P_X$的第$i$个主对角元。
\end{theorem}
\begin{note}
	Cook统计量$D_i$越大表示第$i$个样本对估计或预测造成的影响越大。
\end{note}