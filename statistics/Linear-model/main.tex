\chapter{线性模型}

\section{一般线性模型}
\begin{definition}\label{model:LinearModel}
	称以下模型为\gls{LinearModel}:
	\begin{equation*}
		\begin{cases}
			y=X\beta+\varepsilon \\
			\operatorname{E}(\varepsilon)=\mathbf{0} \\
			\operatorname{Cov}(\varepsilon)=\sigma^2I_n
		\end{cases}
	\end{equation*}
	其中$y$为$n\times 1$观测向量,$X$为$n\times p$设计矩阵,$\beta$为$p\times 1$未知参数向量,$\varepsilon$为随机误差,$\sigma^2$为误差方差。
\end{definition}
\begin{definition}
	称方程$X^TX\beta=X^Ty$为\gls{NormalEquation}。
\end{definition}
\begin{theorem}
	对于\cref{model:LinearModel},$\hat{\beta}=(X^TX)^-X^Ty$是其唯一的最小二乘解。
\end{theorem}
\begin{proof}
	注意到:
	\begin{gather*}
		\begin{aligned}
			Q(\beta)&=||y-X\beta||^2=(y-X\beta)^T(y-X\beta) \\
			&=y^Ty-y^TX\beta-\beta^TX^Ty-\beta^TX^TX\beta \\
			&=y^Ty-2y^TX\beta-\beta^TX^TX\beta
		\end{aligned}\\
		\frac{\partial y^TX\beta}{\beta}=X^Ty,\quad
		\frac{\partial \beta^TX^TX\beta}{\beta}=2X^TX\beta \\
		\frac{\partial Q(\beta)}{\partial\beta}=2X^Ty-2X^TX\beta=0 \\
		X^TX\beta=X^Ty
	\end{gather*}
	由\cref{theo:VectorSpaceAAAT}和\cref{theo:SolutionOfSLE2}(1)可知方程$X^TX\beta=X^Ty$是相容的,根据\cref{theo:InhomogeneousLinearEq'sGeneralSolution2}可知其通解为:
	\begin{equation*}
		\hat{\beta}=(X^TX)^-X^Ty
	\end{equation*}
	其中$(X^TX)^-$是$X^TX$的任意一个广义逆矩阵。\par
	对任意的$\beta$,有:
	\begin{align*}
		Q(\beta)&=||y-X\beta||^2=||y-X\hat{\beta}+X\hat{\beta}-X\beta||^2=||y-X\hat{\beta}+X(\hat{\beta}-\beta)||^2 \\
		&=||y-X\hat{\beta}||^2+||X(\hat{\beta}-\beta)||^2+2(y-X\hat{\beta})^TX(\hat{\beta}-\beta)
	\end{align*}
	注意到正则方程即为:
	\begin{equation*}
		X^T(y-X\beta)=\mathbf{0}
	\end{equation*}
	于是:
	\begin{equation*}
		2(y-X\hat{\beta})^TX(\hat{\beta}-\beta)=2[X^T(y-X\hat{\beta})]^T(\hat{\beta}-\beta)=0
	\end{equation*}
	所以:
	\begin{equation*}
		Q(\beta)=||y-X\hat{\beta}||^2+||X(\hat{\beta}-\beta)||^2
	\end{equation*}
	上第二项总是非负的,由范数的性质其为$0$当且仅当$X\hat{\beta}=X\beta$,即当且仅当$X^TX\beta=X^TX\hat{\beta}=X^Ty$,所以使$Q(\beta)$达到最小值的$\beta$必为正则方程的解$\hat{\beta}=(X^TX)^-X^Ty$。
\end{proof}
\begin{derivation}
	若$\operatorname{rank}(X)=p$,则$X$的列向量组线性无关。考虑二次型$y^TX^TXy$,$y^TX^TXy=0\Leftrightarrow||Xy||=0\Leftrightarrow Xy=\mathbf{0}$,而$X$的列向量是线性无关的,所以不存在非零向量的$y$使得$Xy=\mathbf{0}$,于是$y^TX^TXy$是一个正定二次型,$X^TX$是一个正定矩阵。由\cref{theo:PositiveDefinite}(3)的第五点和\info{行列式等于特征值的积,行列式大于$0$矩阵可逆}可得$X^TX$可逆。此时$\hat{\beta}=(X^TX)^{-1}X^Ty$,称$\hat{\beta}$为$\beta$的\gls{LSE}。
\end{derivation}
\subsection{参数估计}
\subsubsection{回归系数}
\begin{definition}
	若存在$n\times 1$向量$\alpha$使得$\operatorname{E}(\alpha^Ty)=c^T\beta$对一切的$\beta$成立,则称$c^T\beta$为\gls{EstimableFunction}。
\end{definition}
\begin{property}\label{prop:EstimableFunction}
	对于\cref{model:LinearModel},$c^T\beta$和$d^T\beta$是可估函数,$\hat{\beta}$是正则方程的解,则:
	\begin{enumerate}
		\item 使$c^T\beta$成为可估函数的全体向量$c$构成$\mathcal{M}(X^T)$;
		\item 若$c_1^T\beta$和$c_2^T\beta$都是可估函数,则对任意常数$a_1,a_2$,$a_1c_1^T\beta+a_2c_2^T\beta$也是可估函数;
		\item 线性无关的可估函数组最多有$\operatorname{rank}(X)$个可估函数;
		\item $c^T\hat{\beta}$与$(X^TX)^-$的选择无关;
		\item $c^T\hat{\beta}$为$c^T\beta$的无偏估计;
		\item  $\operatorname{Var}(c^T\hat{\beta})=\sigma^2c^T(X^TX)^-c,\;\operatorname{Cov}(c^T\hat{\beta},d^T\hat{\beta})=\sigma^2c^T(X^TX)^-d$,且与$(X^TX)^-$的选择无关;
		\item $c^T\hat{\beta}$是$c^T\beta$唯一的BLUE;
		\item 设$\varphi_i=c_i^T\beta,\;i=1,2,\dots,k$都是可估函数,$\seq{\alpha}{k}\in\mathbb{R}^{}$,则$\varphi=\sum\limits_{i=1}^{k}\alpha_i\varphi_i$也是可估的,且$\hat{\varphi}=\sum\limits_{i=1}^{k}\alpha_ic_i^T\hat{\beta}$是$\varphi$的BLUE。
	\end{enumerate}
\end{property}
\begin{proof}
	(1)$\;c^T\beta$是可估函数$\Leftrightarrow$存在$n\times1$向量$\alpha$使得$\operatorname{E}(\alpha^Ty)=\alpha^T\operatorname{E}(y)=\alpha^TX\beta=c^T\beta$对一切的$\beta$成立$\Leftrightarrow c=X^T\alpha$。\par
	(2)由(1)直接可得。\par
	(3)由(1)直接可得。\par
	(4)因为$c^T\beta$可估,由(1)可知存在$n\times 1$向量$\alpha$使得$c=X^T\alpha$,于是:
	\begin{equation*}
		c^T\hat{\beta}=\alpha^TX(X^TX)^-X^Ty
	\end{equation*}
	由\cref{prop:A-}(4)即可得出结论。\par
	(5)因为$c^T\beta$可估,由(1)可知存在$n\times 1$向量$\alpha$使得$c=X^T\alpha$,根据\cref{prop:A-}(7)可得:
	\begin{equation*}
		\operatorname{E}(c^T\hat{\beta})=\operatorname{E}[c^T(X^TX)^-X^Ty]=c^T(X^TX)^-X^TX\beta=c^T\beta
	\end{equation*}\par
	(6)因为$c^T\beta,d^T\beta$是可估函数,由(1)可知存在$\alpha,\gamma$使得$c=X^T\alpha,d=X^T\gamma$。由\cref{prop:CovMat}(3)和\cref{prop:A-}(6)(7)可知:
	\begin{align*}
		\operatorname{Cov}(c^T\hat{\beta},d^T\hat{\beta})
		&=\operatorname{Cov}[c^T(X^TX)^-X^Ty,d^T(X^TX)^-X^Ty] \\
		&=c^T(X^TX)^-X^T\operatorname{Cov}(y)X[(X^TX)^-]^Td \\
		&=c^T(X^TX)^-X^T\sigma^2I_nX(X^TX)^-d \\
		&=\sigma^2c^T(X^TX)^-d
	\end{align*}
	在上第三行中把$c^T,d$展开为$\alpha^TX,X^T\gamma$,由\cref{prop:A-}(4)即可知$\operatorname{Cov}(c^T\hat{\beta},d^T\hat{\beta})$与$(X^TX)^-$的选择无关。\par
	(7)无偏性由(5)可得,线性性由正则方程可知,下证方差最小。
	设$a^Ty$为$c^T\beta$的任一无偏估计,由(1)的过程可知$c=X^Ta$。根据\cref{prop:A-}(6)和(6)可得:
	\begin{align*}
		\operatorname{Var}(a^Ty)-\operatorname{Var}(c^T\hat{\beta})&=\sigma^2[a^Ta-c^T(X^TX)^-c] \\
		&=\sigma^2[a^T-c^T(X^TX)^-X^T][a-X(X^TX)^-c] \\
		&=\sigma^2||a-X(X^TX)^-c||^2\geqslant0
	\end{align*}
	上式第一行到第二行是由于\cref{prop:A-}(7):
	\begin{align*}
		&[a^T-c^T(X^TX)^-X^T][a-X(X^TX)^-c] \\
		=&a^Ta-a^TX(X^TX)^-c-c^T(X^TX)^-X^Ta+c^T(X^TX)^-X^TX(X^TX)^-c \\
		=&a^Ta-c^T(X^TX)^-c-c^T(X^TX)^-c+c^T(X^TX)^-c \\
		=&a^Ta-c^T(X^TX)^-c
	\end{align*}
	由范数的性质可知$\operatorname{Var}(a^Ty)=\operatorname{Var}(c^T\hat{\beta})$当且仅当$a=X(X^TX)^-c$,由\cref{prop:A+}(3)可知$a=X(X^TX)^-c\Leftrightarrow a^T=c^T(X^TX)^-X^T\Leftrightarrow a^Ty=c^T(X^TX)^-X^Ty=c^T\hat{\beta}$。\par
	(8)因为$\seq{\varphi}{k}$都是可估函数,所以存在$\seq{b}{k}$使得$\operatorname{E}(b_i^Ty)=c_i^T\beta$,于是:
	\begin{equation*}
		\operatorname{E}\left(\sum_{i=1}^{k}\alpha_ib_i^Ty\right)=\sum_{i=1}^{k}\alpha_i\operatorname{E}(b_i^Ty)=\sum_{i=1}^{k}\alpha_ic_i^T\beta=\sum_{i=1}^{k}a_i\varphi_i=\varphi
	\end{equation*}
	所以取$\alpha=\sum\limits_{i=1}^{k}\alpha_ib_i$即可得到$\operatorname{E}(\alpha^Ty)=\varphi$,$\varphi$是可估的。\par
	由(5)可得$c_i^T\hat{\beta}$是$c_i^T\beta$的无偏估计,所以:
	\begin{equation*}
		\operatorname{E}(\hat{\varphi})=\operatorname{E}\left(\sum_{i=1}^{k}\alpha_ic_i^T\hat{\beta}\right)=\sum_{i=1}^{k}\alpha_i\operatorname{E}(c_i^T\hat{\beta})=\sum_{i=1}^{k}\alpha_ic_i^T\beta=\varphi
	\end{equation*}
	即$\hat{\varphi}$是一个无偏估计。\par
	令$c=\sum\limits_{i=1}^{k}\alpha_ic_i$,则$\varphi=c^T\beta$。设$\gamma^Ty$是$\varphi$的一个无偏估计,于是由(7)可得:
	\begin{equation*}
		\operatorname{Var}(\gamma^Ty)-\operatorname{Var}(c^T\hat{\beta})=\sigma^2||\gamma-X(X^TX)^-c||^2
	\end{equation*}
	上式等于$0\Leftrightarrow \gamma^Ty=c^T\hat{\beta}=\hat{\varphi}$,即$\hat{\varphi}$是唯一的BLUE。
\end{proof}
\begin{definition}
	对于\cref{model:LinearModel},若$c^T\beta$是可估函数,称$c^T\hat{\beta}$为$c^T\beta$的LSE,其中$\hat{\beta}$为正则方程的解。
\end{definition}
\subsubsection{残差}
\begin{definition}
	称$\hat{e}=y-X\hat{\beta}$为\gls{ResidualVector},称$\hat{e}^T\hat{e}$为\gls{SSE},记为SSE。
\end{definition}
\begin{property}\label{prop:ehat}
	对于\cref{model:LinearModel},$\hat{\beta}$为正则方程的解,则残差向量$\hat{e}$满足:
	\begin{enumerate}
		\item $\operatorname{E}(\hat{e})=0,\;\operatorname{Cov}(\hat{e})=\sigma^2(I-P_X)$;
		\item $SSE=y^T(I_n-P_X)y$。
	\end{enumerate}
\end{property}
\begin{proof}
	(1)由\cref{prop:OrthogonalProjectionMat}(2)可知向$\mathcal{M}(X)$的正交投影阵$P_X=X(X^TX)^-X^T$,根据\cref{prop:OrthogonalProjectionMat}(4)可知$I_n-P_X$是对称幂等阵,所以由\cref{prop:CovMat}(3)可得:
	\begin{gather*}
		\begin{aligned}
			\operatorname{E}(\hat{e})&=\operatorname{E}(y-X\hat{\beta})=\operatorname{E}[I_ny-X(X^TX)^-X^Ty]=(I_n-P_X)\operatorname{E}(y) \\
			&=(I_n-P_X)X\beta=(X-X)\beta=0
		\end{aligned} \\
		\begin{aligned}
			\operatorname{Cov}(\hat{e})&=\operatorname{Cov}[(I_n-P_X)y]=(I_n-P_X)\operatorname{Cov}(y)(I_n-P_X)^T \\
			&=(I_n-P_X)\operatorname{Cov}(y)(I_n-P_X)=\sigma^2(I_n-P_X)
		\end{aligned}
	\end{gather*}\par
	(2)由(1)的证明过程可知:
	\begin{equation*}
		\hat{e}=(I_n-P_X)y
	\end{equation*}
	且$I_n-P_X$是一个对称幂等阵,于是:
	\begin{equation*}
		\hat{e}^T\hat{e}=y^T(I_n-P_X)^T(I_n-P_X)y=y^T(I_n-P_X)(I_n-P_X)y=y^T(I_n-P_X)y\qedhere
	\end{equation*}
\end{proof}
\subsubsection{误差方差}
\begin{theorem}\label{theo:VarianceOfErrorTerm}
	对于\cref{model:LinearModel},$\hat{\beta}$为正则方程的解,$\operatorname{rank}(X)=r$,则:
	\begin{equation*}
		\hat{\sigma}^2=\frac{SSE}{n-r}
	\end{equation*}
	是$\sigma^2$的无偏估计。
\end{theorem}
\begin{proof}
注意到$(I_n-P_X)X=X-X=\mathbf{0}$,由\cref{prop:ehat}(2)和\cref{theo:ERVQuadraticForm}可得:
	\begin{align*}
		\operatorname{E}(SSE)&=\operatorname{E}[y^T(I_n-P_X)y] =\beta^TX^T(I_n-P_X)X\beta+\operatorname{tr}[(I_n-P_X)\sigma^2I_n] \\
		&=\sigma^2\operatorname{tr}(I_n-P_X)
	\end{align*}
	由\cref{prop:OrthogonalProjectionMat}(4)、\cref{prop:IdempotentMat}(2)(3)和\cref{prop:OrthogonalProjectionMat}(1)可得:
	\begin{equation*}
		\operatorname{tr}(I_n-P_X)=\operatorname{rank}(I_n-P_X)=n-\operatorname{rank}(P_X)=n-\operatorname{rank}(X)=n-r
	\end{equation*}
	即:
	\begin{equation*}
		\operatorname{E}\left(\frac{SSE}{n-r}\right)=\sigma^2\qedhere
	\end{equation*}
\end{proof}
\begin{definition}
	称$\hat{\sigma}^2$为$\sigma^2$的LSE。
\end{definition}

\subsection{约束最小二乘估计}
\begin{theorem}\label{theo:ConstraintLinearModel}
	对于\cref{model:LinearModel},假设:
	\begin{equation*}
		A\beta=b,\quad A\in M_{k\times p}(K),\quad\operatorname{rank}(A)=k,\quad\mathcal{M}(A^T)\subseteq\mathcal{M}(X^T)
	\end{equation*}
	且$A\beta=b$相容,则:
	\begin{equation*}
		\hat{\beta}_A=\hat{\beta}-(X^TX)^-A^T[A(X^TX)^-A^T]^{-1}(A\hat{\beta}-b)
	\end{equation*}
	为$\beta$在约束$A\beta=b$下的约束LS解,$A\hat{\beta}_A$为$A\beta$的约束LSE。
\end{theorem}
\begin{proof}
	使用Lagrange乘子法构造辅助函数($\lambda$为Lagrange乘子,乘子前加上系数$2$是为了下面不出现分数,对结果没有影响):
	\begin{align*}
		F(\beta,\lambda)&=||y-X\beta||^2+2\lambda^T(A\beta-b) \\
		&=y^Ty-y^TX\beta-\beta^TX^Ty+\beta^TX^TX\beta+2\lambda^TA\beta-2\lambda^Tb
	\end{align*}
	于是:
	\begin{equation*}
		\frac{\partial F(\beta,\lambda)}{\partial\beta}=-2X^Ty+2X^TX\beta+2A^T\lambda
	\end{equation*}
	令上式为$0$,得到:
	\begin{equation*}
		X^TX\beta=X^Ty-A^T\lambda
	\end{equation*}
	于是约束下的解即为方程组:
	\begin{equation*}
		\begin{cases}
			X^TX\beta=X^Ty-A^T\lambda \\
			A\beta=b
		\end{cases}
	\end{equation*}
	的解,将其记为$\hat{\beta}_A,\hat{\lambda}$。因为$\mathcal{M}(A^T)\subseteq\mathcal{M}(X^T)$,所以方程组是相容的。由\cref{theo:InhomogeneousLinearEq'sGeneralSolution2}可知:
	\begin{equation*}
		\hat{\beta}_A=(X^TX)^-X^Ty-(X^TX)^-A^T\hat{\lambda}=\hat{\beta}-(X^TX)^-A^T\hat{\lambda}
	\end{equation*}
	代入方程组的第二个方程可得:
	\begin{equation*}
		A\hat{\beta}-A(X^TX)^-A^T\hat{\lambda}=b
	\end{equation*}
	由\info{$A(X^TX)^-A^T$的可逆性}可知:
	\begin{equation*}
		\hat{\lambda}=[A(X^TX)^-A^T]^{-1}(A\hat{\beta}-b)
	\end{equation*}
	于是:
	\begin{equation*}
		\hat{\beta}_A=\hat{\beta}-(X^TX)^-A^T[A(X^TX)^-A^T]^{-1}(A\hat{\beta}-b)
	\end{equation*}
	下证明这个解确实是最小二乘解。\par
	做分解:
	\begin{align*}
		||y-X\beta||^2
		&=||y-X\hat{\beta}+X(\hat{\beta}-\beta)||^2 \\
		&=||y-X\hat{\beta}||^2+2(y-X\hat{\beta})^TX(\hat{\beta}-\beta)+(\hat{\beta}-\beta)^TX^TX(\hat{\beta}-\beta)
	\end{align*}
	由\cref{prop:A-}(5)(6)可得:
	\begin{align*}
		(y-X\hat{\beta})^TX(\hat{\beta}-\beta)&=y^TX(\hat{\beta}-\beta)-\hat{\beta}^TX^TX(\hat{\beta}-\beta) \\
		&=y^TX(\hat{\beta}-\beta)-[(X^TX)^-Xy]^TX^TX(\hat{\beta}-\beta) \\
		&=y^TX(\hat{\beta}-\beta)-y^TX^T(X^TX)^-X^TX(\hat{\beta}-\beta) \\
		&=y^TX(\hat{\beta}-\beta)-y^TX^T(\hat{\beta}-\beta)=0
	\end{align*}
	于是:
	\begin{align*}
		||y-X\beta||^2
		&=||y-X\hat{\beta}||^2+(\hat{\beta}-\beta)^TX^TX(\hat{\beta}-\beta) \\
		&=||y-X\hat{\beta}||^2+(\hat{\beta}-\hat{\beta}_A+\hat{\beta}_A-\beta)^TX^TX(\hat{\beta}-\hat{\beta}_A+\hat{\beta}_A-\beta) \\
		&=||y-X\hat{\beta}||^2+||X(\hat{\beta}-\hat{\beta}_A)||^2+||X(\hat{\beta}_A-\beta)||^2+2(\hat{\beta}-\hat{\beta}_A)^TX^TX(\hat{\beta}_A-\beta)
	\end{align*}
	由\cref{prop:A-}(6)(7)以及$\mathcal{M}(A^T)\subseteq\mathcal{M}(X^T)$可得:
	\begin{align*}
		(\hat{\beta}-\hat{\beta}_A)^TX^TX(\hat{\beta}_A-\beta)
		&=[(X^TX)^-A^T\hat{\lambda}]^TX^TX(\beta_A-\beta)
		=\hat{\lambda}^TA(X^TX)^-X^TX(\beta_A-\beta) \\
		&=\hat{\lambda}^TA(\beta_A-\beta)
		=\hat{\lambda}^T(A\beta_A-A\beta)=0
	\end{align*}
	所以:
	\begin{equation*}
		||y-X\beta||^2=||y-X\hat{\beta}||^2+||X(\hat{\beta}-\hat{\beta}_A)||^2+||X(\hat{\beta}_A-\beta)||^2
	\end{equation*}
	即对任意满足$A\beta=b$的$\beta$都有:
	\begin{equation*}
		||y-X\beta||^2\geqslant||y-X\hat{\beta}||^2+||X(\hat{\beta}-\hat{\beta}_A)||^2
	\end{equation*}
	等号成立当且仅当$\beta=\hat{\beta}_A$,于是$\hat{\beta}_A$是LSE。
\end{proof}
\subsubsection{误差方差}
\begin{theorem}
	在\cref{theo:ConstraintLinearModel}的假设下,在参数区域$A\beta=b$上,
	\begin{equation*}
		\hat{\sigma}_A^2=\frac{||y-X\hat{\beta}_A||^2}{n-r+k}=\frac{SSE_A}{n-r+k}
	\end{equation*}
	是$\sigma^2$的无偏估计。
\end{theorem}
\begin{proof}
	由\cref{theo:ConstraintLinearModel}可知:
	\begin{equation*}
		\operatorname{E}(||y-X\hat{\beta}_A||^2)=\operatorname{E}[||y-X\hat{\beta}||^2+||X(\hat{\beta}-\hat{\beta}_A)||^2]=\operatorname{E}(||y-X\hat{\beta}||^2)+\operatorname{E}[||X(\hat{\beta}-\hat{\beta}_A)||^2]
	\end{equation*}
	根据\cref{theo:VarianceOfErrorTerm}可知:
	\begin{equation*}
		\operatorname{E}(||y-X\hat{\beta}||^2)=(n-r)\sigma^2
	\end{equation*}
	由\cref{prop:A-}(6)(7)、$\mathcal{M}(A^T)\subseteq\mathcal{M}(X^T)$
	\begin{align*}
		||X(\hat{\beta}-\hat{\beta}_A)||&=(\hat{\beta}-\hat{\beta}_A)^TX^TX(\hat{\beta}-\hat{\beta}_A) \\
		&=\{(X^TX)^-A^T[A(X^TX)^-A^T]^{-1}(A\hat{\beta}-b)\}^T \\
		&\quad\cdot X^TX(X^TX)^-A^T[A(X^TX)^-A^T]^{-1}(A\hat{\beta}-b) \\
		&=(A\hat{\beta}-b)^T[A(X^TX)^-A^T]^{-1}A(X^TX)^- \\
		&\quad\cdot X^TX(X^TX)^-A^T[A(X^TX)^-A^T]^{-1}(A\hat{\beta}-b) \\
		&=(A\hat{\beta}-b)^T[A(X^TX)^-A^T]^{-1}A(X^TX)^-A^T[A(X^TX)^-A^T]^{-1}(A\hat{\beta}-b) \\
		&=(A\hat{\beta}-b)^T[A(X^TX)^-A^T]^{-1}(A\hat{\beta}-b)
	\end{align*}
	因为$\mathcal{M}(A^T)\subseteq\mathcal{M}(X^T)$,所以由\cref{prop:EstimableFunction}(1)可知$A\beta$的每一个元素都是可估函数,于是由\cref{theo:ERVQuadraticForm}和\cref{prop:EstimableFunction}(5)(6)可知:
	\begin{align*}
		\operatorname{E}(||X(\hat{\beta}-\hat{\beta}_A)||)
		&=\operatorname{E}\{(A\hat{\beta}-b)^T[A(X^TX)^-A^T]^{-1}(A\hat{\beta}-b)\} \\
		&=(A\beta-b)^T[A(X^TX)^-A^T]^{-1}(A\beta-b) \\
		&\quad+\operatorname{tr}\{[A(X^TX)^-A^T]^{-1}\operatorname{Cov}(A\hat{\beta}-b)\} \\
		&=\operatorname{tr}\{[A(X^TX)^-A^T]^{-1}\sigma^2A(X^TX)^-A^T\}=\sigma^2\operatorname{tr}(I_k)=k\sigma^2
	\end{align*}
	所以:
	\begin{equation*}
		\operatorname{E}(||y-X\hat{\beta}_A||^2)=(n-r+k)\sigma^2
	\end{equation*}
	即在参数区域$A\beta=b$上,
	\begin{equation*}
		\hat{\sigma}_A^2=\frac{||y-X\hat{\beta}_A||^2}{n-r+k}
	\end{equation*}
	是$\sigma^2$的无偏估计。
\end{proof}

\subsection{实际计算}
\begin{theorem}
	对于无约束条件以及约束$A\beta=\mathbf{0}$,有:
	\begin{gather*}
		SSE=||y-X\hat{\beta}||^2=y^Ty-\hat{\beta}^TX^Ty \\
		SSE_A=||y-X\hat{\beta}_A||^2=y^Ty-\hat{\beta}_A^TX^Ty
	\end{gather*}
\end{theorem}
\begin{proof}
	由\cref{prop:A-}(5)可知:
	\begin{align*}
		SSE&=(y-X\hat{\beta})^T(y-X\hat{\beta})=y^Ty-y^TX\hat{\beta}-\hat{\beta}^TX^Ty+\hat{\beta}^TX^TX\hat{\beta} \\
		&=y^Ty-2\hat{\beta}^TX^Ty+\hat{\beta}^TX^TX(X^TX)^-X^Ty=y^Ty-2\hat{\beta}^TX^Ty+\hat{\beta}^TX^Ty \\
		&=y^Ty-\hat{\beta}^TX^Ty
	\end{align*}
	由\cref{theo:ConstraintLinearModel}可知:
	\begin{equation*}
		\begin{cases}
			X^TX\hat{\beta}_A=X^Ty-A^T\hat{\lambda} \\
			A\hat{\beta}_A=\mathbf{0}
		\end{cases}
	\end{equation*}
	其中$\lambda$为Lagrange乘子,于是有:
	\begin{align*}
		SSE_A&=(y-X\hat{\beta}_A)^T(y-X\hat{\beta}_A)=y^Ty-y^TX\hat{\beta}_A-\hat{\beta}_A^TX^Ty+\hat{\beta}_A^TX^TX\hat{\beta}_A \\
		&=y^Ty-\hat{\beta}_A^TX^Ty+\hat{\beta}_A^TX^TX\hat{\beta}_A-\hat{\beta}^TX^Ty=y^Ty-\hat{\beta}_A^TX^Ty+\hat{\beta}_A^T(X^TX\hat{\beta}_A-X^Ty) \\
		&=y^Ty-\hat{\beta}_A^TX^Ty-\hat{\beta}_A^TA^T\hat{\lambda}=y^Ty-\hat{\beta}_A^TX^Ty\qedhere
	\end{align*}
\end{proof}
\begin{definition}
	称$\hat{\beta}^TX^Ty$为\gls{RSS},记为$RSS(\beta)$。称$\hat{\beta}_A^TX^Ty$为约束条件$A\beta=\mathbf{0}$下的回归平方和,记为$RSS_A(\beta)$。
\end{definition}
\begin{note}
	回归平方和表示了数据平方和$y^Ty$中能够由因变量$y$与自变量$\seq{X}{p}$的线性关系解释的部分。
\end{note}

\subsection{预测}
\label{def:LinearModelForcast}
假设要预测$m$个点$x_{0i}=(x_{0i1},x_{0i2},\dots,x_{0ip}),\;i=1,2,\dots,m$,它们所对应的因变量为$y_{01},y_{02},\dots,y_{0m}$。已知$y_{0i}$和历史数据服从同一个线性模型,即:
\begin{gather*}
	y_0=X_0\beta+\varepsilon_0,\quad\operatorname{E}(\varepsilon_0)=\mathbf{0},\quad\operatorname{Cov}(\varepsilon_0)=\sigma^2I_m \\
	y_0=
	\begin{pmatrix}
		y_{01} \\
		y_{02} \\
		\vdots \\
		y_{0m}
	\end{pmatrix},\quad
	X_0=
	\begin{pmatrix}
		x_{011} & x_{012} & \cdots & x_{01p} \\
		x_{021} & x_{022} & \cdots & x_{02p} \\
		\vdots & \vdots & \ddots & \vdots \\
		x_{0m1} & x_{0m2} & \cdots & x_{0mp} 
	\end{pmatrix},\quad
	\varepsilon_0=
	\begin{pmatrix}
		\varepsilon_{01} \\
		\varepsilon_{02} \\
		\vdots \\
		\varepsilon_{0m}
	\end{pmatrix}
\end{gather*}
\textbf{假设$\mathcal{M}(X_0^T)\subseteq\mathcal{M}(X^T)$}。
\subsubsection{被测量与历史数据无关}
\begin{derivation}
	此时使用$\operatorname{E}(y_0)=X_0\beta$的估计$X_0\hat{\beta}=X_0(X^TX)^-X^Ty$去进行预测。\par
	由\cref{theo:VectorSpaceAAAT}可知$\mathcal{M}(X^T)=\mathcal{M}(X^TX)$,于是$\mathcal{M}(X_0^T)\subseteq\mathcal{M}(X^T)=\mathcal{M}(X^TX)$,由\cref{prop:A-}(3)可知$X_0(X^TX)^-X^Ty$与广义逆$(X^TX)^-$的选择无关。\par
	因为$\mathcal{M}(X_0^T)\subseteq\mathcal{M}(X^T)$,所以$X_0\beta$是可估的,它具有可估函数的性质。但需注意,因为$y_0$也是一个随机变量,所以这里的无偏性指的是$\operatorname{E}(\hat{y}_0-y_0)=0$。
\end{derivation}
\subsubsection{被测量与历史数据相关}
\begin{derivation}
	在某些情况下,$y_0$和$y$确实具有一定的相关性,用$\operatorname{Cov}(\varepsilon,\varepsilon_0)=\sigma^2V^T$来度量它们之间的相关性,此时有:
	\begin{equation*}
		\operatorname{Cov}[(y,y_0)^T]=\operatorname{Cov}[(\varepsilon,\varepsilon_0)^T]=\sigma^2
		\begin{pmatrix}
			I_n & V^T \\
			V & I_m 
		\end{pmatrix}
	\end{equation*}
\end{derivation}
\begin{definition}
	记$\hat{y}=Cy$是$y$的一个线性无偏估计,称:
	\begin{equation*}
		\operatorname{PMSE}(\hat{y})=\operatorname{E}[(\hat{y}-y)^TA(\hat{y}-y)]
	\end{equation*}
	为\gls{PMSE},其中$A>0$。
\end{definition}
\begin{theorem}\label{theo:PECorrelatedLinearModelForcast}
	$y_0$在广义预测均方误差意义下的最优线性无偏估计为:
	\begin{equation*}
		\hat{y}_0=X_0\hat{\beta}+V(y-X\hat{\beta})
	\end{equation*}
	其中$\hat{\beta}$为正则方程的解。
\end{theorem}
\begin{proof}
	令$\hat{y}_0=Cy$是一个无偏估计,由\cref{prop:MeasurableIntegral}(6)和\cref{prop:CovMat}(3)可得:
	\begin{gather*}
		\hat{y}_0-y_0=Cy-X_0\beta-\varepsilon_0=CX\beta+C\varepsilon-X_0\beta-\varepsilon_0=(CX-X_0)\beta+C\varepsilon-\varepsilon_0 \\
		\operatorname{E}(\hat{y}_0-y_0)=(CX-X_0)\beta=\mathbf{0}\text{对一切$\beta$成立}\iff CX=X_0 \\
		\begin{aligned}
			\operatorname{Cov}(C\varepsilon-\varepsilon_0)&=\operatorname{E}[(C\varepsilon-\varepsilon_0)(C\varepsilon-\varepsilon_0)^T]=\operatorname{E}(C\varepsilon\varepsilon^TC^T-C\varepsilon\varepsilon_0^T-\varepsilon_0\varepsilon^TC^T+\varepsilon_0\varepsilon_0^T) \\
			&=\operatorname{E}(C\varepsilon\varepsilon^TC^T)-\operatorname{E}(C\varepsilon\varepsilon_0^T)-\operatorname{E}(\varepsilon_0\varepsilon^TC^T)+\operatorname{E}(\varepsilon_0\varepsilon_0^T) \\
			&=C\operatorname{E}(\varepsilon\varepsilon^T)C^T-\operatorname{Cov}(C\varepsilon,\varepsilon_0)-\operatorname{Cov}(\varepsilon_0,C\varepsilon)+\operatorname{Cov}(\varepsilon_0) \\
			&=C\operatorname{Cov}(\varepsilon)C^T-2\operatorname{Cov}(C\varepsilon,\varepsilon_0)+\sigma^2I_m \\
			&=\sigma^2CC^T-2C\operatorname{Cov}(\varepsilon,\varepsilon_0)+\sigma^2I_m \\
			&=\sigma^2CC^T-2\sigma^2CV^T+\sigma^2I_m=\sigma^2(CC^T-2CV^T+I_m)
		\end{aligned}
	\end{gather*}
	由\cref{theo:ERVQuadraticForm}和\cref{prop:Trace}(2)可得:
	\begin{align*}
		\operatorname{PMSE}(\hat{y}_0)&=\operatorname{E}[(\hat{y}_0-y)^TA(\hat{y}_0-y)] =\operatorname{E}[(C\varepsilon-\varepsilon_0)^TA(C\varepsilon-\varepsilon_0)] \\
		&=\operatorname{tr}[A\sigma^2(CC^T-2CV^T+I_m)] =\sigma^2\operatorname{tr}[A(CC^T-2CV^T+I_m)]
	\end{align*}
	接下来的目标就是求解:
	\begin{equation*}
		\min_{C}\operatorname{PMSE}(\hat{y}_0),\quad\operatorname{s.t.}CX=X_0
	\end{equation*}\par
	使用Lagrange乘子法,构造辅助函数(不引入$X_0$是因为它是常数,对结果没影响):
	\begin{equation*}
		F(C,\varLambda)=\sigma^2\operatorname{tr}[A(CC^T-2CV^T+I_m)]-2\operatorname{tr}(CX\varLambda)
	\end{equation*}
	其中$\varLambda$是Lagrange乘子。由矩阵求导和\cref{prop:Trace}(1)可得:
	\begin{align*}
		\frac{\partial\operatorname{PSME}(\hat{y}_0)}{\partial C}&=\sigma^2\frac{\partial\operatorname{tr}(ACC^T)}{\partial C}-2\sigma^2\frac{\partial\operatorname{tr}(ACV^T)}{\partial C}-2\frac{\partial\operatorname{tr}(CX\varLambda)}{\partial C} \\
		&=\sigma^22AC-2\sigma^2AV-2\varLambda^TX^T
	\end{align*}
	令上式为$0$可得:
	\begin{gather*}
		\sigma^2AC=\sigma^2AV+\varLambda^TX^T \\
		C=V+\frac{A^{-1}\varLambda^TX^T}{\sigma^2}
	\end{gather*}
	代入$CX=X_0$可得:
	\begin{equation*}
		VX+\frac{A^{-1}\varLambda^TX^TX}{\sigma^2}=X_0 \\
		\varLambda^TX^TX=\sigma^2A(X_0-VX) \\
		X^TX\varLambda=\sigma^2(X_0^T-X^TV^T)A^T
	\end{equation*}
	由\cref{theo:VectorSpaceAAAT}可知$\mathcal{M}(X^T)=\mathcal{M}(X^TX)$,而$\mathcal{M}(X_0^T)\subseteq\mathcal{M}(X^T)$,所以上式等式右边矩阵的每一列都在$\mathcal{M}(X^TX)$中,即方程组是相容的。由\cref{theo:InhomogeneousLinearEq'sGeneralSolution2}可得:
	\begin{equation*}
		\varLambda=\sigma^2(X^TX)^-(X_0^T-X^TV^T)A^T
	\end{equation*}
	于是由\cref{prop:A-}(6)可得:
	\begin{align*}
		C&=V+A^{-1}A(X_0-VX)(X^TX)^-X^T=V+(X_0-VX)(X^TX)^-X^T \\
		&=X_0(X^TX)^-X^T+V[I_n-X(X^TX)^-X^T]
	\end{align*}
	由\cref{prop:A-}(3)(4)可知$C$是唯一的,与$(X^TX)^-$的选择无关。所以:
	\begin{equation*}
		\hat{y}_0=X_0(X^TX)^-X^Ty+V[I_n-X(X^TX)^-X^T]y=X_0\hat{\beta}+V(y-X\hat{\beta})
	\end{equation*}
	为$y_0$在广义预测均方误差意义下的最优线性无偏估计。
\end{proof}



\section{正态线性模型}
\subsection{参数估计}
\begin{definition}\label{model:NormalLinearModel}
	称以下模型为\gls{NormalLinearModel}:
	\begin{equation*}
		\begin{cases}
			y=X\beta+\varepsilon \\
			\varepsilon\sim\operatorname{N}_n(\mathbf{0},\sigma^2I_n)
		\end{cases}
	\end{equation*}
	其中$y$为$n\times 1$观测向量,$X$为$n\times p$设计矩阵,$\beta$为$p\times 1$未知参数向量,$\varepsilon$为随机误差,$\sigma^2$为误差方差。
\end{definition}
\begin{property}\label{prop:NormalLinearModel}
	对于\cref{model:NormalLinearModel},设$c^T\beta$为可估函数,则:
	\begin{enumerate}
		\item LS估计$c^T\hat{\beta}$是$c^T\beta$的MLE,$\tilde{\sigma}^2=\dfrac{n-r}{n}\hat{\sigma}^2$是$\sigma^2$的MLE。若模型在\cref{theo:ConstraintLinearModel}的约束下,则LS估计$c^T\hat{\beta}_A$是$c^T\beta$的MLE,$\tilde{\sigma}_A^2=\dfrac{n-r+k}{n}\hat{\sigma}_A^2$是$\sigma^2$的MLE;
		\item $c^T\hat{\beta}\sim N[c^T\beta,\sigma^2c^T(X^TX)^-c]$,$\dfrac{(n-r)\hat{\sigma}^2}{\sigma^2}=\dfrac{SSE}{\sigma^2}\sim\chi_{n-r}^2$;
		\item $c^T\hat{\beta}$与$\hat{\sigma}^2$相互独立;
		\item $T_1=y^Ty,\;T_2=X^Ty$为完全充分统计量;
		\item $c^T\hat{\beta}$是$c^T\beta$唯一的MVUE,$\hat{\sigma}^2$为$\sigma^2$唯一的MVUE。
	\end{enumerate}
\end{property}
\begin{proof}
	(1)对于\cref{model:NormalLinearModel},其似然函数为:
	\begin{equation*}
		L(\beta,\sigma^2)=\frac{1}{(2\pi)^{\frac{n}{2}}(\sigma^2)^\frac{n}{2}}\exp\left(-\frac{1}{2\sigma^2}||y-X\beta||^2\right)
	\end{equation*}
	于是对数似然函数为:
	\begin{equation*}
		\ln L(\beta,\sigma^2)=-\frac{n}{2}\ln(2\pi)-\frac{n}{2}\ln(\sigma^2)-\frac{1}{2\sigma^2}||y-X\beta||^2
	\end{equation*}
	固定$\sigma^2$时,由最小二乘法原理可知:
	\begin{equation*}
		||y-X\hat{\beta}||^2=\min||y-X\beta||^2
	\end{equation*}
	当$\beta=\hat{\beta}$时有:
	\begin{equation*}
		\ln L(\hat{\beta},\sigma^2)=-\frac{n}{2}\ln(2\pi)-\frac{n}{2}\ln(\sigma^2)-\frac{1}{2\sigma^2}||y-X\hat{\beta}||^2
	\end{equation*}
	由极值点的必要条件可知:
	\begin{gather*}
		\frac{\dif\ln L(\hat{\beta},\sigma^2)}{\dif \sigma^2}=-\frac{n}{2\sigma^2}+\frac{||y-X\hat{\beta}||^2}{2\sigma^4} \\
		\tilde{\sigma}^2=\frac{1}{n}||y-X\hat{\beta}||^2
	\end{gather*}
	时对数似然函数取极值,注意到:
	\begin{gather*}
		\frac{\dif^2\ln L(\hat{\beta},\sigma^2)}{\dif (\sigma^2)^2}=\frac{n}{2\sigma^4}-\frac{||y-X\hat{\beta}||^2}{\sigma^6} \\
		\begin{aligned}
			\frac{\dif^2\ln L(\hat{\beta},\sigma^2)}{\dif (\sigma^2)^2}\Big|_{\sigma^2=\tilde{\sigma}^2}&=\frac{n^3}{2||y-X\hat{\beta}||^4}-\frac{||y-X\hat{\beta}||^2n^3}{||y-X\hat{\beta}||^6} \\
			&=\frac{n^3}{2||y-X\hat{\beta}||^4}-\frac{n^3}{||y-X\hat{\beta}||^4}=-\frac{n^3}{2||y-X\hat{\beta}||^4}<0
		\end{aligned}
	\end{gather*}
	于是此处取极大值。因为:
	\begin{equation*}
		\tilde{\sigma}^2=\frac{1}{n}||y-X\hat{\beta}||^2=\frac{n-r}{n}\hat{\sigma}^2
	\end{equation*}
	所以$\tilde{\sigma}^2$是$\sigma^2$的MLE。由上可知$\hat{\beta}$是$\beta$的MLE,根据\cref{prop:MLE}(1)可得$c^T\hat{\beta}$是$c^T\beta$的MLE。\par
	约束条件下的情况与上述证明过程类似。\par
	(2)因为$\varepsilon\sim\operatorname{N}_n(\mathbf{0},\sigma^2I_n)$,而$c^T\hat{\beta}=c^T(X^TX)^-X^Ty=c^T(X^TX)^-X^T(X\beta+\varepsilon)$。因为$c^T\beta$是可估函数,所以由\cref{prop:EstimableFunction}(1)可知存在$\alpha$使得$c=X^T\alpha$,根据\cref{prop:A-}(7)可知:
	\begin{equation*}
		c^T(X^TX)^-X^TX\beta=c^T\beta
	\end{equation*}
	由\cref{prop:A-}(6)(7)可知:
	\begin{equation*}
		c^T(X^TX)^-X^T[c^T(X^TX)^-X^T]^T=c^T(X^TX)^-X^TX(X^TX)^-c=c^T(X^TX)^-c
	\end{equation*}
	于是由\cref{theo:MultiNormalLinearTransform}可得:
	\begin{equation*}
		c^T\hat{\beta}\sim\operatorname{N}[c^T\beta,c^T(X^TX)^-c]
	\end{equation*}\par
	因为$(I_n-P_X)X=\mathbf{0}$,根据\cref{prop:OrthogonalProjectionMat}(4)可得$I_n-P_X$是对称阵,所以由\cref{prop:ehat}(2)可知:
	\begin{align*}
		\frac{n-r}{\sigma^2}\hat{\sigma}^2&=\frac{\hat{e}^T\hat{e}}{\sigma^2}=\frac{y^T(I_n-P_X)y}{\sigma^2} \\
		&=\frac{(X\beta+\varepsilon)^T(I_n-P_X)(X\beta+\varepsilon)}{\sigma^2} \\
		&=\frac{(X\beta+\varepsilon)^T(I_n-P_X)X\beta+(X\beta+\varepsilon)^T(I_n-P_X)\varepsilon}{\sigma^2} \\
		&=\frac{(X\beta+\varepsilon)^T(I_n-P_X)\varepsilon}{\sigma^2}=\frac{\beta^TX^T(I_n-P_X)\varepsilon+\varepsilon^T(I_n-P_X)\varepsilon}{\sigma^2} \\
		&=\frac{\beta^T[(I_n-P_X)^TX]^T\varepsilon+\varepsilon^T(I_n-P_X)\varepsilon}{\sigma^2}=\frac{\varepsilon^T(I_n-P_X)\varepsilon}{\sigma^2}
	\end{align*}
	因为$\varepsilon\sim\operatorname{N}_n(\mathbf{0},\sigma^2I_n)$,由\cref{theo:MatNormalLinearTransform}可知$\dfrac{\varepsilon}{\sigma}\sim\operatorname{N}_n(\mathbf{0},I_n)$。根据\cref{prop:OrthogonalProjectionMat}(4)可知$I_n-P_X$是对称幂等阵,由\cref{prop:IdempotentMat}(3)和\cref{prop:OrthogonalProjectionMat}(1)可得$\operatorname{rank}(I_n-P_X)=n-r$,于是根据\cref{theo:XAXChi2}可得:
	\begin{equation*}
		\frac{n-r}{\sigma^2}\hat{\sigma}^2=\frac{\varepsilon^T(I_n-P_X)\varepsilon}{\sigma^2}\sim\chi_{n-r}^2
	\end{equation*}\par
	(3)由\cref{prop:ehat}(2)可知:
	\begin{equation*}
		c^T\hat{\beta}=c^T(X^TX)^-X^Ty,\quad\hat{\sigma}^2=\frac{y^T(I_n-P_X)y}{n-r}
	\end{equation*}
	由\cref{prop:OrthogonalProjectionMat}(4)可知$I_n-P_X$为对称阵,所以$\dfrac{I_n-P_X}{n-r}$也是对称阵。因为:
	\begin{align*}
		c^T(X^TX)^-X^T\frac{I_n-P_X}{n-r}&=\frac{1}{n-r}c^T(X^TX)^-X^T(I_n-P_X) \\
		&=\frac{1}{n-r}c^T(X^TX)^-[(I_n-P_X)X]^T=\mathbf{0}
	\end{align*}
	由\cref{theo:BXXAXIndependent}可知$c^T\hat{\beta}$与$\hat{\sigma}^2$独立。\par
	(4)
\end{proof}

\subsection{假设检验}
\begin{theorem}
	对于\cref{model:NormalLinearModel},假设:
	\begin{equation*}
		A\beta=b,\quad A\in M_{k\times p}(K),\quad\operatorname{rank}(A)=k,\quad\mathcal{M}(A^T)\subseteq\mathcal{M}(X^T)
	\end{equation*}
	且$A\beta=b$相容,则:
	\begin{enumerate}
		\item 似然比检验$H_0:A\beta=b,\;H_1:A\beta\ne b$的似然比为:
		\begin{equation*}
			\lambda(y)=\left(\frac{SSE_A}{SSE}\right)^\frac{n}{2}
		\end{equation*}
		\item $\dfrac{SSE_A-SSE}{\sigma^2}\sim\chi^2_{k,\alpha}$,其中:
		\begin{equation*}
			\alpha=\frac{(A\beta-b)^T[A(X^TX)^-A^T]^{-1}(A\beta-b)}{\sigma^2}
		\end{equation*}
		\item $SSE_A-SSE$与$SSE$相互独立;
		\item 当$A\beta=b$为真时,
		\begin{equation*}
			F=\frac{(SSE_A-SSE)/k}{SSE/(n-r)}\sim F_{k,n-r}
		\end{equation*}
		且上式左侧为$\lambda(y)$的单调增函数;
		\item 似然比检验$H_0:A\beta=b,\;H_1:A\beta\ne b$的拒绝域为$\{F:F>F_{k,n-r}(\alpha)\}$。
	\end{enumerate}
\end{theorem}
\begin{proof}
	(1)由\cref{prop:NormalLinearModel}(1)可知:
	\begin{gather*}
		\begin{aligned}
			\sup_{\beta,\sigma}L(\beta,\sigma^2;y)&=L(\hat{\beta},\tilde{\sigma}^2;y)=(2\pi)^{-\frac{n}{2}}(\tilde{\sigma}^2)^{-\frac{n}{2}}\exp\left(-\frac{||y-X\hat{\beta}||^2}{2\tilde{\sigma}^2}\right) \\
			&=(2\pi)^{-\frac{n}{2}}\left(\frac{||y-X\hat{\beta}||^2}{n}\right)^{-\frac{n}{2}}\exp\left(-\frac{n||y-X\hat{\beta}||^2}{2||y-X\hat{\beta}||^2}\right) \\
			&=(2\pi)^{-\frac{n}{2}}\left(\frac{||y-X\hat{\beta}||^2}{n}\right)^{-\frac{n}{2}}\exp\left(-\frac{n}{2}\right)=\left(\frac{2\pi e}{n}\right)^{-\frac{n}{2}}||y-X\hat{\beta}||^{-n}
		\end{aligned} \\
		\sup_{A\beta=b,\sigma^2}L(\beta,\sigma^2;y)=L(\hat{\beta}_A,\tilde{\sigma}_A^2;y)=\left(\frac{2\pi e}{n}\right)^{-\frac{n}{2}}||y-X\hat{\beta}_A||^{-n}
	\end{gather*}
	于是:
	\begin{equation*}
		\lambda(y)=\frac{L(\hat{\beta},\tilde{\sigma}^2;y)}{L(\hat{\beta}_A,\tilde{\sigma}_A^2;y)}=\left(\frac{SSE_A}{SSE}\right)^\frac{n}{2}
	\end{equation*}\par
	(2)根据\cref{prop:A-}(6)(7)和\cref{prop:OrthogonalProjectionMat}(4)以及$(I_n-P_X)X=\mathbf{0}$对$SSE_A$作分解:
	\begin{align*}
		SSE_A&=||y-X\hat{\beta}_A||^2=\Big\|y-X\{\hat{\beta}-(X^TX)^-A^T[A(X^TX)^-A^T]^{-1}(A\hat{\beta}-b)\}\Big\|^2 \\
		&=\Big\|y-X\hat{\beta}+X(X^TX)^-A^T[A(X^TX)^-A^T]^{-1}(A\hat{\beta}-b)\Big\|^2 \\
		&=||y-X\hat{\beta}||^2+2(y-X\hat{\beta})^TX(X^TX)^-A^T[A(X^TX)^-A^T]^{-1}(A\hat{\beta}-b) \\
		&\quad+\{X(X^TX)^-A^T[A(X^TX)^-A^T]^{-1}(A\hat{\beta}-b)\}^T \\
		&\quad\cdot X(X^TX)^-A^T[A(X^TX)^-A^T]^{-1}(A\hat{\beta}-b) \\
		&=||y-X\hat{\beta}||^2+2[y-X(X^TX)^-X^Ty]^TX(X^TX)^-A^T[A(X^TX)^-A^T]^{-1}(A\hat{\beta}-b) \\
		&\quad+(A\hat{\beta}-b)^T[A(X^TX)^-A^T]^{-1}A(X^TX)^-X^T \\
		&\quad\cdot X(X^TX)^-A^T[A(X^TX)^-A^T]^{-1}(A\hat{\beta}-b) \\
		&=||y-X\hat{\beta}||^2+2[(I_n-P_X)y]^TX(X^TX)^-A^T[A(X^TX)^-A^T]^{-1}(A\hat{\beta}-b) \\
		&\quad+(A\hat{\beta}-b)^T[A(X^TX)^-A^T]^{-1}A(X^TX)^-A^T[A(X^TX)^-A^T]^{-1}(A\hat{\beta}-b) \\
		&=||y-X\hat{\beta}||^2+2y^T(I_n-P_X)X(X^TX)^-A^T[A(X^TX)^-A^T]^{-1}(A\hat{\beta}-b) \\
		&\quad+(A\hat{\beta}-b)^T[A(X^TX)^-A^T]^{-1}(A\hat{\beta}-b) \\
		&=||y-X\hat{\beta}||^2+(A\hat{\beta}-b)^T[A(X^TX)^-A^T]^{-1}(A\hat{\beta}-b)
	\end{align*}
	所以有:
	\begin{equation*}
		SSE_A-SSE=(A\hat{\beta}-b)^T[A(X^TX)^-A^T]^{-1}(A\hat{\beta}-b)
	\end{equation*}
	由\cref{prop:NormalLinearModel}(2)可知:
	\begin{equation*}
		A\hat{\beta}-b\sim\operatorname{N}_k[A\beta-b,\sigma^2A(X^TX)^-A^T]
	\end{equation*}
	根据\info{$A(X^TX)^-A^T$的正定性}可知$A(X^TX)^-A^T$存在平方根阵及平方根阵的逆矩阵,由\cref{cor:MultiNormalLinearTransform}(1)可知:
	\begin{equation*}
		\frac{[A(X^TX)^-A^T]^{-\frac{1}{2}}}{\sigma}(A\hat{\beta}-b)\sim\operatorname{N}_k\{[A(X^TX)^-A^T]^{-\frac{1}{2}}(A\beta-b),I_n\}
	\end{equation*}
	于是由$\chi^2$分布的定义可得:
	\begin{equation*}
		\frac{SSE_A-SSE}{\sigma^2}=\frac{\{[A(X^TX)^-A^T]^{-\frac{1}{2}}(A\hat{\beta}-b)\}^T[A(X^TX)^-A^T]^{-\frac{1}{2}}(A\hat{\beta}-b)}{\sigma^2}\sim\chi_{k,\alpha}^2
	\end{equation*}
	其中:
	\begin{align*}
		\alpha&=\{[A(X^TX)^-A^T]^{-\frac{1}{2}}(A\beta-b)\}^T[A(X^TX)^-A^T]^{-\frac{1}{2}}(A\beta-b) \\
		&=(A\beta-b)^T[A(X^TX)^-A^T]^{-1}(A\beta-b)
	\end{align*}\par
	(3)由\cref{prop:ehat}(2)可知$SSE=y^T(I_n-P_X)y$,由(2)的过程可得:
	\begin{align*}
		SSE_A-SSE&=(A\hat{\beta}-b)^T[A(X^TX)^-A^T]^{-1}(A\hat{\beta}-b) \\
		&=[A(X^TX)^-X^Ty-b]^T[A(X^TX)^-A^T]^{-1}[A(X^TX)^-X^Ty-b] \\
		&=y^TX(X^TX)^-A^T[A(X^TX)^-A^T]^{-1}A(X^TX)^-X^Ty \\
		&\quad-2b^T[A(X^TX)^-A^T]^{-1}A(X^TX)^-X^Ty+b^T[A(X^TX)^-A^T]^{-1}b
	\end{align*}
	因为$(I_n-P_X)X=\mathbf{0}$,由\cref{prop:OrthogonalProjectionMat}(4)可得:
	\begin{gather*}
		(I_n-P_X)X(X^TX)^-A^T[A(X^TX)^-A^T]^{-1}A(X^TX)^-X^T=\mathbf{0} \\
		\begin{aligned}
			&2b^T[A(X^TX)^-A^T]^{-1}A(X^TX)^-X^T(I_n-P_X) \\
			=&2b^T[A(X^TX)^-A^T]^{-1}A(X^TX)^-[(I_n-P_X)X]^T=\mathbf{0}
		\end{aligned}
	\end{gather*}
	于是根据\cref{theo:BXXAXIndependent}和\cref{theo:XAXXBXIndependent}可知$SSE_A-SSE$与$SSE$独立。\par
	(4)当$A\beta=b$为真时,由(2)可知$\dfrac{SSE_A-SSE}{\sigma^2}\sim\chi^2_{k}$。根据(3)和\cref{prop:NormalLinearModel}(2)可得:
	\begin{equation*}
		\frac{(SSE_A-SSE)/(k\sigma^2)}{SSE/[(n-r)\sigma^2]}=\frac{(SSE_A-SSE)/k}{SSE/(n-r)}\sim F_{k,n-r}\qedhere
	\end{equation*}
	由(1)可得:
	\begin{equation*}
		\frac{(SSE_A-SSE)/k}{SSE/(n-r)}=\frac{n-r}{k}[\lambda^{\frac{2}{n}}(y)-1]
	\end{equation*}
	所以它是$\lambda(y)$的单调增函数。\par
	(5)由(1)(4)可立即得出。
\end{proof}

\subsection{置信域}
\subsubsection{置信椭球}
\begin{theorem}\label{theo:NormalLinearModelConfidenceEllipsoid}
	对于\cref{model:NormalLinearModel},若不能接受假设$A\beta=\mathbf{0}$,则$A\beta$置信度为$1-\alpha$的置信椭球为:
	\begin{equation*}
		\{A\beta:(A\beta-A\hat{\beta})^T[A(X^TX)^-A^T]^{-1}(A\beta-A\hat{\beta})\leqslant k\hat{\sigma}^2F_{k,n-r}(\alpha)\}
	\end{equation*}
\end{theorem}
\begin{proof}
	由\cref{prop:NormalLinearModel}(2)可知:
	\begin{equation*}
		A\hat{\beta}\sim\operatorname{N}[A\beta,\sigma^2A(X^TX)^-A^T]
	\end{equation*}
	所以:
	\begin{equation*}
		\frac{A\hat{\beta}-A\beta}{\sigma}\sim\operatorname{N}[\mathbf{0},A(X^TX)^-A^T]
	\end{equation*}
	因为:
	\begin{equation*}
		[A(X^TX)^-A^T]^{-1}A(X^TX)^-A^T[A(X^TX)^-A^T]^{-1}=[A(X^TX)^-A^T]^{-1}
	\end{equation*}
	所以由\cref{cor:XAXChi2}可知:
	\begin{equation*}
		\frac{(A\hat{\beta}-A\beta)^T[A(X^TX)^-A^T]^{-1}(A\hat{\beta}-A\beta)}{\sigma^2}\sim\chi_{k}^2
	\end{equation*}
	根据\cref{prop:NormalLinearModel}(2)(3)可得:
	\begin{align*}
		&\frac{(A\hat{\beta}-A\beta)^T[A(X^TX)^-A^T]^{-1}(A\hat{\beta}-A\beta)}{k\sigma^2}\Big/\frac{(n-r)\hat{\sigma}^2}{(n-r)\sigma^2} \\
		=&\frac{(A\hat{\beta}-A\beta)^T[A(X^TX)^-A^T]^{-1}(A\hat{\beta}-A\beta)}{k\hat{\sigma}^2}\sim F_{k,n-r}
	\end{align*}
	所以对任意的$0<\alpha<1$,有:
	\begin{equation*}
		P\left[\frac{(A\hat{\beta}-A\beta)^T[A(X^TX)^-A^T]^{-1}(A\hat{\beta}-A\beta)}{k\hat{\sigma}^2}\leqslant F_{k,n-r}(\alpha)\right]=1-\alpha
	\end{equation*}
	即$A\beta$置信度为$1-\alpha$的置信椭球为:
	\begin{equation*}
		\{A\beta:(A\beta-A\hat{\beta})^T[A(X^TX)^-A^T]^{-1}(A\beta-A\hat{\beta})\leqslant k\hat{\sigma}^2F_{k,n-r}(\alpha)\}\qedhere
	\end{equation*}
\end{proof}
\begin{note}
	这里其实是一个未知方差构造$F$分布的思想。
\end{note}
\subsubsection{Scheffe置信区间}
\begin{theorem}\label{theo:ScheffeCI}
	对于\cref{model:NormalLinearModel},对任何可估函数$l^T\beta$,其中$l\in\mathcal{M}(A^T)$且$l\ne\mathbf{0}$,其置信度为$1-\alpha$的同时置信区间为:
	\begin{equation*}
		l^T\hat{\beta}\pm[k\hat{\sigma}^2F_{k,n-r}(\alpha)l^T(X^TX)^-l]^{\frac{1}{2}}
	\end{equation*}
\end{theorem}
\begin{proof}
	由\cref{theo:NormalLinearModelConfidenceEllipsoid}、\info{$A(X^TX)^-A^T$的正定性}与\cref{ineq:cauchy-schiwarz-rayleigh}可知:
	\begin{align*}
		1-\alpha&=P\left[(A\beta-A\hat{\beta})^T[A(X^TX)^-A^T]^{-1}(A\beta-A\hat{\beta})\leqslant k\hat{\sigma}^2F_{k,n-r}(\alpha)\right] \\
		&=P\left\{\sup_{b\ne\mathbf{0}}\frac{[(A\hat{\beta}-A\beta)^Tb]^2}{b^TA(X^TX)^-A^Tb}\leqslant k\hat{\sigma}^2F_{k,n-r}(\alpha)\right\} \\
		&=P\left\{\frac{[(A\hat{\beta}-A\beta)^Tb]^2}{b^TA(X^TX)^-A^Tb}\leqslant k\hat{\sigma}^2F_{k,n-r}(\alpha),\;\text{对任意的}b\ne\mathbf{0}\right\} \\ 
		&=P\left\{|(A\hat{\beta}-A\beta)^Tb|\leqslant [k\hat{\sigma}^2F_{k,n-r}(\alpha)b^TA(X^TX)^-A^Tb]^{\frac{1}{2}},\;\text{对任意的}b\ne\mathbf{0}\right\} \\
		&=P\left\{|\hat{\beta}^TA^Tb-\beta^TA^Tb|\leqslant [k\hat{\sigma}^2F_{k,n-r}(\alpha)b^TA(X^TX)^-A^Tb]^{\frac{1}{2}},\;\text{对任意的}b\ne\mathbf{0}\right\}
	\end{align*}
	记$A^Tb=l$,因为$\mathcal{M}(A^T)\subseteq\mathcal{M}(X^T)$,由\cref{prop:EstimableFunction}(1)可知$l^T\beta$也是一个可估函数,于是有:
	\begin{equation*}
		1-\alpha=P\left\{|l^T\hat{\beta}-l^T\beta|\leqslant [k\hat{\sigma}^2F_{k,n-r}(\alpha)l^T(X^TX)^-l]^{\frac{1}{2}},\;\text{对任意的}l\in\mathcal{M}(A^T)\text{且}l\ne\mathbf{0}\right\}\qedhere
	\end{equation*}
\end{proof}
\subsubsection{Bonferroni置信区间}
\begin{theorem}\label{theo:BonferroniCI}
	对于\cref{model:NormalLinearModel},记$A$的行分别为$\seq{a^T}{k}$,则$a_
	i^T\beta$置信度为$1-\alpha$的Bonferroni置信区间为:
	\begin{equation*}
		a_i^T\hat{\beta}\pm t_{n-r}\left(\frac{\alpha}{2k}\right)[\hat{\sigma}^2a_i^T(X^TX)^-a_i]^{\frac{1}{2}}
	\end{equation*}
\end{theorem}
\begin{proof}
	由\cref{theo:NormalLinearModelConfidenceEllipsoid}可得当$k=1$时有:
	\begin{align*}
		1-\alpha&=P\left\{(a_i^T\beta-a_i^T\hat{\beta})^T[a_i^T(X^TX)^-a_i]^{-1}(a_i^T\beta-a_i^T\hat{\beta})\leqslant\hat{\sigma}^2F_{1,n-r}(\alpha)\right\} \\
		&=P\left\{(a_i^T\beta-a_i^T\hat{\beta})^2\leqslant\hat{\sigma}^2F_{1,n-r}(\alpha)[a_i^T(X^TX)^-a_i]\right\}
	\end{align*}
	由\cref{prop:FDistribution}(2)可知$F_{1,n-r}=t^2_{n-r}$,因为服从$t$分布的变量可取负值而服从$F$分布的变量只能为正值,所以上式把平方变成绝对值时应修改对应的$\alpha$为$\dfrac{a}{2}$,即此时:
	\begin{equation*}
		1-\alpha=P\left\{|a_i^T\beta-a_i^T\hat{\beta}|\leqslant t_{n-r}\left(\frac{\alpha}{2}\right)[\hat{\sigma}^2a_i^T(X^TX)^-a_i]^{\frac{1}{2}}\right\}
	\end{equation*}
	由Bonferroni校正法可得出结论。
\end{proof}
\subsubsection{比较}
\begin{derivation}
	Scheffe区间与Bonferroni区间哪个更好?由二者的公式可以看出只需选择:
	\begin{equation*}
		\min\left\{[k\hat{\sigma}^2F_{k,n-r}(\alpha)]^{\frac{1}{2}},\hat{\sigma}t_{n-r}\left(\frac{\alpha}{2k}\right)\right\}
	\end{equation*}
	对应的方法即能得到更短的置信区间。
\end{derivation}

\subsection{区间预测}
\begin{theorem}
	延续\cref{def:LinearModelForcast}处的定义,假设$\varepsilon\sim\operatorname{N}(0,\sigma^2I_n),\;\varepsilon_0\sim\operatorname{N}(0,\sigma^2I_m)$,且有
	\begin{equation*}
		\operatorname{Cov}[(y,y_0)^T]=\operatorname{Cov}[(\varepsilon,\varepsilon_0)^T]=\sigma^2
		\begin{pmatrix}
			I_n & V^T \\
			V & I_m 
		\end{pmatrix}
	\end{equation*}
	则$y_0^{(i)}$的Scheffe置信区间与Bonferroni置信区间分别为:
	\begin{equation*}
		\hat{y}_0^{(i)}\pm[m\hat{\sigma}^2F_{m,n-r}(\alpha)T_{ii}]^{\frac{1}{2}},\quad
		\hat{y}_0^{(i)}\pm t_{n-r}\left(\frac{\alpha}{2m}\right)\hat{\sigma}(T_{ii})^{\frac{1}{2}}
	\end{equation*}
	其中$y_0^{(i)},\hat{y}_0^{(i)}$表示的是$y_0,\hat{y}_0$的第$i$个分量,$\hat{y}_0$由\cref{theo:PECorrelatedLinearModelForcast}给出,$T_{ii}$表示矩阵$T$的$i,i$元,$T=(X_0-VX)(X^TX)^-(X_0-VX)^T+I_m-VV^T$。
\end{theorem}
\begin{proof}
	由\cref{theo:PECorrelatedLinearModelForcast}、\cref{theo:MultiNormalLinearTransform}、\cref{prop:CovMat}(3)(5)和\cref{prop:A-}(6)(7)(5)可得:
	\begin{equation*}
		\hat{y}_0=X_0\hat{\beta}+V(y-X\hat{\beta})\sim\operatorname{N}_m(X_0\beta,\sigma^2VV^T)
	\end{equation*}
	\begin{align*}
		\operatorname{Cov}(\hat{y}_0)&=\operatorname{Cov}(Cy)=C\operatorname{Cov}(y)C^T=\sigma^2CC^T \\
		&=\sigma^2[X_0(X^TX)^-X^T+V-VX(X^TX)^-X^T] \\
		&\quad\cdot[X(X^TX)^-X_0^T+V^T-X(X^TX)^-X^TV^T] \\
		&=\sigma^2[X_0(X^TX)^-X^TX(X^TX)^-X_0^T+X_0(X^TX)^-X^TV^T \\
		&\quad-X_0(X^TX)^-X^TX(X^TX)^-X^TV^T+VX(X^TX)^-X_0^T \\
		&\quad+VV^T-VX(X^TX)^-X^TV^T-VX(X^TX)^-X^TX(X^TX)^-X_0^T \\
		&\quad-VX(X^TX)^-X^TV^T+VX(X^TX)^-X^TX(X^TX)^-X^TV^T] \\
		&=\sigma^2[X_0(X^TX)^-X_0^T+X_0(X^TX)^-X^TV^T-X_0(X^TX)^-X^TV^T \\
		&\quad+VX(X^TX)^-X_0^T+VV^T-VX(X^TX)^-X^TV^T \\
		&\quad-VX(X^TX)^-X_0^T-VX(X^TX)^-X^TV^T+VX(X^TX)^-X^TV^T] \\
		&=\sigma^2[X_0(X^TX)^-X_0^T+VV^T-VX(X^TX)^-X^TV^T]
	\end{align*}
	\begin{align*}
		\operatorname{Cov}(\hat{y}_0,y_0)
		&=\operatorname{Cov}[X_0\hat{\beta}+V(y-X\hat{\beta}),y_0]=\operatorname{Cov}[Vy+(X_0-VX)\hat{\beta},y_0] \\
		&=V\operatorname{Cov}(y,y_0)+(X_0-VX)\operatorname{Cov}(\hat{\beta},y_0) \\
		&=V\operatorname{Cov}(y,y_0)+(X_0-VX)\operatorname{Cov}[(X^TX)^-X^Ty,y_0] \\
		&=V\operatorname{Cov}(y,y_0)+(X_0-VX)(X^TX)^-X^T\operatorname{Cov}\operatorname{Cov}(y,y_0) \\
		&=[V+(X_0-VX)(X^TX)^-X^T]\operatorname{Cov}(y,y_0) \\
		&=\sigma^2[V+(X_0-VX)(X^TX)^-X^T]V^T
	\end{align*}
	令$A=X_0(X^TX)^-X_0^T+VV^T-VX(X^TX)^-X^TV^T,\;B=[V+(X_0-VX)(X^TX)^-X^T]V^T$,所以:
	\begin{equation*}
		\begin{pmatrix}
			\hat{y}_0 \\
			y_0
		\end{pmatrix}\sim\operatorname{N}_{m+n}\left[
		\begin{pmatrix}
			X_0\beta \\
			X_0\beta
		\end{pmatrix},\sigma^2
		\begin{pmatrix}
			A & B \\
			B^T & I_m
		\end{pmatrix}
		\right]
	\end{equation*}
	由\cref{theo:MultiNormalLinearTransform}可知:
	\begin{equation*}
		\hat{y}_0-y_0=
		(1,-1)
		\begin{pmatrix}
			\hat{y}_0 \\
			y_0
		\end{pmatrix}\sim\operatorname{N}_m\{
		\mathbf{0},\sigma^2[(X_0-VX)(X^TX)^-(X_0-VX)+I_m-VV^T]\}
	\end{equation*}\par
	令$T=(X_0-VX)(X^TX)^-(X_0-VX)^T+I_m-VV^T$,仿照\cref{theo:NormalLinearModelConfidenceEllipsoid}中的推导能得到此时$\hat{y}_0-y_0$置信度为$1-\alpha$的置信椭球为:
	\begin{equation*}
		[\hat{y}_0-y_0:(\hat{y}_0-y_0)^T(T)^{-1}(\hat{y}_0-y_0)\leqslant m\hat{\sigma}^2F_{m,n-r}(\alpha)]
	\end{equation*}
	其中$T$的可逆性由\info{T的可逆性证明}保证。\par
	由上述置信椭球,仿照\cref{theo:ScheffeCI}的推导,在其中取$b$为标准基向量即可得到$y_0^{(i)}$的Scheffe置信区间为:
	\begin{equation*}
		\hat{y}_0^{(i)}\pm[m\hat{\sigma}^2F_{m,n-r}(\alpha)T_{ii}]^{\frac{1}{2}}
	\end{equation*}\par
	同理可得到$y_0^{(i)}$的Bonferroni置信区间为:
	\begin{equation*}
		\hat{y}_0^{(i)}\pm t_{n-r}\left(\frac{\alpha}{2m}\right)\hat{\sigma}(T_{ii})^{\frac{1}{2}}\qedhere
	\end{equation*}
\end{proof}


\section{误差协方差推广}
在很多情况下线性模型误差的协方差矩阵都不是$\sigma^2I_n$的形式。
\subsection{广义最小二乘估计}
\begin{definition}\label{model:GeneralizedLinearModel}
	称以下模型为\gls{GeneralizedLinearModel}:
	\begin{equation*}
		\begin{cases}
			y=X\beta+\varepsilon \\
			\operatorname{E}(\varepsilon)=\mathbf{0} \\
			\operatorname{Cov}(\varepsilon)=\sigma^2\Sigma
		\end{cases}
	\end{equation*}
	其中$y$为$n\times 1$观测向量,$X$为$n\times p$设计矩阵,$\beta$为$p\times 1$未知参数向量,$\varepsilon$为随机误差,$\sigma^2\Sigma$为误差协方差矩阵且$\Sigma>0$。
\end{definition}
\begin{derivation}
	因为$\Sigma>0$,所以存在$\Sigma^{-\frac{1}{2}}$。令:
	\begin{equation*}
		y^*=\Sigma^{-\frac{1}{2}}y,\quad X^*=\Sigma^{-\frac{1}{2}}X,\quad\varepsilon^*=\Sigma^{-\frac{1}{2}}\varepsilon
	\end{equation*}
	由\cref{prop:CovMat}(3)可知\cref{model:GeneralizedLinearModel}可化作:
	\begin{equation*}
		\begin{cases}
			y^*=X^*\beta+\varepsilon^* \\
			\operatorname{E}(\varepsilon^*)=\mathbf{0} \\
			\operatorname{Cov}(\varepsilon^*)=\sigma^2I_n
		\end{cases}
	\end{equation*}
	于是我们可以将广义线性模型化作线性模型来处理,由于可估函数的定义与协方差矩阵无关,所以对于线性模型与正态线性模型的那些结论,广义线性模型也可得到。
\end{derivation}

\subsection{最小二乘统一理论}


\section{线性回归模型}
\begin{definition}\label{model:LinearRegressionModel}
	设因变量$Y$和自变量$\seq{X}{p-1}$满足:
	\begin{equation*}
		Y=\beta_0+\beta_1X_1+\beta_2X_2+\cdots+\beta_{p-1}X_{p-1}+e
	\end{equation*}
	若对因变量$Y$和自变量$\seq{X}{p-1}$进行了$n$次观察,得到$n$组数据,它们满足:
	\begin{equation*}
		y_i=\beta_0+\beta_1x_{i1}+\beta_2x_{i2}+\cdots+\beta_{p-1}x_{ip-1}+\varepsilon_i,\quad i=1,2,\dots,n
	\end{equation*}
	记:
	\begin{equation*}
		y=
		\begin{pmatrix}
			y_1 \\
			y_2 \\
			\vdots \\
			y_n
		\end{pmatrix},\quad
		X=
		\begin{pmatrix}
			1 & x_{11} & \cdots & x_{1p-1} \\
			1 & x_{21} & \cdots & x_{2p-1} \\
			\vdots & \vdots & \ddots & \vdots \\
			1 & x_{n1} & \cdots & x_{np-1} \\
		\end{pmatrix},\quad
		\beta=
		\begin{pmatrix}
			\beta_0 \\
			\beta_1 \\
			\vdots \\
			\beta_{p-1}
		\end{pmatrix},\quad
		\varepsilon=
		\begin{pmatrix}
			\varepsilon_1 \\
			\varepsilon_2 \\
			\vdots \\
			\varepsilon_n
		\end{pmatrix}
	\end{equation*}
	且假设$\operatorname{rank}(X)=p,\;\operatorname{E}(\varepsilon)=\mathbf{0},\;\operatorname{Cov}(\varepsilon)=\sigma^2I_n$,则得到\gls{LinearRegressionModel}:
	\begin{equation*}
		y=X\beta+\varepsilon,\quad\operatorname{E}(\varepsilon)=\mathbf{0},\quad\operatorname{Cov}(\varepsilon)=\sigma^2I_n
	\end{equation*}
	称$\beta_0$为常数项,$\beta_I=(\seq{\beta}{p-1})^T$为回归系数。若记$X=(\mathbf{1}_n,\tilde{X})$,其中$\mathbf{1}_n$为由$n$个$1$构成的列向量,则线性回归模型可被改写为:
	\begin{equation*}
		y=\mathbf{1}_n\beta_0+\tilde{X}\beta_I+\varepsilon,\quad\operatorname{E}(\varepsilon)=\mathbf{0},\quad\operatorname{Cov}(\varepsilon)=\sigma^2I_n
	\end{equation*}
\end{definition}
\subsubsection{中心化与标准化处理}
\begin{definition}
	对于\cref{model:LinearRegressionModel},记:
	\begin{equation*}
		\bar{x}_j=\frac{1}{n}\sum_{i=1}^{n}x_{ij},\quad s_j^2=\sum_{i=1}^{n}(x_{ij}-\bar{x}_j)^2
	\end{equation*}
\end{definition}