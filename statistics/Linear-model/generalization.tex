\section{误差协方差推广}
在很多情况下线性模型误差的协方差矩阵都不是$\sigma^2I_n$的形式。
\subsection{广义最小二乘估计}
\begin{definition}\label{model:GeneralizedLinearModel}
	称以下模型为\gls{GeneralizedLinearModel}:
	\begin{equation*}
		\begin{cases}
			y=X\beta+\varepsilon \\
			\operatorname{E}(\varepsilon)=\mathbf{0} \\
			\operatorname{Cov}(\varepsilon)=\sigma^2\Sigma
		\end{cases}
	\end{equation*}
	其中$y$为$n\times 1$观测向量,$X$为$n\times p$设计矩阵,$\beta$为$p\times 1$未知参数向量,$\varepsilon$为随机误差,$\sigma^2\Sigma$为误差协方差矩阵且$\Sigma>\mathbf{0}$。
\end{definition}
\begin{derivation}
	因为$\Sigma>\mathbf{0}$,所以存在$\Sigma^{-\frac{1}{2}}$。令:
	\begin{equation*}
		y^{\star}=\Sigma^{-\frac{1}{2}}y,\quad X^{\star}=\Sigma^{-\frac{1}{2}}X,\quad\varepsilon^{\star}=\Sigma^{-\frac{1}{2}}\varepsilon
	\end{equation*}
	由\cref{prop:CovMat}(3)可知\cref{model:GeneralizedLinearModel}可化作:
	\begin{equation*}
		\begin{cases}
			y^{\star}=X^{\star}\beta+\varepsilon^{\star} \\
			\operatorname{E}(\varepsilon^{\star})=\mathbf{0} \\
			\operatorname{Cov}(\varepsilon^{\star})=\sigma^2I_n
		\end{cases}
	\end{equation*}
	于是我们可以将广义线性模型化作线性模型来处理,由于可估函数的定义与协方差矩阵无关,所以对于线性模型与正态线性模型的那些结论,广义线性模型也可得到。
\end{derivation}

\subsection{最小二乘统一理论}

当$\Sigma$不是正定矩阵的时候,广义最小二乘估计失效,最小二乘统一理论解决了这一问题。
\begin{lemma}
	对于模型:
	\begin{equation*}
		\begin{cases}
			y=X\beta+\varepsilon \\
			\operatorname{E}(\varepsilon)=\mathbf{0} \\
			\operatorname{Cov}(\varepsilon)=\sigma^2\Sigma
		\end{cases}
	\end{equation*}
	其中$y$为$n\times 1$观测向量,$X$为$n\times p$设计矩阵,$\beta$为$p\times 1$未知参数向量,$\varepsilon$为随机误差,$\sigma^2\Sigma$为误差协方差矩阵,有如下结论:
	\begin{enumerate}
		\item $y\in\mathcal{\Sigma,X}$;
	\end{enumerate}
\end{lemma}
\begin{proof}
	由\cref{prop:CovMat}(2)可得$\Sigma\geqslant0$,根据\cref{prop:HermitianMatEigen}(3)、\cref{prop:OrthogonalUnitaryMatrix}(1)和\cref{prop:Transpose}(4)可知:
	\begin{equation*}
		\Sigma=Q\operatorname{diag}\{\seq{\lambda}{n}\}Q^{\top}=Q\operatorname{diag}\{\seq{\lambda^{\frac{1}{2}}}{n}\}\operatorname{diag}\{\seq{\lambda^{\frac{1}{2}}}{n}\}Q^{\top}\coloneq LL^{\top}
	\end{equation*}
	记$\varepsilon=Le$,则
\end{proof}