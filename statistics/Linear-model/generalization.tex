\section{误差协方差推广}
在很多情况下线性模型误差的协方差矩阵都不是$\sigma^2I_n$的形式。
\subsection{广义最小二乘估计}
\begin{definition}\label{model:GeneralizedLinearModel}
	称以下模型为\gls{GeneralizedLinearModel}:
	\begin{equation*}
		\begin{cases}
			y=X\beta+\varepsilon \\
			\operatorname{E}(\varepsilon)=\mathbf{0} \\
			\operatorname{Cov}(\varepsilon)=\sigma^2\Sigma
		\end{cases}
	\end{equation*}
	其中$y$为$n\times 1$观测向量,$X$为$n\times p$设计矩阵,$\beta$为$p\times 1$未知参数向量,$\varepsilon$为随机误差,$\sigma^2\Sigma$为误差协方差矩阵且$\Sigma>0$。
\end{definition}
\begin{derivation}
	因为$\Sigma>0$,所以存在$\Sigma^{-\frac{1}{2}}$。令:
	\begin{equation*}
		y^*=\Sigma^{-\frac{1}{2}}y,\quad X^*=\Sigma^{-\frac{1}{2}}X,\quad\varepsilon^*=\Sigma^{-\frac{1}{2}}\varepsilon
	\end{equation*}
	由\cref{prop:CovMat}(3)可知\cref{model:GeneralizedLinearModel}可化作:
	\begin{equation*}
		\begin{cases}
			y^*=X^*\beta+\varepsilon^* \\
			\operatorname{E}(\varepsilon^*)=\mathbf{0} \\
			\operatorname{Cov}(\varepsilon^*)=\sigma^2I_n
		\end{cases}
	\end{equation*}
	于是我们可以将广义线性模型化作线性模型来处理,由于可估函数的定义与协方差矩阵无关,所以对于线性模型与正态线性模型的那些结论,广义线性模型也可得到。
\end{derivation}

\subsection{最小二乘统一理论}