\chapter{统计初步}

\section{基于测度论的统计简介}
\begin{definition}
	设$(X,\mathscr{A},P)$为概率空间,$f$是从可测空间$(X,\mathscr{A})$到可测空间$(Y,\mathscr{B})$上的随机变量。称概率测度$Pf^{-1}$为\gls{Population},$f$的观测和$f$都被称为\gls{Sample},称$f$的观测的数目为\gls{SampleSize},总体的数字特征即为$f$的数字特征。
\end{definition}
\begin{note}
	由定义可以看出,样本具有两重性,若将随机变量的观测看作样本,则样本是固定的,没有随机性,但在理论分析中我们往往研究的是具有随机性的样本,也即随机变量本身。换句话讲,抽样实施以前,样本被看作随机变量,抽样实施以后,样本是具体的。
\end{note}
\begin{definition}
	设$\mathscr{P}$为可测空间$(X,\mathscr{A})$上的一个概率测度族,则称$(X,\mathscr{A},\mathscr{P})$为\gls{StatisticalStructure}或\gls{StatisticalModel}。若$\mathscr{P}$仅依赖于参数$\theta$,即:
	\begin{equation*}
		\mathscr{P}=\{P_{\theta}:\theta\in\Theta\}
	\end{equation*}
	其中$\Theta$为参数空间,则称此结构为\gls{ParametricStructure},否则称为\gls{Non-parametricStructure}。
\end{definition}
\begin{note}
	引入统计结构是因为在统计学中我们往往没有总体$P$的全部信息,它是不确定的,但我们会根据现有信息去假设$P$可能是怎样的,从而给出一族可能的概率测度$\mathscr{P}$,因此产生了统计学的两大问题:参数估计和假设检验。二者都是在获得样本后对$\mathscr{P}$进行分析的统计工具,其中参数估计又可分为点估计和区间估计。点估计分析$\mathscr{P}$中哪一个概率测度最有可能是真实总体,区间估计是为了得到一个尽可能小的$\mathscr{P}_0$使得真实总体最有可能在这里面,假设检验分析对于$\mathscr{P}_1,\mathscr{P}_2\subset\mathscr{P}$,真实总体最有可能出现在哪一个之中。
\end{note}
\begin{definition}
	设$(X_1,\mathscr{A}_1,\mathscr{P}_1),(X_2,\mathscr{A}_2,\mathscr{P}_2),\dots,(X_n,\mathscr{A}_n,\mathscr{P}_n)$为$n$个统计结构,称:
	\begin{equation*}
		\left(\prod_{i=1}^nX_i,\sigma\left(\prod_{i=1}^n\mathscr{A}_i\right),\prod_{i=1}^n\mathscr{P}_i\right)
	\end{equation*}
	为它们的\gls{ProductStructure},其中$\prod$表示Cartisian积。
\end{definition}
\begin{definition}
	设$(X,\mathscr{A},\mathscr{P})$为统计结构。若可测空间$(X,\mathscr{A})$上存在一个$\sigma$有限测度$\mu$满足对任意的$P\in\mathscr{P}$有$P\ll\mu$,则称$(X,\mathscr{A},\mathscr{P})$是\gls{Controllable},称$\mu$为对应的\gls{ControllingMeasure}。
\end{definition}
\begin{definition}
	设$(X,\mathscr{A},\mathscr{P})$为统计结构,$T$是可测空间$(X,\mathscr{A})$到可测空间$(Y,\mathscr{B})$的可测映射。若$T$不依赖于$\mathscr{P}$,则称$T$为$(X,\mathscr{A},\mathscr{P})$上的\gls{Statistic}。统计量的分布称为\gls{SamplingDistribution}或\gls{InducedDistribution}。
\end{definition}
在测度论的可测映射那一小节我们提到了概率分布,接下来我们所提到的统计量的分布若无特殊说明都默认为是概率分布。

\subsection{次序统计量}
\begin{definition}
	设$\seq{f}{n}$为可测空间$(X,\mathscr{A})$上的随机变量,对每个$x\in X$,将其按大小排列为$f_{(1)}\leqslant f_{(2)}\leqslant\cdots\leqslant f_{(n)}$,若存在相等的情况则可随意排列,由此可得到新的函数$f_{(1)},f_{(2)},\dots,f_{(n)}$,称$(f_{(1)},f_{(2)},\dots,f_{(n)})$为$\seq{f}{n}$的\gls{OrderStatistic}。
\end{definition}
\begin{property}\label{prop:OrderStatistics}
	设$\seq{f}{n}$为可测空间$(X,\mathscr{A})$上的随机变量,$(f_{(1)},f_{(2)},\dots,f_{(n)})$为$\seq{f}{n}$的次序统计量。对任意的$i=1,2,\dots,n$,$f_{(i)}$是$(X,\mathscr{A})$上的随机变量。
\end{property}
\begin{proof}
	注意到对任意的$i=1,2,\dots,n$和任意的$a\in\mathbb{R}^{}$,$f_{(i)}$是从$X$到$\mathbb{R}$上的映射,并且有:
	\begin{equation*}
		\{f_{(i)}\leqslant a\}=\left\{\sum_{j=1}^{n}I_{\{f_j\leqslant a\}}\geqslant i\right\}
	\end{equation*}
	由\cref{prop:MeasurableFunction}(1.b)、\cref{prop:SimpleFunction}(3)(1)、\cref{prop:MeasurableFunction}(5.a)(1.d)即可得到$\{f_{(i)}\leqslant a\}\in\mathscr{A}$,于是$f_{(i)}$是$(X,\mathscr{A})$上的随机变量。
\end{proof}
\begin{lemma}\label{lem:OrderStatistics}
	设总体的分布函数为$F(x)$,概率函数为$f(x)$,$\seq{X}{n}$为从总体中抽取的简单样本,则有:
	\begin{equation*}
		\underset{a<x_1<\cdots<x_n<b}{\int\cdots\int}f(x_1)\cdots f(x_n)\dif x_1\cdots\dif x_n=\frac{1}{n!}[F(b)-F(a)]^n
	\end{equation*}
	其中$a,b\in\overline{\mathbb{R}}$。
\end{lemma}
\begin{proof}
	因为$\seq{X}{n}$独立同分布,所以:
	\begin{equation*}
		\int_{a}^{b}\cdots\int_{a}^{b}f(x_1)\cdots f(x_n)\dif x_1\cdots\dif x_n=\left[\int_{a}^{b}f(x_1)\dif x_1\right]^n=[F(b)-F(a)]^n
	\end{equation*}
	在这个区域上对$\seq{x}{n}$的排序结果一共有$n!$种(不考虑等于的情况,测度为$0$,不影响积分结果),每种排序都是等可能的,于是结论成立。
\end{proof}
\begin{theorem}\label{theo:OrderStatisticsDist}
	设$\seq{X}{n}$是测度空间$(X,\mathscr{A},P)$上独立同分布的随机变量,分布函数和概率函数分别为$F(x),f(x)$。
	\begin{enumerate}
		\item 令$Y_i=X_{(i)},\;i=1,2,\dots,n$,则次序统计量$(\seq{Y}{n})$的联合概率函数为:
		\begin{equation*}
			p(\seq{y}{n})=
			\begin{cases}
				n!f(y_1)f(y_2)\cdots f(y_n),&y_1<y_2<\cdots<y_n \\
				0,&\text{其它}
			\end{cases}
		\end{equation*}
		\item 次序统计量中任意$m\geqslant2$个分量$Y_{i_1},Y_{i_2},\dots,Y_{i_m},\;i_1<i_2<\cdots<i_m$的联合概率函数为:
		\begin{equation*}
			p(y_{i_1},y_{i_2},\dots,y_{i_m})=
			\begin{cases}
				n!\prod\limits_{j=1}^{m}f(y_{i_j})\left\{\prod\limits_{k=2}^{m}\frac{1}{(i_k-i_{k-1}-1)!}[F(y_{i_k})-F(y_{i_{k-1}})]^{i_k-i_{k-1}-1}\right\} & \\
				\quad\frac{F^{i_1-1}(y_{i_1})}{(i_1-1)!}\frac{[1-F(y_{i_m})]^{n-i_m}}{(n-i_m)!},\quad y_{i_1}<y_{i_2}<\cdots<y_{i_m} \\
				0, \quad\text{其他}
			\end{cases}
		\end{equation*}
		\item 次序统计量中单个分量$Y_{m},\;m=1,2,\dots,n$的概率函数为:
		\begin{equation*}
			p(y_{m})=n!f(y_m)\frac{F^{m-1}(y_m)}{(m-1)!}\frac{[1-F(y_m)]^{n-m}}{(n-m)!}
		\end{equation*}
		\item 对于任意的$i,j=1,2,\dots,n$满足$i<j$,令$V=Y_j-Y_i$,则有:
		\begin{equation*}
			p(v)=\int_{-\infty}^{+\infty}p(u,v)\dif u
		\end{equation*}
		其中:
		\begin{equation*}
			p(u,v)=
			\begin{cases}
				\dfrac{n!f(u)f(u+v)}{(i-1)!(j-i-1)!(n-j)!}F^{i-1}(u)[F(u+v)-F(u)]^{j-i-1} \\
				\quad\quad[1-F(u+v)]^{n-j},&v>0 \\
				0,&\text{其它}
			\end{cases}
		\end{equation*}
		\item 次序统计量$(\seq{Y}{n})$极差$V$的概率函数为:
		\begin{equation*}
			p(v)=\int_{-\infty}^{+\infty}p(u,v)\dif u
		\end{equation*}
		其中:
		\begin{equation*}
			p(u,v)=
			\begin{cases}
				n(n-1)f(u)f(u+v)[F(u+v)-F(u)]^{n-2},&v>0 \\
				0,&\text{其它}
			\end{cases}
		\end{equation*}
		分布函数为:
		\begin{equation*}
			F(V\leqslant x)=\int_{-\infty}^{+\infty}nf(u)[F(u+x)-F(u)]^{n-1}\dif u 
		\end{equation*}
	\end{enumerate}
\end{theorem}
\begin{proof}
	(1)在$\mathbb{R}^{n}$中划分$n!$个区域,每个区域分别对应着一个$\seq{i}{n}$使得$x_{i1}<x_{i2}<\cdots<x_{in}$,因为$\seq{x}{n}$的排列一共有$n!$种,所以这$n!$个区域加上包括等于号的一些零测集就构成了整个$\mathbb{R}^{n}$。因为次序统计量的概率函数也是在$\mathbb{R}^{n}$上的一个概率测度,则可以对每个划分的区域求$(\seq{Y}{n})$的概率测度,再对所有区域求和,即可得到次序统计量的联合概率函数。这个过程类似于全概率公式。\par
	任取一个上述区域$A$作变换:
	\begin{equation*}
		y_j=x_{i_j},\;j=1,2,\dots,n,\;x_{i1}<x_{i2}<\cdots<x_{in}
	\end{equation*}
	则该变换的Jacobi行列式为$|\mathbf{J}|=|I_n|=1$,因为$\seq{X}{n}$是简单样本,所以在该区域上的:
	\begin{equation*}
		p(\seq{y}{n}|A)=
		\begin{cases}
			\prod\limits_{i=1}^{n}f(x_i)=\prod\limits_{i=1}^{n}f(y_i),&y_1<y_2<\cdots<y_n \\
			0,&\text{其它}
		\end{cases}
	\end{equation*}
	由区域的任意性可得在整个$\mathbb{R}^{n}$上:
	\begin{equation*}
		p(\seq{y}{n})=
		\begin{cases}
			n!f(y_1)f(y_2)\cdots f(y_n),&y_1<y_2<\cdots<y_n \\
			0,&\text{其它}
		\end{cases}
	\end{equation*}\par
	(2)注意到$Y_{(i_1)},Y_{(i_2)},\dots,Y_{(i_m)}$的联合概率函数是次序统计量的边缘概率函数,所以由(1)和\cref{lem:OrderStatistics}可得:
	\begin{align*}
		p(y_{i_1},y_{i_2},\dots,y_{i_m})
		&=\underset{-\infty<y_1<\cdots<y_n<+\infty}{\int\cdots\int}n!f(y_1)f(y_2)\cdots f(y_n)\dif y_1\cdots\dif y_{i_1-1} \\
		&\quad\dif y_{i_1+1}\cdots\dif y_{i_2-1}\dif y_{i_2+1}\cdots\dif y_{i_m-1}\dif y_{i_m+1}\cdots\dif y_n \\
		&=n!\prod_{j=1}^{m}f(y_{i_j})\underset{-\infty<y_1<\cdots<y_{i_1}}{\int\cdots\int}f(y_1)\cdots f(y_{i_1-1})\dif y_1\cdots\dif y_{i_1-1} \\
		&\quad\times\underset{y_{i_1}<y_{i_1+1}<y_{i_1+2}<\cdots<y_{i_2}}{\int\cdots\int}f(y_{i_1+1})\cdots f(y_{i_2-1})\dif y_{i_1+1}\cdots\dif y_{i_2-1} \\
		&\quad\cdots\cdots \\
		&\quad\times\underset{y_{i_m}<y_{i_m+1}<y_{i_m+2}<\cdots<+\infty}{\int\cdots\int}f(y_{i_m+1})\cdots f(y_n)\dif y_{i_m+1}\cdots\dif y_n \\
		&=n!\prod_{j=1}^{m}f(y_{i_j})\frac{1}{(i_1-1)!}F^{i_1-1}(y_{i_1})\frac{1}{(i_2-i_1-1)!}[F(y_{i_2})-F(y_{i_1})]^{i_2-i_1-1} \\
		&\quad\cdots\frac{1}{(n-i_m)!}[1-F(y_{i_m})]^{n-i_m} \\
		&=n!\prod_{j=1}^{m}f(y_{i_j})\frac{1}{(i_1-1)!}F^{i_1-1}(y_{i_1})\frac{1}{(n-i_m)!}[1-F(y_{i_m})]^{n-i_m} \\
		&\quad\left\{\prod_{k=2}^{m}\frac{1}{(i_k-i_{k-1}-1)!}[F(y_{i_k})-F(y_{i_{k-1}})]^{i_k-i_{k-1}-1}\right\}
	\end{align*}\par
	(3)类似(2)的过程即可得到,省略。\par
	(4)使用增补变量法\info{链接随机变量函数的分布中的增补变量法},做变换:
	\begin{equation*}
		\begin{cases}
			U=Y_i \\
			V=Y_j-Y_i
		\end{cases}
		\Leftrightarrow
		\begin{cases}
			Y_i=U \\
			Y_j=V+U
		\end{cases}
	\end{equation*}
	该变换的Jacobi行列式为:
	\begin{equation*}
		|\mathbf{J}|=
		\begin{vmatrix}
			1 & 0 \\
			1 & 1
		\end{vmatrix}
		=1
	\end{equation*}
	由(2)可得$(Y_i,Y_j)$的联合概率函数:
	\begin{equation*}
		p(y_i,y_j)=
		\begin{cases}
			\dfrac{n!f(y_i)f(y_j)}{(i-1)!(j-i-1)!(n-j)!}F^{i-1}(y_i)[F(y_j)-F(y_i)]^{j-i-1} \\
			\quad\quad[1-F(y_j)]^{n-j},&y_i<y_j \\
			0,&\text{其它}
		\end{cases}
	\end{equation*}
	于是:
	\begin{equation*}
		p(u,v)=
		\begin{cases}
			\dfrac{n!f(u)f(u+v)}{(i-1)!(j-i-1)!(n-j)!}F^{i-1}(u)[F(u+v)-F(u)]^{j-i-1} \\
			\quad\quad[1-F(u+v)]^{n-j},&v>0 \\
			0,&\text{其它}
		\end{cases}
	\end{equation*}
	所以:
	\begin{equation*}
		p(v)=\int_{-\infty}^{+\infty}p(u,v)\dif u
	\end{equation*}\par
	(5)由(4)立即可得概率函数。对于分布函数有:
	\begin{align*}
		F(V\leqslant x)&=\int_{-\infty}^x\int_{-\infty}^{+\infty}g(u,v)\dif u\dif v =\int_{-\infty}^0\int_{-\infty}^{+\infty}0\dif u\dif v \\
		&\quad+\int_{0}^x\int_{-\infty}^{+\infty}n(n-1)f(u)f(u+v)[F(u+v)-F(u)]^{n-2}\dif u\dif v \\
		&=\int_{-\infty}^{+\infty}\dif u\int_{0}^xn(n-1)f(u)f(u+v)[F(u+v)-F(u)]^{n-2}\dif v \\
		&=\int_{-\infty}^{+\infty}nf(u)[F(u+v)-F(u)]^{n-1}\Big|_0^x\dif u \\
		&=\int_{-\infty}^{+\infty}nf(u)[F(u+x)-F(u)]^{n-1}\dif u \qedhere
	\end{align*}
\end{proof}

\subsection{充分统计量}
\begin{definition}
	设$(X,\mathscr{A},\mathscr{P})$是参数结构,$\Theta$为参数空间,$T$是可测空间$(X,\mathscr{A})$到可测空间$(Y,\mathscr{B})$上的统计量。若对任意的$\theta\in\Theta$,样本$\mathbf{X}$在给定$T(\mathbf{X})$下的条件分布$P_{\theta}(\mathbf{X}|T(\mathbf{X}))$与参数$\theta$无关a.s.于$(Y,\mathscr{B},P_{\theta}T^{-1})$\info{这是茆书中的定义,点估计理论与ShaoJun都未提到a.s.,a.s.的对象是否有问题?条件分布是定义在原空间上的,为什么a.s.于推前测度?},则称$T$为$\theta$的\gls{SufficientStatistic},也称$T$为$\mathscr{P}$的充分统计量。
\end{definition}
\begin{theorem}[Factorization Theorem]
	\label{theo:FactorizationTheorem}
	设$(X,\mathscr{A},\mathscr{P})$是可控参数结构,$\mu$是控制测度,$\Theta$是参数空间,$T$是可测空间$(X,\mathscr{A})$到可测空间$(Y,\mathscr{B})$上的统计量。$T$对$\mathscr{P}$充分当且仅当存在非负$\mathscr{B}$可测函数$g_\theta$和非负$\mathscr{A}$可测函数$h$使得:
	\begin{equation*}
		\forall\;\theta\in\Theta,\;\frac{\dif P_\theta}{\dif\mu}(x)=g_\theta[T(x)]h(x)\;\text{a.e.于}(X,\mathscr{A},\mu)
	\end{equation*}
\end{theorem}
\begin{property}\label{prop:SufficientStatistic}
	充分统计量具有如下性质:
	\begin{enumerate}
		\item 设$(\mathbb{R},\mathcal{B},\mathscr{P})$为统计结构,则其上的次序统计量是$\mathscr{P}$的充分统计量;
		\item 充分统计量的可逆变换仍为充分统计量;
	\end{enumerate}
\end{property}
\subsubsection{极小充分统计量}
\begin{definition}
	设$(X,\mathscr{A},\mathscr{P})$是统计结构,$T$是可测空间$(X,\mathscr{A})$到可测空间$(Y,\mathscr{B})$上对$\mathscr{P}$充分的统计量。若对任意$\mathscr{P}$的充分统计量$S$存在$(Y,\mathscr{B})$到$(Y,\mathscr{B})$上的可测映射$\varphi$满足:
	\begin{equation*}
		\forall\;P\in\mathscr{P},\;T=\varphi(S)\;a.s.\text{于}(X,\mathscr{A},P)
	\end{equation*}
	则称$T$为$\mathscr{P}$的\gls{MinimalSufficientStatistic}。
\end{definition}
\begin{property}\label{prop:MinimalSufficientStatistic}
	设$(X,\mathscr{A},\mathscr{P})$是统计结构,$T$是可测空间$(X,\mathscr{A})$到可测空间$(Y,\mathscr{B})$上的统计量。极小充分统计量具有如下性质:
	\begin{enumerate}
		\item 在$X$为欧氏空间且$(X,\mathscr{A},\mathscr{P})$是可控结构的条件下,极小充分统计量一定存在;
		\item 极小充分统计量的可逆可测变换仍为极小充分统计量;
		\item 若将可由一个可逆的可测映射a.s.互变的两个极小充分统计量视为同一个统计量,则极小充分统计量在此意义下是唯一的;
		\item 若$T$是$\mathscr{P}_0\subseteq\mathscr{P}$的极小充分统计量且是$\mathscr{P}$的充分统计量,对任意的$P\in\mathscr{P}_0\;$a.s.就对任意的$P\in\mathscr{P}\;$a.s.,则$T$是$\mathscr{P}$的极小充分统计量;
		\item 若$(X,\mathscr{A},\mathscr{P})$是可控参数结构,$X$是欧氏空间,$\mu$是控制测度,$\Theta$为参数空间,则如果满足:
		\begin{equation*}
			\forall\;x,y\in X,\;\forall\;\theta\in\Theta,\;\frac{\dif P_\theta}{\dif\mu}(x)=\frac{\dif P_\theta}{\dif\mu}(y)f(x,y)\Rightarrow T(x)=T(y)
		\end{equation*}
		其中$f$是一个可测函数,则$T$是$\mathscr{P}$的极小充分统计量;
		\item 若$\mathscr{P}=\{P_n\},\;n=0,1,2,\dots$,则统计量:
		\begin{equation*}
			T(x)=\left(\frac{P_1(x)}{P_0(x)},\frac{P_2(x)}{P_0(x)},\dots\right)
		\end{equation*}
		是$\mathscr{P}$的极小充分统计量;
	\end{enumerate}
\end{property}
\begin{proof}
	(1)不予证明。\par
	(2)由定义即可得出。\par
	(3)\info{JunShao的证明很简单,由定义即可得出,但没想明白}\par
	(4)设$S$是$\mathscr{P}$的充分统计量,所以$S$也是$\mathscr{P}_0$的充分统计量,于是存在可测映射$\varphi$使得:
	\begin{equation*}
		\forall\;P\in\mathscr{P}_0,\;T=\varphi(S)\;a.s.\text{于}(X,\mathscr{A},P)
	\end{equation*}
	由条件即可得到:
	\begin{equation*}
		\forall\;P\in\mathscr{P},\;T=\varphi(S)\;a.s.\text{于}(X,\mathscr{A},P)
	\end{equation*}\par
	(5)\info{证明未完成}
\end{proof}


\subsection{完全性}
\begin{definition}
	设$(X,\mathscr{A},\mathscr{P})$是参数结构,$\Theta$是参数空间,$T$是可测空间$(X,\mathscr{A})$到可测空间$(Y,\mathscr{B})$上的统计量。若对任意的Borel函数$f$有:
	\begin{equation*}
		\forall\;\theta\in\Theta,\;\int_{X}f[T(x)]\dif P_{\theta}=0\Rightarrow f[T(x)]=0\;a.s.\text{于}(X,\mathscr{A},P_{\theta})
	\end{equation*}
	则称$T$为\gls{CompleteStatistic},称$\mathscr{P}T^{-1}$是\textbf{完全的}。若对任意的有界Borel函数有上述结论,则称$T$为\gls{BoundedlyCompleteStatistic}。
\end{definition}
\begin{property}\label{prop:CompleteStatistic}
	完全统计量具有如下性质:
	\begin{enumerate}
		\item 完全统计量是有界完备统计量;
		\item 完全(有界完全)统计量的可测变换仍是完全(有界完全)统计量;
		\item 充分完全统计量是极小充分统计量;
	\end{enumerate}
\end{property}

\subsection{指数族}
\begin{definition}
	设$(X,\mathscr{A},\mathscr{P})$是可控参数结构,$\mu$是控制测度,$\Theta$是参数空间。若$\mathscr{P}$满足:
	\begin{equation*}
		\forall\;\theta\in\Theta,\;\frac{\dif P_{\theta}}{\dif\mu}(x)=\exp\left[\eta^T(\theta)T(x)-\xi(\theta)\right]h(x)=C(\theta)\exp\left[\eta^T(\theta)T(x)\right]h(x)
	\end{equation*}
	且支撑$\left\{x:\frac{\dif P_{\theta}}{\dif\mu}(x)>0\right\}$不依赖于未知参数$\theta$,则称$\mathscr{P}$是\gls{ExponentialFamily}。其中$\eta(\theta),T(x)$是$n$维实向量,$n$被称为该指数族的维数,$h$是非负Borel函数,$\xi(\theta)$是实值函数。称:
	\begin{equation*}
		\frac{\dif P_{\theta}}{\dif\mu}(x)=\exp\left[\eta^TT(x)-\xi(\eta)\right]h(x)=C(\eta)\exp\left[\eta^TT(x)\right]h(x)
	\end{equation*}
	为指数族的\gls{CanonicalForm},此时的新参数$\eta$被称为\gls{NaturalParameter},称:
	\begin{equation*}
		\Xi=\left\{\eta:\int_{X}\exp\left[\eta^TT(x)\right]h(x)\dif\mu<+\infty\right\}
	\end{equation*}
	为\gls{NaturalParameterSpace}。当$T(x)$或$\eta(\theta)$的各分量之间满足线性约束时,此时称$\theta$\textbf{不可识别}。若自然参数空间作为$\mathbb{R}^{n}$的子集包含一个开集,则称该指数族是\textbf{满秩的}。若自然参数之间满足非线性约束,称此时的指数族为\textbf{curved exponential family}。
\end{definition}
\begin{note}
	指数族的很多很好的性质需要指数族是满秩的,请注意,自然参数空间有内点$\iff$自然参数空间中有开集$\iff$自然参数空间中含有一个开矩形。
\end{note}
\begin{property}\label{prop:ExponentialFamily}
	设$(X,\mathscr{A},\mathscr{P})$是可控参数结构,$\mu$是控制测度,$\mathscr{P}$是$n$维指数族,则:
	\begin{enumerate}
		\item $\dfrac{\dif P_{\theta}}{\dif\mu}$表达式不唯一;
		\item 对于$\xi(\theta)$和自然参数$\xi(\eta)$有:
		\begin{gather*}
			\xi(\theta)=\ln\left\{\int_{X}\exp\left[\eta^T(\theta)T(x)\right]h(x)\dif\mu\right\} \\
			\xi(\eta)=\ln\left\{\int_{X}\exp\left[\eta^TT(x)\right]h(x)\dif\mu\right\}
		\end{gather*}
		\item 有限个概率分布是指数族的相互独立的随机变量的联合概率分布仍为指数族,即指数族在简单抽样下是封闭的;
		\item $\mathscr{P}$的自然参数空间$\Xi$是凸集;
		\item 设$f$是$(X,\mathscr{A})$上的Borel函数,令:
		\begin{equation*}
			G(\eta)=\int_{X}f(x)\exp[\eta^TT(x)]h(x)\dif\mu
		\end{equation*}
		记$\Xi_0=\{\eta:G(\eta)\in\mathbb{R}^{}\}$。对$\Xi_0$的任意内点$a$,$G$在$a$点连续,且$G$在$a$点的任意阶偏导数都存在并可将求导与积分交换顺序。
		\item 若$\mathscr{P}$满秩,$T(x)$或$\eta(\theta)$的各分量之间不满足线性约束或非线性约束;
		\item $T(x)$是$\mathscr{P}$的充分统计量;
		\item 若$\mathscr{P}$满秩,则$T(x)$是$\mathscr{P}$的极小充分统计量和完全统计量;
		\item $\operatorname{E}_{\eta}(T_i)=\dfrac{\partial\xi(\eta)}{\partial\eta_i},\;\operatorname{Cov}_{\eta}(T_i,T_j)=\dfrac{\partial^2\xi(\eta)}{\partial\eta_i\partial\eta_j}$;
	\end{enumerate}
\end{property}
\begin{proof}
	(1)显然。\par
	(2)概率测度在$X$上的积分应为$1$。\par
	(3)\info{链接独立的联合密度函数}。\par
	(4)设$\theta,\vartheta\in\Xi$且$\theta\ne\vartheta$。对任意的 $\alpha\in(0,1)$,将$h$吸收进$\mu$得到$\nu$,由\cref{ineq:holder-ineq-Lebesgue}可得:
	\begin{align*}
		&\int_{}\exp\left\{[\alpha\theta+(1-\alpha)\vartheta]^TT(x)\right\}\dif\nu \\
		=&\int_{}\exp\left[\alpha\theta^TT(x)\right]\exp\left[(1-\alpha)\vartheta^TT(x)\right]\dif\nu \\
		\leqslant&\left\{\int_{}\exp\left[\theta^TT(x)\right]\dif\nu\right\}^{\alpha}\left\{\int_{}\exp\left[\vartheta^TT(x)\right]\dif\nu\right\}^{1-\alpha}<+\infty
	\end{align*}\par
	(7)由\cref{theo:FactorizationTheorem}立即可得。\par
	(9)因为:
	\begin{equation*}
		\int_{X}\exp\left[\sum_{i=1}^{n}\eta_iT_i(x)-\xi(\eta)\right]h(x)\dif\mu=1
	\end{equation*}
	由(5)和\cref{prop:MeasurableIntegral}(5)可得\info{$C(\eta)$的可积性与Borel可测性}:
	\begin{gather*}
		\frac{\partial}{\partial\eta_i}\int_{X}^{}\exp\left[\sum_{i=1}^{n}\eta_iT_i(x)-\xi(\eta)\right]h(x)\dif\mu=0 \\
		\int_{X}^{}\left[T_i(x)-\frac{\partial \xi(\eta)}{\partial\eta_i}\right]\exp\left[\sum_{i=1}^{n}\eta_iT_i(x)-\xi(\eta)\right]h(x)\dif\mu=0 \\
		\int_{X}^{}T_i(x)\exp\left[\sum_{i=1}^{n}\eta_iT_i(x)-\xi(\eta)\right]h(x)\dif\mu=\frac{\partial \xi(\eta)}{\partial\eta_i}\int_{X}^{}\exp\left[\sum_{i=1}^{n}\eta_iT_i(x)-\xi(\eta)\right]h(x)\dif\mu \\
		\operatorname{E}_{\eta}(T_i)=\frac{\partial \xi(\eta)}{\partial\eta_i}
	\end{gather*}\par
	对上式求关于$\eta_j$的导数,由\cref{prop:CovMat}(6)可得:
	\begin{gather*}
		\int_{X}^{}T_i(x)\left[T_j(x)-\frac{\partial \xi(\eta)}{\partial\eta_j}\right]\exp\left[\sum_{i=1}^{n}\eta_iT_i(x)-\xi(\eta)\right]h(x)\dif\mu=\frac{\partial^2 \xi(\eta)}{\partial\eta_i\partial\eta_j} \\
		\int_{X}^{}T_i(x)T_j(x)p(x)\dif\mu-\int_{X}^{}T_i(x)\frac{\partial \xi(\eta)}{\partial\eta_j}p(x)\dif\mu=\frac{\partial^2 \xi(\eta)}{\partial\eta_i\partial\eta_j} \\
		\operatorname{E}_{\eta}(T_iT_j)-\frac{\partial \xi(\eta)}{\partial\eta_j}\operatorname{E}_{\eta}(T_i)=\frac{\partial^2 \xi(\eta)}{\partial\eta_i\partial\eta_j} \\
		\operatorname{E}_{\eta}(T_iT_j)-\operatorname{E}_{\eta}(T_i)\operatorname{E}_{\eta}(T_j)=\frac{\partial^2 \xi(\eta)}{\partial\eta_i\partial\eta_j} \\
		\operatorname{Cov}_{\eta}(T_i,T_j)=\frac{\partial^2 \xi(\eta)}{\partial\eta_i\partial\eta_j}\qedhere
	\end{gather*}
\end{proof}
\begin{theorem}
	设$X\sim\operatorname{Binom}(n,p),\;p\in(0,1)$,$\seq{X}{m}$是$X$的一个样本,则:
	\begin{enumerate}
		\item $\seq{X}{m}$的分布是指数族,其标准形式、自然参数、$C(\eta)$、$T(x)$和$h(x)$分别为:
		\begin{gather*}
			p(X_1=x_1,X_2=x_2,\dots,X_m=x_m)=(1+e^{\eta})^{-mn}\exp\left(\eta\sum_{i=1}^{m}x_i\right)\prod_{i=1}^{m}\left[\binom{n}{x_i}I_{\mathbb{N}_0}(x_i)\right] \\
			\eta=\ln\left(\frac{p}{1-p}\right),\quad C(\eta)=(1+e^{\eta})^{-mn} \\ T(x)=\sum_{i=1}^{m}x_i,\quad h(x)=\prod_{i=1}^{m}\left[\binom{n}{x_i}I_{\mathbb{N}_0}(x_i)\right]
		\end{gather*}
		自然参数空间为$\mathbb{R}^{}$;
		\item $\sum\limits_{i=1}^{m}X_i$是该分布族的充分统计量、完备统计量和极小充分统计量;
	\end{enumerate}
\end{theorem}
\begin{proof}
	(1)由二项分布的定义可以得到:
	\begin{align*}
		&p(X_1=x_1,X_2=x_2,\dots,X_m=x_m) \\
		=&\prod_{i=1}^{m}\left[\binom{n}{x_i}p^{x_i}(1-p)^{n-x_i}I_{\mathbb{N}_0}(x_i)\right]=\prod_{i=1}^{m}\left[\binom{n}{x_i}\left(\frac{p}{1-p}\right)^{x_i}(1-p)^nI_{\mathbb{N}_0}(x_i)\right] \\
		=&(1-p)^{mn}\exp\left[\ln\left(\frac{p}{1-p}\right)\sum_{i=1}^{m}x_i\right]\prod_{i=1}^{m}\left[\binom{n}{x_i}I_{\mathbb{N}_0}(x_i)\right]
	\end{align*}
	令:
	\begin{equation*}
		\eta=\ln\left(\frac{p}{1-p}\right)
	\end{equation*}
	可解得:
	\begin{equation*}
		p=\frac{e^{\eta}}{e^{\eta}+1}
	\end{equation*}
	于是有:
	\begin{equation*}
		C(\eta)=(1-p)^{mn}=\left(\frac{1}{e^{\eta}+1}\right)^{mn}=(1+e^{\eta})^{-mn}
	\end{equation*}
	根据$\eta$与$p$的关系可得$\eta\in\mathbb{R}^{}$。\par
	(2)由(1)和\cref{prop:ExponentialFamily}(7)(8)立即可得。
\end{proof}

\begin{theorem}
	设$X\sim\operatorname{NB}(r,p),\;p\in(0,1)$,$\seq{X}{m}$是$X$的一个样本,则:
	\begin{enumerate}
		\item $\seq{X}{m}$的分布是指数族,其标准形式、自然参数、$C(\eta)$、$T(x)$和$h(x)$分别为:
		\begin{gather*}
			\begin{aligned}
				&p(X_1=x_1,X_2=x_2,\dots,X_m=x_m) \\
				=&(1-e^{\eta})^{mr}\exp\left(\eta\sum_{i=1}^{m}x_i\right)\prod_{i=1}^{m}\left[\binom{x_i+r-1}{r-1}I_{\mathbb{N}_0}(x_i)\right]
			\end{aligned} \\
			\eta=\ln(1-p),\quad C(\eta)=(1-e^{\eta})^{mr} \\
			T(x)=\sum_{i=1}^{m}x_i,\quad h(x)=\prod_{i=1}^{m}\left[\binom{x_i+r-1}{r-1}I_{\mathbb{N}_0}(x_i)\right]
		\end{gather*}
		自然参数空间为$\mathbb{R}^{-}$;
		\item $\sum\limits_{i=1}^{m}X_i$是该分布族的充分统计量、完备统计量和极小充分统计量;
	\end{enumerate}
\end{theorem}
\begin{proof}
	(1)由负二项分布的定义可以得到:
	\begin{align*}
		&p(X_1=x_1,X_2=x_2,\dots,X_m=x_m) \\
		=&\prod_{i=1}^{m}\left[\binom{x_i+r-1}{r-1}p^r(1-p)^{x_i}I_{\mathbb{N}_0}(x_i)\right] \\
		=&p^{mr}\exp\left[\ln(1-p)\sum_{i=1}^{m}x_i\right]\prod_{i=1}^{m}\left[\binom{x_i+r-1}{r-1}I_{\mathbb{N}_0}(x_i)\right]
	\end{align*}
	令$\eta=\ln(1-p)$,可解得$p=1-e^{\eta}$,于是有$C(\eta)=p^{mr}=(1-e^{\eta})^{mr}$。由$\eta$与$p$的关系可得$\eta\in\mathbb{R}^{-}$。\par
	(2)由(1)和\cref{prop:ExponentialFamily}(7)(8)立即可得。
\end{proof}

%\begin{theorem}
%	设$X\sim\operatorname{Geom}(p),\;p\in(0,1)$,$\seq{X}{m}$是$X$的一个样本,则:
%	\begin{enumerate}
%		\item $\seq{X}{m}$的分布是指数族,其标准形式、自然参数、$C(\eta)$、$T(x)$和$h(x)$分别为:
%		\begin{gather*}
%			p(X_1=x_1,X_2=x_2,\dots,X_m=x_m)=(1-e^{\eta})^m\exp\left(\eta\sum_{i=1}^{m}x_i\right)\prod_{i=1}^{m}I_{\mathbb{N}_0}(x_i) \\
%			\eta=\ln(1-p),\quad C(\eta)=(1-e^{\eta})^m,\quad T(x)=\sum_{i=1}^{m}x_i,\quad h(x)=\prod_{i=1}^{m}I_{\mathbb{N}_0}(x_i)
%		\end{gather*}
%		自然参数空间为$\mathbb{R}^{-}$;
%		\item $\sum\limits_{i=1}^{m}X_i$是该分布族的充分统计量、完备统计量和极小充分统计量;
%	\end{enumerate}
%\end{theorem}
%\begin{proof}
%	(1)由几何分布的定义可以得到:
%	\begin{align*}
%		&p(X_1=x_1,X_2=x_2,\dots,X_m=x_m) \\
%		=&\prod_{i=1}^{m}[(1-p)^{x_i}pI_{\mathbb{N}_0}(x_i)] =p^m\exp\left[\ln(1-p)\sum_{i=1}^{m}x_i\right]\prod_{i=1}^{m}I_{\mathbb{N}_0}(x_i)
%	\end{align*}
%	令$\eta=\ln(1-p)$,可解得$p=1-e^{\eta}$,于是有$C(\eta)=p^m=(1-e^{\eta})^m$。由$\eta$与$p$的关系可得$\eta\in\mathbb{R}^{-}$。\par
%	(2)由(1)和\cref{prop:ExponentialFamily}(7)(8)立即可得。
%\end{proof}

\begin{theorem}
	设$X\sim\operatorname{Poisson}(\lambda),\;\lambda\in\mathbb{R}^+$,$\seq{X}{m}$是$X$的一个样本,则:
	\begin{enumerate}
		\item $\seq{X}{m}$的分布是指数族,其标准形式、自然参数、$C(\eta)$、$T(x)$和$h(x)$分别为:
		\begin{gather*}
			p(X_1=x_1,X_2=x_2,\dots,X_m=x_m)=\exp(-me^{\eta})\exp\left(\eta\sum_{i=1}^{m}x_i\right)\prod_{i=1}^{m}\frac{I_{\mathbb{N}_0}(x_i)}{x_i!} \\
			\eta=\ln\lambda,\quad C(\eta)=\exp(-me^{\eta}),\quad T(x)=\sum_{i=1}^{m}x_i,\quad h(x)=\prod_{i=1}^{m}\frac{I_{\mathbb{N}_0}(x_i)}{x_i!}
		\end{gather*}
		自然参数空间为$\mathbb{R}^{}$;
		\item $\sum\limits_{i=1}^{m}X_i$是该分布族的充分统计量、完备统计量和极小充分统计量;
	\end{enumerate}
\end{theorem}
\begin{proof}
	(1)由Poisson分布的定义可以得到:
	\begin{align*}
		&p(X_1=x_1,X_2=x_2,\dots,X_m=x_m) \\
		=&\prod_{i=1}^{m}\left(\frac{\lambda^{x_i}}{x_i!}e^{-\lambda}I_{\mathbb{N}_0}(x_i)\right) =e^{-m\lambda}\exp\left[\ln(\lambda)\sum_{i=1}^{m}x_i\right]\prod_{i=1}^{m}\frac{I_{\mathbb{N}_0}(x_i)}{x_i!}
	\end{align*}
	令$\eta=\ln\lambda$,可解得$\lambda=e^{\eta}$,于是有$C(\eta)=e^{-m\lambda}=\exp(-me^{\eta})$。由$\eta$与$\lambda$的关系可得$\eta\in\mathbb{R}^{}$。\par
	(2)由(1)和\cref{prop:ExponentialFamily}(7)(8)立即可得。
\end{proof}

\begin{theorem}
	设$X\sim\operatorname{Power}(\theta),\;\theta\in\mathbb{R}^{+}$,$\seq{X}{m}$是$X$的一个样本,则:
	\begin{enumerate}
		\item $\seq{X}{m}$的分布是指数族,其标准形式、自然参数、$C(\eta)$、$T(x)$和$h(x)$分别为:
		\begin{gather*}
			p(X_1=x_1,X_2=x_2,\dots,X_m=x_m)=(\eta+1)^m\exp\left(\eta\sum_{i=1}^{m}\ln x_i\right)\prod_{i=1}^{m}I_{(0,1)}(x_i) \\
			\eta=\theta-1,\quad C(\eta)=(\eta+1)^m,\quad
			T(x)=\sum_{i=1}^{m}\ln x_i,\quad h(x)=\prod_{i=1}^{m}I_{(0,1)}(x_i)
		\end{gather*}
		自然参数空间为$(-1,+\infty)$;
		\item $\sum\limits_{i=1}^{m}\ln X_i$是该分布族的充分统计量、完备统计量和极小充分统计量;
	\end{enumerate}
\end{theorem}
\begin{proof}
	(1)由幂分布的定义可以得到:
	\begin{align*}
		&p(X_1=x_1,X_2=x_2,\dots,X_m=x_m) \\
		=&\prod_{i=1}^{m}\left[\theta x_i^{\theta-1}I_{(0,1)}(x_i)\right]=\theta^m\exp\left[(\theta-1)\sum_{i=1}^{m}\ln x_i\right]\prod_{i=1}^{m}I_{(0,1)}(x_i)
	\end{align*}
	令$\eta=\theta-1$,于是有$C(\eta)=\theta^{m}=(\eta+1)^m$。由$\eta$与$\theta$的关系可得$\eta\in(-1,+\infty)$。\par
	(2)由(1)和\cref{prop:ExponentialFamily}(7)(8)立即可得。
\end{proof}

\begin{theorem}
	设$X\sim\operatorname{Weibull}(n,\theta),\;\theta\in\mathbb{R}^{+}$,$\seq{X}{m}$是$X$的一个样本,则:
	\begin{enumerate}
		\item $\seq{X}{m}$的分布是指数族,其标准形式、自然参数、$C(\eta)$、$T(x)$和$h(x)$分别为:
		\begin{gather*}
			p(X_1=x_1,X_2=x_2,\dots,X_m=x_m)=(-\eta)^{m}\exp\left(\eta\sum_{i=1}^{m}x_i^n\right)\prod_{i=1}^{m}\left[nx_i^{n-1}I_{(0,+\infty)}(x_i)\right] \\
			\eta=-\theta^{-n},\quad C(\eta)=(-\eta)^{m},\quad
			T(x)=\sum_{i=1}^{m}x_i^n,\quad h(x)=\prod_{i=1}^{m}\left[nx_i^{n-1}I_{(0,+\infty)}(x_i)\right]
		\end{gather*}
		自然参数空间为$\mathbb{R}^{-}$;
		\item $\sum\limits_{i=1}^{m}X_i^n$是该分布族的充分统计量、完备统计量和极小充分统计量;
	\end{enumerate}
\end{theorem}
\begin{proof}
	(1)由Weibull分布的定义可以得到:
	\begin{align*}
		&p(X_1=x_1,X_2=x_2,\dots,X_m=x_m) \\
		=&\prod_{i=1}^{m}\left\{nx_i^{n-1}\theta^{-n}\exp\left[-\left(\frac{x_i}{\theta}\right)^n\right]I_{(0,+\infty)}(x_i)\right\} \\
		=&\theta^{-mn}\exp\left(-\frac{1}{\theta^n}\sum_{i=1}^{m}x_i^n\right)\prod_{i=1}^{m}\left[nx_i^{n-1}I_{(0,+\infty)}(x_i)\right]
	\end{align*}
	令$\eta=-\theta^{-n}$,于是有$C(\eta)=\theta^{-mn}=(-\eta)^{m}$。由$\eta$与$\theta$的关系可得$\eta\in\mathbb{R}^{-}$。\par
	(2)由(1)和\cref{prop:ExponentialFamily}(7)(8)立即可得。
\end{proof}

\begin{theorem}
	设$X\sim\operatorname{Pareto}(\alpha,\beta),\;\beta\in\mathbb{R}^{+}$,$\seq{X}{m}$是$X$的一个样本,则:
	\begin{enumerate}
		\item $\seq{X}{m}$的分布是指数族,其标准形式、自然参数、$C(\eta)$、$T(x)$和$h(x)$分别为:
		\begin{gather*}
			\begin{aligned}
				&p(X_1=x_1,X_2=x_2,\dots,X_m=x_m) \\
				=&(-\eta-1)^m\alpha^{-m(\eta+1)}\exp\left(\eta\sum_{i=1}^{m}\ln x_i\right)\prod_{i=1}^{m}I_{(\alpha,+\infty)}(x_i)
			\end{aligned} \\
			\eta=-(\beta+1),\quad C(\eta)=(-\eta-1)^m\alpha^{-m(\eta+1)} \\
			T(x)=\sum_{i=1}^{m}\ln x_i,\quad h(x)=\prod_{i=1}^{m}I_{(\alpha,+\infty)}(x_i)
		\end{gather*}
		自然参数空间为$(-\infty,-1)$;
		\item $\sum\limits_{i=1}^{m}\ln X_i$是该分布族的充分统计量、完备统计量和极小充分统计量;
	\end{enumerate}
\end{theorem}
\begin{proof}
	(1)由Pareto分布的定义可以得到:
	\begin{align*}
		&p(X_1=x_1,X_2=x_2,\dots,X_m=x_m) \\
		=&\prod_{i=1}^{m}\left[\beta\alpha^{\beta}x_i^{-(\beta+1)}I_{(\alpha,+\infty)}(x_i)\right]=\beta^m\alpha^{m\beta}\exp\left[-(\beta+1)\sum_{i=1}^{m}\ln x_i\right]\prod_{i=1}^{m}I_{(\alpha,+\infty)}(x_i)
	\end{align*}
	令$\eta=-(\beta+1)$,于是有$C(\eta)=\beta^{m}\alpha^{m\beta}=(-\eta-1)^m\alpha^{-m(\eta+1)}$。由$\eta$与$\beta$的关系可得$\eta\in(-\infty,-1)$。\par
	(2)由(1)和\cref{prop:ExponentialFamily}(7)(8)立即可得。
\end{proof}

\begin{theorem}
	设$X\sim\operatorname{N}(\mu,\sigma^2),\;\mu\in\mathbb{R}^{},\sigma^2>0$,$\seq{X}{m}$是$X$的一个样本,则:
	\begin{enumerate}
		\item $\seq{X}{m}$的分布是指数族,其标准形式、自然参数、$C(\eta)$、$T(x)$和$h(x)$分别为:
		\begin{gather*}
			\begin{aligned}
				&p(X_1=x_1,X_2=x_2,\dots,X_m=x_m) \\
				=&\left(-\frac{\pi}{\eta_1}\right)^{-\frac{m}{2}}\exp\left(\frac{m\eta_2^2}{4\eta_1}\right)\exp\left(\eta_1\sum_{i=1}^{m}x_i^2+\eta_2\sum_{i=1}^{m}x_i\right)
			\end{aligned} \\
			\eta=\left(-\frac{1}{2\sigma^2},\frac{\mu}{\sigma^2}\right),\quad C(\eta)=\left(-\frac{\pi}{\eta_1}\right)^{-\frac{m}{2}}\exp\left(\frac{m\eta_2^2}{4\eta_1}\right) \\
			T(x)=\left(\sum_{i=1}^{m}x_i^2,\sum_{i=1}^{m}x_i\right),\quad h(x)=1
		\end{gather*}
		自然参数空间为$\mathbb{R}^-\times\mathbb{R}^{}$;
		\item $\left(\sum\limits_{i=1}^{m}X_i^2,\sum\limits_{i=1}^{m}X_i\right)$是该分布族的充分统计量、完备统计量和极小充分统计量;
	\end{enumerate}
\end{theorem}
\begin{proof}
	(1)由正态分布的定义可以得到:
	\begin{align*}
		&p(X_1=x_1,X_2=x_2,\dots,X_m=x_m)=\prod_{i=1}^{m}\left\{(2\pi\sigma^2)^{-\frac{1}{2}}\exp\left[-\frac{(x_i-\mu)^2}{2\sigma^2}\right]\right\} \\
		=&(2\pi\sigma^2)^{-\frac{m}{2}}\prod_{i=1}^{m}\left[\exp\left(-\frac{x_i^2-2x_i\mu+\mu^2}{2\sigma^2}\right)\right] \\
		=&(2\pi\sigma^2)^{-\frac{m}{2}}\exp\left(-\frac{1}{2\sigma^2}\sum_{i=1}^{m}x_i^2+\frac{\mu}{\sigma^2}\sum_{i=1}^{m}x_i-\frac{m\mu^2}{2\sigma^2}\right) \\
		=&(2\pi\sigma^2)^{-\frac{m}{2}}\exp\left(-\frac{m\mu^2}{2\sigma^2}\right)\exp\left(-\frac{1}{2\sigma^2}\sum_{i=1}^{m}x_i^2+\frac{\mu}{\sigma^2}\sum_{i=1}^{m}x_i\right)
	\end{align*}
	令:
	\begin{equation*}
		\eta=\left(-\frac{1}{2\sigma^2},\frac{\mu}{\sigma^2}\right)
	\end{equation*}
	可解得:
	\begin{equation*}
		\mu=-\frac{\eta_2}{2\eta_1},\quad\sigma^2=-\frac{1}{2\eta_1}
	\end{equation*}
	于是有:
	\begin{equation*}
		C(\eta)=(2\pi\sigma^2)^{-\frac{m}{2}}\exp\left(-\frac{m\mu^2}{2\sigma^2}\right)=\left(-\frac{\pi}{\eta_1}\right)^{-\frac{m}{2}}\exp\left(\frac{m\eta_2^2}{4\eta_1}\right)
	\end{equation*}
	由$\eta$与$(\mu,\sigma^2)$的关系即可得到$\eta\in\mathbb{R}^{-}\times\mathbb{R}^{}$。\par
	(2)由(1)和\cref{prop:ExponentialFamily}(7)(8)立即可得。
\end{proof}

\begin{theorem}
	设$X\sim\operatorname{LN}(\mu,\sigma^2),\;\mu\in\mathbb{R}^{},\sigma^2>0$,$\seq{X}{m}$是$X$的一个样本,则:
	\begin{enumerate}
		\item $\seq{X}{m}$的分布是指数族,其标准形式、自然参数、$C(\eta)$、$T(x)$和$h(x)$分别为:
		\begin{gather*}
			\begin{aligned}
				&p(X_1=x_1,X_2=x_2,\dots,X_m=x_m) \\
				=&\left(-\frac{\pi}{\eta_1}\right)^{-\frac{m}{2}}\exp\left(\frac{m\eta_2^2}{4\eta_1}\right)\exp\left(\eta_1\sum_{i=1}^{m}\ln^2 x_i+\eta_2\sum_{i=1}^{m}\ln x_i\right)\prod_{i=1}^{m}\frac{I_{(0,+\infty)}(x_i)}{x_i}
			\end{aligned} \\
			\eta=\left(-\frac{1}{2\sigma^2},\frac{\mu}{\sigma^2}\right),\quad C(\eta)=\left(-\frac{\pi}{\eta_1}\right)^{-\frac{m}{2}}\exp\left(\frac{m\eta_2^2}{4\eta_1}\right) \\
			T(x)=\left(\sum_{i=1}^{m}\ln^2 x_i,\sum_{i=1}^{m}\ln x_i\right),\quad h(x)=\prod_{i=1}^{m}\frac{I_{(0,+\infty)}(x_i)}{x_i}
		\end{gather*}
		自然参数空间为$\mathbb{R}^{-}\times\mathbb{R}^{}$;
		\item $\left(\sum\limits_{i=1}^{m}\ln^2 X_i,\sum\limits_{i=1}^{m}\ln X_i\right)$是该分布族的充分统计量、完备统计量和极小充分统计量;;
	\end{enumerate}
\end{theorem}
\begin{proof}
	(1)由对数正态分布的定义可以得到:
	\begin{align*}
		&p(X_1=x_1,X_2=x_2,\dots,X_m=x_m) \\
		=&\prod_{i=1}^{m}\left\{(2\pi\sigma^2)^{-\frac{1}{2}}\frac{1}{x_i}\exp\left[-\frac{(\ln x_i-\mu)^2}{2\sigma^2}\right]I_{(0,+\infty)}(x_i)\right\} \\
		=&(2\pi\sigma^2)^{-\frac{m}{2}}\prod_{i=1}^{m}\left[\exp\left(-\frac{\ln^2 x_i-2\mu\ln x_i+\mu^2}{2\sigma^2}\right)\frac{I_{(0,+\infty)}(x_i)}{x_i}\right] \\
		=&(2\pi\sigma^2)^{-\frac{m}{2}}\exp\left(-\frac{1}{2\sigma^2}\sum_{i=1}^{m}\ln^2 x_i+\frac{\mu}{\sigma^2}\sum_{i=1}^{m}\ln x_i-\frac{m\mu^2}{2\sigma^2}\right)\prod_{i=1}^{m}\frac{I_{(0,+\infty)}(x_i)}{x_i} \\
		=&(2\pi\sigma^2)^{-\frac{m}{2}}\exp\left(-\frac{m\mu^2}{2\sigma^2}\right)\exp\left(-\frac{1}{2\sigma^2}\sum_{i=1}^{m}\ln^2 x_i+\frac{\mu}{\sigma^2}\sum_{i=1}^{m}\ln x_i\right)\prod_{i=1}^{m}\frac{I_{(0,+\infty)}(x_i)}{x_i}
	\end{align*}
	令:
	\begin{equation*}
		\eta=\left(-\frac{1}{2\sigma^2},\frac{\mu}{\sigma^2}\right)
	\end{equation*}
	可解得:
	\begin{equation*}
		\mu=-\frac{\eta_2}{2\eta_1},\quad\sigma^2=-\frac{1}{2\eta_1}
	\end{equation*}
	于是有:
	\begin{equation*}
		C(\eta)=(2\pi\sigma^2)^{-\frac{m}{2}}\exp\left(-\frac{m\mu^2}{2\sigma^2}\right)=\left(-\frac{\pi}{\eta_1}\right)^{-\frac{m}{2}}\exp\left(\frac{m\eta_2^2}{4\eta_1}\right)
	\end{equation*}
	由$\eta$与$(\mu,\sigma^2)$的关系即可得到$\eta\in\mathbb{R}^{-}\times\mathbb{R}^{}$。\par
	(2)由(1)和\cref{prop:ExponentialFamily}(7)(8)立即可得。
\end{proof}

\begin{theorem}
	设$X\sim\operatorname{Exp}(\lambda),\;\lambda\in\mathbb{R}^{+}$,$\seq{X}{m}$是$X$的一个样本,则:
	\begin{enumerate}
		\item $\seq{X}{m}$的分布是指数族,其标准形式、自然参数、$C(\eta)$、$T(x)$和$h(x)$分别为:
		\begin{gather*}
			p(X_1=x_1,X_2=x_2,\dots,X_m=x_m) =(-\eta)^m\lambda^m\exp\left(\eta\sum_{i=1}^{m}x_i\right)\prod_{i=1}^mI_{\mathbb{R}^{}\backslash\mathbb{R}^{-}}(x_i) \\
			\eta=-\lambda,\quad C(\eta)=(-\eta)^m,\quad
			T(x)=\sum_{i=1}^{m}x_i,\quad h(x)=\prod_{i=1}^{m}I_{\mathbb{R}^{}\backslash\mathbb{R}^{-}}(x_i)
		\end{gather*}
		自然参数空间为$\mathbb{R}^{-}$;
		\item $\sum\limits_{i=1}^{m}X_i$为该分布族的充分统计量、完备统计量、极小充分统计量;
	\end{enumerate}
\end{theorem}
\begin{proof}
	(1)由指数分布的定义可以得到:
	\begin{align*}
		&p(X_1=x_1,X_2=x_2,\dots,X_m=x_m) \\
		=&\prod_{i=1}^{m}\left[\lambda\exp(-\lambda x_i)I_{\mathbb{R}^{}\backslash\mathbb{R}^{-}}(x_i)\right]=\lambda^m\exp\left(-\lambda\sum_{i=1}^{m}x_i\right)\prod_{i=1}^mI_{\mathbb{R}^{}\backslash\mathbb{R}^{-}}(x_i)
	\end{align*}
	令$\eta=-\lambda$,于是有$C(\eta)=\lambda^m=(-\eta)^m$。由$\eta$与$\lambda$的关系可得$\eta\in\mathbb{R}^{-}$。\par
	(2)由(1)和\cref{prop:ExponentialFamily}(7)(8)立即可得。
\end{proof}

\begin{theorem}
	设$X\sim\operatorname{Laplace}(\theta),\;\theta\in\mathbb{R}^{+}$,$\seq{X}{m}$是$X$的一个样本,则:
	\begin{enumerate}
		\item $\seq{X}{m}$的分布是指数族,其标准形式、自然参数、$C(\eta)$、$T(x)$和$h(x)$分别为:
		\begin{gather*}
			p(X_1=x_1,X_2=x_2,\dots,X_m=x_m) =(-\eta)^m\exp\left(\eta\sum_{i=1}^{m}|x_i|\right) \\
			\eta=-\theta^{-1},\quad C(\eta)=(-\eta)^m,\quad
			T(x)=\sum_{i=1}^{m}|x_i|,\quad h(x)=1
		\end{gather*}
		自然参数空间为$\mathbb{R}^{-}$;
		\item $\sum\limits_{i=1}^{m}|X_i|$为该分布族的充分统计量、完备统计量、极小充分统计量;
	\end{enumerate}
\end{theorem}
\begin{proof}
	(1)由Laplace分布的定义可以得到:
	\begin{align*}
		&p(X_1=x_1,X_2=x_2,\dots,X_m=x_m) \\
		=&\prod_{i=1}^{m}\left[\frac{1}{\theta}\exp\left(-\frac{|x_i|}{\theta}\right)\right]=\frac{1}{\theta^m}\exp\left(-\frac{1}{\theta}\sum_{i=1}^{m}|x_i|\right)
	\end{align*}
	令$\eta=-\theta^{-1}$,于是有$C(\eta)=\theta^{-m}=(-\eta)^m$。由$\eta$与$\theta$的关系可得$\eta\in\mathbb{R}^{-}$。\par
	(2)由(1)和\cref{prop:ExponentialFamily}(7)(8)立即可得。
\end{proof}

\begin{theorem}
	设$X\sim\operatorname{Rayleigh}(\lambda),\;\lambda\in\mathbb{R}^{+}$,$\seq{X}{m}$是$X$的一个样本,则:
	\begin{enumerate}
		\item $\seq{X}{m}$的分布是指数族,其标准形式、自然参数、$C(\eta)$、$T(x)$和$h(x)$分别为:
		\begin{gather*}
			p(X_1=x_1,X_2=x_2,\dots,X_m=x_m) =(-\eta)^m\exp\left(\eta\sum_{i=1}^{m}x_i^2\right)\prod_{i=1}^{m}\left[2x_iI_{\mathbb{R}^{}\backslash\mathbb{R}^{-}}(x_i)\right] \\
			\eta=-\lambda,\quad C(\eta)=(-\eta)^m,\quad
			T(x)=\sum_{i=1}^{m}x_i^2,\quad h(x)=\prod_{i=1}^{m}\left[2x_iI_{\mathbb{R}^{}\backslash\mathbb{R}^{-}}(x_i)\right]
		\end{gather*}
		自然参数空间为$\mathbb{R}^{-}$;
		\item $\sum\limits_{i=1}^{m}X_i^2$为该分布族的充分统计量、完备统计量、极小充分统计量;
	\end{enumerate}
\end{theorem}
\begin{proof}
	(1)由Rayleigh分布的定义可以得到:
	\begin{align*}
		&p(X_1=x_1,X_2=x_2,\dots,X_m=x_m)=\prod_{i=1}^{m}\left[2\lambda x_i\exp(-\lambda x_i^2)I_{\mathbb{R}^{}\backslash\mathbb{R}^{-}}(x_i)\right] \\
		=&\lambda^m\exp\left(-\lambda\sum_{i=1}^{m}x_i^2\right)\prod_{i=1}^{m}\left[2x_iI_{\mathbb{R}^{}\backslash\mathbb{R}^{-}}(x_i)\right]
	\end{align*}
	令$\eta=-\lambda$,于是有$C(\eta)=\lambda^{m}=(-\eta)^m$。由$\eta$与$\lambda$的关系可得$\eta\in\mathbb{R}^{-}$。\par
	(2)由(1)和\cref{prop:ExponentialFamily}(7)(8)立即可得。
\end{proof}

\begin{theorem}
	设$X\sim\operatorname{Gamma}(\alpha,\lambda),\;\alpha,\lambda\in\mathbb{R}^{+}$,$\seq{X}{m}$是$X$的一个样本,则:
	\begin{enumerate}
		\item $\seq{X}{m}$的分布是指数族,其标准形式、自然参数、$C(\eta)$、$T(x)$和$h(x)$分别为:
		\begin{gather*}
			\begin{aligned}
				&p(X_1=x_1,X_2=x_2,\dots,X_m=x_m) \\
				=&\frac{(-\eta_2)^{m(\eta_1+1)}}{\Gamma^m(\eta_1+1)}\exp\left[\eta_1\sum_{i=1}^{m}\ln x_i+\eta_2\sum_{i=1}^{m}x_i\right]\prod_{i=1}^{m}I_{\mathbb{R}^{}\backslash\mathbb{R}^{-}}(x_i)
			\end{aligned} \\
			\eta=(\alpha-1,-\lambda),\quad C(\eta)=\frac{(-\eta_2)^{m(\eta_1+1)}}{\Gamma^m(\eta_1+1)} \\
			T(x)=\left(\sum_{i=1}^{m}\ln x_i,\sum_{i=1}^{m}x_i\right),\quad h(x)=\prod_{i=1}^{m}I_{\mathbb{R}^{}\backslash\mathbb{R}^{-}}(x_i)
		\end{gather*}
		自然参数空间为$(-1,+\infty)\times\mathbb{R}^{-}$;
		\item $\left(\sum\limits_{i=1}^{m}\ln X_i,\sum\limits_{i=1}^{m}X_i\right)$为该分布族的充分统计量、完备统计量、极小充分统计量;
	\end{enumerate}
\end{theorem}
\begin{proof}
	(1)由Gamma分布的定义可以得到:
	\begin{align*}
		&p(X_1=x_1,X_2=x_2,\dots,X_m=x_m)=\prod_{i=1}^{m}\left[\frac{\lambda^{\alpha}}{\Gamma(\alpha)}x_i^{\alpha-1}e^{-\lambda x_i}I_{\mathbb{R}^{}\backslash\mathbb{R}^{-}}(x_i)\right] \\
		=&\frac{\lambda^{m\alpha}}{\Gamma^m(\alpha)}\exp\left[(\alpha-1)\sum_{i=1}^{m}\ln x_i-\lambda\sum_{i=1}^{m}x_i\right]\prod_{i=1}^{m}I_{\mathbb{R}^{}\backslash\mathbb{R}^{-}}(x_i)
	\end{align*}
	令$\eta=(\alpha-1,-\lambda)$,于是有:
	\begin{equation*}
		C(\eta)=\frac{\lambda^{m\alpha}}{\Gamma^m(\alpha)}=\frac{(-\eta_2)^{m(\eta_1+1)}}{\Gamma^m(\eta_1+1)}
	\end{equation*}
	由$\eta$与$(\alpha,\lambda)$的关系可得$\eta\in(-1,+\infty)\times\mathbb{R}^{-}$。\par
	(2)由(1)和\cref{prop:ExponentialFamily}(7)(8)立即可得。
\end{proof}

\begin{theorem}
	设$X\sim\operatorname{Beta}(\alpha,\beta),\;\alpha,\beta\in\mathbb{R}^{+}$,$\seq{X}{m}$是$X$的一个样本,则:
	\begin{enumerate}
		\item $\seq{X}{m}$的分布是指数族,其标准形式、自然参数、$C(\eta)$、$T(x)$和$h(x)$分别为:
		\begin{gather*}
			\begin{aligned}
				&p(X_1=x_1,X_2=x_2,\dots,X_m=x_m) \\
				=&\left[\frac{\Gamma(\eta_1+\eta_2+2)}{\Gamma(\eta_1+1)\Gamma(\eta_2+1)}\right]^m\exp\left[\eta_1\sum_{i=1}^{m}\ln x_i+\eta_2\sum_{i=1}^{m}\ln(1-x_i)\right]\prod_{i=1}^{m}I_{(0,1)}(x_i)
			\end{aligned} \\
			\eta=(\alpha-1,\beta-1),\quad C(\eta)=\left[\frac{\Gamma(\eta_1+\eta_2+2)}{\Gamma(\eta_1+1)\Gamma(\eta_2+1)}\right]^m \\
			T(x)=\left(\sum_{i=1}^{m}\ln x_i,\sum_{i=1}^{m}\ln(1-x_i)\right),\quad h(x)=\prod_{i=1}^{m}I_{(0,1)}(x_i)
		\end{gather*}
		自然参数空间为$(-1,+\infty)\times(-1,+\infty)$;
		\item $\left(\sum\limits_{i=1}^{m}\ln X_i,\sum\limits_{i=1}^{m}\ln(1-X_i)\right)$为该分布族的充分统计量、完备统计量、极小充分统计量;
	\end{enumerate}
\end{theorem}
\begin{proof}
	(1)由Beta分布的定义可以得到:
	\begin{align*}
		&p(X_1=x_1,X_2=x_2,\dots,X_m=x_m)=\prod_{i=1}^{m}\left[\frac{\Gamma(\alpha+\beta)}{\Gamma(\alpha)\Gamma(\beta)}x_i^{\alpha-1}(1-x_i)^{\beta-1}I_{(0,1)}(x_i)\right] \\
		=&\left[\frac{\Gamma(\alpha+\beta)}{\Gamma(\alpha)\Gamma(\beta)}\right]^m\exp\left[(\alpha-1)\sum_{i=1}^{m}\ln x_i+(\beta-1)\sum_{i=1}^{m}\ln(1-x_i)\right]\prod_{i=1}^{m}I_{(0,1)}(x_i)
	\end{align*}
	令$\eta=(\alpha-1,\beta-1)$,于是有:
	\begin{equation*}
		C(\eta)=\left[\frac{\Gamma(\alpha+\beta)}{\Gamma(\alpha)\Gamma(\beta)}\right]^m=\left[\frac{\Gamma(\eta_1+\eta_2+2)}{\Gamma(\eta_1+1)\Gamma(\eta_2+1)}\right]^m
	\end{equation*}
	由$\eta$与$(\alpha,\beta)$的关系可得$\eta\in(-1,+\infty)\times(-1,+\infty)$。\par
	(2)由(1)和\cref{prop:ExponentialFamily}(7)(8)立即可得。
\end{proof}


\section{抽样分布}

\subsection{一维总体}
\begin{theorem}
	设$\seq{X}{m}\;\text{i.i.d.}\sim N(\mu,\sigma^2)$,$\seq{Y}{n}\;\text{i.i.d.}\sim N(\nu,\sigma^2)$,$\seq{X}{m}$和$\seq{Y}{n}$相互独立,$\overline{X},\overline{Y}$为样本均值,$S_X^2,S_Y^2$为样本方差,则:
	\begin{enumerate}
		\item $\overline{X}\sim N\left(\mu,\dfrac{\sigma^2}{m}\right)$;
		\item $\dfrac{(m-1)S_X^2}{\sigma^2}\sim\chi_{m-1}^2$;
		\item $\overline{X}$与$S_X^2$独立;
		\item $\dfrac{\sqrt{m}(\overline{X}-\mu)}{S_X}\sim t_{m-1}$;
		\item $\dfrac{\overline{X}-\overline{Y}-(\mu-\nu)}{S_w}\sqrt{\dfrac{mn}{m+n}}\sim t_{m+n-2}$,其中:
		\begin{equation*}
			(m+n-2)S_w^2=(m-1)S_X^2+(n-1)S_Y^2
		\end{equation*}
		\item 若$\seq{X}{m}\;\text{i.i.d.}\sim N(\mu,\sigma_1^2)$,$\seq{Y}{n}\;\text{i.i.d.}\sim N(\nu,\sigma_2^2)$,其它条件不变,则:
		\begin{equation*}
			\frac{S_X^2\sigma_2^2}{S_Y^2\sigma_1^2}\sim F_{m-1,n-1}
		\end{equation*}
	\end{enumerate}
\end{theorem}
\begin{proof}
	令$\mathbf{X}=(\seq{X}{m})^T,\;\mathbf{Y}=(\seq{Y}{n})^T$。因为$\seq{X}{m}\;\text{i.i.d.}\sim N(\mu,\sigma^2)$,$\seq{Y}{n}\;\text{i.i.d.}\sim N(\nu,\sigma^2)$,所以$\mathbf{X}\sim N_m(\boldsymbol{\mu},\Sigma_m),\;\mathbf{Y}\sim N_n(\boldsymbol{\nu},\Sigma_n)$,其中:
	\begin{equation*}
		\boldsymbol{\mu}=\mu\mathbf{1}_m,\quad\Sigma_m=\sigma^2I_m,\quad\boldsymbol{\nu}=\nu\mathbf{1}_n,\quad\Sigma_n=\sigma^2I_n
	\end{equation*}\par
	(1)令$m$维行向量$c=\left(\dfrac{1}{m},\dfrac{1}{m},\dots,\dfrac{1}{m}\right)$,
	由\cref{prop:MultiNormal}(2)可知:
	\begin{equation*}
		\overline{X}=c\mathbf{X}\sim N(c\boldsymbol{\mu},c\Sigma c^T)
	\end{equation*}
	而:
	\begin{equation*}
		c\boldsymbol{\mu}=\sum_{i=1}^{m}\frac{\mu}{m}=\mu,\;c\Sigma c^T=\sum_{i=1}^{m}\frac{\sigma^2}{m^2}=\frac{\sigma^2}{m}
	\end{equation*}
	所以$\overline{X}\sim N\left(\mu,\dfrac{\sigma^2}{m}\right)$。\par
	(2)由Schmidit正交化\info{考虑链接什么过来}可知存在正交矩阵:
	\begin{equation*}
		A=
		\begin{pmatrix}
			\frac{1}{\sqrt{m}} & \frac{1}{\sqrt{m}} & \cdots & \frac{1}{\sqrt{m}} \\
			a_{21} & a_{22} & \cdots & a_{2m} \\
			\vdots & \vdots & \ddots & \vdots \\
			a_{m1} & a_{m2} & \cdots & a_{mm}
		\end{pmatrix}
	\end{equation*}
	令$\mathbf{Z}=A\mathbf{X}$,由\cref{prop:MultiNormal}(2)可知$\mathbf{Z}\sim N_m(A\boldsymbol{\mu},\sigma^2I_m)$。由\cref{prop:MultiNormal}(3)可知$\mathbf{Z}_i\sim N(\mu_i,\sigma^2)$,其中:
	\begin{equation*}
		\mu_i=\mu\sum_{j=1}^{m}a_{ij}
	\end{equation*}
	因为$A$是一个正交矩阵,所以:
	\begin{equation*}
		\mu_i=\sqrt{m}\mu\sum_{j=1}^{m}\frac{1}{\sqrt{m}}a_{ij}=\sqrt{m}\mu\left(\dfrac{1}{\sqrt{m}},\dfrac{1}{\sqrt{m}},\dots,\dfrac{1}{\sqrt{m}}\right)(a_{i1},a_{i2},\dots,a_{im})^T=0
	\end{equation*}
	由\cref{prop:MultiNormal}(2)即可得:
	\begin{equation*}
		\frac{\mathbf{Z}_i}{\sigma}\sim N(0,1),\;\forall\;i=1,2,\dots,m
	\end{equation*}
	因为$\mathbf{Z}=A\mathbf{X}$,所以:
	\begin{equation*}
		\mathbf{Z}_1=\frac{1}{\sqrt{m}}\sum_{i=1}^{m}X_i=\sqrt{m}\overline{X}
	\end{equation*}
	因为:
	\begin{equation*}
		\mathbf{Z}^T\mathbf{Z}=\sum_{i=1}^{m}\mathbf{Z}_i^2=\mathbf{X}^TA^TA\mathbf{X}=\mathbf{X}^T\mathbf{X}=\sum_{i=1}^{m}X_i^2
	\end{equation*}
	于是:
	\begin{equation*}
		(m-1)S_X^2=\sum_{i=1}^{m}(X_i-\overline{X})^2=\sum_{i=1}^{m}X_i^2-m\overline{X}^2=\sum_{i=1}^{m}\mathbf{Z}_i^2-\mathbf{Z}_1^2=\sum_{i=2}^{m}\mathbf{Z}_i^2
	\end{equation*}
	所以:
	\begin{equation*}
		\frac{(m-1)S_X^2}{\sigma^2}=\sum_{i=2}^{m}\left(\frac{\mathbf{Z}_i}{\sigma}\right)^2\sim\chi_{m-1}^2
	\end{equation*}\par
	(3)由(2)的证明过程可得$\mathbf{Z}\sim N_m(A\boldsymbol{\mu},\sigma^2I_m)$,根据\cref{prop:MultiNormal}(8)可知$\seq{\mathbf{Z}}{m}$相互独立。而:
	\begin{equation*}
		S_X^2=\frac{\sum\limits_{i=2}^{m}\mathbf{Z}_i^2}{(m-1)},\;\overline{X}=\frac{\mathbf{Z}_1}{\sqrt{m}}
	\end{equation*}
	所以$S_X^2$与$\overline{X}$独立。\par
	(4)对$\overline{X}$进行标准化可得:
	\begin{equation*}
		\frac{\overline{X}-\mu}{\sqrt{\frac{\sigma^2}{m}}}=\frac{\sqrt{m}(\overline{X}-\mu)}{\sigma}\sim N(0,1)
	\end{equation*}
	由(2)和(3)进一步可得:
	\begin{equation*}
		\frac{\dfrac{\sqrt{m}(\overline{X}-\mu)}{\sigma}}{\sqrt{\dfrac{(m-1)S_X^2}{\sigma^2(m-1)}}}=\frac{\sqrt{m}(\overline{X}-\mu)}{S_X}\sim t_{m-1}
	\end{equation*}\par
	(5)由(1)(得到$\overline{X}$和$\overline{Y}$的分布)、$\seq{X}{m}$与$\seq{Y}{n}$相互独立(由\cref{prop:MultiNormal}(6)得到二维随机向量$(\overline{X},\overline{Y})^T$的分布)和\cref{prop:MultiNormal}(2)(对$(\overline{X},\overline{Y})^T$用二维行向量$(1,-1)$做线性变换)可得:
	\begin{equation*}
		\overline{X}-\overline{Y}\sim N\left(\mu-\nu,\frac{\sigma^2}{m}+\frac{\sigma^2}{n}\right)
	\end{equation*}
	于是:
	\begin{equation*}
		\frac{\overline{X}-\overline{Y}-(\mu-\nu)}{\sqrt{\dfrac{m+n}{mn}\sigma^2}}\sim N(0,1)
	\end{equation*}
	由(2)可得:
	\begin{equation*}
		\frac{(m-1)S_X^2}{\sigma^2}\sim\chi_{m-1}^2,\;
		\frac{(n-1)S_Y^2}{\sigma^2}\sim\chi_{n-1}^2
	\end{equation*}
	由\cref{prop:Chi2Distribution}(1)可得:
	\begin{equation*}
		\frac{(m-1)S_X^2+(n-1)S_Y^2}{\sigma^2}\sim\chi_{m+n-2}^2
	\end{equation*}
	于是:
	\begin{equation*}
		\frac{(m+n-2)S_w^2}{\sigma^2}\sim\chi_{m+n-2}^2
	\end{equation*}
	由(3)可得$\overline{X}$与$S_X^2$独立、$\overline{Y}$与$S_Y^2$独立,所以:
	\begin{equation*}
		\frac{\dfrac{\overline{X}-\overline{Y}-(\mu-\nu)}{\sqrt{\dfrac{m+n}{mn}\sigma^2}}}{\sqrt{\dfrac{(m+n-2)S_w^2}{\sigma^2(m+n-2)}}}=\frac{\overline{X}-\overline{Y}-(\mu-\nu)}{S_w}\sqrt{\dfrac{mn}{m+n}}\sim t_{m+n-2}
	\end{equation*}\par
	(6)由(2)可知:
	\begin{equation*}
		\frac{(m-1)S_X^2}{\sigma_1^2}\sim\chi_{m-1}^2,\;
		\frac{(n-1)S_Y^2}{\sigma_2^2}\sim\chi_{n-1}^2
	\end{equation*}
	因为$\seq{X}{m}$和$\seq{Y}{n}$相互独立,所以上两式也相互独立。由$F$分布的定义即可得:
	\begin{equation*}
		\frac{\dfrac{(m-1)S_X^2}{\sigma_1^2(m-1)}}{\dfrac{(n-1)S_Y^2}{\sigma_2^2(n-1)}}=\frac{S_X^2\sigma_2^2}{S_Y^2\sigma_1^2}\sim F_{m-1,n-1}\qedhere
	\end{equation*}
\end{proof}

\subsection{多维总体}
\begin{theorem}\label{theo:MultiVariateSamplingDist}
	设$\mathbf{X_1},\mathbf{X_2},\dots,\mathbf{X_m}\;\text{i.i.d.}\sim\operatorname{N}_p(\boldsymbol{\mu},\Sigma)$,$\mathbf{Y_1},\mathbf{Y_2},\dots,\mathbf{Y_n}\;\text{i.i.d.}\sim \operatorname{N}_p(\boldsymbol{\nu},\Sigma)$,$\Sigma>\mathbf{0}$,$\mathbf{X_1},\mathbf{X_2},\dots,\mathbf{X_m}$和$\mathbf{Y_1},\mathbf{Y_2},\dots,\mathbf{Y_n}$相互独立,$\overline{\mathbf{X}},\overline{\mathbf{Y}}$为样本均值向量,$S_X,S_Y$为样本协方差矩阵,则:
	\begin{enumerate}
		\item $\overline{\mathbf{X}}\sim\operatorname{N}_p\left(\boldsymbol{\mu},\dfrac{1}{m}\Sigma\right)$;
		\item $(m-1)S_X\sim\operatorname{W}_p(m-1,\Sigma)$;
		\item $\overline{\mathbf{X}}$与$S_X$相互独立;
		\item $m(\overline{\mathbf{X}}-\boldsymbol{\mu})^TS_X^{-1}(\overline{\mathbf{X}}-\boldsymbol{\mu})\sim T^2(p,m-1)$;
		\item 若$\boldsymbol{\mu}=\boldsymbol{\nu}$,则$\dfrac{mn}{m+n}(\overline{\mathbf{X}}-\overline{\mathbf{Y}})^TS_{w}^{-1}(\overline{\mathbf{X}}-\overline{\mathbf{Y}})\sim T^2(p,m+n-2)$,其中:
		\begin{equation*}
			(m+n-2)S_w=(m-1)S_X+(n-1)S_Y 
		\end{equation*}
	\end{enumerate}
\end{theorem}
\begin{proof}
	(1)\par
	(2)\par
	(3)\par
	(4)由(1)(2)(3)和\cref{prop:T^2}(1)可得:
	\begin{equation*}
		(m-1)m(\overline{\mathbf{X}}-\boldsymbol{\mu})^T[(m-1)S_X]^{-1}(\overline{\mathbf{X}}-\boldsymbol{\mu})=m(\overline{\mathbf{X}}-\boldsymbol{\mu})^TS_X^{-1}(\overline{\mathbf{X}}-\boldsymbol{\mu})\sim T^2(p,m-1)
	\end{equation*}\par
	(5)由(1)(得到$\overline{\mathbf{X}}$和$\overline{\mathbf{Y}}$的分布)、\cref{prop:MultiNormal}(2)(得到$-\overline{\mathbf{Y}}$的分布)、\cref{prop:MultiNormal}(7)以及$\overline{\mathbf{X}}$与$\overline{\mathbf{Y}}$的独立性可得
	\begin{equation*}
		\overline{\mathbf{X}}-\overline{\mathbf{Y}}\sim \operatorname{N}_p\left(\boldsymbol{\mu}-\boldsymbol{\nu},\frac{\Sigma}{m}+\frac{\Sigma}{n}\right)
	\end{equation*}
	于是当$\boldsymbol{\mu}=\boldsymbol{\nu}$时由\cref{prop:MultiNormal}(2)可得:
	\begin{equation*}
		\overline{\mathbf{X}}-\overline{\mathbf{Y}}-(\mu-\nu)=\overline{\mathbf{X}}-\overline{\mathbf{Y}}\sim \operatorname{N}_p\left(\mathbf{0},\frac{\Sigma}{m}+\frac{\Sigma}{n}\right)
	\end{equation*}
	由(2)、$\mathbf{X_1},\mathbf{X_2},\dots,\mathbf{X_m}$和$\mathbf{Y_1},\mathbf{Y_2},\dots,\mathbf{Y_m}$之间的独立性以及\cref{prop:Wishart}(1)可得:
	\begin{gather*}
		(m-1)S_X\sim\operatorname{W}_p(m-1,\Sigma),\quad(n-1)S_Y\sim\operatorname{W}_p(n-1,\Sigma) \\
		(m+n-2)S_w=(m-1)S_X+(n-1)S_Y\sim\operatorname{W}_p(m+n-2,\Sigma)
	\end{gather*}
	于是:
	\begin{align*}
		&\frac{m+n-2}{\dfrac{m+n}{mn}}(\overline{\mathbf{X}}-\overline{\mathbf{Y}})^T[(m+n-2)S_{w}]^{-1}(\overline{\mathbf{X}}-\overline{\mathbf{Y}}) \\
		=&\frac{mn}{m+n}(\overline{\mathbf{X}}-\overline{\mathbf{Y}})^TS_{w}^{-1}(\overline{\mathbf{X}}-\overline{\mathbf{Y}})\sim T^2(p,m+n-2)\qedhere
	\end{align*}
\end{proof}

\subsection{多维正态总体参数的检验}
\subsubsection{均值的检验}
\begin{theorem}
	设$\mathbf{X_1},\mathbf{X_2},\dots,\mathbf{X_n}\;\text{i.i.d.}\sim\operatorname{N}_p(\boldsymbol{\mu},\Sigma)$,$\Sigma>\mathbf{0}$,$\overline{\mathbf{X}}$为样本均值向量,$S$为样本协方差矩阵。假设检验问题:
	\begin{equation*}
		H_0:\boldsymbol{\mu}=\boldsymbol{\mu}_0,\quad H_1:\boldsymbol{\mu}\ne\boldsymbol{\mu}_0
	\end{equation*}
	的检验统计量与拒绝域为:
	\begin{gather*}
		\begin{cases}
			\chi_0^2=n(\overline{\mathbf{X}}-\boldsymbol{\mu}_0)^T\Sigma^{-1}(\overline{\mathbf{X}}-\boldsymbol{\mu}_0)\sim\chi_p^2,&\Sigma\text{已知} \\
			T^2=n(\overline{\mathbf{X}}-\boldsymbol{\mu}_0)^TS^{-1}(\overline{\mathbf{X}}-\boldsymbol{\mu}_0)\sim T^2(p,n-1),&\Sigma\text{未知}
		\end{cases} \\
		\begin{cases}
			\{\chi_0^2>\chi_p^2(\alpha)\},&\Sigma\text{已知} \\
			\{T^2>T^2_{p,n-1}(\alpha)\},&\Sigma\text{未知} \\
		\end{cases}
	\end{gather*}
\end{theorem}
\begin{proof}
	\textbf{(1)$\;\Sigma$已知:}因为$\Sigma>\mathbf{0}$,所以$\Sigma^{-1},\Sigma^{-\frac{1}{2}}$存在\info{正定矩阵可逆}。在原价设成立的情况下,由\cref{theo:MatNormalLinearTransform}和\cref{theo:MultiVariateSamplingDist}(1)可知:
	\begin{equation*}
		\sqrt{n}\Sigma^{-\frac{1}{2}}(\overline{\mathbf{X}}-\boldsymbol{\mu}_0)\sim\operatorname{N}_p(\mathbf{0},I_p)
	\end{equation*}
	由\cref{prop:ReverseSquareRootMat}(3)可得:
	\begin{align*}
		n(\overline{\mathbf{X}}-\boldsymbol{\mu}_0)^T\Sigma^{-1}(\overline{\mathbf{X}}-\boldsymbol{\mu}_0)
		&=(\overline{\mathbf{X}}-\boldsymbol{\mu}_0)^T\left(\frac{1}{n}\Sigma\right)^{-1}(\overline{\mathbf{X}}-\boldsymbol{\mu}_0) \\
		&=(\overline{\mathbf{X}}-\boldsymbol{\mu}_0)^T\left(\sqrt{n}\Sigma^{-\frac{1}{2}}\right)^{2}(\overline{\mathbf{X}}-\boldsymbol{\mu}_0) \\
		&=[\sqrt{n}\Sigma^{-\frac{1}{2}}(\overline{\mathbf{X}}-\boldsymbol{\mu}_0)]^T[\sqrt{n}\Sigma^{-\frac{1}{2}}(\overline{\mathbf{X}}-\boldsymbol{\mu}_0)]\sim\chi^2_p
	\end{align*}
	上式值越大,说明样本均值与$\mu_0$的Mahalanobis距离越大,所以拒绝域取右侧单尾。\par
	\textbf{(2)$\;\Sigma$未知:}由\cref{theo:MultiVariateSamplingDist}(4)可得统计量的分布,而$S$是$\Sigma$的无偏估计,所以该统计量值越大说明样本均值与$\mu_0$的Mahalanobis距离越大,拒绝域取右侧单尾。
\end{proof}
\subsubsection{两总体均值的比较}
\begin{theorem}
	设$\mathbf{X_1},\mathbf{X_2},\dots,\mathbf{X_m}\;\text{i.i.d.}\sim\operatorname{N}_p(\boldsymbol{\mu},\Sigma),\;\mathbf{Y_1},\mathbf{Y_2},\dots,\mathbf{Y_n}\;\text{i.i.d.}\sim\operatorname{N}_p(\boldsymbol{\nu},\Sigma)$,$\Sigma>\mathbf{0}$,$\mathbf{X_1},\mathbf{X_2},\dots,\mathbf{X_m}$和$\mathbf{Y_1},\mathbf{Y_2},\dots,\mathbf{Y_n}$相互独立,$\overline{\mathbf{X}},\overline{\mathbf{Y}}$为样本均值向量,$S_X,S_Y$为样本协方差矩阵。假设检验问题:
	\begin{equation*}
		H_0:\boldsymbol{\mu}=\boldsymbol{\nu},\quad H_1:\boldsymbol{\mu}\ne\boldsymbol{\nu}
	\end{equation*}
	的检验统计量与拒绝域为:
	\begin{equation*}
		T^2=\frac{mn}{m+n}(\overline{\mathbf{X}}-\overline{\mathbf{Y}})^TS_{w}^{-1}(\overline{\mathbf{X}}-\overline{\mathbf{Y}})\sim T^2(p,m+n-2),\quad\{T^2>T^2_{p,m+n-2}(\alpha)\}
	\end{equation*}
	其中$(m+n-2)S_w=(m-1)S_X+(n-1)S_Y$。
\end{theorem}
\begin{proof}
	由\cref{theo:MultiVariateSamplingDist}(4)可得统计量的分布,该统计量值越大说明两个总体均值相近的可能性越小,所以拒绝域取右侧单尾。
\end{proof}

