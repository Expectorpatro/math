\section{主成分分析}

\gls{PCA}的目的是:对数据进行一个线性变换,在最大程度保留原始信息的前提下去除数据中彼此相关的信息。反映在变量上就是说,对所有的变量进行一个线性变换,使得变换后得到的变量彼此之间不相关,并且是所有可能的线性变换中方差最大的一些变量(我们认为方差体现了信息量的大小)。

\subsection{总体主成分分析}
\begin{definition}
	设$\mathbf{X}$是一个$n$维随机向量,其均值向量为$\mu=(\seq{\mu}{n})$、协方差矩阵为$\Sigma=(\sigma_{ij}),\;i,j=1,2,\dots,n$。对$\mathbf{X}$进行一个线性变换$\mathcal{T}$得到一个$n$维随机向量$\mathbf{Y}=(\seq{Y}{n})^{\top}$,$\mathcal{T}$的矩阵为$A=(\alpha_1^{\top};\alpha_2^{\top};\cdots;\alpha_n^{\top})$。若:
	\begin{enumerate}
		\item $AA^{\top}=I_n$;
		\item $\operatorname{Cov}(\mathbf{Y})$是一个对角矩阵,即$\operatorname{Cov}(Y_i,Y_j)=0,\;i\ne j$;
		\item $Y_1$是所有对$\mathbf{X}$进行线性变换后得到的随机变量中方差最大的随机变量,$Y_2$是与$Y_1$不相关的所有对$\mathbf{X}$进行线性变换后得到的随机变量中方差第二大的随机变量,以此类推。
	\end{enumerate}
	则分别称$\seq{Y}{n}$是第一、第二、……、第$n$主成分。
\end{definition}
\begin{theorem}
	若不对$\mathcal{T}$的矩阵$A$作出相应的限制,即去除$AA^{\top}=I_n$的限制,对$\mathbf{X}$进行线性变换后得到的$Y_i,\;i=1,2,\dots,n$的方差可以任意大。
\end{theorem}
\begin{proof}
	由\cref{prop:CovMat}(3)可知:
	\begin{equation*}
		\operatorname{Var}(Y_i)=\operatorname{Cov}(\mathbf{Y})(i,i)=\operatorname{Cov}(A\mathbf{X})(i,i)=(A\Sigma A^{\top})(i,i)=\alpha_i^{\top}\Sigma\alpha_i
	\end{equation*}
	若$\operatorname{Var}(Y_i)>0$,取矩阵$B=kA$,$\mathbf{Z}=kA\mathbf{X}$,则:
	\begin{equation*}
		\operatorname{Var}(Z_i)=(k\alpha_i)^{\top}\Sigma(k\alpha_i)=k^2\alpha_i^{\top}\Sigma\alpha_i
	\end{equation*}
	即改变$k$的值就可对$Y_i,\;i=1,2,\dots,n$的方差进行任意的放缩。
\end{proof}
\begin{theorem}\label{theo:PCA}
	设$\mathbf{X}$是一个$n$维随机向量,$\Sigma$是其协方差矩阵,$\Sigma$的特征值\footnote{若特征多项式有重根,则标准正交化特征向量组不唯一,主成分也不唯一。}从大到小记作$\seq{\lambda}{n}$,$\seq{\varphi}{n}$为对应的标准正交化特征向量,则$\mathbf{X}$的第$i$个主成分以及其方差为:
	\begin{equation*}
		Y_i=\varphi_i^{\top}\mathbf{X},\;\operatorname{Var}(Y_i)=\varphi_i^{\top}\Sigma\varphi_i=\lambda_i
	\end{equation*}
\end{theorem}
\begin{proof}
	考虑到:
	\begin{equation*}
		\operatorname{Var}(Y_i)=\alpha_i^{\top}\Sigma\alpha_i,\quad
		\operatorname{Cov}(Y_i,Y_j)=\alpha_i^{\top}\Sigma\alpha_j
	\end{equation*}
	求解主成分的过程即为求解:
	\begin{gather*}
		\alpha_i=\arg\max\alpha_i^{\top}\Sigma\alpha_i,\quad
		\operatorname{s.t.}
		\begin{cases}
			||\alpha_i||=1,\;&i=1,2,\dots,n\\
			\alpha_i^{\top}\Sigma\alpha_j=0,\;&j<i
		\end{cases}
	\end{gather*}
	由\cref{theo:maxminxAx/xx}可知上述结论成立。
\end{proof}
\begin{definition}
	将第$i$个主成分$Y_i$与变量$X_j$的相关系数$\rho(Y_i,X_j)$称为\gls{FactorLoading}。
\end{definition}
\begin{definition}
	称主成分$\seq{Y}{k}$与变量$X_j$之间的复相关系数的平方$R^2$为$\seq{Y}{k}$对$X_j$的贡献率。
\end{definition}
\begin{property}\label{prop:PCA}
	总体主成分具有如下性质:
	\begin{enumerate}
		\item $\operatorname{Cov}(\mathbf{Y})=\operatorname{diag}\{\seq{\lambda}{n}\}$;
		\item $\mathbf{Y}$的方差之和等于$\mathbf{X}$的方差之和,即$\sum\limits_{i=1}^{n}\lambda_i=\sum\limits_{i=1}^{n}\sigma_{ii}$;
		\item $\rho(Y_i,X_j)=\dfrac{\sqrt{\lambda_i}\alpha_{ij}}{\sqrt{\sigma_{jj}}},\;i,j=1,2,\dots,n$;
		\item 第$i$个主成分与原变量的因子负荷量满足$\sum\limits_{j=1}^{n}\sigma_{jj}\rho^2(Y_i,X_j)=\lambda_i$;
		\item 原变量的第$j$个分量与所有主成分的因子负荷量满足$\sum\limits_{i=1}^{n}\rho^2(Y_i,X_j)=1$;
		\item 若$\Sigma>\mathbf{0}$,则原变量的第$j$个分量与前$k$个主成分之间的复相关系数满足:
		\begin{equation*}
			R^2=\sum_{i=1}^{k}\rho^2(Y_i,X_j)=\sum_{i=1}^{k}\frac{\lambda_i\alpha_{ij}^2}{\sigma_{ii}}
		\end{equation*}
	\end{enumerate}
\end{property}
\begin{proof}
	(1)由\cref{theo:PCA}直接可得。\par
	(2)由\cref{prop:CovMat}(3)和\cref{prop:Trace}(3)可得:
	\begin{align*}
		\sum_{i=1}^{n}\operatorname{Var}(Y_i)
		&=\operatorname{tr}[\operatorname{Cov}(\mathbf{Y})]=\operatorname{tr}[\operatorname{Cov}(A\mathbf{X})]=\operatorname{tr}(A\Sigma A^{\top}) \\
		&=\operatorname{tr}(\Sigma A^{\top}A)=\operatorname{tr}(\Sigma)=\sum_{i=1}^{n}\operatorname{Var}(X_i)
	\end{align*}\par
	(3)由\cref{prop:CovMat}(3)可得:
	\begin{align*}
		\rho(Y_i,X_j)
		&=\frac{\operatorname{Cov}(Y_i,X_j)}{\sqrt{\operatorname{Var}(Y_i)\operatorname{Var}(X_j)}}=\frac{\operatorname{Cov}(\alpha_i^{\top}\mathbf{X},e_j^{\top}\mathbf{X})}{\sqrt{\lambda_i\sigma_{jj}}} \\
		&=\frac{\alpha_i^{\top}\Sigma e_j}{\sqrt{\lambda_i\sigma_{jj}}}=\frac{e_j^{\top}\Sigma\alpha_i}{\sqrt{\lambda_i\sigma_{jj}}}=\frac{e_j^{\top}\lambda_i\alpha_i}{\sqrt{\lambda_i\sigma_{jj}}}=\frac{\sqrt{\lambda_i}\alpha_{ij}}{\sqrt{\sigma_{jj}}}
	\end{align*}\par
	(4)由(3)可得:
	\begin{equation*}
		\sum_{j=1}^{n}\sigma_{jj}\rho^2(Y_i,X_j)=\sum_{j=1}^{n}\lambda_i\alpha_{ij}^2=\lambda_i\alpha_i^{\top}\alpha_i=\lambda_i
	\end{equation*}\par
	(5)由\cref{prop:MultipleCorrelationCoefficient}(3)和\cref{prop:CorrelationCoefficient}(1)立即可得。\par
	(6)当$\Sigma>\mathbf{0}$时,由\cref{theo:PositiveDefinite}(3.5)可知$\Sigma$的特征值都大于$0$,根据\cref{theo:PCA}即$\operatorname{Var}(Y_i)>0$,于是由\cref{prop:MultipleCorrelationCoefficient}(1)和(3)可立即得出结论。
\end{proof}
\begin{definition}
	称第$i$个主成分$Y_i$的方差与所有主成分方差之和的比值为$Y_i$的方差贡献率,记为$\eta_i$,即$\eta_i=\lambda_i\Big/\sum\limits_{j=1}^{n}\lambda_j$。将$\sum\limits_{i=1}^{k}\lambda_i\Big/\sum\limits_{i=1}^{n}\lambda_i$称为主成分$\seq{Y}{k}$的累计方差贡献率。
\end{definition}
由前述,我们一般通过选择主成分的个数来实现对数据的降维,即选择主成分的个数使它们的累计方差贡献率达到一定比例(一般为$85\%$)。

\subsection{样本主成分分析}
假设对$n$维随机变量$\mathbf{X}$进行$m$次独立观测,得到$m$个$n$维样本$\seq{x}{m}$。在样本主成分分析中,我们使用样本来估计$\mathbf{X}$的协方差矩阵,即:
\begin{equation*}
	\hat{\boldsymbol{\mu}}=\frac{1}{m}\sum_{i=1}^{m}x_i,\;S=(s_{ij})=\frac{1}{m-1}\sum_{i=1}^{m}(x_i-\hat{\boldsymbol{\mu}})(x_i-\hat{\boldsymbol{\mu}})^{\top},\;i,j=1,2,\dots,n
\end{equation*}
其余步骤与总体主成分分析一致。

\subsection{注意事项}
\subsubsection{多重共线性问题}
当原始变量出现多重共线性时,重复的信息在方差占比中会重复进行计算,导致方差占比被改变。为了达到一定的累积贡献率,就会不可避免地使得主成分也融入重复信息(可以从一个极端案例来想:添加一个重复变量到原先一点都不重复的数据集中),所以主成分是无法解决重复信息问题的。主成分分析从另一个角度是依赖于多重共线性的:我们需要舍弃方差贡献率小的主成分。如果没有方差贡献率较小的主成分,这说明原数据几乎没有多重共线性情况,此时无论舍弃哪个主成分都将导致严重的信息损失。
\subsubsection{相关矩阵导出主成分}
上面我们都是对协方差矩阵的特征值分解进行计算,但在现实中,各变量的量纲可能不一样,这就导致各变量方差的量级会不一样,方差越大,由\cref{prop:PCA}(2)可知必然会导致该变量在主成分构成中的占比变大,但此时的大方差可能仅仅是因为量纲的不同,从变异系数的角度可能各变量的波动差距没有那么明显。为了消除量纲带来的影响,我们会对数据进行标准化,注意到标准化后数据的协方差矩阵即为相关矩阵,此时将相关矩阵作对应的特征值分解即可。但需要注意:标准化后\textbf{各变量方差相等均为$1$,损失了部分信息},所以会使得\textbf{标准化后的各变量在对主成分构成中的作用趋于相等}。因此,取值范围在同量级的数据建议使用协方差矩阵直接求解主成分,若变量之间数量级差异较大,再使用相关矩阵求解主成分。
\begin{algorithm}[H]
	\caption{主成分分析(PCA)}
	\begin{algorithmic}[1]
		\State \textbf{Input:} 原始数据矩阵 $\mathbf{X}$
		\State \textbf{Output:} 主成分得分 $\mathbf{Y}$,选定的主成分个数 $k$
		
		\If{变量量纲差异明显}
		\State 将$\mathbf{X}$标准化得到$\mathbf{Z}$
		\Else
		\State 保留原始数据 $\mathbf{Z} = \mathbf{X}$
		\EndIf
		
		\If{存在多重共线性(最小特征值 $\approx 0$)}
		\State 删除或合并相关变量
		\Else
		\State 保留原始变量
		\EndIf
		
		\State 计算协方差矩阵:$\Sigma = \text{cov}(\mathbf{Z})$,对 $\Sigma$ 进行特征值分解:$\Sigma = A \Lambda A^{\top}$
		\State 初始化 $k \gets 1$, 累计贡献率 $\gets 0$
	
		\Repeat
		\State 选择第 $k$ 个主成分,更新累计贡献率
		\If{累计贡献率 $< $ 阈值}
		\State $k \gets k + 1$
		\EndIf
		\Until{累计贡献率 $\geqslant$ 阈值}
		
		\State 计算主成分得分:$\mathbf{Y}=A_k\mathbf{Z}$($A_k$为矩阵$A$的前$k$行构成的矩阵),输出 $\mathbf{Y}$ 和 $k$
	\end{algorithmic}
\end{algorithm}

\subsection{应用:主成分回归}
\begin{note}
	我们将介绍PCA在解决线性模型复共线性问题上的应用,接下来将延续在岭估计处提到的线性回归模型的典则形式,并将$\operatorname{diag}\{\seq{\lambda}{p-1}\}$设置为$\lambda_1>\lambda_2>\cdots>\lambda_{p-1}$的形式。由\info{矩阵多项式的特征向量}可知$Y$实际上是经过PCA处理后的数据,当$X^{\top}X$存在很小的特征值时,不妨设它们为$\lambda_{r+1},\lambda_{r+2},\dots,\lambda_{p-1}$,由\cref{prop:PCA}(1)可知$Y_{r+1},Y_{r+2},\dots,Y_{p-1}$的方差很小,于是我们可以舍弃它们,从而得到如下模型:
	\begin{equation*}
		y=\alpha_0\mathbf{1}_n+\mathbf{Y}_1\delta_1+\varepsilon,\quad\operatorname{E}(\varepsilon)=\mathbf{0},\quad\operatorname{Cov}(\varepsilon)=\sigma^2I_n
	\end{equation*}
	其中$\mathbf{Y}_1$是$Y$的前$r$列,$\delta_1$是$\delta_I$的前$r$行。称如上模型给出的估计$\hat{\delta}_I=(\hat{\delta}_1,\mathbf{0})^{\top}$为$\hat{\delta}$的\gls{PrincipalComponentEstimate}。
\end{note}
\begin{property}
	中心化模型的主成分回归具有如下性质:
	\begin{enumerate}
		\item $\hat{\alpha}_0=\overline{y},\;\hat{\delta}_1=(\mathbf{Y}_1^{\top}\mathbf{Y}_1)^{-1}\mathbf{Y}_1^{\top}y=\operatorname{diag}\left\{\dfrac{1}{\lambda_{1}},\dfrac{1}{\lambda_{2}},\dots,\dfrac{1}{\lambda_{r}}\right\}\mathbf{Y}_1^{\top}y=\operatorname{diag}\left\{\dfrac{1}{\lambda_{1}},\dfrac{1}{\lambda_{2}},\dots,\dfrac{1}{\lambda_{r}}\right\}Q_1\tilde{X}_c^{\top}y,\hat{\alpha}_I=Q_1^{\top}\operatorname{diag}\left\{\dfrac{1}{\lambda_{1}},\dfrac{1}{\lambda_{2}},\dots,\dfrac{1}{\lambda_{r}}\right\}\mathbf{Y}_1^{\top}y=Q_1^{\top}\operatorname{diag}\left\{\dfrac{1}{\lambda_{1}},\dfrac{1}{\lambda_{2}},\dots,\dfrac{1}{\lambda_{r}}\right\}Q\tilde{X}_c^{\top}y$,其中$Q_1$是$Q$的前$r$行;
		\item 主成分估计一般是有偏估计;
		\item 当原中心化设计阵存在复共线性关系时,适当选择保留的主成分个数可以使得主成分估计比LSE具有更小的$\operatorname{MSE}$。
	\end{enumerate}
\end{property}
\begin{proof}
	(1)显然$\mathbf{Y}_1$也是中心化的,所以第一式成立。由\cref{prop:OrthogonalUnitaryMatrix}(1)和\cref{prop:InvertibleMatrix}(16)可知$Y$是列满秩的,所以由\cref{prop:LinearlyDependent}(1)可得$\mathbf{Y}_1$是列满秩的,根据\cref{theo:LinearRegressionModel}(1)可得$(\mathbf{Y}_1^{\top}\mathbf{Y}_1)^{-1}$存在。由正则方程即可得$\hat{\delta}_1=(\mathbf{Y}_1^{\top}\mathbf{Y}_1)^{-1}\mathbf{Y}_1^{\top}y$。根据\cref{prop:Transpose}(4)可知:
	\begin{equation*}
		\mathbf{Y}_1^{\top}\mathbf{Y}_1=Q_1\tilde{X}_c^{\top}\tilde{X}_cQ_1^{\top}=\operatorname{diag}\{\seq{\lambda}{r}\}
	\end{equation*}
	其中$Q_1$是$Q$的前$r$行,于是有:
	\begin{equation*}
		\hat{\delta}_1=(\mathbf{Y}_1^{\top}\mathbf{Y}_1)^{-1}\mathbf{Y}_1^{\top}y=\operatorname{diag}\left\{\dfrac{1}{\lambda_{1}},\dfrac{1}{\lambda_{2}},\dots,\dfrac{1}{\lambda_{r}}\right\}\mathbf{Y}_1^{\top}y=\operatorname{diag}\left\{\dfrac{1}{\lambda_{1}},\dfrac{1}{\lambda_{2}},\dots,\dfrac{1}{\lambda_{r}}\right\}Q_1\tilde{X}_c^{\top}y
	\end{equation*}
	由\cref{prop:Transpose}(4)、\cref{prop:InvertibleMatrix}(11)和\cref{theo:LinearRegressionModel}(3)可得:
	\begin{equation*}
		\hat{\alpha}_I=(\tilde{X}_c^{\top}\tilde{X}_c)^{-1}\tilde{X}_c^{\top}y=Q^{\top}\operatorname{diag}\left\{\dfrac{1}{\lambda_{1}},\dfrac{1}{\lambda_{2}},\dots,\dfrac{1}{\lambda_{p-1}}\right\}Q\tilde{X}_c^{\top}y
	\end{equation*}
\end{proof}






