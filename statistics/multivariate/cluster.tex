\section{聚类分析}

\gls{ClusteringAnalysis}是一类在\textbf{不依赖任何先验标签信息}的前提下,根据样本之间的相似性或距离关系,对数据进行划分的方法。形式上,聚类算法为每个样本分配一个簇标签,但从方法论角度看,\textbf{聚类分析并不以标签本身为研究终点},而是试图通过这些标签揭示数据中潜在的结构性特征。

更准确地说,聚类分析的本质是一种\textbf{结构发现(structure discovery)}方法。它旨在回答如下问题:数据是否可以被视为由若干相对同质的子群体构成?这些子群体在统计特征、生成机制或行为模式上是否存在系统性差异?因此,聚类标签应被理解为对数据结构的一种\textbf{抽象描述},而非具有直接语义的分类结果。

在统计学习与数据分析框架下,聚类的目的可以从以下几个相互关联的层面来理解。

\begin{enumerate}
	\item \textbf{结构发现与探索性分析:}通过样本间相似性关系揭示数据中潜在的分组结构与异质性特征,例如多模态、分层或局部聚集等现象。从方法论上看,聚类属于典型的探索性数据分析手段,其结果主要用于辅助理解数据结构、生成研究假设并为后续建模提供依据,而非作为最终分析结论;
	\item \textbf{数据抽象与降低复杂度:}在高维或大规模数据情形下,直接以单个样本为分析对象往往会带来计算和解释上的困难。聚类通过将大量样本映射为有限个簇,实现了从“个体层面”到“群体层面”的抽象。这种抽象使得研究者可以以簇为基本单元,对数据进行更高层次的描述、比较与总结。例如,可以利用簇的中心、分布特征或代表性样本来刻画原始数据,从而提高分析的可解释性与可操作性;
	\item \textbf{异质性刻画与分群建模:}在许多应用场景中,整体数据往往并非来自单一同质总体,而是由多个具有不同统计特性或生成机制的子群体混合而成。如果忽略这种异质性而直接建立统一的全局模型,往往会导致模型拟合能力下降,甚至产生误导性的结论。此时聚类分析可以帮助识别潜在的异质子群体,并为后续的分群建模提供依据。例如,可以在不同簇内分别建立回归模型、风险模型或预测模型,比较模型参数在不同簇之间的差异,从而揭示不同子群体可能遵循的不同机制;
\end{enumerate}

\subsection{评价指标}
聚类评价指标的核心目标是量化:
\begin{itemize}
	\item \textbf{簇内紧凑性(compactness)}:同一簇里的点是否彼此接近;
	\item \textbf{簇间分离性(separation)}:不同簇之间是否彼此远离;
	\item \textbf{与外部标签一致性(agreement with ground truth)}:当存在真实标签时,聚类是否复现了真实分组。
\end{itemize}
具体而言,聚类评价指标可以分为下述几类:
\begin{enumerate}
	\item \textbf{内部指标(Internal Validation)}:只依赖数据与聚类标签(不需要真实标签);
	\item \textbf{外部指标(External Validation)}:需要真实标签。
\end{enumerate}\par
需要强调的是,外部评价指标的引入并不意味着聚类任务转化为监督学习问题。外部指标的核心作用在于:当研究者在特定观测样本集中事先掌握某种参考划分时,用以分析既有标签与数据内在结构之间的一致性关系。从而对聚类算法、距离度量或参数设置进行事后评估与比较。因此,外部指标并不参与聚类过程本身,也不作为聚类目标函数进行优化,而是作为一种方法学意义上的对照标准,为研究者理解聚类结果与现实语义之间的关系提供了一种量化参照,帮助研究者理解数据自身在多大程度上支持、修正或挑战已有的分类体系。\par
当外部一致性较高时,说明在当前特征表示与距离度量下,数据的几何或统计结构能够较好地支持既定的类别划分;例如,在文本主题分析或基因表达数据中,若聚类结果与人工标注的主题类别或生物亚型高度一致,则可认为该标签在数据空间中具有较清晰的结构基础。相反,当外部一致性较低时,这并不必然意味着聚类算法失效,而可能揭示出标签本身的局限性:例如,在心理测量或社会科学调查中,理论上区分的若干类型可能在实际数据中呈现出连续分布或显著重叠;又如,在医学诊断中,同一疾病标签下的样本可能包含多个具有不同分子机制的亚群。此时,聚类结果与外部标签之间的不一致,反而为研究者提供了重新审视标签定义、发现潜在异质结构或子群划分的线索。\par
设观测样本集$\mathcal{D}=\{x_i\}_{i=1}^n,\ x_i\in\mathbb{R}^m$,聚类输出标签$y_i\in\{1,\dots,K\}$(或包含噪声$-1$),第$k$簇为$C_k=\{i:y_i=k\}$,簇大小$n_k=|C_k|$,簇中心和总体均值定义为:
\begin{equation*}
	\mu_k=\frac{1}{n_k}\sum_{i\in C_k}x_i,\quad
	\bar{x}=\frac{1}{n}\sum_{i=1}^n x_i
\end{equation*}
\subsubsection{内部指标}
\begin{definition}
	称$\operatorname{WCSS}=\sum\limits_{k=1}^{K}\sum\limits_{i\in C_k}\|x_i-\mu_k\|^2$为\gls{WCSS}。
\end{definition}
\begin{property}[最优WCSS关于簇数的单调性]\label{prop:WCSSMonotonicity}
	定义:
	\begin{equation*}
		\operatorname{WCSS}^\star(K)
		\coloneq\min_{\{C_1,\dots,C_K\}}
		\;\sum_{k=1}^K \min_{\mu_k\in\mathbb{R}^m}\sum_{i\in C_k}\|x_i-\mu_k\|^2
	\end{equation*}
	则对任意$K\geqslant1$有$\operatorname{WCSS}^\star(K+1)\leqslant \operatorname{WCSS}^\star(K)$。
\end{property}
\begin{proof}
	取任意一个划分$C=\{C_1,\dots,C_K\}$和一个非空簇(如 $C_1\ne\varnothing$),将其拆成两个非空子集$A,B$:
	\begin{equation*}
		C_1=A\cup B,\;A\ne\varnothing,\;B\ne\varnothing
	\end{equation*}
	并令新划分为$C'=\{A,B,C_2,\dots,C_K\}$。对新划分的目标值有:
	\begin{equation*}
		\Phi(C')
		= \min_{\alpha}\sum_{i\in A}\|x_i-\alpha\|^2
		+\min_{\beta}\sum_{i\in B}\|x_i-\beta\|^2
		+\sum_{k=2}^K \min_{\mu_k\in\mathbb{R}^{m}}\sum_{i\in C_k}\|x_i-\mu_k\|^2
	\end{equation*}
	对任意给定向量$\gamma\in\mathbb{R}^m$有:
	\begin{equation*}
		\min_{\alpha}\sum_{i\in A}\|x_i-\alpha\|^2 \leqslant\sum_{i\in A}\|x_i-\gamma\|^2,\quad
		\min_{\beta}\sum_{i\in B}\|x_i-\beta\|^2 \leqslant\sum_{i\in B}\|x_i-\gamma\|^2
	\end{equation*}
	取$\gamma$为原簇 $C_1$ 的最优中心,记为 $\mu_1^\star$,得到:
	\begin{equation*}
		\min_{\alpha}\sum_{i\in A}\|x_i-\alpha\|^2
		+\min_{\beta}\sum_{i\in B}\|x_i-\beta\|^2
		\leqslant
		\sum_{i\in A}\|x_i-\mu_1^\star\|^2 + \sum_{i\in B}\|x_i-\mu_1^\star\|^2
		=
		\sum_{i\in C_1}\|x_i-\mu_1^\star\|^2
	\end{equation*}
	因此:
	\begin{equation*}
		\Phi(C')\leqslant
		\sum_{i\in C_1}\|x_i-\mu_1^\star\|^2
		+\sum_{k=2}^K \min_{\mu_k\in\mathbb{R}^{m}}\sum_{i\in C_k}\|x_i-\mu_k\|^2
		= \Phi(\mathcal{C})
	\end{equation*}
	由最小值的不等式性可得:
	\begin{equation*}
		\operatorname{WCSS}^\star(K+1)\leqslant\operatorname{WCSS}^\star(K)\qedhere
	\end{equation*}
\end{proof}
\begin{note}
	$\operatorname{WCSS}$是以欧氏距离为基础的簇内紧凑性度量,其数值大小直接反映了样本在各自簇内围绕质心的离散程度。一般而言,$\operatorname{WCSS}$越小,表示簇内样本越集中、聚类结果越紧凑,因而在同一簇数$K$下聚类质量越好。需要注意的是:
	\begin{enumerate}
		\item 对固定观测样本集和以$\operatorname{WCSS}$为目标函数的算法,由\cref{prop:WCSSMonotonicity}可知$\operatorname{WCSS}$随簇数$K$的增大单调下降;极端情况下,当$K=n$时,每个样本单独成簇,$\operatorname{WCSS}=0$,因此$\operatorname{WCSS}$本身不能直接用于比较不同$K$下的聚类优劣。尽管$\operatorname{WCSS}$随$K$单调下降,但其下降幅度通常在某一$K$值之后显著减缓。通过考察$\operatorname{WCSS}$随$K$变化的曲线,并选取曲线由“陡降”转为“平缓”的拐点,可以在模型复杂度与聚类紧凑性之间取得平衡,从而确定一个合理的簇数,这一方法被称之为\gls{ElbowMethod};
		\item 由于$\operatorname{WCSS}$基于质心与距离定义,其隐含假设簇内样本呈近似球形或凸形分布,对非凸结构或密度差异较大的簇刻画能力有限;
		\item $\operatorname{WCSS}$对特征量纲高度敏感,不同尺度的特征会不均衡地影响距离计算,因此在实际应用中通常需要对特征进行标准化或归一化处理。
	\end{enumerate}
\end{note}
\begin{definition}
	称:
	\begin{equation*}
		\operatorname{CH}=\frac{\operatorname{tr}(B)/(K-1)}{\operatorname{tr}(W)/(n-K)}
	\end{equation*}
	为\gls{CH},其中:
	\begin{equation*}
		W=\sum_{k=1}^K\sum_{i\in C_k}(x_i-\mu_k)(x_i-\mu_k)^\top,\quad
		B=\sum_{k=1}^K n_k(\mu_k-\bar{x})(\mu_k-\bar{x})^\top
	\end{equation*}
\end{definition}
\begin{note}
	$\operatorname{CH}$指数本质上与单因素方差分析中的$F$统计量高度一致。在几何意义上,当簇中心彼此远离且簇内样本高度集中时,$\operatorname{tr}(B)$ 相对较大而 $\operatorname{tr}(W)$ 相对较小,从而$\operatorname{CH}$取得较大值,表明聚类结构清晰。
\end{note}
\begin{definition}
	对样本$x_i$,定义:
	\begin{gather*}
		a_i\coloneq\frac{1}{|C_{y_i}|-1}\sum_{\substack{j\in C_{y_i}\\ j\neq i}} d(x_i,x_j),\quad b_i\coloneq\min_{\ell\neq y_i}\frac{1}{|C_{\ell}|}\sum_{j\in C_{\ell}} d(x_i,x_j) \\
		s_i\coloneq
		\begin{cases}
			\dfrac{b_i-a_i}{\max\{a_i,b_i\}},&\max\{a_i,b_i\}\ne0 \\
			0,&\max\{a_i,b_i\}=0
		\end{cases},\quad\operatorname{Silhouette}\coloneq\frac{1}{n}\sum_{i=1}^ns_i
	\end{gather*}
\end{definition}
\begin{property}[轮廓系数的取值范围]\label{prop:Silhouette}
	$\operatorname{Silhouette}$轮廓系数的取值范围为$[-1,1]$。
\end{property}
\begin{proof}
	若$\max\{a_i,b_i\}=0$,则$a_i=b_i=0$,此时$s_i=0$。下面假设$\max\{a_i,b_i\}>0$。分两种情形讨论:
	\begin{enumerate}
		\item 若$b_i\geqslant a_i$,则$\max\{a_i,b_i\}=b_i$,从而:
		\begin{equation*}
			s_i=\frac{b_i-a_i}{b_i}=1-\frac{a_i}{b_i}
		\end{equation*}
		由于$0\leqslant a_i\leqslant b_i$,可得$0\leqslant \dfrac{a_i}{b_i}\leqslant 1$,因此$0\leqslant s_i\leqslant 1$。		
		\item 若$a_i>b_i$,则$\max\{a_i,b_i\}=a_i$,从而:
		\begin{equation*}
			s_i=\frac{b_i-a_i}{a_i}=\frac{b_i}{a_i}-1
		\end{equation*}
		由于$0\leqslant b_i<a_i$,可得$0\leqslant\dfrac{b_i}{a_i}<1$,因此$-1\leqslant s_i<0.$
	\end{enumerate}
	综上,$s_i\in[-1,1]$,即$\operatorname{Silhouette}\in[-1,1]$。
\end{proof}
\begin{note}[Silhouette 的几何解释]
	轮廓系数从几何角度刻画了单个样本在其所属簇中的相对位置。其中,$a_i$衡量样本$x_i$到其所在簇$C_k$内部的平均距离,反映簇内的局部紧凑性;而$b_i$刻画样本$x_i$到最近其他簇的平均距离,反映其与邻近簇之间的分离程度。由定义可以看出,$s_i$本质上是簇内距离与最近簇间距离之间的\emph{相对差异}。当$a_i\ll b_i$时,样本$x_i$位于簇的内部区域,与其他簇之间存在明显间隔,此时$s_i$接近$1$;当$a_i\approx b_i$时,样本$x_i$位于簇的边界附近,对不同簇的归属不具有明显偏好,此时$s_i$接近$0$;当$a_i>b_i$时,样本$x_i$在几何意义上更接近其他簇而非当前簇,通常意味着该样本可能被错误分配,此时$s_i<0$。\par	
	因此,Silhouette 系数可被视为一种基于距离比值的局部几何判别准则,它同时综合了簇内紧凑性与簇间分离性,并能够在样本层面反映聚类结构的合理性。
\end{note}
\subsubsection{外部指标}
\begin{definition}
	考虑所有样本对$(i,j)$,定义:
	\begin{itemize}
		\item $a$:在真实标签同类,且聚类也同簇的样本对个数;
		\item $b$:在真实标签不同类,且聚类也不同簇的样本对个数;
	\end{itemize}
	称:
	\begin{equation*}
		\operatorname{RI}=\frac{a+b}{\binom{n}{2}}
	\end{equation*}
	为\gls{RI},称:
	\begin{equation*}
		\operatorname{ARI}=\frac{\operatorname{RI}-\operatorname{E}(\operatorname{RI})}{1-\operatorname{E}(\operatorname{RI})}
	\end{equation*}
	为\gls{ARI}。
\end{definition}
\begin{theorem}
	在固定边际(列联表个数统计值)的情况下,设:
	\begin{table}[H]
		\centering
		\caption{真实类别与聚类簇的列联表}
		\begin{tabular}{c|cccc|c}
			\hline
			& $L_1$ & $L_2$ & $\cdots$ & $L_r$ & 合计 \\ \hline
			$C_1$ & $N_{11}$ & $N_{12}$ & $\cdots$ & $N_{1r}$ & $n_1$ \\
			$C_2$ & $N_{21}$ & $N_{22}$ & $\cdots$ & $N_{2r}$ & $n_2$ \\
			$\vdots$ & $\vdots$ & $\vdots$ & $\ddots$ & $\vdots$ & $\vdots$ \\
			$C_K$ & $N_{K1}$ & $N_{K2}$ & $\cdots$ & $N_{Kr}$ & $n_K$ \\ \hline
			合计 & $m_1$ & $m_2$ & $\cdots$ & $m_r$ & $n$ \\ \hline
		\end{tabular}
	\end{table}
	其中$C_i$表示第$i$个聚类簇,$L_j$表示第$j$个真实类别,$N_{ij}=|C_i\cap L_j|$为同时属于簇$C_i$且属于类别$L_j$的样本个数,并满足行边际与列边际约束:
	\begin{equation*}
		\sum_{j=1}^rN_{ij}=n_i,\quad
		\sum_{i=1}^KN_{ij}=m_j,\quad
		\sum_{i=1}^K\sum_{j=1}^rN_{ij}=n
	\end{equation*}
	则:
	\begin{equation*}
		\operatorname{E}(\operatorname{RI})=1+\frac{2\sum\limits_{i=1}^{K}\binom{n_i}{2}\sum\limits_{j=1}^{r}\binom{m_j}{2}}{[\binom{n}{2}]^2}-\frac{\sum\limits_{i=1}^{K}\binom{n_i}{2}+\sum\limits_{j=1}^{r}\binom{m_j}{2}}{\binom{n}{2}}
	\end{equation*}
\end{theorem}
\begin{proof}
	由\cref{prop:MeasurableIntegral}(5)可知:
	\begin{equation*}
		\operatorname{E}(\operatorname{RI})=\frac{\operatorname{E}(a)+\operatorname{E}(b)}{\binom{n}{2}}
	\end{equation*}\par
	对于$\operatorname{E}(a)$,根据\cref{prop:MeasurableIntegral}(5)可得\info{写非参数列联表的时候把这边综合起来}:
	\begin{equation*}
		\operatorname{E}(a)=\operatorname{E}\left[\sum_{i=1}^{K}\sum_{j=1}^{r}\binom{N_{ij}}{2}\right]=\sum_{i=1}^{K}\sum_{j=1}^{r}\frac{1}{2}\operatorname{E}[N_{ij}(N_{ij}-1)]=\sum_{i=1}^{K}\sum_{j=1}^{r}\frac{1}{2}[\operatorname{E}(N_{ij}^2)-\operatorname{E}(N_{ij})]
	\end{equation*}
	而:
	\begin{equation*}
		N_{ij}\sim\operatorname{Hyper}(n_i,m_j,n),\quad\operatorname{E}(N_{ij})=\frac{n_im_j}{n},\quad\operatorname{Var}(N_{ij})=\frac{n_im_j(n-n_i)(n-m_j)}{n^2(n-1)}
	\end{equation*}
	由\cref{prop:Variance}(1)可知:
	\begin{align*}
		&\operatorname{E}(N_{ij}^2)=\operatorname{Var}(N_{ij})+[\operatorname{E}(N_{ij})]^2=\frac{n_im_j(n-n_i)(n-m_j)}{n^2(n-1)}+\frac{n_i^2m_j^2}{n^2} \\
		=&\frac{n_im_j(n-n_i)(n-m_j)+n_i^2m_j^2(n-1)}{n^2(n-1)}=\frac{n_im_j[(n-n_i)(n-m_j)+n_im_j(n-1)]}{n^2(n-1)} \\
		=&\frac{n_im_j(n^2-m_jn-n_in+n_im_j+n_im_jn-n_im_j)}{n^2(n-1)}=\frac{n_im_j(n^2-m_jn-n_in+n_im_jn)}{n^2(n-1)} \\
		=&\frac{n_im_j(n-m_j-n_i+n_im_j)}{n(n-1)} 
	\end{align*}
	于是:
	\begin{align*}
		&\operatorname{E}(N_{ij}^2)-\operatorname{E}(N_{ij})=\frac{n_im_j(n-m_j-n_i+n_im_j)}{n(n-1)}-\frac{n_im_j}{n} \\
		=&\frac{n_im_j(n-m_j-n_i+n_im_j)-(n-1)n_im_j}{n(n-1)}=\frac{n_im_j(1-m_j-n_i+n_im_j)}{n(n-1)} \\
		=&\frac{n_i(n_i-1)m_j(m_j-1)}{n(n-1)}
	\end{align*}
	所以:
	\begin{equation*}
		\operatorname{E}(a)=\sum_{i=1}^{K}\sum_{j=1}^{r}\frac{1}{2}\frac{n_i(n_i-1)m_j(m_j-1)}{n(n-1)}=\sum_{i=1}^{K}\sum_{j=1}^{r}\frac{\binom{n_i}{2}\binom{m_j}{2}}{\binom{n}{2}}=\frac{\sum\limits_{i=1}^{K}\binom{n_i}{2}\sum\limits_{j=1}^{r}\binom{m_j}{2}}{\binom{n}{2}}
	\end{equation*}\par
	对于$\operatorname{E}(b)$,注意到同簇样本个数$A$与同类样本个数$B$分别为:
	\begin{equation*}
		A=\sum_{i=1}^{K}\binom{n_i}{2},\quad B=\sum_{j=1}^{r}\binom{m_j}{2}
	\end{equation*}
	于是:
	\begin{equation*}
		b=\binom{n}{2}-a-(B-a)-(A-a)=\binom{n}{2}-A-B+a
	\end{equation*}
	根据\cref{prop:MeasurableIntegral}(5)可得:
	\begin{equation*}
		\operatorname{E}(b)=\binom{n}{2}-A-B+\operatorname{E}(a)
	\end{equation*}
	所以:
	\begin{align*}
		&\operatorname{E}(\operatorname{RI})=\frac{2\operatorname{E}(a)+\binom{n}{2}-A-B}{\binom{n}{2}}=\frac{2\operatorname{E}(a)}{\binom{n}{2}}+1-\frac{A+B}{\binom{n}{2}} \\
		=&1+\frac{2\sum\limits_{i=1}^{K}\binom{n_i}{2}\sum\limits_{j=1}^{r}\binom{m_j}{2}}{[\binom{n}{2}]^2}-\frac{\sum\limits_{i=1}^{K}\binom{n_i}{2}+\sum\limits_{j=1}^{r}\binom{m_j}{2}}{\binom{n}{2}}\qedhere
	\end{align*}
\end{proof}
\begin{property}
	$\operatorname{RI}\in[0,1],\;\operatorname{ARI}\leqslant1$。
\end{property}
\begin{proof}
	$\operatorname{RI}\in[0,1]$是显然的。\par
	若真实划分与聚类划分不是几乎必然完全一致,则随机生成的聚类结果不可能以概率$1$与真实标签完全一致,因此$\operatorname{E}(\operatorname{RI})<1$。\par
	由$\operatorname{RI}\leqslant1$可知:
	\begin{equation*}
		\operatorname{RI}-\operatorname{E}(\operatorname{RI})\leqslant1-\operatorname{E}(\operatorname{RI})
	\end{equation*}
	由于$1-\operatorname{E}(\operatorname{RI})>0$,两边同除以$1-\operatorname{E}(\operatorname{RI})$可得:
	\begin{equation*}
		\operatorname{ARI}
		=\frac{\operatorname{RI}-\operatorname{E}(\operatorname{RI})}{1-\operatorname{E}(\operatorname{RI})}\leqslant1\qedhere
	\end{equation*}
\end{proof}
\begin{note}
	$\operatorname{RI}$基于样本对一致性来评价聚类结果,$\operatorname{RI}$ 越大表示聚类划分与真实标签的一致性越强;若 $\operatorname{RI}$ 接近 $1$,说明聚类结果与真实划分高度一致;若接近 $0$,说明一致性极弱。\par
	为了理解$\operatorname{RI}$的统计行为,需要引入\textbf{随机聚类模型}:在该模型中,真实类别个数与各簇规模均被视为固定,而聚类标签在所有满足给定簇规模约束的分配方式中等概率随机生成。该模型刻画了“聚类结果与真实标签在统计意义上相互独立”的基准情形。\par
	在这一随机模型下,$\operatorname{RI}$往往具有非零且相对较大的期望值。其根本原因在于样本对中真实不同类的情形通常占据主导地位,而这类样本对在随机聚类中被分配到不同簇的概率本身较高。例如,设$n=100$,真实标签包含$10$个类别、每类$10$个样本,则共有$\binom{100}{2}=4950$个样本对,其中真实不同类的样本对数为$4500$。即使聚类结果完全随机,只要簇规模与真实类别规模相近,大多数真实异类样本对仍会落入不同簇,从而使得 $\operatorname{RI}$取得看似较好的数值。这表明$\operatorname{RI}$对随机一致性并不敏感。\par
	$\operatorname{ARI}$正是基于上述随机模型对$\operatorname{RI}$进行期望校正,该校正保证了随机聚类的期望$\operatorname{ARI}$为$0$,完美一致时取值为$1$,并允许在一致性劣于随机时取负值。因此,$\operatorname{ARI}$相较于$\operatorname{RI}$提供了一个以随机一致性为基准的、更为稳健的外部聚类评价指标。
\end{note}
\begin{definition}
	考虑所有样本对$(i,j)$,定义:
	\begin{itemize}
		\item $TP$:真实标签同类,且聚类结果同簇的样本对数;
		\item $FP$:真实标签不同类,但聚类结果同簇的样本对数;
		\item $FN$:真实标签同类,但聚类结果不同簇的样本对数。
	\end{itemize}
	称:
	\begin{equation*}
		\operatorname{FMI}
		=\sqrt{
			\frac{TP}{TP+FP}
			\cdot
			\frac{TP}{TP+FN}
		}
	\end{equation*}
	为\gls{FMI}。
\end{definition}
\begin{property}\label{prop:FMI}
	$\operatorname{FMI}\in[0,1]$。
\end{property}
\begin{proof}
	由定义立即可得。
\end{proof}
\begin{note}
	Fowlkes-Mallows指数从样本对一致性的角度同时兼顾了聚类结果的精确性与完整性。在真实标签给定的条件下,$\operatorname{FMI}$越大表示聚类结果与真实划分在样本对层面上的一致性越强;当聚类结果与真实标签完全一致时,$\operatorname{FMI}=1$;当不存在任何同类同簇的样本对时,$\operatorname{FMI}=0$。与$\operatorname{RI}$将异类异簇样本对也视为正确判断不同,$\operatorname{FMI}$仅关注真实同类样本是否被成功聚合,从而在类别数量较多或类别规模不平衡的情形下,往往比 RI 更为保守。
\end{note}

\subsection{K-means}
\begin{algorithm}[H]
	\caption{K-means Clustering (Lloyd's Algorithm)}
	\label{alg:Kmeans}
	\begin{algorithmic}[1]
		\Require
		Data points $X=\{x_n\}_{i=1}^{n}\subseteq\mathbb{R}^m$;
		number of clusters $K$;
		maximum number of iterations $T$;
		tolerance $\varepsilon$
		\Ensure
		Cluster assignments $\{C_1,\dots,C_K\}$;
		centroids $\{\mu_1,\dots,\mu_K\}$
		
		\State Select the first centroid $\mu_1$ uniformly at random from $X$
		\Comment{Initialization step}
		\For{$k=2$ to $K$}
		\For{each data point $x_i \in \mathcal{X}$}
		\State $D_i^2 \gets \min\limits_{1\leqslant j < k} \|x_i - \mu_j\|^2$
		\EndFor
		\State Select $\mu_k$ from $X$ with probability
		\[
		P(x_i) = \frac{D_i^2}{\sum\limits_{j=1}^n D_j^2}
		\]
		\EndFor
		
		\For{$t=1$ to $T$}
		\Comment{Assignment step}
		\For{each data point $x_i\in X$}
		\State $y_i \gets \underset{k=1,2,\dots,K}{\arg\min}\|x_i-\mu_k\|^2$, assign $x_i$ to cluster $C_{y_i}$
		\EndFor
		
		\For{$k=1$ to $K$}
		\Comment{Update step}
		\State $\mu_k^{\text{new}} \gets \dfrac{1}{|C_k|}
		\sum\limits_{x_i\in C_k} x_i$
		\EndFor
		
		\If{$\max\limits_{k} \|\mu_k^{\text{new}}-\mu_k\| < \varepsilon$}
		\State \textbf{break}
		\EndIf
		
		\State $\mu_k \gets \mu_k^{\text{new}},\;\forall\;k=1,\dots,K$
		\EndFor
	\end{algorithmic}
\end{algorithm}
\cref{alg:Kmeans}给出了经典的K-means聚类算法(亦称Lloyd算法)的完整流程。该算法的目标是在给定样本集合的情况下,将样本划分为$K$个互不相交的簇,使得簇内样本在欧氏距离意义下尽可能紧密。从优化角度看,K-means试图最小化如下的类内平方和目标函数:
\begin{equation*}
	\Phi(\{\mu_k\},\{C_k\})=\sum_{k=1}^K\sum_{x_i\in C_k}\|x_i-\mu_k\|^2
\end{equation*}\par
由于每一次分配或更新步骤都会使目标函数$\Phi$的值非增\info{平方分解均值最小,下面两个算法也需要链接},且$\Phi$在下界零处有界,根据\cref{prop:RSeq}(7)可知K-means算法在有限步内必然收敛到某一局部最优解。然而,由于目标函数是非凸的\info{证明},算法的最终结果依赖于初始中心的选取,这也是在实际应用中通常采用k-means++初始化(即\cref{alg:Kmeans}中的1-7步)而非完全随机初始化的原因。
\subsubsection{K-modes}
\begin{algorithm}[H]
	\caption{K-modes Clustering Algorithm with Cao Initialization}
	\label{alg:Kmodes}
	\begin{algorithmic}[1]
		\Require
		Categorical data $X=\{x_i\}_{i=1}^n$, where $x_i=(x_{i1},\dots,x_{ip})$;
		number of clusters $K$;
		maximum number of iterations $T$
		\Ensure
		Cluster assignments $\{C_1,\dots,C_K\}$;
		modes $\{m_1,\dots,m_K\}$
		
		\State For each categorical variable $j=1,\dots,p$, let $\mathcal C_j$ denote its set of possible categories.
		For each $c\in\mathcal C_j$, compute
		\[
		f_j(c)
		=
		\frac{1}{n}
		\sum_{i=1}^n I(x_{ij}=c)
		\]
		\State Define sample density: $\mathrm{dens}(x_i)=\sum\limits_{j=1}^p f_j(x_{ij})$
		\State Select the first mode: $m_1 \gets \underset{x_i\in X}{\arg\max}\;\mathrm{dens}(x_i)$
		\For{$k=2$ to $K$}
		\State Select
		\[
		m_k \gets \underset{x_i\in X}{\arg\max}
		\Bigl(
		\mathrm{dens}(x_i)\cdot
		\min_{1\leqslant r<k}\sum_{j=1}^p I(x_{ij}\neq m_{rj})
		\Bigr)
		\]
		\EndFor
		
		\For{$t=1$ to $T$}
		\Comment{Assignment step}
		\For{each data point $x_i\in X$}
		\State $y_i \gets \underset{k=1,2,\dots,K}{\arg\min}
		\sum\limits_{j=1}^p I(x_{ij}\neq m_{kj})$, assign $x_i$ to cluster $C_{y_i}$
		\EndFor
		
		\For{$k=1$ to $K$}
		\Comment{Update step}
		\For{$j=1$ to $p$}
		\State $m_{kj} \gets \operatorname{mode}\{x_{ij}:x_i\in C_k\}$
		\EndFor
		\EndFor
		
		\If{cluster assignments do not change}
		\State \textbf{break}
		\EndIf
		\EndFor
	\end{algorithmic}
\end{algorithm}
\cref{alg:Kmodes} 给出了K-modes聚类算法的标准形式。该算法专门用于处理分类型数据,其目标是在简单匹配不相似度下最小化类内不一致性。对应的优化目标函数为:
\begin{equation*}
	\Psi({m_k},{C_k})=
	\sum_{k=1}^K\sum_{x_i\in C_k}\sum_{j=1}^p I(x_{ij}\neq m_{kj})
\end{equation*}
该目标函数在簇分配或逐维众数更新时均单调不增,因此算法在有限步内收敛到某一局部最优解。与K-means类似,由于目标函数的非凸性,K-modes的结果依赖于初始化方式,实际应用中通常采用Cao初始化(即\cref{alg:Kmodes}中的1-6步)。\par
Cao初始化的核心思想在于:初始聚类原型应当同时具备代表性和区分性。具体而言,算法首先通过各分类变量的经验频率刻画样本在全体数据中的密度,从而选取在每一维上均取高频类别的样本作为第一个初始中心,该样本可被视为总体经验分布下的一个典型点。随后,在选取其余初始中心时,Cao初始化在保证中心密度较高的同时,引入与已有中心之间的简单匹配不相似度,以避免多个初始原型集中于同一高频区域。
\subsubsection{K-prototypes}
\begin{algorithm}[H]
	\caption{K-prototypes Clustering Algorithm}
	\label{alg:Kprototypes}
	\begin{algorithmic}[1]
		\Require
		Mixed-type data $X=\{x_i=(x_i^{(n)},x_i^{(c)})\}_{i=1}^n$;
		number of clusters $K$;
		trade-off parameter $\gamma>0$;
		maximum number of iterations $T$
		\Ensure
		Cluster assignments $\{C_1,\dots,C_K\}$;
		prototypes $\{(\mu_k,m_k)\}_{k=1}^K$
		
		\Comment{Initialization step}
		\State Initialize numerical prototypes $\mu_1,\dots,\mu_K$
		by applying K-means++ initialization
		
		\State Initialize categorical prototypes $m_1,\dots,m_K$
		by applying Cao initialization
		
		\For{$t=1$ to $T$}
		\Comment{Assignment step}
		\For{each data point $x_i\in X$}
		\State $y_i \gets \underset{k=1,2,\dots,K}{\arg\min}
		\Big(
		\|x_i^{(n)}-\mu_k\|^2
		+
		\gamma\sum\limits_{j=1}^q I(x_{ij}^{(c)}\neq m_{kj})
		\Big)$, assign $x_i$ to cluster $C_{y_i}$
		\EndFor
		
		\For{$k=1$ to $K$}
		\Comment{Update step}
		\State $\mu_k \gets \dfrac{1}{|C_k|}
		\sum\limits_{x_i\in C_k} x_i^{(n)}$
		\For{$j=1$ to $\dim(x_i^{(c)})$}
		\State $m_{kj} \gets
		\operatorname{mode}\{x_{ij}^{(c)}:x_i\in C_k\}$
		\EndFor
		\EndFor
		
		\If{cluster assignments do not change}
		\State \textbf{break}
		\EndIf
		\EndFor
	\end{algorithmic}
\end{algorithm}

\cref{alg:Kprototypes} 描述了K-prototypes聚类算法的完整流程。该算法同时处理数值型与分类型变量,其优化目标为:
\begin{equation*}
	\Omega({\mu_k,m_k},{C_k})=
	\sum_{k=1}^K\sum_{x_i\in C_k}
	\Big(\|x_i^{(n)}-\mu_k\|^2 + \gamma\sum_{j=1}^{\dim(x_i^{(c)})} I(x_{ij}^{(c)}\neq m_{kj})\Big)
\end{equation*}
其中参数$\gamma$用于平衡数值误差与分类型不匹配误差的相对尺度,实践中,通常在对数值型变量进行标准化处理后,根据数值型与分类型变量的维数比例设置$\gamma\approx p/q$,其中$p$与$q$分别表示数值型与分类型变量的维数。与 K-means和K-modes 类似,K-prototypes通过交替最小化策略保证目标函数单调下降,但最终仅能保证收敛到局部最优解。


\subsection{凝聚型层次聚类}
\begin{algorithm}[H]
	\caption{Agglomerative Hierarchical Clustering}
	\label{alg:AgglomerativeHierarchicalClustering}
	\begin{algorithmic}[1]
		\Require
		Data points $X=\{x_i\}_{i=1}^{n}\subseteq\mathbb{R}^m$;
		linkage criterion $L$;
		distance metric $\rho(\cdot,\cdot)$;
		target number of clusters $K$ \textbf{or} distance threshold $h_0$
		\Ensure
		Cluster assignments $\{C_1,\dots,C_{K}\}$;
		merge history (dendrogram)
		
		\State Initialize clusters $C_i \gets \{x_i\}$ for $i=1,\dots,n$
		\State Compute initial pairwise dissimilarities $D_L(C_p,C_q)$
		
		\While{stopping criterion not satisfied}
		\State $(p,q)\gets \underset{p\neq q}{\arg\min}D_L(C_p,C_q)$
		\If{$D_L(C_p,C_q) > h_0$}
		\State \textbf{break}
		\EndIf
		\State Merge clusters $C_p$ and $C_q$ into $C_{\text{new}}$
		\State Record merge height $D_L(C_p,C_q)$
		\State Remove $C_p,C_q$ and add $C_{\text{new}}$
		\State Update dissimilarities involving $C_{\text{new}}$
		\If{number of clusters $= K$}
		\State \textbf{break}
		\EndIf
		\EndWhile
	\end{algorithmic}
\end{algorithm}
\cref{alg:AgglomerativeHierarchicalClustering}给出了\gls{AgglomerativeHierarchicalClustering}的基本流程。算法从最细粒度的划分开始,将每一个样本视为一个单独的簇。随后,在当前簇集合中,根据预先选定的簇间距离准则(linkage criterion),反复选择距离最近的两个簇并将其合并。该合并过程持续进行,直到簇的数量减少至预先指定的目标簇数$K$,或最小簇间距离大于给定值$h_0$。\par
层次聚类的核心在于如何定义两个簇之间的距离。设$A,B\subseteq X$为两个簇,$\rho$ 为样本空间中的距离度量,则簇间距离$D_L(A,B)$由 linkage 准则给出。不同的 linkage 定义对应着不同的几何偏好与聚类行为。\par
常见的 linkage 定义如下:
\begin{enumerate}
	\item \textbf{单链接(Single linkage):}$D_{\text{single}}(A,B)=\min\limits_{x\in A,\,y\in B} \rho(x,y)$;
	\item \textbf{全链接(Complete linkage):}$D_{\text{complete}}(A,B)=\max\limits_{x\in A,\,y\in B} \rho(x,y)$;
	\item \textbf{平均链接(Average linkage):}$D_{\text{average}}(A,B)=\dfrac{1}{|A||B|}\sum\limits_{x\in A}\sum\limits_{y\in B}\rho(x,y)$;
	\item \textbf{Ward 方法(Ward linkage):}
	$\Delta(A,B)=\operatorname{SSE}(A\cup B)-\operatorname{SSE}(A)-\operatorname{SSE}(B)$,Ward 方法并不直接基于样本对距离,而是选择使合并后类内平方和增量最小的两个簇
\end{enumerate}

\subsection{DBSCAN}
\begin{algorithm}[H]
	\caption{DBSCAN (Density-Based Spatial Clustering of Applications with Noise)}
	\label{alg:DBSCAN}
	\begin{algorithmic}[1]
		\Require
		Data points $X=\{x_i\}_{i=1}^{n}\subseteq\mathbb{R}^m$;
		neighborhood radius $\varepsilon$;
		minimum number of points $\operatorname{MinPts}$
		\Ensure
		Cluster assignments $\{C_1,\dots,C_K\}$;
		noise set $\mathcal{N}$
		
		\State Mark all points as \textbf{unvisited}
		\State $K \gets 0$, $\mathcal{N}\gets\varnothing$
		
		\For{each point $x_i\in X$}
		\If{$x_i$ is visited}
		\State \textbf{continue}
		\EndIf
		\State Mark $x_i$ as visited
		\State $N_\varepsilon(x_i)\gets\{x_j\in X: \rho(x_j,x_i)\leqslant\varepsilon\}$
		
		\If{$|N_\varepsilon(x_i)|<\operatorname{MinPts}$}
		\State Label $x_i$ as \textbf{noise}; $\mathcal{N}\gets \mathcal{N}\cup\{x_i\}$
		\Else
		\State $K\gets K+1$
		\State Create new cluster $C_K$ and add $x_i$ to $C_K$
		\State $S \gets N_\varepsilon(x_i)\setminus\{x_i\}$
		
		\While{$S\ne \varnothing$}
		\State Select $x_p\in S$ and remove it from $S$
		
		\If{$x_p$ is not visited}
		\State Mark $x_p$ as visited
		\State $N_\varepsilon(x_p)\gets\{x_j\in X: \rho(x_j,x_p)\leqslant\varepsilon\}$
		\If{$|N_\varepsilon(x_p)|\geqslant\operatorname{MinPts}$}
		\State $S \gets S \cup [N_\varepsilon(x_p)\setminus\{x_p\}]$
		\EndIf
		\EndIf
		
		\If{$x_p$ is \textbf{unvisited} or \textbf{noise}}
		\State Add $x_p$ to $C_K$; $\mathcal{N}\gets \mathcal{N}\setminus\{x_p\}$
		\EndIf
		\EndWhile
		\EndIf
		\EndFor
	\end{algorithmic}
\end{algorithm}
\cref{alg:DBSCAN}给出了DBSCAN算法的完整流程。与基于中心的K-means不同,DBSCAN不依赖簇中心或簇数量的预先设定,而是通过样本在空间中的密度分布来刻画簇结构。\par
给定半径参数$\varepsilon$与最小点数阈值$\operatorname{MinPts}$,定义点$x_i$的$\varepsilon$-邻域为:
\begin{equation*}
	N_\varepsilon(x_i)=\{x_j\in X:\|x_j-x_i\|\leqslant\varepsilon\}
\end{equation*}
若$|N_\varepsilon(x_i)|\leqslant\operatorname{MinPts}$,则称$x_i$为核心点;若点本身不是核心点,但位于某一核心点的$\varepsilon$-邻域内,则称其为边界点;其余点被视为噪声。\par
DBSCAN从任一未访问样本出发,若其为核心点,则以该点为起始,通过密度可达关系不断扩张簇;若样本不是核心点,则将其标记为噪声或边界点。该扩张过程持续进行,直至当前簇无法继续增长。\par
由于DBSCAN能够识别任意形状的簇并显式区分噪声点,其在处理非凸簇结构时具有显著优势。然而,算法对参数 $\varepsilon$ 与$\operatorname{MinPts}$较为敏感,且在高维空间中邻域查询的效率可能显著下降,这在一定程度上限制了其适用范围。

\subsection{OPTICS}
\begin{algorithm}[H]
	\caption{OPTICS (Ordering Points To Identify the Clustering Structure)}
	\label{alg:OPTICS}
	\begin{algorithmic}[1]
		\Require
		Data points $X=\{x_i\}_{i=1}^{n}\subseteq\mathbb{R}^m$;
		maximum neighborhood radius $\varepsilon_{\max}$;
		minimum number of points $\operatorname{MinPts}$
		\Ensure
		Cluster ordering $\mathcal{O}$;
		reachability distances $\{d_{\text{reach}}(x_i)\}$
		
		\State Mark all points as \textbf{unprocessed}
		\State Initialize $d_{\text{reach}}(x_i)\gets +\infty$ for all $x_i$ and empty ordering list $\mathcal{O}$
		
		\For{each point $x_i\in X$}
		\If{$x_i$ is processed}
		\State \textbf{continue}
		\EndIf
		
		\State Compute neighborhood $N_{\varepsilon_{\max}}(x_i)$
		\State Mark $x_i$ as processed; Append $x_i$ to $\mathcal{O}$
		
		\If{$|N_{\varepsilon_{\max}}(x_i)|\geqslant \operatorname{MinPts}$}
		\State Compute core distance
		\[
		d_{\text{core}}(x_i)
		=\text{distance to the $\operatorname{MinPts}$-th nearest neighbor of }x_i
		\]
		
		\State Initialize empty priority queue $S$
		
		\For{each unprocessed point $x_j\in N_{\varepsilon_{\max}}(x_i)$}
		\State $d_{\text{new}}
		\gets \max\{d_{\text{core}}(x_i),\,\rho(x_i,x_j)\}$
		\State $d_{\text{reach}}(x_j)
		\gets \min\{d_{\text{reach}}(x_j),\,d_{\text{new}}\}$
		\State Insert or update $x_j$ in $S$ with key $d_{\text{reach}}(x_j)$
		\EndFor
		
		\While{$S\ne \varnothing$}
		\State Extract $x_p$ with smallest $d_{\text{reach}}(x_p)$ from $S$
		\State Compute neighborhood $N_{\varepsilon_{\max}}(x_p)$
		\State Mark $x_p$ as processed
		\State Append $x_p$ to $\mathcal{O}$
		
		\If{$|N_{\varepsilon_{\max}}(x_p)|\geqslant \operatorname{MinPts}$}
		\State Compute $d_{\text{core}}(x_p)$
		\For{each unprocessed point $x_q\in N_{\varepsilon_{\max}}(x_p)$}
		\State $d_{\text{new}}
		\gets \max\{d_{\text{core}}(x_p),\,\rho(x_p,x_q)\}$
		\State $d_{\text{reach}}(x_q)
		\gets \min\{d_{\text{reach}}(x_q),\,d_{\text{new}}\}$
		\State Insert or update $x_q$ in $S$ with key $d_{\text{reach}}(x_q)$
		\EndFor
		\EndIf
		\EndWhile
		\EndIf
		\EndFor
	\end{algorithmic}
\end{algorithm}
\cref{alg:OPTICS} 给出了OPTICS算法的完整流程。OPTICS属于基于密度的聚类方法,其基本密度可达性定义与 DBSCAN 保持一致,但其核心目标并非直接给出单一的聚类划分结果,而是通过对样本点进行有序遍历,显式刻画数据在不同密度尺度下的潜在聚类结构。\par
在 OPTICS 中,引入\gls{CoreDistance}以刻画样本点局部密度水平。对于样本点$x_i$,其核心距离定义为:\par
当点$x_i$的$\varepsilon_{\max}$-邻域中包含的样本数不少于$\operatorname{MinPts}$时,$x_i$被视为核心点,其核心距离反映了达到该局部密度所需的最小尺度;否则,该点被视为非核心点,其核心距离定义为无穷大。\par
基于核心距离,OPTICS 进一步定义样本$x_j$相对于核心点$x_i$的\gls{ReachabilityDistance} 。可达距离刻画了在保持局部密度约束的前提下,从核心点$x_i$触及样$x_j$所需的最小距离尺度,是 OPTICS 构建样本排序的关键量。\par
OPTICS 从任一未处理样本出发,若该样本为核心点,则利用可达距离对其邻域样本进行基于优先级队列的扩张,并不断选择当前可达距离最小的样本进行处理。通过该过程,算法在不固定单一密度阈值的情况下,生成一个包含所有样本的线性排序 $\mathcal{O}$,并为每个样本记录其对应的可达距离。\par
最终得到的可达距离图通过按照排序顺序绘制样本的可达距离,以一维形式揭示数据在不同密度尺度下的结构特征。在该图中,连续的低可达距离区域通常对应于高密度簇,而不同密度簇之间则表现为明显的“高度跃迁”或“谷地边界”。通过在可达距离图上选择不同的高度阈值进行切割,可以在后处理阶段提取出对应于不同密度水平的聚类结果。\par

从方法论角度看,DBSCAN 可被视为 OPTICS 在固定邻域半径 $\varepsilon$ 下的特例:当在可达距离图中选定单一阈值时,OPTICS 所得到的聚类结果与 DBSCAN 在相同参数设置下是等价的。相比之下,OPTICS 通过保留完整的密度排序信息,能够同时刻画多尺度密度结构,在不同密度簇共存的观测样本集中具有更强的表达能力。\par

在计算复杂度方面,OPTICS 的主要开销来自于邻域查询与优先队列操作。在采用空间索引结构(如 $k$-d tree 或 ball tree)的情况下,其时间复杂度通常与 DBSCAN 相当;在最坏情况下,时间复杂度可达 $O(n^2)$。算法的空间复杂度主要由邻域信息与可达距离存储决定,通常为 $O(n)$ 至 $O(n^2)$,具体取决于实现方式与数据分布。