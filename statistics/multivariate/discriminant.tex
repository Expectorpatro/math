\section{判别分析}

判别分析的目的为判断样本最有可能属于哪一个总体,这属于一个分类问题。它利用已知分类的样本(称之为训练样本,训练样本的集合称为训练集)去估计\textbf{判别函数}(即一个以样本为自变量的函数,我们利用该函数的取值将样本进行分类)的具体形式。

\subsection{距离判别}
\begin{definition}
	设有$n$个总体$D_i$,它们的均值向量与协方差矩阵分别为$\boldsymbol{\mu_i},\Sigma_i,\;i=1,2,\dots,n$。现存在一个从上述$n$个总体中抽取出的样本$x$,若将其判给满足:
	\begin{equation*}
		j=\arg\min d^2_m(x,D_i)=(x-\boldsymbol{\mu})^T\Sigma^{-1}(x-\boldsymbol{\mu})
	\end{equation*}
	的分布$D_j$,称这种判别方法为\textbf{距离判别}。距离判别的判别函数为:
	\begin{equation*}
		(x-\boldsymbol{\mu})^T\hat{\Sigma}^{-1}(x-\boldsymbol{\mu})
	\end{equation*}
\end{definition}

\subsection{Bayesian判别}

\subsection{Fisher判别}
\begin{definition}
	设有$n$个总体$D_i$,它们的均值向量与协方差矩阵分别为$\boldsymbol{\mu_i},\Sigma_i,\;i=1,2,\dots,n$。现存在一个从上述$n$个总体中抽取出的未知分类的样本$x$,记线性判别函数为$u(y)=\alpha^Ty$,并且记:
	\begin{equation*}
		u_i=\operatorname{E}[u(\mathbf{X})|\mathbf{X}\in D_i],\quad\overline{u}=\frac{1}{n}\sum_{i=1}^{n}u_i,\quad\sigma_i^2=\operatorname{Var}[u(\mathbf{X})|\mathbf{X}\in D_i]
	\end{equation*}
	取$\alpha$满足:
	\begin{equation*}
		\alpha=\arg\max\frac{\sum\limits_{i=1}^{n}(u_i-\overline{u})^2}{\sum\limits_{i=1}^{n}\sigma_i^2}
	\end{equation*}
	Fisher提议将$x$判给满足:
	\begin{equation*}
		j=\arg\min|u(x)-u_i| 
	\end{equation*}
	的分布$D_j$,称这种判别方法为\textbf{Fisher判别}。Fisher判别的判别函数为:
	\begin{equation*}
		|\hat{u}(x)-\hat{u}_i|
	\end{equation*}
\end{definition}
\begin{note}
	在实际分析中,我们会取:
	\begin{equation*}
		\alpha=\arg\max\frac{\sum\limits_{i=1}^{n}n_i(u_i-\overline{u})^2}{\sum\limits_{i=1}^{n}(n_i-1)\sigma_i^2}
	\end{equation*}
	其中$n_i$为训练集中属于总体$D_i$的样本数。从接下来的分析可以看出这样作实际上有利于求解$\alpha$。
\end{note}
\begin{theorem}
	设从总体$D_i$中获取到了$n_i$个样本$x^{(i)}_j,\;i=1,2,\dots,n,\;j=1,2,\dots,n_i$。Fisher中的$\alpha$为矩阵$B$关于矩阵$S$的最大特征值对应的特征向量,$B$与$S$的定义请见证明过程。
\end{theorem}
\begin{proof}
	注意到此时有:
	\begin{gather*}
		\hat{u}_i=\frac{1}{n_i}\sum_{j=1}^{n_i}\alpha^Tx^{(i)}_j=\alpha^T\overline{x^{(i)}},\;i=1,2,\dots,n \\
		\hat{\overline{u}}=\frac{1}{n}\sum_{i=1}^{n}\hat{u}_i=\frac{1}{n}\sum_{i=1}^{n}\alpha^T\overline{x^{(i)}}=\alpha^T\overline{x} \\
		\begin{aligned}
			\hat{\sigma}_i^2
			&=\frac{1}{n_i-1}\sum_{j=1}^{n_i}[u(x^{(i)}_j)-\hat{u}_i]^2=\frac{1}{n_i-1}\sum_{j=1}^{n_i}[\alpha^Tx^{(i)}_j-\alpha^T\overline{x^{(i)}}]^2 \\
			&=\frac{1}{n_i-1}\sum_{j=1}^{n_i}[\alpha^T(x^{(i)}_j-\overline{x^{(i)}})]^2=\frac{1}{n_i-1}\sum_{j=1}^{n_i}\alpha^T(x^{(i)}_j-\overline{x^{(i)}})(x^{(i)}_j-\overline{x^{(i)}})^T\alpha \\
			&=\frac{1}{n_i-1}\alpha^T\left[\sum_{j=1}^{n_i}(x^{(i)}_j-\overline{x^{(i)}})(x^{(i)}_j-\overline{x^{(i)}})^T\right]\alpha\coloneqq\frac{1}{n_i-1}\alpha^TS_i\alpha
		\end{aligned}
	\end{gather*}
	于是:
	\begin{gather*}
		\sum_{i=1}^{n}(n_i-1)\sigma_i^2=\sum_{i=1}^{n}(n_i-1)\frac{1}{n_i-1}\alpha^TS_i\alpha\coloneqq\alpha^TS\alpha \\
		\sum_{i=1}^{n}n_i(\hat{u}_i-\hat{\overline{u}})^2=\sum_{i=1}^{n}n_i[\alpha^T\overline{x^{(i)}}-\alpha^T\overline{x}]^2=\sum_{i=1}^{n}n_i\alpha^T(\overline{x^{(i)}}-\overline{x})(\overline{x^{(i)}}-\overline{x})^T\alpha\coloneqq\alpha^TB\alpha
	\end{gather*}
	所以:
	\begin{equation*}
		\alpha=\arg\max\frac{\alpha^TB\alpha}{\alpha^TS\alpha}
	\end{equation*}
	但若此时不作任何限制,由于对$\alpha$作任意倍的放缩都满足上式,所以解得的$\alpha$并不唯一,我们人为地添加约束条件$\alpha^TS\alpha=1$,于是该最优化问题变为:
	\begin{gather*}
		\alpha=\arg\max\frac{\alpha^TB\alpha}{\alpha^TS\alpha} \\
		\operatorname{s.t.}\alpha^TS\alpha=1
	\end{gather*}
	使用Lagrange乘子法进行求解,引入Lagrange乘子$\lambda$构造Lagrange函数:
	\begin{equation*}
		L(\alpha,\lambda)=\alpha^TB\alpha-\lambda(\alpha^TS\alpha-1)
	\end{equation*}
	求导可得:
	\begin{equation*}
		\frac{\partial L(\alpha,\lambda)}{\partial\alpha}=2B\alpha-2\lambda S\alpha,\quad\frac{\partial L(\alpha,\lambda)}{\partial\lambda}=1-\alpha^TS\alpha=0
	\end{equation*}
	于是只需令上第一式为$0$,即:
	\begin{equation*}
		B\alpha=\lambda S\alpha
	\end{equation*}
	所以$\lambda$为矩阵$B$关于矩阵$S$的广义特征值,此时目标函数为:
	\begin{equation*}
		\alpha^TB\alpha=\lambda\alpha^TS\alpha=\lambda
	\end{equation*}
	由此可以看出该最优化问题的最优解为$\lambda$是最大的广义特征值且$\alpha$为对应于$\lambda$的特征向量。
\end{proof}