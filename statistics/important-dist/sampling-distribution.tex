\section{$\chi^2$分布,$t$分布和$F$分布}

\subsection{$\chi^2$分布}
\begin{definition}
	设$\mathbf{X}\sim N_n(\boldsymbol{\mu}, I_n)$,则随机变量$\mathbf{Y}=X^TX$的分布称为自由度为$n$、非中心参数为$\lambda=\boldsymbol{\mu}^T\boldsymbol{\mu}$的$\chi^2$分布,记为$\mathbf{Y}\sim \chi^2_{n,\lambda}$。当$\lambda=0$时,称$\mathbf{Y}$的分布为中心$\chi^2$分布,记为$Y\sim\chi_n^2$。
\end{definition}
\begin{property}\label{prop:Chi2Distribution}
	$\chi^2$分布具有如下性质:
	\begin{enumerate}
		\item 设$Y_i\sim\chi_{n_i,\lambda_i}^2,\;i=1,2,\dots,k$相互独立,则:
		\begin{gather*}
			\sum_{i=1}^{k}Y_i\sim\chi_{n,\lambda}^2,\quad\text{其中}
			n=\sum_{i=1}^{k}n_i,\;\lambda=\sum_{i=1}^{k}\lambda_i
		\end{gather*}
		\item 设$Y\sim\chi_{n,\lambda}^2$,则$\operatorname{E}(Y)=n+\lambda,\;\operatorname{Var}(Y)=2n+4\lambda$;
		\item 设$Y\sim\chi_{n,\lambda}^2$,$\mathbf{X}\sim N_n(\boldsymbol{\mu},I_n),\;\mathbf{Y}=\mathbf{X}^T\mathbf{X}$,则:
		\begin{equation*}
			\varphi_{\mathbf{Y}}(t)=(1-2it)^{-\frac{n}{2}}\exp\left\{\frac{it\lambda}{1-2it}\right\}
		\end{equation*}
	\end{enumerate}
\end{property}
\begin{proof}
	(1)设$Y_i=\mathbf{X_i}^T\mathbf{X_i}$,其中$\mathbf{X_i}\sim N_{n_i}(\boldsymbol{\mu_i},I_{n_i})$。令$\mathbf{X}=(\mathbf{X_1},\mathbf{X_2},\dots,\mathbf{X_k})^T$,则有
	\begin{equation*}
		\sum_{i=1}^{k}Y_i=\sum_{i=1}^{k}\mathbf{X_i}^T\mathbf{X_i}=(\mathbf{X_1},\mathbf{X_2},\dots,\mathbf{X_k})(\mathbf{X_1},\mathbf{X_2},\dots,\mathbf{X_k})^T=\mathbf{X}^T\mathbf{X}
	\end{equation*}
	因为$Y_i$相互独立,所以$\mathbf{X_i}$也相互独立,于是$\mathbf{X}\sim N_n(\boldsymbol{\mu},I_n)$,其中:
	\begin{equation*}
		n=\sum_{i=1}^{k}n_i,\;\boldsymbol{\mu}=(\boldsymbol{\mu_1},\boldsymbol{\mu_2},\dots,\boldsymbol{\mu_n})^T
	\end{equation*}
	因此有:
	\begin{equation*}
		\sum_{i=1}^{k}Y_i\sim\chi_{n,\lambda}^2,\;\lambda=\boldsymbol{\mu}^T\boldsymbol{\mu}=\sum_{i=1}^{k}\boldsymbol{\mu_i}^T\boldsymbol{\mu_i}=\sum_{i=1}^{k}\lambda_i
	\end{equation*}\par
	(2)因为$Y\sim\chi_{n,\lambda}^2$,由定义可知$Y$可以表示为:
	\begin{equation*}
		Y=\sum_{i=1}^{n}X_i^2,\;X_i\sim N(\mu_i,1),\;\sum_{i=1}^{n}\mu_i^2=\lambda
	\end{equation*}
	其中$X_i$相互独立。由\cref{prop:Variance}(1)可知:
	\begin{equation*}
		\operatorname{E}(Y)=\operatorname{E}\left(\sum_{i=1}^{n}X_i^2\right)=\sum_{i=1}^{n}\operatorname{E}(X_i^2)=\sum_{i=1}^{n}\{\operatorname{Var}(X_i)+[\operatorname{E}(X_i)]^2\}=\sum_{i=1}^{n}(1+\mu_i^2)=n+\lambda
	\end{equation*}
	因为$X_i$相互独立,由\info{链接独立方差等于和}可知:
	\begin{align*}
		\operatorname{Var}(Y)
		&=\operatorname{Var}\left(\sum_{i=1}^{n}X_i^2\right)=\sum_{i=1}^{n}\operatorname{Var}(X_i^2)=\sum_{i=1}^{n}\{\operatorname{E}(X_i^4)-[\operatorname{E}(X_i^2)]^2\} \\
		&=\sum_{i=1}^{n}\operatorname{E}(X_i^4)-\sum_{i=1}^{n}[\operatorname{E}(X_i^2)]^2
	\end{align*}
	由\cref{prop:Variance}(1)可知:
	\begin{equation*}
		\operatorname{E}(X_i^2)=\operatorname{Var}(X_i)+[\operatorname{E}(X_i)]^2=1+\mu_i^2
	\end{equation*}
	所以:
	\begin{equation*}
		\sum_{i=1}^{n}[\operatorname{E}(X_i^2)]^2=\sum_{i=1}^{n}(\mu_i^4+2\mu_i^2+1)=\sum_{i=1}^{n}\mu_i^4+2\sum_{i=1}^{n}\mu_i^2+n=\sum_{i=1}^{n}\mu_i^4+2\lambda+n
	\end{equation*}
	而:
	\begin{equation*}
		\operatorname{E}(X_i^4)=\mu_i^4+6\mu_i^2+3
	\end{equation*}
	于是:
	\begin{align*}
		\operatorname{Var}(Y)
		&=\sum_{i=1}^{n}\operatorname{E}(X_i^4)-\sum_{i=1}^{n}[\operatorname{E}(X_i^2)]^2 \\
		&=\sum_{i=1}^{n}\mu_i^4+6\sum_{i=1}^{n}\mu_i^2+3n-\sum_{i=1}^{n}\mu_i^4-2\lambda-n \\
		&=6\lambda+3n-2\lambda-n=2n+4\lambda
	\end{align*}\par
	(3)因为$\mathbf{X}\sim N_n(\boldsymbol{\mu},I_n)$,由\cref{prop:MultiNormal}(8)可知$\mathbf{X}_i$相互独立,所以$\mathbf{X}_i^2$相互独立。因为$\mathbf{Y}=\mathbf{X}^T\mathbf{X}=\sum\limits_{i=1}^n\mathbf{X}_i^2$,由\cref{prop:CharacteristicFunction}(4)可知:
	\begin{equation*}
		\varphi_{\mathbf{Y}}(t)=\prod_{i=1}^n\varphi_{\mathbf{X}_i^2}(t)
	\end{equation*}
	下面来求$\varphi_{\mathbf{X}_i^2}$。\par
	由\cref{prop:MultiNormal}(3)可知$\mathbf{X}_i\sim N(\mu_i,1)$,于是:
	\begin{align*}
		\varphi_{\mathbf{X}_i^2}(t)
		&=\operatorname{E}(e^{it\mathbf{X}_i^2}) \\
		&=\int_{-\infty}^{+\infty}e^{itx^2}\frac{1}{\sqrt{2\pi}}e^{-\frac{(x-\mu_i)^2}{2}}\dif x \\
		&=\frac{1}{\sqrt{2\pi}}\int_{-\infty}^{+\infty}\exp\left\{-\frac{x^2 - 2\mu_ix + \mu_i^2}{2} + itx^2\right\}\dif x \\
		&=\frac{1}{\sqrt{2\pi}}\int_{-\infty}^{+\infty}\exp\left\{-\frac{x^2}{2} (1 - 2it) + \mu_ix - \frac{\mu_i^2}{2}\right\}\dif x \\
		&=\frac{1}{\sqrt{2\pi}}e^{-\frac{\mu_i^2}{2}} \int_{-\infty}^{\infty} \exp\left\{-\frac{x^2}{2} (1 - 2it) + \mu_ix\right\}\dif x
	\end{align*}
	这是一个Gaussian积分,由Gaussian积分公式可得:
	\begin{align*}
		\varphi_{\mathbf{X}_i^2}(t)
		&=\frac{1}{\sqrt{2\pi}}e^{-\frac{\mu_i^2}{2}} \int_{-\infty}^{\infty}\exp\left\{-\frac{x^2}{2} (1 - 2it) + \mu_i x\right\}\dif x \\
		&=\frac{1}{\sqrt{2\pi}}e^{-\frac{\mu_i^2}{2}}\sqrt{\frac{2\pi}{1-2it}}e^{\frac{\mu_i^2}{2-4it}}
		=(1-2it)^{-\frac{1}{2}}\exp\left\{\frac{\mu_i^2}{2-4it}-\frac{\mu_i^2}{2}\right\} \\
		&=(1-2it)^{-\frac{1}{2}}\exp\left\{\frac{it\mu_i^2}{1-2it}\right\}
	\end{align*}
	于是:
	\begin{equation*}
		\varphi_{\mathbf{Y}}=\prod_{i=1}^n(1-2it)^{-\frac{1}{2}}\exp\left\{\frac{it\mu_i^2}{1-2it}\right\}=(1-2it)^{-\frac{n}{2}}\exp\left\{\frac{it\lambda}{1-2it}\right\}
	\end{equation*}
\end{proof}

\subsection{$t$分布}
\begin{definition}
	设随机变量$X\sim N(0,1),\;Y\sim\chi_n^2$且$X$与$Y$独立,则称:
	\begin{equation*}
		T=\frac{X}{\sqrt{Y/ n}}
	\end{equation*}
	为自由度是$n$的$t$变量,其分布称为自由度为$n$的$t$分布,记为$T\sim t_n$。
\end{definition}

\subsection{$F$分布}
\begin{definition}
	设随机变量$X\sim \chi_m^2,\;Y\sim\chi_n^2$且$X$与$Y$独立,则称:
	\begin{equation*}
		F=\frac{X/m}{Y/n}
	\end{equation*}
	为自由度是$m$和$n$的$F$变量,其分布称为自由度为$m$和$n$的$F$分布,记为$F\sim F_{m,n}$。
\end{definition}
\begin{property}\label{prop:FDistribution}
	$F$分布具有如下性质:
	\begin{enumerate}
		\item 若$F\sim F_{m,n}$,则有$\frac{1}{F}\sim F_{n,m}$;
		\item 若$T\sim t_n$,则有$T^2\sim F_{1,n}$;
		\item $F_{m,n}(1-\alpha)=\dfrac{1}{F_{n,m}(\alpha)}$;
	\end{enumerate}
\end{property}
\begin{proof}
	(1)由$F$分布的定义直接可得。\par
	(2)设:
	\begin{equation*}
		T=\frac{X}{\sqrt{Y/n}}
	\end{equation*}
	其中$X\sim N(0,1),\;Y\sim\chi_n^2$且$X$与$Y$独立,于是:
	\begin{equation*}
		T^2=\frac{X^2}{Y/n}=\frac{X^2/1}{Y/n}
	\end{equation*}
	注意到$X^2\sim\chi_1^2$且有$X^2$与$Y$独立,由$F$分布的定义即可得到$T^2\sim F_{1,n}$。\par
	(3)由分位数的定义:
	\begin{gather*}
		P[F>F_{m,n}(1-\alpha)]=1-\alpha \\
		P\left[\frac{X/m}{Y/n}>F_{m,n}(1-\alpha)\right]=1-\alpha \\
		P\left[\frac{Y/n}{X/m}<\frac{1}{F_{m,n}(1-\alpha)}\right]=1-\alpha \\
		P\left[\frac{Y/n}{X/m}\geqslant\frac{1}{F_{m,n}(1-\alpha)}\right]=\alpha \\
		P\left[\frac{Y/n}{X/m}>\frac{1}{F_{m,n}(1-\alpha)}\right]=\alpha
	\end{gather*}
	即:
	\begin{equation*}
		F_{m,n}(1-\alpha)=\frac{1}{F_{n,m}(\alpha)}
	\end{equation*}
\end{proof}