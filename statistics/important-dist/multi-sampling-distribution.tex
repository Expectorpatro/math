\section{多元分布}

\subsection{Wishart分布}
\begin{definition}
	设$\mathbf{X_i}\text{i.i.d.}\sim\operatorname{N}_p(\mathbf{0},\Sigma),\;i=1,2,\dots,n,\;\Sigma>0,n\geqslant p$,记$\mathbf{X}=(\mathbf{X_1},\mathbf{X_2},\dots,\mathbf{X_n})^T$,称随机矩阵:
	\begin{equation*}
		\mathbf{W}=\mathbf{X}^T\mathbf{X}=\sum_{i=1}^{n}\mathbf{X_i}\mathbf{X_i}^T
	\end{equation*}
	所服从的分布为自由度为$n$的$p$维中心Wishart分布,记为$\mathbf{W}\sim\operatorname{W}_p(n,\Sigma)$。
\end{definition}
\begin{property}\label{prop:Wishart}
	Wishart分布具有如下性质:
	\begin{enumerate}
		\item 若$\mathbf{W}_i\sim\operatorname{W}_p(n_i,\Sigma),\;i=1,2,\dots,m$且相互独立,则:
		\begin{equation*}
			\sum_{i=1}^{m}\mathbf{W}_i\sim\operatorname{W}_p\left(\sum_{i=1}^{m}n_i,\Sigma\right)
		\end{equation*}
		\item 若$\mathbf{W}\sim\operatorname{W}_p(n,\Sigma)$,$C\in M_{m\times p}(\mathbb{R}^{})$且可逆,则:
		\begin{equation*}
			C\mathbf{W}C^T\sim\operatorname{W}_m(n, C\Sigma C^T)
		\end{equation*}
		\item 若$\mathbf{W}\sim\operatorname{W}_p(n,\Sigma)$,$\alpha\in\mathbb{R}^{p}$是任意常数向量,且$\alpha^T\Sigma\alpha\ne0$,则:
		\begin{equation*}
			\frac{\alpha^T\mathbf{W}\alpha}{\alpha^T\Sigma\alpha}\sim\chi_n^2
		\end{equation*}
		\item 若$\mathbf{W}\sim\operatorname{W}_p(n,\Sigma)$,$\alpha\in\mathbb{R}^{p}$是任一非零常数向量,则:
		\begin{equation*}
			\frac{\alpha^T\Sigma^{-1}\alpha}{\alpha^T\mathbf{W}^{-1}\alpha}\sim\chi_{n-p+1}^2
		\end{equation*}
	\end{enumerate}
\end{property}
\begin{proof}
	(1)由Wishart分布的定义和:
	\begin{equation*}
		\sum_{i=1}^{m}\mathbf{W}_i=\sum_{i=1}^{m}\sum_{j=1}^{n_i}\mathbf{X_{ij}}\mathbf{X_{ij}}^T
	\end{equation*}
	立即可得。\par
	(2)因为$C\mathbf{X}^T=(C\mathbf{X_1},C\mathbf{X_2},\dots,C\mathbf{X_n})$,由\cref{theo:MultiNormalLinearTransform}可得$C\mathbf{X_i}\sim\operatorname{N}_m(\mathbf{0},C\Sigma C^T)$且相互独立。由定义即可得到:
	\begin{equation*}
		C\mathbf{W}C^T=C\mathbf{X}^T\mathbf{X}C^T\sim\operatorname{W}_m(n,C\Sigma C^T)
	\end{equation*}\par
	(3)注意到:
	\begin{equation*}
		\alpha^T\mathbf{W}\alpha=\sum_{i=1}^{n}\alpha^T\mathbf{X_i}\mathbf{X_i}^T\alpha=\sum_{i=1}^{n}(\alpha^T\mathbf{X_i})^2
	\end{equation*}
	由\cref{theo:MatNormalLinearTransform}可知$\alpha^T\mathbf{X_i}\sim\operatorname{N}(0,\alpha^T\Sigma\alpha)$且相互独立。因为$\Sigma>0,\;\alpha^T\Sigma\alpha\ne0$,所以$\alpha^T\Sigma\alpha>0$,即$(\alpha^T\Sigma\alpha)^{-\frac{1}{2}}$存在。由\cref{theo:MatNormalLinearTransform}可知:
	\begin{equation*}
		\frac{\alpha^T\mathbf{X_i}}{(\alpha^T\Sigma\alpha)^{\frac{1}{2}}}\sim\operatorname{N}(0,1)
	\end{equation*}
	所以:
	\begin{equation*}
		\frac{\alpha^T\mathbf{W}\alpha}{\alpha^T\Sigma\alpha}=\frac{\sum\limits_{i=1}^{n}(\alpha^T\mathbf{X_i})^2}{\alpha^T\Sigma\alpha}=\sum_{i=1}^{n}\left(\frac{\alpha^T\mathbf{X_i}}{(\alpha^T\Sigma\alpha)^{\frac{1}{2}}}\right)^2\sim\chi_n^2
	\end{equation*}\par
	(4)不给予证明。
\end{proof}

\subsection{$T^2$分布}
\begin{definition}
	设$\mathbf{W}\sim\operatorname{W}_p(n,\Sigma),\;\mathbf{X}\sim\operatorname{N}_p(\mathbf{0},c\Sigma),\;c>0,\Sigma>0,n\geqslant p$,$\mathbf{W}$与$\mathbf{X}$相互独立,则称随机变量:
	\begin{equation*}
		T^2=\frac{n}{c}\mathbf{X}^T\mathbf{W}^{-1}\mathbf{X}
	\end{equation*}
	所服从的分布为第一自由度为$p$、第二自由度为$n$的中心$T^2$分布,记为$T^2\sim T^2(p,n)$。
\end{definition}
\begin{property}\label{prop:T^2}
	$T^2$分布有如下性质:
	\begin{enumerate}
		\item 设$\mathbf{X}\sim\operatorname{N}_p(\boldsymbol{\mu},c\Sigma),\;\mathbf{W}\sim\mathbf{W}_p(n,\Sigma)$,$\mathbf{X}$与$\mathbf{W}$相互独立,则:
		\begin{equation*}
			\frac{n}{c}(\mathbf{X}-\boldsymbol{\mu})^T\mathbf{W}^{-1}(\mathbf{X}-\boldsymbol{\mu})\sim T^2(p,n)
		\end{equation*}
		\item $T^2$分布可化为中心$F$分布:
		\begin{equation*}
			\frac{n-p+1}{pn}T^2(p,n)=F(p,n-p+1)
		\end{equation*}
	\end{enumerate}
\end{property}
\begin{proof}
	(1)由$T^2$分布的定义立即可得。\par
	(2)不给予证明。
\end{proof}

\subsection{Wilks分布}
\begin{definition}
	设$\mathbf{W}_1\sim\operatorname{W}_p(m,\Sigma),\;\mathbf{W}_2\sim\operatorname{W}_p(n,\Sigma),\;\Sigma>0,\;m,n>p$,$\mathbf{W}_1$与$\mathbf{W}_2$独立,称随机变量:
	\begin{equation*}
		\varLambda=\frac{\det\mathbf{W}_1}{\det(\mathbf{W}_1+\mathbf{W}_2)}
	\end{equation*}
	所服从的分布为维数为$p$、第一自由度为$m$、第二自由度为$n$的Wilks$\varLambda$分布,记为$\varLambda\sim\varLambda(p,m,n)$。
\end{definition}