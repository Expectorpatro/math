\chapter{假设检验理论}
\begin{definition}
	设$(X,\mathscr{A},\mathscr{P})$是统计结构,则称$\mathscr{P}$的非空子集为\gls{Hypothesis}。若$(X,\mathscr{A},\mathscr{P})$是参数结构,$\Theta$是参数空间,则$\Theta$的非空子集也被称之为假设。设$\mathbf{X}$是从总体$F$中抽取的简单样本,判断$F$是否服从某一假设的问题是\gls{HypothesisTesting}问题。称要检验的假设$\mathscr{P}_0(\Theta_0)$为\gls{NullHypothesis},记为$H_0$,$\mathscr{P_1}\subseteq\mathscr{P}\backslash\mathscr{P}_0(\Theta_1\subseteq\Theta\backslash\Theta_0)$被称之为\gls{AlternativeHypothesis},记为$H_1$。称$(H_0,H_1)$为检验问题。检验指的是:
	\begin{enumerate}
		\item 将$X$划分为互不相交的两个$\mathscr{P}$可测集,即$X=W\cup\overline{W},\;W\cap\overline{W}=\varnothing,\;W,\overline{W}\in\mathscr{A}$;
		\item 作如下规定,若$\mathbf{X}\in W$,则拒绝原假设$H_0$,接受备择假设$H_1$;若$\mathbf{X}\in\overline{W}$,则接受原假设$H_0$,拒绝备择假设$H_1$。
	\end{enumerate}
	称$W$为\gls{RejectRegion}。
\end{definition}
\begin{definition}
	在检验问题$(H_0,H_1)$中,若$H_0$为真但由于样本的随机性拒绝了$H_0$,称此时犯的错误为\gls{Type1Error}或\textbf{拒真错误};若$H_0$为伪但由于样本的随机性接受了$H_0$,称此时犯的错误为\gls{Type2Error}或\textbf{取伪错误}。
\end{definition}
\begin{note}
	理想中的情况是有方法(即存在$X$的一个划分)能够同时降低犯两类错误的可能,但实际情况是降低第一类错误发生的概率就会提高第二类错误发生的概率,反之降低第二类错误发生的概率就会提高第一类错误发生的概率。Neyman和Pearson提出了一个基本思想:先使得犯第一类错误的可能性尽可能地小,然后尽可能地降低犯第二类错误的可能性,即拒真优先。
\end{note}
\begin{definition}
	设$(X,\mathscr{A},\mathscr{P})$是统计结构,$\mathbf{X}$为从总体$F$中抽取的简单样本,$(H_0,H_1)$是检验问题。称:
	\begin{equation*}
		\forall\;P\in\mathscr{P},\;g(P)=P(W)
	\end{equation*}
	为检验问题$(H_0,H_1)$的\gls{PowerFunction}。当$F\in H_0$时,$g(F)\leqslant\alpha$的检验被称之为\gls{SignificanceLevel}为$\alpha$的检验。
\end{definition}
\begin{note}
	当$F\in H_0$时,$g(F)$为犯第一类错误的概率;当$F\in H_1$时,$1-g(F)$为犯第二类错误的概率。\par
	实际中我们根本不知道$F$到底是在$H_0$中还是在$H_1$中,假设检验本就是为了判断这个问题才出现的,为了得到显著性为$\alpha$的检验,实际上我们做的是找到$X$的一个划分使得:
	\begin{equation*}
		\sup_{P\in H_0}g(P)\leqslant\alpha
	\end{equation*}
	这就会导致当$F$真的在$H_0$中时,犯第一类错误的概率实际上有可能小于$\alpha$。这代表着什么?根据前述同时降低两类错误的矛盾性和Neyman、Pearson的基本思想,如果犯第一类错误的概率实际上小于$\alpha$,那么在显著性水平为$\alpha$的检验中犯第二类错误的概率不是最低的,检验还能进一步优化。由此引入下述随机化检验的概念。
\end{note}
\begin{definition}
	设$\varphi$是可测空间$(X,\mathscr{A})$上的可测函数。若$0\leqslant\varphi\leqslant1$,则称$\varphi$是\gls{TestFunction}。当$\varphi$只取$0,1$两个值时,称对应的检验为\gls{NonrandomizedTesting},否则称为\gls{RandomizedTesting}。$\varphi$的势函数为$\operatorname{E}(\varphi)$。
\end{definition}
\begin{definition}
	设$(H_0,H_1)$是统计结构$(X,\mathscr{A},\mathscr{P})$上的一个假设检验问题,$\mathbf{X}$是从总体$F$中抽取的简单样本,利用$\mathbf{X}$能够作出拒绝原假设的最小显著性水平被称为检验的\gls{PValue}。
\end{definition}
