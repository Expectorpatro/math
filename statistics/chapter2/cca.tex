\section{典型相关分析}
\gls{CCA}
\subsection{总体典型相关分析}
\begin{definition}
	设$\mathbf{X},\mathbf{Y}$分别是$m,n$维随机向量,$m+n$维随机向量$(\mathbf{X},\mathbf{Y})^T$的均值向量$\boldsymbol{\mu}=\mathbf{0}$,协方差矩阵$\Sigma$是正定阵。对$\mathbf{X},\mathbf{Y}$分别进行线性变换$\mathcal{T}_1,\mathcal{T}_2$得到两个$p$维随机向量$\mathbf{U}=\seq{\mathbf{U}}{p},\;\mathbf{V}=\seq{\mathbf{V}}{p}$,$\mathcal{T}_1$和$\mathcal{T}_2$的矩阵分别为:
	\begin{equation*}
		A=
		\begin{pmatrix}
			\alpha_1^T \\
			\alpha_2^T \\
			\vdots \\
			\alpha_p^T
		\end{pmatrix},\quad
		B=
		\begin{pmatrix}
			\beta_1^T \\
			\beta_2^T \\
			\vdots \\
			\beta_p^T
		\end{pmatrix}
	\end{equation*}
	若:
	\begin{enumerate}
		\item $\mathbf{U}_i,\mathbf{V}_i$和之前的$i-1$对变量都不相关,即:
		\begin{equation*}
			\operatorname{Corr}(\mathbf{U}_i,\mathbf{U}_j)=\operatorname{Corr}(\mathbf{U}_i,\mathbf{V}_j)=\operatorname{Corr}(\mathbf{V}_i,\mathbf{V}_j)=\operatorname{Corr}(\mathbf{V}_i,\mathbf{U}_j)=0,\quad j=1,2,\dots,i-1
		\end{equation*}
		\item $\operatorname{Corr}(\mathbf{U}_i,\mathbf{V}_i)=\max\limits_{\alpha,\beta}\operatorname{Corr}(\alpha^T\mathbf{X},\beta^T\mathbf{Y})$。
	\end{enumerate}
	则称$\mathbf{U}_i,\mathbf{V}_i$是第$i$对\gls{CanonicalVariable},$i=1,2,\dots,p$。
\end{definition}
这一定义是否足够?
\begin{theorem}
	若不对$\mathcal{T}_1,\mathcal{T}_2$的矩阵$A,B$作出相应的限制,任何一对典型变量都不唯一。
\end{theorem}
\begin{proof}
	由相关系数的性质可知,对随机变量作线性变换不影响它们之间的线性关系,即对任意的$a,b,c,d\in\mathbb{R}^{}$,有:
	\begin{equation*}
		\operatorname{Corr}(\alpha_i^T\mathbf{X},\beta_i^T\mathbf{Y})=\operatorname{Corr}[a(\alpha_i^T\mathbf{X})+b,c(\beta_i^T\mathbf{Y})+d],\quad i=1,2,\dots,p
	\end{equation*}
	所以任何一对典型变量都不唯一。
	\info{相关系数的性质}
\end{proof}
因此我们需要对$A,B$进行相应的限制。
\begin{definition}
	设$\mathbf{X},\mathbf{Y}$分别是$m,n$维随机向量,$m+n$维随机向量$(\mathbf{X},\mathbf{Y})^T$的均值向量$\boldsymbol{\mu}=\mathbf{0}$,协方差矩阵$\Sigma$是正定阵。对$\mathbf{X},\mathbf{Y}$分别进行线性变换$\mathcal{T}_1,\mathcal{T}_2$得到两个$p$维随机向量$\mathbf{U}=\seq{\mathbf{U}}{p},\;\mathbf{V}=\seq{\mathbf{V}}{p}$,$\mathcal{T}_1$和$\mathcal{T}_2$的矩阵分别为:
	\begin{equation*}
		A=
		\begin{pmatrix}
			\alpha_1^T \\
			\alpha_2^T \\
			\vdots \\
			\alpha_p^T
		\end{pmatrix},\quad
		B=
		\begin{pmatrix}
			\beta_1^T \\
			\beta_2^T \\
			\vdots \\
			\beta_p^T
		\end{pmatrix}
	\end{equation*}
	若:
	\begin{enumerate}
		\item $\mathbf{U}_i,\mathbf{V}_i$和之前的$i-1$对变量都不相关,即:
		\begin{equation*}
			\operatorname{Corr}(\mathbf{U}_i,\mathbf{U}_j)=\operatorname{Corr}(\mathbf{U}_i,\mathbf{V}_j)=\operatorname{Corr}(\mathbf{V}_i,\mathbf{V}_j)=\operatorname{Corr}(\mathbf{V}_i,\mathbf{U}_j)=0,\quad j=1,2,\dots,i-1
		\end{equation*}
		\item $\operatorname{Corr}(\mathbf{U}_i,\mathbf{V}_i)=\max\limits_{\alpha,\beta}\operatorname{Corr}(\alpha^T\mathbf{X},\beta^T\mathbf{Y})$;
		\item $\operatorname{Var}(\mathbf{U}_i)=\operatorname{Var}(\mathbf{V}_i)=1,\;i=1,2,\dots,p$。
	\end{enumerate}
	则称$\mathbf{U}_i,\mathbf{V}_i$是第$i$对典型变量,$\operatorname{Corr}(\mathbf{U}_i,\mathbf{V}_i)$为第$i$典型相关系数,$i=1,2,\dots,p$。
\end{definition}
\begin{theorem}
	设$\mathbf{X},\mathbf{Y}$分别是$m,n$维随机向量,$m+n$维随机向量$\mathbf{Z}=(\mathbf{X},\mathbf{Y})^T$的均值向量$\boldsymbol{\mu}=\mathbf{0}$,协方差矩阵$\Sigma$是正定阵,其中:
	\begin{equation*}
		\Sigma=
		\begin{pmatrix}
			\Sigma_{11} & \Sigma_{12} \\
			\Sigma_{21} & \Sigma_{22}
		\end{pmatrix}
	\end{equation*}
	令$M=\Sigma_{11}^{-1}\Sigma_{12}\Sigma_{22}^{-1}\Sigma_{21}$,$M$的特征值从大到小为$\seq{\lambda}{m}$,对应的正交化特征向量为$\seq{a}{m}$,设:
	\begin{equation*}
		b_i=\Sigma_{22}^{-1}\Sigma_{21}a_i,\quad\alpha_i=\frac{a_i}{\sqrt{a_i^T\Sigma_{11}a_i}},\quad\beta_i=\frac{b_i}{\sqrt{b_i^T\Sigma_{22}b_i}}
	\end{equation*}
	则$\sqrt{\lambda_i^2}$为第$i$对典型相关系数,$\mathbf{U}_i=\alpha_i^T\mathbf{X},\mathbf{V}_i=\beta_i^T\mathbf{Y}$是第$i$对典型变量。
\end{theorem}
\begin{proof}
	求第一对典型相关系数的过程等价于求解下述约束优化问题:
	\begin{gather*}
		\max\operatorname{Corr}(\mathbf{U}_1,\mathbf{V_1})=\alpha_1^T\Sigma_{12}\beta_1 \\
		\operatorname{s.t.}
		\begin{cases}
			\operatorname{Var}(\mathbf{U}_1)=\alpha_1^T\Sigma_{11}\alpha_1=1 \\
			\operatorname{Var}(\mathbf{V}_1)=\beta_1^T\Sigma_{22}\beta_1=1
		\end{cases}
	\end{gather*}
	使用Lagrange乘子法进行求解,构造Lagrange函数:
	\begin{equation*}
		L(\alpha_1,\beta_1,s,t)=\alpha_1^T\Sigma_{12}\beta_1-\frac{s}{2}(\alpha_1^T\Sigma_{11}\alpha_1-1)-\frac{t}{2}(\beta_1^T\Sigma_{22}\beta_1-1)
	\end{equation*}
	利用矩阵求导可得:
	\begin{gather*}
		\frac{\partial L(\alpha_1,\beta_1,s,t)}{\partial\alpha_1}=\Sigma_{12}\beta_1-s\Sigma_{11}\alpha_1 \\
		\frac{\partial L(\alpha_1,\beta_1,s,t)}{\partial\beta_1}=\Sigma_{21}\alpha_1-t\Sigma_{22}\beta_1 \\
		\frac{\partial L(\alpha_1,\beta_1,s,t)}{\partial s}=\alpha_1^T\Sigma_{11}\alpha_1-1 \\
		\frac{\partial L(\alpha_1,\beta_1,s,t)}{\partial t}=\beta_1^T\Sigma_{22}\beta_1-1
	\end{gather*}
	令上四式全部为$0$,即:
	\begin{equation*}
		\Sigma_{12}\beta_1-s\Sigma_{11}\alpha_1=0,\quad\Sigma_{21}\alpha_1-t\Sigma_{22}\beta_1=0,\quad\alpha_1^T\Sigma_{11}\alpha_1-1=0,\quad\beta_1^T\Sigma_{22}\beta_1-1=0
	\end{equation*}
	在上第一式和第二式两边分别同乘$\alpha_1^T$和$\beta_1^T$可得$s=t$。因为$\Sigma$正定,\info{正定阵可逆}所以$\Sigma$可逆,于是$\Sigma_{11}^{-1},\Sigma_{22}^{-1}$存在。注意到:
	\begin{equation*}
		s\alpha_1=\Sigma_{11}^{-1}\Sigma_{12}\beta_1,\quad s\beta_1=\Sigma_{22}^{-1}\Sigma_{21}\alpha
	\end{equation*}
	于是有:
	\begin{equation*}
		s^2\alpha_1=\Sigma_{11}^{-1}\Sigma_{12}s\beta_1=\Sigma_{11}^{-1}\Sigma_{12}\Sigma_{22}^{-1}\Sigma_{21}\alpha_1,\quad
		s^2\beta_1=\Sigma_{22}^{-1}\Sigma_{21}s\alpha_1=\Sigma_{22}^{-1}\Sigma_{21}\Sigma_{11}^{-1}\Sigma_{12}\beta_1
	\end{equation*}
	即$s^2$是$\Sigma_{11}^{-1}\Sigma_{12}\Sigma_{22}^{-1}\Sigma_{21}$和$\Sigma_{22}^{-1}\Sigma_{21}\Sigma_{11}^{-1}\Sigma_{12}$共同的特征值。由\cref{prop:Eigenvector}(3)可知$\Sigma_{11}^{-1}\Sigma_{12}\Sigma_{22}^{-1}\Sigma_{21}$和$\Sigma_{22}^{-1}\Sigma_{21}\Sigma_{11}^{-1}\Sigma_{12}$具有相同的非零特征值。因为:
	\begin{equation*}
		\operatorname{Corr}(\mathbf{U}_1,\mathbf{V}_1)=\alpha_1^T\Sigma_{12}\beta_1=\alpha_1^Ts\Sigma_{11}\alpha_1=s\alpha_1^T\Sigma_{11}\alpha_1=s
	\end{equation*}
	所以优化问题的解为$\Sigma_{11}^{-1}\Sigma_{12}\Sigma_{22}^{-1}\Sigma_{21}$和$\Sigma_{22}^{-1}\Sigma_{21}\Sigma_{11}^{-1}\Sigma_{12}$的最大特征值$s^2$、$\Sigma_{11}^{-1}\Sigma_{12}\Sigma_{22}^{-1}\Sigma_{21}$的特征值$s^2$对应的满足$\alpha_1^T\Sigma_{11}\alpha_1=1$的特征向量$\alpha_1$和$\Sigma_{22}^{-1}\Sigma_{21}\Sigma_{11}^{-1}\Sigma_{12}$的特征值$s^2$对应的满足$\beta_1^T\Sigma_{22}\beta_1=1$的特征向量$\beta_1$。显然结论对第$1$对典型变量成立。\par
	考虑第$i$对典型变量,此时的优化问题为:
	\begin{gather*}
		\max\operatorname{Corr}(\mathbf{U}_i,\mathbf{V}_i)=\alpha_i^T\Sigma_{12}\beta_i \\
		\operatorname{s.t.}
		\begin{cases}
			\operatorname{Var}(\mathbf{U}_i)=\alpha_i^T\Sigma_{11}\alpha_i=1 \\
			\operatorname{Var}(\mathbf{V}_i)=\beta_i^T\Sigma_{22}\beta_i=1 \\
			\operatorname{Cov}(\mathbf{U}_i,\mathbf{U}_j)=\alpha_i^T\Sigma_{11}\alpha_j=0,&j=1,2,\dots,i-1 \\
			\operatorname{Cov}(\mathbf{V}_i,\mathbf{V}_j)\beta_i^T\Sigma_{12}\beta_j=0,&j=1,2,\dots,i-1 \\
			\operatorname{Cov}(\mathbf{U}_i,\mathbf{V_j})=\alpha_i^T\Sigma_{12}\beta_j=0,&j=1,2,\dots,i-1 \\
			\operatorname{Cov}(\mathbf{V}_i,\mathbf{U}_j)=\beta_i^T\Sigma_{21}\alpha_j=0,&j=1,2,\dots,i-1
		\end{cases}
	\end{gather*}
	因为$\Sigma$正定,所以$\Sigma_{11}$和$\Sigma_{22}$正定,即存在$\Sigma_{11}^{-\frac{1}{2}}$和$\Sigma_{22}^{-\frac{1}{2}}$。令:
	\begin{equation*}
		\gamma=\Sigma_{11}^{\frac{1}{2}}\alpha_i,\quad\delta=\Sigma_{22}^{\frac{1}{2}}\beta_i,\quad\gamma_j=\Sigma_{11}^{\frac{1}{2}}\alpha_j,\quad\delta_j=\Sigma_{22}^{\frac{1}{2}}\beta_j,\quad j=1,2,\dots,i-1
	\end{equation*}
	同时$\gamma$和$\alpha_i$、$\delta$和$\beta_i$、$\gamma_j$和$\alpha_j$、$\delta_j$和$\beta_j$都是一一对应的关系。由\cref{prop:ReverseSquareRootMat}(3),此时约束条件可改写为:
	\begin{equation*}
		\begin{cases}
			\operatorname{Var}(\mathbf{U}_i)=\alpha_i^T\Sigma_{11}\alpha_i=||\gamma||=1 \\
			\operatorname{Var}(\mathbf{V}_i)=\beta_i^T\Sigma_{22}\beta_i=||\delta||=1 \\
			\operatorname{Cov}(\mathbf{U}_i,\mathbf{U}_j)=\alpha_i^T\Sigma_{11}\alpha_j=\gamma^T\gamma_j=0 \\
			\operatorname{Cov}(\mathbf{V}_i,\mathbf{V}_j)=\beta_i^T\Sigma_{12}\beta_j=\delta^T\delta_j=0 \\
			\operatorname{Cov}(\mathbf{U}_i,\mathbf{V_j})=\alpha_i^T\Sigma_{12}\beta_j=(\Sigma_{11}^{-\frac{1}{2}}\gamma)^T\Sigma_{12}\Sigma_{22}^{-\frac{1}{2}}\delta_j=\gamma^T\Sigma_{11}^{-\frac{1}{2}}\Sigma_{12}\Sigma_{22}^{-\frac{1}{2}}\delta_j=0 \\
			\operatorname{Cov}(\mathbf{V}_i,\mathbf{U}_j)=\beta_i^T\Sigma_{21}\alpha_j=(\Sigma_{22}^{-\frac{1}{2}}\delta)^T\Sigma_{21}(\Sigma_{11}^{-\frac{1}{2}}\gamma_j)=\delta^T\Sigma_{22}^{-\frac{1}{2}}\Sigma_{21}\Sigma_{11}^{-\frac{1}{2}}\gamma_j=0
		\end{cases}
	\end{equation*}
	令$K=\Sigma_{11}^{-\frac{1}{2}}\Sigma_{12}\Sigma_{22}^{-\frac{1}{2}}$,则最后两个约束条件可改写为$\gamma^TK\delta_j=0,\;\delta^TK^T\gamma_j=0,\;j=1,2,\dots,i-1$,目标函数可改写为$\alpha_i^T\Sigma_{12}\beta_i=(\Sigma_{11}^{-\frac{1}{2}}\gamma)^T\Sigma_{12}(\Sigma_{22}^{-\frac{1}{2}}\delta)=\gamma^T\Sigma_{11}^{-\frac{1}{2}}\Sigma_{12}\Sigma_{22}^{-\frac{1}{2}}\delta=\gamma^TK\delta$。由\cref{ineq:cauchy-schiwarz-inner-product}可得:
	\begin{equation*}
		|\gamma^TK\delta|\leqslant||K^T\gamma||||\delta||
	\end{equation*}
	由取等条件和第二个约束条件可知固定$\gamma$情况下$\delta$的最优取值为$\dfrac{K^T\gamma}{||K^T\gamma||}$\info{未完成}
\end{proof}