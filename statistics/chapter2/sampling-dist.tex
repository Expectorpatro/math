\section{抽样分布}

\begin{theorem}
	设$\seq{X}{m}\;\text{i.i.d.}\sim N(\mu,\sigma^2)$,$\seq{Y}{n}\;\text{i.i.d.}\sim N(\nu,\sigma^2)$,$\seq{X}{m}$和$\seq{Y}{n}$相互独立,$\overline{X},\overline{Y}$为样本均值,$S_X^2,S_Y^2$为样本方差,则:
	\begin{enumerate}
		\item $\overline{X}\sim N\left(\mu,\dfrac{\sigma^2}{m}\right)$;
		\item $\dfrac{(m-1)S_X^2}{\sigma^2}\sim\chi_{m-1}^2$;
		\item $\overline{X}$与$S_X^2$独立;、
		\item $\dfrac{\sqrt{m}(\overline{X}-\mu)}{S_X}\sim t_{m-1}$;
		\item $\dfrac{\overline{X}-\overline{Y}-(\mu-\nu)}{S_w}\sqrt{\dfrac{mn}{m+n}}\sim t_{m+n-2}$,其中:
		\begin{equation*}
			(m+n-2)S_w^2=(m-1)S_X^2+(n-1)S_Y^2
		\end{equation*}
		\item 若$\seq{X}{m}\;\text{i.i.d.}\sim N(\mu,\sigma_1^2)$,$\seq{Y}{n}\;\text{i.i.d.}\sim N(\nu,\sigma_2^2)$,其它条件不变,则:
		\begin{equation*}
			\frac{S_X^2\sigma_2^2}{S_Y^2\sigma_1^2}\sim F_{m-1,n-1}
		\end{equation*}
	\end{enumerate}
\end{theorem}
\begin{proof}
	令$\mathbf{X}=(\seq{X}{m})^T,\;\mathbf{Y}=(\seq{Y}{n})^T$。因为$\seq{X}{m}\;\text{i.i.d.}\sim N(\mu,\sigma^2)$,$\seq{Y}{n}\;\text{i.i.d.}\sim N(\nu,\sigma^2)$,所以$\mathbf{X}\sim N_m(\boldsymbol{\mu},\Sigma_m),\;\mathbf{Y}\sim N_n(\boldsymbol{\nu},\Sigma_n)$,其中:
	\begin{equation*}
		\boldsymbol{\mu}=
		\begin{pmatrix}
			\mu \\
			\mu \\
			\vdots \\
			\mu
		\end{pmatrix},\;
		\Sigma_m=
		\begin{pmatrix}
			\sigma^2 & 0 & \cdots & 0 \\
			0 & \sigma^2 & \cdots & 0 \\
			\vdots & \vdots & \ddots & 0 \\
			0 & 0 & \cdots & \sigma^2
		\end{pmatrix},\;
			\boldsymbol{\nu}=
		\begin{pmatrix}
			\nu \\
			\nu \\
			\vdots \\
			\nu
		\end{pmatrix},\;
		\Sigma_n=
		\begin{pmatrix}
			\sigma^2 & 0 & \cdots & 0 \\
			0 & \sigma^2 & \cdots & 0 \\
			\vdots & \vdots & \ddots & 0 \\
			0 & 0 & \cdots & \sigma^2
		\end{pmatrix}
	\end{equation*}
	\par
	(1)令$m$维行向量$c=\left(\dfrac{1}{n},\dfrac{1}{n},\dots,\dfrac{1}{n}\right)$,
	由\cref{theo:MultiNormalLinearTransform}可知:
	\begin{equation*}
		\overline{X}=c\mathbf{X}\sim N(c\boldsymbol{\mu},c\Sigma c^T)
	\end{equation*}
	而:
	\begin{equation*}
		c\boldsymbol{\mu}=\sum_{i=1}^{m}\frac{\mu}{m}=\mu,\;c\Sigma c^T=\sum_{i=1}^{m}\frac{\sigma^2}{m^2}=\frac{\sigma^2}{m}
	\end{equation*}
	所以$\overline{X}\sim N\left(\mu,\dfrac{\sigma^2}{m}\right)$。\par
	(2)由Schmidit正交化\info{考虑链接什么过来}可知存在正交矩阵:
	\begin{equation*}
		A=
		\begin{pmatrix}
			\frac{1}{\sqrt{m}} & \frac{1}{\sqrt{m}} & \cdots & \frac{1}{\sqrt{m}} \\
			a_{21} & a_{22} & \cdots & a_{2m} \\
			\vdots & \vdots & \ddots & \vdots \\
			a_{m1} & a_{m2} & \cdots & a_{mm}
		\end{pmatrix}
	\end{equation*}
	令$\mathbf{Z}=A\mathbf{X}$,由\cref{cor:MultiNormalLinearTransform}(2)可知$\mathbf{Z}\sim N_m(A\boldsymbol{\mu},\sigma^2I_m)$。由\cref{cor:MultiNormalLinearTransform}(4)可知$\mathbf{Z}_i\sim N(\mu_i,\sigma^2)$,其中:
	\begin{equation*}
		\mu_i=\mu\sum_{j=1}^{m}a_{ij}
	\end{equation*}
	因为$A$是一个正交矩阵,所以:
	\begin{equation*}
		\mu_i=\sqrt{m}\mu\sum_{j=1}^{m}\frac{1}{\sqrt{m}}a_{ij}=\sqrt{m}\mu\left(\dfrac{1}{\sqrt{m}},\dfrac{1}{\sqrt{m}},\dots,\dfrac{1}{\sqrt{m}}\right)(a_{i1},a_{i2},\dots,a_{im})^T=0
	\end{equation*}
	由\cref{theo:MultiNormalLinearTransform}即可得:
	\begin{equation*}
		\frac{\mathbf{Z}_i}{\sigma}\sim N(0,1),\;\forall\;i=1,2,\dots,m
	\end{equation*}
	因为$\mathbf{Z}=A\mathbf{X}$,所以:
	\begin{equation*}
		\mathbf{Z}_1=\frac{1}{\sqrt{m}}\sum_{i=1}^{m}X_i=\sqrt{m}\overline{X}
	\end{equation*}
	因为:
	\begin{equation*}
		\mathbf{Z}^T\mathbf{Z}=\sum_{i=1}^{m}\mathbf{Z}_i^2=\mathbf{X}^TA^TA\mathbf{X}=\mathbf{X}^T\mathbf{X}=\sum_{i=1}^{m}X_i^2
	\end{equation*}
	于是:
	\begin{equation*}
		(m-1)S_X^2=\sum_{i=1}^{m}(X_i-\overline{X})^2=\sum_{i=1}^{m}X_i^2-m\overline{X}^2=\sum_{i=1}^{m}\mathbf{Z}_i^2-\mathbf{Z}_1^2=\sum_{i=2}^{m}\mathbf{Z}_i^2
	\end{equation*}
	所以:
	\begin{equation*}
		\frac{(m-1)S_X^2}{\sigma^2}=\sum_{i=2}^{m}\left(\frac{\mathbf{Z}_i}{\sigma}\right)^2\sim\chi_{m-1}^2
	\end{equation*}\par
	(3)由(2)的证明过程可得$\mathbf{Z}\sim N_m(A\boldsymbol{\mu},\sigma^2I_m)$,根据\cref{theo:IndependentCorrelationNormal}可知$\seq{\mathbf{Z}}{m}$相互独立。而:
	\begin{equation*}
		S_X^2=\frac{\sum\limits_{i=2}^{m}\mathbf{Z}_i^2}{(m-1)},\;\overline{X}=\frac{\mathbf{Z}_1}{\sqrt{m}}
	\end{equation*}
	所以$S_X^2$与$\overline{X}$独立。\par
	(4)对$\overline{X}$进行标准化可得:
	\begin{equation*}
		\frac{\overline{X}-\mu}{\sqrt{\frac{\sigma^2}{m}}}=\frac{\sqrt{m}(\overline{X}-\mu)}{\sigma}\sim N(0,1)
	\end{equation*}
	由(2)和(3)进一步可得:
	\begin{equation*}
		\frac{\dfrac{\sqrt{m}(\overline{X}-\mu)}{\sigma}}{\sqrt{\dfrac{(m-1)S_X^2}{\sigma^2(m-1)}}}=\frac{\sqrt{m}(\overline{X}-\mu)}{S_X}\sim t_{m-1}
	\end{equation*}\par
	(5)由(1)(得到$\overline{X}$和$\overline{Y}$的分布)、$\seq{X}{m}$与$\seq{Y}{n}$相互独立(得到二维随机向量$(\overline{X},\overline{Y})$的分布)和\cref{theo:MultiNormalLinearTransform}(对$(\overline{X},\overline{Y})$用二维行向量$(1,-1)$做线性变换)可得:
	\begin{equation*}
		\overline{X}-\overline{Y}\sim N\left(\mu-\nu,\frac{\sigma^2}{m}+\frac{\sigma^2}{n}\right)
	\end{equation*}
	于是:
	\begin{equation*}
		\frac{\overline{X}-\overline{Y}-(\mu-\nu)}{\sqrt{\dfrac{m+n}{mn}\sigma^2}}\sim N(0,1)
	\end{equation*}
	由(2)可得:
	\begin{equation*}
		\frac{(m-1)S_X^2}{\sigma^2}\sim\chi_{m-1}^2,\;
		\frac{(n-1)S_Y^2}{\sigma^2}\sim\chi_{n-1}^2
	\end{equation*}
	由\cref{prop:Chi2Distribution}(1)可得:
	\begin{equation*}
		\frac{(m-1)S_X^2+(n-1)S_Y^2}{\sigma^2}\sim\chi_{m+n-2}^2
	\end{equation*}
	于是:
	\begin{equation*}
		\frac{(m+n-2)S_w^2}{\sigma^2}\sim\chi_{m+n-2}^2
	\end{equation*}
	由(3)可得$\overline{X}$与$S_X^2$独立、$\overline{Y}$与$S_Y^2$独立,所以:
	\begin{equation*}
		\frac{\dfrac{\overline{X}-\overline{Y}-(\mu-\nu)}{\sqrt{\dfrac{m+n}{mn}\sigma^2}}}{\sqrt{\dfrac{(m+n-2)S_w^2}{\sigma^2(m+n-2)}}}=\frac{\overline{X}-\overline{Y}-(\mu-\nu)}{S_w}\sqrt{\dfrac{mn}{m+n}}\sim t_{m+n-2}
	\end{equation*}\par
	(6)由(2)可知:
	\begin{equation*}
		\frac{(m-1)S_X^2}{\sigma_1^2}\sim\chi_{m-1}^2,\;
		\frac{(n-1)S_Y^2}{\sigma_2^2}\sim\chi_{n-1}^2
	\end{equation*}
	因为$\seq{X}{m}$和$\seq{Y}{n}$相互独立,所以上两式也相互独立。由$F$分布的定义即可得:
	\begin{equation*}
		\frac{\dfrac{(m-1)S_X^2}{\sigma_1^2(m-1)}}{\dfrac{(n-1)S_Y^2}{\sigma_2^2(n-1)}}=\frac{S_X^2\sigma_2^2}{S_Y^2\sigma_1^2}\sim F_{m-1,n-1}\qedhere
	\end{equation*}
\end{proof}