\section{次序统计量}

\begin{definition}
	设$\seq{X}{n}$为从总体中抽取的样本,将其按大小排列为$X_{(1)}\leqslant X_{(2)}\leqslant\cdots\leqslant X_{(n)}$,称$(X_{(1)},X_{(2)},\dots,X_{(n)})$为样本$\seq{X}{n}$的\gls{OrderStatistics}。
\end{definition}

\subsubsection{次序统计量的联合分布}
\begin{theorem}\label{theo:OrderStatisticsDist}
	设总体的密度函数为$f(x),\;x\in\mathbb{R}$,$\seq{X}{n}$为从总体中抽取的简单样本。令$Y_i=X_{(i)},\;i=1,2,\dots,n$,则次序统计量$(\seq{Y}{n})$的联合密度为:
	\begin{equation*}
		g(\seq{y}{n})=
		\begin{cases}
			n!f(y_1)f(y_2)\cdots f(y_n),&y_1<y_2<\cdots<y_n \\
			0,&\text{其它}
		\end{cases}
	\end{equation*}
\end{theorem}
\begin{proof}
	在$\mathbb{R}^{n}$中划分$n!$个区域,每个区域分别对应着一个$\seq{i}{n}$使得$x_{i1}<x_{i2}<\cdots<x_{in}$,因为$\seq{x}{n}$的排列一共有$n!$种,所以这$n!$个区域加上包括等于号的一些零测集就构成了整个$\mathbb{R}^{n}$。因为次序统计量的密度函数也是在$\mathbb{R}^{n}$上的一个概率测度,则可以对每个划分的区域求$(\seq{Y}{n})$的概率测度,再对所有区域求和,即可得到次序统计量的联合密度。这个过程类似于全概率公式。\par
	任取一个上述区域$A$作变换:
	\begin{equation*}
		y_j=x_{i_j},\;j=1,2,\dots,n,\;x_{i1}<x_{i2}<\cdots<x_{in}
	\end{equation*}
	则该变换的Jacobi行列式为$|\mathbf{J}|=|I_n|=1$,因为$\seq{X}{n}$是简单样本,所以在该区域上的:
	\begin{equation*}
		g(\seq{y}{n}|A)=
		\begin{cases}
			\prod\limits_{i=1}^{n}f(x_i)=\prod\limits_{i=1}^{n}f(y_i),&y_1<y_2<\cdots<y_n \\
			0,&\text{其它}
		\end{cases}
	\end{equation*}
	由区域的任意性可得在整个$\mathbb{R}^{n}$上:
	\begin{equation*}
		g(\seq{y}{n})=
		\begin{cases}
			n!f(y_1)f(y_2)\cdots f(y_n),&y_1<y_2<\cdots<y_n \\
			0,&\text{其它}
		\end{cases}\qedhere
	\end{equation*}
\end{proof}

\subsubsection{部分次序统计量的联合分布}
\begin{lemma}\label{lem:OrderStatistics}
	设总体的分布函数为$F(x)$,密度函数为$f(x)$,$\seq{X}{n}$为从总体中抽取的简单样本,则有:
	\begin{equation*}
		\underset{a<x_1<\cdots<x_n<b}{\int\cdots\int}f(x_1)\cdots f(x_n)\dif x_1\cdots\dif x_n=\frac{1}{n!}[F(b)-F(a)]^n
	\end{equation*}
	其中$a,b\in\overline{\mathbb{R}}$。
\end{lemma}
\begin{proof}
	因为$\seq{X}{n}$独立同分布,所以:
	\begin{equation*}
		\int_{a}^{b}\cdots\int_{a}^{b}f(x_1)\cdots f(x_n)\dif x_1\cdots\dif x_n=\left[\int_{a}^{b}f(x_1)\dif x_1\right]^n=[F(b)-F(a)]^n
	\end{equation*}
	在这个区域上对$\seq{x}{n}$的排序结果一共有$n!$种(不考虑等于的情况,测度为$0$,不影响积分结果),每种排序都是等可能的,于是结论成立。
\end{proof}
\begin{theorem}\label{theo:PartialOrderStatisticsDist}
	设$\seq{X}{n}$为从总体中抽取的简单样本,总体的分布函数和密度函数分别为$F(x),f(x)$,则样本次序统计量中任意$m$个分量$Y_{(i_1)},Y_{(i_2)},\dots,Y_{(i_m)},\;i_1<i_2<\cdots<i_m$的联合密度为:
	\begin{align*}
		g(y_{i_1},y_{i_2},\dots,y_{i_m})
		&=n!\prod_{j=1}^{m}f(y_{i_j})\frac{1}{(i_1-1)!}F^{i_1-1}(y_{i_1})\frac{1}{(n-i_m)!}[1-F(y_{i_m})]^{n-i_m} \\
		&\quad\left\{\prod_{k=2}^{m}\frac{1}{(i_k-i_{k-1}-1)!}[F(y_{i_k})-F(y_{i_{k-1}})]^{i_k-i_{k-1}-1}\right\}
	\end{align*}
\end{theorem}
\begin{proof}
	注意到$Y_{(i_1)},Y_{(i_2)},\dots,Y_{(i_m)}$的联合密度是次序统计量的边缘密度,所以由\cref{lem:OrderStatistics}可得:
	\begin{align*}
		g(y_{i_1},y_{i_2},\dots,y_{i_m})
		&=\underset{-\infty<y_1<\cdots<y_n<+\infty}{\int\cdots\int}n!f(y_1)f(y_2)\cdots f(y_n)\dif y_1\cdots\dif y_{i_1-1} \\
		&\quad\dif y_{i_1+1}\cdots\dif y_{i_2-1}\dif y_{i_2+1}\cdots\dif y_{i_m-1}\dif y_{i_m+1}\cdots\dif y_n \\
		&=n!\prod_{j=1}^{m}f(y_{i_j})\underset{-\infty<y_1<\cdots<y_{i_1}}{\int\cdots\int}f(y_1)\cdots f(y_{i_1-1})\dif y_1\cdots\dif y_{i_1-1} \\
		&\quad\times\underset{y_{i_1}<y_{i_1+1}<y_{i_1+2}<\cdots<y_{i_2}}{\int\cdots\int}f(y_{i_1+1})\cdots f(y_{i_2-1})\dif y_{i_1+1}\cdots\dif y_{i_2-1} \\
		&\quad\cdots\cdots \\
		&\quad\times\underset{y_{i_m}<y_{i_m+1}<y_{i_m+2}<\cdots<+\infty}{\int\cdots\int}f(y_{i_m+1})\cdots f(y_n)\dif y_{i_m+1}\cdots\dif y_n \\
		&=n!\prod_{j=1}^{m}f(y_{i_j})\frac{1}{(i_1-1)!}F^{i_1-1}(y_{i_1})\frac{1}{(i_2-i_1-1)!}[F(y_{i_2})-F(y_{i_1})]^{i_2-i_1-1} \\
		&\quad\cdots\frac{1}{(n-i_m)!}[1-F(y_{i_m})]^{n-i_m} \\
		&=n!\prod_{j=1}^{m}f(y_{i_j})\frac{1}{(i_1-1)!}F^{i_1-1}(y_{i_1})\frac{1}{(n-i_m)!}[1-F(y_{i_m})]^{n-i_m} \\
		&\quad\left\{\prod_{k=2}^{m}\frac{1}{(i_k-i_{k-1}-1)!}[F(y_{i_k})-F(y_{i_{k-1}})]^{i_k-i_{k-1}-1}\right\}\qedhere
	\end{align*}
\end{proof}

\subsubsection{极差的分布}
\begin{theorem}
	设$\seq{X}{n}$为从总体中抽取的简单样本,总体的分布函数和密度函数分别为$F(x),f(x)$,样本的次序统计量为$(\seq{Y}{n})$,则对于任意的$i,j=1,2,\dots,n$满足$i<j$,令$V=Y_j-Y_i$,则有:
	\begin{equation*}
		g(v)=\int_{-\infty}^{+\infty}g(u,v)\dif u
	\end{equation*}
	其中:
	\begin{equation*}
		g(u,v)=
		\begin{cases}
			\dfrac{n!f(u)f(u+v)}{(i-1)!(j-i-1)!(n-j)!}F^{i-1}(u)[F(u+v)-F(u)]^{j-i-1} \\
			\quad\quad[1-F(u+v)]^{n-j},&v>0 \\
			0,&\text{其它}
		\end{cases}
	\end{equation*}
\end{theorem}
\begin{proof}
	使用增补变量法\info{链接随机变量函数的分布中的增补变量法},做变换:
	\begin{equation*}
		\begin{cases}
			U=Y_i \\
			V=Y_j-Y_i
		\end{cases}
		\Leftrightarrow
		\begin{cases}
			Y_i=U \\
			Y_j=V+U
		\end{cases}
	\end{equation*}
	该变换的Jacobi行列式为:
	\begin{equation*}
		|\mathbf{J}|=
		\begin{vmatrix}
			1 & 0 \\
			1 & 1
		\end{vmatrix}
		=1
	\end{equation*}
	由\cref{theo:PartialOrderStatisticsDist}可得$(Y_i,Y_j)$的联合密度:
	\begin{equation*}
		g(y_i,y_j)=
		\begin{cases}
			\dfrac{n!f(y_i)f(y_j)}{(i-1)!(j-i-1)!(n-j)!}F^{i-1}(y_i)[F(y_j)-F(y_i)]^{j-i-1} \\
			\quad\quad[1-F(y_j)]^{n-j},&y_i<y_j \\
			0,&\text{其它}
		\end{cases}
	\end{equation*}
	于是:
	\begin{equation*}
		g(u,v)=
		\begin{cases}
			\dfrac{n!f(u)f(u+v)}{(i-1)!(j-i-1)!(n-j)!}F^{i-1}(u)[F(u+v)-F(u)]^{j-i-1} \\
			\quad\quad[1-F(u+v)]^{n-j},&v>0 \\
			0,&\text{其它}
		\end{cases}
	\end{equation*}
	所以:
	\begin{equation*}
		g(v)=\int_{-\infty}^{+\infty}g(u,v)\dif u\qedhere
	\end{equation*}
\end{proof}
\begin{corollary}
	次序统计量$(\seq{Y}{n})$极差的分布为:
	\begin{equation*}
		g(v)=\int_{-\infty}^{+\infty}g(u,v)\dif u
	\end{equation*}
	其中:
	\begin{equation*}
		g(u,v)=
		\begin{cases}
			n(n-1)f(u)f(u+v)[F(u+v)-F(u)]^{n-2},&v>0 \\
			0,&\text{其它}
		\end{cases}
	\end{equation*}
\end{corollary}