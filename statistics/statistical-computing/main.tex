\chapter{统计计算方法}

\section{非线性方程的求解}
\subsubsection{二分法}
\begin{method}[Bisection Method]
	设函数$f$在$[a,b]$上连续,且满足$f(a)f(b)<0$。为求解非线性方程$f(x)=0$,由\cref{lem:IntermediateValueR}可知,存在至少一个根$x^*\in(a,b)$。二分法通过反复对区间进行二等分来逐步缩小根所在的区间:令:
	\begin{equation*}
		c=\frac{a+b}{2}
	\end{equation*}
	若$f(c)=0$,则$c$即为方程的解;否则根据符号判定准则,取满足:
	\begin{equation*}
		f(a)f(c)<0\quad\text{或}\quad f(c)f(b)<0
	\end{equation*}
	的子区间作为新的区间$[a,b]$,并重复上述过程。\par
	经过$n$次迭代后,所得区间长度不超过$(b-a)/2^n$,任意迭代点$x_n$与真根$x^*$的误差满足:
	\begin{equation*}
		|x_n-x^*|\leqslant\frac{b-a}{2^n}
	\end{equation*}
	因此,可通过给定误差容许值$\varepsilon$,选取满足:
	\begin{equation*}
		\frac{b-a}{2^n}\leqslant\varepsilon
	\end{equation*}
	的最小整数$n$来控制计算精度。
\end{method}
\subsubsection{不动点迭代法}
\begin{method}[Fixed-point Iteration Method]
	为求解非线性方程$f(x)=0$,可将其改写为等价的形式$x=T(x)$,若$T$是一个压缩映射,由\cref{theo:ContractionMapTheorem}可求得非线性方程的根。对于迭代次数问题,可用\cref{theo:ContractionMapTheorem}中的第二种误差估计公式来控制精度。
\end{method}
\begin{definition}
	设映射$T$有不动点$x^*$。若存在$x^*$的邻域$\bar{U}(x^*,\delta)$满足对任意的$x\in\bar{U}(x^*,\delta)$,序列$\{x_n=T^nx\}\to x^*$,则称此时的迭代法\gls{LocalConvergence}。
\end{definition}
\begin{theorem}\label{theo:FixedPointLocalConvergence}
	设映射$T$有不动点$x^*$,$T'$在$x^*$的某个邻域内连续,且$|T'x^*|<1$,则此时的迭代法局部收敛。
\end{theorem}
\begin{proof}
	由条件和\cref{prop:RSeq}(5)可知存在$x^*$的邻域$U$和$L$,使得对任意的$x\in U$有$|T'x|\leqslant L<1$。根据\cref{theo:LagrangeMeanValueTheorem}可知对任意的$x'\in U$有:
	\begin{equation*}
		|Tx-Tx^*|\leqslant L|x-x^*|
	\end{equation*}
	即$T$在$x^*$的某个闭邻域内是压缩映射,由\cref{theo:RnComplete}、\cref{prop:CompleteMetricSpace}(1)和\cref{theo:ContractionMapTheorem}可知此时的迭代法局部收敛。
\end{proof}
\begin{definition}
	设迭代过程$x_{n+1}=Tx_n$收敛于$T$的不动点$x^*$。令$\varepsilon_n=x_n-x^*$,若有:
	\begin{equation*}
		\lim_{n\to+\infty}\frac{\varepsilon_{n+1}}{\varepsilon_n^p}=C\in\mathbb{R}^{}\backslash\{0\}
	\end{equation*}
	则称该迭代过程是$p$阶收敛的。特别的,$p=1$且$|C|<1$时称为\gls{LinearConvergence},$p>1$时称为\gls{SuperlinearConvergence}。
\end{definition}
\begin{theorem}\label{theo:FixedPointOrderPConvergence}
	设迭代过程$x_{n+1}=Tx_n$。若对于$p\in\mathbb{N}^+$,$T^{(p)}$在$T$的不动点$x^*$的某个邻域内连续,并且有:
	\begin{equation*}
		T'x^*=T''x^*=\cdots=T^{(p-1)}x^*=0,\;T^{(p)}x^*\ne0
	\end{equation*}
	则该迭代过程在$x^*$邻近是$p$阶收敛的。
\end{theorem}
\begin{proof}
	由$T'x^*=0$和\cref{theo:FixedPointLocalConvergence}可知该迭代过程是局部收敛的。根据\info{泰勒展开Lagrange余项}可得:
	\begin{equation*}
		Tx_n=Tx^*+\frac{T^{(p)}\xi_n}{p!}(x_n-x^*)^p,\quad\xi_n\text{在$x_n$与$x^*$之间}
	\end{equation*}
	所以:
	\begin{equation*}
		\frac{Tx_n-x^*}{(x_n-x^*)^p}=\frac{T^{(p)}\xi_n}{p!}
	\end{equation*}
	由$T^{(p)}$在$x^*$某个邻域内的连续性和\cref{prop:RSeq}(8.c)(4.a)可得:
	\begin{equation*}
		\lim_{n\to+\infty}\frac{Tx_n-x^*}{(x_n-x^*)^p}=\lim_{n\to+\infty}\frac{T^{(p)}\xi_n}{p!}=\frac{1}{p!}T^{(p)}\left(\lim_{n\to+\infty}\xi_n\right)=\frac{T^{(p)}x^*}{p!}\qedhere
	\end{equation*}
\end{proof}
\subsubsection{牛顿法}
\begin{method}[Newton Method]
	设$f$在根$x^*$的某个邻域内是$C^2$类函数,且满足$f(x^*)=0,\;f'(x^*)\ne0$。为求解非线性方程$f(x)=0$,牛顿法对$f$在当前迭代点$x_n$处作一阶Taylor展开,得到:
	\begin{equation*}
		f(x)\approx f(x_n)+f'(x_n)(x-x_n)
	\end{equation*}
	令上式为$0$得到迭代格式为:
	\begin{equation*}
		x_{n+1}=x_n-\frac{f(x_n)}{f'(x_n)}
	\end{equation*}\par
	牛顿法的迭代函数为:
	\begin{equation*}
		Tx=x-\frac{f(x)}{f'(x)}
	\end{equation*}
	于是有:
	\begin{equation*}
		T'x=1-\frac{[f'(x)]^2-f(x)f''(x)}{[f'(x)]^2}=\frac{f(x)f''(x)}{[f'(x)]^2}
	\end{equation*}
	根据\cref{theo:FixedPointOrderPConvergence}可知牛顿法至少是二阶收敛的。
\end{method}


\paragraph{设定与记号}
固定 \(a>0\)。定义迭代
\[
x_{n+1}=\tfrac12\!\left(x_n+\frac{a}{x_n}\right),\qquad n=0,1,2,\dots
\]
设真值 \(r=\sqrt a>0\),误差 \(e_n:=x_n-r\)。

\begin{theorem}[Heron 法是 Newton--Raphson 的特例]
	令 \(f(x)=x^2-a\)。对方程 \(f(x)=0\) 应用 Newton 法
	\[
	x_{n+1}=x_n-\frac{f(x_n)}{f'(x_n)}=x_n-\frac{x_n^2-a}{2x_n}
	=\frac12\!\left(x_n+\frac{a}{x_n}\right),
	\]
	即为所给迭代。
\end{theorem}

\begin{theorem}[不变性、单调性与收敛]
	若 \(x_0>0\),则对所有 \(n\) 都有 \(x_n>0\)。进一步:
	\begin{enumerate}
		\item 若 \(x_0>r\),则 \(x_{n+1}\in(r,x_n)\),序列严格单调递减并下有界,极限为 \(r\)。
		\item 若 \(0<x_0<r\),则 \(x_{n+1}\in(x_n,r)\),序列严格单调递增并上有界,极限为 \(r\)。
	\end{enumerate}
\end{theorem}

\begin{proof}
	首先 \(x_n>0\Rightarrow x_{n+1}=\frac12(x_n+a/x_n)>0\),故正性不变。
	
	设 \(x_n>r\)。则
	\[
	x_{n+1}-r=\frac{x_n^2+a}{2x_n}-r
	=\frac{x_n^2-2rx_n+r^2}{2x_n}=\frac{(x_n-r)^2}{2x_n}>0,
	\]
	且
	\[
	x_n-x_{n+1}=x_n-\frac12\!\left(x_n+\frac{a}{x_n}\right)
	=\frac{x_n^2-a}{2x_n}=\frac{(x_n-r)(x_n+r)}{2x_n}>0,
	\]
	故 \(x_{n+1}\in(r,x_n)\) 并单调递减。由下界 \(r\) 与单调性知收敛,极限必为 \(f\) 的正根 \(r\)。
	
	当 \(0<x_n<r\) 时同理得
	\[
	x_{n+1}-r=\frac{(x_n-r)^2}{2x_n}<0,\qquad
	x_{n+1}-x_n=\frac{a-x_n^2}{2x_n}>0,
	\]
	故 \(x_{n+1}\in(x_n,r)\) 并单调递增,极限同为 \(r\)。
\end{proof}

\begin{theorem}[误差精确递推与二次收敛]
	令 \(e_n=x_n-r\)。对任意 \(x_n>0\) 有
	\[
	e_{n+1}=x_{n+1}-r=\frac{(x_n-r)^2}{2x_n}=\frac{e_n^2}{2x_n}.
	\]
	因此当 \(x_n\to r\) 时
	\[
	e_{n+1}=\frac{e_n^2}{2r}\,(1+o(1)),
	\]
	即误差二次收敛,渐近误差常数为 \(1/(2r)\)。
\end{theorem}

\begin{proof}
	代数恒等式直接给出
	\(
	x_{n+1}-r=\frac{(x_n-r)^2}{2x_n}.
	\)
	当 \(x_n\to r>0\) 时,\(\frac{1}{2x_n}\to \frac{1}{2r}\),得所述渐近式。
\end{proof}

\begin{remark}[相对误差形式]
	相对误差 \(\delta_n:=\frac{x_n-r}{r}\) 满足
	\[
	\delta_{n+1}=\frac{\delta_n^2}{2(1+\delta_n)}=\frac12\delta_n^2+O(\delta_n^3),
	\]
	同样显示二次收敛。
\end{remark}

\begin{theorem}[倒数平方根的牛顿迭代与一次修正]
	为求 \(y=1/\sqrt a\),考虑
	\(
	\phi(y)=y^{-2}-a=0
	\)。
	Newton 法给出
	\[
	y_{k+1}=y_k-\frac{\phi(y_k)}{\phi'(y_k)}
	=y_k-\frac{y_k^{-2}-a}{-2y_k^{-3}}
	=y_k\!\left(\frac32-\frac{a y_k^2}{2}\right).
	\]
	设初值 \(y_0\) 来自硬件 \texttt{rsqrt} 指令(粗近似)。则该迭代二次收敛到 \(1/\sqrt a\)。
	得到 \(y_m\) 后,可令
	\(
	x=a\,y_m
	\)
	作为 \(\sqrt a\) 的近似;若需再提高精度,对 \(x\) 再做一次 Heron 步
	\[
	x\ \leftarrow\ \frac12\!\left(x+\frac{a}{x}\right),
	\]
	误差继续二次缩小。
\end{theorem}

\begin{proof}
	\(\phi'(y)=-2y^{-3}\neq0\) 于 \(y>0\)。标准 Newton 收敛定理表明在真根邻域二次收敛。误差递推可直接线化得到
	\(
	\varepsilon_{k+1}\approx \tfrac{3}{2}\,a\,\varepsilon_k^2
	\)
	(此处 \(\varepsilon_k\) 为 \(y_k\) 的绝对误差),从而二次收敛。令 \(x=a\,y\) 则
	\(x=\sqrt a\,(1+\eta)\)
	的相对误差与 \(y\) 的相对误差一致;随后的 Heron 一步按前述定理使相对误差平方级下降。
\end{proof}

\begin{corollary}[一次修正的效果评估]
	若 \(\tilde y=\frac{1}{\sqrt a}(1+\epsilon)\) 为倒数平方根的相对误差近似,则
	\(
	\tilde x=a\tilde y=\sqrt a(1+\epsilon)
	\)
	为平方根近似。对 \(\tilde x\) 做一轮 Heron:
	\[
	x^+=\tfrac12\!\left(\tilde x+\frac{a}{\tilde x}\right)
	=\sqrt a\left(1+\frac{\epsilon^2}{2}+O(\epsilon^3)\right),
	\]
	相对误差由 \(O(\epsilon)\) 降为 \(O(\epsilon^2)\)。
\end{corollary}
