\chapter{统计计算方法}

\section{非线性方程的求解}
\subsubsection{二分法}
\begin{method}[Bisection Method]
	设函数$f$在$[a,b]$上连续,且满足$f(a)f(b)<0$。为求解非线性方程$f(x)=0$,由\cref{lem:IntermediateValueR}可知,存在至少一个根$x^{\star}\in(a,b)$。二分法通过反复对区间进行二等分来逐步缩小根所在的区间:令:
	\begin{equation*}
		c=\frac{a+b}{2}
	\end{equation*}
	若$f(c)=0$,则$c$即为方程的解;否则根据符号判定准则,取满足:
	\begin{equation*}
		f(a)f(c)<0\quad\text{或}\quad f(c)f(b)<0
	\end{equation*}
	的子区间作为新的区间$[a,b]$,并重复上述过程。\par
	经过$n$次迭代后,所得区间长度不超过$(b-a)/2^n$,任意迭代点$x_n$与真根$x^{\star}$的误差满足:
	\begin{equation*}
		|x_n-x^{\star}|\leqslant\frac{b-a}{2^n}
	\end{equation*}
	因此,可通过给定误差容许值$\varepsilon$,选取满足:
	\begin{equation*}
		\frac{b-a}{2^n}\leqslant\varepsilon
	\end{equation*}
	的最小整数$n$来控制计算精度。
\end{method}
\subsubsection{不动点迭代法}
\begin{method}[Fixed-point Iteration Method]
	为求解非线性方程$f(x)=0$,可将其改写为等价的形式$x=T(x)$,若$T$是一个压缩映射,由\cref{theo:ContractionMapTheorem}可求得非线性方程的根。对于迭代次数问题,可用\cref{theo:ContractionMapTheorem}中的第二种误差估计公式来控制精度。
\end{method}
\begin{definition}
	设映射$T$有不动点$x^{\star}$。若存在$x^{\star}$的邻域$\bar{U}(x^{\star},\delta)$满足对任意的$x\in\bar{U}(x^{\star},\delta)$,序列$\{x_n=T^nx\}\to x^{\star}$,则称此时的迭代法\gls{LocalConvergence}。
\end{definition}
\begin{theorem}\label{theo:FixedPointLocalConvergence}
	设映射$T$有不动点$x^{\star}$,$T'$在$x^{\star}$的某个邻域内连续,且$|T'x^{\star}|<1$,则此时的迭代法局部收敛。
\end{theorem}
\begin{proof}
	由条件和\cref{prop:RSeq}(5)可知存在$x^{\star}$的邻域$U$和$L$,使得对任意的$x\in U$有$|T'x|\leqslant L<1$。根据\cref{theo:LagrangeMeanValueTheorem}可知对任意的$x'\in U$有:
	\begin{equation*}
		|Tx-Tx^{\star}|\leqslant L|x-x^{\star}|
	\end{equation*}
	即$T$在$x^{\star}$的某个闭邻域内是压缩映射,由\cref{theo:RnComplete}、\cref{prop:CompleteMetricSpace}(1)和\cref{theo:ContractionMapTheorem}可知此时的迭代法局部收敛。
\end{proof}
\begin{definition}
	设迭代过程$x_{n+1}=Tx_n$收敛于$T$的不动点$x^{\star}$。令$\varepsilon_n=x_n-x^{\star}$,若有:
	\begin{equation*}
		\lim_{n\to+\infty}\frac{\varepsilon_{n+1}}{\varepsilon_n^p}=C\in\mathbb{R}^{}\setminus\{0\}
	\end{equation*}
	则称该迭代过程是$p$阶收敛的。特别的,$p=1$且$|C|<1$时称为\gls{LinearConvergence},$p>1$时称为\gls{SuperlinearConvergence}。
\end{definition}
\begin{theorem}\label{theo:FixedPointOrderPConvergence}
	设迭代过程$x_{n+1}=Tx_n$。若对于$p\in\mathbb{N}^+$,$T^{(p)}$在$T$的不动点$x^{\star}$的某个邻域内连续,并且有:
	\begin{equation*}
		T'x^{\star}=T''x^{\star}=\cdots=T^{(p-1)}x^{\star}=0,\;T^{(p)}x^{\star}\ne0
	\end{equation*}
	则该迭代过程在$x^{\star}$邻近是$p$阶收敛的。
\end{theorem}
\begin{proof}
	由$T'x^{\star}=0$和\cref{theo:FixedPointLocalConvergence}可知该迭代过程是局部收敛的。根据\info{泰勒展开Lagrange余项}可得:
	\begin{equation*}
		Tx_n=Tx^{\star}+\frac{T^{(p)}\xi_n}{p!}(x_n-x^{\star})^p,\quad\xi_n\text{在$x_n$与$x^{\star}$之间}
	\end{equation*}
	所以:
	\begin{equation*}
		\frac{Tx_n-x^{\star}}{(x_n-x^{\star})^p}=\frac{T^{(p)}\xi_n}{p!}
	\end{equation*}
	由$T^{(p)}$在$x^{\star}$某个邻域内的连续性和\cref{prop:RSeq}(8.c)(4.a)可得:
	\begin{equation*}
		\lim_{n\to+\infty}\frac{Tx_n-x^{\star}}{(x_n-x^{\star})^p}=\lim_{n\to+\infty}\frac{T^{(p)}\xi_n}{p!}=\frac{1}{p!}T^{(p)}\left(\lim_{n\to+\infty}\xi_n\right)=\frac{T^{(p)}x^{\star}}{p!}\qedhere
	\end{equation*}
\end{proof}
\subsubsection{牛顿法}
\begin{method}[Newton Method]
	设$f$在根$x^{\star}$的某个邻域内是$C^2$类函数,且满足$f(x^{\star})=0,\;f'(x^{\star})\ne0$。为求解非线性方程$f(x)=0$,牛顿法对$f$在当前迭代点$x_n$处作一阶Taylor展开,得到:
	\begin{equation*}
		f(x)\approx f(x_n)+f'(x_n)(x-x_n)
	\end{equation*}
	令上式为$0$得到迭代格式为:
	\begin{equation*}
		x_{n+1}=x_n-\frac{f(x_n)}{f'(x_n)}
	\end{equation*}\par
	牛顿法的迭代函数为:
	\begin{equation*}
		Tx=x-\frac{f(x)}{f'(x)}
	\end{equation*}
	于是有:
	\begin{equation*}
		T'x=1-\frac{[f'(x)]^2-f(x)f''(x)}{[f'(x)]^2}=\frac{f(x)f''(x)}{[f'(x)]^2}
	\end{equation*}
	根据\cref{theo:FixedPointOrderPConvergence}可知牛顿法至少是二阶收敛的。
\end{method}


\paragraph{设定与记号}
固定 \(a>0\)。定义迭代
\[
x_{n+1}=\tfrac12\!\left(x_n+\frac{a}{x_n}\right),\qquad n=0,1,2,\dots
\]
设真值 \(r=\sqrt a>0\),误差 \(e_n\coloneq x_n-r\)。

\begin{theorem}[Heron 法是 Newton--Raphson 的特例]
	令 \(f(x)=x^2-a\)。对方程 \(f(x)=0\) 应用 Newton 法
	\[
	x_{n+1}=x_n-\frac{f(x_n)}{f'(x_n)}=x_n-\frac{x_n^2-a}{2x_n}
	=\frac12\!\left(x_n+\frac{a}{x_n}\right),
	\]
	即为所给迭代。
\end{theorem}

\begin{theorem}[不变性、单调性与收敛]
	若 \(x_0>0\),则对所有 \(n\) 都有 \(x_n>0\)。进一步:
	\begin{enumerate}
		\item 若 \(x_0>r\),则 \(x_{n+1}\in(r,x_n)\),序列严格单调递减并下有界,极限为 \(r\)。
		\item 若 \(0<x_0<r\),则 \(x_{n+1}\in(x_n,r)\),序列严格单调递增并上有界,极限为 \(r\)。
	\end{enumerate}
\end{theorem}

\begin{proof}
	首先 \(x_n>0\Rightarrow x_{n+1}=\frac12(x_n+a/x_n)>0\),故正性不变。
	
	设 \(x_n>r\)。则
	\[
	x_{n+1}-r=\frac{x_n^2+a}{2x_n}-r
	=\frac{x_n^2-2rx_n+r^2}{2x_n}=\frac{(x_n-r)^2}{2x_n}>0,
	\]
	且
	\[
	x_n-x_{n+1}=x_n-\frac12\!\left(x_n+\frac{a}{x_n}\right)
	=\frac{x_n^2-a}{2x_n}=\frac{(x_n-r)(x_n+r)}{2x_n}>0,
	\]
	故 \(x_{n+1}\in(r,x_n)\) 并单调递减。由下界 \(r\) 与单调性知收敛,极限必为 \(f\) 的正根 \(r\)。
	
	当 \(0<x_n<r\) 时同理得
	\[
	x_{n+1}-r=\frac{(x_n-r)^2}{2x_n}<0,\qquad
	x_{n+1}-x_n=\frac{a-x_n^2}{2x_n}>0,
	\]
	故 \(x_{n+1}\in(x_n,r)\) 并单调递增,极限同为 \(r\)。
\end{proof}

\begin{theorem}[误差精确递推与二次收敛]
	令 \(e_n=x_n-r\)。对任意 \(x_n>0\) 有
	\[
	e_{n+1}=x_{n+1}-r=\frac{(x_n-r)^2}{2x_n}=\frac{e_n^2}{2x_n}.
	\]
	因此当 \(x_n\to r\) 时
	\[
	e_{n+1}=\frac{e_n^2}{2r}\,(1+o(1)),
	\]
	即误差二次收敛,渐近误差常数为 \(1/(2r)\)。
\end{theorem}

\begin{proof}
	代数恒等式直接给出
	\(
	x_{n+1}-r=\frac{(x_n-r)^2}{2x_n}.
	\)
	当 \(x_n\to r>0\) 时,\(\frac{1}{2x_n}\to \frac{1}{2r}\),得所述渐近式。
\end{proof}

\begin{remark}[相对误差形式]
	相对误差 \(\delta_n\coloneq\frac{x_n-r}{r}\) 满足
	\[
	\delta_{n+1}=\frac{\delta_n^2}{2(1+\delta_n)}=\frac12\delta_n^2+O(\delta_n^3),
	\]
	同样显示二次收敛。
\end{remark}

\begin{theorem}[倒数平方根的牛顿迭代与一次修正]
	为求 \(y=1/\sqrt a\),考虑
	\(
	\phi(y)=y^{-2}-a=0
	\)。
	Newton 法给出
	\[
	y_{k+1}=y_k-\frac{\phi(y_k)}{\phi'(y_k)}
	=y_k-\frac{y_k^{-2}-a}{-2y_k^{-3}}
	=y_k\!\left(\frac32-\frac{a y_k^2}{2}\right).
	\]
	设初值 \(y_0\) 来自硬件 \texttt{rsqrt} 指令(粗近似)。则该迭代二次收敛到 \(1/\sqrt a\)。
	得到 \(y_m\) 后,可令
	\(
	x=a\,y_m
	\)
	作为 \(\sqrt a\) 的近似;若需再提高精度,对 \(x\) 再做一次 Heron 步
	\[
	x\ \leftarrow\ \frac12\!\left(x+\frac{a}{x}\right),
	\]
	误差继续二次缩小。
\end{theorem}

\begin{proof}
	\(\phi'(y)=-2y^{-3}\neq0\) 于 \(y>0\)。标准 Newton 收敛定理表明在真根邻域二次收敛。误差递推可直接线化得到
	\(
	\varepsilon_{k+1}\approx \tfrac{3}{2}\,a\,\varepsilon_k^2
	\)
	(此处 \(\varepsilon_k\) 为 \(y_k\) 的绝对误差),从而二次收敛。令 \(x=a\,y\) 则
	\(x=\sqrt a\,(1+\eta)\)
	的相对误差与 \(y\) 的相对误差一致;随后的 Heron 一步按前述定理使相对误差平方级下降。
\end{proof}

\begin{corollary}[一次修正的效果评估]
	若 \(\tilde y=\frac{1}{\sqrt a}(1+\epsilon)\) 为倒数平方根的相对误差近似,则
	\(
	\tilde x=a\tilde y=\sqrt a(1+\epsilon)
	\)
	为平方根近似。对 \(\tilde x\) 做一轮 Heron:
	\[
	x^+=\tfrac12\!\left(\tilde x+\frac{a}{\tilde x}\right)
	=\sqrt a\left(1+\frac{\epsilon^2}{2}+O(\epsilon^3)\right),
	\]
	相对误差由 \(O(\epsilon)\) 降为 \(O(\epsilon^2)\)。
\end{corollary}

\section{插值法}
\begin{definition}
	设函数$f$在$[a,b]$上有定义,且已知在点$a\leqslant x_0<x_1<\cdots<x_n\leqslant b$上的值$y_0,y_1,\dots,y_n$。若存在一个简单函数$P(x)$使得:
	\begin{equation*}
		P(x_i)=y_i,\quad i=0,1,\dots,n
	\end{equation*}
	则称$P$为$f$的\gls{InterpolatingFunction},点$x_0,x_1,\dots,x_n$被称为\gls{InterpolationNodes},$[a,b]$为\gls{InterpolationInterval},求插值函数$P$的方法称为\gls{InterpolationMethod}。若$P$是不超过$n$次的多项式,即:
	\begin{equation*}
		P(x)=\sum_{i=0}^{n}a_ix^i,\quad a_i\in\mathbb{R}^{},\;i=0,2,\dots,n
	\end{equation*}
	则称$P$为\gls{InterpolationPolynomial},对应的插值法被称为\gls{PolynomialInterpolation}。若$P$是分段的多项式,则称对应的插值法为\gls{piecewise interpolation}。
\end{definition}
\begin{theorem}\label{theo:ExistenceUniquenessInterpolationPolynomial}
	设函数$f$在$[a,b]$上有定义,且已知在点$a\leqslant x_0<x_1<\cdots<x_n\leqslant b$上的值$y_0,y_1,\dots,y_n$。$f$在$[a,b]$上次数不超过$n$的插值多项式存在且唯一。
\end{theorem}
\begin{proof}
	设$P(x)=\sum\limits_{i=0}^{n}a_ix^i$,则由条件可得到:
	\begin{equation*}
		\begin{pmatrix}
			1 & x_0 & x_0^2 & \cdots & x_0^n \\
			1 & x_1 & x_1^2 & \cdots & x_1^n \\
			\vdots & \vdots & \vdots & \ddots & \vdots \\
			1 & x_n & x_n^2 & \cdots & x_n^n
		\end{pmatrix}
		\begin{pmatrix}
			a_0 \\
			a_1 \\ 
			\vdots \\ 
			a_n
		\end{pmatrix}
		=
		\begin{pmatrix}
			y_0 \\ 
			y_1 \\ 
			\vdots \\ 
			y_n
		\end{pmatrix}
	\end{equation*}
	根据\cref{prop:Determinant}(13)、\cref{prop:Rank}(1)和\cref{theo:SolutionOfSLE2}(2)可知上述线性方程组的解存在且唯一,即$f$在$[a,b]$上次数不超过$n$的插值多项式存在且唯一。
\end{proof}
\begin{definition}
	若$n$次多项式$l_i(x),\;i=0,1,\dots,n$在$n+1$个节点$x_0<x_1<\cdots<x_n$上满足:
	\begin{equation*}
		l_i(x_j)=
		\begin{cases}
			1,& i=j \\
			0,& i\ne j
		\end{cases},\quad
		i,j=0,1,\dots,n
	\end{equation*}
	则称$l_0(x),l_1(x),\dots,l_n(x)$为节点$x_0,x_1,\dots,x_n$上的$n$次\gls{InterpolationBasisFunctions}。
\end{definition}
\begin{definition}
	设函数$f$在$[a,b]$上有定义,且已知在点$a\leqslant x_0<x_1<\cdots<x_n\leqslant b$上的值$y_0,y_1,\dots,y_n$。定义:
	\begin{equation*}
		l_i(x)=\frac{(x-x_0)(x-x_1)\dots(x-x_{i-1})(x-x_{i+1})\dots(x-x_n)}{(x_i-x_0)(x_i-x_1)\dots(x_i-x_{i-1})(x_i-x_{i+1})\dots(x_i-x_n)},\quad i=0,1,\dots,n
	\end{equation*}
	称:
	\begin{equation*}
		L_n(x)\coloneq\sum_{i=0}^{n}y_il_i(x)
	\end{equation*}
	为\gls{LagrangeInterpolationPolynomial}。
\end{definition}
\begin{property}\label{prop:LagrangeInterpolationPolynomial}
	设函数$f$在$[a,b]$上有定义,且已知在点$a\leqslant x_0<x_1<\cdots<x_n\leqslant b$上的值$y_0,y_1,\dots,y_n$,$f^{(n)}$在$[a,b]$上连续,$f^{(n+1)}$在$(a,b)$上存在,$L_n$为$f$在节点$x_0,x_1,\dots,x_n$上的Lagrange插值多项式。$L_n$具有如下性质:
	\begin{enumerate}
		\item $\sum\limits_{i=0}^{n}l_i(x)=1$;
		\item 记$w_{n+1}(x)=(x-x_0)(x-x_1)\cdots(x-x_n)$,则:
		\begin{equation*}
			L_n(x)=\sum_{i=0}^{n}y_i\frac{w_{n+1}(x)}{(x-x_i)w_{n+1}'(x_i)}
		\end{equation*}
		\item $L_n$有如下形式:
		\begin{equation*}
			L_n(x)=\frac{\sum\limits_{i=0}^{n}y_i\dfrac{1}{(x-x_i)w_{n+1}'(x_i)}}{\sum\limits_{i=0}^{n}\dfrac{1}{(x-x_i)w_{n+1}'(x_i)}}
		\end{equation*}
		\item 对于任意的$x\in[a,b]$,有误差估计:
		\begin{equation*}
			f(x)-L_n(x)=\frac{f^{(n+1)}(\xi)}{(n+1)!}w_{n+1}(x)
		\end{equation*}
		其中$\xi\in(a,b)$。
	\end{enumerate}
\end{property}
\begin{proof}
	(1)设$f$在$[a,b]$上恒等于$1$,由\info{$n$次多项式在复数域上最多有$n$个零点}即可得出结论。\par
	(2)由\info{乘积的微分}可知:
	\begin{equation*}
		w_{n+1}'(x_i)=(x_i-x_0)(x_i-x_1)\dots(x_i-x_{i-1})(x_i-x_{i+1})\dots(x_i-x_n)
	\end{equation*}
	于是:
	\begin{equation*}
		l_i(x)=\frac{w_{n+1}(x)}{(x-x_i)w_{n+1}'(x_i)}
	\end{equation*}\par
	(3)由(1)(2)可得:
	\begin{equation*}
		1=\sum_{i=0}^{n}l_i(x)=\sum_{i=0}^{n}\frac{w_{n+1}(x)}{(x-x_i)w_{n+1}'(x_i)}=w_{n+1}(x)\sum_{i=0}^{n}\frac{1}{(x-x_i)w_{n+1}'(x_i)}
	\end{equation*}
	所以:
	\begin{equation*}
		w_{n+1}(x)=\frac{1}{\sum\limits_{i=0}^{n}\dfrac{1}{(x-x_i)w_{n+1}'(x_i)}}
	\end{equation*}
	于是有:
	\begin{align*}
		L_n(x)&=\sum_{i=0}^{n}y_i\frac{w_{n+1}(x)}{(x-x_i)w_{n+1}'(x_i)}=w_{n+1}(x)\sum_{i=0}^{n}y_i\frac{1}{(x-x_i)w_{n+1}'(x_i)} \\
		&=\frac{\sum\limits_{i=0}^{n}y_i\dfrac{1}{(x-x_i)w_{n+1}'(x_i)}}{\sum\limits_{i=0}^{n}\dfrac{1}{(x-x_i)w_{n+1}'(x_i)}}
	\end{align*}\par
	(4)由条件可设$f(x)-L_n(x)=K(x)w_{n+1}(x)$,其中$K(x)$是一个待定函数。作辅助函数:
	\begin{equation*}
		\varphi(t)=f(t)-L_n(t)-K(x)w_{n+1}(t)
	\end{equation*}
	由\cref{prop:FrechetDifferential}(4)可知$\varphi^{(n)}$在$[a,b]$上连续,$\varphi^{(n+1)}$在$(a,b)$上存在。因为$\varphi$在$x_0,x_1,\dots,x_n$和$x$处均为$0$,所以$\varphi$在$[a,b]$上有$n+2$个零点,根据\cref{theo:RolleTheorem}可知$\varphi'$在$\varphi$的两个零点间至少存在一个零点,所以$\varphi'$在$[a,b]$上至少存在$n+1$个零点。对$\varphi'$再使用\cref{theo:RolleTheorem}可知$\varphi''$在$[a,b]$上至少存在$n$个零点。依此类推可知$\varphi^{(n+1)}$在$(a,b)$上至少存在一个零点$\xi$,即:
	\begin{equation*}
		\varphi^{(n+1)}(\xi)=f^{(n+1)}(\xi)-K(x)(n+1)!=0
	\end{equation*}
	于是:
	\begin{equation*}
		K(x)=\frac{f^P(n+1)(\xi)}{(n+1)!}
	\end{equation*}
	即:
	\begin{equation*}
		f(x)-L_n(x)=\frac{f^{(n+1)}(\xi)}{(n+1)!}w_{n+1}(x)\qedhere
	\end{equation*}
\end{proof}
\begin{note}
	\cref{prop:LagrangeInterpolationPolynomial}(3)中$L_n(x)$的形式又被称为\textbf{Barycentric形式}。该表示形式在数学上与拉格朗日插值多项式完全等价,但在数值计算中避免了高次多项式乘积的显式计算,即计算:
	\begin{equation*}
		w_{n+1}(x)=(x-x_0)(x-x_1)\cdots(x-x_n)
	\end{equation*}
	从而减少了由大数乘积和相消所引起的数值误差,具有显著更优的稳定性。基于上述原因,在实际数值计算中通常优先采用Barycentric形式来实现多项式插值。
\end{note}
\begin{minted}[linenos,breaklines]{python}
    import numpy as np
    from scipy.interpolate import BarycentricInterpolator
    x_nodes = np.array([0.0,1.0,2.0,3.0])
    y_nodes = np.array([1.0,2.0,0.0,5.0])
    interp = BarycentricInterpolator(x_nodes,y_nodes)
    print(interp(1.5))
\end{minted}

