\section{平稳序列的Wold表示}
\begin{definition}
	设$\{X_t\}$是平稳序列,$\operatorname{E}(X_t)=\mu$,记:
	\begin{equation*}
		\mathbf{X}_{t,n}=(X_t,X_{t-1},\dots,X_{t-n+1}),\quad\hat{X}_{t+k,n}=L(X_{t+k}|\mathbf{X}_{t,n}),\quad\sigma_{k,n}^2=\operatorname{E}[(X_{t+k}-\hat{X}_{t+k,n})^2]
	\end{equation*}
	由\cref{prop:BestLinearForcast}(12)可得$\sigma_{k,n}^2$随着$n$的增大单调不增且有下界$0$,\info{单调有界必收敛}所以定义:
	\begin{equation*}
		\sigma_k^2=\lim_{n\to+\infty}\sigma_{k,n}^2
	\end{equation*}
\end{definition}
\begin{property}
	设$\{X_t\}$是平稳序列,$\operatorname{E}(X_t)=\mu$,则:
	\begin{enumerate}
		\item $\sigma_k^2$与$t$无关;
		\item $\sigma_k^2\geqslant\sigma_{k-1}^2$。
	\end{enumerate}
\end{property}
\begin{proof}
	(1)设$\alpha$满足用$\mathbf{X}_{t,n}$预测$X_{t+k}$时的预测方程,因为$\{X_t\}$是平稳序列,所以$\alpha=(\seq{a}{n})^T$与$t$无关。因为:
	\begin{align*}
		&(X_{t+k}-\hat{X}_{t+k,n})^2=\left[X_{t+k}-\sum_{i=1}^{n}a_i(X_{t+1-i}-\mu)-\mu\right]^2 \\
		=&(X_{t+k}-\mu)^2+\left[\sum_{i=1}^{n}a_i(X_{t+1-i}-\mu)\right]^2-\sum_{i=1}^{n}a_i(X_{t+1-i}-\mu)(X_{t+k}-\mu) \\
		=&(X_{t+k}-\mu)^2+\sum_{i=1}^{n}a_i^2(X_{t+1-i}-\mu)^2+\sum_{i=1}^{n}\sum_{j=1}^{n}a_ia_j(X_{t+1-i}-\mu)(X_{t+1-j}-\mu) \\
		&-\sum_{i=1}^{n}a_i(X_{t+1-i}-\mu)(X_{t+k}-\mu)
	\end{align*}
	由\cref{prop:MeasurableIntegral}(5)可得:
	\begin{equation*}
		\sigma_{k,n}^2=\operatorname{E}[(X_{t+k}-\hat{X}_{t+k,n})^2]=\sum_{i=1}^{n}(1+a_i^2)\gamma(0)+\sum_{i=1}^{n}\sum_{j=1}^{n}a_ia_j\gamma(j-i)-\sum_{i=1}^{n}a_i\gamma(1-i-k)
	\end{equation*}
	与$t$无关,所以$\sigma_k^2$与$t$也无关。\par
	(2)由\cref{prop:BestLinearForcast}可得:
	\begin{align*}
		\sigma_k^2&=\lim_{n\to+\infty}\operatorname{E}\{[X_{t+k}-L(X_{t+k}|\mathbf{X}_{t,n})]^2\} \\
		&=\lim_{n\to+\infty}\operatorname{E}\{[X_{t+k-1}-L(X_{t+k-1}|\mathbf{X}_{t-1,n})]^2\} \\
		&\geqslant\lim_{n\to+\infty}\operatorname{E}\{[X_{t+k-1}-L(X_{t+k-1}|\mathbf{X}_{t,n+1})]^2\}=\sigma_{k-1}^2\qedhere
	\end{align*}
\end{proof}
\begin{definition}
	设$\{X_t\}$是平稳序列。
	\begin{enumerate}
		\item 若$\sigma_1^2=0$,称$\{X_t\}$是\textbf{决定性平稳序列};
		\item 若$\sigma_1^2>0$,称$\{X_t\}$是\textbf{非决定性平稳序列},并称$\sigma_1^2$为$\{X_t\}$的\textbf{$1$步预测误差的方差};
		\item 若$\lim\limits_{k\to+\infty}\sigma_k^2=\gamma(0)$,则称$\{X_t\}$是\textbf{纯非决定性的}。
	\end{enumerate}
\end{definition}
\begin{note}
	如果$\{X_t\}$是纯非决定性的,说明用充分多的历史对遥远的未来进行预测和用$\mu$对其进行预测的效果差不多,因为:
	\begin{equation*}
		\operatorname{MSE}(X_t-\mu)=\operatorname{Var}(X_t)=\gamma(0)=\lim_{k\to+\infty}\sigma_k^2
	\end{equation*}
	即预测方差相近并且都为无偏估计。
\end{note}

\begin{lemma}
	设$\{X_t\}$是一个零均值平稳序列,令:
	\begin{gather*}
		K_n=\left\{\sum_{i=0}^{m}c_iX_{n-i}:c_i\in\mathbb{R}^{},\;m\in\mathbb{N}^+\right\},\quad
		H_n=\overline{\operatorname{sp}}\{X_n,X_{n-1},X_{n-2},\cdots\}
	\end{gather*}
	则对任意的$\xi\in H_n$,存在$K_n$中的点列$\{\xi_n\}$使得:
	\begin{equation*}
		\lim_{n\to+\infty}\operatorname{E}[(\xi_n-\xi)^2]=0
	\end{equation*}
\end{lemma}
\begin{proof}
	用$\overline{K}_n$表示$K_n$中的随机变量和它们均方极限,只需证明$\overline{K}_n=H_n$。先证明$\overline{K}_n$是完备的。任取$\overline{K}_n$中的一个Cauchy点列$\{x_n\}$,因为\info{未完成}
\end{proof}
