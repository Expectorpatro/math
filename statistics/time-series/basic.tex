\section{基本时间序列}

\subsection{预处理}
\subsection{title}
\begin{gather*}
	\bar{x}=\frac{\sum\limits_{t=1}^{n}x_t}{n} \\
	\hat{\gamma}(k)=\frac{\sum_{t=1}^{n-k}(x_t-\bar{x})(x_{t+k}-\bar{x})}{n-k},\;
	\hat{\gamma}(0)=s^2 \\
	\hat{\rho}_k=\frac{\hat{\gamma}(k)}{\hat{\gamma}(0)}
\end{gather*}

acf(x, lag=)
虚线为自相关系数$2$倍标准差位置

\subsubsection{平稳性检验}
\begin{enumerate}
	\item 时序图观察
	\item 自相关系数图acf函数,应呈现出迅速衰减向$0$
\end{enumerate}

\subsubsection{白噪声检验}
同均值同方差不相关
\begin{theorem}\label{theo:Barlett}
	如果一个时间序列是白噪声,得到一个观察期数为$n$的观察序列$\{x_t\}$,那么有:
	\begin{equation*}
		\hat{\rho}_k\sim\operatorname{N}\left(0,\frac{1}{n}\right),\;\forall\;k\ne0
	\end{equation*}
	近似成立。
\end{theorem}

\begin{derivation}
	构建假设:
	\begin{enumerate}
		\item 原假设:延迟期数小于或等于$m$期的序列值之间不相关,即:
		\begin{equation*}
			H_0:\rho_1=\rho_2=\cdots=\rho_m=0
		\end{equation*}
		\item 备择假设:延迟期数小于或等于$m$期的序列值之间有相关性,即:
		\begin{equation*}
			\text{至少存在某个}\rho_k\ne0,\;k\leqslant m
		\end{equation*}
	\end{enumerate}
	构建$Q$统计量:
	\begin{equation*}
		Q=n\sum_{k=1}^{m}\hat{\rho}_k^2
	\end{equation*}
	若原假设成立,则$\hat{\rho}_k^2$之间也彼此独立,于是:
	\begin{equation*}
		Q=\sum_{k=1}^{m}(\sqrt{n}\hat{\rho}_k)^2=n\sum_{k=1}^{m}\hat{\rho}_k^2\sim\chi^2_m
	\end{equation*}
	当原假设不成立时,$Q$统计量的值应该偏大,于是拒绝域取$\chi^2_m$分布的上$\alpha$分位点。\par
	Box和Ljung为了弥补小样本情况时$Q$统计量效果较差的问题,推导出了LB统计量:
	\begin{equation*}
		LB=n(n+2)\sum_{k=1}^{m}\left(\frac{\hat{\rho}_k^2}{n-k}\right)\sim\chi^2_m
	\end{equation*}\par
	两种检验的代码为Box.test(x, type=, lag=)
\end{derivation}

\subsection{arima}
\subsubsection{wold分解定理}
任意一个平稳时间序列$\{X_t\}$都可以分解为两个不相关的平稳序列之和:
\begin{gather*}
	X_t=V_t+\xi_t,\quad\operatorname{Cov}(V_t,\xi_s)=0,\;\forall\;t\ne s \\ V_t=\sum_{j=0}^{+\infty}\varphi_jX_{t-j},\quad\xi_t=\sum_{j=0}^{+\infty}\theta_j\varepsilon_{t-j},\;\varepsilon_t\sim\operatorname{WN}(0,\sigma^2),\theta_0=1,\{\theta_t\}\in l^2
\end{gather*}
称$\{V_t\}$为确定性序列,$\{\xi_t\}$为随机平稳序列。\par
若使用线性函数对$Y_t$进行预测的方差随着自变量个数的增大趋于$0$,称序列为确定性序列;若方差趋于$\operatorname{Var}(Y_t)$,称为纯随机序列;介于二者之间的是随机序列。
\subsubsection{ar模型}
白噪声、$\varepsilon_t$与$X_t$不相关,中心化序列,自回归系数多项式。\par
齐次线性差分方程的解$\sum\limits_{i=1}^{p}c_i\lambda_i^t$。非齐次特解用常数特解。\par
特征多项式的根都在单位元内,自回归系数多项式的根要在单位圆外。AR(2)平稳域为三角形(x从-2到2,y从-1到1)\par
均值,方差(wold系数满足$\psi_0=1$,从1开始满足回归方程),协方差(满足回归方程),自相关(满足回归方程,指数衰减、拖尾性证明),偏自相关系数(用Yule-Walker方程构造线性方程组求解,求kk)
\subsubsection{ma模型}
移动平均系数多项式\par
均值,方差,自协方差函数(),自相关系数\par
可逆性(特征方程)\par
逆函数公式

\subsection{指数平滑模型}

\begin{definition}
	设$\{X_t\}$是一个时间序列,若存在不可观测状态$\{Y_t\}$以及随机误差$\{\varepsilon_t\}$,使得模型可以写为:
	\begin{gather*}
		X_t=f(Y_t)+g(Y_{t-1},\varepsilon_t) \\
		Y_t=p(Y_{t-1})+q(Y_{t-1},\varepsilon_t)
	\end{gather*}
	则称时间序列$\{X_t\}$具有\gls{StateSpace}表示,上第一式称为\gls{MeasurementEquation},第二式称为\gls{StateEquation}。
\end{definition}

\begin{definition}
	设时间序列$\{X_t\}$具有状态空间表示。若其观测方程与状态方程分别为:
	\begin{gather*}
		X_t=Y_{t-1}+\varepsilon_t \\
		Y_t=Y_{t-1}+\alpha\varepsilon_t,
		\quad 0<\alpha<1
	\end{gather*}
	且$\operatorname{E}(\varepsilon_t)=0$,则称$\{X_t\}$为\gls{SES}模型,其中$Y_t$称为序列在时刻$t$的\gls{LevelState},参数$\alpha$称为\gls{SmoothingParameter}。
\end{definition}
\begin{property}\label{prop:SES}
	设$\{X_t\}$是简单指数平滑模型,则:
	\begin{enumerate}
		\item $\{Y_t\}$具有递推形式:
		\begin{equation*}
			Y_t=\alpha X_t+(1-\alpha)Y_{t-1}=\alpha\sum_{i=0}^{T-1}(1-\alpha)^i X_{T-i}+(1-\alpha)^T Y_0
		\end{equation*}
		\item $\hat{X}_{t+h|t}=Y_t,\quad h\geqslant1$。
	\end{enumerate}
\end{property}
\begin{proof}
	(1)
\end{proof}
\subsection{Holt 线性趋势(加性趋势)}

当序列呈现稳定的线性趋势时,Holt 方法通过水平 \(\ell_t\) 与斜率 \(b_t\) 两个状态来描述。加性误差的观测方程和状态更新方程为【243587268339714†L333-L358】:
\begin{align*}
	y_t &= \ell_{t-1} + b_{t-1} + \varepsilon_t,\\
	\ell_t &= \ell_{t-1} + b_{t-1} + \alpha\,\varepsilon_t,\\
	b_t &= b_{t-1} + \beta\,\varepsilon_t,
\end{align*}
其中 \(\alpha,\beta\in[0,1]\)。相应的乘性误差模型为:
\begin{align*}
	y_t &= (\ell_{t-1} + b_{t-1})\,\varepsilon_t,\\
	\ell_t &= (\ell_{t-1} + b_{t-1})\bigl(1 + \alpha(\varepsilon_t-1)\bigr),\\
	b_t &= b_{t-1}\,\bigl(1 + \beta(\varepsilon_t-1)\bigr).
\end{align*}
两类误差的\textbf{一步预测}均为
\begin{equation*}
	\hat{y}_{t+1|t} = \ell_t + b_t,
\end{equation*}
即根据当前水平和斜率进行线性外推。

\paragraph{多步预测公式推导} 根据状态更新方程可证,\(\ell_t\) 和 \(b_t\) 在预测期内保持不变。因此 \(h\) 步预测为
\begin{equation*}
	\hat{y}_{t+h|t} = \ell_t + h\,b_t.
\end{equation*}
证明可用数学归纳法:对 \(h=1\) 成立,假设对 \(h\) 成立,则 \(\hat{y}_{t+h+1|t} = \hat{y}_{t+h|t} + b_t = (\ell_t + h\,b_t) + b_t = \ell_t + (h+1)b_t\)。

\paragraph{性质} Holt 方法的长期预测呈直线增长或下降,其斜率由 \(b_t\) 决定。当 \(\beta\) 较小或为零时,斜率更新缓慢,趋势相对平滑。

\subsection{阻尼趋势}

阻尼趋势引入阻尼因子 \(\phi \in (0,1)\) 抑制远期线性增长。加性误差的模型形式为【243587268339714†L376-L392】:
\begin{align*}
	y_t &= \ell_{t-1} + \phi\,b_{t-1} + \varepsilon_t,\\
	\ell_t &= \ell_{t-1} + \phi\,b_{t-1} + \alpha\,\varepsilon_t,\\
	b_t &= \phi\,b_{t-1} + \beta\,\varepsilon_t.
\end{align*}
乘性误差模型类似,只是观测与状态更新中出现乘性误差。一步预测为 \(\hat{y}_{t+1|t} = \ell_t + \phi\,b_t\)。

\paragraph{多步预测公式推导} 阻尼趋势的 \(h\) 步预测为【243587268339714†L376-L392】:
\begin{equation*}
	\hat{y}_{t+h|t} = \ell_t + b_t\,\phi\,\frac{1 - \phi^{\,h}}{1 - \phi}.
\end{equation*}
这一公式来自求和等比数列 \(\phi + \phi^2 + \cdots + \phi^h = \phi\,(1 - \phi^{h})/(1 - \phi)\)。随着 \(h\to\infty\),预测逼近极限 \(\ell_t + b_t\,\phi/(1-\phi)\),因此阻尼方法的长期预测趋于常数而非无限增长【243587268339714†L376-L392】。

\section{季节模型}

季节性反映周期性波动,可通过季节因子 \(s_t\) 表示长度为 \(m\) 的周期。本节讨论无季节、加性季节和乘性季节三种情况。令 \(k=\lfloor (h-1)/m\rfloor\) 表示预测步长中完整季节的个数。

\subsection{无季节}

若序列不存在季节波动,则季节项 \(s_t\) 缺省。本情形下的预测公式已在前述趋势模型部分给出。

\subsection{加性季节}

加性季节假设季节效应以绝对差异形式叠加。加性误差模型的状态空间方程【243587268339714†L494-L504】:
\begin{align*}
	y_t &= \ell_{t-1} + b_{t-1} + s_{t-m} + \varepsilon_t,\\
	\ell_t &= \ell_{t-1} + b_{t-1} + \alpha\,\varepsilon_t,\\
	b_t &= b_{t-1} + \beta\,\varepsilon_t,\\
	s_t &= s_{t-m} + \gamma\,\varepsilon_t,
\end{align*}
其中 \(\gamma\in[0,1]\) 为季节平滑参数。乘性误差模型改为 \(y_t = (\ell_{t-1}+b_{t-1}+s_{t-m})\,\varepsilon_t\),其余状态更新乘上 \((1+\cdot)\) 的误差调整因子。一步预测为
\begin{equation*}
	\hat{y}_{t+1|t} = \ell_t + b_t + s_{t+1-m}.
\end{equation*}

\paragraph{多步预测公式推导} 在预测阶段,状态不再更新,因此季节因子以周期 \(m\) 轮回。加性季节的 \(h\) 步预测为
\begin{equation*}
	\hat{y}_{t+h|t} = \ell_t + h\,b_t + s_{t+h-m(k+1)},
\end{equation*}
其中最后一个季节项索引 \((t+h-m(k+1))\) 确保使用最近完整周期的估计【243587268339714†L494-L504】。

\subsection{乘性季节}

乘性季节假设季节效应按比例放大或缩小。加性误差的观测方程和更新方程【243587268339714†L518-L530】:
\begin{align*}
	y_t &= (\ell_{t-1} + b_{t-1})\,s_{t-m} + \varepsilon_t,\\
	\ell_t &= \ell_{t-1} + b_{t-1} + \alpha\,\frac{\varepsilon_t}{s_{t-m}},\\
	b_t &= b_{t-1} + \beta\,\frac{\varepsilon_t}{s_{t-m}},\\
	s_t &= s_{t-m} + \gamma\,\frac{\varepsilon_t}{\ell_{t-1}+b_{t-1}},
\end{align*}
乘性误差模型则将 \(\varepsilon_t\) 以乘法形式作用于观测和状态更新。一步预测为
\begin{equation*}
	\hat{y}_{t+1|t} = (\ell_t + b_t)\,s_{t+1-m}.
\end{equation*}

\paragraph{多步预测公式推导} 与加性季节类似,但预测值与季节因子相乘:
\begin{equation*}
	\hat{y}_{t+h|t} = \bigl(\ell_t + h\,b_t\bigr)\,s_{t+h-m(k+1)}.
\end{equation*}
如果趋势采用阻尼形式,则把 \(h\,b_t\) 替换为 \(\phi\)-阻尼和,乘性季节预测则为 \(\bigl[\ell_t + b_t\,\phi(1-\phi^{h})/(1-\phi)\bigr]\,s_{t+h-m(k+1)}\)【243587268339714†L599-L608】。

\section{误差类型:加性与乘性}

\begin{itemize}
	\item \textbf{加性误差(A)}:观测方程中误差以加法形式出现,如 \(y_t = \text{系统值} + \varepsilon_t\)。状态更新也通过加法调整。加性误差适用于方差相对稳定的时间序列。
	\item \textbf{乘性误差(M)}:误差以乘法形式出现,如 \(y_t = \text{系统值} \times \varepsilon_t\)。相对误差用于更新状态,使得状态变量保持正值。乘性误差适用于方差随水平变化的序列。
\end{itemize}
虽然误差类型影响状态更新和预测区间,但不改变点预测的形式;因此上述多步预测公式同时适用于两类误差。

\section{三种趋势 $\times$ 三种季节 $\times$ 两种误差的模型表}

表~\ref{tab:ets} 汇总了 \(3\times 3\times 2=18\) 种 ETS 模型的组合。表中列出误差类型 \(E\)、趋势类型 \(T\)、季节类型 \(S\)、模型记号 \(\mathrm{ETS}(E,T,S)\) 以及 \(h\) 步点预测公式。符号 \(\ell_t,b_t,s_t\) 分别表示时间 \(t\) 的水平、斜率和季节状态,\(\phi\) 为阻尼系数,\(k=\lfloor (h-1)/m\rfloor\)。

\begin{table}[h]
	\centering
	\small
	\caption{18 种 ETS 模型及其多步预测公式}
	\label{tab:ets}
	\begin{tabular}{cccc}
		\toprule
		误差 $E$ & 趋势 $T$ & 季节 $S$ & $h$ 步预测公式 \\
		\midrule
		A/M & N (SES) & N & $\hat{y}_{t+h|t} = \ell_t$ \\
		A/M & A (Holt) & N & $\hat{y}_{t+h|t} = \ell_t + h\,b_t$ \\
		A/M & Ad (阻尼) & N & $\hat{y}_{t+h|t} = \ell_t + b_t\,\phi\,\dfrac{1 - \phi^{\,h}}{1-\phi}$ \\
		\midrule
		A/M & N (SES) & A (加性季节) & $\hat{y}_{t+h|t} = \ell_t + s_{t+h-m(k+1)}$ \\
		A/M & A (Holt) & A & $\hat{y}_{t+h|t} = \ell_t + h\,b_t + s_{t+h-m(k+1)}$ \\
		A/M & Ad (阻尼) & A & $\hat{y}_{t+h|t} = \ell_t + b_t\,\phi\,\dfrac{1 - \phi^{\,h}}{1-\phi} + s_{t+h-m(k+1)}$ \\
		\midrule
		A/M & N (SES) & M (乘性季节) & $\hat{y}_{t+h|t} = \ell_t\,s_{t+h-m(k+1)}$ \\
		A/M & A (Holt) & M & $\hat{y}_{t+h|t} = \bigl(\ell_t + h\,b_t\bigr)\,s_{t+h-m(k+1)}$ \\
		A/M & Ad (阻尼) & M & $\hat{y}_{t+h|t} = \Bigl[\ell_t + b_t\,\phi\,\dfrac{1 - \phi^{\,h}}{1-\phi}\Bigr]\,s_{t+h-m(k+1)}$ \\
		\bottomrule
	\end{tabular}
\end{table}

\section{模型的数学性质与证明}

本节讨论一些指数平滑模型的重要数学性质并给出证明。

\subsection{加权移动平均性质}

简单指数平滑的平滑方程 \(\ell_t = \alpha y_t + (1-\alpha)\ell_{t-1}\) 可以展开为加权移动平均。在初始水平 \(\ell_0\) 给定的情况下,通过递推可得
\begin{align*}
	\ell_t &= \alpha y_t + (1-\alpha)\ell_{t-1}\\
	&= \alpha y_t + (1-\alpha)\bigl[\alpha y_{t-1} + (1-\alpha)\ell_{t-2}\bigr]\\
	&= \alpha y_t + \alpha(1-\alpha)\,y_{t-1} + (1-\alpha)^2\ell_{t-2}.
\end{align*}
继续展开得到
\begin{equation*}
	\ell_T = \alpha\sum_{j=0}^{T-1}(1-\alpha)^j\,y_{T-j} \;+\; (1-\alpha)^T\,\ell_0,
\end{equation*}
说明各历史观测以指数衰减的权重贡献于当前水平【243587268339714†L108-L124】。

\subsection{阻尼趋势的极限}

在阻尼趋势模型中,多步预测的趋势部分为几何级数和 \(\phi + \phi^2 + \cdots + \phi^h\)。利用等比数列求和公式可得
\[
\phi + \phi^2 + \cdots + \phi^h = \phi\,\frac{1-\phi^{\,h}}{1-\phi}.
\]
因此 \(h\) 步预测为 \(\hat{y}_{t+h|t} = \ell_t + b_t\,\phi(1-\phi^{h})/(1-\phi)\)。当 \(h\to\infty\) 时,几何级数趋于 \(\phi/(1-\phi)\),预测逼近常数 \(\ell_t + b_t\,\phi/(1-\phi)\),说明阻尼方法避免了长期线性爆炸。

\subsection{季节模型的周期性}

在加性或乘性季节模型中,季节状态 \(s_t\) 以周期 \(m\) 轮回:\(s_t=s_{t-m}\) 当不考虑误差更新时。因此在预测阶段,\(\hat{y}_{t+h|t}\) 使用索引 \(t+h-m(k+1)\) 的季节因子来选择最后一个完整周期的估计值【243587268339714†L494-L504】。这意味着预测的季节模式与最近一年(或最近 \(m\) 个周期)的形状一致。
