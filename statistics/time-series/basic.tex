\section{本科时间序列}

\subsection{预处理}
\subsection{title}
\begin{gather*}
	\bar{x}=\frac{\sum\limits_{t=1}^{n}x_t}{n} \\
	\hat{\gamma}(k)=\frac{\sum_{t=1}^{n-k}(x_t-\bar{x})(x_{t+k}-\bar{x})}{n-k},\;
	\hat{\gamma}(0)=s^2 \\
	\hat{\rho}_k=\frac{\hat{\gamma}(k)}{\hat{\gamma}(0)}
\end{gather*}

acf(x, lag=)
虚线为自相关系数$2$倍标准差位置

\subsubsection{平稳性检验}
\begin{enumerate}
	\item 时序图观察
	\item 自相关系数图acf函数,应呈现出迅速衰减向$0$
\end{enumerate}

\subsubsection{白噪声检验}
同均值同方差不相关
\begin{theorem}\label{theo:Barlett}
	如果一个时间序列是白噪声,得到一个观察期数为$n$的观察序列$\{x_t\}$,那么有:
	\begin{equation*}
		\hat{\rho}_k\sim\operatorname{N}\left(0,\frac{1}{n}\right),\;\forall\;k\ne0
	\end{equation*}
	近似成立。
\end{theorem}

\begin{derivation}
	构建假设:
	\begin{enumerate}
		\item 原假设:延迟期数小于或等于$m$期的序列值之间不相关,即:
		\begin{equation*}
			H_0:\rho_1=\rho_2=\cdots=\rho_m=0
		\end{equation*}
		\item 备择假设:延迟期数小于或等于$m$期的序列值之间有相关性,即:
		\begin{equation*}
			\text{至少存在某个}\rho_k\ne0,\;k\leqslant m
		\end{equation*}
	\end{enumerate}
	构建$Q$统计量:
	\begin{equation*}
		Q=n\sum_{k=1}^{m}\hat{\rho}_k^2
	\end{equation*}
	若原假设成立,则$\hat{\rho}_k^2$之间也彼此独立,于是:
	\begin{equation*}
		Q=\sum_{k=1}^{m}(\sqrt{n}\hat{\rho}_k)^2=n\sum_{k=1}^{m}\hat{\rho}_k^2\sim\chi^2_m
	\end{equation*}
	当原假设不成立时,$Q$统计量的值应该偏大,于是拒绝域取$\chi^2_m$分布的上$\alpha$分位点。\par
	Box和Ljung为了弥补小样本情况时$Q$统计量效果较差的问题,推导出了LB统计量:
	\begin{equation*}
		LB=n(n+2)\sum_{k=1}^{m}\left(\frac{\hat{\rho}_k^2}{n-k}\right)\sim\chi^2_m
	\end{equation*}\par
	两种检验的代码为Box.test(x, type=, lag=)
\end{derivation}

\subsection{arima}
\subsubsection{wold分解定理}
任意一个平稳时间序列$\{X_t\}$都可以分解为两个不相关的平稳序列之和:
\begin{gather*}
	X_t=V_t+\xi_t,\quad\operatorname{Cov}(V_t,\xi_s)=0,\;\forall\;t\ne s \\ V_t=\sum_{j=0}^{+\infty}\varphi_jX_{t-j},\quad\xi_t=\sum_{j=0}^{+\infty}\theta_j\varepsilon_{t-j},\;\varepsilon_t\sim\operatorname{WN}(0,\sigma^2),\theta_0=1,\{\theta_t\}\in l^2
\end{gather*}
称$\{V_t\}$为确定性序列,$\{\xi_t\}$为随机平稳序列。\par
若使用线性函数对$Y_t$进行预测的方差随着自变量个数的增大趋于$0$,称序列为确定性序列;若方差趋于$\operatorname{Var}(Y_t)$,称为纯随机序列;介于二者之间的是随机序列。
\subsubsection{ar模型}
白噪声、$\varepsilon_t$与$X_t$不相关,中心化序列,自回归系数多项式。\par
齐次线性差分方程的解$\sum\limits_{i=1}^{p}c_i\lambda_i^t$。非齐次特解用常数特解。\par
特征多项式的根都在单位元内,自回归系数多项式的根要在单位圆外。AR(2)平稳域为三角形(x从-2到2,y从-1到1)\par
均值,方差(wold系数满足$\psi_0=1$,从1开始满足回归方程),协方差(满足回归方程),自相关(满足回归方程,指数衰减、拖尾性证明),偏自相关系数(用Yule-Walker方程构造线性方程组求解,求kk)
\subsubsection{ma模型}
移动平均系数多项式\par
均值,方差,自协方差函数(),自相关系数\par
可逆性(特征方程)\par
逆函数公式