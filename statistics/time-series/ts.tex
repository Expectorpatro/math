\section{本科时间序列}


\begin{gather*}
	\bar{x}=\frac{\sum\limits_{t=1}^{n}x_t}{n} \\
	\hat{\gamma}(k)=\frac{\sum_{t=1}^{n-k}(x_t-\bar{x})(x_{t+k}-\bar{x})}{n-k},\;
	\hat{\gamma}(0)=s^2 \\
	\hat{\rho}_k=\frac{\hat{\gamma}(k)}{\hat{\gamma}(0)}
\end{gather*}

acf(x, lag=)
虚线为自相关系数$2$倍标准差位置

\subsubsection{平稳性检验}
\begin{enumerate}
	\item 时序图观察
	\item 自相关系数图acf函数,应呈现出迅速衰减向$0$
\end{enumerate}

\subsubsection{白噪声检验}
同均值同方差不相关
\begin{theorem}\label{theo:Barlett}
	如果一个时间序列是白噪声,得到一个观察期数为$n$的观察序列$\{x_t\}$,那么有:
	\begin{equation*}
		\hat{\rho}_k\sim\operatorname{N}\left(0,\frac{1}{n}\right),\;\forall\;k\ne0
	\end{equation*}
	近似成立
\end{theorem}

\begin{derivation}
	构建假设:
	\begin{enumerate}
		\item 原假设:延迟期数小于或等于$m$期的序列值之间相互独立,即:
		\begin{equation*}
			H_0:\rho_1=\rho_2=\cdots=\rho_m=0
		\end{equation*}
		\item 备择假设:延迟期数小于或等于$m$期的序列值之间有相关性\info{确定这里是有相关性而不是不独立吗?},即:
		\begin{equation*}
			\text{至少存在某个}\rho_k\ne0,\;k\leqslant m
		\end{equation*}
	\end{enumerate}
	构建$Q$统计量:
	\begin{equation*}
		Q=n\sum_{k=1}^{m}\hat{\rho}_k^2
	\end{equation*}
	若原假设成立,则$\hat{\rho}_k^2$之间也彼此独立,于是:
	\begin{equation*}
		Q=\sum_{k=1}^{m}(\sqrt{n}\hat{\rho}_k)^2\sim\chi^2_m
	\end{equation*}
	当原假设不成立时,$Q$统计量的值应该偏大,于是拒绝域取$\chi^2_m$分布的上$\alpha$分位点。\par
	Box和Ljung为了弥补小样本情况时$Q$统计量效果较差的问题,推导出了LB统计量:
	\begin{equation*}
		LB=n(n+2)\sum_{k=1}^{m}\left(\frac{\hat{\rho}_k^2}{n-k}\right)\sim\chi^2_m
	\end{equation*}\par
	两种检验的代码为Box.test(x, type=, lag=)
\end{derivation}

\section{线性差分方程理论}

\subsection{差分与位移}
\begin{definition}
	设$f(x)$是定义在$\mathbb{R}^{}$上的函数,称$\Delta f(x)=f(x)-f(x-1)$为$f(x)$在$x$处的$1$阶\textbf{差分},称$\Delta$为\textbf{差分算子},对$n-1$阶差分后的函数再进行一次$1$阶差分运算称为$n$阶差分,记$\Delta^nf(x)$为$f(x)$的$n$阶差分,则:
	\begin{equation*}
		\Delta^nf(x)=\Delta^{n-1}f(x)-\Delta^{n-1}f(x-1)
	\end{equation*}\par
	将$\mathcal{B}f(x)=f(x-1)$称为$f(x)$在$x$处的$1$步\textbf{位移},称$\mathcal{B}$为位移算子,对$n-1$步位移后的函数再进行一次$1$步位移运算称为$n$步位移,记$\mathcal{B}^nf(x)$为$f(x)$的$n$步位移,则:
	\begin{equation*}
		\mathcal{B}^nf(x)=f(x-n)
	\end{equation*}
\end{definition}
\begin{property}
	设$f(x),g(x)$为定义在$\mathbb{R}^{}$上的函数,$\alpha,\beta$是任意常数,$\mathcal{B}$为位移算子。差分算子$\Delta$和位移算子$\mathcal{B}$具有如下性质:
	\begin{enumerate}
		\item 差分算子是线性算子,即:
		\begin{equation*}
			\Delta[\alpha f(x)+\beta g(x)]=\alpha\Delta f(x)+\beta\Delta g(x)
		\end{equation*}
		\item $\Delta[f(x)g(x)]=\Delta f(x)g(x)+\Delta g(x)\mathcal{B}f(x)$;
		\item $\Delta\left[\dfrac{f(x)}{g(x)}\right]=\dfrac{\Delta f(x)\mathcal{B}g(x)-\Delta g(x)\mathcal{B}f(x)}{g(x)\mathcal{B}g(x)}$;
		\item $\mathcal{B}\alpha=\alpha$;
		\item 位移算子是线性算子,即:
		\begin{equation*}
			\mathcal{B}[\alpha f(x)+\beta g(x)]=\alpha\mathcal{B}f(x)+\beta\mathcal{B}g(x)
		\end{equation*}
		\item 对于多项式$\varphi(x)=\sum\limits_{i=0}^{m}a_ix^i,\;\psi(x)=\sum\limits_{i=0}^{n}b_ix^i$的乘积$A(x)=\varphi(x)\psi(x)$,有:
		\begin{equation*}
			A(\mathcal{B})f(x)=\varphi(\mathcal{B})[\psi(\mathcal{B})f(x)]=\psi(\mathcal{B})[\varphi(\mathcal{B})f(x)]
		\end{equation*}
		\item 差分算子与位移算子具有如下关系:
		\begin{gather*}
			\Delta^n=(I-\mathcal{B})^n=\sum_{i=0}^{n}\binom{n}{i}(-1)^i\mathcal{B}^i \\
			\mathcal{B}^n=(I-\Delta)^n=\sum_{i=0}^{n}\binom{n}{i}(-1)^i\Delta^i \\
			\Delta\mathcal{B}=\mathcal{B}\Delta
		\end{gather*}
	\end{enumerate}
\end{property}
\begin{proof}
	(1)(4)(5)(7)是显然的。\par
	(2)注意到:
	\begin{align*}
		\Delta[f(x)g(x)]&=f(x)g(x)-f(x-1)g(x-1) \\
		&=f(x)g(x)-f(x-1)g(x)+g(x)f(x-1)-g(x-1)f(x-1) \\
		&=[f(x)-f(x-1)]g(x)+[g(x)-g(x-1)]f(x-1) \\
		&=\Delta f(x)g(x)+\Delta g(x)\mathcal{B}f(x)
	\end{align*}\par
	(3)注意到:
	\begin{align*}
		\Delta\left[\frac{f(x)}{g(x)}\right]
		&=\frac{f(x)}{g(x)}-\frac{f(x-1)}{g(x-1)}
		=\frac{f(x)g(x-1)-f(x-1)g(x)}{g(x)g(x-1)} \\
		&=\frac{f(x)g(x-1)-f(x-1)g(x-1)+f(x-1)g(x-1)-f(x-1)g(x)}{g(x)\mathcal{B}g(x)} \\
		&=\frac{[f(x)-f(x-1)]g(x-1)-[g(x)-g(x-1)]f(x-1)}{g(x)\mathcal{B}g(x)} \\
		&=\frac{\Delta f(x)\mathcal{B}g(x)-\Delta g(x)\mathcal{B}f(x)}{g(x)\mathcal{B}g(x)}
	\end{align*}\par
	(6)注意到:
	\begin{equation*}
		A(x)=\varphi(x)\psi(x)=\left(\sum_{i=0}^{m}a_ix^i\right)\left(\sum_{i=0}^{n}b_ix^i\right)=\sum_{i=0}^{m}\sum_{j=0}^{n}a_ib_j x^{i+j}
	\end{equation*}
	所以:
	\begin{align*}
		A(\mathcal{B})f(x)&=\sum_{i=0}^{m}\sum_{j=0}^{n}a_ib_j\mathcal{B}^{i+j}f(x)
		=\sum_{i=0}^{m}\sum_{j=0}^{n}a_ib_j \mathcal{B}^i\mathcal{B}^j f(x) \\
		&=\sum_{i=0}^{m}a_i\mathcal{B}^i\left[\sum_{j=0}^{n}b_j\mathcal{B}^j f(x)\right]
		=\varphi(\mathcal{B})[\psi(\mathcal{B})f(x)]
	\end{align*}
	同理可证$A(\mathcal{B})f(x)=\psi(\mathcal{B})[\varphi(\mathcal{B})f(x)]$。
\end{proof}


\subsection{线性差分方程}
\begin{definition}
	称方程:
	\begin{equation*}
		x_m+a_1(m)x_{m-1}+a_2(m)x_{m-2}+\cdots+a_n(m)x_{m-n}=f(m)
	\end{equation*}
	为关于$\{x_m\}$的$n$阶线性差分方程,其中$f(x),a_1(x),a_2(x),\dots,a_n(x)$为定义在$\mathbb{Z}^{}$上的函数,且$a_n(x)\ne0$在$x\in\mathbb{Z}^{}$恒成立。当$f(x)=0$时,称上述方程为关于$\{x_m\}$的$n$阶齐次线性差分方程。
\end{definition}
\begin{note}
	这里的线性指的是方程关于$\{x_m\}$是线性的。$x_m-\sin(x_{m-1})$不是一个线性方程。
\end{note}
\begin{derivation}
	定义算子:
	\begin{equation*}
		L(x_m)=I+a_1(m)\mathcal{B}x_m+a_2(m)\mathcal{B}^2x_m+\cdots+a_n(m)\mathcal{B}^nx_m
	\end{equation*}
	则线性差分方程可表示为:
	\begin{equation*}
		L(x_m)=f(m)
	\end{equation*}
\end{derivation}
\begin{theorem}\label{theo:LinearDifferenceEqExistenceUniqueness}
	初值问题:
	\begin{equation*}
		L(x_m)=f(m),\;x_0=y_0,\;x_1=y_1,\;\cdots,\;x_{n-1}=y_{n-1}
	\end{equation*}
	有唯一的解。
\end{theorem}
\begin{proof}
	注意到关系:
	\begin{gather*}
		x_m+a_1(m)x_{m-1}+a_2(m)x_{m-2}+\cdots+a_n(m)x_{m-n}=f(m),\;\forall\;m\geqslant n \\
		x_{m}=-\frac{x_{m+n}}{a_n(m+n)}-\frac{a_1(m)x_{m+n-1}}{a_n(m+n)}-\cdots-\frac{a_{n-1}x_{m+1}}{a_n(m+n)}+\frac{f(m)}{a_n(m+n)},\;\forall\;m<0
	\end{gather*}
	于是当如上$n$个初始值给定时,$\{x_m\}$的所有值都可以由上述递推关系唯一得到。
\end{proof}
\begin{definition}
	若存在不全为$0$的常数$\seq{c}{k}$使得序列$\{x_m^{(1)}\},\{x_m^{(2)}\},\dots,\{x_m^{(k)}\}$满足:
	\begin{equation*}
		c_1x_m^{(1)}+c_2x_m^{(2)}+\cdots+c_kx_m^{(k)}=0,\;\forall\;m\in\mathbb{Z}^{}
	\end{equation*}
	则称这些序列\textbf{线性相关},否则就称这些序列\textbf{线性无关}。
\end{definition}
\begin{definition}
	称矩阵:
	\begin{equation*}
		C(m)=
		\begin{pmatrix}
			x_m^{(1)} & x_m^{(2)} & \cdots & x_m^{(k)} \\
			\mathcal{B}x_m^{(1)} & \mathcal{B}x_m^{(2)} & \cdots & \mathcal{B}x_m^{(k)} \\
			\vdots & \vdots & \ddots & \vdots \\
			\mathcal{B}^{k-1}x_m^{(1)} & \mathcal{B}^{k-1}x_m^{(2)} & \cdots & \mathcal{B}^{k-1}x_m^{(k)} \\
		\end{pmatrix}
	\end{equation*}
	为序列$\{x_m^{(1)}\},\{x_m^{(2)}\},\dots,\{x_m^{(k)}\}$的\textbf{Casorati矩阵},将$\det C(m)$称为其\textbf{Casorati行列式}。
\end{definition}
\begin{theorem}\label{theo:CasoratiEq0LinearDependent}
	若序列$\{x_m^{(1)}\},\{x_m^{(2)}\},\dots,\{x_m^{(k)}\}$线性相关,则其Casorati行列式$\det C(m)$在$m\in\mathbb{Z}^{}$上恒为$0$。
\end{theorem}
\begin{proof}
	因为$\{x_m^{(1)}\},\{x_m^{(2)}\},\dots,\{x_m^{(k)}\}$线性相关,所以存在不全为$0$的常数$\seq{c}{k}$使得
	\begin{equation*}
		c_1x_m^{(1)}+c_2x_m^{(2)}+\cdots+c_kx_m^{(k)}=0,\;\forall\;m\in\mathbb{Z}^{}
	\end{equation*}
	于是以$C(m)$为系数矩阵的线性方程组有非零解。由\info{链接线性方程组理论}可知$\det C(m)=0$在$m\in\mathbb{Z}^{}$上恒成立。
\end{proof}
\begin{corollary}\label{cor:CasoratiNe0LinearIndependent}
	若存在$m\in\mathbb{Z}^{}$使得$\det C(m)\ne0$,则序列$\{x_m^{(1)}\},\{x_m^{(2)}\},\dots,\{x_m^{(k)}\}$线性无关。
\end{corollary}
\subsubsection{齐次线性差分方程解的一般理论}
\begin{theorem}\label{theo:LinearDifferenceLinearCombination}
	如果$\{x_m^{(1)}\},\{x_m^{(2)}\},\dots,\{x_m^{(k)}\}$是$n$阶常系数齐次线性差分方程$L(x_m)=0$的$k$个解,则它们的线性组合:
	\begin{equation*}
		\{x_m=c_1x_m^{(1)}+c_2x_m^{(2)}+\cdots+c_kx_m^{(k)}\}
	\end{equation*}
	也是解,其中$c_i$为任意常数,$i=1,2,\dots,k$。
\end{theorem}
\begin{proof}
	因为$x_m^{(j)},\;j=1,2,\dots,k$是$L(x_m)=0$的解,所以:
	\begin{equation*}
		x_m^{(j)}+a_{1}x_{m-1}^{(j)}+\cdots+a_nx_{m-n}^{(j)}=0
	\end{equation*}
	于是:
	\begin{equation*}
		\sum_{j=1}^{k}x_m^{(j)}+a_1\sum_{j=1}^{k}x_{m-1}^{(j)}+\cdots+a_n\sum_{j=1}^{k}x_{m-n}^{j}=0
	\end{equation*}
	即:
	\begin{equation*}
		x_m+a_{1}x_{m-1}+\cdots+a_nx_{m-n}=0\qedhere
	\end{equation*}
\end{proof}
\begin{theorem}
	$n$阶齐次线性差分方程$L(x_m)=0$一定存在$n$个线性无关的解。
\end{theorem}
\begin{proof}
	由\cref{theo:LinearDifferenceEqExistenceUniqueness}可知$L(x_m)=0$满足初值条件:
	\begin{equation*}
		\begin{cases}
			x_0=1,\;x_1=0,\;\dots,\;x_{n-1}=0 \\
			x_0=0,\;x_1=1,\;\dots,\;x_{n-1}=0 \\
			\cdots\cdots\cdots \\
			x_0=0,\;x_1=0,\;\dots,\;x_{n-1}=1 \\
		\end{cases}
	\end{equation*}
	的解一定存在,由\cref{cor:CasoratiNe0LinearIndependent}可知这$n$个解线性无关。
\end{proof}
\begin{theorem}\label{theo:GeneralSolutionHomogeneousLinearDifferenceEquation}
	设$\{x_m^{(1)}\},\{x_m^{(2)}\},\dots,\{x_m^{(n)}\}$是$n$阶齐次线性差分方程$L(x_m)=0$的$n$个线性无关的解,则方程的通解可以表示为:
	\begin{equation*}
		x_m=c_1x_m^{(1)}+c_2x_m^{(2)}+\cdots+c_nx_m^{(n)},\;\forall\;m\in\mathbb{Z}^{}
	\end{equation*}
	其中$\seq{c}{n}$是任意常数。
\end{theorem}
\begin{proof}
	因为$\{x_m^{(1)}\},\{x_m^{(2)}\},\dots,\{x_m^{(n)}\}$线性无关,由\cref{theo:CasoratiEq0LinearDependent}可知存在$m\in\mathbb{Z}^{}$使得Casorati行列式$\det C(m)\ne0$,于是此时的Casorati矩阵$C(n)$可逆。任取$L(x_m)=0$的一个解$\{y_m\}$,则$y_m,\mathcal{B}y_m,\dots,\mathcal{B}^{n-1}y_m$是确定的数,于是关于$(\seq{c}{n})^T$的线性方程组:
	\begin{equation*}
		\begin{pmatrix}
			x_m^{(1)} & x_m^{(2)} & \cdots & x_m^{(n)} \\
			\mathcal{B}x_m^{(1)} & \mathcal{B}x_m^{(2)} & \cdots & \mathcal{B}x_m^{(n)} \\
			\vdots & \vdots & \ddots & \vdots \\
			\mathcal{B}^{n-1}x_m^{(1)} & \mathcal{B}^{n-1}x_m^{(2)} & \cdots & \mathcal{B}^{n-1}x_m^{(n)} \\
		\end{pmatrix}
		\begin{pmatrix}
			c_1 \\
			c_2 \\
			\cdots \\
			c_n
		\end{pmatrix}=
		\begin{pmatrix}
			y_m \\
			\mathcal{B}y_m \\
			\vdots \\
			\mathcal{B}^{n-1}y_m
		\end{pmatrix}
	\end{equation*}
	存在唯一解$(\seq{c^*}{n})^T$,即在这$n$个位置处$\{y_m\}$可以由$\{x_m^{(1)}\},\{x_m^{(2)}\},\dots,\{x_m^{(n)}\}$线性表出。令:
	\begin{equation*}
		z_m=c_1^*x_m^{(1)}+c_2^*x_m^{(2)}+\cdots+c_n^*x_m^{(n)},\;\forall\;m\in\mathbb{Z}^{}
	\end{equation*}
	则$\{z_m\}$在这$n$个位置处的值等于$\{y_m\}$对应位置上的值,且$\{z_m\}$可以由$\{x_m^{(1)}\},\{x_m^{(2)}\},\dots,\{x_m^{(n)}\}$线性表出。而由\cref{theo:LinearDifferenceEqExistenceUniqueness}可知$L(x_m)=0$的解由任意的$n$个初始值唯一确定,于是$\{y_m\}$就是$\{z_m\}$,所以它也可以由$\{x_m^{(1)}\},\{x_m^{(2)}\},\dots,\{x_m^{(n)}\}$线性表出。由$\{y_m\}$的任意性,结论成立。
\end{proof}
\begin{corollary}\label{cor:LinearSpaceSolutionHomogeneousLinearDifferenceEquation}
	$n$阶齐次线性差分方程的所有解构成一个$n$维线性空间。
\end{corollary}
\begin{definition}
	称$n$阶齐次线性差分方程的任意$n$个线性无关的解为其\textbf{基本解组}。
\end{definition}
\subsubsection{非齐次线性差分方程解的一般理论}
\begin{theorem}\label{theo:DifStillSolutionHomogeneousLinearDifferenceEquation}
	$n$阶非齐次线性差分方程$L(x_m)=f(m)$任意两个解的差是$n$阶齐次线性差分方程$L(x_m)=0$的解。
\end{theorem}
\begin{theorem}\label{theo:AddStillSolutionNonhomogeneousLinearDifferenceEquation}
	$n$阶非齐次线性差分方程$L(x_m)=f(m)$任意一个解与$n$阶齐次线性差分方程$L(x_m)=0$任意一个解的和还是$L(x_m)=f(m)$的解。
\end{theorem}
\begin{theorem}\label{theo:GeneralSolutionNonhomogeneousLinearDifferenceEquation}
	设$\{x_m^{(1)}\},\{x_m^{(2)}\},\dots,\{x_m^{(n)}\}$是$n$阶齐次线性差分方程$L(x_m)=0$的基本解组,$\{y_m\}$是$n$阶非齐次线性差分方程$L(x_m)=f(m)$的一个解,则$L(x_m)=f(m)$的通解可表示为:
	\begin{equation*}
		x_m=y_m+c_1x_m^{(1)}+c_2x_m^{(2)}+\cdots+c_nx_m^{(n)},\;\forall\;m\in\mathbb{Z}^{}
	\end{equation*}
\end{theorem}
\begin{proof}
	对任意的$\{y_m\}$,由\cref{theo:DifStillSolutionHomogeneousLinearDifferenceEquation}和\cref{theo:GeneralSolutionHomogeneousLinearDifferenceEquation}可知$L(x_m)=f(m)$的任意一个解可以表示为上述形式。由$\{y_m\}$的任意性可得出结论。
\end{proof}

\subsection{$n$阶常系数线性差分方程}
\begin{definition}
	若$n$阶线性差分方程:
	\begin{equation*}
		x_m+a_1(m)x_{m-1}+a_2(m)x_{m-2}+\cdots+a_n(m)x_{m-n}=f(m)
	\end{equation*}
	中$a_1(x),a_2(x),\dots,a_n(x)$都是常数,则称上述方程为关于$\{x_m\}$的$n$阶常系数线性差分方程。
\end{definition}
\subsubsection{常系数齐次线性差分方程解的一般理论}
\begin{derivation}
	取一个一阶常系数齐次线性差分方程$x_m+a_1x_{m-1}=0$,可以看出该方程的通解为$x_m=C(-a_1)^m,\;\forall\;m\in\mathbb{Z}^{}$,其中$C$为任意常数。受此启发,我们对于$n$阶常系数齐次线性差分方程寻找指数形式的解。
\end{derivation}
\begin{definition}
	称:
	\begin{equation*}
		\lambda^n+a_1\lambda^{n-1}+a_2\lambda^{n-2}+\cdots+a_{n-1}\lambda+a_n=0
	\end{equation*}
	为$n$阶常系数齐次线性差分方程:
	\begin{equation*}
		x_m+a_1x_{m-1}+a_2x_{m-2}+\cdots+a_nx_{m-n}=0
	\end{equation*}
	的\textbf{特征方程},等号左边关于$\lambda$的多项式被称为\textbf{特征多项式},记为$l(\lambda)$。
\end{definition}
\begin{theorem}\label{theo:GeneralSolutionHomogeneousLinearDifferenceEqConstantCoefficients}
	设$\seq{\lambda}{s}$是$n$阶常系数齐次线性差分方程$L(x_m)=0$的特征方程$l(\lambda)=0$的解,其重数分别为$\seq{r}{s}$,则$L(x_m)=0$的通解可以表示为:
	\begin{equation*}
		x_m=\sum_{i=1}^{s}\sum_{j=0}^{r_i-1}c_{ij}m^{j}\lambda_i^m,\;\forall\;m\in\mathbb{Z}^{}
	\end{equation*}
	其中$c_{ij},\;i=1,2,\dots,s,\;j=0,1,\dots,r_i-1$为常数。
\end{theorem}
\begin{proof}
	\textbf{情况一:}此时特征方程有$n$个互不相同的实根,分别设为$\seq{\lambda}{n}$。将$\{\lambda_i^m\},\;i=1,2,\dots,n$代入$L(x_m)=0$中可以发现:
	\begin{align*}
		&\lambda_i^m+a_1\lambda_i^{m-1}+a_2\lambda_i^{m-2}+\cdots+a_n\lambda_i^{m-n} \\
		=&\lambda_i^{m-n}(\lambda_i^n+a_1\lambda_i^{n-1}+a_2\lambda_i^{n-2}+\cdots+a_n)=0
	\end{align*}
	所以它们就是$L(x_m)=0$的解。由$n$阶线性差分方程的定义,$a_n\ne0$,所以$\lambda_i\ne0$。于是有:
	\begin{equation*}
		\det C(0)=
		\begin{vmatrix}
			1 & 1 & \cdots & 1 \\
			\lambda_1^{-1} & \lambda_2^{-1} & \cdots & \lambda_n^{-1} \\
			\lambda_1^{-2} & \lambda_2^{-2} & \cdots & \lambda_n^{-2} \\
			\vdots & \vdots & \ddots & \vdots \\
			\lambda_1^{-(n-1)} & \lambda_2^{-(n-1)} & \cdots & \lambda_n^{-(n-1)} \\
		\end{vmatrix}
		=\prod_{1\leqslant i<j\leqslant n}(\lambda_j-\lambda_i)\ne0
	\end{equation*}
	由\cref{cor:CasoratiNe0LinearIndependent}可知这$n$个解线性无关,所以构成一个基本解组。\par
	\textbf{情况二:}此时特征方程存在重实根。假设根$\lambda_i$的重数为$r_i$,代入可知$\{\lambda_i^m\}$是$L(x_m)=0$的解。下证明对$1\leqslant b\leqslant r_i-1$且$b\in\mathbb{N}^+$,$\{m^b\lambda_i^m\}$都是$L(x_m)=0$的解。设$a_0=1$,则:
	\begin{align*}
		&m^{b}\lambda_i^m+a_1(m-1)^{b}\lambda_i^{m-1}+\cdots+a_{n}(m-n)^{b}\lambda_i^{m-n} \\
		=&\sum_{k=0}^{n}a_k(m-k)^b\lambda_i^{m-k} =\sum_{k=0}^{n}a_k\sum_{j=0}^{b}\binom{b}{j}m^j(-k)^{b-j}\lambda_i^{m-k} \\
		=&\sum_{k=0}^{n}\sum_{j=0}^{b}a_k\binom{b}{j}m^j(-k)^{b-j}\lambda_i^{m-k}
		=\sum_{j=0}^{b}\sum_{k=0}^{n}a_k\binom{b}{j}m^j(-k)^{b-j}\lambda_i^{m-k} \\
		=&\sum_{j=0}^{b}\binom{b}{j}m^j\sum_{k=0}^{n}a_k(-k)^{b-j}\lambda_i^{m-k}=\lambda_i^m\sum_{j=0}^{b}\binom{b}{j}m^j\sum_{k=0}^{n}a_k(-k)^{b-j}\lambda_i^{-k}
	\end{align*}
	取函数$f(\lambda_i)=\sum\limits_{k=0}^{n}a_k\lambda_i^{-k}$,则:
	\begin{equation*}
		\sum_{k=0}^{n}a_k(-k)^{b-j}\lambda_i^{-k}
	\end{equation*}
	可以表示为$f(\lambda_i),f'(\lambda_i),\dots,f^{(b-j)}(\lambda_i)$的线性组合。考虑特征多项式$l(\lambda_i)$:
	\begin{equation*}
		l(\lambda_i)=\lambda_i^n+a_1\lambda_i^{n-1}+a_2\lambda_i^{n-2}+\cdots+a_{n-1}\lambda_i+a_n=\sum_{k=0}^{n}a_k\lambda_i^{n-k}=\lambda_i^nf(\lambda_i)
	\end{equation*}
	于是有:
	\begin{equation*}
		f(\lambda_i)=\frac{l(\lambda_i)}{\lambda_i^n}
	\end{equation*}
	所以$f^{(b-j)}(\lambda_i)$可以由$l(\lambda_i),l'(\lambda_i),\dots,l^{(b-j)}(\lambda_i)$线性表出。于是:
	\begin{align*}
		&m^{b}\lambda_i^m+a_1(m-1)^{b}\lambda_i^{m-1}+\cdots+a_{n}(m-n)^{b}\lambda_i^{m-n} \\
		=&\lambda_i^m\sum_{j=0}^{b}\binom{b}{j}m^j\sum_{k=0}^{n}a_k(-k)^{b-j}\lambda_i^{-k}=\lambda_i^m\sum_{j=0}^{b}\binom{b}{j}m^jg[l(\lambda_i),l'(\lambda_i),\dots,l^{(b-j)}(\lambda_i)]
	\end{align*}
	其中$g$是$l(\lambda_i),l'(\lambda_i),\dots,l^{(b-j)}(\lambda_i)$的线性函数。因为$\lambda_i$是特征方程的$r_i$重根,所以:
	\begin{equation*}
		l(\lambda)=(\lambda-\lambda_i)^{r_i}h(\lambda)=0
	\end{equation*}
	求导可知$l(\lambda_i)=l'(\lambda_i)=\cdots=l^{(b-j)}(\lambda_i)=0$,所以$g=0$,即:
	\begin{equation*}
		m^{b}\lambda_i^m+a_1(m-1)^{b}\lambda_i^{m-1}+\cdots+a_{n}(m-n)^{b}\lambda_i^{m-n}=0
	\end{equation*}
	所以$\{m^b\lambda_i^m\},\;b=0,1,2,\dots,r_i-1$都是$L(x_m)=0$的解。\par
	若$\{m^b\lambda_i^m\},\;b=0,1,2,\dots,r_i-1$线性相关,则存在不全为$0$的$\seq{c}{r_i}$使得:
	\begin{equation*}
		c_1\lambda_i^m+c_2m\lambda_i^m+\cdots+c_{r_i}m^{r_i-1}\lambda_i^m=\lambda_i^m(c_1+c_2m+\cdots+c_{r_i}m^{r_i-1})=0
	\end{equation*}
	对任意的$m\in\mathbb{Z}$成立。由$n$阶线性差分方程的定义,$a_n\ne0$,所以$\lambda_i\ne0$,于是需要上式中关于整数$m$的多项式恒等于$0$,此时应有$c_1=c_2=\cdots=c_{r_i}=0$,矛盾,所以$\{m^b\lambda_i^m\},\;b=0,1,2,\dots,r_i-1$线性无关。\info{不同$\lambda_i$之间的线性无关性涉及到广义Vandermonde行列式,以后再写}\par
	\textbf{情形三:}此时特征方程存在一对共轭复根$a+bi,a-bi$,令:
	\begin{equation*}
		\rho=\sqrt{a^2+b^2},\;\theta=\arctan\left(\frac{b}{a}\right)
	\end{equation*}
	则:
	\begin{gather*}
		(a+bi)^m=(\rho e^{i\theta})^m=\rho^m[\cos(m\theta)+i\sin(m\theta)],\;\forall\;m\in\mathbb{Z}^{} \\
		(a-bi)^m=(\rho e^{-i\theta})^m=\rho^m[\cos(m\theta)-i\sin(m\theta)],\;\forall\;m\in\mathbb{Z}^{}
	\end{gather*}
	都是$L(x_m)$的解。\par
	综上可得出定理的结论。
\end{proof}
\subsubsection{常系数齐次线性差分方程解的收敛性}
\begin{derivation}
	设$\seq{\lambda}{s}$为$n$阶常系数齐次线性差分方程$L(x_m)=0$的特征方程$l(\lambda)=0$的解。\par
	\textbf{(1)$\;\lambda_i$都在单位圆内:}此时存在$\alpha$使得:
	\begin{equation*}
		\min\{|\lambda_i|:i=1,2,\dots,s\}<\alpha<1
	\end{equation*}
	于是$L(x_m)=0$的任何解$\{x_m\}$满足:
	\begin{equation*}
		|x_m|=\left|\sum_{i=1}^{s}\sum_{j=0}^{r_i-1}c_{ij}m^{j}\lambda_i^m\right|\leqslant\sum_{i=1}^{s}\sum_{j=0}^{r_i-1}|c_{ij}m^{j}|\;|\lambda_i^m|\leqslant\sum_{i=1}^{s}\sum_{j=0}^{r_i-1}|c_{ij}m^{j}|\alpha^m
	\end{equation*}
	由指数函数$\alpha^m$与幂函数$m^j$的收敛速度比较可得:
	\begin{equation*}
		\lim_{m\to+\infty}|x_m|=0
	\end{equation*}
	此时称$\{x_m\}$\textbf{以负指数收敛到$0$}。\par
	\textbf{(2)$\;\lambda_i$在单位圆上:}设$\lambda_i=a+bi$,由\cref{theo:GeneralSolutionHomogeneousLinearDifferenceEqConstantCoefficients}可知此时$L(x_m)=0$有解:
	\begin{equation*}
		x_m=\cos(m\theta)+i\sin(m\theta),\;\theta=\arctan\left(\frac{b}{a}\right),\;\forall\;m\in\mathbb{Z}^{}
	\end{equation*}
	当$\theta=\dfrac{\pi}{4}$时,这个解不收敛,是一个周期解。\par
	\textbf{(3)$\;\lambda_i$在单位圆外:}显然此时存在发散于$+\infty$的解。
\end{derivation}
\subsubsection{常系数非齐次线性差分方程解的一般理论}
\begin{theorem}\label{theo:GeneralSolutionNonhomogeneousLinearDifferenceEqConstantCoefficients}
	设$\seq{\lambda}{s}$是$n$阶常系数齐次线性差分方程$L(x_m)=0$的特征方程$l(\lambda)=0$的互异根,其重数分别为$\seq{r}{s}$,$n$阶常系数非齐次线性差分方程$L(x_m)=f(m)$的一个特解为$\{\lambda^m\}$,则$L(x_m)=f(m)$的通解可以表示为:
	\begin{equation*}
		x_m=\lambda^m+\sum_{i=1}^{s}\sum_{j=0}^{r_i-1}c_{ij}m^{j}\lambda_i^m,\;\forall\;m\in\mathbb{Z}^{}
	\end{equation*}
	其中$c_{ij},\;i=1,2,\dots,s,\;j=0,1,\dots,r_i-1$为常数。
\end{theorem}
\begin{proof}
	由\cref{theo:GeneralSolutionNonhomogeneousLinearDifferenceEquation}和\cref{theo:GeneralSolutionHomogeneousLinearDifferenceEqConstantCoefficients}可直接得到。
\end{proof}


\subsection{AR模型}
\begin{definition}
	如果$\{\varepsilon_t\}$是$\operatorname{WN}(0,\sigma^2)$,实数$\seq{a}{p}(a_p\ne0)$使得多项式$A(z)=0$的根都在单位圆外:
	\begin{equation*}
		A(z)=1-\sum_{i=1}^{p}a_iz^i\ne0,\;\forall\;|z|\leqslant1
	\end{equation*}
	则称$p$阶常系数线性差分方程:
	\begin{equation*}
		X_t=\sum_{i=1}^{p}a_iX_{t-i}+\varepsilon_t
	\end{equation*}
	为$p$\textbf{阶自回归模型},简记为$\operatorname{AR}(p)$模型。称满足$\operatorname{AR}(p)$模型的平稳时间序列$\{X_t\}$为\textbf{平稳解}或$\operatorname{AR}(p)$序列,称$\seq{a}{p}$为$\operatorname{AR}(p)$模型的\textbf{自回归系数},多项式$A(z)=0$的根都在单位圆外这一条件被称为\textbf{稳定性条件},分别称$A(z)$和$A(\mathcal{B})$为$\operatorname{AR}(p)$模型的\textbf{特征多项式}和\textbf{自回归系数多项式}。可以用$A(\mathcal{B})$将模型改写为:
	\begin{equation*}
		A(\mathcal{B})X_t=\varepsilon_t
	\end{equation*}
\end{definition}
\begin{derivation}
	$A(z)=0$的解为$p$阶常系数线性差分方程特征方程:
	\begin{equation*}
		\lambda^p-a_1\lambda^{p-1}-\cdots-a_{p-1}\lambda-a_p=0
	\end{equation*}
	的解的倒数。取特征方程的任一根$\lambda_j$,则:
	\begin{equation*}
		A\left(\frac{1}{\lambda_j}\right)=1-\sum_{i=1}^{p}a_i\frac{1}{\lambda_j^i}=\frac{1}{\lambda_j^p}(\lambda_j^p-a_1\lambda_j^{p-1}-\cdots-a_{p-1}\lambda_j-a_{p})=0
	\end{equation*}
	所以上述稳定性条件即为要求$p$阶常系数线性差分方程特征方程的根都在单位圆内。
\end{derivation}
\begin{theorem}
	设$\operatorname{AR}(p)$模型的特征多项式$A(z)=0$有$s$个互异根$\seq{z}{s}$,根的重数分别为$\seq{r}{s}$,则:
	\begin{enumerate}
		\item $\operatorname{AR}(p)$模型的唯一平稳解是:
		\begin{equation*}
			X_t=\sum_{i=0}^{+\infty}\psi_i\varepsilon_{t-i}
		\end{equation*}
		其中$\psi_j$满足:
		\begin{equation*}
			A^{-1}()
		\end{equation*}
	\end{enumerate}
\end{theorem}

