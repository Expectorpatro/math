\section{平稳时间序列}

\begin{definition}
	如果时间序列$\{X_t\}=\{X_t:t\in\mathbb{N}\}$满足:
	\begin{enumerate}
		\item 对任何的$t\in\mathbb{N}$,有$\operatorname{E}(X_t^2)<+\infty$;
		\item 对任何的$t\in\mathbb{N}$,有$\operatorname{E}(X_t)=\mu$;
		\item 对任何的$t,s\in\mathbb{N}$,有$\operatorname{Cov}(X_t,X_s)=\operatorname{E}[(X_t-\mu)(X_s-\mu)]=\gamma(t-s)$,其中$\gamma(t-s)$是$t-s$的实值函数,被称为$\{X_t\}$的\gls{AutocovarianceF}。
	\end{enumerate}
	则称$\{X_t\}$是一个\gls{StationaryTimeSeries}。
\end{definition}

\subsection{平稳时间序列的性质}
\subsubsection{线性变换}
\begin{theorem}
	平稳时间序列$\{X_t\}$经过线性变换后得到的还是平稳时间序列。
\end{theorem}
\begin{proof}
	只需证明$\{Y_t=aX_t+b:t\in\mathbb{N}\}$对任意的$a,b\in\mathbb{R}$是平稳时间序列。设$\operatorname{E}(X_t)=\mu$。\par
	(1)对于线性变换后时间序列的期望,有:
	\begin{equation*}
		\operatorname{E}(Y_t)=\operatorname{E}(aX_t+b)=a\mu+b
	\end{equation*}\par
	(2)对于线性变换后时间序列的二阶原点矩,有:
	\begin{equation*}
		\operatorname{E}(Y_t^2)=\operatorname{Var}(Y_t)+[\operatorname{E}(Y_t)]^2=a^2\gamma(0)+(a\mu+b)^2<+\infty\qedhere
	\end{equation*}\par
	(3)对于线性变换后时间序列的协方差,有:
	\begin{equation*}
		\operatorname{Cov}(Y_t,Y_s)=\operatorname{E}[(aX_t+b-a\mu-b)(aX_s+b-a\mu-b)]=a^2\gamma(t-s)
	\end{equation*}
\end{proof}
\subsubsection{平稳时间序列的谱函数}
\begin{definition}
	设平稳时间序列$\{X_t\}$的自协方差函数为$\gamma(n)$。
	\begin{enumerate}
		\item 若存在$[-\pi,\pi]$上单调不减且右连续的函数$F(\lambda)$使得:
		\begin{equation*}
			\gamma(n)=\int_{[-\pi,\pi]}e^{in\lambda}\dif F(\lambda),\quad F(-\pi)=0,\;k\in\mathbb{Z}^{}
		\end{equation*}
		则称$F(\lambda)$为$\{X_t\}$或$\{\gamma(n)\}$的\gls{SpectralDistributionFunction},简称为\textbf{谱函数};
		\item 若存在$[-\pi,\pi]$上的非负函数$f(\lambda)$使得:
		\begin{equation*}
			\gamma(n)=\int_{[-\pi,\pi]}f(\lambda)e^{in\lambda}\dif\lambda,\quad n\in\mathbb{Z}^{}
		\end{equation*}
		则称$f(\lambda)$为$\{X_t\}$或$\{\gamma(n)\}$的\gls{SpectralDensityFunction},简称为\textbf{谱密度}。
	\end{enumerate}
\end{definition}
\begin{theorem}[Herglotz theorem]
	平稳时间序列的谱函数存在且唯一。
\end{theorem}
\begin{theorem}\label{theo:SpectralGamma}
	若平稳时间序列$\{X_t\}$的自协方差函数$\{\gamma(n)\}\in l^1$,则$\{X_t\}$有谱密度:
	\begin{equation*}
		f(\lambda)=\frac{1}{2\pi}\sum_{n=-\infty}^{+\infty}\gamma(n)e^{-in\lambda}
	\end{equation*}
\end{theorem}
\subsubsection{正交与不相关}
\begin{definition}
	设$\{X_t\}$和$\{Y_t\}$是平稳时间序列。
	\begin{enumerate}
		\item 若对任何的$s,t\in\mathbb{N}$,有$\operatorname{E}(X_tY_s)=0$,则称$\{X_t\}$和$\{Y_t\}$\textbf{正交}。
		\item 若对任何的$s,t\in\mathbb{N}$,有$\operatorname{Cov}(X_t,Y_s)=0$,则称$\{X_t\}$和$\{Y_t\}$\textbf{不相关}。
	\end{enumerate}
\end{definition}
\begin{theorem}\label{theo:StationarySeriesOrthogonalUncorrelated}
	对于期望为$0$的平稳时间序列,正交性和不相关性等价。
\end{theorem}
\begin{proof}
	若$\mu=0$,则:
	\begin{equation*}
		\operatorname{Cov}(X_t,X_s)=\operatorname{E}[(X_t-\mu)(X_s-\mu)]=\operatorname{E}(X_tX_s)\qedhere
	\end{equation*}
\end{proof}
\begin{theorem}\label{theo:StationarySeriesSum}
	设$\gamma_X(t)$和$\gamma_Y(t)$分别是平稳时间序列$\{X_t\}$和$\{Y_t\}$的自协方差函数。记$\mu_X=\operatorname{E}(X_t),\;\mu_Y=\operatorname{E}(Y_t)$,定义:
	\begin{equation*}
		Z_t=X_t+Y_t,\;t\in\mathbb{N}
	\end{equation*}
	则:
	\begin{enumerate}
		\item 若$\{X_t\}$和$\{Y_t\}$正交,则$\{Z_t\}$是平稳时间序列,有自协方差函数:
		\begin{equation*}
			\gamma_Z(t)=\gamma_X(t)+\gamma_Y(t)-2\mu_X\mu_Y,\;t\in\mathbb{N}
		\end{equation*}
		\item 若$\{X_t\}$和$\{Y_t\}$不相关,则$\{Z_t\}$是平稳时间序列,有自协方差函数:
		\begin{equation*}
			\gamma_Z(t)=\gamma_X(t)+\gamma_Y(t),\;t\in\mathbb{N}
		\end{equation*}
	\end{enumerate}
\end{theorem}
\begin{proof}
	因为对任意的$t\in\mathbb{N}$,有$(X_t+Y_t)^2\leqslant2X_t^2+2Y_t^2$,所以:
	\begin{equation*}
		\operatorname{E}(Z_t^2)\leqslant2\operatorname{E}(X_t^2)+2\operatorname{E}(Y_t^2)<+\infty
	\end{equation*}\par
	显然:
	\begin{equation*}
		\operatorname{E}(Z_t)=\operatorname{E}(X_t)+\operatorname{E}(Y_t)=\mu_X+\mu_Y
	\end{equation*}
	与$t$无关。因为:
	\begin{align*}
		\operatorname{Cov}(Z_t,Z_s)
		&=\operatorname{Cov}(X_t+Y_t,X_s+Y_s) \\
		&=\operatorname{Cov}(X_t,X_s)+\operatorname{Cov}(X_t,Y_s)+\operatorname{Cov}(Y_t,X_s)+\operatorname{Cov}(Y_t,Y_s)
	\end{align*}\par
	(1)由于$\{X_t\}$和$\{Y_t\}$正交,所以$\operatorname{E}(X_tY_s)=\operatorname{E}(X_sY_t)=0$,于是由\cref{prop:CovMat}(6)可得:
	\begin{align*}
		\operatorname{Cov}(Z_t,Z_s)
		&=\gamma_X(t-s)+\gamma_Y(t-s)+\operatorname{E}(X_tY_s)-\operatorname{E}(X_t)\operatorname{E}(Y_s)+\operatorname{E}(X_sY_t)-\operatorname{E}(Y_t)\operatorname{E}(X_s) \\
		&=\gamma_X(t-s)+\gamma_Y(t-s)-2\mu_X\mu_Y
	\end{align*}
	所以$\operatorname{Cov}(Z_t,Z_s)$只与$(t-s)$有关。综上,$\{Z_t\}$是平稳时间序列,且
	\begin{equation*}
		\gamma_Z(t)=\gamma_X(t)+\gamma_Y(t)-2\mu_X\mu_Y,\;t\in\mathbb{N}
	\end{equation*}\par
	(2)因为$\{X_t\}$和$\{Y_t\}$不相关,所以$\operatorname{Cov}(X_t,Y_s)=0=\operatorname{Cov}(Y_t,X_s)=0$,于是:
	\begin{equation*}
		\operatorname{Cov}(Z_t,Z_s)=\operatorname{Cov}(X_t,X_s)+\operatorname{Cov}(Y_t,Y_s)=\gamma_X(t-s)+\gamma_Y(t-s)
	\end{equation*}
	所以$\operatorname{Cov}(Z_t,Z_s)$只与$(t-s)$有关。综上,$\{Z_t\}$是平稳时间序列,且
	\begin{equation*}
		\gamma_Z(t)=\gamma_X(t)+\gamma_Y(t),\;t\in\mathbb{N}\qedhere
	\end{equation*}
\end{proof}
\begin{theorem}
	设$\{X_t\}$和$\{Y_t\}$是正交的平稳时间序列,且$\operatorname{E}(X_t)=\operatorname{E}(Y_t)=0$,$c$是常数,定义:
	\begin{equation*}
		Z_t=X_t+Y_t+c,\quad t\in\mathbb{Z}^{}
	\end{equation*}
	\begin{enumerate}
		\item 如果$\{X_t\}$和$\{Y_t\}$分别有谱函数$F_X(\lambda)$和$F_Y(\lambda)$,则平稳时间序列$\{Z_t\}$有谱函数$F_Z(\lambda)=F_X(\lambda)+F_Y(\lambda)$;
		\item 如果$\{X_t\}$和$\{Y_t\}$分别有谱密度$f_X(\lambda)$和$f_Y(\lambda)$,则平稳时间序列$\{Z_t\}$有谱密度$f_Z(\lambda)=f_X(\lambda)+f_Y(\lambda)$。
	\end{enumerate}
\end{theorem}
\begin{proof}
	有\cref{theo:StationarySeriesOrthogonalUncorrelated}和\cref{theo:StationarySeriesSum}可知$\{Z_t\}$是平稳时间序列且有自协方差函数:
	\begin{equation*}
		\gamma_Z(n)=\gamma_X(n)+\gamma_Y(n)
	\end{equation*}\par
	(1)此时有:
	\begin{align*}
		\gamma_Z(n)&=\gamma_X(n)+\gamma_Y(n) \\
		&=\int_{[-\pi,\pi]}e^{in\lambda}\dif F_X(\lambda)+\int_{[-\pi,\pi]}e^{in\lambda}\dif F_Y(\lambda) \\
		&=\int_{[-\pi,\pi]}e^{in\lambda}\dif [F_X(\lambda)+F_Y(\lambda)]
	\end{align*}\par
	(2)此时有:
	\begin{align*}						     \gamma_Z(n)&=\gamma_X(n)+\gamma_Y(n) \\
		&=\int_{[-\pi,\pi]}f_X(\lambda)e^{in\lambda}\dif\lambda+\int_{[-\pi,\pi]}f_Y(\lambda)e^{in\lambda}\dif\lambda \\
		&=\int_{[-\pi,\pi]}[f_X(\lambda)+f_Y(\lambda)]e^{in\lambda}\dif\lambda\qedhere
	\end{align*}
\end{proof}
\subsubsection{自协方差函数}
\begin{property}
	平稳时间序列的自协方差函数具有如下基本性质:
	\begin{enumerate}
		\item \textbf{对称性:}$\forall\;n\in\mathbb{N},\;\gamma(n)=\gamma(-n)$;
		\item \textbf{半正定性:}对任意的$n\in\mathbb{N}^+$,$n$阶自协方差矩阵:
		\begin{equation*}
			\Gamma_n=
			\begin{pmatrix}
				\gamma(0) & \gamma(1) & \cdots & \gamma(n-1) \\
				\gamma(1) & \gamma(0) & \cdots & \gamma(n-2) \\
				\vdots & \vdots & \ddots & \vdots \\
				\gamma(n-1) & \gamma(n-2) & \cdots & \gamma(0) \\
			\end{pmatrix}
		\end{equation*}
		是半正定矩阵;
		\item \textbf{有界性:}$|\gamma(n)|\leqslant|\gamma(0)|$对所有的$n\in\mathbb{N}$成立。
	\end{enumerate}
\end{property}
\begin{proof}
	(1)由协方差的定义:
	\begin{equation*}
		\gamma(n)=\operatorname{E}[(X_{t+n}-\mu)(X_t-\mu)]=\operatorname{E}[(X_t-\mu)(X_{t+n}-\mu)]=\gamma(-n)
	\end{equation*}\par
	(2)任取$n$维实数向量$\alpha=(\seq{a}{n})^T$有:
	\begin{align*}
		a^T\Gamma_na
		&=\sum_{i=1}^{n}\sum_{j=1}^{n}a_ia_j\gamma_{i-j} \\
		&=\sum_{i=1}^{n}\sum_{j=1}^{n}a_ia_j\operatorname{E}[(X_i-\mu)(X_j-\mu)]\\
		&=\operatorname{E}\left[\sum_{i=1}^{n}\sum_{j=1}^{n}a_ia_j(X_i-\mu)(X_j-\mu)\right] \\
		&=\operatorname{E}\left[\sum_{i=1}^{n}a_i(X_i-\mu)\right]^2\geqslant0
	\end{align*}\par
	(3)对任意的$n\in \mathbb{N}$,由\cref{ineq:cauchy-schiwarz-expectations}可得:
	\begin{align*}
		|\gamma(n)|
		&=\Bigl|\operatorname{E}[(X_{n+1}-\mu)(X_1-\mu)]\Bigr| \\
		&\leqslant\sqrt{\operatorname{E}[(X_{n+1}-\mu)^2]\operatorname{E}[(X_1-\mu)^2]} \\
		&=\sqrt{\gamma(0)\gamma(0)}=|\gamma(0)|\qedhere
	\end{align*}
\end{proof}
\begin{definition}
	任何满足上述三个基本性质的实数序列都被称为\textbf{非负定序列}。
\end{definition}
\begin{theorem}\label{theo:GammaPositiveDefinite}
	设$\Gamma_n$是平稳时间序列$\{X_t\}$的$n$阶自协方差矩阵。
	\begin{enumerate}
		\item 如果$\{X_t\}$的谱密度$f(\lambda)$存在,则对任何的$n\geqslant1$,$\Gamma_n$正定;
		\item 如果当$n\to+\infty$时有$\gamma(n)\to0$,则对任何的$n\geqslant1$,$\Gamma_n$正定。
	\end{enumerate}
\end{theorem}
\begin{proof}
	(1)任取$n$维实向量$\alpha=(\seq{a}{n})^T\ne\mathbf{0}$,
\end{proof}
\subsubsection{自相关函数}
\begin{definition}
	$\{X_t\}$是一个平稳时间序列,称平稳时间序列:
	\begin{equation*}
		Y_t=\frac{X_t-\mu}{\sqrt{\gamma(0)}},\;t\in\mathbb{N}
	\end{equation*}
	为$\{X_t\}$的标准化序列。称$\{Y_t\}$的自协方差函数$\rho(t)$为$\{X_t\}$的\gls{AutocorrelationF}。因为自协方差函数都是非负定序列,所以$\rho(t)$也是非负定序列。
\end{definition}
\begin{theorem}\label{theo:RhotGamma0}
	设$\{X_t\}$的自协方差函数为$\gamma(t)$,$\{Y_t\}$是$\{X_t\}$的标准化序列同时它的自协方差函数为$\rho(t)$,则:
	\begin{equation*}
		\rho(t)=\frac{\gamma(t)}{\gamma(0)},\;t\in\mathbb{N}
	\end{equation*}
\end{theorem}
\begin{proof}
	显然$\mu_Y=\operatorname{E}(Y_t)=0$,由自协方差函数的定义:
	\begin{equation*}
		\rho(t)=\operatorname{E}[(Y_t-\mu_Y)(Y_0-\mu_Y)]=\operatorname{E}(Y_tY_0)=\operatorname{E}\left[\left(\frac{X_t-\mu}{\sqrt{\gamma(0)}}\right)\left(\frac{X_0-\mu}{\sqrt{\gamma(0)}}\right)\right]=\frac{\gamma(t)}{\gamma(0)}\qedhere
	\end{equation*}
\end{proof}
\subsubsection{线性相关}
\begin{definition}\label{def:RVLinearlyDependent}
	对于随机变量$\seq{X}{n}$,若存在非零的$n$维实向量$\alpha=(\seq{a}{n})^T$使得:
	\begin{equation*}
		\operatorname{Var}\left[\sum_{i=1}^{n}a_i(X_i-\mu)\right]=0
	\end{equation*}
	则称$\seq{X}{n}$线性相关。
\end{definition}
\begin{lemma}\label{lem:aTaInvertible}
	对称阵$A\in M_{n}(\mathbb{R})$退化的充分必要条件为存在$\mathbb{R}^{n}$中的非零向量$\alpha$使得$\alpha^TA\alpha=0$。
\end{lemma}
\begin{proof}
	\textbf{(1)充分性:}由实对称矩阵的正交相似,有$Q^TAQ=B$,其中$Q$是正交矩阵,$B=\operatorname{diag}\{b_1,b_2,\dots,b_n\}$是主对角线上为$A$特征值的对角矩阵。假设此时$A$可逆,则它的所有特征值都不为$0$(否则就有$|A|=|Q^TBQ|=|Q^T|\;|B|\;|Q|=|B|=0$)。由题设存在非零向量$\alpha$使得$\alpha^TA\alpha=0$,因为$Q$是正交矩阵,所以$Q^{-1}$可逆,于是$Q^{-1}x=\mathbf{0}$只有零解,所以$Q^{-1}\alpha\ne\mathbf{0}$。设$Q^{-1}\alpha=\beta=(y_1,y_2,\dots,y_n)^T$,则有:
	\begin{equation*}
		\beta^TB\beta=\alpha^T(Q^{-1})^TBQ^{-1}\alpha=\alpha^T(Q^T)^{-1}BQ^{-1}\alpha=\alpha^TA\alpha=0
	\end{equation*}
	注意到:
	\begin{equation*}
		\beta^TB\beta=\sum_{i=1}^{n}b_iy_i^2
	\end{equation*}
	所以有:
	\begin{equation*}
		\sum_{i=1}^{n}b_iy_i^2=0
	\end{equation*}
	但是此时$b_i$都不是$0$,若上式成立需要$y_i$都为$0$,这就与$\beta\ne\mathbf{0}$矛盾,所以$A$退化。\par
	\textbf{(2)必要性:}如果$A$退化,则存在非零向量$\alpha$使得$A\alpha=\mathbf{0}$,显然此时$\alpha^TA\alpha=0$。
\end{proof}
\begin{theorem}\label{theo:TSLinearlyIndependent}
	$\{X_t\}$是一个平稳时间序列,$\Gamma_n$是$\{X_t\}$的$n$阶自协方差矩阵,则$\Gamma_n$退化的充要条件是对任意的$t\in\mathbb{N}$,$X_{t+1},X_{t+2},\dots,X_{t+n}$线性相关。
\end{theorem}
\begin{proof}
	由\cref{lem:aTaInvertible}可知$\Gamma_n$退化的充要条件是存在非零的$n$维实向量$\alpha=(\seq{a}{n})^T$使得$\alpha^T\Gamma_n\alpha=0$,而:
	\begin{align*}
		\alpha^T\Gamma_n\alpha
		&=\sum_{i=1}^{n}\sum_{j=1}^{n}a_ia_j\operatorname{E}[(X_i-\mu)(X_j-\mu)] =\sum_{i=1}^{n}\sum_{j=1}^{n}a_ia_j\operatorname{E}[(X_{t+i}-\mu)(X_{t+j}-\mu)] \\
		&=\operatorname{E}\left[\sum_{i=1}^{n}a_i(X_{t+i}-\mu)\right]^2 =\operatorname{E}\left[\sum_{i=1}^{n}a_i(X_{t+i}-\mu)\right]^2+0 \\
		&=\operatorname{E}\left[\sum_{i=1}^{n}a_i(X_{t+i}-\mu)\right]^2+\left\{\operatorname{E}\left[\sum_{i=1}^{n}a_i(X_{t+i}-\mu)\right]\right\}^2 =\operatorname{Var}\left[\sum_{i=1}^{n}a_i(X_{t+i}-\mu)\right]
	\end{align*}
	由随机变量线性相关的定义结论得证。
\end{proof}
\begin{theorem}
	$\{X_t\}$是一个平稳时间序列,$\Gamma_n$是$\{X_t\}$的$n$阶自协方差矩阵。若$\Gamma_n$退化,只要$m>n$,则有$\Gamma_m$退化,也即若$X_{t+1},X_{t+2},\dots,X_{t+n}$线性相关,只要$m>n$,则有$X_{t+1},X_{t+2},\dots,X_{t+m}$线性相关。
\end{theorem}
\begin{proof}
	由\cref{def:RVLinearlyDependent}与\cref{theo:TSLinearlyIndependent}可直接得出,只需在方差公式中取$a_{n+1}=a_{n+2}=\cdots=a_{m}=0$即可。
\end{proof}
\subsubsection{最小序列}
\begin{definition}
	用$L^2(X)$表示平稳时间序列$\{X_t\}$中有限个随机变量线性组合的全体:
	\begin{equation*}
		L^2(X)=\left\{\sum_{i=1}^{n}a_iX_{t_i}:a_i\in\mathbb{R}^{},\;t_i\in\mathbb{Z},\;k\in\mathbb{N}^+\right\}
	\end{equation*}
\end{definition}
\begin{property}\label{prop:L2TSX}
	$L^2(X)$有如下性质:
	\begin{enumerate}
		\item $L^2(X)\subset L^2$;
		\item 在$L^2(X)$中定义内积$(X,Y)=\operatorname{E}(XY)$,则$L^2(X)$成为一个Hilbert空间。
	\end{enumerate}
\end{property}
\begin{definition}
	设$\{X_t\}$是平稳序列,用$H_x$表示$L^2(X)$,用$H_x(s)$表示$\{X_t:t\ne s\}$产生的Hilbert空间。若存在$s\in\mathbb{Z}^{}$使得$H_x\ne H_x(s)$,则称$\{X_t\}$是\textbf{最小序列}。
\end{definition}
\begin{property}
	设$\{X_t\}$是平稳序列,有谱密度$f(\lambda)$,则:
	\begin{enumerate}
		\item 若$\{X_t\}$是最小序列,则对所有的$t\in\mathbb{Z}^{}$有$H_x\ne H_x(s)$;
		\item $\{X_t\}$是最小序列的充分必要条件为:
		\begin{equation*}
			\int_{[-\pi,\pi]}\frac{\dif\lambda}{f(\lambda)}<+\infty
		\end{equation*}
		\item 若$f(\lambda)$连续且恒正,则$\{X_t\}$是最小序列。
	\end{enumerate}
\end{property}
\subsubsection{长短记忆}
\begin{definition}
	根据自协方差函数$\gamma(n)$收敛到$0$的速度将平稳时间序列$\{X_t\}$分为\textbf{长记忆序列}和\textbf{短记忆序列}。\par
	对实数$d<0.5$,若:
	\begin{equation*}
		\lim_{n\to+\infty}\frac{\gamma(n)}{n^{2d-1}}>0
	\end{equation*}
	即$\gamma(n)$与$n^{2d-1}$是同阶无穷小,则称$\{X_t\}$是长记忆序列。
\end{definition}

\subsection{线性平稳序列}
\subsubsection{白噪声}
\begin{definition}
	设$\{\varepsilon_t\}$是一个平稳时间序列。如果对任何的$s,t\in\mathbb{N}$,有:
	\begin{equation*}
		\operatorname{E}(\varepsilon_t)=\mu,\;
		\operatorname{Cov}(\varepsilon_t,\varepsilon_s)=
		\begin{cases}
			\sigma^2, & t=s \\
			0, & t\ne s
		\end{cases}
	\end{equation*}
	则称$\{\varepsilon_t\}$是\gls{WhiteNoise},记作WN$(\mu,\sigma^2)$。当$\{\varepsilon_t\}$是独立序列时,称$\{\varepsilon_t\}$是\textbf{独立白噪声};当$\mu=0$时,称$\{\varepsilon_t\}$是\textbf{零均值白噪声};当$\mu=0,\;\sigma^2=1$时,称$\{\varepsilon_t\}$是\textbf{标准白噪声};当$\varepsilon_t$服从正态分布时,称$\{\varepsilon_t\}$是\textbf{正态白噪声}。
\end{definition}
\subsubsection{有限滑动平均}
\begin{definition}
	$\{\varepsilon_t\}=\{\varepsilon_t:t\in\mathbb{Z}\}$是$WN(0,\sigma^2)$。称:
	\begin{equation*}
		X_t=a_0\varepsilon_t+a_1\varepsilon_{t-1}+\cdots+a_q\varepsilon_{t-q}
	\end{equation*}
	是白噪声$\{\varepsilon_t\}$的\gls{FiniteMovingAverage},其中$q\in\mathbb{N}$,$a_0,a_1,\dots,a_q$为常数。
\end{definition}
\begin{theorem}\label{theo:ECovFiniteMovingAverage}
	$\{\varepsilon_t\}=\{\varepsilon_t:t\in\mathbb{Z}\}$是$WN(0,\sigma^2)$,则该白噪声的有限滑动平均:
	\begin{equation*}
		X_t=a_0\varepsilon_t+a_1\varepsilon_{t-1}+\cdots+a_q\varepsilon_{t-q}
	\end{equation*}
	构成的序列$\{X_t\}$具有如下均值与自协方差函数:
	\begin{equation*}
		\operatorname{E}(X_t)=0,\;
		\gamma(n)=
		\begin{cases}
			\sigma^2\sum\limits_{i=0}^{q-n}a_ia_{i+n}, & 0\leqslant n\leqslant q \\
			0, & n>q
		\end{cases},\;t,n\in\mathbb{Z}
	\end{equation*}
\end{theorem}
\begin{proof}
	由有限滑动平均的定义,对于$\{X_t\}$的期望有:
	\begin{equation*}
		\operatorname{E}(X_t)=\operatorname{E}\left(\sum_{i=0}^{q}a_i\varepsilon_{t-i}\right)=\sum_{i=0}^{q}a_i\operatorname{E}(\varepsilon_{t-i})=0,\;t\in\mathbb{Z}
	\end{equation*}
	对于$\{X_t\}$的自协方差函数,当$n>q$时,$t+n-i>t-j$恒成立,由白噪声的定义可得$\operatorname{E}(\varepsilon_{t+n-i}\varepsilon_{t-j})=0,\;\forall\;i,j=0,1,2,\dots,q$,于是$\gamma(n)=0$。当$0\leqslant n\leqslant q$时,有:
	\begin{align*}
		\gamma(n)
		&=\operatorname{E}(X_{t+n}X_t) \\
		&=\operatorname{E}[(a_{n}\varepsilon_{t}+a_{n+1}\varepsilon_{t-1}+\cdots+a_{q}\varepsilon_{t+n-q})(a_0\varepsilon_t+a_1\varepsilon_{t-1}+\cdots+a_{q-n}\varepsilon_{t+n-q})] \\
		&=\operatorname{E}(a_na_0\varepsilon_t^2+a_{n+1}a_1\varepsilon_{t-1}^2+\cdots+a_qa_{q-n}\varepsilon_{t+n-q}^2) \\
		&=\sum_{i=0}^{q-n}a_ia_{i+n}\operatorname{E}(\varepsilon_{t-i}^2)
		=\sum_{i=0}^{q-n}a_ia_{i+n}\sigma^2
	\end{align*}
	综上可得:
	\begin{equation*}
		\gamma(n)=
		\begin{cases}
			\sigma^2\sum\limits_{i=0}^{q-n}a_ia_{i+n}, & 0\leqslant n\leqslant q \\
			0, & n>q
		\end{cases},\;n\in\mathbb{N}\qedhere
	\end{equation*}
\end{proof}
\subsubsection{单边滑动平均}
\begin{definition}
	$\{\varepsilon_t\}=\{\varepsilon_t:t\in\mathbb{Z}\}$是$WN(0,\sigma^2)$。称:
	\begin{equation*}
		X_t=\sum_{i=0}^{+\infty}a_i\varepsilon_{t-i},\;t\in\mathbb{Z}
	\end{equation*}
	是白噪声$\{\varepsilon_t\}$的\textbf{单边滑动平均},其中$a_0,a_1,\dots$为常数。它表明当前的观测$X_t$只与$t$时刻以及之前时刻的白噪声相关,与$t$时刻之后的白噪声无关。
\end{definition}
\subsubsection{无穷滑动平均}
\begin{definition}
	$\{\varepsilon_t\}=\{\varepsilon_t:t\in\mathbb{Z}\}$是$WN(0,\sigma^2)$,$\{a_n\}\in l^1$。称:
	\begin{equation*}
		X_t=\sum_{i=-\infty}^{+\infty}a_i\varepsilon_{t-i}
	\end{equation*}
	是白噪声$\{\varepsilon_t\}$的\gls{InfiniteMovingAverage}。
\end{definition}
\begin{theorem}\label{theo:l1LinearlyStationarySeries}
	$\{\varepsilon_t\}=\{\varepsilon_t:t\in\mathbb{Z}\}$是$WN(0,\sigma^2)$,则该白噪声的无穷滑动平均:
	\begin{equation*}
		X_t=\sum_{i=-\infty}^{+\infty}a_i\varepsilon_{t-i}
	\end{equation*}
	构成的序列$\{X_t\}$是平稳序列且具有如下均值与自协方差函数:
	\begin{equation*}
		\operatorname{E}(X_t)=0,\;\gamma(n)=\sigma^2\sum_{i=-\infty}^{+\infty}a_ia_{i+n}
	\end{equation*}
\end{theorem}
\begin{proof}
	由\cref{prop:NonnegativeMeasurableIntegral}(4)(6)可得:
	\begin{align*}
		\operatorname{E}\left(\sum_{i=-\infty}^{+\infty}|a_i\varepsilon_{t-i}|\right)&=\operatorname{E}\left[\lim_{n\to+\infty}\left(\sum_{i=-n}^{n}|a_i||\varepsilon_{t-i}|\right)\right]=\lim_{n\to+\infty}\operatorname{E}\left(\sum_{i=-n}^{n}|a_i||\varepsilon_{t-i}|\right) \\
		&=\lim_{n\to+\infty}\left[\sum_{i=-n}^{n}|a_i|\operatorname{E}(|\varepsilon_{t-i})\right]=\sum_{i=-\infty}^{+\infty}|a_i|\operatorname{E}(|\varepsilon_{t-i})
	\end{align*}
	由\cref{ineq:cauchy-schiwarz-expectations}可得:
	\begin{equation*}
		\operatorname{E}(|\varepsilon_{t-i}|)=\Big|\operatorname{E}(|\varepsilon_{t-i}|\cdot1)\Big|\leqslant\sqrt{\operatorname{E}(\varepsilon_{t-i}^2)\operatorname{E}(1)}=\sigma
	\end{equation*}
	于是:
	\begin{equation*}
		\operatorname{E}\left(\sum_{i=-\infty}^{+\infty}|a_i\varepsilon_{t-i}|\right)=\sum_{i=-\infty}^{+\infty}|a_i|\operatorname{E}(|\varepsilon_{t-i}|)\leqslant\sigma\sum_{i=-\infty}^{+\infty}|a_i|<+\infty
	\end{equation*}
	由\cref{prop:NonnegativeMeasurableIntegral}(9)可得对任意的$t\in\mathbb{Z}^{}$,$X_t\;$a.e.有限,即:
	\begin{equation*}
		X_t=\sum_{i=-\infty}^{+\infty}a_i\varepsilon_{t-i}
	\end{equation*}
	右式a.e.收敛。取控制函数$\sum\limits_{i=-\infty}^{+\infty}|a_i\varepsilon_{t-i}|$,由\cref{theo:DominatedConvergenceTheorem}和\cref{prop:MeasurableIntegral}(6)可得
	\begin{equation*}
		\operatorname{E}(X_t)=\operatorname{E}\left[\lim_{n\to+\infty}\left(\sum_{i=-n}^{n}a_i\varepsilon_{t-i}\right)\right]=\lim_{n\to+\infty}\left[\operatorname{E}\left(\sum_{i=-n}^{n}a_i\varepsilon_{t-i}\right)\right]=0
	\end{equation*}\par
	对$t,s\in\mathbb{Z}$定义:
	\begin{equation*}
		\varphi_n=\sum_{i=-n}^{n}a_i\varepsilon_{t-i},\quad\psi_n=\sum_{j=-n}^{n}a_j\varepsilon_{s-j}
	\end{equation*}
	则有$\varphi_n\psi_n\to X_tX_s$,因为对任意的$t,s\in\mathbb{Z}^{}$,$X_t\;$a.e.有限,所以$X_tX_s\;$a.e.有限。由\cref{prop:NonnegativeMeasurableIntegral}(4)(6)可得:
	\begin{align*}
		\operatorname{E}\left(\sum_{i=-\infty}^{+\infty}\sum_{j=-\infty}^{+\infty}|a_ia_j\varepsilon_{t-i}\varepsilon_{s-j}|\right)=\sum_{i=-\infty}^{+\infty}\sum_{j=-\infty}^{+\infty}|a_ia_j|\operatorname{E}(|\varepsilon_{t-i}\varepsilon_{s-j}|)
	\end{align*}
	由\cref{ineq:cauchy-schiwarz-expectations}可得:
	\begin{equation*}
		\operatorname{E}(|\varepsilon_{t-i}\varepsilon_{s-j}|)=\Big|\operatorname{E}(|\varepsilon_{t-i}\varepsilon_{s-j}|)\Big|\leqslant\sqrt{\operatorname{E}(\varepsilon_{t-i}^2)\operatorname{E}(\varepsilon_{s-j}^2)}=\sigma^2
	\end{equation*}
	于是:
	\begin{equation*}
		\operatorname{E}\left(\sum_{i=-\infty}^{+\infty}\sum_{j=-\infty}^{+\infty}|a_ia_j\varepsilon_{t-i}\varepsilon_{s-j}|\right)\leqslant\sigma^2\sum_{i=-\infty}^{+\infty}\sum_{j=-\infty}^{+\infty}|a_ia_j|=\sigma^2\left(\sum_{i=-\infty}^{+\infty}|a_i|\right)^2<+\infty
	\end{equation*}
	取控制函数$\sum\limits_{i=-\infty}^{+\infty}\sum\limits_{j=-\infty}^{+\infty}|a_ia_j\varepsilon_{t-i}\varepsilon_{s-j}|$,由\cref{theo:DominatedConvergenceTheorem}和\cref{prop:MeasurableIntegral}(6)可得:
	\begin{align*}
		\operatorname{Cov}(X_t,X_s)&=\operatorname{E}(X_tX_s)=\operatorname{E}\left(\lim_{n\to+\infty}\varphi_n\psi_n\right)=\lim_{n\to+\infty}\operatorname{E}(\varphi_n\psi_n) \\
		&=\lim_{n\to+\infty}\operatorname{E}\left[\left(\sum_{i=-n}^{n}a_i\varepsilon_{t-i}\right)\left(\sum_{j=-n}^{n}a_j\varepsilon_{s-j}\right)\right] \\
		&=\lim_{n\to+\infty}\operatorname{E}\left(\sum_{i=-n}^{n}\sum_{j=-n}^{n}a_ia_j\varepsilon_{t-i}\varepsilon_{s-j}\right) \\
		&=\lim_{n\to+\infty}\left[\sum_{i=-n}^{n}\sum_{j=-n}^{n}a_ia_j\operatorname{E}(\varepsilon_{t-i}\varepsilon_{s-j})\right] \\
		&=\sum_{i=-\infty}^{+\infty}a_ia_{i-(t-s)}\sigma^2=\sigma^2\sum_{i=-\infty}^{+\infty}a_ia_{i+(t-s)}
	\end{align*}
	上式最后一步是因为$t-i=s-j$时期望才不为$0$,并且$n$是逐渐变大趋于无穷,所以也不用考虑$n$为定值时$t,s$相差过大导致索引越界的问题。由协方差公式可以看出其只与$t-s$相关。\par
	注意到:
	\begin{equation*}
		\operatorname{Var}(X_t)=\sigma^2\sum_{i=-\infty}^{+\infty}a_i^2
	\end{equation*}
	因为$l^1\subset l^2$,$\{a_n\}\in l^1$,所以上式也收敛。\par
	综上,$\{X_t\}$是一个平稳序列。
\end{proof}
\subsubsection{线性平稳序列}
\begin{definition}
	$\{\varepsilon_t\}=\{\varepsilon_t:t\in\mathbb{Z}\}$是$WN(0,\sigma^2)$,$\{a_n\}\in l^2$。称:
	\begin{equation*}
		X_t=\sum_{i=-\infty}^{+\infty}a_i\varepsilon_{t-i}
	\end{equation*}
	构成的序列$\{X_t\}$为\gls{LinearlyStationarySeries}。
\end{definition}
一
\begin{property}\label{prop:LinearlyStationarySeries}
	$\{\varepsilon_t\}=\{\varepsilon_t:t\in\mathbb{Z}\}$是$WN(0,\sigma^2)$,$\{a_n\}\in l^2$。线性平稳序列:
	\begin{equation*}
		X_t=\sum_{i=-\infty}^{+\infty}a_i\varepsilon_{t-i}
	\end{equation*}
	具有如下性质:
	\begin{enumerate}
		\item 线性平稳序列是平稳序列,且有:
		\begin{equation*}
			\operatorname{E}(X_t)=0,\;\gamma(n)=\sigma^2\sum_{i=-\infty}^{+\infty}a_ia_{i+n}
		\end{equation*}
		\item 对于自协方差函数$\gamma(n)$,有:
		\begin{equation*}
			\lim_{n\to\infty}\gamma(n)=0
		\end{equation*}\par
		\item 线性平稳序列的谱密度为($\{a_n\}$为实数列):
		\begin{equation*}
			f(\lambda)=\frac{\sigma^2}{2\pi}\left|\sum_{j=-\infty}^{+\infty}a_je^{ij\lambda}\right|^2,\quad\lambda\in[-\pi,\pi]
		\end{equation*}
		\item 自协方差矩阵$\Gamma_n$是正定矩阵;
	\end{enumerate}
\end{property}
\begin{proof}
	(1)在$L^2$空间中定义内积$(X,Y)=\operatorname{E}(XY)$,定义:
	\begin{equation*}
		\varphi_n=\sum_{i=-n}^{n}a_i\varepsilon_{t-i}
	\end{equation*}
	则对$m<n,\;n\to+\infty$有:
	\begin{equation*}
		||\varphi_n-\varphi_m||^2=\left\|\sum_{i=m+1}^{n}a_i\varepsilon_{t-i}+\sum_{i=-n}^{-m-1}a_i\varepsilon_{t-i}\right\|^2=\sigma^2\left(\sum_{i=m+1}^{n}a_i^2+\sum_{i=-n}^{-m-1}a_i^2\right)\to0
	\end{equation*}
	由\cref{theo:LpBanach}可知$X_t\in L_2$。由内积的连续性和\cref{prop:MeasurableIntegral}(6)可得:
	\begin{gather*}
		\operatorname{E}(X_t)=(X_t,1)=\lim_{n\to+\infty}(\varphi_n,1)=\lim_{n\to+\infty}\operatorname{E}(\varphi_n)=0 \\
		\begin{aligned}
			\operatorname{Cov}(X_t,X_s)&=\operatorname{E}(X_tX_s)=\lim_{n\to+\infty}\left(\sum_{i=-n}^{n}a_i\varepsilon_{t-i},\sum_{i=-n}^{n}a_i\varepsilon_{s-i}\right) \\
			&=\lim_{n\to+\infty}\operatorname{E}\left(\sum_{i=-n}^{n}a_i\varepsilon_{t-i}\sum_{j=-n}^{n}a_j\varepsilon_{s-j}\right)=\lim_{n\to+\infty}\operatorname{E}\left(\sum_{i=-n}^{n}\sum_{j=-n}^{n}a_ia_j\varepsilon_{t-i}\varepsilon_{s-j}\right) \\
			&=\sigma^2\sum_{i=-\infty}^{+\infty}a_ia_{i+(t-s)}
		\end{aligned}
	\end{gather*}
	上式倒数第二步到最后一步和\cref{theo:l1LinearlyStationarySeries}中是一样的,同时平稳性的分析也与之一样,故省略。\par
	(2)由\cref{ineq:cauchy-ineq-R}可得:
	\begin{align*}
		|\gamma(n)|
		&=\sigma^2\Bigl|\sum_{i=-\infty}^{+\infty}a_ia_{i+n}\Bigr| \leqslant\sigma^2\sum_{i=-\infty}^{+\infty}|a_ia_{i+n}| \\
		&=\sigma^2\sum_{|i|\leqslant n/2}|a_i||a_{i+n}|+\sigma^2\sum_{|i|> n/2}|a_i||a_{i+n}| \\
		&\leqslant\sigma^2\left(\sum_{|i|\leqslant n/2}a_i^2\sum_{|i|\leqslant n}a_{i+n}^2\right)^\frac{1}{2}+\sigma^2\left(\sum_{|i|>n/2}a_i^2\sum_{|i|>n}a_{i+n}^2\right)^\frac{1}{2} \\
		&\leqslant\sigma^2\left(\sum_{i=-\infty}^{+\infty}a_i^2\sum_{|i|\leqslant n/2}a_{i+n}^2\right)^\frac{1}{2}+\sigma^2\left(\sum_{i=-\infty}^{+\infty}a_i^2\sum_{|i|>n/2}a_{i+n}^2\right)^\frac{1}{2}
	\end{align*}
	注意到$|i|\leqslant\dfrac{n}{2}$时$\dfrac{n}{2}\leqslant i+n\leqslant\dfrac{3n}{2}$,所以有:
	\begin{equation*}
		\sum_{|i|\leqslant n/2}a_{i+n}^2\leqslant\sum_{|i|>n/2}a_{i}^2
	\end{equation*}
	结合$\{a_n\}\in l^2$即可得:
	\begin{equation*}
		|\gamma(n)|\leqslant2\sigma^2\left(\sum_{i=-\infty}^{+\infty}a_i^2\sum_{|i|>n/2}a_{i}^2\right)^\frac{1}{2}=2\sigma^2\left(\sum_{i=-\infty}^{+\infty}a_i^2\right)^{\frac{1}{2}}\left(\sum_{|i|>n/2}^{}a_{i}^2\right)^{\frac{1}{2}}\to0
	\end{equation*}\par
	(3)设随机变量$Y$在$[-\pi,\pi]$上服从均匀分布,定义$\varepsilon_n=e^{inY},\;n\in\mathbb{Z}$,于是有:
	\begin{gather*}
		\begin{aligned}
			\operatorname{E}(\varepsilon_n)&=\int_{[-\pi,\pi]}\frac{1}{2\pi}e^{iny}\dif y=\frac{e^{in\pi}-e^{-in\pi}}{2\pi in} \\
			&=\frac{\cos(n\pi)+i\sin(n\pi)-\cos(-n\pi)-i\sin(-n\pi)}{2\pi in} \\
			&=\frac{2i\sin(n\pi)}{2\pi in}=0,\quad n\ne0
		\end{aligned} \\
		\operatorname{E}(\varepsilon_n)=\int_{[-\pi,\pi]}\frac{1}{2\pi}\dif y=1,\quad n=0 \\
		\begin{aligned}
			\operatorname{E}(\varepsilon_n\overline{\varepsilon}_m)&=\operatorname{E}(e^{i(n-m)Y})=\int_{[-\pi,\pi]}\frac{1}{2\pi}e^{i(n-m)y}\dif y=\frac{e^{i(n-m)\pi}-e^{-i(n-m)\pi}}{2\pi i(n-m)} \\
			&=\frac{\cos[(n-m)\pi]+i\sin[(n-m)\pi]-\cos[-(n-m)\pi]-i\sin[-(n-m)\pi]}{2\pi i(n-m)} \\
			&=\frac{2i\sin[(n-m)\pi]}{2\pi i(n-m)}=0,\quad n\ne m
		\end{aligned} \\
		\operatorname{E}(\varepsilon_n\overline{\varepsilon}_m)=\operatorname{E}(e^{i(n-m)y})=\int_{[-\pi,\pi]}\frac{1}{2\pi}\dif y=1,\quad n=m
	\end{gather*}
	即:
	\begin{equation*}
		\operatorname{E}(\varepsilon_n)=\delta_n,\quad\operatorname{E}(\varepsilon_n\overline{\varepsilon}_m)=\delta_{n-m},\qquad n,m\in\mathbb{Z}^{}
	\end{equation*}\par
	令:
	\begin{equation*}
		Z_n=\sum_{j=-\infty}^{+\infty}a_j\varepsilon_{n-j}=\sum_{j=-\infty}^{+\infty}a_je^{i(n-j)Y},\quad n\in\mathbb{Z}^{}
	\end{equation*}
	由内积的连续性和\cref{prop:MeasurableIntegral}(6)可得:
	\begin{align*}
		\operatorname{E}(Z_n)&=(Z_n,1)=\left(\sum_{j=-\infty}^{+\infty}a_j\varepsilon_{n-j},1\right)=\left[\lim_{k\to+\infty}\left(\sum_{j=k}^{-k}a_j\varepsilon_{n-j}\right),1\right] \\
		&=\lim_{k\to+\infty}\left(\sum_{j=-k}^{k}a_j\varepsilon_{n-j},1\right)=\lim_{k\to+\infty}\operatorname{E}\left(\sum_{j=-k}^{k}a_j\varepsilon_{n-j}\right) \\
		&=\lim_{k\to+\infty}\left[\sum_{j=-k}^{k}\operatorname{E}(a_j\varepsilon_{n-j})\right]=a_n
	\end{align*}
	\begin{align*}
		\operatorname{E}(Z_n\overline{Z}_m)&=(Z_n,Z_m)=\left(\sum_{j=-\infty}^{+\infty}a_j\varepsilon_{n-j},\sum_{j=-\infty}^{+\infty}a_j\varepsilon_{m-j}\right) \\
		&=\left[\lim_{k\to+\infty}\left(\sum_{j=-k}^{k}a_j\varepsilon_{n-j}\right),\lim_{k\to+\infty}\left(\sum_{j=-k}^{k}a_j\varepsilon_{m-j}\right)\right] \\
		&=\lim_{k\to+\infty}\left(\sum_{j=-k}^{k}a_j\varepsilon_{n-j},\sum_{j=-k}^{k}a_j\varepsilon_{m-j}\right)=\lim_{k\to+\infty}\operatorname{E}\left(\sum_{j=-k}^{k}a_j\varepsilon_{n-j}\sum_{l=-k}^{k}a_l\varepsilon_{m-l}\right) \\
		&=\lim_{k\to+\infty}\operatorname{E}\left(\sum_{j=-k}^{k}\sum_{l=-k}^{k}a_j\varepsilon_{n-j}a_l\varepsilon_{m-l}\right)=\sum_{j=-\infty}^{+\infty}a_ja_{j+(n-m)} 
	\end{align*}
	另一方面又有:
	\begin{align*}
		\operatorname{E}(Z_n\overline{Z}_m)&=\operatorname{E}\left[\left(\sum_{j=-\infty}^{+\infty}a_je^{i(n-j)Y}\right)\left(\sum_{k=-\infty}^{+\infty}a_ke^{-i(m-k)Y}\right)\right] \\
		&=\frac{1}{2\pi}\int_{[-\pi,\pi]}\left(\sum_{j=-\infty}^{+\infty}a_je^{i(n-j)y}\right)\left(\sum_{k=-\infty}^{+\infty}a_ke^{-i(m-k)y}\right)\dif y \\
		&=\frac{1}{2\pi}\int_{[-\pi,\pi]}\left(\sum_{j=-\infty}^{+\infty}a_je^{-ijy}\right)\left(\sum_{k=-\infty}^{+\infty}a_ke^{iky}\right)e^{i(n-m)y}\dif y \\
		&\frac{1}{2\pi}=\int_{[-\pi,\pi]}\left|\sum_{j=-\infty}^{+\infty}a_je^{-ijy}\right|^2e^{i(n-m)y}\dif y
	\end{align*}
	所以有:
	\begin{gather*}
		\sum_{j=-\infty}^{+\infty}a_ja_{j+(n-m)}=\frac{1}{2\pi}\int_{[-\pi,\pi]}\left|\sum_{j=-\infty}^{+\infty}a_je^{-ijy}\right|^2e^{i(n-m)y}\dif y \\
		\sigma^2\sum_{j=-\infty}^{+\infty}a_ja_{j+n}=\frac{\sigma^2}{2\pi}\int_{[-\pi,\pi]}\left|\sum_{j=-\infty}^{+\infty}a_je^{-ijy}\right|^2e^{iny}\dif y
	\end{gather*}
	由(1)可得:
	\begin{equation*}
		\gamma(n)=\frac{\sigma^2}{2\pi}\int_{[-\pi,\pi]}\left|\sum_{j=-\infty}^{+\infty}a_je^{-ijy}\right|^2e^{iny}\dif y
	\end{equation*}
	所以有:
	\begin{equation*}
		f(\lambda)=\frac{\sigma^2}{2\pi}\left|\sum_{j=-\infty}^{+\infty}a_je^{-ij\lambda}\right|^2=\frac{\sigma^2}{2\pi}\left|\sum_{j=-\infty}^{+\infty}a_je^{ij\lambda}\right|^2
	\end{equation*}\par
	(4)由(3)和\cref{theo:GammaPositiveDefinite}立即可得。
\end{proof}









