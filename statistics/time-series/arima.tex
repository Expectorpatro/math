\section{ARIMA}
\subsection{AR模型}
\begin{definition}
	如果$\{\varepsilon_t\}$是$\operatorname{WN}(0,\sigma^2)$,实数$\seq{a}{p}(a_p\ne0)$使得多项式$A(z)=0$的根都在单位圆外:
	\begin{equation*}
		A(z)=1-\sum_{i=1}^{p}a_iz^i\ne0,\quad\forall\;|z|\leqslant1
	\end{equation*}
	则称$p$阶常系数线性差分方程:
	\begin{equation*}
		X_t=\sum_{i=1}^{p}a_iX_{t-i}+\varepsilon_t
	\end{equation*}
	为$p$\textbf{阶自回归模型},简记为$\operatorname{AR}(p)$模型。称满足$\operatorname{AR}(p)$模型的平稳时间序列$\{X_t\}$为$\operatorname{AR}(p)$序列,称$\seq{a}{p}$为$\operatorname{AR}(p)$模型的\textbf{自回归系数},多项式$A(z)=0$的根都在单位圆外这一条件被称为\textbf{稳定性条件},分别称$A(z)$和$A(\mathcal{B})$为$\operatorname{AR}(p)$模型的\textbf{特征多项式}和\textbf{自回归系数多项式}。可以用$A(\mathcal{B})$将模型改写为$A(\mathcal{B})X_t=\varepsilon_t$。
\end{definition}
\begin{derivation}
	$A(z)=0$的解为$p$阶常系数线性差分方程特征方程:
	\begin{equation*}
		\lambda^p-a_1\lambda^{p-1}-\cdots-a_{p-1}\lambda-a_p=0
	\end{equation*}
	的解的倒数。取特征方程的任一根$\lambda_j$,则:
	\begin{equation*}
		A\left(\frac{1}{\lambda_j}\right)=1-\sum_{i=1}^{p}a_i\frac{1}{\lambda_j^i}=\frac{1}{\lambda_j^p}(\lambda_j^p-a_1\lambda_j^{p-1}-\cdots-a_{p-1}\lambda_j-a_{p})=0
	\end{equation*}
	所以上述稳定性条件即为要求$p$阶常系数线性差分方程特征方程的根都在单位圆内。
\end{derivation}
\begin{theorem}\label{theo:ARPSolution}
	设$\operatorname{AR}(p)$模型的特征多项式$A(z)=0$有$s$个互异根$\seq{z}{s}$,根的重数分别为$\seq{r}{s}$,$1<\rho<\min\limits_i\{|z_i|\}$,则:
	\begin{enumerate}
		\item $\operatorname{AR}(p)$模型的唯一平稳解是:
		\begin{equation*}
			X_t=\sum_{i=0}^{+\infty}\psi_i\varepsilon_{t-i}
		\end{equation*}
		其中$\psi_i$为$A^{-1}(z)$在$\{z:|z|\leqslant\rho\}$内展开的幂级数的系数,称之为\textbf{Wold系数},$\{\psi_n\}\in l^1$。对$j<0$定义$\psi_j=0$,那么它具有递推公式:
		\begin{equation*}
			\psi_0=1,\quad\psi_j=\sum_{i=1}^{p}a_i\psi_{j-i},\;\forall\;j\geqslant1
		\end{equation*}
		\item $\operatorname{AR}(p)$模型的通解为:
		\begin{equation*}
			\sum_{i=0}^{+\infty}\psi_i\varepsilon_{t-i}+\sum_{i=1}^{s}\sum_{j=0}^{r_i-1}c_{ij}t^{j}z_i^{-t}
		\end{equation*}
		其中$c_{ij}$为任意常数;
		\item $\operatorname{AR}(p)$模型的任一解都以负指数阶的速度收敛到平稳解,$\min\limits_i{|z_i|}$越大,收敛越快。
	\end{enumerate}
\end{theorem}
\begin{proof}
	(1)由复变函数的知识,$A^{-1}(z)$在$\{z:|z|\leqslant\rho\}$内解析,即$A^{-1}(z)$有如下展开:
	\begin{equation*}
		A^{-1}(z)=\sum_{i=0}^{+\infty}\psi_iz^i,\quad|z|\leqslant\rho
	\end{equation*}
	且这个级数是绝对收敛的。由绝对收敛性可知$|\psi_iz^i|\to0$,即$|\psi_i|=o(\rho^{-i})$,所以$\{\psi_i\}\in l^1$,由\cref{theo:l1LinearlyStationarySeries}可知$\{X_t\}$是平稳序列。\par
	令$a_0=-1$,对$k<0$定义$\psi_k=0$。注意到:
	\begin{align*}
		A(\mathcal{B})X_t&=\left(1-\sum_{i=1}^{p}a_i\mathcal{B}^i\right)X_t=\left(-a_0-\sum_{i=1}^{p}a_i\mathcal{B}^i\right)X_t=-\sum_{i=0}^{p}a_i\mathcal{B}^iX_t \\
		&=-\sum_{i=0}^{p}a_iX_{t-i}=-\sum_{i=0}^{p}a_{i}\sum_{j=0}^{+\infty}\psi_j\varepsilon_{t-i-j}=-\sum_{j=0}^{+\infty}\sum_{i=0}^{p}a_i\psi_j\varepsilon_{t-i-j} \\
		&=-\sum_{j=0}^{+\infty}\sum_{i=0}^{p}a_i\psi_{j-i}\varepsilon_{t-j}=-\sum_{j=0}^{+\infty}\left(\sum_{i=0}^{p}a_i\psi_{j-i}\right)\varepsilon_{t-j}
	\end{align*}
	因为$\{\psi_i\}\in l^1$,所以对$|z|\leqslant1$级数$\sum\limits_{j=0}^{+\infty}\psi_jz^j$是良定义的。由:
	\begin{align*}
		1&=A(z)A^{-1}(z)=\left(1-\sum_{i=1}^{p}a_iz^i\right)A^{-1}(z)=-\sum_{i=0}^{p}a_iz^i\sum_{j=0}^{+\infty}\psi_jz^j \\
		&=-\sum_{i=0}^{p}\sum_{j=0}^{+\infty}a_iz^i\psi_jz^j=-\sum_{j=0}^{+\infty}\sum_{i=0}^{p}a_i\psi_jz^{i+j}=-\sum_{j=0}^{+\infty}\left(\sum_{i=0}^{p}a_i\psi_{j-i}\right)z^j
	\end{align*}
	最后一步是求和换元后的结果。对比系数可得(递推公式):
	\begin{equation*}
		-\sum_{i=0}^{p}a_i\psi_{-i}=1,\quad-\sum_{i=0}^{p}a_i\psi_{j-i}=0,\;\forall\;j\geqslant1
	\end{equation*}
	于是:
	\begin{equation*}
		A(\mathcal{B})X_t=-\sum_{j=0}^{+\infty}\left(\sum_{i=0}^{p}a_i\psi_{j-i}\right)\varepsilon_{t-j}=-\sum_{i=0}^{p}a_i\psi_{-i}\varepsilon_t-\sum_{j=1}^{+\infty}\left(\sum_{i=0}^{p}a_i\psi_{j-i}\right)\varepsilon_{t-j}=\varepsilon_t
	\end{equation*}
	所以$\{X_t\}$是解。\par
	设还有另一平稳解$\{Y_t\}$,即$A(\mathcal{B})Y_t=\varepsilon_t$且$A^{-1}(\mathcal{B})$存在,则:
	\begin{equation*}
		Y_t=A^{-1}(\mathcal{B})\varepsilon_t=A^{-1}(\mathcal{B})A(\mathcal{B})X_t=X_t
	\end{equation*}\par
	综上,$\{X_t\}$是$\operatorname{AR}(p)$模型唯一的平稳解。\par
	(2)由(1)、\cref{theo:GeneralSolutionNonhomogeneousLinearDifferenceEquation}与\cref{theo:GeneralSolutionHomogeneousLinearDifferenceEqConstantCoefficients}即可得通解为:
	\begin{equation*}
		X_t+\sum_{i=1}^{s}\sum_{j=0}^{r_i-1}c_{ij}t^{j}z_i^{-t}=\sum_{i=0}^{+\infty}\psi_i\varepsilon_{t-i}+\sum_{i=1}^{s}\sum_{j=0}^{r_i-1}c_{ij}t^{j}z_i^{-t}
	\end{equation*}\par
	(3)由(1)(2)可得对于$\operatorname{AR}(p)$模型的任一解$\{Y_t\}$有:
	\begin{equation*}
		|X_t-Y_t|=\left|\sum_{i=1}^{s}\sum_{j=0}^{r_i-1}c_{ij}t^{j}z_i^{-t}\right|\leqslant O\left[\left(\min_i\{|z_i|\}\right)^{-t}\right]\qedhere
	\end{equation*}
\end{proof}
\begin{note}
	上述定理给了我们一个产生$\operatorname{AR}(p)$序列的方式。先任意选择$p$个初始值,然后根据自回归系数产生序列$\{Y_t\}$。因为任意的$\{Y_t\}$都以负指数阶的速度收敛到平稳解,取一个较大的$m$然后令$X_t=Y_{m+t}$即可得到近似的$\operatorname{AR}(p)$序列$\{X_t\}$。
\end{note}
\begin{property}\label{prop:ARp}
	$\operatorname{AR}(p)$序列$\{X_t\}$具有如下性质:
	\begin{enumerate}
		\item 对任意的$i\geqslant1$且$i\in\mathbb{N}^+$,$X_t$与$\varepsilon_{t+i}$不相关;
		\item $\gamma(n)=\sigma^2\sum\limits_{i=0}^{+\infty}\psi_i\psi_{i+n}$;
		\item (Yule-Walker方程)$\;\{X_t\}$的自协方差函数满足:
		\begin{gather*}
			\boldsymbol{\gamma}_n=\Gamma_n\alpha,\;
			\gamma(0)=\boldsymbol{\gamma}_n^T\alpha+\sigma^2,\quad n\geqslant p \\
			\boldsymbol{\gamma}_n=
			\begin{pmatrix}
				\gamma(1) \\
				\gamma(2) \\
				\vdots \\
				\gamma(n)
			\end{pmatrix},\;
			\Gamma_n=
			\begin{pmatrix}
				\gamma(0) & \gamma(1) & \cdots & \gamma(n-1) \\
				\gamma(1) & \gamma(0) & \cdots & \gamma(n-2) \\
				\vdots & \vdots & \ddots & \vdots \\
				\gamma(n-1) & \gamma(n-2) & \cdots & \gamma(0) \\
			\end{pmatrix} \\
			\alpha=(\seq{a}{p},0,0,\dots,0)^T
		\end{gather*}
		\item $\operatorname{AR}(p)$序列的自协方差函数和自相关函数满足和$\operatorname{AR}(p)$模型相对应的常系数齐次线性差分方程:
		\begin{gather*}
			\gamma(n)=a_1\gamma(n-1)+a_2\gamma(n-2)\cdots+a_p\gamma(n-p) \\
			\rho(n)=a_1\rho(n-1)+a_2\rho(n-2)\cdots+a_p\rho(n-p)
		\end{gather*}
		\item $\{X_t\}$的自协方差函数与自相关函数具有拖尾性,即$\gamma(n)$和$\rho(n)$始终不为$0$,且二者的模随着$n$的增大指数衰减到$0$;
		\item $\{X_t\}$具有如下谱密度函数:
		\begin{equation*}
			f(\lambda)=\frac{\sigma^2}{2\pi}\left|\sum_{j=0}^{+\infty}\psi_je^{ij\lambda}\right|^2=\frac{\sigma^2}{2\pi|A(e^{i\lambda})|^2}=\frac{1}{2\pi}\sum_{n=-\infty}^{+\infty}\gamma(n)e^{in\lambda},\quad\lambda\in[-\pi,\pi]
		\end{equation*}
		\item $\{X_t\}$的自协方差矩阵为正定矩阵;
		\item $\{X_t\}$是最小序列;
	\end{enumerate}
\end{property}
\begin{proof}
	(1)因为$\operatorname{E}(\varepsilon_{t+i})=0$,由\cref{prop:NonnegativeMeasurableIntegral}(9)可知$\varepsilon_{t+i}\;$a.e.有限,由\cref{theo:l1LinearlyStationarySeries}可知$X_t\;$a.e.有限,于是$X_t\varepsilon_{t+i}$良定义。根据\cref{prop:NonnegativeMeasurableIntegral}(4)(6)可知:
	\begin{align*}
		\operatorname{E}\left(\sum_{j=0}^{+\infty}|\psi_j\varepsilon_{t-j}\varepsilon_{t+i}|\right)
		&=\lim_{n\to+\infty}\operatorname{E}\left(\sum_{j=0}^{n}|\psi_j\varepsilon_{t-j}\varepsilon_{t+i}|\right)
		=\lim_{n\to+\infty}\left[\sum_{j=0}^{n}|\psi_j|\operatorname{E}(|\varepsilon_{t-j}\varepsilon_{t+i}|)\right] \\
		&=\sum_{j=0}^{+\infty}|\psi_j|\operatorname{E}(|\varepsilon_{t-j}\varepsilon_{t+i}|)
	\end{align*}
	由\cref{ineq:cauchy-schiwarz-expectations}可知:
	\begin{equation*}
		\operatorname{E}(|\varepsilon_{t-j}\varepsilon_{t+i}|)=\Big|\operatorname{E}(|\varepsilon_{t-j}\varepsilon_{t+i}|)\Big|\leqslant\sqrt{\operatorname{E}(\varepsilon_{t-j}^2)\operatorname{E}(\varepsilon_{t+i}^2)}=\sigma^2
	\end{equation*}
	由\cref{theo:ARPSolution}(1)可得$\{\psi_i\}\in l^1$,所以:
	\begin{equation*}
		\operatorname{E}\left(\sum_{j=0}^{+\infty}|\psi_j\varepsilon_{t-j}\varepsilon_{t+i}|\right)\leqslant\sigma^2\sum_{j=0}^{+\infty}|\psi_j|<+\infty
	\end{equation*}
	取控制函数$\sum\limits_{j=0}^{+\infty}|\psi_j\varepsilon_{t-j}\varepsilon_{t+i}|$,由\cref{theo:DominatedConvergenceTheorem}和\cref{prop:NonnegativeMeasurableIntegral}(6)可得:
	\begin{align*}
		\operatorname{E}(X_t\varepsilon_{t+i})&=\operatorname{E}\left(\sum_{j=0}^{+\infty}\psi_j\varepsilon_{t-j}\varepsilon_{t+i}\right)=\operatorname{E}\left[\lim_{n\to+\infty}\left(\sum_{j=0}^{n}\psi_j\varepsilon_{t-j}\varepsilon_{t+i}\right)\right] \\
		&=\lim_{n\to+\infty}\operatorname{E}\left(\sum_{j=0}^{n}\psi_j\varepsilon_{t-j}\varepsilon_{t+i}\right)=\lim_{n\to+\infty}\left[\sum_{j=0}^{n}\psi_j\operatorname{E}(\varepsilon_{t-j}\varepsilon_{t+i})\right]=0
	\end{align*}\par
	(2)由\cref{prop:LinearlyStationarySeries}(1)和\cref{theo:ARPSolution}(1)立即可得。\par
	(3)由$\operatorname{AR}(p)$模型的定义可得:
	\begin{equation*}
		\begin{pmatrix}
			X_t \\
			X_{t+1} \\
			\vdots \\
			X_{t+n-1}
		\end{pmatrix}=
		\begin{pmatrix}
			X_{t-1} & X_{t-2} & \cdots & X_{t-n} \\
			X_{t} & X_{t-1} & \cdots & X_{t-n+1} \\
			\vdots & \vdots & \ddots & \vdots \\
			X_{t+n-2} & X_{t+n-3} & \cdots & X_{t-1}
		\end{pmatrix}\alpha+
		\begin{pmatrix}
			\varepsilon_t \\
			\varepsilon_{t+1} \\
			\vdots \\
			\varepsilon_{t+n-1}
		\end{pmatrix}
	\end{equation*}
	于是:
	\begin{gather*}
		X_{t-1}
		\begin{pmatrix}
			X_t \\
			X_{t+1} \\
			\vdots \\
			X_{t+n-1}
		\end{pmatrix}=X_{t-1}
		\begin{pmatrix}
			X_{t-1} & X_{t-2} & \cdots & X_{t-n} \\
			X_{t} & X_{t-1} & \cdots & X_{t-n+1} \\
			\vdots & \vdots & \ddots & \vdots \\
			X_{t+n-2} & X_{t+n-3} & \cdots & X_{t-1}
		\end{pmatrix}\alpha_n+X_{t-1}
		\begin{pmatrix}
			\varepsilon_t \\
			\varepsilon_{t+1} \\
			\vdots \\
			\varepsilon_{t+n-1}
		\end{pmatrix} \\
		\boldsymbol{\gamma}_n=\Gamma_n\alpha,\quad n\geqslant p
	\end{gather*}
	第二行是对第一行取期望的结果。对于$\gamma(0)$,由(1)可得:
	\begin{align*}
		\gamma(0)&=\operatorname{E}(X_t^2)=\operatorname{E}\left(\sum_{i=1}^{p}a_iX_{t-i}+\varepsilon_t\right)^2=\operatorname{E}\left(\sum_{i=1}^{p}a_iX_{t-i}\right)^2+\operatorname{E}(\varepsilon_t)^2 \\
		&=\alpha^T\Gamma_n\alpha+\sigma^2=\alpha^T\boldsymbol{\gamma}_n+\sigma^2,\quad n\geqslant p
	\end{align*}\par
	(4)由(3)和\cref{theo:RhotGamma0}立即可得。\par
	(5)由\cref{theo:GeneralSolutionHomogeneousLinearDifferenceEqConstantCoefficients}和(4)可知$\gamma(n)$和$\rho(n)$的通解都具有如下形式:
	\begin{equation*}
		\sum_{i=1}^{s}\sum_{j=0}^{r_i-1}c_{ij}n^{j}\lambda_i^n,\;\forall\;n\in\mathbb{N}
	\end{equation*}
	其中$\seq{\lambda}{s}$为对应常系数齐次线性差分方程的特征方程的根,$c_{ij}$为任意常数。若$c_{ij}$全为$0$,则$\gamma(0),\rho(0)$都为$0$,所以它们不可能全为$0$.由$\operatorname{AR}(p)$序列的定义,$|\lambda_i|<1$,因为$a_p\ne0$,所以$\lambda_i\ne0$,由此可得$\gamma(n)$和$\rho(n)$始终不为$0$。由指数函数与幂函数的收敛速度比较可知$\gamma(n)$和$\rho(n)$的模随着$n$的增大将以指数阶的速度减小。\par
	(6)由\cref{prop:LinearlyStationarySeries}和\cref{theo:ARPSolution}(1)可知:
	\begin{equation*}
		f(\lambda)=\frac{\sigma^2}{2\pi}\left|\sum_{j=0}^{+\infty}\psi_je^{ij\lambda}\right|^2
	\end{equation*}
	由\cref{theo:SpectralGamma}和\cref{theo:ARPSolution}(1)可得:
	\begin{equation*}
		f(\lambda)=\frac{1}{2\pi}\sum_{n=-\infty}^{+\infty}\gamma(n)e^{-in\lambda}
	\end{equation*}
	根据\cref{theo:ARPSolution}(1)中的论述,有:
	\begin{equation*}
		A^{-1}(z)=\sum_{i=0}^{+\infty}\psi_iz^i,\quad|z|\leqslant\rho
	\end{equation*}
	因为$|e^{i\lambda}|=1<\rho$,所以:
	\begin{equation*}
		\sum_{j=0}^{+\infty}\psi_je^{ij\lambda}=A^{-1}(e^{i\lambda})
	\end{equation*}
	于是:
	\begin{equation*}
		f(\lambda)=\frac{\sigma^2}{2\pi}\left|A^{-1}(e^{i\lambda})\right|^2=\frac{\sigma^2}{2\pi|A(e^{i\lambda})|^2}
	\end{equation*}\par
	(7)由\cref{prop:LinearlyStationarySeries}(4)立即可得。
\end{proof}

\subsection{MA模型}
\begin{definition}
	设$\{\varepsilon_t\}$是$\operatorname{WN}(0,\sigma^2)$,实数$\seq{b}{q}(b_q\ne0)$使得多项式$B(z)=0$的根都不在单位圆内:
	\begin{equation*}
		B(z)=1+\sum_{i=1}^{q}b_iz^i\ne0,\quad\forall\;|z|<1
	\end{equation*}
	则称:
	\begin{equation*}
		X_t=\varepsilon_t+\sum_{i=1}^{q}b_i\varepsilon_{t-i},\quad t\in\mathbb{Z}
	\end{equation*}
	为\textbf{$q$阶滑动平均模型},简记为$\operatorname{MA}(q)$模型。称满足$\operatorname{MA}(q)$模型的平稳时间序列$\{X_t\}$为$\operatorname{MA}(q)$序列。若$B(z)\ne0$对$|z|\leqslant1$成立,则称对应的$\operatorname{MA}(q)$模型为\textbf{可逆的$\operatorname{MA}(q)$模型},相应的$\operatorname{MA}(q)$序列为\textbf{可逆的$\operatorname{MA}(q)$序列}。可以用$B(\mathcal{B})$将模型改写为$X_t=B(\mathcal{B})\varepsilon_t$。对于可逆的$\operatorname{MA}(q)$模型,$B^{-1}(z)$有如下展开:
	\begin{equation*}
		B^{-1}(z)=\sum_{i=0}^{+\infty}\psi_iz^i,\quad |z|\leqslant1
	\end{equation*}
	于是在可逆情况下还可以将模型改写为:
	\begin{equation*}
		\varepsilon_t=B^{-1}(\mathcal{B})X_t=\sum_{i=0}^{+\infty}\psi_iX_{t-i},\quad t\in\mathbb{Z}^{}
	\end{equation*}
\end{definition}
\begin{theorem}
	设零均值平稳时间序列$\{X_t\}$有自协方差函数$\{\gamma(n)\}$,则$\{X_t\}$是$\operatorname{MA}(q)$序列的充分必要条件为:
	\begin{equation*}
		\gamma(q)\ne 0,\quad\gamma(k)=0,\;\forall\;|k|>q
	\end{equation*}
\end{theorem}
\begin{property}
	$\operatorname{MA}(q)$序列$\{X_t\}$具有如下性质:
	\begin{enumerate}
		\item 令$b_0=1$,则:
		\begin{equation*}
			\operatorname{E}(X_t)=0,\quad\gamma(n)=
			\begin{cases}
				\sigma^2\sum\limits_{i=0}^{q-n}b_ib_{i+n},&0\leqslant k\leqslant q \\
				0, &k>q
			\end{cases}
		\end{equation*}
		\item $\{X_t\}$的自协方差函数$\gamma(n)$和自相关函数$\rho(n)$都具有截尾性,即对任意的$k>q$有$\gamma(k)=\rho(k)=0$;
		\item 若$\{X_t\}$可逆,对$j<0$定义$\psi_j=0$,那么$\{\psi_n\}$具有递推公式:
		\begin{equation*}
			\psi_0=1,\quad\psi_j=-\sum_{i=1}^{q}b_i\psi_{j-i},\;\forall\;j\geqslant1
		\end{equation*}
		\item $\{X_t\}$具有谱密度:
		\begin{equation*}
			f(\lambda)=\frac{\sigma^2}{2\pi}\left|\sum_{j=0}^{q}b_je^{ij\lambda}\right|^2=\frac{\sigma^2}{2\pi}|B(e^{i\lambda})|^2=\frac{1}{2\pi}\sum_{n=-k}^{k}\gamma(n)e^{-in\lambda},\quad\lambda\in[-\pi,\pi]
		\end{equation*}
		\item $\{X_t\}$的自协方差矩阵为正定矩阵;
		\item 若$\{X_t\}$可逆,则$\{X_t\}$是最小序列;若$\{X_t\}$不可逆,则$\{X_t\}$不是最小序列;
	\end{enumerate}
\end{property}
\begin{proof}
	(1)由\cref{theo:ECovFiniteMovingAverage}立即可得。\par
	(2)由(1)和\cref{theo:RhotGamma0}立即得出。\par
	(3)对$k<0$定义$\psi_k=0$,因为$\{X_t\}$可逆,所以:
	\begin{align*}
		1&=B(z)B^{-1}(z)=\left(1+\sum_{i=1}^{q}b_iz^i\right)B^{-1}(z)=\sum_{i=0}^{q}b_iz^i\sum_{j=0}^{+\infty}\psi_jz^j \\
		&=\sum_{i=0}^{q}\sum_{j=0}^{+\infty}b_iz^i\psi_jz^j=\sum_{j=0}^{+\infty}\sum_{i=0}^{q}b_i\psi_jz^{i+j}=\sum_{j=0}^{+\infty}\left(\sum_{i=0}^{q}b_i\psi_{j-i}\right)z^j
	\end{align*}
	对比系数可得(递推公式):
	\begin{equation*}
		\sum_{i=0}^{q}b_i\psi_{-i}=1,\quad\sum_{i=0}^{q}b_i\psi_{j-i}=0,\;\forall\;j\geqslant1
	\end{equation*}\par
	(4)由\cref{prop:ARp}(6)和$B(z)$的定义立即可得,第三种表达方式由(1)和\cref{theo:SpectralGamma}立即可得。\par
	(5)由(4)和\cref{theo:GammaPositiveDefinite}立即可得。
\end{proof}

\subsection{ARMA}
\begin{definition}
	设$\{\varepsilon_t\}$是$\operatorname{WN}(0,\sigma^2)$,实系数多项式$A(z)=0$和$B(z)=0$没有公共根且满足$b_0=1,\;a_pb_q\ne0$和:
	\begin{equation*}
		A(z)=1-\sum_{i=1}^{p}a_iz^i\ne0,\;|z|\leqslant1\quad B(z)=\sum_{i=0}^{q}b_iz^i\ne0,\;|z|<1
	\end{equation*}
	则称差分方程:
	\begin{equation*}
		X_t=\sum_{i=1}^{p}a_iX_{t-i}+\sum_{i=0}^{q}b_i\varepsilon_{t-i}
	\end{equation*}
	为\textbf{自回归滑动平均模型},简称为$\operatorname{ARMA}(p,q)$模型。称满足$\operatorname{ARMA}(p,q)$模型的平稳时间序列$\{X_t\}$为$\operatorname{ARMA}(p,q)$序列。可以用$A(\mathcal{B})$与$B(\mathcal{B})$将模型改写为$A(\mathcal{B})X_t=B(\mathcal{B})\varepsilon_t$。若要求$B(z)$在单位圆上也没有根,则称此时的$\operatorname{ARMA}(p,q)$模型为\textbf{可逆的$\operatorname{ARMA}(p,q)$模型},相应的$\operatorname{ARMA}(p,q)$序列为\textbf{可逆的$\operatorname{ARMA}(p,q)$序列}。对于可逆的$\operatorname{ARMA}(p,q)$模型,令$\seq{z}{s}$为$B(z)=0$的全部互异根,$1<\rho<\min\limits_i\{|z_i|\}$,则$B^{-1}(z)A(z)$在$\{z:|z|\leqslant\rho\}$内解析,从而有如下展开式:
	\begin{equation*}
		B^{-1}(z)A(z)=\sum_{i=0}^{+\infty}\varphi_iz^i,\quad|z|\leqslant\rho
	\end{equation*}
	且这个级数是绝对收敛的。由绝对收敛性可知$|\psi_iz^i|\to0$,即$|\psi_i|=o(\rho^{-i})$,所以$\{\psi_i\}\in l^1$。在$A(\mathcal{B})X_t=B(\mathcal{B})\varepsilon_t$两边同乘$B^{-1}(\mathcal{B})$即可将模型改写为:
	\begin{equation*}
		\varepsilon_t=B^{-1}(\mathcal{B})A(\mathcal{B})=\sum_{i=0}^{+\infty}\varphi_iX_{t-i}
	\end{equation*}\par
	若不对$A(z)$的根与$B(z)$的根做任何限制,称差分方程$A(\mathcal{B})X_t=B(\mathcal{B})\varepsilon_t$为\textbf{广义的$\operatorname{ARMA}(p,q)$模型}。
\end{definition}
\begin{theorem}\label{theo:ARMAPQSolution}
	在$\operatorname{ARMA}(p,q)$模型中,设$A(z)=0$有$s$个互异根$\seq{z}{s}$,根的重数分别为$\seq{r}{s}$,$1<\rho<\min\limits_i\{|z_i|\}$,则:
	\begin{enumerate}
		\item $\operatorname{ARMA}(p,q)$模型的唯一平稳解是:
		\begin{equation*}
			X_t=\sum_{i=0}^{+\infty}\psi_i\varepsilon_{t-i}
		\end{equation*}
		其中$\psi_i$为$\Phi(z)=A^{-1}(z)B(z)$在$\{z:|z|\leqslant\rho\}$内展开的幂级数的系数,称之为\textbf{Wold系数},$\{\psi_n\}\in l^1$。对$j<0$定义$\psi_j=0$,对$j>q$定义$b_j=0$,那么它具有递推公式:
		\begin{equation*}
			\psi_0=1,\quad\psi_j=b_j+\sum_{i=1}^{p}a_i\psi_{j-i},\;\forall\;j\geqslant1
		\end{equation*}
		\item $\operatorname{ARMA}(p,q)$模型的通解为:
		\begin{equation*}
			\sum_{i=0}^{+\infty}\psi_i\varepsilon_{t-i}+\sum_{i=1}^{s}\sum_{j=0}^{r_i-1}c_{ij}t^{j}z_i^{-t}
		\end{equation*}
		其中$c_{ij}$为任意常数;
		\item $\operatorname{ARMA}(p,q)$模型的任一解都以负指数阶的速度收敛到平稳解,$\min\limits_i{|z_i|}$越大,收敛越快。
	\end{enumerate}
\end{theorem}
\begin{proof}
	(1)因为$A(z)$满足平稳性条件,所以存在$\rho>1$使得在$\{z:|z|\leqslant\rho\}$内$A^{-1}(z)B(z)$解析,从而有展开式:
	\begin{equation*}
		\Phi(z)=A^{-1}(z)B(z)=\sum_{i=0}^{+\infty}\psi_iz^i,\quad|z|\leqslant\rho
	\end{equation*}
	且这个级数是绝对收敛的。由绝对收敛性可知$|\psi_iz^i|\to0$,即$|\psi_i|=o(\rho^{-i})$,所以$\{\psi_i\}\in l^1$,由\cref{theo:l1LinearlyStationarySeries}可知$\{X_t\}$是平稳序列。在
	\begin{equation*}
		X_t=\sum_{i=0}^{+\infty}\psi_i\varepsilon_{t-i}
	\end{equation*}
	两边同乘$A(\mathcal{B})$即可得到$A(\mathcal{B})X_t=B(\mathcal{B})\varepsilon_t$,所以$\{X_t\}$是$\operatorname{ARMA}(p,q)$模型的平稳解。\par
	设还有另一平稳解$\{Y_t\}$,即$A(\mathcal{B})Y_t=B(\mathcal{B})\varepsilon_t$且$A^{-1}(\mathcal{B})$存在,则:
	\begin{equation*}
		Y_t=A^{-1}(\mathcal{B})B(\mathcal{B})\varepsilon_t=X_t
	\end{equation*}\par
	综上,$\{X_t\}$是$\operatorname{ARMA}(p,q)$模型唯一的平稳解。\par
	定义$a_0=-1$,则:
	\begin{align*}
		A(z)\Phi(z)&=\left(1-\sum_{i=1}^{p}a_iz^i\right)\Phi(z)=-\sum_{i=0}^{p}a_iz^i\sum_{j=0}^{+\infty}\psi_jz^j \\
		&=-\sum_{i=0}^{p}\sum_{j=0}^{+\infty}a_iz^i\psi_jz^j=-\sum_{j=0}^{+\infty}\sum_{i=0}^{p}a_i\psi_jz^{i+j}=-\sum_{j=0}^{+\infty}\left(\sum_{i=0}^{p}a_i\psi_{j-i}\right)z^j	
	\end{align*}
	最后一步是求和换元后的结果。又有:
	\begin{equation*}
		A(z)\Phi(z)=B(z)=\sum_{i=0}^{q}b_iz^i
	\end{equation*}
	对比系数可得(递推公式):
	\begin{equation*}
		-\sum_{i=0}^{p}a_i\psi_{-i}=b_0,\quad-\sum_{i=0}^{p}a_i\psi_{j-i}=b_i,\;\forall\;j\geqslant1
	\end{equation*}\par
	(2)由(1)、\cref{theo:GeneralSolutionNonhomogeneousLinearDifferenceEquation}与\cref{theo:GeneralSolutionHomogeneousLinearDifferenceEqConstantCoefficients}即可得通解为:
	\begin{equation*}
		X_t+\sum_{i=1}^{s}\sum_{j=0}^{r_i-1}c_{ij}t^{j}z_i^t=\sum_{i=0}^{+\infty}\psi_i\varepsilon_{t-i}+\sum_{i=1}^{s}\sum_{j=0}^{r_i-1}c_{ij}t^{j}z_i^{-t}
	\end{equation*}\par
	(3)由(1)(2)可得对于$\operatorname{ARMA}(p,q)$模型的任一解$\{Y_t\}$有:
	\begin{equation*}
		|X_t-Y_t|=\left|\sum_{i=1}^{s}\sum_{j=0}^{r_i-1}c_{ij}t^{j}z_i^{-t}\right|\leqslant O\left[\left(\min_i\{|z_i|\}\right)^{-t}\right]\qedhere
	\end{equation*}
\end{proof}
\begin{note}
	上述定理给了我们一个产生$\operatorname{ARMA}(p,q)$序列的方式。先任意选择$p$个初始值,然后根据自回归系数和白噪声序列产生序列$\{Y_t\}$。因为任意的$\{Y_t\}$都以负指数阶的速度收敛到平稳解,取一个较大的$m$然后令$X_t=Y_{m+t}$即可得到近似的$\operatorname{ARMA}(p,q)$序列$\{X_t\}$。
\end{note}
\begin{property}
	$\operatorname{ARMA}(p,q)$序列$\{X_t\}$具有如下性质:
	\begin{enumerate}
		\item 对任意的$i\geqslant1$且$i\in\mathbb{N}^+$,$X_t$与$\varepsilon_{t+i}$不相关;
		\item $\gamma(n)=\sigma^2\sum\limits_{i=0}^{+\infty}\psi_i\psi_{i+n}$;
		\item (Yule-Walker方程)$\;\{X_t\}$的自协方差函数满足:
		\begin{gather*}
			\gamma(n)-\sum_{i=1}^{p}a_i\gamma(n-i)=
			\begin{cases}
				\sigma^2\sum\limits_{i=n}^{q}b_i\psi_{i-n},&1\leqslant n\leqslant q \\
				0 & n>q
			\end{cases} \\
			\begin{pmatrix}
				\gamma(q+1) \\
				\gamma(q+2) \\
				\vdots \\
				\gamma(q+p)
			\end{pmatrix}=
			\begin{pmatrix}
				\gamma(q) & \gamma(q-1) & \cdots & \gamma(q-p+1) \\
				\gamma(q+1) & \gamma(q) & \cdots & \gamma(q-p+2) \\
				\vdots & \vdots & \ddots & \vdots \\
				\gamma(q+p-1) & \gamma(q+p-2) & \cdots & \gamma(q) \\
			\end{pmatrix}
			\begin{pmatrix}
				a_1 \\
				a_2 \\
				\vdots \\
				a_p
			\end{pmatrix}
		\end{gather*}
		记上述矩阵为$\Gamma_{p,q}$;
		\item $\{X_t\}$的自协方差函数与自相关函数具有截尾性,即对任意的$k>q$有$\gamma(k)=\rho(k)=0$,且二者的模随着$n$的增大指数衰减;
		\item $\{X_t\}$具有谱密度:
		\begin{equation*}
			f(\lambda)=\frac{\sigma^2}{2\pi}\left|\sum_{j=0}^{+\infty}\psi_je^{ij\lambda}\right|^2=\frac{\sigma^2}{2\pi}\left|\frac{B(e^{i\lambda})}{A(e^{i\lambda})}\right|^2=\frac{1}{2\pi}\sum_{n=-\infty}^{+\infty}\gamma(n)e^{-in\lambda}
		\end{equation*}
		\item $\{X_t\}$的自协方差矩阵为正定矩阵;
	\end{enumerate}
\end{property}
\begin{proof}
	(1)由\cref{theo:ARMAPQSolution}(1)和\cref{prop:ARp}(1)立即可得。\par
	(2)由\cref{theo:ARMAPQSolution}(1)和\cref{prop:LinearlyStationarySeries}(1)立即可得。\par
	(3)对$j<0$定义$\psi_j=0$,则:
	\begin{align*}
		\gamma(n)&=\operatorname{E}(X_tX_{t-n})=\operatorname{E}\left[\left(\sum_{i=1}^{p}a_iX_{t-i}+\sum_{i=0}^{q}b_i\varepsilon_{t-i}\right)X_{t-n}\right] \\
		&=\operatorname{E}\left(\sum_{i=1}^{p}a_iX_{t-i}X_{t-n}\right)+\operatorname{E}\left(\sum_{i=0}^{q}b_i\varepsilon_{t-i}X_{t-n}\right) \\
		& =\sum_{i=1}^{p}a_i\gamma(n-i)+\operatorname{E}\left(\sum_{i=0}^{q}b_i\varepsilon_{t-i}\sum_{j=0}^{+\infty}\psi_j\varepsilon_{t-n-j}\right) \\
		&=\sum_{i=1}^{p}a_i\gamma(n-i)+\sigma^2\sum_{j=0}^{q}b_j\psi_{j-n}=\sum_{i=1}^{p}a_i\gamma(n-i)+\sigma^2\sum_{j=n}^{q}b_j\psi_{j-n}
	\end{align*}\par
	(4)截尾性由(3)立即得出。由\cref{ineq:cauchy-ineq-R}、(2)和\cref{theo:ARMAPQSolution}(1)中的$|\psi_i|=o(\rho^{-1})$可得:
	\begin{align*}
		|\gamma(n)|&=\left|\sigma^2\sum_{i=0}^{+\infty}\psi_i\psi_{i+n}\right|\leqslant\sigma^2\sum_{i=0}^{+\infty}|\psi_i||\psi_{i+n}|\leqslant\sigma^2\left(\sum_{i=0}^{+\infty}\psi_i^2\sum_{i=0}^{+\infty}\psi_{i+n}^2\right)^{\frac{1}{2}} \\
		&\leqslant c_0\left(\sum_{i=n}^{+\infty}\rho^{-2i}\right)^{\frac{1}{2}}\leqslant c_1\rho^{-n}
	\end{align*}
	上式第一行到第二行的过程中利用$\{\psi_n\}\in l^1$将第一个求和和$\sigma^2$放缩为$c_0$,最后一个放缩利用$\{\rho^{-2i}\}$的收敛性显然可得,于是由自协方差函数的模随着$n$的增大呈现出指数衰减,根据\cref{theo:RhotGamma0}可得自相关函数的对应结论。\par
	(5)由\cref{prop:LinearlyStationarySeries}(3)和$\Phi(z)$的定义立即可得,第三个表示方法由\cref{theo:ARMAPQSolution}(1)和\cref{theo:SpectralGamma}立即可得。\par
	(6)由(5)和\cref{theo:GammaPositiveDefinite}立即可得。
\end{proof}

\subsection{ARIMA模型}
\begin{definition}
	设$\{\varepsilon_t\}$是$\operatorname{WN}(0,\sigma^2)$,实系数多项式$A(z)=0$和$B(z)=0$没有公共根且满足$b_0=1,\;a_pb_q\ne0$和:
	\begin{equation*}
		A(z)=1-\sum_{i=1}^{p}a_iz^i\ne0,\;|z|\leqslant1\quad B(z)=\sum_{i=0}^{q}b_iz^i\ne0,\;|z|<1
	\end{equation*}
	则称差分方程:
	\begin{equation*}
		A(\mathcal{B})(1-\mathcal{B})^dX_t=B(\mathcal{B})\varepsilon_t
	\end{equation*}
	为\textbf{ARIMA模型},记为$\operatorname{ARMA}(p,d,q)$模型。若$Y_t=(1-\mathcal{B})^dX_t$是$\operatorname{ARMA}(p,q)$序列,则称$\{X_t\}$是\textbf{$\operatorname{ARIMA}(p,d,q)$序列}。
\end{definition}
\begin{theorem}
	$\operatorname{ARIMA}(p,d,q)$模型的通解为:
	\begin{equation*}
		X_t=\sum_{i=0}^{d-1}c_it^{i}+\sum_{n_{d-1}=1}^{t}\cdots\sum_{n_1=1}^{n_2}\sum_{i=1}^{n_1}Y_i
	\end{equation*}
	其中$c_i$为随机变量。
\end{theorem}
\begin{proof}
	当$k=1$时,给定初值$X_0$,有:
	\begin{align*}
		X_t=X_{t-1}+Y_t=X_{t-2}+Y_{t-1}+Y_t=\cdots=X_0+\sum_{i=1}^{t}Y_i=X_0t^0+\sum_{i=1}^{t}Y_i
	\end{align*}
	即$k=1$时结论成立。设$k=d-1$时结论成立,即:
	\begin{equation*}
		X_t=\sum_{i=1}^{d-2}c_it^{i}+\sum_{n_{d-2}=1}^{t}\cdots\sum_{n_1=1}^{n_2}\sum_{i=1}^{n_1}Y_i
	\end{equation*}
	则$k=d$时有:
	\begin{equation*}
		Y_t=(1-\mathcal{B})^dX_t=(1-\mathcal{B})^{d-1}(1-\mathcal{B})X_t
	\end{equation*}
	令$(1-\mathcal{B})X_t=X_t-X_{t-1}=Z_t$,由归纳假设可知:
	\begin{equation*}
		Z_t=\sum_{i=1}^{d-2}c_i't^{i}+\sum_{n_{d-2}=1}^{t}\cdots\sum_{n_1=1}^{n_2}\sum_{i=1}^{n_1}Y_i
	\end{equation*}
	所以:
	\begin{align*}
		X_t&=X_{t-1}+Z_t=X_{t-2}+Z_{t-1}+Z_t=X_0+\sum_{i=1}^{t}Z_i \\
		&=X_0+\sum_{i=1}^{t}\left(\sum_{j=1}^{d-2}c_j'i^{j}+\sum_{n_{d-2}=1}^{i}\cdots\sum_{n_1=1}^{n_2}\sum_{j=1}^{n_1}Y_j\right) \\
		&=X_0+\sum_{i=1}^{t}\sum_{j=1}^{d-2}c_j'i^{j}+\sum_{i=1}^{t}\sum_{n_{d-2}=1}^{i}\cdots\sum_{n_1=1}^{n_2}\sum_{j=1}^{n_1}Y_j \\
		&=X_0+\sum_{j=1}^{d-2}\sum_{i=1}^{t}c_j'i^{j}+\sum_{n_{d-1}=1}^{t}\sum_{n_{d-2}=1}^{n_{d-1}}\cdots\sum_{n_1=1}^{n_2}\sum_{j=1}^{n_1}Y_j \\
		&=X_0t^0+\sum_{i=1}^{d-1}C_it^i+\sum_{n_{d-1}=1}^{t}\sum_{n_{d-2}=1}^{n_{d-1}}\cdots\sum_{n_1=1}^{n_2}\sum_{j=1}^{n_1}Y_j \\
		&=\sum_{i=0}^{d-1}C_it^{i}+\sum_{n_{d-1}=1}^{t}\cdots\sum_{n_1=1}^{n_2}\sum_{i=1}^{n_1}Y_i
	\end{align*}
	倒数第三行到倒数第二行使用了多项式的结论,其中$C_i$由$\{X_t\}$初始的$d$个值决定。
\end{proof}
\subsubsection{单位根模型}
\begin{definition}
	称$\operatorname{ARIMA}(p,1,q)$模型为\textbf{单位根模型}。
\end{definition}