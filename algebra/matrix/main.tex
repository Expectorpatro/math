\chapter{矩阵}

\section{矩阵空间}
\begin{definition}
	由$m\times n$个数排成$m$行、$n$列的一张表称为一个$m\times n$\gls{Matrix},其中的每一个数称为这个矩阵的一个元素,第$i$行与第$j$列交叉位置的元素称为矩阵的$(i,j)$元,记作$A(i;j)$。一个$m\times n$矩阵可以简单地记作$A_{m\times n}$。如果矩阵$A$的$(i,j)$元是$a_{ij}$,那么可以记作$A=(a_{ij})$。如果一个矩阵的行数和列数相同,则称它为\gls{SquareMatrix},$n$行$n$列的方阵也成为$n$阶矩阵。对于两个矩阵$A$和$B$,如果它们的行数都等于$m$且列数都等于$n$,同时还有$A(i;j)=B(i;j),i=1,2,\dots,m,\;j=1,2,\dots,n$,那么称$A$和$B$相等,记作$A=B$。
\end{definition}
\begin{definition}
	称元素全为$0$的矩阵为\gls{ZeroMatrix},记作$\mathbf{0}$。
\end{definition}
\begin{definition}
	称主对角线元素都为$1$其他位置元素都为$0$的$n$阶方阵为$n$阶\gls{IdentityMatrix},记为$I_n$。
\end{definition}
\begin{definition}
	若一个$n$阶方阵的主对角线下方的元素全为$0$,即$a_{ij}=0(i>j)$,称该矩阵为$n$阶\gls{UpperTriangularMatrix}。对角线元素均为$1$的上三角矩阵被称为\textbf{单位上三角矩阵}。若一个$n$阶方阵的主对角线上方的元素全为$0$,即$a_{ij}=0(i<j)$,称该矩阵为$n$阶\gls{LowerTriangularMatrix}。对角线元素均为$1$的下三角矩阵被称为\textbf{单位下三角矩阵}。
\end{definition}


\subsection{矩阵的运算}
\subsubsection{加减法与数量乘法}
\begin{definition}
	将数域$K$上所有$m\times n$矩阵组成的集合记作$M_{m\times n}(K)$,当$m=n$时,$M_{m\times m}(K)$可以简记作$M_m(K)$。在$M_{m\times n}(K)$中定义如下运算:
	\begin{enumerate}
		\item \textbf{加法:} 
		\begin{equation*}
			\forall\;A=(a_{ij}),B=(b_{ij})\in M_{m\times n}(K),\;A+B=(a_{ij}+b_{ij})
		\end{equation*}
		\item \textbf{纯量乘法:}
		\begin{equation*}
			\forall\;k\in K,\;\forall\;A=(a_{ij}),\;kA=(ka_{ij})
		\end{equation*}
	\end{enumerate}
	那么$M_{m\times n}(K)$构成一个线性空间。
\end{definition}
\begin{proof}
	由数域中加法和乘法的封闭性可知如上定义的加法和纯量乘法对$M_{m\times n}(K)$是封闭的。\par
	接下来证明如上定义的加法和纯量乘法满足线性空间中的$8$条运算法则:
	\begin{enumerate}
		\item 因为数域内的数满足加法交换律与加法结合律,所以$M_{m\times n}(K)$上的加法满足线性空间运算法则(1)(2);
		\item 对任意的$A\in M_{m\times n}(K)$,有$A+\mathbf{0}=A$,因此$M_{m\times n}(K)$中存在零元且它就是$\mathbf{0}$,$M_{m\times n}(K)$上的加法满足线性空间运算法则(3);
		\item 对任意的$A\in M_{m\times n}(K)$,取$-A=(-a_{ij})$,则有$A+(-A)=(a_{ij}-a_{ij})=\mathbf{0}$。由$A$的任意性,$M_{m\times n}(K)$中的每个元素都具有负元。因此,$M_{m\times n}(K)$上的加法满足线性空间运算法则(4);
		\item 因为数域内的数满足乘法结合律和乘法分配律,同时它们乘$1$的积是自身,所以$M_{m\times n}$上的纯量乘法满足线性空间运算法则(5)(6)(7)(8)。
	\end{enumerate}
	证明完毕。
\end{proof}
\begin{definition}
	定义$M_{m\times n}(K)$上矩阵的减法如下:设$A,B\in M_{m\times n}(K)$,则:
	\begin{equation*}
		A-B\coloneq A+(-B)
	\end{equation*}
\end{definition}
\subsubsection{乘法}
\begin{definition}
	设$A=(a_{ij})_{s\times n}\in M_{s\times n}(K),\;B=(b_{ij})_{n\times m}\in M_{n\times m}(K)$,令$C=(c_{ij})_{s\times m}\in M_{s\times m}(K)$,其中:
	\begin{equation*}
		c_{ij}=\sum_{k=1}^{n}a_{ik}b_{kj},\;i=1,2,\dots,s,\;j=1,2,\dots,m
	\end{equation*}
	则称$C$为矩阵$A$与$B$的乘积,记作$C=AB$。
\end{definition}
\begin{property}\label{prop:MatrixMultiplication}
	矩阵乘法具有如下性质:
	\begin{enumerate}
		\item 设$A=(a_{ij})\in M_{m\times n}(K),\;B=(b_{ij})\in M_{n\times p}(K)$,则$AB$有如下四种理解:
		\begin{gather*}
			A=(\alpha_1^{\top};\alpha_2^{\top};\dots;\alpha_m^{\top}),\;B=(\seq{\beta}{p})\longrightarrow AB=(\alpha_i^{\top}\beta_j) \\
			A=(\seq{\alpha}{n})\longrightarrow AB=\left(\sum_{i=1}^{n}b_{i1}\alpha_i,\sum_{i=1}^{n}b_{i2}\alpha_i,\dots,\sum_{i=1}^{n}b_{ip}\alpha_i\right) \\
			B=(\beta_1^{\top};\beta_2^{\top};\beta_n^{\top})\longrightarrow AB=\left(\sum_{j=1}^{n}a_{1i}\beta_i^{\top};\sum_{j=1}^{n}a_{2i}\beta_i^{\top};\dots;\sum_{j=1}^{n}a_{mi}\beta_i^{\top}\right) \\
			A=(\seq{\alpha}{n}),\;B=(\beta_1^{\top};\beta_2^{\top};\beta_n^{\top})\longrightarrow AB=\sum_{i=1}^{n}\alpha_i\beta_i^{\top}
		\end{gather*}
		\item \textbf{结合律:}设$A\in M_{m\times n}(K),\;B\in M_{n\times p}(K),\;C\in M_{p\times q}(K)$,则$(AB)C=A(BC)$;
		\item \textbf{分配律:}设$A\in M_{m\times n}(K),\;B\in M_{n\times p}(K),\;C\in M_{n\times p}(K),\;D\in M_{p\times q}(K)$,则$A(B+C)=AB+AC,\;(B+C)D=BD+CD$;
		\item 设$A\in M_{m\times n}(K),\;B\in M_{n\times p}(K),\;k\in K$,则$k(AB)=(kA)B=A(kB)$;
		\item 设$A\in M_{m\times n}(K)$,则$AI_n=I_mA=A$;
		\item 两个上三角矩阵的乘积还是上三角矩阵,两个下三角矩阵的乘积还是下三角矩阵。
	\end{enumerate}
\end{property}
\begin{proof}
	证明过于机械,略去。
\end{proof}
\begin{definition}
	根据\cref{prop:MatrixMultiplication}(2),定义$m$阶方阵的非负幂整数幂如下:
	\begin{equation*}
		A^0=I_m,\quad A^n=\underbrace{A\cdot A \cdots A}_{n\text{个}A},\;n\in\mathbb{N}^+
	\end{equation*}
\end{definition}
\subsubsection{转置}
\begin{definition}
	设$A\in M_{m\times n}(K)$,定义$A$的\gls{Transpose}$A^{\top}\in M_{n\times m}(K)$,它的第$i$行是$A$的第$i$列,第$j$列是$A$的第$j$行。若$K=\mathbb{C}^{}$,称$A^H=\overline{A}^{\top}$为$A$的Hermitian转置或\gls{ConjugateTranspose},它是对$A$的每个元素先取复共轭再经过转置后得到的矩阵。
\end{definition}
\begin{property}\label{prop:Transpose}
	转置与共轭转置具有如下性质:
	\begin{enumerate}
		\item $A^H=\overline{A}^{\top}=\overline{A^{\top}}$;
		\item $(A^H)^H=A$,$(A^{\top})^{\top}=A$;
		\item $(A+B)^H=A^H+B^H$,$(A+B)^{\top}=A^{\top}+B^{\top}$;
		\item $(AB)^H=B^HA^H$,$(AB)^{\top}=B^{\top}A^{\top}$。
	\end{enumerate}
\end{property}
\begin{proof}
	证明过于机械,略去。
\end{proof}
\begin{definition}
	若$A^{\top}=A$,则称$A$为\gls{SymmetricMatrix}。若$A^H=A$,则称$A$为\gls{HermitianMatrix}。
\end{definition}
\subsubsection{初等变换}
\begin{definition}
	称以下变换为矩阵的\gls{ElementaryRowOperation}:
	\begin{enumerate}
		\item 把一行的倍数加到另一行上;
		\item 互换两行的位置;
		\item 用一个非零数乘某一行。
	\end{enumerate}
	称以下变换为矩阵的\gls{ElementaryColumnOperation}:
	\begin{enumerate}
		\item 把一列的倍数加到另一列上;
		\item 互换两列的位置;
		\item 用一个非零数乘某一列。
	\end{enumerate}
\end{definition}
\begin{definition}
	一个矩阵被称为\gls{REF},如果它满足以下条件:
	\begin{enumerate}
		\item 所有零行(全为零的行)位于非零行的下方;
		\item 若某一行非零,则该行的首个非零元素(称为\gls{Pivot})位于该行之前所有行的主元右侧。
	\end{enumerate}
	一个矩阵被称为\gls{RREF},如果满足以下条件:
	\begin{enumerate}
		\item 它是阶梯形矩阵;
		\item 每个非零行的主元都是$1$;
		\item 每个主元所在列的其他元素均为$0$。
	\end{enumerate}
\end{definition}
\begin{theorem}\label{theo:RREFExistence}
	任意一个矩阵都可以经过一系列初等行变换化成行阶梯形矩阵,进而可以经过一系列初等行变换化成简化行阶梯形矩阵。
\end{theorem}
\begin{proof}
	数学归纳法,化为行阶梯形矩阵只需经过前两种初等变换,由行阶梯形矩阵化为简化行阶梯形矩阵只需经过后两种初等变换。
\end{proof}
\begin{definition}
	将由单位矩阵经过一次初等行(列)变换后得到的矩阵称为\gls{ElementaryMatrix}。
\end{definition}
\begin{property}\label{prop:ElementaryMatrix}
	$n$阶初等矩阵具有如下性质:
	\begin{enumerate}
		\item 初等矩阵有且仅有三种类型,分别记为:
		\begin{gather*}
			P[j,i(k)]:\text{将$I_n$的第$j$行加上第$i$行的$k$倍,或将$I_n$的第$i$列加上第$j$列的$k$倍} \\
			P[i,j]:\text{交换$I_n$的第$i$行和第$j$行,或交换$I_n$的第$i$列和第$j$列} \\
			P[i(c)]:\text{将$I_n$的第$i$行乘非零常数$c$,或将$I_n$的第$i$列乘非零常数$c$}
		\end{gather*}
		\item 用初等矩阵左乘一个矩阵,就相当于对矩阵作从单位矩阵得到该初等矩阵的初等行变换;用初等矩阵右乘一个矩阵,就相当于对矩阵作从单位矩阵得到该初等矩阵的初等列变换;
		\item 第二种初等矩阵$P[i,j]$可以由第一种和第三种表示,即矩阵的第二种初等变换可以由一系列其它两种初等变换得到。
	\end{enumerate}
\end{property}
\begin{proof}
	(1)(2)证明过于机械,略去。\par
	(3)只需注意到$P[i,j]=P[i(-1)]P[i,j(-1)]P[j,i(1)]P[i,j(-1)]$。由(2)即可得到矩阵的第二种初等变换可以由一系列其它两种初等变换得到。
\end{proof}
\begin{definition}
	若一个$n$阶方阵的每一行与每一列中恰有一个元素为$1$,其余元素均为$0$,称该矩阵为$n$阶\gls{PermutationMatrix}。
\end{definition}
\begin{property}\label{prop:PermutationMatrix}
	用置换矩阵乘上一个矩阵表示对该矩阵作矩阵的第二类初等变换,即互换行、列的位置。
\end{property}
\begin{proof}
	置换矩阵可由单位矩阵经矩阵的第二类初等变换得到,由\cref{prop:ElementaryMatrix}(2)即可得出结论。
\end{proof}
\subsubsection{迹}
\begin{definition}
	$A=(a_{ij})\in M_{n}(K)$的主对角线上的元素之和称为$A$的\gls{Trace},记作$\operatorname{tr}(A)$,即:
	\begin{equation*}
		\operatorname{tr}(A)=\sum_{i=1}^{n}a_{ii}
	\end{equation*}
\end{definition}
\begin{property}\label{prop:Trace}
	设$A,B\in M_{n}(K),\;k\in K$,矩阵的迹具有如下性质:
	\begin{enumerate}
		\item $\operatorname{tr}(A+B)=\operatorname{tr}(A)+\operatorname{tr}(B)$;
		\item $\operatorname{tr}(kA)=k\operatorname{tr}(A)$;
		\item $\operatorname{tr}(AB)=\operatorname{tr}(BA)$。
	\end{enumerate}
\end{property}
\begin{proof}
	(1)(2)是显然的;\par
	(3)显然:
	\begin{gather*}
		\operatorname{tr}(AB)=\sum_{i=1}^{n}\sum_{j=1}^{n}a_{ij}b_{ji}=\sum_{j=1}^{n}\sum_{i=1}^{n}b_{ji}a_{ij}=\operatorname{tr}(BA)\qedhere
	\end{gather*}
\end{proof}

\subsection{矩阵的行列式}
\subsubsection{排列}
\begin{definition}
	$n$个不同的正整数的一个全排列称为一个\gls{NPermutation}。
\end{definition}
\begin{definition}
	在$n$元排列$a_1a_2\cdots a_n$中,从左到右任取一对数$a_ia_j(i<j)$,若$a_i<a_j$,则称这一对数构成一个\gls{NaturalOrder};若$a_i>a_j$,则称这一对数构成一个\gls{Inversion}。
\end{definition}
\begin{definition}
	一个$n$元排列$a_1a_2\cdots a_n$中逆序的总数称为\gls{NumberOfInversion},记作$\tau(a_1a_2\cdots a_n)$。逆序数为奇数的排列称为\gls{OddPermutation},逆序数为偶数的排列称为\gls{EvenPermutation}。
\end{definition}
\begin{definition}
	将一个排列$a_1a_2\cdots a_n$中第$i$个位置的元素$a_i$与第$j$个位置的元素$a_j$交换位置的运算称为\gls{Transposition},记作$(a_i,a_j)$。
\end{definition}
\begin{property}\label{prop:Transposition}
	$n$元排列的对换具有如下性质:
	\begin{enumerate}
		\item 对换改变$n$元排列的奇偶性;
		\item 任一$n$元排列可以经过一系列对换变为逆序数为$0$的排列,且所作的对换的次数与这个$n$元排列具有相同的奇偶性;
		\item 在所有由正整数$\seq{a}{n}(n>1)$构成的$n$元排列中,偶排列数等于奇排列数;
		\item 设$c_1c_2\cdots c_kd_1d_2\cdots d_{n-k}$是由$1,2,\dots,n$构成的一个$n$元排列,则:
		\begin{equation*}
			(-1)^{\tau(c_1c_2\cdots c_kd_1d_2\cdots d_{n-k})}=(-1)^{\tau(c_1c_2\cdots c_k)+\tau(d_1d_2\cdots d_{n-k})}\cdot(-1)^{c_1+c_2+\cdots+c_k}\cdot(-1)^{\frac{k(k+1)}{2}}
		\end{equation*}
	\end{enumerate}
\end{property}
\begin{proof}
	(1)任取一个$n$元排列$a_1a_2\cdots a_n$,若对换$(a_i,a_j)$的两个数$a_i,a_j$相邻,即:
	\begin{equation*}
		a_1a_2\cdots a_ia_j\cdots a_n\longrightarrow a_1a_2\cdots a_ja_i\cdots a_n
	\end{equation*}
	这个对换只改变了$a_i$与$a_j$构成的数对的顺逆序关系,于是对换前后排列的奇偶性相反。\par
	对于一般情况:
	\begin{equation*}
		a_1a_2\cdots a_ik_1k_2\cdots k_sa_j\cdots a_n\longrightarrow a_1a_2\cdots a_jk_2\cdots k_sa_i\cdots a_n
	\end{equation*}
	这个对换只需作$2s+1$次相邻数的对换即可达到:
	\begin{equation*}
		(a_i,k_1),\;(a_i,k_2),\;\dots,(a_i,k_s),\;(a_i,a_j),\;(k_s,a_j),\;(k_{s-1},a_j)\;\dots,(k_1,a_j)
	\end{equation*}
	所以对换前后排列的奇偶性相反。\par
	综上,对换改变$n$元排列的奇偶性。\par
	(2)满足逆序数为$0$的排列是一个偶排列,于是由(1)立即可得出结论。\par
	(3)将所有奇排列构成的集合记作$A_n$,偶排列构成的集合记作$B_n$。由(1)可知$f:(a_1,a_2)$是一个$A_n$与$B_n$之间的双射,所以奇排列数等于偶排列数。\par
	(4)设$c_1c_2\cdots c_k$经过$s$次对换得到$a_1a_2\cdots a_k$,其中$\tau(a_1a_2\cdots a_k)=0$,由(2)可得:
	\begin{align*}
		&(-1)^{\tau(c_1c_2\cdots c_kd_1d_2\cdots d_{n-k})}=(-1)^s(-1)^{\tau(a_1a_2\cdots a_kd_1d_2\cdots d_{n-k})} \\
		=&(-1)^{\tau(c_1c_2\cdots c_k)}(-1)^{\tau(d_1d_2\cdots d_{n-k})}(-1)^{(a_1-1)+(a_2-1)+\cdots+(a_k-1)} \\
		=&(-1)^{\tau(c_1c_2\cdots c_k)+\tau(d_1d_2\cdots d_{n-k})}\cdot(-1)^{c_1+c_2+\cdots+c_k}\cdot(-1)^{\frac{k(k+1)}{2}}\qedhere
	\end{align*}
\end{proof}
\subsubsection{行列式的定义与性质}
\begin{definition}
	定义$A=(a_{ij})\in M_{n}(K)$的\gls{Determinant}$\det A$为:
	\begin{equation*}
		\begin{vmatrix}
			a_{11} & a_{12} & \cdots & a_{1n} \\
			a_{21} & a_{22} & \cdots & a_{2n} \\
			\vdots & \vdots & \ddots & \vdots \\
			a_{n1} & a_{n2} & \cdots & a_{nn}
		\end{vmatrix}=
		\sum_{j_1j_2\cdots j_n}^{}(-1)^{\tau(j_1j_2\cdots j_n)}a_{1j_1}a_{2j_2}\cdots a_{nj_n}
	\end{equation*}
	其中$j_1j_2\cdots j_n$是由$1$到$n$构成的$n$元排列,$\det A$也记作$|A|$。
\end{definition} 
\begin{definition}
	设$A\in M_{n}(K)$。在$A$中任意取定$k(1\leqslant k<n)$行($i_1,i_2,\dots,i_k$)、$k$列($j_1,j_2,\dots,j_k$),位于这些行和列交叉处的$k^2$个元素按原来的次序组成的$k$阶矩阵的行列式称为$A$的一个\gls{KMinor},记作:
	\begin{equation*}
		A\left\{
		\begin{array}{*{4}{c}}
			i_1 & i_2 & \dots & i_k \\
			j_1 & j_2 & \dots & j_k
		\end{array}\right\}
	\end{equation*}
	其余元素按原来的次序组成的$n-k$阶矩阵的行列式称为上式的\gls{Minor},称:
	\begin{equation*}
		(-1)^{(i_1+i_2+\cdots+i_k)+(j_1+j_2+\cdots+j_k)}A\left\{
		\begin{array}{*{4}{c}}
			i_1' & i_2' & \dots & i_{n-k}' \\
			j_1' & j_2' & \dots & j_{n-k}'
		\end{array}\right\}
	\end{equation*}
	为其\gls{Cofactor},其中:
	\begin{gather*}
		\{i_1',i_2',\dots,i_{n-k}'\}=\{1,2,\dots,n\}\setminus\{i_1,i_2,\dots,i_k\} \\
		\{j_1',j_2',\dots,j_{n-k}'\}=\{1,2,\dots,n\}\setminus\{j_1,j_2,\dots,j_k\}
	\end{gather*}
	特别的,$A$的$(i,j)$元的余子式记作$M_{ij}$,代数余子式记作$A_{ij}$。若选取的行号和列号相同,则称这些行和列交叉处的$k^2$个元素按原来的次序组成的$k$阶矩阵为\gls{PrincipalKSubmatrix},其行列式为$A$的\gls{PrincipalKMinor};若选取的行和列是前$k$行与前$k$列,则称这些行和列交叉处的$k^2$个元素按原来的次序组成的$k$阶矩阵为\gls{LeadingPrincipalKSubmatrix},其行列式为$A$的\gls{LeadingPrincipalKMinor}。对于非方阵的矩阵,可类似定义。
\end{definition}
\begin{property}\label{prop:Determinant}
	矩阵$A=(a_{ij})\in M_{n}(K)$的行列式$\det A$具有如下性质:
	\begin{enumerate}
		\item $\det A$有等价表达式:
		\begin{gather*}
			\sum_{k_1k_2\cdots k_n}^{}(-1)^{\tau(i_1i_2\cdots i_n)+\tau(k_1k_2\cdots k_n)}a_{i_1k_1}a_{i_2k_2}\cdots a_{i_nk_n} \\
			\sum_{i_1i_2\cdots i_n}^{}(-1)^{\tau(i_1i_2\cdots i_n)+\tau(k_1k_2\cdots k_n)}a_{i_1k_1}a_{i_2k_2}\cdots a_{i_nk_n} \\
			\sum_{i_1i_2\cdots i_n}^{}(-1)^{\tau(i_1i_2\cdots i_n)}a_{i_11}a_{i_22}\cdots a_{i_nn}
		\end{gather*}
		\item $\det A^{\top}=\det A,\;|A^H|=\overline{|A|}$;
		\item 行列式一行(列)的公因子可以提出去:
		\begin{gather*}
			\begin{vmatrix}
				a_{11} & a_{12} & \cdots & a_{1n} \\
				\vdots & \vdots & \ddots & \vdots \\
				ka_{i1} & ka_{i2} & \cdots & ka_{in} \\
				\vdots & \vdots & \ddots & \vdots \\
				a_{n1} & a_{n2} & \cdots & a_{nn}
			\end{vmatrix}=k
			\begin{vmatrix}
				a_{11} & a_{12} & \cdots & a_{1n} \\
				\vdots & \vdots & \ddots & \vdots \\
				a_{i1} & a_{i2} & \cdots & a_{in} \\
				\vdots & \vdots & \ddots & \vdots \\
				a_{n1} & a_{n2} & \cdots & a_{nn}
			\end{vmatrix} \\
			\begin{vmatrix}
				a_{11} & \cdots & ka_{1j} & \cdots & a_{1n} \\
				a_{21} & \cdots & ka_{2j} & \cdots & a_{2n} \\
				\vdots & \ddots & \vdots & \ddots & \vdots \\
				a_{n1} & \cdots & ka_{nj} & \cdots & a_{nn}
			\end{vmatrix}=k
			\begin{vmatrix}
				a_{11} & \cdots & a_{1j} & \cdots & a_{1n} \\
				a_{21} & \cdots & a_{2j} & \cdots & a_{2n} \\
				\vdots & \ddots & \vdots & \ddots & \vdots \\
				a_{n1} & \cdots & a_{nj} & \cdots & a_{nn}
			\end{vmatrix}
		\end{gather*}
		\item 行列式的行(列)可以作拆分:
		\begin{gather*}
			\begin{vmatrix}
				a_{11} & a_{12} & \cdots & a_{1n} \\
				\vdots & \vdots & \ddots & \vdots \\
				a_{i1}+a_{i1}' & a_{i2}+a_{i2}' & \cdots & a_{in}+a_{in}' \\
				\vdots & \vdots & \ddots & \vdots \\
				a_{n1} & a_{n2} & \cdots & a_{nn}
			\end{vmatrix}=
			\begin{vmatrix}
				a_{11} & a_{12} & \cdots & a_{1n} \\
				\vdots & \vdots & \ddots & \vdots \\
				a_{i1} & a_{i2} & \cdots & a_{in} \\
				\vdots & \vdots & \ddots & \vdots \\
				a_{n1} & a_{n2} & \cdots & a_{nn}
			\end{vmatrix}+
			\begin{vmatrix}
				a_{11} & a_{12} & \cdots & a_{1n} \\
				\vdots & \vdots & \ddots & \vdots \\
				a_{i1}' & a_{i2}' & \cdots & a_{in}' \\
				\vdots & \vdots & \ddots & \vdots \\
				a_{n1} & a_{n2} & \cdots & a_{nn}
			\end{vmatrix} \\
			\begin{vmatrix}
				a_{11} & \cdots & a_{1j}+a_{1j}' & \cdots & a_{1n} \\
				a_{21} & \cdots & a_{2j}+a_{2j}' & \cdots & a_{2n} \\
				\vdots & \ddots & \vdots & \ddots & \vdots \\
				a_{n1} & \cdots & a_{nj}+a_{nj}' & \cdots & a_{nn}
			\end{vmatrix}=
			\begin{vmatrix}
				a_{11} & \cdots & a_{1j} & \cdots & a_{1n} \\
				a_{21} & \cdots & a_{2j} & \cdots & a_{2n} \\
				\vdots & \ddots & \vdots & \ddots & \vdots \\
				a_{n1} & \cdots & a_{nj} & \cdots & a_{nn}
			\end{vmatrix}+
			\begin{vmatrix}
				a_{11} & \cdots & a_{1j}' & \cdots & a_{1n} \\
				a_{21} & \cdots & a_{2j}' & \cdots & a_{2n} \\
				\vdots & \ddots & \vdots & \ddots & \vdots \\
				a_{n1} & \cdots & a_{nj}' & \cdots & a_{nn}
			\end{vmatrix}
		\end{gather*}
		\item 行列式两行(列)互换,行列式反号:
		\begin{gather*}
			\begin{vmatrix}
				a_{11} & a_{12} & \cdots & a_{1n} \\
				\vdots & \vdots & \ddots & \vdots \\
				a_{i1} & a_{i2} & \cdots & a_{in} \\
				\vdots & \vdots & \ddots & \vdots \\
				a_{j1} & a_{j2} & \cdots & a_{jn} \\
				\vdots & \vdots & \ddots & \vdots \\
				a_{n1} & a_{n2} & \cdots & a_{nn}
			\end{vmatrix}=-
			\begin{vmatrix}
				a_{11} & a_{12} & \cdots & a_{1n} \\
				\vdots & \vdots & \ddots & \vdots \\
				a_{j1} & a_{j2} & \cdots & a_{jn} \\
				\vdots & \vdots & \ddots & \vdots \\
				a_{i1} & a_{i2} & \cdots & a_{in} \\
				\vdots & \vdots & \ddots & \vdots \\
				a_{n1} & a_{n2} & \cdots & a_{nn}
			\end{vmatrix} \\
			\begin{vmatrix}
				a_{11} & \cdots & a_{1i} & \cdots & a_{1j} & \cdots & a_{1n} \\
				a_{21} & \cdots & a_{2i} & \cdots & a_{2j} & \cdots & a_{2n} \\
				\vdots & \ddots & \vdots & \ddots & \vdots & \ddots & \vdots \\
				a_{n1} & \cdots & a_{ni} & \cdots & a_{nj} & \cdots & a_{nn}
			\end{vmatrix}=-
			\begin{vmatrix}
				a_{11} & \cdots & a_{1j} & \cdots & a_{1i} & \cdots & a_{1n} \\
				a_{21} & \cdots & a_{2j} & \cdots & a_{2i} & \cdots & a_{2n} \\
				\vdots & \ddots & \vdots & \ddots & \vdots & \ddots & \vdots \\
				a_{n1} & \cdots & a_{nj} & \cdots & a_{ni} & \cdots & a_{nn}
			\end{vmatrix}
		\end{gather*}
		\item 行列式两行(列)成比例则值为$0$:
		\begin{equation*}
			\begin{vmatrix}
			a_{11} & a_{12} & \cdots & a_{1n} \\
			\vdots & \vdots & \ddots & \vdots \\
			a_{i1} & a_{i2} & \cdots & a_{in} \\
			\vdots & \vdots & \ddots & \vdots \\
			ka_{i1} & ka_{i2} & \cdots & ka_{in} \\
			\vdots & \vdots & \ddots & \vdots \\
			a_{n1} & a_{n2} & \cdots & a_{nn}
			\end{vmatrix}=
			\begin{vmatrix}
			a_{11} & \cdots & a_{1j} & \cdots & ka_{1j} & \cdots & a_{1n} \\
			a_{21} & \cdots & a_{2j} & \cdots & ka_{2j} & \cdots & a_{2n} \\
			\vdots & \ddots & \vdots & \ddots & \vdots & \ddots & \vdots \\
			a_{n1} & \cdots & a_{nj} & \cdots & ka_{nj} & \cdots & a_{nn}
			\end{vmatrix}=0
		\end{equation*}
		\item 把一行(列)的倍数加到另一行(列)上行列式的值不变:
		\begin{gather*}
			\begin{vmatrix}
				a_{11} & a_{12} & \cdots & a_{1n} \\
				\vdots & \vdots & \ddots & \vdots \\
				a_{i1} & a_{i2} & \cdots & a_{in} \\
				\vdots & \vdots & \ddots & \vdots \\
				a_{j1}+ka_{i1} & a_{j2}+ka_{i2} & \cdots & a_{jn}+ka_{in} \\
				\vdots & \vdots & \ddots & \vdots \\
				a_{n1} & a_{n2} & \cdots & a_{nn}
			\end{vmatrix}=
			\begin{vmatrix}
				a_{11} & a_{12} & \cdots & a_{1n} \\
				\vdots & \vdots & \ddots & \vdots \\
				a_{i1} & a_{i2} & \cdots & a_{in} \\
				\vdots & \vdots & \ddots & \vdots \\
				a_{j1} & a_{j2} & \cdots & a_{jn} \\
				\vdots & \vdots & \ddots & \vdots \\
				a_{n1} & a_{n2} & \cdots & a_{nn}
			\end{vmatrix} \\
			\begin{vmatrix}
				a_{11} & \cdots & a_{1i} & \cdots & a_{1j}+ka_{1i} & \cdots & a_{1n} \\
				a_{21} & \cdots & a_{2i} & \cdots & a_{2j}+ka_{2i} & \cdots & a_{2n} \\
				\vdots & \ddots & \vdots & \ddots & \vdots & \ddots & \vdots \\
				a_{n1} & \cdots & a_{ni} & \cdots & a_{nj}+ka_{ni}  & \cdots & a_{nn}
			\end{vmatrix}=
			\begin{vmatrix}
				a_{11} & \cdots & a_{1i} & \cdots & a_{1j} & \cdots & a_{1n} \\
				a_{21} & \cdots & a_{2i} & \cdots & a_{2j} & \cdots & a_{2n} \\
				\vdots & \ddots & \vdots & \ddots & \vdots & \ddots & \vdots \\
				a_{n1} & \cdots & a_{ni} & \cdots & a_{nj} & \cdots & a_{nn}
			\end{vmatrix}
		\end{gather*}
		\item $A$的行列式等于它的第$i$行(第$j$列)元素与自己代数余子式的乘积之和,即:
		\begin{equation*}
			\det A=\sum_{j=1}^{n}a_{ij}A_{ij}=\sum_{i=1}^{n}a_{ij}A_{ij}
		\end{equation*}
		\item $A$的第$i$行(列)元素与第$j(j\ne i)$行(列)相应元素的代数余子式之和等于$0$,即:
		\begin{equation*}
			\sum_{k=1}^{n}a_{ik}A_{jk}=\sum_{k=1}^{n}a_{ki}A_{kj}=0
		\end{equation*}
		\item (Laplace Theorem) 取定$A$的第$i_1,i_2,\dots,i_k(i_1<i_2<\cdots<i_k)$行,则这$k$行元素构成的所有$k$阶子式与它们自己的代数余子式的乘积之和等于$\det A$,即:
		\begin{align*}
			\det A&=\sum_{1\leqslant j_1<j_2<\cdots<j_k\leqslant n}A\left\{
			\begin{array}{*{4}{c}}
				i_1 & i_2,\dots & i_k \\
				j_1 & j_2,\dots & j_k
			\end{array}\right\} \\
			&\quad(-1)^{(i_1+i_2+\cdots+i_k)+(j_1+j_2+\cdots+j_k)}A\left\{
			\begin{array}{*{4}{c}}
				i_1' & i_2' & \dots & i_{n-k}' \\
				j_1' & j_2' & \dots & j_{n-k}'
			\end{array}\right\}
		\end{align*}
		\item (Cauchy-Binet Formula) 设$P=(p_{ij})\in M_{m\times n}(K),Q=(q_{ij})\in M_{n\times m}(K)$,$r\in\mathbb{N}^+$且$r\leqslant m$。
		\begin{enumerate}
			\item 若$r>n$,则$PQ$的任意一个$r$阶子式为$0$;
			\item 若$r\leqslant n$,则:
			\begin{align*}
				(PQ)\left\{
				\begin{array}{*{4}{c}}
					i_1 & i_2 & \dots & i_r \\
					j_1 & j_2 & \dots & j_r
				\end{array}\right\}&=\sum_{1\leqslant k_1<k_2<\cdots<k_r\leqslant n}^{}P\left\{
				\begin{array}{*{4}{c}}
					i_1 & i_2 & \dots & i_r \\
					k_1 & k_2 & \dots & k_r
				\end{array}\right\} \\
				&\quad Q\left\{
				\begin{array}{*{4}{c}}
					k_1 & k_2 & \dots & k_r \\
					j_1 & j_2 & \dots & j_r
				\end{array}\right\}
			\end{align*}
		\end{enumerate}
		上述结论内含当$m=n$时,$\det(PQ)=\det P\det Q$;
		\item 若$A$是上三角矩阵或下三角矩阵,则$\det A=\prod\limits_{i=1}^na_{ii}$;
		\item 称下述行列式为\textbf{Vandermonde行列式}:
		\begin{equation*}
			\begin{vmatrix}
				1 & 1 & 1 & \cdots & 1 \\
				x_1 & x_2 & x_3 & \cdots & x_n \\
				x_1^2 & x_2^2 & x_3^2 & \cdots & x_n^2 \\
				\vdots & \vdots & \vdots & \ddots & \vdots \\
				x_1^{n-2} & x_2^{n-2} & x_3^{n-2} & \cdots & x_n^{n-2} \\
				x_1^{n-1} & x_2^{n-1} & x_3^{n-1} & \cdots & x_n^{n-1}
			\end{vmatrix}=\prod_{1\leqslant i<j\leqslant n}(x_j-x_i)
		\end{equation*}
	\end{enumerate}
\end{property}
\begin{proof}
	(1)对排列$a_{i_1k_1}a_{i_2k_2}\cdots a_{i_nk_n}$进行考察,设其进行了$s$次对换得到$a_{1j_1}a_{2j_2}\cdots a_{nj_n}$。由\cref{prop:Transposition}(2)可知:
	\begin{equation*}
		(-1)^{\tau(i_1i_2\cdots i_n)}(-1)^{s}=(-1)^{\tau(12\cdots n)}=1,\quad
		(-1)^{\tau(k_1k_2\cdots k_n)}(-1)^{s}=(-1)^{\tau(j_1j_2\cdots j_n)}
	\end{equation*}
	所以:
	\begin{equation*}
		(-1)^{\tau(i_1i_2\cdots i_n)}(-1)^{s}(-1)^{\tau(k_1k_2\cdots k_n)}(-1)^{s}=(-1)^{\tau(j_1j_2\cdots j_n)}
	\end{equation*}
	即:
	\begin{equation*}
		(-1)^{\tau(i_1i_2\cdots i_n)+\tau(k_1k_2\cdots k_n)}=(-1)^{\tau(j_1j_2\cdots j_n)}
	\end{equation*}
	于是:
	\begin{equation*}
		\sum_{j_1j_2\cdots j_n}^{}(-1)^{\tau(j_1j_2\cdots j_n)}a_{1j_1}a_{2j_2}\cdots a_{nj_n}=\sum_{k_1k_2\cdots k_n}^{}(-1)^{\tau(i_1i_2\cdots i_n)+\tau(k_1k_2\cdots k_n)}a_{i_1k_1}a_{i_2k_2}\cdots a_{nk_n}
	\end{equation*}\par
	第二式同理可得,第三式可由第二式推出。\par
	(2)利用(1)将行列式分别按行顺序与列顺序展开即可得到。共轭转置的情况由\info{乘积的共轭等于共轭的乘积}即可得出。\par
	(3)由定义可得:
	\begin{equation*}
		\sum_{j_1j_2\cdots j_n}^{}(-1)^{\tau(j_1j_2\cdots j_n)}a_{1j_1}a_{2j_2}\cdots ka_{ij_i}\cdots a_{nj_n}=k\sum_{j_1j_2\cdots j_n}^{}(-1)^{\tau(j_1j_2\cdots j_n)}a_{1j_1}a_{2j_2}\cdots a_{ij_i}\cdots a_{nj_n}
	\end{equation*}
	列的结果由(2)和行的结果即可得到。\par
	(4)由定义可得:
	\begin{align*}
		&\sum_{j_1j_2\cdots j_n}^{}(-1)^{\tau(j_1j_2\cdots j_n)}a_{1j_1}a_{2j_2}\cdots (a_{ij_i}+a_{ij_i}')\cdots a_{nj_n} \\
		=&\sum_{j_1j_2\cdots j_n}^{}(-1)^{\tau(j_1j_2\cdots j_n)}a_{1j_1}a_{2j_2}\cdots a_{ij_i}\cdots a_{nj_n} \\
		&+\sum_{j_1j_2\cdots j_n}^{}(-1)^{\tau(j_1j_2\cdots j_n)}a_{1j_1}a_{2j_2}\cdots a_{ij_i}'\cdots a_{nj_n}
	\end{align*}
	列的结果由(2)和行的结果即可得到。\par
	(5)由定义和\cref{prop:Transposition}(1)可得:
	\begin{align*}
		&\sum_{j_1j_2\cdots j_n}^{}(-1)^{\tau(j_1j_2\cdots j_i\cdots j_k\cdots j_n)}a_{1j_1}a_{2j_2}\cdots a_{ij_i}\cdots a_{kj_k}\cdots a_{nj_n} \\
		=&\sum_{j_1j_2\cdots j_n}^{}(-1)^{\tau(j_1j_2\cdots j_k\cdots j_i\cdots j_n)}a_{1j_1}a_{2j_2}\cdots a_{kj_k}\cdots a_{ij_i}\cdots a_{nj_n} \\
		=&(-1)\sum_{j_1j_2\cdots j_n}^{}(-1)^{\tau(j_1j_2\cdots j_i\cdots j_k\cdots j_n)}a_{1j_1}a_{2j_2}\cdots a_{ij_i}\cdots a_{kj_k}\cdots a_{nj_n}
	\end{align*}\par
	(6)由(3)(5)可得。\par
	(7)由(3)(6)可得。\par
	(8)由(1)可得:
	\begin{align*}
		\det A&=\sum_{k_1k_2\cdots k_{i-1}jk_{i+1}\cdots k_n}^{}(-1)^{\tau(k_1k_2\cdots k_{i-1}jk_{i+1}\cdots k_n)}a_{1k_1}a_{2k_2}\cdots a_{(i-1)k_{i-1}}a_{ij}a_{(i+1)k_{i+1}}\cdots a_{nk_n} \\
		&=\sum_{jk_1k_2\cdots k_{i-1}k_{i+1}\cdots k_n}^{}(-1)^{\tau[i12\cdots(i-1)(i+1)\cdots n]+\tau(jk_1k_2\cdots k_{i-1}k_{i+1}\cdots k_n)} \\
		&\quad a_{ij}a_{1k_1}a_{2k_2}\cdots a_{(i-1)k_{i-1}}a_{(i+1)k_{i+1}}\cdots a_{nk_n} \\
		&=\sum_{jk_1k_2\cdots k_{i-1}k_{i+1}\cdots k_n}^{}(-1)^{i-1}(-1)^{j-1}(-1)^{\tau(k_1k_2\cdots k_{i-1}k_{i+1}\cdots k_n)} \\
		&\quad a_{ij}a_{1k_1}a_{2k_2}\cdots a_{(i-1)k_{i-1}}a_{(i+1)k_{i+1}}\cdots a_{nk_n} \\
		&=\sum_{j=1}^{n}(-1)^{i-1}(-1)^{j-1}a_{ij}\sum_{k_1k_2\cdots k_{i-1}k_{i+1}\cdots k_n}(-1)^{\tau(k_1k_2\cdots k_{i-1}k_{i+1}\cdots k_n)} \\
		&\quad a_{1k_1}a_{2k_2}\cdots a_{(i-1)k_{i-1}}a_{(i+1)k_{i+1}}\cdots a_{nk_n} \\
		&=\sum_{j=1}^{n}(-1)^{i+j}a_{ij}M_{ij}=\sum_{j=1}^{n}a_{ij}A_{ij}
	\end{align*}
	列的结果由(2)和行的结果即可得到。\par
	(9)由(8)(6)(2)即可得到。\par
	(10)给定$A$的一个行指标$i_1i_2\cdots i_ki_1'i_2'\cdots i_{n-k}'$,由(1)可得:
	\begin{align*}
		\det A&=\sum_{\mu_1\mu_2\cdots\mu_k\nu_1\nu_2\cdots\nu_{n-k}}(-1)^{\tau(i_1i_2\cdots i_ki_1'i_2'\cdots i_{n-k}')+\tau(\mu_1\mu_2\cdots\mu_k\nu_1\nu_2\cdots\nu_{n-k})} \\
		&\quad a_{i_1\mu_1}a_{i_2\mu_2}\cdots a_{i_k\mu_k}a_{i_1'\nu_1}a_{i_2'\nu_2}\cdots a_{i_{n-k}'\nu_{n-k}}
	\end{align*}
	将这$n!$项进行分组:任意取定$\seq{j}{k}$列,其中$\tau(j_1j_2\cdots j_k)=0$,对应于选定结果的$n$元排列形如:
	\begin{equation*}
		\mu_1\mu_2\cdots\mu_k\nu_1\nu_2\cdots\nu_{n-k}
	\end{equation*}
	其中$\mu_1\mu_2\cdots\mu_k$是$\seq{j}{k}$形成的排列,$\nu_1\nu_2\cdots\nu_{n-k}$是$\{1,2,\dots,n\}\setminus\{\seq{j}{k}\}=\{j_1',j_2',\dots,j_{n-k}'\}$形成的排列。根据\cref{prop:Transposition}(4)可得:
	\begin{align*}
		\det A&=\sum_{1\leqslant j_1<\cdots<j_k\leqslant n}^{}\sum_{\mu_1\mu_2\cdots\mu_k}^{}\sum_{\nu_1\nu_2\cdots\nu_{n-k}}^{}(-1)^{\tau(i_1i_2\cdots i_ki_1'i_2'\cdots i_{n-k}')+\tau(\mu_1\mu_2\cdots\mu_k\nu_1\nu_2\cdots\nu_{n-k})} \\
		&\quad a_{i_1\mu_1}a_{i_2\mu_2}\cdots a_{i_k\mu_k}a_{i_1'\nu_1}a_{i_2'\nu_2}\cdots a_{i_{n-k}'\nu_{n-k}} \\
		&=\sum_{1\leqslant j_1<\cdots<j_k\leqslant n}^{}\sum_{\mu_1\mu_2\cdots\mu_k}^{}\sum_{\nu_1\nu_2\cdots\nu_{n-k}}^{}(-1)^{(i_1-1)+(i_2-1)+\cdots+(i_{k}-1)}(-1)^{\tau(\mu_1\mu_2\cdots\mu_k)+\tau(\nu_1\nu_2\cdots\nu_{n-k})} \\
		&\quad(-1)^{j_1+j_2+\cdots+j_k}(-1)^{\frac{k(k+1)}{2}}a_{i_1\mu_1}a_{i_2\mu_2}\cdots a_{i_k\mu_k}a_{i_1'\nu_1}a_{i_2'\nu_2}\cdots a_{i_{n-k}'\nu_{n-k}} \\
		&=\sum_{1\leqslant j_1<\cdots<j_k\leqslant n}^{}\sum_{\mu_1\mu_2\cdots\mu_k}^{}\sum_{\nu_1\nu_2\cdots\nu_{n-k}}^{}(-1)^{i_1+i_2+\cdots+i_k}(-1)^{\tau(\mu_1\mu_2\cdots\mu_k)+\tau(\nu_1\nu_2\cdots\nu_{n-k})} \\
		&\quad(-1)^{j_1+j_2+\cdots+j_k}a_{i_1\mu_1}a_{i_2\mu_2}\cdots a_{i_k\mu_k}a_{i_1'\nu_1}a_{i_2'\nu_2}\cdots a_{i_{n-k}'\nu_{n-k}} \\
		&=\sum_{1\leqslant j_1<\cdots<j_k\leqslant n}^{}(-1)^{(i_1+i_2+\cdots+i_k)+(j_1+j_2+\cdots+j_k)} \\
		&\quad\sum_{\mu_1\mu_2\cdots\mu_k}^{}(-1)^{\tau(\mu_1\mu_2\cdots\mu_k)}a_{i_1\mu_1}a_{i_2\mu_2}\cdots a_{i_k\mu_k} \\
		&\quad\sum_{\nu_1\nu_2\cdots\nu_{n-k}}^{}(-1)^{\tau(\nu_1\nu_2\cdots\nu_{n-k})}a_{i_1'\nu_1}a_{i_2'\nu_2}\cdots a_{i_{n-k}'\nu_{n-k}} \\
		&=\sum_{1\leqslant j_1<j_2<\cdots<j_k\leqslant n}(-1)^{(i_1+i_2+\cdots+i_k)+(j_1+j_2+\cdots+j_k)} \\
		&\quad A\left\{
		\begin{array}{*{4}{c}}
			i_1 & i_2 & \dots & i_k \\
			j_1 & j_2 & \dots & j_k
		\end{array}\right\}A\left\{
		\begin{array}{*{4}{c}}
			i_1' & i_2' & \dots & i_{n-k}' \\
			j_1' & j_2' & \dots & j_{n-k}'
		\end{array}\right\}
	\end{align*}\par
	(11)先证明:
	\begin{enumerate}
		\item 若$m>n$,则$\det(PQ)=0$;
		\item 若$m\leqslant n$,则:
		\begin{equation*}
			\det(PQ)=\sum_{1\leqslant j_1<j_2<\cdots<j_m\leqslant n}^{}P
			\left\{\begin{array}{*{4}{c}}
				1 & 2 & \dots & m \\
				j_1 & j_2 & \dots & j_m
			\end{array} \right\}Q
			\left\{\begin{array}{*{4}{c}}
				j_1 & j_2 & \dots & j_m \\
				1 & 2 & \dots & m
			\end{array}\right\}
		\end{equation*}
	\end{enumerate}\par
	令:
	\begin{equation*}
		C=
		\begin{pmatrix}
			P & \mathbf{0} \\
			-I_n & Q
		\end{pmatrix}
	\end{equation*}
	由(7)、(10)和(3)可得:
	\begin{align*}
		&\det C=\begin{vmatrix}
			P & \mathbf{0} \\
			-I_n & Q
		\end{vmatrix}=
		\begin{vmatrix}
			\mathbf{0} & PQ \\
			-I_n & Q
		\end{vmatrix} \\
		=&\det(PQ)(-1)^{(1+2+\cdots+m)+[(n+1)+(n+2)+\cdots+(n+m)]}\det(-I_n) \\
		=&\det(PQ)(-1)^{2(1+2+\cdots+m)+mn}(-1)^n=\det(PQ)(-1)^{mn+n}=\det(PQ)(-1)^{n(m+1)}
	\end{align*}\par
	当$m>n$时,$C$的前$m$行中的所有$m$阶子式至少有一列全为$0$,由(10)可得$\det C=0$,于是$\det(PQ)=0$。\par
	当$m\leqslant n$时,由(10)可得:
	\begin{align*}
		\det C&=\sum_{1\leqslant j_1<j_2<\cdots<j_m\leqslant m+n}C\left\{ \begin{array}{*{4}{c}}
			1 & 2 & \dots & m \\
			j_1 & j_2 & \dots & j_m
		\end{array}\right\} \\
		&\quad(-1)^{(1+2+\cdots+m)+(j_1+j_2+\cdots+j_m)}C\left\{
		\begin{array}{*{4}{c}}
			m+1 & m+2 & \dots & m+n \\
			j_1' & j_2' & \dots & j_{n}'
		\end{array}\right\} \\
		&=\sum_{1\leqslant j_1<j_2<\cdots<j_m\leqslant n}P\left\{
		\begin{array}{*{4}{c}}
			1 & 2 & \dots & m \\
			j_1 & j_2 & \dots & j_m
		\end{array}\right\} \\
		&\quad(-1)^{(1+2+\cdots+m)+(j_1+j_2+\cdots+j_m)}C\left\{
		\begin{array}{*{4}{c}}
			m+1 & m+2 & \dots & m+n \\
			j_1' & j_2' & \dots & j_{n}'
		\end{array}\right\}
	\end{align*}
	令$\{\seq{i}{n-m}\}=\{1,2,\dots,n\}\setminus\{	j_1,j_2,\dots,j_m\}$,$e_{i_k}$为第$i_k$维为$1$其余维度元素都为$0$的$n$维列向量,由(2)(10)可得:
	\begin{align*}
		&C\left\{
		\begin{array}{*{4}{c}}
			m+1 & m+2 & \dots & m+n \\
			j_1' & j_2' & \dots & j_{n}'
		\end{array}\right\}=
		\begin{vmatrix}
			-e_{i_1} & -e_{i_2} & \cdots & -e_{i_{n-m}} & Q
		\end{vmatrix} \\
		=&(-1)^{n-m}(-1)^{(1+2+\cdots+n-m)+(i_1+i_2+\cdots+i_{n-m})}Q\left\{ \begin{array}{*{4}{c}}
			j_1 & j_2 & \dots & j_m \\
			1   & 2   & \dots & m
		\end{array}\right\}
	\end{align*}
	所以:
	\begin{align*}
		\det C&=\sum_{1\leqslant j_1<j_2<\cdots<j_m\leqslant n}P\left\{ \begin{array}{*{4}{c}}
			1 & 2 & \dots & m \\
			j_1 & j_2 & \dots & j_m
		\end{array}\right\}Q\left\{
		\begin{array}{*{4}{c}}
			j_1 & j_2 & \dots & j_m \\
			1 & 2 & \dots & m
		\end{array}\right\}  \\
		&\quad(-1)^{(1+2+\cdots+m)+(j_1+j_2+\cdots+j_m)}(-1)^{n-m}(-1)^{(1+2+\cdots+n-m)+(i_1+i_2+\cdots+i_{n-m})} \\
		&=\sum_{1\leqslant j_1<j_2<\cdots<j_m\leqslant n}P\left\{
		\begin{array}{*{4}{c}}
			1 & 2 & \dots & m \\
			j_1 & j_2 & \dots & j_m
		\end{array}\right\}Q\left\{\begin{array}{*{4}{c}}
			j_1 & j_2 & \dots & j_m \\
			1 & 2 & \dots & m
		\end{array}\right\}  \\
		&\quad(-1)^{(1+2+\cdots+m)+(1+2+\cdots+n)+(1+2+\cdots+n-m)+n-m}
	\end{align*}
	于是:
	\begin{align*}
		\det(PQ)&=\sum_{1\leqslant j_1<j_2<\cdots<j_m\leqslant n}P\left\{
		\begin{array}{*{4}{c}}
			1 & 2 & \dots & m \\
			j_1 & j_2 & \dots & j_m
		\end{array}\right\}Q\left\{\begin{array}{*{4}{c}}
			j_1 & j_2 & \dots & j_m \\
			1 & 2 & \dots & m
		\end{array}\right\}  \\
		&\quad(-1)^{(1+2+\cdots+m)+(1+2+\cdots+n)+(1+2+\cdots+n-m)+n-m+n(m+1)}
	\end{align*}
	因为:
	\begin{align*}
		&(-1)^{(1+2+\cdots+m)+(1+2+\cdots+n)+(1+2+\cdots+n-m)+n-m+n(m+1)} \\
		=&(-1)^{m^2+n^2+3n-m}=(-1)^{m(m-1)+n(n+3)}=1
	\end{align*}
	所以:
	\begin{equation*}
		\det(PQ)=\sum_{1\leqslant j_1<j_2<\cdots<j_m\leqslant n}P\left\{
		\begin{array}{*{4}{c}}
			1 & 2 & \dots & m \\
			j_1 & j_2 & \dots & j_m
		\end{array}\right\}Q\left\{\begin{array}{*{4}{c}}
			j_1 & j_2 & \dots & j_m \\
			1 & 2 & \dots & m
		\end{array}\right\}
	\end{equation*}\par
	对于$PQ$的$r$阶子式,只需注意到:
	\begin{equation*}
		(PQ)\left\{
		\begin{array}{*{4}{c}}
			i_1 & i_2 & \dots & i_r \\
			j_1 & j_2 & \dots & j_r
		\end{array}\right\}
	\end{equation*}
	是由$P$的$i_1,i_2,\dots,i_r$行构成的矩阵和$Q$的$j_1,j_2,\dots,j_r$列构成的矩阵的乘积,直接由前面已经证明过的结论即可得出最终结论。\par
	(13)由行列式的定义立即可得。\par
	(14)当$n=2$时结论显然成立。\par
	假设对$n-1$阶Vandermonde行列式结论成立,对于$n$阶Vandermonde行列式从最后一行到第二行将上一行的$-x_1$倍加到下一行上,由(7)(3)(8)和归纳假设可得:
	\begin{align*}
		&
		\begin{vmatrix}
			1 & 1 & 1 & \cdots & 1 \\
			x_1 & x_2 & x_3 & \cdots & x_n \\
			x_1^2 & x_2^2 & x_3^2 & \cdots & x_n^2 \\
			\vdots & \vdots & \vdots & \ddots & \vdots \\
			x_1^{n-2} & x_2^{n-2} & x_3^{n-2} & \cdots & x_n^{n-2} \\
			x_1^{n-1} & x_2^{n-1} & x_3^{n-1} & \cdots & x_n^{n-1}
		\end{vmatrix} \\
		=&
		\begin{vmatrix}
			1 & 1 & 1 & \cdots & 1 \\
			0 & x_2-x_1 & x_3-x_1 & \cdots & x_n-x_1 \\
			0 & x_2^2-x_2x_1 & x_3^2-x_3x_1 & \cdots & x_n^2-x_nx_1 \\
			\vdots & \vdots & \vdots & \ddots & \vdots \\
			0 & x_2^{n-2}-x_2^{n-3}x_1 & x_3^{n-2}-x_3^{n-3}x_1 & \cdots & x_n^{n-2}-x_n^{n-3}x_1 \\
			0 & x_2^{n-1}-x_2^{n-2}x_1 & x_3^{n-1}-x_3^{n-2}x_1 & \cdots & x_n^{n-1}-x_n^{n-2}x_1
		\end{vmatrix} \\
		=&\prod_{i=2}^{n}(x_i-x_1)
		\begin{vmatrix}
			1 & \frac{1}{x_2-x_1} & \frac{1}{x_3-x_1} & \cdots & \frac{1}{x_n-x_1} \\
			0 & 1 & 1 & \cdots & 1 \\
			0 & x_2& x_3 & \cdots & x_n \\
			\vdots & \vdots & \vdots & \ddots & \vdots \\
			0 & x_2^{n-3} & x_3^{n-3}& \cdots & x_n^{n-3}\\
			0 & x_2^{n-2} & x_3^{n-2} & \cdots & x_n^{n-2}
		\end{vmatrix} 
		=\prod_{i=2}^{n}(x_i-x_1)\prod_{2\leqslant i<j\leqslant n}(x_j-x_i) \\
		=&\prod_{1\leqslant i<j\leqslant n}^{}(x_j-x_i)\qedhere
	\end{align*}
\end{proof}


\subsection{矩阵的秩}
\begin{definition}
	设$K$是一个数域,$n$是给定的正整数,令:
	\begin{equation*}
		K^n=\{(\seq{a}{n}):a_i\in K,\;i=1,2,\dots,n\}
	\end{equation*}
	称$K^n$为\gls{NDimensionalVectorSpace},其中的元素为\gls{NDimensionalVector},将$(\seq{a}{n})$写成一行称为\gls{RowVector},写成一列称为\gls{ColumnVector}。若$n$维向量$(\seq{a}{n})$与$\seq{b}{n}$满足$a_1=b_1,\;a_2=b_2,\dots\;a_n=b_n$,则称二者相等。在$K^n$中定义如下运算:
	\begin{enumerate}
		\item \textbf{加法:} 
		\begin{equation*}
			(\seq{a}{n})+(\seq{b}{n})\coloneq(a_1+b_1,a_2+b_2,\dots,a_n+b_n)
		\end{equation*}
		\item \textbf{纯量乘法:}
		\begin{equation*}
			\forall\;k\in K,\; k(\seq{a}{n})=(\seq{ka}{n})
		\end{equation*}
	\end{enumerate}
	那么$M_{m\times n}(K)$构成一个线性空间\footnote{证明略去}。
\end{definition}
根据$n$维向量空间的定义,可以将矩阵$A\in M_{m\times n}(K)$的列向量组视为$K^m$中的元素,行向量组也可视为$K^n$中的元素。接下来我们来讨论矩阵的秩。
\begin{definition}
	矩阵$A$的列向量组的秩称为$A$的\textbf{列秩},行向量组的秩称为$A$的\textbf{行秩}。
\end{definition}
\begin{lemma}\label{lem:REFRankColumnRow}
	阶梯形矩阵$J$的行秩与等于列秩且都等于非零行数,$J$的主元所在的行构成行向量组的一个极大线性无关组,主元所在列构成列向量组的一个极大线性无关组。
\end{lemma}
\begin{proof}
	先证明后两句结论(这是显然的),即可得到第一句结论。
\end{proof}
\begin{lemma}\label{lem:ElementaryRowColumnTransRank}
	矩阵的初等行变换不改变行秩,初等列变换不改变列秩。
\end{lemma}
\begin{proof}
	证明三种变换前后的向量组是等价的,由\cref{prop:Rank}(3)即可得出结论。列变换的情况可由转置与行变换的结论得到。
\end{proof}
\begin{lemma}\label{lem:ElementaryRowSameColumn}
	矩阵的初等行变换不改变矩阵列向量组之间的线性相关性:
	\begin{enumerate}
		\item 设矩阵$A$经过初等行变换变成矩阵$B$,则$A$的列向量组线性相关当且仅当$B$的列向量组线性相关;
		\item 设矩阵$A$经过初等行变换变成矩阵$B$,若$B$的第$\seq{j}{r}$列构成$B$的列向量组的一个极大线性无关组,则$A$的第$\seq{j}{r}$列也构成$A$的列向量组的一个极大线性无关组\footnote{与\cref{lem:REFRankColumnRow}联合起来提供了求矩阵列向量组的极大线性无关组的方法。}。
	\end{enumerate}
\end{lemma}
\begin{proof}
	(1)将矩阵$A,B$看作齐次线性方程组的矩阵,因为$Ax=\mathbf{0}$和$Bx=\mathbf{0}$同解,于是$Ax=\mathbf{0}$有非零解当且仅当$Bx=\mathbf{0}$有非零解,即$A$的列向量组线性相关当且仅当$B$的列向量组线性相关。\par
	(2)$\;A$的第$\seq{j}{r}$列经过初等行变换构成$B$的第$\seq{j}{r}$列,由(1)可知它们线性无关。任取其它列第$l$列,则$A$的第$\seq{j}{r},l$列经过初等行变换构成$B$的第$\seq{j}{r},l$列,因为$B$的第$\seq{j}{r}$列构成$B$的列向量组的一个极大线性无关组,所以$B$的第$\seq{j}{r},l$列线性相关,由(1)可知$A$的第$\seq{j}{r},l$列也线性相关,所以$A$的第$\seq{j}{r},l$列构成$A$的一个极大线性无关组。
\end{proof}
\begin{property}\label{prop:MatrixRank}
	矩阵的秩具有如下性质:
	\begin{enumerate}
		\item 任意矩阵的行秩都等于列秩,所以将矩阵$A$的行秩和列秩统称为矩阵$A$的秩,记为$\operatorname{rank}(A)$;
		\item 矩阵的初等变换不改变矩阵的秩;
		\item 非零矩阵的秩等于它的不为$0$的子式的最高阶数;
		\item 矩阵$A$的不为$0$的$\operatorname{rank}(A)$阶子式所在的行(列)构成$A$的行(列)向量组的一个极大线性无关组;
		\item $\operatorname{rank}(A+B)\leqslant\operatorname{rank}(A)+\operatorname{rank}(B)$;
		\item 若$k\ne0$,则$\operatorname{rank}(kA)=\operatorname{rank}(A)$;
		\item 转置与Hermitian转置不改变矩阵的秩;
		\item 设$A\in M_{m\times n}(K)$,则有:
		\begin{equation*}
			\operatorname{rank}(AA^{\top})=\operatorname{rank}(A^{\top}A)=\operatorname{rank}(A)
		\end{equation*}
		若$K=\mathbb{C}$,则有:
		\begin{equation*}
			\operatorname{rank}(AA^H)=\operatorname{rank}(A^HA)=\operatorname{rank}(A)
		\end{equation*}
		\item $\operatorname{rank}(AB)\leqslant\min\{\operatorname{rank}(A),\operatorname{rank}(B)\}$;
		\item 设$A\in M_{m\times n}(K),\;B\in M_{s\times t}(K)$,则:
		\begin{equation*}
			\operatorname{rank}\left[
			´\begin{pmatrix}
				A & \mathbf{0} \\
				\mathbf{0} & B
			\end{pmatrix}
			\right]=\operatorname{rank}(A)+\operatorname{rank}(B)
		\end{equation*}
		\item 设$A\in M_{m\times n}(K),\;B\in M_{s\times t}(K),\;C\in M_{m\times t}(K)$,则:
		\begin{equation*}
			\operatorname{rank}\left[
			´\begin{pmatrix}
				A & C\\
				\mathbf{0} & B
			\end{pmatrix}
			\right]\geqslant\operatorname{rank}(A)+\operatorname{rank}(B)
		\end{equation*}
		当$A$和$B$都行满秩或列满秩时等号成立;
	\end{enumerate}
\end{property}
\begin{proof}
	(1)任取矩阵$A$,根据\cref{theo:RREFExistence},记$A$的阶梯形矩阵为$J$。由\cref{lem:ElementaryRowColumnTransRank}可知则$A$的行秩等于$J$的行秩,由\cref{lem:REFRankColumnRow}可知$J$的行秩等于$J$的列秩,由\cref{lem:ElementaryRowSameColumn}(2)可知$J$的列秩等于$A$的列秩,于是$A$的行秩等于$A$的列秩。由$A$的任意性,结论成立。\par
	(2)由\cref{lem:ElementaryRowColumnTransRank}和(1)立即得到。\par
	(3)设矩阵$A\in M_{m\times n}(K),\;\operatorname{rank}(A)=r$,由(1)可得$A$有$r$行、$r$列线性无关,将其对应的$r^2$个元素按原本的顺序排成的矩阵记为$A_1$,由(2)、\cref{lem:REFRankColumnRow}可知$|A_1|\ne0$,所以$A$存在一个$r$阶子式。\par
	设$s>r$且$s\leqslant\min\{m,n\}$,任取$A$的一个$s$阶子式:
	\begin{equation*}
		A\left\{ \begin{array}{l}
			i_1,i_2,\dots,i_s \\
			j_1,j_2,\ \dots,j_s
		\end{array}\right\}
	\end{equation*}
	因为$\operatorname{rank}(A)=r$,所以$A$的列向量组的极大线性无关组由$r$个向量组成,而$A$的第$\seq{j}{s}$列可以由$A$的列向量组的极大线性无关组表出,且$s>r$,根据\cref{prop:LinearlyDependent}(7)可得$A$的第$\seq{j}{s}$列线性相关。由\cref{prop:Determinant}(7)可知该$m$阶子式为$0$。\par
	综上,$A$的不为$0$的子式的最高阶数为$\operatorname{rank}(A)$。\par
	(4)由\cref{prop:Determinant}(7)可知该子式对应的矩阵的行(列)向量组线性无关,从而其延伸组也线性无关,即对应于$A$的行(列)线性无关。因为该向量组的向量个数等于$\operatorname{rank}(A)$,由\cref{prop:Rank}(5)可知它是$A$的行(列)向量组的一个极大线性无关组。\par
	(5)因为$A+B$的列向量组可由$A$的列向量组与$B$的列向量组线性表出,所以$A+B$的极大线性无关组可以由$A$的极大线性无关组与$B$的极大线性无关组线性表出,由\cref{prop:LinearlyDependent}(8)即可得出结论。\par
	(6)显然。\par
	(7)转置由(1)可得,Hermitian转置只需注意到若虚部线性无关则乘上$-1$也线性无关。\par
	(8)只需要证明Hermitian转置的情况,转置是Hermitian转置在实数域上的特例。\par
	由\cref{prop:HomogeneousSLESolution}(3)可知只需证明方程$A^HAx=\mathbf{0}$与$Ax=\mathbf{0}$同解。注意到$Ax=\mathbf{0}$则必然有$A^HAx=\mathbf{0}$,而若$A^HAx=\mathbf{0}$,则必有$x^HA^HAx=||Ax||=0$,所以$Ax=\mathbf{0}$。于是:
	\begin{equation*}
		n-\operatorname{rank}(A^HA)=n-\operatorname{rank}(A)
	\end{equation*}
	所以:
	\begin{equation*}
		\operatorname{rank}(A^HA)=\operatorname{rank}(A)
	\end{equation*}
	同理由(7)可得:
	\begin{equation*}
		\operatorname{rank}(AA^H)=\operatorname{rank}(A^H)=\operatorname{rank}(A)
	\end{equation*}
	于是有:
	\begin{equation*}
		\operatorname{rank}(AA^H)=\operatorname{rank}(A^HA)=\operatorname{rank}(A)
	\end{equation*}\par
	(9)由\cref{prop:MatrixMultiplication}(1)的第二种理解方式,$AB$的列向量组可以由$A$的列向量组线性表出,由\cref{prop:Rank}(2)可得$\operatorname{rank}(AB)\leqslant\operatorname{rank}(A)$。由(7)同理可得$\operatorname{rank}(AB)=\operatorname{rank}(B^{\top}A^{\top})\leqslant\operatorname{rank}(B^{\top})=\operatorname{rank}(B)$,所以结论成立。\par
	(10)根据\cref{theo:RREFExistence},将矩阵$\begin{pmatrix}
		A & \mathbf{0} \\
		\mathbf{0} & B
	\end{pmatrix}$通过初等变换化作阶梯形矩阵,由(2)和\cref{lem:REFRankColumnRow}即可得到结论。\par
	(11)根据(3)可知$A$有一个$\operatorname{rank}(A)$阶非零子式,$B$有一个$\operatorname{rank}(B)$阶非零子式,将它们分别记为$A_1,B_1$。由\cref{prop:Determinant}(10)可知$\begin{pmatrix}
		A & C \\
		\mathbf{0} & B
	\end{pmatrix}$有一个$\operatorname{rank}(A)+\operatorname{rank}(B)$阶子式:
	\begin{equation*}
		\begin{vmatrix}
			A_1 & C_1 \\
			\mathbf{0} & B_1
		\end{vmatrix}=|A_1||B_1|\ne0
	\end{equation*}
	于是由(3)可得:
	\begin{equation*}
		\operatorname{rank}\left[
		\begin{pmatrix}
			A & C \\
			\mathbf{0} & B
		\end{pmatrix}
		\right]\geqslant\operatorname{rank}(A)+\operatorname{rank}(B)
	\end{equation*}
	类似(10)可得到当$A$和$B$都行满秩或列满秩时等号成立。
\end{proof}

\subsection{矩阵的逆}
\begin{definition}
	对于数域$K$上的矩阵$A$,如果存在数域$K$上的矩阵$B$使得:
	\begin{equation*}
		AB=BA=I
	\end{equation*}
	则称$A$是\gls{InvertibleMatrix}或\gls{NonSingularMatrix},$B$是$A$的逆矩阵,记为$A^{-1}$。不存在逆矩阵的方阵被称为\gls{SingularMatrix}。
\end{definition}
\begin{definition}
	设$A=(a_{ij})\in M_{n}(K)$,称:
	\begin{equation*}
		A^{\star}=
		\begin{pmatrix}
			A_{11} & A_{21} & \cdots & A_{n1} \\
			A_{12} & A_{22} & \cdots & A_{n2} \\
			\vdots & \vdots & \ddots &\vdots \\
			A_{1n} & A_{n2} & \cdots & A_{nn} 
		\end{pmatrix}
	\end{equation*}
	为$A$的\gls{AdjointMatrix},其中$A_{ij}$为$a_{ij}$的代数余子式。
\end{definition}
\begin{property}\label{prop:AdjointMatrix}
	设$A=(a_{ij})\in M_{n}(K)$,则$AA^{\star}=|A|I_n$。
\end{property}
\begin{proof}
	由\cref{prop:Determinant}(8)(9)立即可得。
\end{proof}
\begin{property}\label{prop:InvertibleMatrix}
	关于矩阵的逆有如下结论:
	\begin{enumerate}
		\item 若矩阵$A$可逆,则逆矩阵是唯一的;
		\item 可逆矩阵是方阵;
		\item 设$A\in M_{n}(K)$,则$A$可逆的充要条件为:
		\begin{enumerate}
			\item $\det A\ne0$;
			\item $\operatorname{rank}(A)=n$;
			\item $A$的行(列)向量组线性无关;
			\item $A$的行(列)向量组为$K^n$的一个基;
		\end{enumerate}
		若$A$可逆,有:
		\begin{equation*}
			A^{-1}=\frac{1}{|A|}A^{\star}
		\end{equation*}
		\item 可逆矩阵经过初等行变换化成的简化行阶梯形矩阵一定是单位矩阵;
		\item 设$A$是一个可逆矩阵,则线性方程组$Ax=\mathbf{0}$只有零解;
		\item $\det A^{-1}=(\det A)^{-1}$;
		\item 若$A,B\in M_{n}(K)$且$AB=I_n$,则$A,B$都可逆,且$A^{-1}=B,\;B^{-1}=A$;
		\item 单位矩阵可逆;
		\item 初等矩阵可逆且逆矩阵仍为同类初等矩阵;
		\item 若$A$可逆,则$A^{-1}$可逆且$(A^{-1})^{-1}=A$;
		\item 若$A,B\in M_{n}(K)$都可逆,则$AB$也可逆,且$(AB)^{-1}=B^{-1}A^{-1}$;
		\item 若$A$可逆,则$A^{\top}$可逆且$(A^{\top})^{-1}=(A^{-1})^{\top}$,$A^H$也可逆且$(A^H)^{-1}=(A^{-1})^H$;
		\item 若对称矩阵$A$可逆,则$A^{-1}$仍是对称矩阵;若Hermitian矩阵$A$可逆,则$A^{-1}$仍是Hermitian矩阵;
		\item 设$A\in M_{n}(K)$,则$A$可逆的充要条件为它可以表示为一些初等矩阵的乘积;
		\item 求解逆矩阵的\textbf{初等变换法}:$(A, I)\overrightarrow{初等行变换}(I,A^{-1})$;
		\item 设$A\in M_{n}(K)$可逆,$B\in M_{m\times n}(K)$,则$\operatorname{rank}(B)=\operatorname{rank}(BA)=\operatorname{rank}(AB^{\top})$,即用一个可逆矩阵左(右)乘一个矩阵不会改变该矩阵的秩;
		\item 设$A=
		\begin{pmatrix}
			A_{11} & A_{12} \\
			A_{21} & A_{22}
		\end{pmatrix}$可逆。若$A_{11}$可逆,则:
		\begin{equation*}
			A^{-1}=
			\begin{pmatrix}
				A_{11}^{-1}+A_{11}^{-1}A_{12}(A_{22}-A_{21}A_{11}^{-1}A_{12})^{-1}A_{21}A_{11}^{-1} & -A_{11}^{-1}A_{12}(A_{22}-A_{21}A_{11}^{-1}A_{12})^{-1} \\
				-(A_{22}-A_{21}A_{11}^{-1}A_{12})^{-1}A_{21}A_{11}^{-1} & (A_{22}-A_{21}A_{11}^{-1}A_{12})^{-1}
			\end{pmatrix}
		\end{equation*}
		若$A_{22}$可逆,则:
		\begin{equation*}
			A^{-1}=
			\begin{pmatrix}
				(A_{11}-A_{12}A_{22}^{-1}A_{21})^{-1} & -(A_{11}-A_{12}A_{22}^{-1}A_{21})^{-1}A_{12}A_{22}^{-1} \\
				-A_{22}^{-1}A_{21}(A_{11}-A_{12}A_{22}^{-1}A_{21})^{-1} & A_{22}^{-1}+A_{22}^{-1}A_{21}(A_{11}-A_{12}A_{22}^{-1}A_{21})^{-1}A_{12}A_{22}^{-1}
			\end{pmatrix}
		\end{equation*}
		\item 若下(上)三角矩阵$L$可逆,则$L^{-1}$仍为下(上)三角矩阵。特别的,单位下(上)三角矩阵的逆矩阵仍为单位下(上)三角矩阵,且有:
		\begin{gather*}
			L=
			\begin{pmatrix}
				1 & & & &  \\
				l_{21} & 1 & & &  \\
				l_{31} & l_{32} & 1 & &  \\
				\vdots & \vdots & \vdots & \ddots &  \\
				l_{n1} & l_{n2} & l_{n3} & \cdots & 1
			\end{pmatrix},\quad
			L^{-1}=
			\begin{pmatrix}
				1 & & & &  \\
				-l_{21} & 1 & & &  \\
				-l_{31} & -l_{32} & 1 & &  \\
				\vdots & \vdots & \vdots & \ddots &  \\
				-l_{n1} & -l_{n2} & -l_{n3} & \cdots & 1 \\
			\end{pmatrix} \\
			U=
			\begin{pmatrix}
				1 & u_{12} & u_{13} & \cdots & u_{1n} \\
				& 1 & u_{23} & \cdots & u_{2n} \\
				&       & 1 & \cdots & u_{3n} \\
				&  &  & \ddots & \vdots \\
				&       &       &  & 1
			\end{pmatrix},\quad
			U^{-1}=
			\begin{pmatrix}
				1 & -u_{12} & -u_{13} & \cdots & -u_{1n} \\
				& 1 & -u_{23} & \cdots & -u_{2n} \\
				&       & 1 & \cdots & -u_{3n} \\
				&  &  & \ddots & \vdots \\
				&       &       &  & 1
			\end{pmatrix}
		\end{gather*}
	\end{enumerate}
\end{property}
\begin{proof}
	(1)设$A$有逆矩阵$B_1,B_2$且$B_1\ne B_2$,由\cref{prop:MatrixMultiplication}(5)(3)可得:
	\begin{equation*}
		B_1=B_1I=B_1AB_2=(B_1A)B_2=IB_2=B_2
	\end{equation*}
	矛盾。\par
	(2)由可逆矩阵的定义即可得到。\par
	(3)\textbf{ a:}当$A$可逆时有$AA^{-1}=I_n$,所以由\cref{prop:Determinant}(11)可得$|AA^{-1}|=|A||A^{-1}|=1$,于是$\det A\ne0$。当$\det A\ne0$时,由\cref{prop:Determinant}(8)(9)可知$AA^{\star}=A^{\star}A=|A|E$,所以$A^{-1}=|A|^{-1}A^{\star}$,$A$可逆。\par
	\textbf{b:}由(a)和\cref{prop:MatrixRank}(3)立即可得。\par
	\textbf{c:}由(b)和矩阵秩的定义立即可得。\par
	\textbf{d:}由(c)、\cref{prop:nDimensionalLinearSpace}和$\dim(K^n)=n$立即可得。\par
	(4)由(3.b)、\cref{lem:REFRankColumnRow}和\cref{prop:MatrixRank}(2)立即可得。\par
	(5)由(3.b)立即可得。\par
	(6)当$A$可逆时有$AA^{-1}=I_n$,所以由\cref{prop:Determinant}(11)可得$|AA^{-1}|=|A||A^{-1}|=1$,即$\det A^{-1}=(\det A)^{-1}$。\par
	(7)由\cref{prop:Determinant}(11)可得$|AB|=|A||B|=|I_n|=1$,所以$|A|,|B|\ne0$,由(3.a)可知$A,B$都可逆。注意到$A^{-1}AB=B=A^{-1}$,同理可得$A=B^{-1}$。\par
	(8)显然。\par
	(9)只需取相反的初等矩阵即可。\par
	(10)由可逆矩阵的定义立即可得。\par
	(11)代入定义验证即可得到。\par
	(12)由\cref{prop:Transpose}(4)可得$(AA^{-1})^{\top}=(A^{-1})^{\top}A^{\top}=I,\;(AA^{-1})^H=(A^{-1})^HA^H=I$,根据(7)结论成立。\par
	(13)对称矩阵和Hermitian矩阵的结论由(12)立即可得。\par
	(14)由(4)(7)(10)(9)(11)可得必要性,由(9)(7)可得充分性。\par
	(15)由(4)可知存在一系列单位矩阵$\seq{P}{n}$使得$P_nP_{n-1}\cdots P_1A=I$,根据(7)可得$P_nP_{n-1}\cdots P_1=A^{-1}$,即$P_nP_{n-1}\cdots P_1I=A^{-1}$,由\cref{prop:ElementaryMatrix}(2)可得即可得出结论。\par
	(16)由(14)和\cref{prop:MatrixRank}(2)立即可得。\par
	(17)若$A_{11}$可逆,则:
	\begin{equation*}
		\begin{pmatrix}
			I & \mathbf{0} \\
			-A_{21}A_{11}^{-1} & I
		\end{pmatrix}
		\begin{pmatrix}
			A_{11} & A_{12} \\
			A_{21} & A_{22}
		\end{pmatrix}
		\begin{pmatrix}
			I & -A_{11}^{-1}A_{12} \\
			\mathbf{0} & I
		\end{pmatrix}=
		\begin{pmatrix}
			A_{11} & \mathbf{0} \\
			\mathbf{0} & A_{22}-A_{21}A_{11}^{-1}A_{12}
		\end{pmatrix}
	\end{equation*}
	由\cref{prop:Determinant}(10)(11)和(3.a)可得$A_{22}-A_{21}A_{11}^{-1}A_{12}$和上式左边除了$A$以外的两个矩阵可逆。由(11)即可得到:
	\begin{gather*}
		\begin{pmatrix}
			I & -A_{11}^{-1}A_{12} \\
			\mathbf{0} & I
		\end{pmatrix}^{-1}A^{-1}
		\begin{pmatrix}
			I & \mathbf{0} \\
			-A_{21}A_{11}^{-1} & I
		\end{pmatrix}^{-1}=
		\begin{pmatrix}
			A_{11}^{-1} & \mathbf{0} \\
			\mathbf{0} & (A_{22}-A_{21}A_{11}^{-1}A_{12})^{-1}
		\end{pmatrix} \\
		A^{-1}=
		\begin{pmatrix}
			I & -A_{11}^{-1}A_{12} \\
			\mathbf{0} & I
		\end{pmatrix}
		\begin{pmatrix}
			A_{11}^{-1} & \mathbf{0} \\
			\mathbf{0} & (A_{22}-A_{21}A_{11}^{-1}A_{12})^{-1}
		\end{pmatrix}
		\begin{pmatrix}
			I & \mathbf{0} \\
			-A_{21}A_{11}^{-1} & I
		\end{pmatrix} \\
		A^{-1}=
		\begin{pmatrix}
			A_{11}^{-1}+A_{11}^{-1}A_{12}(A_{22}-A_{21}A_{11}^{-1}A_{12})^{-1}A_{21}A_{11}^{-1} & -A_{11}^{-1}A_{12}(A_{22}-A_{21}A_{11}^{-1}A_{12})^{-1} \\
			-(A_{22}-A_{21}A_{11}^{-1}A_{12})^{-1}A_{21}A_{11}^{-1} & (A_{22}-A_{21}A_{11}^{-1}A_{12})^{-1}
		\end{pmatrix}
	\end{gather*}
	$A_{22}$可逆的情况类似可得。\par
	(18)设下三角矩阵$L\in M_{n}(K)$可逆,由(3.a)和\cref{prop:Determinant}(12)可知$L$的对角线元素均不为$0$。对$L$的阶数作归纳假设。当$n=1$时结论显然成立,假设矩阵阶数为$n-1$时结论成立,下面证明矩阵阶数为$n$时结论也成立。将$L$分块为:
	\begin{equation*}
		L=
		\begin{pmatrix}
			A & b \\
			\mathbf{0} & \alpha
		\end{pmatrix}
	\end{equation*}
	其中$A$为$n-1$阶上三角矩阵矩阵,由(3.a)和\cref{prop:Determinant}(12)可知$A$可逆。根据(17)可得:
	\begin{equation*}
		L^{-1}=
		\begin{pmatrix}
			A^{-1} & -A^{-1}b\alpha^{-1} \\
			\mathbf{0} & \alpha^{-1}
		\end{pmatrix}
	\end{equation*}
	由归纳假设可知$A^{-1}$是下三角矩阵,于是$L^{-1}$是下三角矩阵。类似可得上三角矩阵时的情况。
\end{proof}

\subsection{正交矩阵与Euclid空间}
\begin{definition}
	在$\mathbb{R}^{n}$中,对任意的$\alpha=(\seq{a}{n}),\;\beta=(\seq{b}{n})$定义:
	\begin{equation*}
		(\alpha,\beta)\coloneq\sum_{i=1}^{n}a_ib_i
	\end{equation*}
	则$(\alpha,\beta)$是一个内积,称之为标准内积。将有了标准内积后的$\mathbb{R}^{n}$称为\gls{EuclidSpace}。
\end{definition}
\begin{proof}
	需要证明上述定义满足内积的定义,由于过于简单所以略去。
\end{proof}
\begin{definition}
	在Euclid空间$\mathbb{R}^{n}$中,定义向量$\alpha$的长度$|\alpha|$为:
	\begin{equation*}
		|\alpha|=\coloneq\sqrt{(\alpha,\alpha)}
	\end{equation*}
	长度为$1$的向量被称为\gls{UnitVector}。把非零向量$\alpha$乘以$\frac{1}{|\alpha|}$的操作称为将$\alpha$\gls{Normalization}。
\end{definition}
\begin{definition}
	若$A\in M_{n}(\mathbb{R}^{})$满足$A^{\top}A=I_n$,则称$A$是\gls{OrthogonalMatrix}。若$A\in M_{n}(\mathbb{C}^{})$满足$A^HA=I_n$,则称$A$是\gls{UnitaryMatrix}。
\end{definition}
\begin{property}\label{prop:OrthogonalUnitaryMatrix}
	正交矩阵与酉矩阵具有如下性质:
	\begin{enumerate}
		\item 若$A$是一个正交(酉)矩阵,则$A$可逆且$A^{\top}=A^{-1}$($A^H=A^{-1}$);
		\item 若$A$是一个正交(酉)矩阵,则$\det A=\pm1$($|\det A|=1$);
		\item 单位矩阵是正交(酉)矩阵;
		\item 若$A,B$是正交(酉)矩阵,则$AB$也是正交(酉)矩阵;
		\item 若$A$是正交(酉)矩阵,则$A^{-1}$也是正交(酉)矩阵;
		\item 若$A$是正交(酉)矩阵且为上(下)三角矩阵,则$A$是对角矩阵且主对角线元素模为$1$;
	\end{enumerate}
\end{property}
\begin{proof}
	(1)由定义立即可得。\par
	(2)由\cref{prop:Determinant}(2)(11)可得:
	\begin{equation*}
		|AA^{\top}|=|A|^2=1,\quad|AA^H|=|A||A^H|=|A|\overline{|A|}=1
	\end{equation*}\par
	(3)显然。\par
	(4)根据\cref{prop:Transpose}(4)代入验证即可。\par
	(5)由\cref{prop:Transpose}(2)和正交矩阵、酉矩阵的定义即可得到。\par
	(6)设$A\in M_{n}(K)$为上三角矩阵且$A$是酉矩阵。当$n=1$时,由$A^HA=|a_{11}|^2=1$可知$|a_{11}|=1$,命题成立。假设当矩阵阶数为$n$时结论成立,即若$A\in M_{n-1}(K)$为上三角且$A^HA=I_{n-1}$,则$A$为对角矩阵且$|a_{ii}|=1$,下证明矩阵阶数为$n$时结论也成立。\par
	由于$A$是上三角矩阵,其第$n$行只有一个可能非零的元素$a_{nn}$,因此:
	\begin{equation*}
		AA^H(n;n)=\sum_{j=1}^{n} |a_{nj}|^2=|a_{nn}|^2=1
	\end{equation*}
	从而$|a_{nn}|=1$。\par
	当$n>k$时有:
	\begin{equation*}
		0=AA^H(n;k)=\sum_{j=1}^{n}a_{nj}\overline{a_{kj}}=a_{nn}\overline{a_{kn}}
	\end{equation*}
	由于$|a_{nn}|=1$,所以$a_{kn}=0,\;\forall k<n$。因此,第$n$列除$a_{nn}$外全为零,所以:
	\begin{equation*}
		A= 
		\begin{pmatrix}
			B & 0 \\
			0 & a_{nn}
		\end{pmatrix}
	\end{equation*}
	其中$B\in M_{n-1}(K)$为上三角矩阵。因为$A$是酉矩阵,所以$B$也是酉矩阵,由归纳假设可知$B$是对角矩阵且主对角线模为$1$,从而$A$也是对角矩阵且主对角线模为$1$。\par
	当$A$是下三角矩阵时可知$A^{\top}$是对角矩阵且主对角线元素模为$1$,也即$A$是对角矩阵且主对角线元素模为$1$。
\end{proof}
\section{矩阵的向量空间}

\begin{definition}
	设$A=(\seq{\alpha}{n})\in M_{m\times n}(K)$,将:
	\begin{equation*}
		\left\{\sum_{i=1}^{n}k_i\alpha_i:k_i\in K\right\}\overset{def}{=}\mathcal{M}(A)
	\end{equation*}
\end{definition}

\begin{theorem}\label{theo:VectorSpaceAAAT}
	设$A\in M_{m\times n}(K)$,则:
	\begin{equation*}
		\mathcal{M}(A)=\mathcal{M}(AA^T)
	\end{equation*}
\end{theorem}
\begin{proof}
	由定义,显然$\mathcal{M}(AA^T)\subset\mathcal{M}(A)$。对于任意的$x\perp\mathcal{M}(AA^T)$,有$x^TAA^T=\mathbf{0}$,于是$||A^Tx||^2=x^TAA^Tx=0$,即$A^Tx=\mathbf{0}$,于是$x\perp\mathcal{M}(A)$。\info{回头改证明,同时注意数域问题}
\end{proof}
\section{线性方程组}
\begin{definition}
	设 $x_1, x_2, \dots, x_n$ 为 $n$ 个未知数,若一个方程具有如下形式:
	\[
	a_1 x_1 + a_2 x_2 + \dots + a_n x_n = b
	\]
	其中,$a_1, a_2, \dots, a_n$ 为系数,$b$为常数项,则称该方程为\gls{LinearEquation}。
	由$m$个形如上式的方程组成的方程组:
	\[
	\begin{cases}
		a_{11} x_1 + a_{12} x_2 + \dots + a_{1n} x_n = b_1 \\
		a_{21} x_1 + a_{22} x_2 + \dots + a_{2n} x_n = b_2 \\
		\quad \vdots \\
		a_{m1} x_1 + a_{m2} x_2 + \dots + a_{mn} x_n = b_m
	\end{cases}
	\]
	被称为$n$元\gls{SLE}。由矩阵乘法的定义,该方程组也可以写作矩阵形式:
	\[
	Ax=b
	\]
	其中:
	\[
	A =
	\begin{pmatrix}
		a_{11} & a_{12} & \dots & a_{1n} \\
		a_{21} & a_{22} & \dots & a_{2n} \\
		\vdots & \vdots & \ddots & \vdots \\
		a_{m1} & a_{m2} & \dots & a_{mn}
	\end{pmatrix}, \quad
	x =
	\begin{pmatrix}
		x_1 \\ x_2 \\ \vdots \\ x_n
	\end{pmatrix}, \quad
	b =
	\begin{pmatrix}
		b_1 \\ b_2 \\ \vdots \\ b_m
	\end{pmatrix}
	\]
\end{definition}
\begin{definition}
	给定线性方程组$Ax=b$,称如下矩阵:
	\[
	\begin{pmatrix}
		a_{11} & a_{12} & \dots & a_{1n} & b_1 \\
		a_{21} & a_{22} & \dots & a_{2n} & b_2 \\
		\vdots & \vdots & \ddots & \vdots & \vdots \\
		a_{m1} & a_{m2} & \dots & a_{mn} & b_m
	\end{pmatrix}.
	\]
	为该线性方程组的\gls{AugmentedMatrix},记为$[A|b]$。
\end{definition}
\begin{definition}
	设增广矩阵化简后变为阶梯形矩阵,称每一行主元所在列所对应的未知数为\gls{PivotVariable},同时称非主元所在列对应的未知数为\gls{FreeVariable}。
\end{definition}
\begin{theorem}\label{theo:SolutionOfSLE1}	
	数域$K$上的$n$元线性方程组的解的情况只有三种可能:
	\begin{enumerate}
		\item \textbf{无解:}增广矩阵化成的阶梯形方程出现$0=d$且$d\ne0$;
		\item 有解:
		\begin{enumerate}
			\item \textbf{唯一解:}阶梯形矩阵的非零行数$r$等于未知量个数$n$;
			\item \textbf{无穷多解:}阶梯形矩阵的非零行数$r$小于未知量个数$n$;
		\end{enumerate}
	\end{enumerate}
	这导致:
	\begin{enumerate}
		\item 数域$K$上$n$元齐次线性方程组有非零解的充分必要条件为:系数矩阵经过初等行变换化成的阶梯形矩阵中非零行数$r<n$;
		\item 数域$K$上$n$元齐次线性方程组的方程数$m$若小于未知量数$n$,则一定有非零解。
	\end{enumerate}
\end{theorem}
\begin{proof}
	证明较为简单,略去
\end{proof}
\begin{theorem}\label{theo:SolutionOfSLE2}
	数域$K$上$n$元线性方程组$Ax=b$(即$\sum\limits_{i=1}^{n}\alpha_ix_i=b$,其中$\alpha_i$为$A$的列向量)有解的充分必要条件为:
	\begin{enumerate}
		\item $b\in<\seq{\alpha}{n}>$;
		\item $\operatorname{rank}(A)=\operatorname{rank}([A|b])$;
	\end{enumerate}
	进一步可得唯一解与无穷多解的判别方法:
	\begin{enumerate}
		\item \textbf{唯一解:}$\operatorname{rank}(A)=n$;
		\item \textbf{无穷多解:}$\operatorname{rank}(A)<n$。
	\end{enumerate}
	这导致齐次线性方程组有非零解的充分必要条件为$\operatorname{rank}(A)<n$。
\end{theorem}
\begin{proof}
	(1)显然。\par
	(2)由\cref{prop:SpanSubspace}(5)(3)可得$Ax=b$有解$\iff b\in<\seq{\alpha}{n}>\iff<\seq{\alpha}{n},b>=<\seq{\alpha}{n}>\iff\dim<\seq{\alpha}{n},b>=\dim<\seq{\alpha}{n}>\iff\operatorname{rank}(A)=\operatorname{rank}([A|b])$。\par
	(3)若$\operatorname{rank}(A)=n$,由\cref{lem:ElementaryRowColumnTransRank}和\cref{lem:REFRankColumnRow}可知阶梯形矩阵的非零行数$r=n$,由\cref{theo:SolutionOfSLE1}可得此时有唯一解。\par
	(4)与(3)类似。
\end{proof}
\subsubsection{齐次线性方程组解的解构}
\begin{property}\label{prop:HomogeneousSLESolution}
	数域$K$上$n$元齐次线性方程组$Ax=\mathbf{0}$的解具有如下性质:
	\begin{enumerate}
		\item 若$\alpha,\beta$是解,对任意的$c_1,c_2\in K$,$k_1\alpha+k_2\beta$也是解;
		\item 解空间$W$构成$K^n$的一个子空间;
		\item 解空间$W$满足$\dim(W)=n-\operatorname{rank}(A)$。
	\end{enumerate}
\end{property}
\begin{proof}
	(1)$A(k_1\alpha+k_2\beta)=k_1A\alpha+k_2A\beta=\mathbf{0}$。\par
	(2)由(1)立即可得。\par
	(3)设$A$的列向量组为$\seq{\alpha}{n}$,$A$的行数为$m$。定义线性映射$\mathcal{T}:\alpha\longrightarrow A\alpha$,则$\mathcal{T}$是$K^n$到$K^{m}$的一个线性映射。于是有:
	\begin{gather*}
		\operatorname{Ker}(\mathcal{T})=\{\alpha\in K^n:\mathcal{T}\alpha=\mathbf{0}\}=\{\alpha\in K^n:A\alpha=\mathbf{0}\}=W \\
		\operatorname{Im}(\mathcal{T})=\{A\alpha:\alpha\in K^n\}=<\seq{\alpha}{n}>
	\end{gather*}
	所以由\cref{prop:SpanSubspace}(3)可得:
	\begin{gather*}
		\dim(\operatorname{Ker}\mathcal{T})=\dim(W) \\
		\operatorname{rank}(A)=\operatorname{rank}\{\seq{\alpha}{n}\}=\dim<\seq{\alpha}{n}>=\dim(\operatorname{Im}\mathcal{T})
	\end{gather*}
	由\cref{prop:LinearMapping}(10)即可得到:
	\begin{equation*}
		\dim(K^n)=\dim(\operatorname{Ker}\mathcal{T})+\dim(\operatorname{Im}\mathcal{T})=\dim(W)+\operatorname{rank}(A)
	\end{equation*}
	即$n=\dim(W)+\operatorname{rank}(A)$。
\end{proof}
\begin{definition}
	设数域$K$上$n$元齐次线性方程组$Ax=\mathbf{0}$有非零解,称它的解空间$W$的一组基为\gls{FundamentalSolutionSet}。
\end{definition}
\subsubsection{非齐次线性方程组解的结构}
\begin{property}\label{prop:InhomogeneousSLESolution}
	数域$K$上$n$数域$K$上$n$元非齐次线性方程组$Ax=b$的解具有如下性质:
	\begin{enumerate}
		\item 若$\alpha,\beta$是解,则$\alpha-\beta$为$Ax=\mathbf{0}$的解;
		\item 设$W$为$Ax=\mathbf{0}$的解空间,若$\alpha$是$Ax=b$的解,则对任意的$\beta\in W$,$\alpha+\beta$也是$Ax=b$的解;
		\item 设$W$为$Ax=\mathbf{0}$的解空间,则$Ax=b$的解集$U$可以表示为:
		\begin{equation*}
			U=\{\alpha+\beta:\beta\in W\}
		\end{equation*}
		其中$\alpha$为$Ax=b$的任意一个解;
		\item $Ax=b$的解唯一当且仅当$Ax=\mathbf{0}$的解空间为零空间。
	\end{enumerate}
\end{property}
\begin{proof}
	(1)$A(\alpha-\beta)=A\alpha-A\beta=b-b=\mathbf{0}$。\par
	(2)$A(\alpha+\beta)=A\alpha+A\beta=b+\mathbf{0}=b$。\par
	(3)由(1)(2)可得。\par
	(4)由(3)立即可得。
\end{proof}
\begin{algorithm}
	\caption{Gaussian Elimination}
	\label{alg:gauss}
	\begin{algorithmic}[1]
		\Require Augmented matrix $[A|\mathbf{b}] \in \mathbb{R}^{m \times (n+1)}$
		\Ensure Solution status and expression
		\State \textbf{Step 1: Forward Elimination}
		\For{$j = 1$ to $n$}
		\State Find $p$ such that $|a_{pj}|$ is maximum for $j \leq p \leq m$ \Comment{Pivoting}
		\If{$|a_{pj}| < \varepsilon$}
		\State \textbf{continue} \Comment{Skip zero column}
		\EndIf
		\State Swap row $p$ and row $j$
		\For{$i = j+1$ to $m$}
		\For{$k = j$ to $n+1$}
		\State $a_{ik} \gets a_{ik} - a_{ij} / a_{jj} \cdot a_{jk}$
		\EndFor
		\State $a_{ij} \gets 0$ \Comment{Explicitly zero to avoid error}
		\EndFor
		\EndFor
		
		\State \textbf{Step 2: Inconsistency Check}
		\For{$i = 1$ to $m$}
		\If{All $a_{ij} = 0$ for $j = 1$ to $n$ \textbf{and} $a_{i(n+1)} \neq 0$}
		\State \Return ``No solution (inconsistent row)''
		\EndIf
		\EndFor
		
		\State \textbf{Step 3: Identify Pivot and Free Variables}
		\State $\mathcal{P} \gets$ set of index of pivot columns in row echelon form, $\mathcal{F} \gets \{1,\dots,n\}\backslash\mathcal{P}$, $r \gets |\mathcal{P}|$
		\If{$r = n$}
		\State \textbf{Back substitution: unique solution}
		\State Initialize $\mathbf{x} \gets (0,\dots,0)^{\top}$
		\For{$i = n$ downto $1$}
		\State $x_i \gets \left(a_{i(n+1)} - \sum\limits_{k=i+1}^{n} a_{ik} x_k\right) / a_{ii}$
		\EndFor
		\State \Return $\mathbf{x} = (x_1,\dots,x_n)^{\top}$
		\EndIf
	\end{algorithmic}
\end{algorithm}
\begin{algorithm}
	\caption{Gaussian Elimination (Part 2): General Solution via RREF}
	\begin{algorithmic}[1]
		\Require Row echelon form (REF) matrix $[A|\mathbf{b}]$ with $r < n$
		\Ensure General solution $\mathbf{x} = \mathbf{x}_p + \sum t_j \mathbf{v}_j$
		
		\State \textbf{Construct general solution: infinite solutions}
		\State \textbf{Step 4: Transform to Reduced Row Echelon Form (RREF)}
		\For{$j = r$ downto $1$}
		\State Let $i$ be the row where pivot in column $\mathcal{P}[j]$ appears
		\State Divide entire row $i$ by $a_{i\mathcal{P}[j]}$ to make pivot = 1
		\For{$k = 1$ to $i-1$}
		\For{$l = \mathcal{P}[j]$ to $n+1$}
		\State $a_{kl} \gets a_{kl} - a_{k\mathcal{P}[j]} \cdot a_{il}$
		\EndFor
		\EndFor
		\EndFor
		
		\State \textbf{Step 5: Compute Particular Solution $\mathbf{x}_p$}
		\State Initialize $\mathbf{x}_p = (0, 0, \dots, 0)^{\top}$
		\For{$j = 1$ to $r$}
		\State Let $i$ be the row where pivot in column $\mathcal{P}[j]$ appears
		\State $\mathbf{x}_{p\mathcal{P}[j]}\gets a_{i(n+1)}$
		\EndFor
		
		\State \textbf{Step 6: Compute Basis Vectors $\{\mathbf{v}_j\}$}
		\For{$j = 1$ to $n-r$}
		\State Initialize $\mathbf{v}_j = (0, 0, \dots, 0)^{\top}$
		\State $\mathbf{v}_{j\mathcal{F}[j]}\gets1$
		\State Let $i$ be the first row such that the pivot column index is greater than $\mathcal{F}[j]$
		\For{$k=1$ to $i-1$}
		\State $\mathbf{v}_{jk}\gets-a_{k\mathcal{F}[j]}$
		\EndFor
		\State Store $\mathbf{v}_j$
		\EndFor
		
		\State \Return General solution: $\mathbf{x} = \mathbf{x}_p + \sum\limits_{j=1}^{n-r} c_j \mathbf{v}_j$, where $c_j\in\mathbb{R}^{}$
	\end{algorithmic}
\end{algorithm}

\section{矩阵的等价关系}

\subsection{相抵}
\begin{definition}
	$A,B\in M_{s\times m}(K)$,如果满足下述条件中的任意一个:
	\begin{enumerate}
		\item $A$能够通过初等行变换和初等列变换变成$B$;
		\item 存在数域$K$上的$s$阶初等矩阵$P_1,P_2,\dots,P_t$与$m$阶初等矩阵$Q_1,Q_2,\dots,Q_n$使得:
		\begin{equation*}
			P_t\cdots P_2P_1AQ_1Q_2\cdots Q_n=B
		\end{equation*}
		\item 存在数域$K$上的$s$阶可逆矩阵$P$与$m$阶可逆矩阵$Q$使得:
		\begin{equation*}
			PAQ=B
		\end{equation*}
	\end{enumerate}
	则称$A$与$B$\gls{Equivalent}。
\end{definition}
上述三个条件显然是等价的。
\begin{theorem}
	相抵是$M_{s\times m}(K)$上的一个等价关系。在相抵关系下,矩阵$A$的等价类称为$A$的\textbf{相抵类}。
\end{theorem}
\begin{proof}
	证明是显然的。
\end{proof}
\begin{theorem}
	设$A\in M_{s\times m}(K)$,且$\operatorname{rank}(A)=r$。如果$r>0$,那么$A$相抵于如下形式的矩阵:
	\begin{equation*}
		\begin{pmatrix}
			I_r & \mathbf{0} \\
			\mathbf{0} & \mathbf{0}
		\end{pmatrix}
	\end{equation*}
	称该矩阵为$A$的\textbf{相抵标准形}。如果$r=0$,则$A$相抵于零矩阵,此时称零矩阵为$A$的\textbf{相抵标准形}。
\end{theorem}
\begin{proof}
	一个矩阵通过初等行变换一定可以变成一个简化行阶梯型矩阵,再由初等列变换即可得到上述矩阵。
\end{proof}
\begin{theorem}[相抵的完全不变量]
	$A,B\in M_{s\times m}(K)$,$A$与$B$相抵当且仅当它们的秩相同。
\end{theorem}
\begin{proof}
	\textbf{(1)必要性:}初等行变换和初等列变换不改变矩阵的秩。\par
	\textbf{(2)充分性:}若$A,B$的秩相同,则它们的相抵标准形相同。因为相抵是一个等价关系,由等价关系的对称性与传递性即可得到$A$与$B$相抵。
\end{proof}

\subsection{相似}
\begin{definition}
	$A,B\in M_{n}(K)$。如果存在可逆矩阵$P\in M_{n}(K)$,使得:
	\begin{equation*}
		P^{-1}AP=B
	\end{equation*}
	则称$A$与$B$\gls{Similar}。
\end{definition}
\begin{theorem}
	相似是$M_{n}(K)$上的一个等价关系。在相似关系下,矩阵$A$的等价类称为$A$的\textbf{相似类}。
\end{theorem}
\begin{proof}
	证明是显然的。
\end{proof}
\begin{property}[相似的不变量]\label{prop:Similar}
	相似的矩阵具有相同的行列式值、秩、迹、特征多项式、特征值(包括重数相同)。
\end{property}
\begin{proof}
	设$A,B\in M_{n}(K)$且$A$与$B$相似,于是存在可逆矩阵$P\in M_{n}(K)$使得$P^{-1}AP=B$。\par
	(1)$|A|=|P^{-1}AP|=|P^{-1}|\;|B|\;|P|=|P^{-1}|\;|P|\;|B|=|B|$。\par
	(2)初等行变换与初等列变换不改变矩阵的秩。\par
	(3)由\cref{prop:Trace}(3)可得$\operatorname{tr}(A)=\operatorname{tr}(P^{-1}BP)=\operatorname{tr}(BPP^{-1})=\operatorname{tr}(B)$。\par
	(4)(5)参考\cref{theo:SameEigenvalue}。
\end{proof}

\subsection{合同}
\begin{definition}
	$A,B\in M_{n}(K)$。如果存在可逆矩阵$C\in M_{n}(K)$,使得:
	\begin{equation*}
		C^TAC=B
	\end{equation*}
	则称$A$与$B$\gls{Congruent},记作$A\cong B$。如果对称矩阵$A$合同于一个对角矩阵,那么称这个对角矩阵为$A$的一个\textbf{合同标准形}。
\end{definition}
\begin{theorem}
	合同是$M_{n}(K)$上的一个等价关系。在合同关系下,矩阵$A$的等价类称为$A$的\textbf{合同类}。
\end{theorem}
\begin{proof}
	证明是显然的。
\end{proof}
\begin{definition}
	对$n$阶矩阵的行作初等行变换,再对该矩阵的同样标号的列作相同的初等列变换,这种变换被称为\textbf{成对初等行、列变换}。
\end{definition}
\begin{lemma}\label{lem:CTAC}
	$A,B\in M_{n}(K)$,则$A$合同于$B$当且仅当$A$经过一系列成对初等行、列变换可以变成$B$,此时对$I$作其中的初等列变换即可得到可逆矩阵$C$,使得$C^TAC=B$。
\end{lemma}
\begin{proof}
	由可逆矩阵的初等矩阵分解,可得:
	\begin{gather*}
		\begin{aligned}
			A\cong B
			&\iff\text{存在数域$K$上的可逆矩阵$C$,使得}C^TAC=B \\
			&\iff\text{存在数域$K$上的初等矩阵$P_1,P_2,\dots,P_t$使得}
		\end{aligned}\\
		C=P_1P_2\cdots P_t \\
		P_t^T\cdots P_2^TP_1^TAP_1P_2\cdots P_t=B\qedhere
	\end{gather*}
\end{proof}
\begin{theorem}\label{theo:AllCongruent}
	数域$K$上的任一对称矩阵都合同于一个对角矩阵。
\end{theorem}
\begin{proof}
	对数域$K$上对称矩阵的阶数$n$作数学归纳法,。\par
	当$n=1$时,因为矩阵合同于自身,同时一阶矩阵都是对角矩阵,所以结论成立。\par
	假设$n-1$阶对称矩阵都合同于对角矩阵,考虑$n$阶矩阵$A=(a_{ij})$。\par
	\textbf{情形一:$a_{11}\ne 0$}\par
	把$A$写成分块矩阵的形式,然后对$A$作初等行变换与初等列变换可得:
	\begin{equation*}
		\begin{pmatrix}
			a_{11} & A_1 \\
			A_1^T & A_2
		\end{pmatrix}
		\longrightarrow
		\begin{pmatrix}
			a_{11} & A_1 \\
			\mathbf{0} & A_2-a_{11}^{-1}A_1^TA_1
		\end{pmatrix}
		\longrightarrow
		\begin{pmatrix}
			a_{11} & \mathbf{0} \\
			\mathbf{0} & A_2-a_{11}^{-1}A_1^TA_1
		\end{pmatrix}
	\end{equation*}
	于是有:
	\begin{equation*}
		\begin{pmatrix}
			1 & \mathbf{0} \\
			-a_{11}^{-1}A_1^T & I_{n-1}
		\end{pmatrix}
		\begin{pmatrix}
			a_{11} & A_1 \\
			A_1^T & A_2
		\end{pmatrix}
		\begin{pmatrix}
			1 & -a_{11}^{-1}A_1 \\
			\mathbf{0} & I_{n-1}
		\end{pmatrix}
		=
		\begin{pmatrix}
			a_{11} & \mathbf{0} \\
			\mathbf{0} & A_2-a_{11}^{-1}A_1^TA_1
		\end{pmatrix}
	\end{equation*}
	因为$A$是一个对称矩阵,所以$A_2$是一个对称矩阵,于是:
	\begin{equation*}
		(A_2-a_{11}^{-1}A_1^TA_1)^T=A_2^T-a_{11}^{-1}A_1^T(A_1^T)^T=A_2-a_{11}^{-1}A_1^TA_1
	\end{equation*}
	所以$A_2-a_{11}^{-1}A_1'A_1$是$n-1$阶对称矩阵。由归纳假设可知存在可逆矩阵$C\in M_{n-1}(K)$使得$C^T(A_2-a_{11}^{-1}A_1'A_1)C=D$,其中$D$是一个对角矩阵,即:
	\begin{equation*}
		\begin{pmatrix}
			1 & \mathbf{0} \\
			\mathbf{0} & C^T
		\end{pmatrix}
		\begin{pmatrix}
			a_{11} & \mathbf{0} \\
			\mathbf{0} & A_2-a_{11}^{-1}A_1^TA_1
		\end{pmatrix}
		\begin{pmatrix}
		1 & \mathbf{0} \\
		\mathbf{0} & C
		\end{pmatrix}
		=
		\begin{pmatrix}
			a_{11} & \mathbf{0} \\
			\mathbf{0} & D
		\end{pmatrix}
	\end{equation*}
	于是有:
	\begin{equation*}
		\begin{pmatrix}
			1 & \mathbf{0} \\
			\mathbf{0} & C^T
		\end{pmatrix}
		\begin{pmatrix}
			1 & \mathbf{0} \\
			-a_{11}^{-1}A_1 & I_{n-1}
		\end{pmatrix}
		\begin{pmatrix}
			a_{11} & A_1 \\
			A_1^T & A_2
		\end{pmatrix}
		\begin{pmatrix}
			1 & -a_{11}^{-1}A_1 \\
			\mathbf{0} & I_{n-1}
		\end{pmatrix}
		\begin{pmatrix}
			1 & \mathbf{0} \\
			\mathbf{0} & C
		\end{pmatrix}
		=
		\begin{pmatrix}
			a_{11} & \mathbf{0} \\
			\mathbf{0} & D
		\end{pmatrix}
	\end{equation*}
	因为:
	\begin{equation*}
		\left[
		\begin{pmatrix}
			1 & -a_{11}^{-1}A_1 \\
			\mathbf{0} & I_{n-1}
		\end{pmatrix}
		\begin{pmatrix}
			1 & \mathbf{0} \\
			\mathbf{0} & C
		\end{pmatrix}
		\right]^T
		=
		\begin{pmatrix}
			1 & \mathbf{0} \\
			\mathbf{0} & C
		\end{pmatrix}^T
		\begin{pmatrix}
			1 & -a_{11}^{-1}A_1 \\
			\mathbf{0} & I_{n-1}
		\end{pmatrix}^T
		=
		\begin{pmatrix}
			1 & \mathbf{0} \\
			\mathbf{0} & C^T
		\end{pmatrix}
		\begin{pmatrix}
			1 & \mathbf{0} \\
			-a_{11}^{-1}A_1 & I_{n-1}
		\end{pmatrix}
	\end{equation*}
	并且:
	\begin{equation*}
		\begin{pmatrix}
			1 & -a_{11}^{-1}A_1 \\
			\mathbf{0} & I_{n-1}
		\end{pmatrix}
		\begin{pmatrix}
			1 & \mathbf{0} \\
			\mathbf{0} & C
		\end{pmatrix}
	\end{equation*}
	是一个可逆矩阵,所以$A$合同于对角矩阵:
	\begin{equation*}
		\begin{pmatrix}
			a_{11} & \mathbf{0} \\
			\mathbf{0} & D
		\end{pmatrix}
	\end{equation*}\par
	\textbf{情形二:$a_{11}=0,\;\text{存在$i\ne 1$使得$a_{ii}\ne0$}$}\par
	把$A$的第$1,i$行呼唤,再把所得矩阵的第$1,i$列呼唤,得到的矩阵$B$的$(1,1)$元即为$a_{ii}\ne0$。根据情形一的讨论,$B$合同于一个对角矩阵。因为$B$是由$A$作成对初等行、列变换得到的,由\cref{lem:CTAC}可得$A\cong B$。由合同的传递性,$A$也合同于一个对角矩阵。\par
	\textbf{情形三:$a_{ii}=0,\;\forall\;i=1,2,\dots,n,\;\text{存在$a_{ij}\ne 0,\;i\ne j$}$}\par
	把$A$的第$j$行加到第$i$行上,再把所得矩阵的第$j$列加到第$i$列上,得到的矩阵$E$的$(i,i)$元即为$2a_{ij}\ne0$。由情形二的讨论,$E$合同于一个对角矩阵。因为$E$是由$A$作成对初等行、列变换得到的,由\cref{lem:CTAC}可得$A\cong E$。由合同的传递性,$A$也合同于一个对角矩阵。\par
	\textbf{情形四:$A=\mathbf{0}$}\par
	因为$\mathbf{0}$是一个对角矩阵,所以结论显然成立。
\end{proof}
\begin{theorem}\label{theo:CongruentRank}
	设对角矩阵$B$是对称矩阵$A$的合同标准形,则$B$对角线上不为$0$的元素的个数等于$A$的秩。
\end{theorem}
\begin{proof}
	因为$A\cong B$,所以存在可逆矩阵$C$使得$C^TAC=B$,于是$\operatorname{rank}(A)=\operatorname{rank}(B)$。
\end{proof}
\subsubsection{实对称矩阵的合同规范形}
\begin{theorem}\label{theo:Congruent1-10}
	对于任意的对称矩阵$A\in M_{n}(\mathbb{R})$,$A$都合同于对角矩阵\\$\operatorname{diag}\{1,1,\dots,1,-1,-1,\dots,-1,0,0,\dots,0\}$,系数为$1$的平方项个数称为$A$的\gls{PositiveInertiaIndex},系数为$-1$的平方项个数称为$A$的\gls{NegativeInertiaIndex},这个对角矩阵称为$A$的\textbf{合同规范形}。
\end{theorem}
\begin{proof}
	任取矩阵$A\in M_{n}(\mathbb{R})$,由\cref{theo:AllCongruent}可得$A$合同一个对角矩阵$B$。对$B$作成对初等行、列变换可将$B$对角线上的元素重新排列,使得正值在前,负值在中间,零值在最后,如此得到对角矩阵$C$,$C$可写作:
	\begin{equation*}
		C=
		\left(
		\begin{array}{*{11}c}
			c_1 & 0 & \cdots & 0 & 0 & 0 & \cdots & 0 & 0 & \cdots & 0 \\
			0 & c_2 & \cdots & 0 & 0 & 0 & \cdots & 0 & 0 & \cdots & 0 \\
			\vdots & \vdots & \ddots & \vdots & \vdots & \vdots & \ddots & \vdots & \vdots & \ddots & \vdots \\
			0 & 0 & \cdots & c_p & 0 & 0 & \cdots & 0 & 0 & \cdots & 0 \\
			0 & 0 & \cdots & 0 & -c_{p+1} & 0 & \cdots & 0 & 0 & \cdots & 0 \\
			0 & 0 & \cdots & 0 & 0 & -c_{p+2} & \cdots & 0 & 0 & \cdots & 0 \\
			\vdots & \vdots & \ddots & \vdots & \vdots & \vdots & \ddots & \vdots & \vdots & \ddots & \vdots \\
			0 & 0 & \cdots & 0 & 0 & 0 & \cdots & -c_r & 0 & \cdots & 0 \\
			0 & 0 & \cdots & 0 & 0 & 0 & \cdots & 0 & 0 & \cdots & 0 \\
			\vdots & \vdots & \ddots & \vdots & \vdots & \vdots & \ddots & \vdots & \vdots & \ddots & \vdots \\
			0 & 0 & \cdots & 0 & 0 & 0 & \cdots & 0 & 0 & \cdots & 0 \\
		\end{array}
		\right)
	\end{equation*}
	其中$c_1,c_2,\dots,c_r>0$。再对$C$作成对初等行、列变换,即先对第$i$行除$\sqrt{c_i}$,再对第$i$列除$\sqrt{c_i},\;i=1,2,\dots,n$,即可得到对角矩阵$D$:
	\begin{equation*}
		D=
		\left(
		\begin{array}{*{11}c}
			1 & 0 & \cdots & 0 & 0 & 0 & \cdots & 0 & 0 & \cdots & 0 \\
			0 & 1 & \cdots & 0 & 0 & 0 & \cdots & 0 & 0 & \cdots & 0 \\
			\vdots & \vdots & \ddots & \vdots & \vdots & \vdots & \ddots & \vdots & \vdots & \ddots & \vdots \\
			0 & 0 & \cdots & 1 & 0 & 0 & \cdots & 0 & 0 & \cdots & 0 \\
			0 & 0 & \cdots & 0 & -1 & 0 & \cdots & 0 & 0 & \cdots & 0 \\
			0 & 0 & \cdots & 0 & 0 & -1 & \cdots & 0 & 0 & \cdots & 0 \\
			\vdots & \vdots & \ddots & \vdots & \vdots & \vdots & \ddots & \vdots & \vdots & \ddots & \vdots \\
			0 & 0 & \cdots & 0 & 0 & 0 & \cdots & -1 & 0 & \cdots & 0 \\
			0 & 0 & \cdots & 0 & 0 & 0 & \cdots & 0 & 0 & \cdots & 0 \\
			\vdots & \vdots & \ddots & \vdots & \vdots & \vdots & \ddots & \vdots & \vdots & \ddots & \vdots \\
			0 & 0 & \cdots & 0 & 0 & 0 & \cdots & 0 & 0 & \cdots & 0 \\
		\end{array}
		\right)
	\end{equation*}
	由\cref{lem:CTAC}可得,$D\cong C$,$C\cong B$,又因为$A\cong B$,由合同的传递性与对称性即可得$A\cong D$。由$A$的任意性结论得证。
\end{proof}
\subsubsection{复对称矩阵的合同规范形}
\begin{theorem}\label{theo:Congruent10}
	对于任意的$A\in M_{n}(\mathbb{C})$,$A$都合同于对角矩阵$\operatorname{diag}\{1,1,\dots,1,0,0,\dots,0\}$,这个对角矩阵称为$A$的合同规范形。
\end{theorem}
\begin{proof}
	任取矩阵$A\in M_{n}(\mathbb{C})$,由\cref{theo:AllCongruent}可得$A\cong B=\operatorname{diag}\{b_1,b_2,\dots,b_r,0,0,\dots,0\}$,其中$r$是矩阵$B$的秩,$b_1,b_2,\dots,b_r\ne0$。设$b_j=r_j\cos\theta_j+ir_j\sin\theta_j,\;\theta_j\in[0,2\pi),\;j=1,2,\dots,r$。因为:
	\begin{equation*}
		\left[\sqrt{r_j}\left(\cos\frac{\theta_j}{2}+i\sin\frac{\theta_j}{2}\right)\right]^2=b_j
	\end{equation*}
	将$\sqrt{r_j}\left(\cos\dfrac{\theta_j}{2}+i\sin\dfrac{\theta_j}{2}\right)$记作$\sqrt{b_j}$,作成对初等行、列变换,即先对第$j$行除$\sqrt{b_j}$,再对第$j$列除$\sqrt{b_j}$,则可得到矩阵$C=\operatorname{diag}\{1,1,\dots,1,0,0\dots,0\}$,其中$1$的个数为$r$。由\cref{lem:CTAC}可得,$B\cong C$。因为$A\cong B$,由合同的传递性,$A\cong C$。由$A$的任意性,结论成立。
\end{proof}
\section{相抵的应用}

\subsection{广义逆}
\begin{definition}
	设$A\in M_{m\times n}(K)$,一切满足方程组:
	\begin{equation*}
		AXA=A
	\end{equation*}
	的矩阵$X$都被称为是$A$的\gls{GeneralizedInverse},记为$A^-$。
\end{definition}
 \begin{theorem}\label{theo:ExistenceOfGeneralizedInverse}
	设非零矩阵$A\in M_{m\times n}(K)$,$\operatorname{rank}(A)=r$且:
	\begin{equation*}
		A=P
		\begin{pmatrix}
			I_r & \mathbf{0} \\
			\mathbf{0} & \mathbf{0}
		\end{pmatrix}
		Q
	\end{equation*}
	其中$P,Q$分别为数域$K$上的$m$阶可逆矩阵和$n$阶可逆矩阵,则矩阵方程:
	\begin{equation*}
		AXA=A
	\end{equation*}
	一定有解,且其通解可表示为:
	\begin{equation*}
		X=Q^{-1}
		\begin{pmatrix}
			I_r & B \\
			C & D
		\end{pmatrix}
		P^{-1}
	\end{equation*}
	其中$B,C,D$分别为数域$K$上任意的$r\times (m-r),\;(n-r)\times r,\;(n-r)\times(m-r)$矩阵。
\end{theorem}
\begin{proof}
	若$X$是上述矩阵方程的一个解,则:
	\begin{gather*}
		P
		\begin{pmatrix}
			I_r & \mathbf{0} \\
			\mathbf{0} & \mathbf{0}
		\end{pmatrix}
		Q
		X
		P
		\begin{pmatrix}
			I_r & \mathbf{0} \\
			\mathbf{0} & \mathbf{0}
		\end{pmatrix}
		Q
		=
		P
		\begin{pmatrix}
			I_r & \mathbf{0} \\
			\mathbf{0} & \mathbf{0}
		\end{pmatrix}
		Q  \\
		\begin{pmatrix}
			I_r & \mathbf{0} \\
			\mathbf{0} & \mathbf{0}
		\end{pmatrix}
		Q
		X
		P
		\begin{pmatrix}
			I_r & \mathbf{0} \\
			\mathbf{0} & \mathbf{0}
		\end{pmatrix}
		=
		\begin{pmatrix}
			I_r & \mathbf{0} \\
			\mathbf{0} & \mathbf{0}
		\end{pmatrix}
	\end{gather*}
	将$QXP$写作如下分块矩阵的形式:
	\begin{equation*}
		QXP=
		\begin{pmatrix}
			H & B \\
			C & D
		\end{pmatrix}
	\end{equation*}
	其中$H,B,C,D$分别为数域$K$上任意的$r\times r,\;r\times (m-r),\;(n-r)\times r,\;(n-r)\times(m-r)$矩阵。于是:
	\begin{gather*}
		\begin{pmatrix}
			I_r & \mathbf{0} \\
			\mathbf{0} & \mathbf{0}
		\end{pmatrix}
		\begin{pmatrix}
			H & B \\
			C & D
		\end{pmatrix}
		\begin{pmatrix}
			I_r & \mathbf{0} \\
			\mathbf{0} & \mathbf{0}
		\end{pmatrix}
		=
		\begin{pmatrix}
			I_r & \mathbf{0} \\
			\mathbf{0} & \mathbf{0}
		\end{pmatrix} \\
		\begin{pmatrix}
			H & \mathbf{0} \\
			\mathbf{0} & \mathbf{0}
		\end{pmatrix}
		=
		\begin{pmatrix}
			I_r & \mathbf{0} \\
			\mathbf{0} & \mathbf{0}
		\end{pmatrix}
	\end{gather*}
	所以$H=I_r$,因此:
	\begin{equation*}
		X=Q^{-1}
		\begin{pmatrix}
			I_r & B \\
			C & D
		\end{pmatrix}
		P^{-1}\qedhere
	\end{equation*}
\end{proof}
\begin{property}\label{prop:A-}
	设$A\in M_{m\times n}(K),\;B\in M_{m\times q}(K),\;C\in M_{p\times n}(K)$,则广义逆$A^-$具有如下性质:
	\begin{enumerate}
		\item $A^-$唯一的充分必要条件为$A$可逆,此时$A^-=A^{-1}$;
		\item $\operatorname{rank}(A^-)\geqslant\operatorname{rank}(A)=\operatorname{rank}(AA^-)=\operatorname{rank}(A^-A)$;
		\item 若$\mathcal{M}(B)\subseteq\mathcal{M}(A),\mathcal{M}(C)\subseteq\mathcal{M}(A^T)$,则$C^TA^-B$是唯一的\info{两个唯一性的证明都问题};
		\item $A(A^TA)^-A^T$是唯一的;
		\item 若$A$对称,则$(A^-)^T=A^-$;
		\item $A(A^TA)^-A^TA=A,\;A^TA(A^TA)^-A^T=A^T$;
		\item 若存在$\alpha$使得$c=A^T\alpha$,则$c^T(A^TA)^-A^TA=c^T$。
	\end{enumerate}
\end{property}
\begin{proof}
	(1)\textbf{充分性:}若$A$可逆,由\cref{prop:InvertibleMatrix}(3)可知$r=n$,由\cref{theo:ExistenceOfGeneralizedInverse}中$A^-$的通解公式,显然此时$A^-$唯一。\par
	\textbf{必要性:}若$A^-$唯一,则$r=n$,由\cref{prop:InvertibleMatrix}(3)可知此时$A$可逆。\par
	(2)由$A^-$的通解公式,$\operatorname{rank}(A^-)\geqslant r=\operatorname{rank}(A)$。因为:
	\begin{gather*}
		AA^-=P
		\begin{pmatrix}
			I_r & \mathbf{0} \\
			\mathbf{0} & \mathbf{0}
		\end{pmatrix}
		QQ^{-1}
		\begin{pmatrix}
			I_r & B \\
			C & D
		\end{pmatrix}
		P^{-1}
		=P
		\begin{pmatrix}
			I_r & B \\
			\mathbf{0} & \mathbf{0}
		\end{pmatrix}
		p^{-1} \\
		A^-A=Q^{-1}
		\begin{pmatrix}
			I_r & B \\
			C & D
		\end{pmatrix}
		P^{-1}P
		\begin{pmatrix}
			I_r & \mathbf{0} \\
			\mathbf{0} & \mathbf{0}
		\end{pmatrix}
		Q
		=Q^{-1}
		\begin{pmatrix}
			I_r & \mathbf{0} \\
			C & \mathbf{0}
		\end{pmatrix}
		Q
	\end{gather*}
	显然,$\operatorname{rank}(AA^-)=\operatorname{rank}(A^-A)=\operatorname{rank}(A)=r$。\par
	(3)由已知条件,存在矩阵$D_1,D_2$使得$B=AD_1,\;C=A^TD_2$,于是:
	\begin{equation*}
		C^TA^-B=D_2^TAA^-AD_1=D_2^TAD_1
	\end{equation*}\par
	(4)由\cref{prop:OrthogonalProjectionMat}(2)可知$A(A^TA)^-A^T$是向$\mathcal{M}(A)$的正交投影阵,由\cref{prop:ProjectionTransformation}(7)和\cref{theo:LinearTransformationMatrix}可得$A(A^TA)^-A$是唯一的,与$(A^TA)^-$的选择无关。\par
	(5)此时有:
	\begin{equation*}
		AXA=A\iff A^TX^TA^T=A^T\iff AX^TA=A
	\end{equation*}\par
	(6)设$B=A(A^TA)^-A^TA-A$,则:
	\begin{align*}
		B^TB
		&=\{A^TA[(A^TA)^-]^TA^T-A^T\}[A(A^TA)^-A^TA-A] \\
		&=A^TA[(A^TA)^-]^TA^TA(A^TA)^-A^TA-A^TA[(A^TA)^-]^TA^TA \\
		&\quad-A^TA(A^TA)^-A^TA+A^TA \\
		&=A^TA[(A^TA)^-]^TA^TA-A^TA[(A^TA)^-]^TA^TA-A^TA+A^TA=\mathbf{0}
	\end{align*}
	所以$B=\mathbf{0}$(考虑$B^TB$主对角线上的元素),于是$A(A^TA)^-A^TA=A$。第二个等式由\cref{prop:Transpose}(4)和第一个等式即可得到。\par
	(7)由(6)和\cref{prop:Transpose}(4)可得:
	\begin{equation*}
		c^T(A^TA)^-A^TA=\alpha^TA(A^TA)^-A^TA=\alpha^TA=c^T\qedhere
	\end{equation*}
\end{proof}

\subsection{Moore-Penrose广义逆}
\begin{definition}
	设$A\in M_{m\times n}(\mathbb{C})$。若$X\in M_{n\times m}(\mathbb{C})$满足:
	\begin{equation*}
		\begin{cases}
			AXA=A \\
			XAX=X \\
			(AX)^H=AX \\
			(XA)^H=XA
		\end{cases}
	\end{equation*}
	则称$X$为$A$的Moore-Penrose广义逆,记作$A^+$,上述方程组被称为$A$的Penrose方程组。
\end{definition}
\subsubsection{满秩分解导出的广义逆}
\begin{theorem}
	设$A\in M_{m\times n}(\mathbb{C})$,则$A$的Penrose方程组一定有唯一解。对$A$进行满秩分解,设$A=BC$,其中$B,C$分别为列满秩矩阵与行满秩矩阵,则$A$的Penrose方程组的解可表示为:
	\begin{equation*}
		X=C^H(CC^H)^{-1}(B^HB)^{-1}B^H
	\end{equation*}
\end{theorem}
\begin{proof}
	由\cref{prop:MatrixRank}(8)可知$(B^HB)^{-1},(CC^H)^{-1}$存在,将上述$X$代入$A$的Penrose方程组可得:
	\begin{align*}
		&\begin{aligned}
			XAX
			&=C^H(CC^H)^{-1}(B^HB)^{-1}B^HBCC^H(CC^H)^{-1}(B^HB)^{-1}B^H \\
			&=C^H(CC^H)^{-1}(B^HB)^{-1}B^H=X
		\end{aligned} \\
		&AXA=BCC^H(CC^H)^{-1}(B^HB)^{-1}B^HBC=BC=A \\
		&\begin{aligned}
			(AX)^H
			&=X^HA^H=B[(B^HB)^{-1}]^H[(CC^H)^{-1}]^HCC^HB^H \\
			&=B[(B^HB)^{-1}]^H[(CC^H)^{-1}]^HCC^HB^H \\
			&=B[(B^HB)^H]^{-1}[(CC^H)^H]^{-1}CC^HB^H \\
			&=B(B^HB)^{-1}(CC^H)^{-1}CC^HB^H \\
			&=B(B^HB)^{-1}B^H \\
			&=B(CC^H)(CC^H)^{-1}(B^HB)^{-1}B^H=AX
		\end{aligned} \\
		&\begin{aligned}
			(XA)^H
			&=A^HX^H=C^HB^HB[(B^HB)^{-1}]^H[(CC^H)^{-1}]^HC \\
			&=C^H(CC^H)^{-1}C=C^H(CC^H)^{-1}(B^HB)^{-1}(B^HB)C=XA
		\end{aligned}
	\end{align*}
	于是$X$与$A$的Penrose方程组相容,所以$X$是解。
\end{proof}
\subsubsection{奇异值分解导出的广义逆}
\begin{theorem}\label{theo:A+SVD}
	设$A\in M_{m\times n}(\mathbb{C})$,则有:
	\begin{equation*}
		A^+=Q
		\begin{pmatrix}
			\varLambda^{-1} & \mathbf{0} \\
			\mathbf{0} & \mathbf{0}
		\end{pmatrix}
		P^H
	\end{equation*}
	其中$P,Q,\varLambda$为$A$的奇异值分解中相关矩阵。
\end{theorem}
\begin{proof}
	将之代入到$A$的Penrose方程组中可得:
	\begin{gather*}
		\begin{aligned}
			AQ
			\begin{pmatrix}
				\varLambda^{-1} & \mathbf{0} \\
				\mathbf{0} & \mathbf{0}
			\end{pmatrix}
			P^HA
			&=P
			\begin{pmatrix}
				\varLambda & \mathbf{0} \\
				\mathbf{0} & \mathbf{0}
			\end{pmatrix}Q^HQ
			\begin{pmatrix}
				\varLambda^{-1} & \mathbf{0} \\
				\mathbf{0} & \mathbf{0}
			\end{pmatrix}
			P^HP
			\begin{pmatrix}
				\varLambda & \mathbf{0} \\
				\mathbf{0} & \mathbf{0}
			\end{pmatrix}Q^H \\
			&=P\begin{pmatrix}
				\varLambda & \mathbf{0} \\
				\mathbf{0} & \mathbf{0}
			\end{pmatrix}Q^H
			=A
		\end{aligned} \\
		\begin{aligned}
			Q
			\begin{pmatrix}
				\varLambda^{-1} & \mathbf{0} \\
				\mathbf{0} & \mathbf{0}
			\end{pmatrix}
			P^HA
			Q
			\begin{pmatrix}
				\varLambda^{-1} & \mathbf{0} \\
				\mathbf{0} & \mathbf{0}
			\end{pmatrix}
			P^H
			&=Q
			\begin{pmatrix}
				\varLambda^{-1} & \mathbf{0} \\
				\mathbf{0} & \mathbf{0}
			\end{pmatrix}
			P^HP
			\begin{pmatrix}
				\varLambda & \mathbf{0} \\
				\mathbf{0} & \mathbf{0}
			\end{pmatrix}
			Q^HQ
			\begin{pmatrix}
				\varLambda^{-1} & \mathbf{0} \\
				\mathbf{0} & \mathbf{0}
			\end{pmatrix}
			P^H \\
			&=Q
			\begin{pmatrix}
				\varLambda^{-1} & \mathbf{0} \\
				\mathbf{0} & \mathbf{0}
			\end{pmatrix}
			P^H
		\end{aligned} \\
		AQ
		\begin{pmatrix}
			\varLambda^{-1} & \mathbf{0} \\
			\mathbf{0} & \mathbf{0}
		\end{pmatrix}
		P^H=Q
		\begin{pmatrix}
			\varLambda^{-1} & \mathbf{0} \\
			\mathbf{0} & \mathbf{0}
		\end{pmatrix}
		P^HA=I
	\end{gather*}
	因为$I$是Hermitian矩阵,于是$Q
	\begin{pmatrix}
		\varLambda^{-1} & \mathbf{0} \\
		\mathbf{0} & \mathbf{0}
	\end{pmatrix}
	P^H$与$A$的Penrose方程组相容,所以它是解。
\end{proof}
\subsubsection{Moore-Penrose广义逆的性质}
\begin{property}\label{prop:A+}
	设$A\in M_{m\times n}(\mathbb{C})$,则$A$的Moore-Penrose广义逆$A^+$具有如下性质:
	\begin{enumerate}
		\item $A^+$是唯一的;
		\item $(A^+)^+=A$;
		\item $(A^+)^H=(A^H)^+$;
		\item $\operatorname{rank}(A^+)=\operatorname{rank}(A)$;
		\item 若$A$是一个Hermitian矩阵,则:
		\begin{equation*}
			A^+=Q
			\begin{pmatrix}
				\varLambda^{-1} & \mathbf{0} \\
				\mathbf{0} & \mathbf{0}
			\end{pmatrix}Q^H
		\end{equation*}
		其中$\varLambda$为$A$的非零特征值构成的对角矩阵,$Q$是一个正交矩阵;
		\item 若$\alpha$是一个非零向量,则$\alpha^+=\frac{\alpha^H}{||\alpha||^2}$;
		\item $I-A^+A\geqslant \mathbf{0}$;
		\item $(A^HA)^+=A^+(A^H)^+$;
		\item $A^+=(A^HA)^+A^H=A^H(AA^H)^+$。
	\end{enumerate}
\end{property}
\begin{proof}
	(1)设$X_1,X_2$都是$A$的Penrose方程组的解,则:
	\begin{align*}
		X_1
		&=X_1AX_1=X_1(AX_2A)X_1=X_1(AX_2)(AX_1) \\
		&=X_1(AX_2)^H(AX_1)^H=X_1(AX_1AX_2)^H=X_1X_2^H(AX_1A)^H \\
		&=X_1X_2^HA^H=X_1(AX_2)^H=X_1AX_2=X_1(AX_2A)X_2 \\
		&=(X_1A)(X_2A)X_2=(X_1A)^H(X_2A)^HX_2=(X_2AX_1A)^HX_2 \\
		&=(X_2A)^HX_2=X_2AX_2=X_2
	\end{align*}
	所以Penrose方程组的解是唯一的。\par
	(2)由Penrose方程的对称性可直接得到。\par
	(3)由$A^+$的奇异值分解表示(\cref{theo:A+SVD})和\cref{prop:Transpose}(4)可得:
	\begin{align*}
		(A^+)^H&=\left[Q
		\begin{pmatrix}
			\varLambda^{-1} & \mathbf{0} \\
			\mathbf{0} & \mathbf{0}
		\end{pmatrix}P^H\right]^H
		=P
		\begin{pmatrix}
			\varLambda^{-1} & \mathbf{0} \\
			\mathbf{0} & \mathbf{0}
		\end{pmatrix}^HQ^H \\
		&=P
		\begin{pmatrix}
			(\varLambda^{-1})^H & \mathbf{0} \\
			\mathbf{0} & \mathbf{0}
		\end{pmatrix}Q^H
		=P
		\begin{pmatrix}
			(\varLambda^H)^{-1} & \mathbf{0} \\
			\mathbf{0} & \mathbf{0}
		\end{pmatrix}Q^H
	\end{align*}
	将其代入$A^H$的Penrose方程组可得:
	\begin{gather*}
		\begin{aligned}
			A^H(A^+)^HA^H
			&=Q
			\begin{pmatrix}
				\varLambda^H & \mathbf{0} \\
				\mathbf{0} & \mathbf{0}
			\end{pmatrix}
			P^HP
			\begin{pmatrix}
				(\varLambda^H)^{-1} & \mathbf{0} \\
				\mathbf{0} & \mathbf{0}
			\end{pmatrix}
			Q^HQ
			\begin{pmatrix}
				\varLambda^H & \mathbf{0} \\
				\mathbf{0} & \mathbf{0}
			\end{pmatrix}
			P^H \\
			&=Q
			\begin{pmatrix}
				\varLambda^H & \mathbf{0} \\
				\mathbf{0} & \mathbf{0}
			\end{pmatrix}
			P^H
			=A^H
		\end{aligned} \\
		\begin{aligned}
			(A^+)^HA^H(A^+)^H
			&=P
			\begin{pmatrix}
				(\varLambda^H)^{-1} & \mathbf{0} \\
				\mathbf{0} & \mathbf{0}
			\end{pmatrix}
			Q^H\begin{pmatrix}
				\varLambda^H & \mathbf{0} \\
				\mathbf{0} & \mathbf{0}
			\end{pmatrix}
			P^HP
			\begin{pmatrix}
				(\varLambda^H)^{-1} & \mathbf{0} \\
				\mathbf{0} & \mathbf{0}
			\end{pmatrix}Q^H \\
			&=P
			\begin{pmatrix}
				(\varLambda^H)^{-1} & \mathbf{0} \\
				\mathbf{0} & \mathbf{0}
			\end{pmatrix}Q^H
			=(A^+)^H
		\end{aligned} \\
		[A^H(A^+)^H]^H=[(A^+)^HA^H]^H=A^+A=I
	\end{gather*}
	因为$I$是Hermitian矩阵,于是$(A^+)^H$与$A^H$的Penrose方程组相容,所以$(A^+)^H=(A^H)^+$。\par
	(4)由$A^+$的奇异值分解表示(\cref{theo:A+SVD})和\cref{prop:InvertibleMatrix}(16)可得$\operatorname{rank}(A^+)=\operatorname{rank}(\varLambda)$,而$\operatorname{rank}(\varLambda)=\operatorname{rank}(A)$,所以有$\operatorname{rank}(A^+)=\operatorname{rank}(A)$。\par
	(5)因为$A$是一个Hermitian矩阵,由\cref{prop:HermitianMatEigen}(3)可知存在酉矩阵$Q$使得:
	\begin{equation*}
		A=Q
		\begin{pmatrix}
			\varLambda & \mathbf{0} \\
			\mathbf{0} & \mathbf{0}
		\end{pmatrix}Q^H
	\end{equation*}
	由\cref{prop:Transpose}(4),将$Q
	\begin{pmatrix}
		\varLambda^{-1} & \mathbf{0} \\
		\mathbf{0} & \mathbf{0}
	\end{pmatrix}Q^H$代入$A$的Penrose方程组可得:
	\begin{gather*}
		\begin{aligned}
			AQ
			\begin{pmatrix}
				\varLambda^{-1} & \mathbf{0} \\
				\mathbf{0} & \mathbf{0}
			\end{pmatrix}Q^HA
			&=Q
			\begin{pmatrix}
				\varLambda & \mathbf{0} \\
				\mathbf{0} & \mathbf{0}
			\end{pmatrix}
			Q^HQ
			\begin{pmatrix}
				\varLambda^{-1} & \mathbf{0} \\
				\mathbf{0} & \mathbf{0}
			\end{pmatrix}
			Q^HQ
			\begin{pmatrix}
				\varLambda & \mathbf{0} \\
				\mathbf{0} & \mathbf{0}
			\end{pmatrix}
			Q^H \\
			&=Q
			\begin{pmatrix}
				\varLambda & \mathbf{0} \\
				\mathbf{0} & \mathbf{0}
			\end{pmatrix}
			Q^H
			=A
		\end{aligned} \\
		\begin{aligned}
			Q
			\begin{pmatrix}
				\varLambda^{-1} & \mathbf{0} \\
				\mathbf{0} & \mathbf{0}
			\end{pmatrix}
			Q^HAQ
			\begin{pmatrix}
				\varLambda^{-1} & \mathbf{0} \\
				\mathbf{0} & \mathbf{0}
			\end{pmatrix}
			Q^H
			&=Q
			\begin{pmatrix}
				\varLambda^{-1} & \mathbf{0} \\
				\mathbf{0} & \mathbf{0}
			\end{pmatrix}
			Q^HQ
			\begin{pmatrix}
				\varLambda & \mathbf{0} \\
				\mathbf{0} & \mathbf{0}
			\end{pmatrix}
			Q^HQ
			\begin{pmatrix}
				\varLambda^{-1} & \mathbf{0} \\
				\mathbf{0} & \mathbf{0}
			\end{pmatrix}
			Q^H \\
			&=Q
			\begin{pmatrix}
				\varLambda^{-1} & \mathbf{0} \\
				\mathbf{0} & \mathbf{0}
			\end{pmatrix}
			Q^H
		\end{aligned} \\
		\left[AQ
		\begin{pmatrix}
			\varLambda^{-1} & \mathbf{0} \\
			\mathbf{0} & \mathbf{0}
		\end{pmatrix}
		Q^H\right]^H
		=
		\left[Q
		\begin{pmatrix}
			\varLambda^{-1} & \mathbf{0} \\
			\mathbf{0} & \mathbf{0}
		\end{pmatrix}
		Q^HA\right]^H
		=Q
		\begin{pmatrix}
			I & \mathbf{0} \\
			\mathbf{0} & \mathbf{0}
		\end{pmatrix}
		Q^H
	\end{gather*}
	因为$Q
	\begin{pmatrix}
		I & \mathbf{0} \\
		\mathbf{0} & \mathbf{0}
	\end{pmatrix}
	Q^H$是Hermitian矩阵,于是$Q
	\begin{pmatrix}
		\varLambda^{-1} & \mathbf{0} \\
		\mathbf{0} & \mathbf{0}
	\end{pmatrix}Q^H$与$A$的Penrose方程组相容,所以$Q
	\begin{pmatrix}
	\varLambda^{-1} & \mathbf{0} \\
	\mathbf{0} & \mathbf{0}
	\end{pmatrix}Q^H=A^+$。\par
	(6)将$\frac{\alpha^H}{||\alpha||^2}$代入$\alpha$的Penrose方程组可得:
	\begin{gather*}
		\alpha\frac{\alpha^H}{||\alpha||^2}\alpha=\alpha  \\
		\frac{\alpha^H}{||\alpha||^2}\alpha\frac{\alpha^H}{||\alpha||^2}=\frac{\alpha^H}{||\alpha||^2} \\
		\left(\alpha\frac{\alpha^H}{||\alpha||^2}\right)^H=\left(\frac{\alpha^H}{||\alpha||^2}\alpha\right)^H=1
	\end{gather*}
	显然$\dfrac{\alpha^H}{||\alpha||^2}=\alpha^+$。\par
	(7)由$A^+$的奇异值分解表示(\cref{theo:A+SVD})可得:
	\begin{align*}
		I-A^+A&=I-Q
		\begin{pmatrix}
			\varLambda^{-1} & \mathbf{0} \\
			\mathbf{0} & \mathbf{0}
		\end{pmatrix}
		P^HP
		\begin{pmatrix}
			\varLambda & \mathbf{0} \\
			\mathbf{0} & \mathbf{0}
		\end{pmatrix}Q^H
		=I-Q
		\begin{pmatrix}
			I & \mathbf{0} \\
			\mathbf{0} & \mathbf{0}
		\end{pmatrix}Q^H \\
		&=I-\begin{pmatrix}
			I & \mathbf{0} \\
			\mathbf{0} & \mathbf{0}
		\end{pmatrix}
		=\begin{pmatrix}
			\mathbf{0} & \mathbf{0} \\
			\mathbf{0} & I
		\end{pmatrix}\cong
		\begin{pmatrix}
			I & \mathbf{0} \\
			\mathbf{0} & \mathbf{0}
		\end{pmatrix}
	\end{align*}
	由\cref{theo:PositiveSemidefinite}(3.3)可知$I-A^+A\geqslant\mathbf{0}$。\par
	(8)由(3)可得:
	\begin{align*}
		A^+(A^H)^+&=A^+(A^+)^H=Q
		\begin{pmatrix}
			\varLambda^{-1} & \mathbf{0} \\
			\mathbf{0} & \mathbf{0}
		\end{pmatrix}
		P^HP
		\begin{pmatrix}
			(\varLambda^H)^{-1} & \mathbf{0} \\
			\mathbf{0} & \mathbf{0}
		\end{pmatrix}Q^HQ^H \\
		&=Q\begin{pmatrix}
			\varLambda^{-1}(\varLambda^H)^{-1} & \mathbf{0} \\
			\mathbf{0} & \mathbf{0}
		\end{pmatrix}Q^H
		=\begin{pmatrix}
			\varLambda^{-1}(\varLambda^{H})^{-1} & \mathbf{0} \\
			\mathbf{0} & \mathbf{0}
		\end{pmatrix}
	\end{align*}
	由$A$的奇异值分解(\cref{theo:SVD})可得:
	\begin{align*}
		A^HA&=\left[P
		\begin{pmatrix}
			\varLambda & \mathbf{0} \\
			\mathbf{0} & \mathbf{0}
		\end{pmatrix}Q^H\right]^HP
		\begin{pmatrix}
			\varLambda & \mathbf{0} \\
			\mathbf{0} & \mathbf{0}
		\end{pmatrix}Q^H \\
		=&Q
		\begin{pmatrix}
			\varLambda^H & \mathbf{0} \\
			\mathbf{0} & \mathbf{0}
		\end{pmatrix}
		P^HP
		\begin{pmatrix}
			\varLambda & \mathbf{0} \\
			\mathbf{0} & \mathbf{0}
		\end{pmatrix}Q^H
		=\begin{pmatrix}
			\varLambda^H\varLambda & \mathbf{0} \\
			\mathbf{0} & \mathbf{0}
		\end{pmatrix}
	\end{align*}
	将$A^+(A^H)^+$代入$A^HA$的Penrose方程组中即可验证得到$(A^HA)^+=A^+(A^H)^+$。\par
	(9)由(8)、(3)和$A^+$的奇异值分解表示(\cref{theo:A+SVD})可得:
	\begin{gather*}
		\begin{aligned}
			(A^HA)^+A^H
			&=A^+(A^H)^+A^H
			=A^+(A^+)^HA^H \\
			&=Q
			\begin{pmatrix}
				\varLambda^{-1} & \mathbf{0} \\
				\mathbf{0} & \mathbf{0}
			\end{pmatrix}
			P^HP
			\begin{pmatrix}
				(\varLambda^H)^{-1} & \mathbf{0} \\
				\mathbf{0} & \mathbf{0}
			\end{pmatrix}
			Q^HQ
			\begin{pmatrix}
				\varLambda^H & \mathbf{0} \\
				\mathbf{0} & \mathbf{0}
			\end{pmatrix}
			P^H \\
			&=Q
			\begin{pmatrix}
				\varLambda^{-1} & \mathbf{0} \\
				\mathbf{0} & \mathbf{0}
			\end{pmatrix}
			P^H=A^+
		\end{aligned} \\
		\begin{aligned}
			A^H(AA^H)^+
			&=A^H(A^H)^+A^+
			=A^H(A^+)^HA^+ \\
			&=Q
			\begin{pmatrix}
				\varLambda^H & \mathbf{0} \\
				\mathbf{0} & \mathbf{0}
			\end{pmatrix}
			P^HP
			\begin{pmatrix}
				(\varLambda^H)^{-1} & \mathbf{0} \\
				\mathbf{0} & \mathbf{0}
			\end{pmatrix}
			Q^HQ
			\begin{pmatrix}
				\varLambda^{-1} & \mathbf{0} \\
				\mathbf{0} & \mathbf{0}
			\end{pmatrix}
			P^H \\
			&=Q
			\begin{pmatrix}
				\varLambda^{-1} & \mathbf{0} \\
				\mathbf{0} & \mathbf{0}
			\end{pmatrix}
			P^H=A^+
		\end{aligned}\qedhere
	\end{gather*}
\end{proof}

\subsection{线性方程组的解}
\begin{theorem}\label{theo:ConsistentLinearEqCondition}
	数域$K$上$n$元非齐次线性方程组$Ax=\beta$有解的充分必要条件为对$A$的任一广义逆$A^-$都有:
	\begin{equation*}
		\beta=AA^-\beta
	\end{equation*}
\end{theorem}
\begin{proof}
	\textbf{(1)必要性:}若$Ax=\beta$有解,取其一个解$\alpha$,于是对$A$的任一广义逆有:
	\begin{equation*}
		\beta=A\alpha=AA^-A\alpha=AA^-\beta
	\end{equation*}
	\textbf{(2)充分性:}若此时对$A$的任一广义逆$A^-$有$\beta=AA^-\beta$,则方程组可化为:
	\begin{equation*}
		Ax=AA^-\beta
	\end{equation*}
	容易看出$A^-\beta$就是$Ax=\beta$的一个解。
\end{proof}
\subsubsection{齐次方程组解的结构}
\begin{theorem}\label{theo:HomogeneousLinearEq'sGeneralSolution}
	若数域$K$上$n$元齐次线性方程组$Ax=\mathbf{0}$有解,则它的通解为:
	\begin{equation*}
		x=(I_n-A^-A)y
	\end{equation*}
	其中$A^-$是$A$的任意一个给定的广义逆,$y$取遍$K^n$中的列向量。
\end{theorem}
\begin{proof}
	任取$y\in K^n$,有:
	\begin{equation*}
		A(I_n-A^-A)y=Ay-AA^-Ay=Ay-Ay=\mathbf{0}
	\end{equation*}
	所以对任意的$y\in K^n$,$(I_n-A^-A)y$都是$Ax=\mathbf{0}$的解。\par
	若$\eta$是$Ax=\mathbf{0}$的一个解,则:
	\begin{equation*}
		(I_n-A^-A)\eta=\eta-A^-A\eta=\eta-A^-\mathbf{0}=\eta
	\end{equation*}
	所以$Ax=\mathbf{0}$的任意一个解$x$都可以表示为$(I_n-A^-A)x$的形式。\par
	综上,$Ax=\mathbf{0}$的通解为$x=(I_n-A^-A)y$。
\end{proof}
\subsubsection{非齐次方程组解的结构}
\begin{theorem}[结构1]
	\label{theo:InhomogeneousLinearEq'sGeneralSolution1}
	若数域$K$上$n$元非齐次线性方程组$Ax=\beta$有解,则它的通解为:
	\begin{equation*}
		x=A^-\beta+(I_n-A^-A)y
	\end{equation*}
	其中$A^-$是$A$的任意一个给定的广义逆,$y$取遍$K^n$中的列向量。
\end{theorem}
\begin{proof}
	由\cref{theo:ConsistentLinearEqCondition}的充分性可知对于给定的这一$A^-$,$A^-\beta$为$Ax=\beta$的一个特解,而由\cref{theo:HomogeneousLinearEq'sGeneralSolution}可知齐次线性方程组$Ax=\mathbf{0}$的通解为$(I_n-A^-A)y$,由\cref{prop:InhomogeneousSLESolution}(3)可得$Ax=\beta$的通解为$x=A^-\beta+(I_n-A^-A)y$。
\end{proof}
\begin{theorem}[结构2]
	\label{theo:InhomogeneousLinearEq'sGeneralSolution2}
	若数域$K$上$n$元非齐次线性方程组$Ax=\beta$有解,则它的通解为:
	\begin{equation*}
		x=A^-\beta
	\end{equation*}
	$A^-$取遍$A$的所有广义逆。
\end{theorem}
\begin{proof}
	由\cref{theo:ConsistentLinearEqCondition}的充分性可知对于任意的$A^-$,$A^-\beta$都是$Ax=\beta$的解。\par
	对于$Ax=\beta$的任意一个解$y$,由\cref{theo:InhomogeneousLinearEq'sGeneralSolution1}可知存在$A$的一个广义逆$G$和$K^n$上的一个列向量$z$,使得:
	\begin{equation*}
		y=G\beta+(I_n-GA)z
	\end{equation*}
	因为$\beta\ne\mathbf{0}$,所以$\beta^H\beta\ne0$,于是存在数域$K$上的矩阵$B=z(\beta^H\beta)^{-1}\beta^H$使得$B\beta=z$,于是:
	\begin{equation*}
		y=G\beta+(I_n-GA)B\beta=[G+(I_n-GA)B]\beta
	\end{equation*}
	因为:
	\begin{align*}
		A[G+(I_n-GA)B]A
		&=AGA+A(I_n-GA)BA \\
		&=A+ABA-AGABA \\
		&=A+ABA-ABA=A
	\end{align*}
	所以$G+(I_n-GA)B$是$A$的一个广义逆,即$Ax=\beta$的任一解可以表示为$A^-\beta$。
\end{proof}
\begin{theorem}
	在数域$K$上相容线性方程组$Ax=\beta$的解集中,$x_0=A^+\beta$为长度最小者。
\end{theorem}
\begin{proof}
	由\cref{theo:InhomogeneousLinearEq'sGeneralSolution1}可知,$Ax=\beta$的通解可以表示为:
	\begin{equation*}
		x=A^+\beta+(I-A^+A)y
	\end{equation*}
	于是:
	\begin{align*}
		||x||
		&=[A^+\beta+(I-A^+A)y]^H[A^+\beta+(I-A^+A)y] \\
		&=||x_0||+\beta^H(A^+)^H(I-A^+A)y \\
		&\quad+y^H(I-A^+A)^HA^+\beta+y^H(I-A^+A)^H(I-A^+A)y \\
		&=||x_0||+2\beta^H(A^+)^H(I-A^+A)y+||(I-A^+A)y||
	\end{align*}
	由\cref{prop:A+}(9)可得:
	\begin{align*}
		(A^+)^H(I-A^+A)
		&=(A^+)^H-(A^+)^HA^+A=(A^H)^+-(A^H)^+A^+A \\
		&=(A^H)^+-[A(A^H)]^+A=\mathbf{0}
	\end{align*}
	于是有$2\beta^H(A^+)^H(I-A^+A)y=0$。因为$||(I-A^+A)y||\geqslant0$,等号成立当且仅当$(I-A^+A)y=\mathbf{0}$,所以$x=A^+\beta=x_0$时长度最小。
\end{proof}
\section{相似的应用}

\subsection{特征值与特征向量}
\begin{definition}
	$A\in M_{n}(K)$。如果$K^n$中存在非零列向量$\alpha$,使得:
	\begin{equation*}
		A\alpha=\lambda\alpha,\;\lambda\in K
	\end{equation*}
	则称$\lambda$是$A$的一个\gls{Eigenvalue},$\alpha$是$A$属于特征值$\lambda$的一个\gls{Eigenvector}。
\end{definition}
\subsubsection{求解特征值与特征向量}
\begin{definition}
	$A\in M_{n}(K)$,称$|\lambda I-A|$为$A$的\gls{CharacteristicPolynomial}。
\end{definition}
\begin{theorem}
	$A\in M_{n}(K)$,则:
	\begin{enumerate}
		\item $\lambda$是$A$的一个特征值当且仅当$\lambda$是$A$的特征多项式在数域$K$中的一个根;
		\item $\alpha$是$A$属于特征值$\lambda$的一个特征向量当且仅当$\alpha$是齐次线性方程组$(\lambda I-A)x=\mathbf{0}$的一个非零解。
	\end{enumerate}
\end{theorem}
\begin{proof}
	显然:
	\begin{align*}
		&\lambda\text{是}A\text{的一个特征值,}\alpha\text{是}A\text{属于}\lambda\text{的一个特征向量} \\		\iff&A\alpha=\lambda\alpha,\;\alpha\ne\mathbf{0},\;\lambda\in K \\
		\iff&(\lambda I-A)\alpha=\mathbf{0},\;a\ne\mathbf{0},\;\lambda\in K \\
		\iff&\alpha\text{是齐次线性方程组}(\lambda I-A)x=\mathbf{0}\text{的一个非零解,}\lambda\in K \\		\iff&|\lambda I-A|=0,\;\alpha\text{是齐次线性方程组}(\lambda I-A)x=\mathbf{0}\text{的一个非零解,}\lambda\in K \\
		\iff&\lambda\text{是多项式}|\lambda I-A|\text{在}K\text{中的一个根,} \\
		&\alpha\text{是齐次线性方程组}(\lambda I-A)x=\mathbf{0}\text{的一个非零解,}\lambda\in K\qedhere
	\end{align*}
\end{proof}
\subsubsection{特征向量的性质}
\begin{property}\label{prop:Eigenvector}
	特征向量具有如下性质:
	\begin{enumerate}
		\item 设$\lambda$是$A\in M_{n}(K)$的一个特征值,则$A$属于$\lambda$的所有特征向量构成$K^n$的一个子空间。因此,把齐次线性方程组$(\lambda I-A)x=\mathbf{0}$的解空间称为$A$属于$\lambda$的\gls{Eigenspace},记为$W_{\lambda}$;
		\item $A\in M_{n}(K)$的属于不同特征值的特征向量是线性无关的;
		\item 设$A\in M_{m\times n}(K),\;B\in M_{n\times m}(K)$,则$AB$和$BA$具有相同的非零特征值;
	\end{enumerate}
\end{property}
\begin{proof}
	(1)任取$k_1,k_2\in K$和$A$属于特征值$\lambda$的两个特征向量$\alpha,\beta$,则
	\begin{equation*}
		A(k_1\alpha+k_2\beta)=k_1A\alpha+k_2A\beta=k_1\lambda\alpha+k_2\lambda\beta=\lambda(k_1\alpha+k_2\beta)
	\end{equation*}
	于是$k_1\alpha+k_2\beta$也是$A$属于特征值$\lambda$的特征向量。由\cref{theo:Subspace}可知$A$属于$\lambda$的所有特征向量构成$K^n$的一个子空间。\par
	(2)我们来证明:设$\seq{\lambda}{m}$是$A\in M_{n}(K)$的不同的特征值,$a_{j1},a_{j2},\dots,a_{jr_j}$是$A$属于$\lambda_j$的线性无关的特征向量,$j=1,2,\dots,m$,则向量组:
	\begin{equation*}
		a_{11},a_{12},\dots,a_{1r_1},a_{21},a_{22},\dots,a_{2r_2},a_{m1},a_{m2},\dots,a_{mr_m}
	\end{equation*}
	线性无关。\par
	\textbf{1.证明对$n=2$成立:}对于$\lambda_1$和$\lambda_2$的线性无关的特征向量$a_{11},a_{12},\dots,a_{1r_1}$和$a_{21},a_{22},\dots,a_{2r_2}$,设:
	\begin{equation*}
		k_1a_{11}+k_2a_{12}+\cdots+k_{r_1}a_{1r_1}+l_1a_{21}+l_2a_{22}+\cdots+l_{r_2}a_{2r_2}=\mathbf{0}
	\end{equation*}
	两边同乘$A$可得:
	\begin{gather*}
		k_1Aa_{11}+k_2Aa_{12}+\cdots+k_{r_1}Aa_{1r_1}+l_1Aa_{21}+l_2Aa_{22}+\cdots+l_{r_2}Aa_{2r_2}=\mathbf{0} \\
		k_1\lambda_1a_{11}+k_2\lambda_1a_{12}+\cdots+k_{r_1}\lambda_1a_{1r_1}+l_1\lambda_2a_{21}+l_2\lambda_2a_{22}+\cdots+l_{r_2}\lambda_2a_{2r_2}=\mathbf{0}
	\end{gather*}
	因为$\lambda_1\ne\lambda_2$,所以$\lambda_1,\lambda_2$不全为$0$。设$\lambda_2\ne0$,在上上上个式子两端乘以$\lambda_2$(若$\lambda_2=0$,则同乘$\lambda_1$)得:
	\begin{equation*}
		k_1\lambda_2a_{11}+k_2\lambda_2a_{12}+\cdots+k_{r_1}\lambda_2a_{1r_1}+l_1\lambda_2a_{21}+l_2\lambda_2a_{22}+\cdots+l_{r_2}\lambda_2a_{2r_2}=\mathbf{0}
	\end{equation*}
	于是:
	\begin{equation*}
		k_1(\lambda_1-\lambda_2)a_{11}+k_2(\lambda_1-\lambda_2)a_{12}+\cdots+k_{r_1}(\lambda_1-\lambda_2)a_{1r_1}=\mathbf{0}
	\end{equation*}
	因为$\lambda_1\ne\lambda_2$,所以:
	\begin{equation*}
		k_1a_{11}+k_2a_{12}+\cdots+k_{r_1}a_{1r_1}=\mathbf{0}
	\end{equation*}
	因为$a_{11},a_{12},\dots,a_{1r_1}$线性无关,所以$k_1=k_2=\cdots=k_{r_1}=0$,从而:
	\begin{equation*}
		l_1a_{21}+l_2a_{22}+\cdots+l_{r_2}a_{2r_2}=\mathbf{0}
	\end{equation*}
	因为$a_{21},a_{22},\dots,a_{2r_2}$线性无关,所以$l_1=l_2=\cdots=l_{r_2}=0$。\par
	综上,向量组$a_{11},a_{12},\dots,a_{1r_1},a_{21},a_{22},\dots,a_{2r_2}$线性无关。\par
	\textbf{2.归纳假设:}假设对$n$个不同的特征值都有上述结论(即$n$个不同特征值的线性无关的特征向量构成的向量组线性无关),下面来证明对$n+1$个不同的特征值也成立。\par
	设:
	\begin{equation*}
		k_{11}a_{11}+k_{12}a_{12}+\cdots k_{1r_1}a_{1r_1}+\cdots+k_{nr_n}a_{nr_n}+l_1a_{(n+1)1}+l_2a_{(n+1)2}+\cdots+l_{r_{n+1}}a_{(n+1)r_{n+1}}=\mathbf{0}
	\end{equation*}
	两边同乘$A$可得:
	\begin{align*}
		&k_{11}Aa_{11}+k_{12}Aa_{12}+\cdots k_{1r_1}Aa_{1r_1}+\cdots+k_{nr_n}Aa_{nr_n} \\
		+&l_1Aa_{(n+1)1}+l_2Aa_{(n+1)2}+\cdots+l_{r_{n+1}}Aa_{(n+1)r_{n+1}}=\mathbf{0} \\
		&k_{11}\lambda_1a_{11}+k_{12}\lambda_1a_{12}+\cdots k_{1r_1}\lambda_1a_{1r_1}+\cdots+k_{nr_n}\lambda_na_{nr_n} \\
		+&l_1\lambda_{n+1}a_{(n+1)1}+l_2\lambda_{n+1}a_{(n+1)2}+\cdots+l_{r_{n+1}}\lambda_{n+1}a_{(n+1)r_{n+1}}=\mathbf{0}
	\end{align*}\par
	\textbf{2.1.$\lambda_{n+1}\ne0$:}若$\lambda_{n+1}\ne0$,则在上上上式两边同乘$\lambda_{n+1}$可得:
	\begin{align*}
		&k_{11}\lambda_{n+1}a_{11}+k_{12}\lambda_{n+1}a_{12}+\cdots k_{1r_1}\lambda_{n+1}a_{1r_1}+\cdots+k_{nr_n}\lambda_{n+1}a_{nr_n} \\
		+&l_1\lambda_{n+1}a_{(n+1)1}+l_2\lambda_{n+1}a_{(n+1)2}+\cdots+l_{r_{n+1}}\lambda_{n+1}a_{(n+1)r_{n+1}}=\mathbf{0}
	\end{align*}
	于是有:
	\begin{equation*}
		k_{11}(\lambda_{n+1}-\lambda_1)a_{11}+k_{12}(\lambda_{n+1}-\lambda_1)a_{12}+\cdots k_{1r_1}(\lambda_{n+1}-\lambda_1)a_{1r_1}+\cdots+k_{nr_n}(\lambda_{n+1}-\lambda_n)a_{nr_n}=\mathbf{0}
	\end{equation*}
	由归纳假定$a_{11},a_{12},\dots,a_{1r_1},\dots,a_{nr_n}$线性无关,所以
	\begin{equation*}
		k_{11}(\lambda_{n+1}-\lambda_1)=k_{12}(\lambda_{n+1}-\lambda_1)=\cdots=k_{1r_1}(\lambda_{n+1}-\lambda_1)=\cdots=k_{nr_n}(\lambda_{n+1}-\lambda_n)=0
	\end{equation*}
	因为$\lambda_i,\;i=1,2,\dots,n$之间互不相同,所以$\lambda_{n+1}-\lambda_1,\lambda_{n+1}-\lambda_2,\dots,\lambda_{n+1}-\lambda_n$不为$0$,于是$k_{11}=k_{12}=\cdots=k_{1r_1}=\cdots=k_{nr_n}=0$,所以:
	\begin{equation*}
		l_1a_{(n+1)1}+l_2a_{(n+1)2}+\cdots+l_{r_{n+1}}a_{(n+1)r_{n+1}}=\mathbf{0}
	\end{equation*}
	因为$a_{(n+1)1},a_{(n+1)2},\dots,a_{(n+1)r_{n+1}}$线性无关,所以有$l_1=l_2=\cdots=l_{r_{n+1}}=0$。\par
	综上$a_{11},a_{12},\dots,a_{1r_1},\dots,a_{nr_n},a_{(n+1)1},a_{(n+1)2},\dots,a_{(n+1)r_{n+1}}$线性无关。\par
	\textbf{2.2.$\lambda_{n+1}=0$:}若$\lambda_{n+1}=0$,则此时有:
	\begin{gather*}
		k_{11}\lambda_1a_{11}+k_{12}\lambda_1a_{12}+\cdots k_{1r_1}\lambda_1a_{1r_1}+\cdots+k_{nr_n}\lambda_na_{nr_n} \\
		+l_1\lambda_{n+1}a_{(n+1)1}+l_2\lambda_{n+1}a_{(n+1)2}+\cdots+l_{r_{n+1}}\lambda_{n+1}a_{(n+1)r_{n+1}} \\
		=k_{11}\lambda_1a_{11}+k_{12}\lambda_1a_{12}+\cdots k_{1r_1}\lambda_1a_{1r_1}+\cdots+k_{nr_n}\lambda_na_{nr_n}=\mathbf{0}
	\end{gather*}
	由归纳假定$a_{11},a_{12},\dots,a_{1r_1},\dots,a_{nr_n}$线性无关,所以$k_{11}\lambda_1=k_{12}\lambda_1=\cdots=k_{1r_1}\lambda_1=\cdots=k_{nr_n}\lambda_n=0$。因为$\seq{\lambda}{n}$都不是$0$($\lambda_i,\;i=1,2,\dots,n+1$互不相同,已经有$\lambda_{n+1}=0$了),所以$k_{11}=k_{12}=\cdots=k_{1r_1}=\cdots=k_{nr_n}=0$,于是有:
	\begin{equation*}
		l_1a_{(n+1)1}+l_2a_{(n+1)2}+\cdots+l_{r_{n+1}}a_{(n+1)r_{n+1}}=\mathbf{0}
	\end{equation*}
	因为$a_{(n+1)1},a_{(n+1)2},\dots,a_{(n+1)r_{n+1}}$线性无关,所以有$l_1=l_2=\cdots=l_{r_{n+1}}=0$。\par
	综上,$a_{11},a_{12},\dots,a_{1r_1},\dots,a_{nr_n},a_{(n+1)1},a_{(n+1)2},\dots,a_{(n+1)r_{n+1}}$线性无关。\par
	假设存在属于不同特征值的特征向量$\alpha_1,\alpha_2,\dots,\alpha_m$线性相关,则有:
	\begin{equation*}
		k_1\alpha_1+k_2\alpha_2+\cdots+k_m\alpha_m=\mathbf{0}
	\end{equation*}
	其中$k_1,k_2,\dots,k_m$不全为$0$。注意到$\alpha_i,\;i=1,2,\dots,m$可由其对应特征值的特征子空间中的一组基线性表出,于是有:
	\begin{equation*}
		\alpha_i=\sum_{n=1}^{r_i}l_n\beta_{in}
	\end{equation*}
	其中$\beta_{in},\;n=1,2,\dots,r_i$为$\alpha_i$对应特征值的特征子空间的一组基,所以:
	\begin{equation*}
		\sum_{i=1}^{m}k_i\sum_{n=1}^{r_i}l_n\beta_{in}=	\sum_{i=1}^{m}\sum_{n=1}^{r_i}k_il_n\beta_{in}=\mathbf{0}
	\end{equation*}
	而$\beta_{in},\;i=1,2,\dots,m,\;n=1,2,\dots,r_i$是线性无关的,所以:
	\begin{equation*}
		k_il_n=0,\;\forall\;i=1,2,\dots,m,\;n=1,2,\dots,r_i
	\end{equation*}
	因为$k_1,k_2,\dots,k_m$不全为$0$,所以存在一组$l_n$全为$0$,于是$\alpha_i$中存在零向量,而特征向量不是零向量,矛盾。\par
	(3)设$\lambda$是$AB$的一个非零特征值,$\alpha$是其对应的特征向量,则有$AB\alpha=\lambda\alpha$,$B\alpha\ne0$,否则$\alpha=\mathbf{0}$。注意到$BA(B\alpha)=B\lambda\alpha=\lambda(B\alpha)$,所以$\lambda$是$BA$的一个非零特征值。反之可得$BA$的特征值也是$AB$的特征值,于是结论成立。
\end{proof}
\begin{theorem}\label{theo:SameEigenvalue}
	相似的矩阵有相同的特征多项式,进而有相同的特征值(包括重数相同)。
\end{theorem}
\begin{proof}
	设$A,B\in M_{n}(K)$且$A$与$B$相似,于是存在可逆矩阵$P\in M_{n}(K)$使得$P^{-1}AP=B$,由\cref{prop:Determinant}(11)就有:
	\begin{align*}
		|\lambda I-B|
		&=|\lambda I-P^{-1}AP|=|P^{-1}\lambda IP-P^{-1}AP| \\
		&=|P^{-1}(\lambda I-A)P|=|P^{-1}|\;|\lambda I-A|\;|P|=|\lambda I-A|\qedhere
	\end{align*}
\end{proof}
\subsubsection{几何重数与代数重数}
\begin{definition}
	$A\in M_{n}(K)$,$\lambda$是$A$的一个特征值。把$A$属于$\lambda$的特征子空间的维数叫作$\lambda$的\gls{GeometricMultiplicity},把$\lambda$作为$A$的特征多项式的根的重数叫作$\lambda$的\gls{AlgebraicMultiplicity}。
\end{definition}
\begin{theorem}\label{theo:AlgebraicMultiplicityGeometricMultiplicity}
	$A\in M_{n}(K)$,$\lambda_1$是$A$的一个特征值,则$\lambda_1$的几何重数不超过它的代数重数。
\end{theorem}
\begin{proof}
	设$A$属于特征值$\lambda_1$的特征子空间$W_1$的维数为$r$。在$W_1$中取一组基$\alpha_1,\alpha_2,\dots,\alpha_r$,把它扩充为$K^n$的一组基$\alpha_1,\alpha_2,\dots,\alpha_r,\beta_1,\beta_2,\dots,\beta_{n-r}$。令:
	\begin{equation*}
		P=(\alpha_1,\alpha_2,\dots,\alpha_r,\beta_1,\beta_2,\dots,\beta_{n-r})
	\end{equation*}
	则$P$是数域$K$上的$n$阶可逆矩阵,并且有:
	\begin{align*}
		P^{-1}AP
		&=P^{-1}(A\alpha_1,A\alpha_2,\dots,A\alpha_r,A\beta_1,A\beta_2,\dots,A\beta_{n-r}) \\
		&=P^{-1}(\lambda_1\alpha_1,\lambda_1\alpha_2,\dots,\lambda_1\alpha_r,A\beta_1,A\beta_2,\dots,A\beta_{n-r}) \\
		&=(\lambda_1\varepsilon_1,\lambda_1\varepsilon_2,\dots,\lambda_1\varepsilon_r,P^{-1}A\beta_1,P^{-1}A\beta_2,\dots,P^{-1}A\beta_{n-r}) \\
		&=
		\begin{pmatrix}
			\lambda_1I_r & B \\
			\mathbf{0} & C
		\end{pmatrix}
	\end{align*}
	由\cref{theo:SameEigenvalue}和\cref{prop:Determinant}(10)可得:
	\begin{align*}
		|\lambda I-A|&=
		\begin{vmatrix}
			\lambda I_r-\lambda_1I_r & -B \\
			\mathbf{0} & \lambda I_{n-r}-C
		\end{vmatrix} \\
		&=|\lambda I_r-\lambda_1I_r|\;|\lambda I_{n-r}-C| \\
		&=(\lambda-\lambda_1)^r|\lambda I_{n-r}-C|
	\end{align*}
	即$\lambda_1$的几何重数小于或等于$r$,也即$\lambda_1$的几何重数小于或等于它的代数重数。
\end{proof}

\subsection{矩阵的对角化}
\begin{definition}
	如果$n$阶矩阵$A$能够相似于一个对角矩阵,那么称$A$\gls{Diagonalizable}。
\end{definition}
研究矩阵是否可对角化是为了计算矩阵的幂,因为对角矩阵的幂是很好计算的。
\begin{theorem}\label{theo:DiagCondition}
	$A\in M_{n}(K)$可对角化的充分必要条件为
	\begin{enumerate}
		\item $A$有$n$个线性无关的特征向量$\seq{\alpha}{n}$;
		\item $A$的属于不同特征值的特征子空间的维数之和等于$n$;
		\item $A$的特征多项式的全部复根都属于$K$,且$A$的每个特征值的几何重数等于它的代数重数;
		\item 设$\seq{\lambda}{m}$是$A$全部的不同的特征值,则:
		\begin{equation*}
			K^n=W_{\lambda_1}\oplus W_{\lambda_2}\oplus\cdots\oplus W_{\lambda_m}
		\end{equation*}
	\end{enumerate}
	此时令$P=(\seq{\alpha}{n})$,则:
	\begin{equation*}
		P^{-1}AP=\operatorname{diag}\{\seq{\lambda}{n}\}
	\end{equation*}
	其中$\lambda_i$是$\alpha_i$所属的特征值,$i=1,2,\dots,n$。上述对角矩阵称为$A$的\textbf{相似标准形},除了主对角线上元素的排列次序外,$A$的相似标准形是唯一的;
\end{theorem}
\begin{proof}
	(1)显然:
	\begin{align*}
		&A\text{与}\text{对角矩阵}D=\operatorname{diag}\{\seq{\lambda}{n}\}\text{相似},\;\text{其中}\lambda_i\in K,\;i=1,2,\dots,n \\
		\iff&\text{存在可逆矩阵$P\in M_{n}(K)$,使得}P^{-1}AP=D \\
		&\text{即}AP=PD \\
		&\text{即}A(\seq{\alpha}{n})=(\seq{\alpha}{n})D \\
		&\text{即}(A\alpha_1,A\alpha_2,\dots,A\alpha_n)=(\lambda_1\alpha_1,\lambda_2\alpha_2,\dots,\lambda_n\alpha_n) \\
		\iff&K^{n}\text{中有$n$个线性无关的列向量}\seq{\alpha}{n}\text{使得}A\alpha_i=\lambda_i\alpha_i,\;i=1,2,\dots,n\qedhere
	\end{align*}\par
	(2)\textbf{充分性:}由\cref{prop:Eigenvector}(2)和(1)的充分性可直接得出。\par
	\textbf{必要性:}设$A$的所有不同的特征值是$\seq{\lambda}{m}$,它们的几何重数分别为$r_1,r_2,\dots,r_m$。若此时$A$的属于不同特征值的特征子空间的维数之和不等于$n$,由\cref{theo:AlgebraicMultiplicityGeometricMultiplicity}可知此时$r_1+r_2+\cdots+r_m<n$,那么$A$没有$n$个线性无关的特征向量,由(1)的必要性可得$A$不可以对角化。\par
	(3)\textbf{充分性:}由(2)的充分性可直接得到。\par
	\textbf{必要性:}因为$A$可对角化,由可对角化的定义可知$A$相似于:
	\begin{equation*}
		\operatorname{diag}(\lambda_1,\cdots,\lambda_1,\dots,\lambda_m,\dots,\lambda_m)\in M_{n}(K)
	\end{equation*}
	其中$\seq{\lambda}{m}$是$A$的全部不同的特征值,每个特征值重复的次数为对应特征子空间的维数,$\lambda_i$对应特征子空间的维数记为$r_i,\;i=1,2,\dots,m$。因为相似的矩阵具有相同的特征多项式,所以:
	\begin{equation*}
		|\lambda I-A|=(\lambda-\lambda_1)^{r_1}(\lambda-\lambda_2)^{r_2}\cdots(\lambda-\lambda_m)^{r_m}
	\end{equation*}
	于是$A$的特征多项式的根为$\seq{\lambda}{m}$。因为$\operatorname{diag}(\lambda_1,\cdots,\lambda_1,\dots,\lambda_m,\dots,\lambda_m)\in M_{n}(K)$,所以$\seq{\lambda}{m}\in K$,于是$A$的特征多项式的全部根都属于$K$且每一个特征值的代数重数等于它的几何重数。\par
	(4)由(2)、\cref{prop:Eigenvector}(2)、\cref{prop:nDimensionalLinearSpace}和\cref{theo:DirectSum}(5)可得:
	\begin{align*}
		A\text{可对角化}
		\iff&\sum_{i=1}^{m}\dim(W_{\lambda_i})=n \\
		\iff&W_{\lambda_i},i=1,2,\dots,m\text{的基合起来是$n$个线性无关的向量} \\
		\iff&W_{\lambda_i},i=1,2,\dots,m\text{的基合起来是$K^n$的一组基} \\
		\iff&K^n=W_{\lambda_1}\oplus W_{\lambda_2}\oplus\cdots\oplus W_{\lambda_m}\qedhere
	\end{align*}
\end{proof}

\subsection{Hermitian矩阵的对角化}
\begin{definition}
	若对于$A,B\in M_{n}(\mathbb{C})$,存在一个$n$阶正交(酉)矩阵$Q$,使得$Q^{-1}AQ=B$,则称$A$\textbf{正交(酉)相似}于$B$。
\end{definition}
\begin{property}\label{prop:OrthogonalUnitarySimilarity}
	正交相似与酉相似具有如下性质:
	\begin{enumerate}
		\item 正交(酉)相似是$M_{n}(\mathbb{C})$上的一个等价关系;
		\item $A\in\ M_n(\mathbb{R})$($\mathbb{C}^{}$)。若$A$正交(酉)相似于一个对角矩阵$D$,则$A$一定是对称(Hermitian)矩阵;
		\item 正交(酉)相似一定相似,相似不一定正交(酉)相似。
	\end{enumerate}
\end{property}
\begin{proof}
	(1)由\cref{prop:OrthogonalUnitaryMatrix}(4)(5)可得。\par
	(2)仅证明酉相似时的情况,正交相似可看作酉相似的特例。因为$A$酉相似于$D$,所以存在酉矩阵$Q$使得$Q^{-1}AQ=D$,即$A=QDQ^{-1}$,于是由\cref{prop:OrthogonalUnitaryMatrix}(4)和\cref{prop:Transpose}(2)可得:
	\begin{equation*}
		A^H=(QDQ^{-1})^H=(Q^{-1})^HD^HQ^H=(Q^H)^HDQ^{-1}=QDQ^{-1}=A
	\end{equation*}
	所以$A$是一个Hermitian矩阵。\par
	(3)可举例论证。
\end{proof}
\begin{property}\label{prop:HermitianMatEigen}
	设Hermitian矩阵$A,B\in M_{n}(\mathbb{C})$(对称矩阵$A,B\in M_{n}(\mathbb{R})$),则:
	\begin{enumerate}
		\item $A$的特征多项式的每一个根都是实数,从而都是$A$的特征值;
		\item $A$属于不同特征值的特征向量是正交的;
		\item $A$一定酉(正交)相似于由它的特征值构成的对角矩阵;
		\item $A$与$B$酉(正交)相似的充分必要条件为$A$与$B$相似。
	\end{enumerate}
\end{property}
\begin{proof}
	(1)设$\lambda$是$A$的特征多项式的任意一个根,将$A$看作是复数域$\mathbb{C}$上的矩阵,取$A$属于特征值$\lambda$的一个特征向量$\alpha$,考虑$\mathbb{C}^{n}$中的内积,由\cref{prop:Transpose}(4)可得:
	\begin{gather*}
		(A\alpha,\alpha)=(\lambda\alpha,\alpha)=\lambda(\alpha,\alpha) \\
		(\alpha,A\alpha)=(\alpha,\lambda\alpha)=\overline{\lambda}(\alpha,\alpha) \\
		(A\alpha,\alpha)=(A\alpha)^H\alpha=\alpha^HA^H\alpha=\alpha^H A\alpha=(\alpha,A\alpha)
	\end{gather*}
	所以$\lambda(\alpha,\alpha)=\overline{\lambda}(\alpha,\alpha)$。因为$\alpha$是特征向量,所以$\alpha\ne\mathbf{0}$,于是$\lambda=\overline{\lambda}$,因此$\lambda$是一个实数。由$\lambda$的任意性,结论成立。\par
	(2)设$\lambda_1,\lambda_2$是$A$的不同的特征值(由(1)得它们都是实数),$\alpha_1,\alpha_2$分别是$A$属于$\lambda_1,\lambda_2$的一个特征向量,考虑$\mathbb{C}^{n}$上的标准内积:
	\begin{align*}
		&\quad\lambda_1(\alpha_1,\alpha_2)
		=(\lambda_1\alpha_1,\alpha_2)=(A\alpha_1,\alpha_2)=(\alpha_1,A^H\alpha_2) \\
		&=(\alpha_1,A\alpha_2)=(\alpha_1,\lambda_2\alpha_2)=\overline{\lambda_2}(\alpha_1,\alpha_2)=\lambda_2(\alpha_1,\alpha_2)
	\end{align*}
	于是有$(\lambda_1-\lambda_2)(\alpha_1,\alpha_2)=0$。因为$\lambda_1\ne\lambda_2$,所以$(\alpha_1,\alpha_2)=0$。\par
	(3)对$n$作数学归纳法。\par
	当$n=1$时,$(1)^{-1}A(1)=A$,结论成立。\par
	假设对于$n-1$阶的实对称矩阵都成立,考虑$n$阶实对称矩阵$A$。\par
	由(2)可知$A$必有特征值,取$A$的一个特征值$\lambda_1$和$A$属于$\lambda_1$的一个特征向量$\eta_1$,满足$||\eta_1||=1$。把$\eta_1$扩充为$\mathbb{C}^{n}$的一组基并进行Schimidt正交化和单位化,可得到$\mathbb{C}^{n}$的一个标准正交基$\eta_1,\eta_2,\dots,\eta_n$。令:
	\begin{equation*}
		Q_1=(\eta_1,\eta_2,\dots,\eta_n)
	\end{equation*}
	显然$Q_1$是一个正交矩阵,于是有$Q_1^{-1}Q_1=(Q_1^{-1}\eta_1,Q_1^{-1}\eta_2,\dots,Q_1^{-1}\eta_n)=(e_1,e_2,\dots,e_n)$。注意到:
	\begin{equation*}
		Q_1^{-1}AQ_1=Q_1^{-1}(A\eta_1,A\eta_2,\dots,A\eta_n)=(Q_1^{-1}\lambda\eta_1,Q_1^{-1}A\eta_2,\dots,Q_1^{-1}A\eta_n)
		=
		\begin{pmatrix}
			\lambda_1 & \alpha \\
			\mathbf{0} & B
		\end{pmatrix}
	\end{equation*}
	因为$(Q_1^{-1}AQ_1)^H=Q_1^HA^H(Q_1^{-1})^H=Q_1^{-1}A(Q_1^H)^H=Q_1^{-1}AQ_1$,所以$Q_1^{-1}AQ_1$是一个对称阵,于是$\alpha=\mathbf{0}$,$B$是一个$n-1$阶Hermitian矩阵。由归纳假设,存在$n-1$阶正交矩阵$Q_2$使得$Q_2^{-1}BQ_2=\operatorname{diag}\{\lambda_2,\lambda_3,\dots,\lambda_n\}$。令:
	\begin{equation*}
		Q=Q_1
		\begin{pmatrix}
			1 & \mathbf{0} \\
			\mathbf{0} & Q_2
		\end{pmatrix}
	\end{equation*}
	则:
	\begin{equation*}
		Q^HQ=
		\begin{pmatrix}
			1 & \mathbf{0} \\
			\mathbf{0} & Q_2^H
		\end{pmatrix}
		Q_1^HQ_1
		\begin{pmatrix}
			1 & \mathbf{0} \\
			\mathbf{0} & Q_2
		\end{pmatrix}
		=I
	\end{equation*}
	即$Q$是一个正交矩阵。同时:
	\begin{align*}
		Q^{-1}AQ
		&=
		\begin{pmatrix}
			1 & \mathbf{0} \\
			\mathbf{0} & Q_2^H
		\end{pmatrix}
		Q_1^HAQ_1
		\begin{pmatrix}
			1 & \mathbf{0} \\
			\mathbf{0} & Q_2
		\end{pmatrix}
		=
		\begin{pmatrix}
			1 & \mathbf{0} \\
			\mathbf{0} & Q_2^H
		\end{pmatrix}
		\begin{pmatrix}
			\lambda_1 & \mathbf{0} \\
			\mathbf{0} & B
		\end{pmatrix}
		\begin{pmatrix}
			1 & \mathbf{0} \\
			\mathbf{0} & Q_2
		\end{pmatrix} \\
		&=
		\begin{pmatrix}
			\lambda_1 & \mathbf{0} \\
			\mathbf{0} & Q_2^HBQ_2
		\end{pmatrix}
		=
		\begin{pmatrix}
			\lambda_1 & \mathbf{0} \\
			\mathbf{0} & \operatorname{diag}\{\lambda_2,\lambda_3,\dots,\lambda_n\}
		\end{pmatrix}
		=\operatorname{diag}\{\seq{\lambda}{n}\}
	\end{align*}
	所以$A$正交相似于对角矩阵$\operatorname{diag}\{\seq{\lambda}{n}\}$。由$AQ=Q\operatorname{diag}\{\seq{\lambda}{n}\}$可以得到$\lambda_2,\lambda_3,\dots,\lambda_n$是$A$的特征值。\par
	综上,结论成立。\par
	(4)\textbf{必要性:}正交相似也是相似。\par
	\textbf{充分性:}因为$A$与$B$相似,由\cref{theo:SameEigenvalue}可知$A$与$B$有相同的特征值(包括重数)$\seq{\lambda}{n}$。由(3)可得$A$与$B$都正交相似于$\operatorname{diag}\{\seq{\lambda}{n}\}$(考虑$\lambda_i$的顺序的话只需要更改$Q$中列向量的顺序)。根据\cref{prop:OrthogonalUnitarySimilarity})(1)可得$A$正交相似于$B$。
\end{proof}
\subsubsection{求解正交矩阵$Q$}
\begin{theorem}
	对于Hermitian矩阵$A\in M_{n}(\mathbb{C})$,求正交矩阵$Q$使得$Q^{-1}AQ$为对角阵的步骤如下:
	\begin{enumerate}
		\item 求出$A$的所有特征值$\seq{\lambda}{m}$;
		\item 对于每一个特征值$\lambda_i$,求得其特征子空间的一组基$\alpha_{1i},\alpha_{2i},\dots,\alpha_{r_ii}$,并对它们进行Schimidt正交化与单位化,得到$\eta_{1i},\eta_{2i},\dots,\eta_{r_ii}$;
		\item 令$Q=(\eta_{11},\eta_{21},\dots,\eta_{r_11},\dots,\eta_{r_mm})$,$Q$即为所求。
	\end{enumerate}
\end{theorem}
\begin{proof}
	由\info{Schimidt正交化链接}可知$\eta_{ij},\;i=1,2,\dots,r_j,\;j=1,2,\dots,m$是$A$属于$\lambda_j$的特征值。根据\cref{prop:HermitianMatEigen}(2)可知:
	\begin{align*}
		Q^{-1}AQ
		&=Q^H(A\eta_{11},A\eta_{21},\dots,A\eta_{r_11},\dots,A\eta_{r_mm}) \\
		&=
		\begin{pmatrix}
			\eta_{11}^H \\
			\eta_{21}^H \\
			\vdots \\
			\eta_{r_11}^H \\
			\vdots \\
			\eta_{r_mm}^H
		\end{pmatrix}
		(\lambda_1\eta_{11},\lambda_1\eta_{21},\dots,\lambda_1\eta_{r_11},\dots,\lambda_m\eta_{r_mm}) \\
		&=\operatorname{diag}\{\lambda_1\eta_{11}^H\eta_{11},\lambda_1\eta_{21}^H\eta_{21},\dots,\lambda_1\eta_{r_11}^H\eta_{r_11},\dots,\lambda_m\eta_{r_mm}^H\eta_{r_mm}\} \\
		&=\operatorname{diag}\{\lambda_1,\dots,\lambda_1,\dots,\lambda_m,\dots,\lambda_m\}\qedhere
	\end{align*}
\end{proof}
\subsubsection{实对称矩阵特征值的极值性质}
\begin{theorem}\label{theo:maxminxAx/xx}
	设$A\in M_{n}(\mathbb{R})$,$A$的特征值从大到小记作$\seq{\lambda}{n}$,$\seq{\varphi}{n}$为对应的标准正交化特征向量,则:
	\begin{equation*}
		\max_{x\ne\mathbf{0}}\frac{x^TAx}{x^Tx}=\lambda_1=\varphi_1^TA\varphi_1\quad
		\min_{x\ne\mathbf{0}}\frac{x^TAx}{x^Tx}=\lambda_n=\varphi_n^TA\varphi_n
	\end{equation*}
\end{theorem} 
\begin{proof}
	由\cref{prop:HermitianMatEigen}(3)可知存在一个正交矩阵$Q$使得$Q^{-1}AQ=\operatorname{diag}\{\seq{\lambda}{n}\}=\varLambda$。对任意的$x\in\mathbb{R}^{n}$,因为$Q$为正交矩阵,$Q$可逆,所以关于$y$的非齐次线性方程组$Qy=x$有唯一解,于是对于这个存在且唯一的$y$,有:
	\begin{gather*}
		\frac{x^TAx}{x^Tx}=\frac{y^TQ^TAQy}{y^TQ^TQy}=\frac{y^T\varLambda y}{y^Ty}=\frac{\sum\limits_{i=1}^{n}\lambda_iy_i^2}{\sum\limits_{i=1}^ny_i^2}\leqslant\lambda_1\frac{\sum\limits_{i=1}^{n}y_i^2}{\sum\limits_{i=1}^ny_i^2}=\lambda_1 \\
		\frac{x^TAx}{x^Tx}=\frac{y^TQ^TAQy}{y^TQ^TQy}=\frac{y^T\varLambda y}{y^Ty}=\frac{\sum\limits_{i=1}^{n}\lambda_iy_i^2}{\sum\limits_{i=1}^ny_i^2}\geqslant\lambda_n\frac{\sum\limits_{i=1}^{n}y_i^2}{\sum\limits_{i=1}^ny_i^2}=\lambda_n
	\end{gather*}
	当$y$为$(1,0,0,\dots,0)^T$时第一式取等号,当$y$为$(0,0,\dots,0,1)^T$时第二式取等号,此时$x$分别为$\varphi_1$和$\varphi_n$。
\end{proof}

\section{合同的应用——二次型}
\begin{definition}
	数域$K$上的一个$n$元\gls{QuadraticForm}是系数在$K$中的$n$个变量的二元齐次多项式,它的一般形式为:
	\begin{equation*}
		f(x_1,x_2,\dots,x_n)=\sum_{i=1}^{n}\sum_{j=1}^{n}a_{ij}x_ix_j
	\end{equation*}
	其中$a_{ij}=a_{ji},\;1\leqslant i,j\leqslant n$。矩阵:
	\begin{equation*}
		A=
		\begin{pmatrix}
			a_{11} & a_{12} & \cdots & a_{1n} \\
			a_{12} & a_{22} & \cdots & a_{2n} \\
			\vdots & \vdots & \ddots & \vdots \\
			a_{1n} & a_{2n} & \cdots & a_{nn}
		\end{pmatrix}
	\end{equation*}
	被称为二次型$f(x_1,x_2,\dots,x_n)$的矩阵,它是一个对称矩阵,主对角元依次是$x_1^2,x_2^2,\dots,x_n^2$的系数,$(i,j)$元是$x_ix_j$系数的一半,其中$i\ne j$。令:
	\begin{equation*}
		x=(x_1,x_2,\dots,x_n)^T
	\end{equation*}
	则二次型$f(x_1,x_2,\dots,x_n)$可写作$x^TAx$。
\end{definition}
\begin{definition}
	令$x=(x_1,x_2,\dots,x_n)^T,\;y=(y_1,y_2,\dots,y_n)^T$,可逆矩阵$C\in M_{n}(K)$,则关系式$x=Cy$称为变量$x_1,x_2,\dots,x_n$到变量$y_1,y_2,\dots,y_n$的一个\gls{InvertibleLinearTransformation}。如果$C$是一个正交矩阵,则称变量变换$x=Cy$为一个\gls{OrthogonalTransformation}。
\end{definition}
\begin{definition}
	对于数域$K$上的两个$n$元二次型$x^TAx$与$y^TAy$,如果存在一个非退化线性变换$x=Cy$,把$x^TAx$变成$y^TBy$,那么称二次型$x^TAx$与$y^TBy$\textbf{等价},记作$x^TAx\cong y^TBy$。如果二次型$x^TAx$等价于一个只含平方项的二次型,那么称这个只含平方项的二次型是$x^TAx$的一个\textbf{标准形}。
\end{definition}
\begin{theorem}\label{theo:QuadraticEquivCongruent}
	数域$K$上两个$n$元二次型$x^TAx$与$y^TBy$等价当且仅当$n$阶对称矩阵$A$与$B$合同,于是二次型的等价也是一个等价关系。
\end{theorem}
\begin{proof}
	\textbf{(1)充分性:}因为$A\cong B$,所以存在可逆矩阵$C$使得$C^TAC=B$。作非退化线性变换$x=Cy$,可得到$(Cy)^TA(Cy)=y^TC^TACy=y^TBy$,所以$x^TAx\cong y^TBy$。\par
	\textbf{(2)必要性:}因为$x^TAx\cong y^TBy$,所以存在非退化线性变换$x=Cy$,$C$是一个可逆矩阵,把$x^TAx$变为$y^TBy$,即$(Cy)^TA(Cy)=y^TC^TACy=y^TBy$,所以$C^TAC=B$,即$A\cong B$。\par
	因为合同是一个等价关系,显然可得二次型的等价也是一个等价关系。
\end{proof}
\begin{theorem}
	数域$K$上任一$n$元二次型都等价于一个只含平方项的二次型。
\end{theorem}
\begin{proof}
	当二次型的矩阵是对角矩阵时该二次型只含平方项,由\cref{prop:CongruentMatrix}(4)与\cref{theo:QuadraticEquivCongruent}可立即得出结论。
\end{proof}
\begin{theorem}
	设$n$元二次型$x^TAx$的矩阵$A$合同于对角矩阵$D=\operatorname{diag}\{d_1,d_2,\dots,d_n\}$,即存在可逆矩阵$C$使得$C^TAC=D$。令$x=Cy$,则可以得到$x^TAx$的一个标准形:
	\begin{equation*}
		d_1y_1^2+d_2y_2^2+\cdots+d_ny_n^2
	\end{equation*}
\end{theorem}
\begin{proof}
	将$x=Cy$代入可得:
	\begin{equation*}
		x^TAx=(Cy)^TA(Cy)=y^TC^TACy=y^TDy=\sum_{i=1}^{n}d_iy_i^2\qedhere
	\end{equation*}
\end{proof}
\begin{theorem}
	数域$K$上$n$元二次型$x^TAx$的任一标准形中,系数不为$0$的平方项个数等于它的矩阵$A$的秩。
\end{theorem}
\begin{proof}
	设$n$元二次型$x^TAx$经过非退化线性变换$x=Cy$化成标准形$d_1y_1^2+d_2y_2^2+\cdots+d_ry_r^2$,其中$d_1,d_2,\dots,d_r$都不为$0$,则:
	\begin{equation*}
		C^TAC=\operatorname{diag}\{d_1,d_2,\dots,d_r,0,\dots,0\}
	\end{equation*}
	于是$\operatorname{diag}\{d_1,d_2,\dots,d_r,0,\dots,0\}$是$A$的一个合同标准形。由\cref{prop:CongruentMatrix}(3)可得$\operatorname{rank}(A)=r$。
\end{proof}
\begin{definition}
	称二次型$x^TAx$的矩阵$A$的秩为二次型$x^TAx$的秩。
\end{definition}

\subsection{二次型的规范形}
\subsubsection{实二次型的规范形}
\begin{definition}
	实数域上的二次型称为\textbf{实二次型}。由\cref{prop:CongruentMatrix}(5)可知$n$元实二次型$x^TAx$的矩阵$A$合同于一个对角矩阵$\operatorname{diag}\{1,1,\dots,1,-1,-1,\dots,-1,0,0,\dots,0\}$,再由\cref{theo:QuadraticEquivCongruent}可知经过一个适当的非退化线性变换可以将$x^TAx$化作:
	\begin{equation*}
		z_1^2+z_2^2+\cdots+z_p^2-z_{p+1}^2-z_{p+2}^2-z_r^2
	\end{equation*}
	称此形式为二次型$x^TAx$的\textbf{规范形},其特征为:只含平方项且平方项系数为$1,-1,0$,系数为$1$的平方项在最前面,系数为$-1$的平方项在中间,系数为$0$的平方项在最后。实二次型$x^TAx$的规范形被两个自然数$p$和$r$决定。
\end{definition}
\begin{theorem}[Sylvester's Law of Inertia]
	\label{theo:Sylvester'sLawOfInertia}
	$n$元实二次型$x^TAx$的规范形是唯一的。
\end{theorem}
\begin{proof}
	设$n$元实二次型$x^TAx$的秩为$r$,假设$x^TAx$分别经过非退化线性变换$x=Cy$和$x=Bz$变成两个规范形:
	\begin{gather*}
		x^TAx=y_1^2+y_2^2+\cdots+y_p^2-y_{p+1}^2-y_{p+2}^2-\cdots-y_r^2 \\
		x^TAx=z_1^2+z_2^2+\cdots+z_q^2-z_{q+1}^2-z_{q+2}^2-\cdots-z_r^2
	\end{gather*}
	要证规范形唯一,即证$p=q$。\par
	由$x=Cy$和$x=Bz$可知,经过非退化线性变换$z=(B^{-1}C)y$后有:
	\begin{equation*}
		z_1^2+z_2^2+\cdots+z_q^2-z_{q+1}^2-z_{q+2}^2-\cdots-z_r^2
		=y_1^2+y_2^2+\cdots+y_p^2-y_{p+1}^2-y_{p+2}^2-\cdots-y_r^2
	\end{equation*}
	记$D=B^{-1}C=(d_{ij})$。假设$p>q$,我们想找到变量$y_1,y_2,\dots,y_n$的一组取值,使得上式右端大于$0$,而左端小于或等于$0$,从而产生矛盾。令:
	\begin{equation*}
		y=(y_1,y_2,\dots,y_p,0,0,\dots,0)^T
	\end{equation*}
	其中$y_1,y_2,\dots,y_p$是待定的实数,使得变量$z_1,z_2,\dots,z_q$的值全为$0$。因为$z=Dy$,所以:
	\begin{equation*}
		\begin{pmatrix}
			d_{11} & d_{12} & \cdots & d_{1p} \\
			d_{21} & d_{22} & \cdots & d_{2p} \\
			\vdots & \vdots & \ddots & \vdots \\
			d_{q1} & d_{q2} & \cdots & d_{qp} \\
		\end{pmatrix}
		\begin{pmatrix}
			y_1 \\
			y_2 \\
			\vdots \\
			y_p
		\end{pmatrix}
		=
		\begin{pmatrix}
			z_1 \\
			z_2 \\
			\vdots \\
			z_q
		\end{pmatrix}
		=
		\begin{pmatrix}
			0 \\
			0 \\
			\vdots \\
			0
		\end{pmatrix}
	\end{equation*}
	因为$p>q$,所以上述齐次线性方程组有非零解,即存在非零向量$y=(y_1,y_2,\dots,y_p,0,0,\dots,0)^T$使得$z_1=z_2=\cdots=z_q=0$。此时有:
	\begin{gather*}
		z_1^2+z_2^2+\cdots+z_q^2-z_{q+1}^2-z_{q+2}^2-\cdots-z_r^2\leqslant0 \\
		y_1^2+y_2^2+\cdots+y_p^2-y_{p+1}^2-y_{p+2}^2-\cdots-y_r^2>0
	\end{gather*}
	矛盾。因此$p\leqslant q$。同理可得$q\leqslant p$,于是$p=q$,规范形唯一。
\end{proof}
\begin{definition}
	在实二次型$x^TAx$的规范形中,系数为$1$的平方项个数$p$称为$x^TAx$的正惯性指数,系数为$-1$的平方项个数$r-p$称为$x^TAx$的负惯性指数,正惯性指数减去负惯性指数所得的差$2p-r$称为$x^TAx$称为$x^TAx$的\gls{Signature}。
\end{definition}
\begin{theorem}\label{theo:RQuadraticFormEquiv}
	两个$n$元实二次型等价\par
	$\iff$它们的规范形相同\par
	$\iff$它们的秩相等,并且正惯性指数也相等。
\end{theorem}
\begin{proof}
	第一条由\cref{theo:Sylvester'sLawOfInertia}以及二次型等价的传递性、对称性可直接得到(必要性的证明中需要考虑规范形的定义,然后使用\cref{theo:Sylvester'sLawOfInertia}),第二条是显然的。
\end{proof}
显然矩阵$A$的正惯性指数与负惯性指数就等于二次型$x^TAx$的正惯性指数与负惯性指数,也等于$A$的合同标准形主对角线上大于$0$的元素的个数与小于$0$的个数。
\begin{theorem}
	两个$n$阶实对称矩阵合同$\iff$它们的秩相等,并且正惯性指数也相等。
\end{theorem}
\begin{proof}
	由\cref{theo:QuadraticEquivCongruent}可得矩阵合同等价于各自对应的二次型等价,再由\cref{theo:RQuadraticFormEquiv}可得两个二次型的秩与正惯性指数都相等。因为矩阵的秩与正惯性指数等于对应的二次型的秩与正惯性指数,所以结论成立。
\end{proof}
\subsubsection{复二次型的规范形}
\begin{definition}
	复数域上的二次型称为\textbf{复二次型}。由\cref{prop:CongruentMatrix}(6)可知$n$元复二次型$x^TAx$的矩阵$A$合同于一个对角矩阵$\operatorname{diag}\{1,1,\dots,1,0,0,\dots,0\}$,再由\cref{theo:QuadraticEquivCongruent}可知经过一个适当的非退化线性变换可以将$x^TAx$化作:
	\begin{equation*}
		z_1^2+z_2^2+\cdots+z_r^2
	\end{equation*}
	称此形式为二次型$x^TAx$的\textbf{规范形},其特征为:只含平方项且平方项系数为$1,0$,系数为$1$的平方项在前面,系数为$0$的平方项在后面。
\end{definition}
\begin{theorem}\label{theo:CQuadraticFormOnly}
	复二次型$x^TAx$的规范形是唯一的。
\end{theorem}
\begin{proof}
	复二次型$x^TAx$的规范形完全由它的秩$r$所决定。
\end{proof}
\begin{theorem}
	两个$n$元复二次型等价\par
	$\iff$它们的规范形相同\par
	$\iff$它们的秩相等。
\end{theorem}
\begin{proof}
	第一条由\cref{theo:CQuadraticFormOnly}以及二次型的传递性、对称性可直接得到(必要性的证明中需要考虑规范形的定义,然后使用\cref{theo:CQuadraticFormOnly}),第二条是显然的。
\end{proof}

\subsection{正定二次型与正定矩阵}
\begin{definition}
	如果对$\mathbb{R}^{n}$中任意非零列向量$\alpha$,都有$\alpha^TA\alpha>0$,则称$n$元实二次型$x^TAx$是\gls{PositiveDefinite}的。
\end{definition}
\begin{definition}
	若实二次型$x^TAx$是正定的,则称实对称矩阵$A$是正定的,并称$A$为\gls{PositiveDefiniteMatrix},记为$A>0$。
\end{definition}
\begin{theorem}
	$n$元实二次型$x^TAx$是正定的当且仅当它的正惯性指数等于$n$。
\end{theorem}
\begin{proof}
	\textbf{(1)必要性:}设$x^TAx$是正定的,作非退化线性变换$x=Cy$化成规范形:
	\begin{equation*}
		y_1^2+y_2^2+\cdots+y_p^2-y_{p+1}^2-y_{p+2}^2-y_r^2
	\end{equation*}
	如果$p<n$,则$y_n^2$的系数为$0$或$-1$,取$y=(0,0,\dots,1)^T$,则有$y^TC^TACy=-y_n^2$为$0$或$-1$,取$\alpha=Cy$即有$\alpha^TA\alpha$为$0$或$-1$,与二次型$x^TAx$的正定性矛盾,所以$p=n$。\par
	\textbf{(2)充分性:}设$x^TAx$的正惯性指数等于$n$,则可以作一个非退化线性变换$x=Cy$将该二次型化作规范形:
	\begin{equation*}
		y^TC^TACy=y_1^2+y_2^2+\cdots+y_n^2
	\end{equation*}
	因为矩阵$C$可逆,所以关于$y$的齐次线性方程组$C^{-1}x=\mathbf{0}$只有零解。任取非零向量$\alpha\in\mathbb{R}^{n}$,则$C^{-1}\alpha$不是零向量,令$y=C^{-1}\alpha$,于是$\alpha^T(C^{-1})^TC^TACC^{-1}\alpha>0$,即$\alpha^TA\alpha>0$。由$\alpha$的任意性,$x^TAx$是正定的。
\end{proof}
\begin{theorem}\label{theo:PositiveDefinite}
	由上述定理可得到如下推论:
	\begin{enumerate}
		\item 对于$n$元实二次型$x^TAx$,下述说法等价:
		\begin{itemize}
			\item $x^TAx$是正定的;
			\item $x^TAx$的规范形为$y_1^2+y_2^2+\cdots+y_n^2$;
			\item $x^TAx$的标准形中的$n$个系数都大于$0$;
		\end{itemize}
		\item 与正定二次型等价的实二次型也是正定的;
		\item 对于$n$阶实对称矩阵$A$,下述说法等价:
		\begin{itemize}
			\item $A$是正定的;
			\item $A$的正惯性指数为$n$;
			\item $A\cong I$;
			\item $A$的合同标准形中主对角元都大于$0$;
			\item $A$的特征值都大于$0$;
			\item $A$的顺序主子式都大于$0$。
		\end{itemize}
		\item 与正定矩阵合同的实对称矩阵也是正定矩阵。
		\item 正定矩阵的行列式大于$0$;
		\item 正定矩阵可逆;
	\end{enumerate}
\end{theorem}
\begin{proof}
	(1)$1\iff2$:由上一定理,$x^TAx$正定当且仅当它的正惯性指数为$n$,而$x^TAx$的正惯性指数为$n$当且仅当它的规范形为$y_1^2+y_2^2+\cdots+y_n^2$。\par
	$2\Rightarrow3$:由标准形化规范形的步骤,若$x^TAx$的规范形为$y_1^2+y_2^2+\cdots+y_n^2$,则其标准形中的$n$个系数必然都大于$0$;\par
	$3\Rightarrow2$:当$x^TAx$的标准形中的$n$个系数都大于$0$时,也必然可以将其化为$y_1^2+y_2^2+\cdots+y_n^2$。\par
	(2)由(4)、\cref{theo:QuadraticEquivCongruent}和正定矩阵的定义可直接得到。\par	
	(3)$1\Rightarrow2$:因为$A$是正定的,所以$n$元二次型$x^TAx$是正定的,由上一定理可得$x^TAx$的正惯性指数为$n$。因为$A$的正惯性指数等于$x^TAx$的正惯性指数,所以$A$的正惯性指数为$n$。\par
	$2\Rightarrow3$:因为$A$的正惯性指数为$n$,由矩阵正惯性指数的定义,$A$合同于$I$。\par
	$3\Rightarrow4$:因为$A$合同于$I$,由合同规范形的定义,$I$是$A$的合同规范形,由合同标准型化合同规范形的步骤,$A$的合同标准型中主对角元都大于$0$。\par
	$4\Rightarrow5$:由\cref{prop:HermitianMatEigen}(3)可知$A\cong\operatorname{diag}\{\seq{\lambda}{n}\}$,其中$\lambda_i,\;i=1,2,\dots,n$是$A$的特征值。显然$\operatorname{diag}\{\seq{\lambda}{n}\}$是$A$的一个合同标准型,因为$A$的合同标准型中主对角元都大于$0$,所以$A$的特征值都大于$0$。
	\par
	$5\Rightarrow2$:显然。\par
	$2\Rightarrow1$:由\cref{theo:QuadraticEquivCongruent}、上一定理和矩阵正定的定义可直接得到。\par
	$1\Rightarrow6$:设$n$阶实对称矩阵$A$是正定的,则对于$k=1,2,\dots,n-1$,把$A$写成分块矩阵:
	\begin{equation*}
		A=
		\begin{pmatrix}
			A_k & B_1 \\
			B_1^T & B_2
		\end{pmatrix}
	\end{equation*}
	其中$|A_k|$是$A$的$k$阶顺序主子式。在$\mathbb{R}^{k}$中任取一个非零向量$\delta$,因为$A$是正定矩阵,所以:
	\begin{equation*}
		\begin{pmatrix}
			\delta \\
			\mathbf{0}
		\end{pmatrix}^T
		A
		\begin{pmatrix}
			\delta \\
			\mathbf{0}
		\end{pmatrix}
		=
		\begin{pmatrix}
			\delta^T & \mathbf{0}
		\end{pmatrix}
		\begin{pmatrix}
			A_k & B_1 \\
			B_1^T & B_2
		\end{pmatrix}
		\begin{pmatrix}
			\delta \\
			\mathbf{0}
		\end{pmatrix}
		=\delta^TA_k\delta>0
	\end{equation*}
	由$\delta$的任意性,$A_k$是正定矩阵。由(5),$|A_k|>0,\;k=1,2,\dots,n-1,\;|A|>0$。\par
	$6\Rightarrow1$:对实对称矩阵$A$的阶数$n$作数学归纳法。\par
	当$n=1$时,因为$A$的顺序主子式都大于$0$,所以$A$的唯一一个元素大于$0$,显然此时$A$是正定矩阵。\par
	假设对于$n-1$阶实对称矩阵命题为真,考虑$n$阶实对称矩阵$A=(a_{ij})$,将其写作分块矩阵的形式:
	\begin{equation*}
		A=
		\begin{pmatrix}
			A_{n-1} & \alpha \\
			\alpha^T & a_{nn}
		\end{pmatrix}
	\end{equation*}
	其中$A_{n-1}$是$n-1$阶实对称矩阵,因为$A_{n-1}$的所有顺序主子式是$A$的$1$到$n-1$阶顺序主子式,它们都大于$0$,由归纳假设可得$A_{n-1}$是正定的。根据(6)可知$A_{n-1}$可逆。由(3)的第三条可知存在可逆矩阵$C\in M_{n-1}(\mathbb{R})$使得$C^TA_{n-1}C=I$。因为:
	\begin{equation*}
		\begin{pmatrix}
			I & \mathbf{0} \\
			-\alpha^TA_{n-1}^{-1} & 1
		\end{pmatrix}
		\begin{pmatrix}
			A_{n-1} & \alpha \\
			\alpha^T & a_{nn}
		\end{pmatrix}
		\begin{pmatrix}
			I & -A_{n-1}^{-1}\alpha \\
			\mathbf{0} & 1
		\end{pmatrix}
		=
		\begin{pmatrix}
			A_{n-1} & \mathbf{0} \\
			\mathbf{0} & a_{nn}-\alpha^TA_{n-1}^{-1}\alpha
		\end{pmatrix}
	\end{equation*}
	注意到:
	\begin{equation*}
		\begin{pmatrix}
			I & \mathbf{0} \\
			-\alpha^TA_{n-1}^{-1} & 1
		\end{pmatrix}^T
		=
		\begin{pmatrix}
			I & (-\alpha^TA_{n-1}^{-1})^T \\
			\mathbf{0} & 1
		\end{pmatrix}
		=
		\begin{pmatrix}
			I & -A_{n-1}^{-1}\alpha \\
			\mathbf{0} & 1
		\end{pmatrix}
	\end{equation*}
	且:
	\begin{equation*}
		\begin{pmatrix}
			I & \mathbf{0} \\
			-\alpha^TA_{n-1}^{-1} & 1
		\end{pmatrix}
	\end{equation*}
	可逆,所以$A$合同于矩阵:
	\begin{equation*}
		\begin{pmatrix}
			A_{n-1} & \mathbf{0} \\
			\mathbf{0} & a_{nn}-\alpha^TA_{n-1}^{-1}\alpha
		\end{pmatrix}
	\end{equation*}
	因为:
	\begin{align*}
		\begin{vmatrix}
			A_{n-1} & \mathbf{0} \\
			\mathbf{0} & a_{nn}-\alpha^TA_{n-1}^{-1}\alpha
		\end{vmatrix}
		&=
		\begin{vmatrix}
			I & \mathbf{0} \\
			-\alpha^TA_{n-1}^{-1} & 1
		\end{vmatrix}\;
		\begin{vmatrix}
			A_{n-1} & \alpha \\
			\alpha^T & a_{nn}
		\end{vmatrix}\;
		\begin{vmatrix}
			I & -A_{n-1}^{-1}\alpha \\
			\mathbf{0} & 1
		\end{vmatrix} \\
		&=
		\begin{vmatrix}
			A_{n-1} & \alpha \\
			\alpha^T & a_{nn}
		\end{vmatrix}
		=|A|
	\end{align*}
	所以$|A_{n-1}|(a_{nn}-\alpha^TA_{n-1}^{-1}\alpha)=|A|>0$,而$|A_{n-1}|>0$,所以$a_{nn}-\alpha^TA_{n-1}^{-1}\alpha>0$。因为:
	\begin{align*}
		&
		\begin{pmatrix}
			C & \mathbf{0} \\
			\mathbf{0} & 1
		\end{pmatrix}^T
		\begin{pmatrix}
			A_{n-1} & \mathbf{0} \\
			\mathbf{0} & a_{nn}-\alpha^TA_{n-1}^{-1}\alpha
		\end{pmatrix}
		\begin{pmatrix}
			C & \mathbf{0} \\
			\mathbf{0} & 1
		\end{pmatrix} \\
		=&
		\begin{pmatrix}
			C^TA_{n-1}C & \mathbf{0} \\
			\mathbf{0} & a_{nn}-\alpha^TA_{n-1}^{-1}\alpha
		\end{pmatrix}
		=
		\begin{pmatrix}
			I & \mathbf{0} \\
			\mathbf{0} & a_{nn}-\alpha^TA_{n-1}^{-1}\alpha
		\end{pmatrix}
	\end{align*}
	而:
	\begin{equation*}
		B=
		\begin{pmatrix}
			I & \mathbf{0} \\
			\mathbf{0} & a_{nn}-\alpha^TA_{n-1}^{-1}\alpha
		\end{pmatrix}
	\end{equation*}
	主对角线上的元素都大于$0$,由(3)的第四条可知$B$是一个正定矩阵。因为$|C|1=|C|\ne0$,根据\cref{prop:InvertibleMatrix}(3.a)可得:
	\begin{equation*}
		\begin{pmatrix}
			C & \mathbf{0} \\
			\mathbf{0} & 1
		\end{pmatrix}
	\end{equation*}
	可逆。于是:
	\begin{equation*}
		\begin{pmatrix}
			A_{n-1} & \mathbf{0} \\
			\mathbf{0} & a_{nn}-\alpha^TA_{n-1}^{-1}\alpha
		\end{pmatrix}
	\end{equation*}
	合同于$B$。根据合同的传递性,$A$合同于正定矩阵$B$。由(4),$A$是一个正定矩阵。\par
	(4)设$A$是一个正定矩阵,$B$是一个实对称矩阵且合同于$A$。由(3)的第三条可知$A$合同于$I$,根据合同的传递性,$B$也合同于$I$。再由(3)的第三条可得$B$也是一个正定矩阵。\par
	(5)设$A$是一个正定矩阵,由(3)的第三条可得$A\cong I$,即存在可逆矩阵$C$,使得$C^TAC=I$,于是:
	\begin{equation*}
		|C^TAC|=|C^T|\;|A|\;|C|=|A|\;|C|^2=1
	\end{equation*}
	因为$|C|^2>0$,所以$|A|>0$。\par
	(6)由(5)和\cref{prop:InvertibleMatrix}(3.a)立即得到。
\end{proof}
\subsubsection{半正定二次型与半正定矩阵}
\begin{definition}
	如果对$\mathbb{R}^{n}$中任意非零列向量$\alpha$,都有$\alpha^TA\alpha\geqslant0$,则称$n$元实二次型$x^TAx$是\gls{PositiveSemidefinite}的。
\end{definition}
\begin{definition}
	若实二次型$x^TAx$是半正定的,则称实对称矩阵$A$是半正定的,并称$A$为\gls{PositiveSemidefiniteMatrix},记为$A\geqslant0$。
\end{definition}
\begin{theorem}\label{theo:PositiveSemidefinite}
	由上述定理可得到如下推论:
	\begin{enumerate}
		\item 对于$n$元实二次型$x^TAx$,$\operatorname{rank}(A)=r$,下述说法等价:
		\begin{itemize}
			\item $x^TAx$是半正定的;
			\item $x^TAx$的正惯性指数等于$r$;
			\item $x^TAx$的规范形为$y_1^2+y_2^2+\cdots+y_r^2$;
			\item $x^TAx$的标准形中的$n$个系数都非负;
		\end{itemize}
		\item 与半正定二次型等价的实二次型也是半正定的;
		\item 对于$n$阶实对称矩阵$A$,$\operatorname{rank}(A)=r$,下述说法等价:
		\begin{itemize}
			\item $A$是半正定的;
			\item $A$的正惯性指数为$r$;
			\item $A\cong
			\begin{pmatrix}
				I_r & \mathbf{0} \\
				\mathbf{0} & \mathbf{0}
			\end{pmatrix}$;
			\item $A$的合同标准形中主对角元都非负;
			\item $A$的特征值都非负;
			\item $A$的主子式都非负。
		\end{itemize}
		\item 与半正定矩阵合同的实对称矩阵也是半正定矩阵。
		\item 半正定矩阵的行列式为$0$;
	\end{enumerate}
\end{theorem}
\begin{proof}
	(1)$1\Rightarrow3$:作非退化线性变换$x=Cy$把$x^TAx$化作规范形:
	\begin{equation*}
		y_1^2+y_2^2+\cdots+y_p^2-y_{p+1}^2-y_{p+2}^2-y_r^2
	\end{equation*}
	若$p<r$,取$\alpha=(0,0,\dots,0,1,0,0,\dots,0)$,其中只有第$r$位为$1$,则$(C\alpha)^TA(C\alpha)=\alpha C^TAC\alpha=-1$,与$x^TAx$的非负定性矛盾,所以$p=r$。\par
	$3\Rightarrow2$:显然。\par
	$2\Rightarrow4$:显然。\par
	$4\Rightarrow1$:作非退化线性变换$x=Cy$把$x^TAx$化作一个标准形$d_1y_1^2+d_2y_2^2+\cdots+d_ny_n^2$,其中$d_i\geqslant0,\;i=1,2,\dots,n$。任取$\alpha\in\mathbb{R}^{n}$且$\alpha\ne\mathbf{0}$。因为$C$可逆,所以$C^{-1}x=\mathbf{0}$只有零解,于是$C^{-1}\alpha=(b_1,b_2,\dots,b_n)\ne\mathbf{0}$,所以:
	\begin{equation*}
		(C^{-1}\alpha)^TC^TACC^{-1}\alpha=\sum_{i=1}^{n}d_ib_i^2\geqslant0
	\end{equation*}
	而:
	\begin{equation*}
		(C^{-1}\alpha)^TC^TACC^{-1}\alpha=\alpha^T(C^{-1})^TC^TACC^{-1}\alpha=\alpha^T(C^T)^{-1}C^TACC^{-1}\alpha=\alpha^TA\alpha 
	\end{equation*}
	所以$\alpha^TA\alpha\geqslant0$。由$\alpha$的任意性,$x^TAx$半正定。\par
	(2)由(4)、\cref{theo:QuadraticEquivCongruent}和半正定矩阵的定义可直接得到。\par	
	(3)$1\Rightarrow2$:因为$A$是半正定的,所以$x^TAx$是半正定的。由(1)的第二条,$x^TAx$的正惯性指数等于$r$,而$A$的正惯性指数等于$x^TAx$的正惯性指数,所以$A$的正惯性指数为$r$。\par
	$2\Rightarrow3$:因为$A$的正惯性指数为$r$,由矩阵正惯性指数的定义,$A\cong\begin{pmatrix}
		I_r & \mathbf{0} \\
		\mathbf{0} & \mathbf{0}
	\end{pmatrix}$。\par
	$3\Rightarrow4$:因为$A\cong C=
	\begin{pmatrix}
		I_r & \mathbf{0} \\
		\mathbf{0} & \mathbf{0}
	\end{pmatrix}$,所以$C$是$A$的合同规范形。由合同标准形化合同规范形的步骤,$A$的合同标准形中主对角元都大于$0$。\par
	$4\Rightarrow5$:由\cref{prop:HermitianMatEigen}(3)可知$A\cong\operatorname{diag}\{\seq{\lambda}{n}\}$,其中$\lambda_i,\;i=1,2,\dots,n$是$A$的特征值。显然$\operatorname{diag}\{\seq{\lambda}{n}\}$是$A$的一个合同标准型,因为$A$的合同标准型中主对角元都非负,所以$A$的特征值都非负。\par
	$5\Rightarrow2$:因为$\operatorname{rank}=r$,所以$A$的相似标准形主对角线上的元素有且只有$r$个非零,由条件它们也非负,于是它们为正数,显然此时$A$的正惯性指数为$r$。\par
	$2\Rightarrow1$:由\cref{theo:QuadraticEquivCongruent}、(1)的第二条和矩阵半正定的定义可直接得到。\par
	$1\Rightarrow6$:\par
	$6\Rightarrow5$:\info{有空证明}\par
	(4)设$A$是一个半正定矩阵,$B$是一个实对称矩阵且合同于$A$。由(3)的第三条可知$A\cong C=\begin{pmatrix}
		I_r & \mathbf{0} \\
		\mathbf{0} & \mathbf{0}
	\end{pmatrix}$
	,根据合同的传递性,$B\cong C$。再由(3)的第三条可得$B$也是一个半正定矩阵。\par
	(5)设$A$是一个$n$阶半正定矩阵,由(3)的第三条,存在可逆矩阵$C$使得:
	\begin{equation*}
		C^TAC=B=
		\begin{pmatrix}
			I_r & \mathbf{0} \\
			\mathbf{0} & \mathbf{0}
		\end{pmatrix}
	\end{equation*}
	而$\operatorname{rank}(B)=r$,因为可逆变换不改变矩阵的秩,所以$\operatorname{rank}(A)=r<n$,于是$|A|=0$。
\end{proof}
\subsubsection{负定矩阵}
\begin{definition}
	如果对$\mathbb{R}^{n}$中任意非零列向量$\alpha$,都有$\alpha^TA\alpha<0$,则称$n$元实二次型$x^TAx$是\gls{NegativeDefinite}的。
\end{definition}
\begin{definition}
	若实二次型$x^TAx$是负定的,则称实对称矩阵$A$是负定的,并称$A$为\gls{NegativeDefiniteMatrix},记为$A<0$。
\end{definition}
\begin{theorem}
	对称矩阵$A\in M_{n}(\mathbb{R})$负定的充分必要条件为:它的奇数阶顺序主子式都小于$0$,偶数阶顺序主子式都大于$0$。
\end{theorem}
\begin{proof}
	设$|A_k|$为$A$的$k$阶顺序主子式,由\cref{theo:PositiveDefinite}(3)的第六条:
	\begin{align*}
		&A\text{是负定矩阵} \\
		\iff&(-A)\text{是正定矩阵} \\
		\iff&(-1)^k|A_k|>0 \\
		\iff&
		\begin{cases}
			|A_k|>0,& k\text{为偶数} \\
			|A_k|<0,& k\text{为奇数} \\
		\end{cases}\qedhere
	\end{align*}
\end{proof}

\section{特殊矩阵}
\subsection{幂等阵}
\subsubsection{幂等阵}
\begin{property}
	设$A\in M_{n}(K)$是一个幂等阵,$\operatorname{rank}(A)=r$,则:
	\begin{enumerate}
		\item $A$的特征值只能是$1$或$0$;
		\item $\operatorname{tr}(A)=\operatorname{rank}(A)$;
		\item $\operatorname{rank}(A)+\operatorname{rank}(I_n-A)=n$;
		\item 存在秩为$r$的$B\in M_{n}(K)$使得$A=B(B^TB)^-B^T$;
	\end{enumerate}
\end{property}
\begin{proof}
	(1)设$\lambda$为$A$的一个特征值,$\varphi$为对应的特征向量,因为$A$是一个幂等阵,所以$A^2\varphi=A\varphi=\lambda\varphi$,又因为:
	\begin{equation*}
		A^2\varphi=AA\varphi=A\lambda\varphi=\lambda A\varphi=\lambda^2\varphi
	\end{equation*}
	所以$(\lambda^2-\lambda)\varphi=\mathbf{0}$。因为$\varphi$是特征向量,所以$\varphi\ne\mathbf{0}$,于是$\lambda^2-\lambda=0$,即$\lambda=1$或$\lambda=0$。由$\lambda$的任意性,结论成立。\par
\end{proof}
\begin{property}\label{prop:HermitianIdempotent}
	设$A\in M_{n}(\mathbb{C})$是一个Hermitian幂等阵,则:
	\begin{enumerate}
		\item 
		\item $\operatorname{rank}(A)=\operatorname{tr}(A)$。
	\end{enumerate}
\end{property}
\begin{proof}
	(1)
	(2)由\cref{prop:HermitianMatEigen}(3)和(1)可知存在一个正交矩阵$Q$使得:
	\begin{equation*}
		A=Q^{-1}
		\begin{pmatrix}
			I_r & \mathbf{0} \\
			\mathbf{0} & \mathbf{0}
		\end{pmatrix}Q
	\end{equation*}
	由\cref{prop:Similar}可得:
	\begin{equation*}
		\operatorname{rank}(A)=\operatorname{rank}(I_r)=r
	\end{equation*}
	根据\cref{prop:Trace}(3)可得:
	\begin{equation*}
		\operatorname{tr}(A)=\operatorname{tr}\left[Q^{-1}
		\begin{pmatrix}
			I_r & \mathbf{0} \\
			\mathbf{0} & \mathbf{0}
		\end{pmatrix}Q\right]
		=\operatorname{tr}\left[
		\begin{pmatrix}
			I_r & \mathbf{0} \\
			\mathbf{0} & \mathbf{0}
		\end{pmatrix}QQ^{-1}\right]
		=\operatorname{tr}\left[
		\begin{pmatrix}
			I_r & \mathbf{0} \\
			\mathbf{0} & \mathbf{0}
		\end{pmatrix}\right]
		=r
	\end{equation*}
	所以有$\operatorname{rank}(A)=\operatorname{tr}(A)$。
\end{proof}
\section{矩阵的分解}

\subsection{SVD分解}
\begin{theorem}\label{theo:AATPositiveSemidefinite}
	设$A\in M_{m\times n}(\mathbb{C})$,则$AA^H,A^HA$是半正定矩阵。
\end{theorem}
\begin{proof}
	设$\lambda_i,\;i=1,2,\dots,n$是矩阵$A^HA$的特征值,$\xi_i$是对应的特征向量,则:
	\begin{align*}
		A^HA\xi_i=\lambda_i\xi_i\rightarrow
		\xi_i^HA^HA\xi_i=\lambda_i\xi_i^H\xi_i\rightarrow
		||A\xi_i||^2=\lambda_i||\xi_i||^2
	\end{align*}
	由于左式非负,所以右式非负,而$||\xi_i||^2$非负,因此$\lambda_i$非负,由\cref{theo:PositiveSemidefinite}(3)的第五条可知$AA^T$是半正定矩阵。
\end{proof}
\begin{theorem}\label{theo:SVD}
	设$A\in M_{m\times n}(\mathbb{C})$,$\operatorname{rank}(A)=r$,则存在两个正交矩阵$P\in M_{m}(\mathbb{C}),\;Q\in M_{n}(\mathbb{C})$使得:
	\begin{equation*}
		A=P
		\begin{pmatrix}
			\varLambda & \mathbf{0} \\
			\mathbf{0} & \mathbf{0}
		\end{pmatrix}Q^H
	\end{equation*}
	其中$\varLambda=\operatorname{diag}\{\lambda_1,\lambda_2,\dots,\lambda_r\}$,$\lambda_i>0$,$\lambda_i^2$为$A^HA$的正特征值。
\end{theorem}
\begin{proof}
	由\cref{theo:RankAAHA}可知$\operatorname{rank}(A^HA)=\operatorname{rank}(A)$。于是$A^HA$确实有$r$个正特征值。因为$A^HA$是一个Hermitian矩阵,由\cref{prop:HermitianMatEigen}可知存在正交矩阵$Q\in M_{n}(\mathbb{C})$使得:
	\begin{equation*}
		Q^HA^HAQ=
		\begin{pmatrix}
			\varLambda^2 & \mathbf{0} \\
			\mathbf{0} & \mathbf{0}
		\end{pmatrix}
	\end{equation*}
	记$B=AQ$,则:
	\begin{equation*}
		B^HB=
		\begin{pmatrix}
			\varLambda^2 & \mathbf{0} \\
			\mathbf{0} & \mathbf{0}
		\end{pmatrix}
	\end{equation*}
	这表明$B$的列向量相互正交,且前$r$个列向量的长度分别为$\lambda_1,\lambda_2,\dots,\lambda_r$,后$n-r$个列向量为零向量,于是存在一个正交矩阵$P\in M_{m}(\mathbb{C})$使得:
	\begin{equation*}
		B=P
		\begin{pmatrix}
			\varLambda & \mathbf{0} \\
			\mathbf{0} & \mathbf{0}
		\end{pmatrix}
	\end{equation*}
	因为$B=AQ$,所以:
	\begin{equation*}
		A=P
		\begin{pmatrix}
			\varLambda & \mathbf{0} \\
			\mathbf{0} & \mathbf{0}
		\end{pmatrix}Q^{-1}
		=P
		\begin{pmatrix}
			\varLambda & \mathbf{0} \\
			\mathbf{0} & \mathbf{0}
		\end{pmatrix}Q^H\qedhere
	\end{equation*}
\end{proof}
\begin{definition}
	设$A\in M_{m\times n}(\mathbb{C})$,$\operatorname{rank}(A)=r$,$A^HA$的正特征值为$\lambda_i,\;i=1,2,\dots,r$,称$\delta_i=\sqrt{\lambda_i}$为矩阵$A$的\gls{SingularValue}。
\end{definition}