\section{相似的应用}

\subsection{特征值与特征向量}
\begin{definition}
	$A\in M_{n}(K)$。如果$K^n$中存在非零列向量$\alpha$,使得:
	\begin{equation*}
		A\alpha=\lambda\alpha,\;\lambda\in K
	\end{equation*}
	则称$\lambda$是$A$的一个\gls{Eigenvalue},$\alpha$是$A$属于特征值$\lambda$的一个\gls{Eigenvector}。
\end{definition}
\subsubsection{求解特征值与特征向量}
\begin{definition}
	$A\in M_{n}(K)$,称$|\lambda I-A|$为$A$的\gls{CharacteristicPolynomial}。
\end{definition}
\begin{theorem}
	$A\in M_{n}(K)$,则:
	\begin{enumerate}
		\item $\lambda$是$A$的一个特征值当且仅当$\lambda$是$A$的特征多项式在数域$K$中的一个根;
		\item $\alpha$是$A$属于特征值$\lambda$的一个特征向量当且仅当$\alpha$是齐次线性方程组$(\lambda I-A)x=\mathbf{0}$的一个非零解。
	\end{enumerate}
\end{theorem}
\begin{proof}
	显然:
	\begin{align*}
		&\lambda\text{是}A\text{的一个特征值,}\alpha\text{是}A\text{属于}\lambda\text{的一个特征向量} \\		\iff&A\alpha=\lambda\alpha,\;\alpha\ne\mathbf{0},\;\lambda\in K \\
		\iff&(\lambda I-A)\alpha=\mathbf{0},\;a\ne\mathbf{0},\;\lambda\in K \\
		\iff&\alpha\text{是齐次线性方程组}(\lambda I-A)x=\mathbf{0}\text{的一个非零解,}\lambda\in K \\		\iff&|\lambda I-A|=0,\;\alpha\text{是齐次线性方程组}(\lambda I-A)x=\mathbf{0}\text{的一个非零解,}\lambda\in K \\
		\iff&\lambda\text{是多项式}|\lambda I-A|\text{在}K\text{中的一个根,} \\
		&\alpha\text{是齐次线性方程组}(\lambda I-A)x=\mathbf{0}\text{的一个非零解,}\lambda\in K\qedhere
	\end{align*}
\end{proof}
\subsubsection{特征向量的性质}
\begin{property}\label{prop:Eigenvector}
	$A\in M_{n}(K)$,其特征向量具有如下性质:
	\begin{enumerate}
		\item 设$\lambda$是$A$的一个特征值,则$A$属于$\lambda$的所有特征向量构成$K^n$的一个子空间。因此,把齐次线性方程组$(\lambda I-A)x=\mathbf{0}$的解空间称为$A$属于$\lambda$的\gls{Eigenspace},记为$W_{\lambda}$;
		\item $A$的属于不同特征值的特征向量是线性无关的。
	\end{enumerate}
\end{property}
\begin{proof}
	(1)任取$k_1,k_2\in K$和$A$属于特征值$\lambda$的两个特征向量$\alpha,\beta$,则
	\begin{equation*}
		A(k_1\alpha+k_2\beta)=k_1A\alpha+k_2A\beta=k_1\lambda\alpha+k_2\lambda\beta=\lambda(k_1\alpha+k_2\beta)
	\end{equation*}
	于是$k_1\alpha+k_2\beta$也是$A$属于特征值$\lambda$的特征向量。由\cref{theo:Subspace}可知$A$属于$\lambda$的所有特征向量构成$K^n$的一个子空间。\par
	(2)我们来证明:设$\seq{\lambda}{m}$是$A\in M_{n}(K)$的不同的特征值,$a_{j1},a_{j2},\dots,a_{jr_j}$是$A$属于$\lambda_j$的线性无关的特征向量,$j=1,2,\dots,m$,则向量组:
	\begin{equation*}
		a_{11},a_{12},\dots,a_{1r_1},a_{21},a_{22},\dots,a_{2r_2},a_{m1},a_{m2},\dots,a_{mr_m}
	\end{equation*}
	线性无关。\par
	\textbf{1.证明对$n=2$成立:}对于$\lambda_1$和$\lambda_2$的线性无关的特征向量$a_{11},a_{12},\dots,a_{1r_1}$和$a_{21},a_{22},\dots,a_{2r_2}$,设:
	\begin{equation*}
		k_1a_{11}+k_2a_{12}+\cdots+k_{r_1}a_{1r_1}+l_1a_{21}+l_2a_{22}+\cdots+l_{r_2}a_{2r_2}=\mathbf{0}
	\end{equation*}
	两边同乘$A$可得:
	\begin{gather*}
		k_1Aa_{11}+k_2Aa_{12}+\cdots+k_{r_1}Aa_{1r_1}+l_1Aa_{21}+l_2Aa_{22}+\cdots+l_{r_2}Aa_{2r_2}=\mathbf{0} \\
		k_1\lambda_1a_{11}+k_2\lambda_1a_{12}+\cdots+k_{r_1}\lambda_1a_{1r_1}+l_1\lambda_2a_{21}+l_2\lambda_2a_{22}+\cdots+l_{r_2}\lambda_2a_{2r_2}=\mathbf{0}
	\end{gather*}
	因为$\lambda_1\ne\lambda_2$,所以$\lambda_1,\lambda_2$不全为$0$。设$\lambda_2\ne0$,在上上上个式子两端乘以$\lambda_2$(若$\lambda_2=0$,则同乘$\lambda_1$)得:
	\begin{equation*}
		k_1\lambda_2a_{11}+k_2\lambda_2a_{12}+\cdots+k_{r_1}\lambda_2a_{1r_1}+l_1\lambda_2a_{21}+l_2\lambda_2a_{22}+\cdots+l_{r_2}\lambda_2a_{2r_2}=\mathbf{0}
	\end{equation*}
	于是:
	\begin{equation*}
		k_1(\lambda_1-\lambda_2)a_{11}+k_2(\lambda_1-\lambda_2)a_{12}+\cdots+k_{r_1}(\lambda_1-\lambda_2)a_{1r_1}=\mathbf{0}
	\end{equation*}
	因为$\lambda_1\ne\lambda_2$,所以:
	\begin{equation*}
		k_1a_{11}+k_2a_{12}+\cdots+k_{r_1}a_{1r_1}=\mathbf{0}
	\end{equation*}
	因为$a_{11},a_{12},\dots,a_{1r_1}$线性无关,所以$k_1=k_2=\cdots=k_{r_1}=0$,从而:
	\begin{equation*}
		l_1a_{21}+l_2a_{22}+\cdots+l_{r_2}a_{2r_2}=\mathbf{0}
	\end{equation*}
	因为$a_{21},a_{22},\dots,a_{2r_2}$线性无关,所以$l_1=l_2=\cdots=l_{r_2}=0$。\par
	综上,向量组$a_{11},a_{12},\dots,a_{1r_1},a_{21},a_{22},\dots,a_{2r_2}$线性无关。\par
	\textbf{2.归纳假设:}假设对$n$个不同的特征值都有上述结论(即$n$个不同特征值的线性无关的特征向量构成的向量组线性无关),下面来证明对$n+1$个不同的特征值也成立。\par
	设:
	\begin{equation*}
		k_{11}a_{11}+k_{12}a_{12}+\cdots k_{1r_1}a_{1r_1}+\cdots+k_{nr_n}a_{nr_n}+l_1a_{(n+1)1}+l_2a_{(n+1)2}+\cdots+l_{r_{n+1}}a_{(n+1)r_{n+1}}=\mathbf{0}
	\end{equation*}
	两边同乘$A$可得:
	\begin{align*}
		&k_{11}Aa_{11}+k_{12}Aa_{12}+\cdots k_{1r_1}Aa_{1r_1}+\cdots+k_{nr_n}Aa_{nr_n} \\
		+&l_1Aa_{(n+1)1}+l_2Aa_{(n+1)2}+\cdots+l_{r_{n+1}}Aa_{(n+1)r_{n+1}}=\mathbf{0} \\
		&k_{11}\lambda_1a_{11}+k_{12}\lambda_1a_{12}+\cdots k_{1r_1}\lambda_1a_{1r_1}+\cdots+k_{nr_n}\lambda_na_{nr_n} \\
		+&l_1\lambda_{n+1}a_{(n+1)1}+l_2\lambda_{n+1}a_{(n+1)2}+\cdots+l_{r_{n+1}}\lambda_{n+1}a_{(n+1)r_{n+1}}=\mathbf{0}
	\end{align*}\par
	\textbf{2.1.$\lambda_{n+1}\ne0$:}若$\lambda_{n+1}\ne0$,则在上上上式两边同乘$\lambda_{n+1}$可得:
	\begin{align*}
		&k_{11}\lambda_{n+1}a_{11}+k_{12}\lambda_{n+1}a_{12}+\cdots k_{1r_1}\lambda_{n+1}a_{1r_1}+\cdots+k_{nr_n}\lambda_{n+1}a_{nr_n} \\
		+&l_1\lambda_{n+1}a_{(n+1)1}+l_2\lambda_{n+1}a_{(n+1)2}+\cdots+l_{r_{n+1}}\lambda_{n+1}a_{(n+1)r_{n+1}}=\mathbf{0}
	\end{align*}
	于是有:
	\begin{equation*}
		k_{11}(\lambda_{n+1}-\lambda_1)a_{11}+k_{12}(\lambda_{n+1}-\lambda_1)a_{12}+\cdots k_{1r_1}(\lambda_{n+1}-\lambda_1)a_{1r_1}+\cdots+k_{nr_n}(\lambda_{n+1}-\lambda_n)a_{nr_n}=\mathbf{0}
	\end{equation*}
	由归纳假定$a_{11},a_{12},\dots,a_{1r_1},\dots,a_{nr_n}$线性无关,所以
	\begin{equation*}
		k_{11}(\lambda_{n+1}-\lambda_1)=k_{12}(\lambda_{n+1}-\lambda_1)=\cdots=k_{1r_1}(\lambda_{n+1}-\lambda_1)=\cdots=k_{nr_n}(\lambda_{n+1}-\lambda_n)=0
	\end{equation*}
	因为$\lambda_i,\;i=1,2,\dots,n$之间互不相同,所以$\lambda_{n+1}-\lambda_1,\lambda_{n+1}-\lambda_2,\dots,\lambda_{n+1}-\lambda_n$不为$0$,于是$k_{11}=k_{12}=\cdots=k_{1r_1}=\cdots=k_{nr_n}=0$,所以:
	\begin{equation*}
		l_1a_{(n+1)1}+l_2a_{(n+1)2}+\cdots+l_{r_{n+1}}a_{(n+1)r_{n+1}}=\mathbf{0}
	\end{equation*}
	因为$a_{(n+1)1},a_{(n+1)2},\dots,a_{(n+1)r_{n+1}}$线性无关,所以有$l_1=l_2=\cdots=l_{r_{n+1}}=0$。\par
	综上$a_{11},a_{12},\dots,a_{1r_1},\dots,a_{nr_n},a_{(n+1)1},a_{(n+1)2},\dots,a_{(n+1)r_{n+1}}$线性无关。\par
	\textbf{2.2.$\lambda_{n+1}=0$:}若$\lambda_{n+1}=0$,则此时有:
	\begin{gather*}
		k_{11}\lambda_1a_{11}+k_{12}\lambda_1a_{12}+\cdots k_{1r_1}\lambda_1a_{1r_1}+\cdots+k_{nr_n}\lambda_na_{nr_n} \\
		+l_1\lambda_{n+1}a_{(n+1)1}+l_2\lambda_{n+1}a_{(n+1)2}+\cdots+l_{r_{n+1}}\lambda_{n+1}a_{(n+1)r_{n+1}} \\
		=k_{11}\lambda_1a_{11}+k_{12}\lambda_1a_{12}+\cdots k_{1r_1}\lambda_1a_{1r_1}+\cdots+k_{nr_n}\lambda_na_{nr_n}=\mathbf{0}
	\end{gather*}
	由归纳假定$a_{11},a_{12},\dots,a_{1r_1},\dots,a_{nr_n}$线性无关,所以$k_{11}\lambda_1=k_{12}\lambda_1=\cdots=k_{1r_1}\lambda_1=\cdots=k_{nr_n}\lambda_n=0$。因为$\seq{\lambda}{n}$都不是$0$($\lambda_i,\;i=1,2,\dots,n+1$互不相同,已经有$\lambda_{n+1}=0$了),所以$k_{11}=k_{12}=\cdots=k_{1r_1}=\cdots=k_{nr_n}=0$,于是有:
	\begin{equation*}
		l_1a_{(n+1)1}+l_2a_{(n+1)2}+\cdots+l_{r_{n+1}}a_{(n+1)r_{n+1}}=\mathbf{0}
	\end{equation*}
	因为$a_{(n+1)1},a_{(n+1)2},\dots,a_{(n+1)r_{n+1}}$线性无关,所以有$l_1=l_2=\cdots=l_{r_{n+1}}=0$。\par
	综上,$a_{11},a_{12},\dots,a_{1r_1},\dots,a_{nr_n},a_{(n+1)1},a_{(n+1)2},\dots,a_{(n+1)r_{n+1}}$线性无关。\par
	假设存在属于不同特征值的特征向量$\alpha_1,\alpha_2,\dots,\alpha_m$线性相关,则有:
	\begin{equation*}
		k_1\alpha_1+k_2\alpha_2+\cdots+k_m\alpha_m=\mathbf{0}
	\end{equation*}
	其中$k_1,k_2,\dots,k_m$不全为$0$。注意到$\alpha_i,\;i=1,2,\dots,m$可由其对应特征值的特征子空间中的一组基线性表出,于是有:
	\begin{equation*}
		\alpha_i=\sum_{n=1}^{r_i}l_n\beta_{in}
	\end{equation*}
	其中$\beta_{in},\;n=1,2,\dots,r_i$为$\alpha_i$对应特征值的特征子空间的一组基,所以:
	\begin{equation*}
		\sum_{i=1}^{m}k_i\sum_{n=1}^{r_i}l_n\beta_{in}=	\sum_{i=1}^{m}\sum_{n=1}^{r_i}k_il_n\beta_{in}=\mathbf{0}
	\end{equation*}
	而$\beta_{in},\;i=1,2,\dots,m,\;n=1,2,\dots,r_i$是线性无关的,所以:
	\begin{equation*}
		k_il_n=0,\;\forall\;i=1,2,\dots,m,\;n=1,2,\dots,r_i
	\end{equation*}
	因为$k_1,k_2,\dots,k_m$不全为$0$,所以存在一组$l_n$全为$0$,于是$\alpha_i$中存在零向量,而特征向量不是零向量,矛盾。
\end{proof}
\begin{theorem}\label{theo:SameEigenvalue}
	相似的矩阵有相同的特征多项式,进而有相同的特征值(包括重数相同)。
\end{theorem}
\begin{proof}
	设$A,B\in M_{n}(K)$且$A$与$B$相似,于是存在可逆矩阵$P\in M_{n}(K)$使得$P^{-1}AP=B$,就有:
	\begin{align*}
		|\lambda I-B|
		&=|\lambda I-P^{-1}AP|=|P^{-1}\lambda IP-P^{-1}AP| \\
		&=|P^{-1}(\lambda I-A)P|=|P^{-1}|\;|\lambda I-A|\;|P|=|\lambda I-A|\qedhere
	\end{align*}
\end{proof}
\subsubsection{几何重数与代数重数}
\begin{definition}
	$A\in M_{n}(K)$,$\lambda$是$A$的一个特征值。把$A$属于$\lambda$的特征子空间的维数叫作$\lambda$的\gls{GeometricMultiplicity},把$\lambda$作为$A$的特征多项式的根的重数叫作$\lambda$的\gls{AlgebraicMultiplicity}。
\end{definition}
\begin{theorem}\label{theo:AlgebraicMultiplicityGeometricMultiplicity}
	$A\in M_{n}(K)$,$\lambda_1$是$A$的一个特征值,则$\lambda_1$的几何重数不超过它的代数重数。
\end{theorem}
\begin{proof}
	设$A$属于特征值$\lambda_1$的特征子空间$W_1$的维数为$r$。在$W_1$中取一组基$\alpha_1,\alpha_2,\dots,\alpha_r$,把它扩充为$K^n$的一组基$\alpha_1,\alpha_2,\dots,\alpha_r,\beta_1,\beta_2,\dots,\beta_{n-r}$。令:
	\begin{equation*}
		P=(\alpha_1,\alpha_2,\dots,\alpha_r,\beta_1,\beta_2,\dots,\beta_{n-r})
	\end{equation*}
	则$P$是数域$K$上的$n$阶可逆矩阵,并且有:
	\begin{align*}
		P^{-1}AP
		&=P^{-1}(A\alpha_1,A\alpha_2,\dots,A\alpha_r,A\beta_1,A\beta_2,\dots,A\beta_{n-r}) \\
		&=P^{-1}(\lambda_1\alpha_1,\lambda_1\alpha_2,\dots,\lambda_1\alpha_r,A\beta_1,A\beta_2,\dots,A\beta_{n-r}) \\
		&=(\lambda_1\varepsilon_1,\lambda_1\varepsilon_2,\dots,\lambda_1\varepsilon_r,P^{-1}A\beta_1,P^{-1}A\beta_2,\dots,P^{-1}A\beta_{n-r}) \\
		&=
		\begin{pmatrix}
			\lambda_1I_r & B \\
			\mathbf{0} & C
		\end{pmatrix}
	\end{align*}
	由\cref{theo:SameEigenvalue}可得:
	\begin{align*}
		|\lambda I-A|&=
		\begin{vmatrix}
			\lambda I_r-\lambda_1I_r & -B \\
			\mathbf{0} & \lambda I_{n-r}-C
		\end{vmatrix} \\
		&=|\lambda I_r-\lambda_1I_r|\;|\lambda I_{n-r}-C| \\
		&=(\lambda-\lambda_1)^r|\lambda I_{n-r}-C|
	\end{align*}
	即$\lambda_1$的几何重数小于或等于$r$,也即$\lambda_1$的几何重数小于或等于它的代数重数。
\end{proof}

\subsection{矩阵的对角化}
\begin{definition}
	如果$n$阶矩阵$A$能够相似于一个对角矩阵,那么称$A$\gls{Diagonalizable}。
\end{definition}
研究矩阵是否可对角化是为了计算矩阵的幂,因为对角矩阵的幂是很好计算的。
\begin{theorem}\label{theo:DiagCondition}
	$A\in M_{n}(K)$可对角化的充分必要条件为
	\begin{enumerate}
		\item $A$有$n$个线性无关的特征向量$\seq{\alpha}{n}$;
		\item $A$的属于不同特征值的特征子空间的维数之和等于$n$;
		\item $A$的特征多项式的全部复根都属于$K$,且$A$的每个特征值的几何重数等于它的代数重数;
		\item 设$\seq{\lambda}{m}$是$A$全部的不同的特征值,则:
		\begin{equation*}
			K^n=W_{\lambda_1}\oplus W_{\lambda_2}\oplus\cdots\oplus W_{\lambda_m}
		\end{equation*}
	\end{enumerate}
	此时令$P=(\seq{\alpha}{n})$,则:
	\begin{equation*}
		P^{-1}AP=\operatorname{diag}\{\seq{\lambda}{n}\}
	\end{equation*}
	其中$\lambda_i$是$\alpha_i$所属的特征值,$i=1,2,\dots,n$。上述对角矩阵称为$A$的\textbf{相似标准形},除了主对角线上元素的排列次序外,$A$的相似标准形是唯一的;
\end{theorem}
\begin{proof}
	(1)显然:
	\begin{align*}
		&A\text{与}\text{对角矩阵}D=\operatorname{diag}\{\seq{\lambda}{n}\}\text{相似},\;\text{其中}\lambda_i\in K,\;i=1,2,\dots,n \\
		\iff&\text{如果存在可逆矩阵$P\in M_{n}(K)$,使得}P^{-1}AP=D \\
		&\text{即}AP=PD \\
		&\text{即}A(\seq{\alpha}{n})=(\seq{\alpha}{n})D \\
		&\text{即}(A\alpha_1,A\alpha_2,\dots,A\alpha_n)=(\lambda_1\alpha_1,\lambda_2\alpha_2,\dots,\lambda_n\alpha_n) \\
		\iff&K^{n}\text{中有$n$个线性无关的列向量}\seq{\alpha}{n}\text{使得}A\alpha_i=\lambda_i\alpha_i,\;i=1,2,\dots,n\qedhere
	\end{align*}\par
	(2)\textbf{充分性:}由\cref{prop:Eigenvector}(2)和(1)的充分性可直接得出。\par
	\textbf{必要性:}设$A$的所有不同的特征值是$\seq{\lambda}{m}$,它们的几何重数分别为$r_1,r_2,\dots,r_m$。若此时$A$的属于不同特征值的特征子空间的维数之和不等于$n$,由\cref{theo:AlgebraicMultiplicityGeometricMultiplicity}可知此时$r_1+r_2+\cdots+r_m<n$,那么$A$没有$n$个线性无关的特征向量,由(1)的必要性可得$A$不可以对角化。\par
	(3)\textbf{充分性:}由(2)的充分性可直接得到。\par
	\textbf{必要性:}因为$A$可对角化,由可对角化的定义可知$A$相似于:
	\begin{equation*}
		\operatorname{diag}(\lambda_1,\cdots,\lambda_1,\dots,\lambda_m,\dots,\lambda_m)\in M_{n}(K)
	\end{equation*}
	其中$\seq{\lambda}{m}$是$A$的全部不同的特征值,每个特征值重复的次数为对应特征子空间的维数,$\lambda_i$对应特征子空间的维数记为$r_i,\;i=1,2,\dots,m$。因为相似的矩阵具有相同的特征多项式,所以:
	\begin{equation*}
		|\lambda I-A|=(\lambda-\lambda_1)^{r_1}(\lambda-\lambda_2)^{r_2}\cdots(\lambda-\lambda_m)^{r_m}
	\end{equation*}
	于是$A$的特征多项式的根为$\seq{\lambda}{m}$。因为$\operatorname{diag}(\lambda_1,\cdots,\lambda_1,\dots,\lambda_m,\dots,\lambda_m)\in M_{n}(K)$,所以$\seq{\lambda}{m}\in K$,于是$A$的特征多项式的全部根都属于$K$且每一个特征值的代数重数等于它的几何重数。\par
	(4)由(2)、\cref{prop:Eigenvector}(2)、\cref{prop:nDimensionalLinearSpace}和\cref{theo:DirectSum}(5)可得:
	\begin{align*}
		A\text{可对角化}
		\iff&\sum_{i=1}^{m}\dim(W_{\lambda_i})=n \\
		\iff&W_{\lambda_i},i=1,2,\dots,m\text{的基合起来是$n$个线性无关的向量} \\
		\iff&W_{\lambda_i},i=1,2,\dots,m\text{的基合起来是$K^n$的一组基} \\
		\iff&K^n=W_{\lambda_1}\oplus W_{\lambda_2}\oplus\cdots\oplus W_{\lambda_m}\qedhere
	\end{align*}
\end{proof}

\subsection{Hermitian矩阵的对角化}
\begin{definition}
	若对于$A,B\in M_{n}(\mathbb{C})$,存在一个$n$阶正交矩阵$Q$,使得$Q^{-1}AQ=B$,则称$A$\textbf{正交相似}于$B$。
\end{definition}
\begin{theorem}
	正交相似是$M_{n}(\mathbb{C})$上的一个等价关系。
\end{theorem}
\begin{theorem}
	$A\in\ M_n(\mathbb{C})$。若$A$正交相似与一个对角矩阵$D$,则$A$一定是Hermitian矩阵。
\end{theorem}
\begin{proof}
	因为$A$正交相似于$D$,所以存在正交矩阵$Q$使得$Q^{-1}AQ=D$,即$A=QDQ^{-1}$,于是有:
	\begin{equation*}
		A^H=(QDQ^{-1})^H=(Q^{-1})^HD^HQ^H=(Q^H)^HDQ^{-1}=QDQ^{-1}=A
	\end{equation*}
	所以$A$是一个Hermitian矩阵。
\end{proof}
\begin{corollary}
	正交相似一定相似,相似不一定正交相似。
\end{corollary}
\begin{proof}
	设非Hermitian矩阵$A$相似于一个对角矩阵$D$,若$A$正交相似于$D$,则$A$得是一个Hermitian矩阵,而$A$不是一个Hermitian矩阵。
\end{proof}
\begin{property}\label{prop:HermitianMatEigen}
	设Hermitian矩阵$A,B\in M_{n}(\mathbb{C})$,则:
	\begin{enumerate}
		\item $A$的特征多项式的每一个根都是实数,从而都是$A$的特征值;
		\item $A$属于不同特征值的特征向量是正交的;
		\item $A$一定正交相似于由它的特征值构成的对角矩阵;
		\item $A$与$B$正交相似的充分必要条件为$A$与$B$相似。
	\end{enumerate}
\end{property}
\begin{proof}
	(1)设$\lambda$是$A$的特征多项式的任意一个根,将$A$看作是复数域$\mathbb{C}$上的矩阵,取$A$属于特征值$\lambda$的一个特征向量$\alpha$,考虑$\mathbb{C}^{n}$中的内积,有:
	\begin{gather*}
		(A\alpha,\alpha)=(\lambda\alpha,\alpha)=\lambda(\alpha,\alpha) \\
		(\alpha,A\alpha)=(\alpha,\lambda\alpha)=\overline{\lambda}(\alpha,\alpha) \\
		(A\alpha,\alpha)=(A\alpha)^H\alpha=\alpha^HA^H\alpha=\alpha^H A\alpha=(\alpha,A\alpha)
	\end{gather*}
	所以$\lambda(\alpha,\alpha)=\overline{\lambda}(\alpha,\alpha)$。因为$\alpha$是特征向量,所以$\alpha\ne\mathbf{0}$,于是$\lambda=\overline{\lambda}$,因此$\lambda$是一个实数。由$\lambda$的任意性,结论成立。\par
	(2)设$\lambda_1,\lambda_2$是$A$的不同的特征值(由(1)得它们都是实数),$\alpha_1,\alpha_2$分别是$A$属于$\lambda_1,\lambda_2$的一个特征向量,考虑$\mathbb{C}^{n}$上的标准内积:
	\begin{align*}
		\lambda_1(\alpha_1,\alpha_2)
		&=(\lambda_1\alpha_1,\alpha_2)=(A\alpha_1,\alpha_2)
		=A(\alpha_1,\alpha_2)=(\alpha_1,A^H\alpha_2) \\
		&=(\alpha_1,A\alpha_2)=(\alpha_1,\lambda_2\alpha_2)=\overline{\lambda_2}(\alpha_1,\alpha_2)=\lambda_2(\alpha_1,\alpha_2)
	\end{align*}
	于是有$(\lambda_1-\lambda_2)(\alpha_1,\alpha_2)=0$。因为$\lambda_1\ne\lambda_2$,所以$(\alpha_1,\alpha_2)=0$。\par
	(3)对$n$作数学归纳法。\par
	当$n=1$时,$(1)^{-1}A(1)=A$,结论成立。\par
	假设对于$n-1$阶的实对称矩阵都成立,考虑$n$阶实对称矩阵$A$。\par
	由(2)可知$A$必有特征值,取$A$的一个特征值$\lambda_1$和$A$属于$\lambda_1$的一个特征向量$\eta_1$,满足$||\eta_1||=1$。把$\eta_1$扩充为$\mathbb{C}^{n}$的一组基并进行Schimidt正交化和单位化,可得到$\mathbb{C}^{n}$的一个标准正交基$\eta_1,\eta_2,\dots,\eta_n$。令:
	\begin{equation*}
		Q_1=(\eta_1,\eta_2,\dots,\eta_n)
	\end{equation*}
	显然$Q_1$是一个正交矩阵,于是有$Q_1^{-1}Q_1=(Q_1^{-1}\eta_1,Q_1^{-1}\eta_2,\dots,Q_1^{-1}\eta_n)=(e_1,e_2,\dots,e_n)$。注意到:
	\begin{equation*}
		Q_1^{-1}AQ_1=Q_1^{-1}(A\eta_1,A\eta_2,\dots,A\eta_n)=(Q_1^{-1}\lambda\eta_1,Q_1^{-1}A\eta_2,\dots,Q_1^{-1}A\eta_n)
		=
		\begin{pmatrix}
			\lambda_1 & \alpha \\
			\mathbf{0} & B
		\end{pmatrix}
	\end{equation*}
	因为$(Q_1^{-1}AQ_1)^H=Q_1^HA^H(Q_1^{-1})^H=Q_1^{-1}A(Q_1^H)^H=Q_1^{-1}AQ_1$,所以$Q_1^{-1}AQ_1$是一个对称阵,于是$\alpha=\mathbf{0}$,$B$是一个$n-1$阶Hermitian矩阵。由归纳假设,存在$n-1$阶正交矩阵$Q_2$使得$Q_2^{-1}BQ_2=\operatorname{diag}\{\lambda_2,\lambda_3,\dots,\lambda_n\}$。令:
	\begin{equation*}
		Q=Q_1
		\begin{pmatrix}
			1 & \mathbf{0} \\
			\mathbf{0} & Q_2
		\end{pmatrix}
	\end{equation*}
	则:
	\begin{equation*}
		Q^HQ=
		\begin{pmatrix}
			1 & \mathbf{0} \\
			\mathbf{0} & Q_2^H
		\end{pmatrix}
		Q_1^HQ_1
		\begin{pmatrix}
			1 & \mathbf{0} \\
			\mathbf{0} & Q_2
		\end{pmatrix}
		=I
	\end{equation*}
	即$Q$是一个正交矩阵。同时:
	\begin{align*}
		Q^{-1}AQ
		&=
		\begin{pmatrix}
			1 & \mathbf{0} \\
			\mathbf{0} & Q_2^H
		\end{pmatrix}
		Q_1^HAQ_1
		\begin{pmatrix}
			1 & \mathbf{0} \\
			\mathbf{0} & Q_2
		\end{pmatrix}
		=
		\begin{pmatrix}
			1 & \mathbf{0} \\
			\mathbf{0} & Q_2^H
		\end{pmatrix}
		\begin{pmatrix}
			\lambda_1 & \mathbf{0} \\
			\mathbf{0} & B
		\end{pmatrix}
		\begin{pmatrix}
			1 & \mathbf{0} \\
			\mathbf{0} & Q_2
		\end{pmatrix} \\
		&=
		\begin{pmatrix}
			\lambda_1 & \mathbf{0} \\
			\mathbf{0} & Q_2^HBQ_2
		\end{pmatrix}
		=
		\begin{pmatrix}
			\lambda_1 & \mathbf{0} \\
			\mathbf{0} & \operatorname{diag}\{\lambda_2,\lambda_3,\dots,\lambda_n\}
		\end{pmatrix}
		=\operatorname{diag}\{\seq{\lambda}{n}\}
	\end{align*}
	所以$A$正交相似于对角矩阵$\operatorname{diag}\{\seq{\lambda}{n}\}$。由$AQ=Q\operatorname{diag}\{\seq{\lambda}{n}\}$可以得到$\lambda_2,\lambda_3,\dots,\lambda_n$是$A$的特征值。\par
	综上,结论成立。\par
	(4)\textbf{必要性:}正交相似也是相似。\par
	\textbf{充分性:}因为$A$与$B$相似,由\cref{theo:SameEigenvalue}可知$A$与$B$有相同的特征值(包括重数)$\seq{\lambda}{n}$。由(3)可得$A$与$B$都正交相似于$\operatorname{diag}\{\seq{\lambda}{n}\}$(考虑$\lambda_i$的顺序的话只需要更改$Q$中列向量的顺序)。因为正交相似具有对称性与传递性,所以$A$正交相似于$B$。
\end{proof}
\subsubsection{求解正交矩阵$Q$}
\begin{theorem}
	对于Hermitian矩阵$A\in M_{n}(\mathbb{C})$,求正交矩阵$Q$使得$Q^{-1}AQ$为对角阵的步骤如下:
	\begin{enumerate}
		\item 求出$A$的所有特征值$\seq{\lambda}{m}$;
		\item 对于每一个特征值$\lambda_i$,求得其特征子空间的一组基$\alpha_{1i},\alpha_{2i},\dots,\alpha_{r_ii}$,并对它们进行Schimidt正交化与单位化,得到$\eta_{1i},\eta_{2i},\dots,\eta_{r_ii}$;
		\item 令$Q=(\eta_{11},\eta_{21},\dots,\eta_{r_11},\dots,\eta_{r_mm})$,$Q$即为所求。
	\end{enumerate}
\end{theorem}
\begin{proof}
	由\info{Schimidt正交化链接}可知$\eta_{ij},\;i=1,2,\dots,r_j,\;j=1,2,\dots,m$是$A$属于$\lambda_j$的特征值。根据\cref{prop:HermitianMatEigen}(2)可知:
	\begin{align*}
		Q^{-1}AQ
		&=Q^H(A\eta_{11},A\eta_{21},\dots,A\eta_{r_11},\dots,A\eta_{r_mm}) \\
		&=
		\begin{pmatrix}
			\eta_{11}^H \\
			\eta_{21}^H \\
			\vdots \\
			\eta_{r_11}^H \\
			\vdots \\
			\eta_{r_mm}^H
		\end{pmatrix}
		(\lambda_1\eta_{11},\lambda_1\eta_{21},\dots,\lambda_1\eta_{r_11},\dots,\lambda_m\eta_{r_mm}) \\
		&=\operatorname{diag}\{\lambda_1\eta_{11}^H\eta_{11},\lambda_1\eta_{21}^H\eta_{21},\dots,\lambda_1\eta_{r_11}^H\eta_{r_11},\dots,\lambda_m\eta_{r_mm}^H\eta_{r_mm}\} \\
		&=\operatorname{diag}\{\lambda_1,\dots,\lambda_1,\dots,\lambda_m,\dots,\lambda_m\}\qedhere
	\end{align*}
\end{proof}
\subsubsection{实对称矩阵特征值的极值性质}
\begin{theorem}\label{theo:maxminxAx/xx}
	设$A\in M_{n}(\mathbb{R})$,$A$的特征值从大到小记作$\seq{\lambda}{n}$,$\seq{\varphi}{n}$为对应的标准正交化特征向量,则:
	\begin{equation*}
		\max_{x\ne\mathbf{0}}\frac{x^TAx}{x^Tx}=\lambda_1=\varphi_1^TA\varphi_1\quad
		\min_{x\ne\mathbf{0}}\frac{x^TAx}{x^Tx}=\lambda_n=\varphi_n^TA\varphi_n
	\end{equation*}
\end{theorem} 
\begin{proof}
	由\cref{prop:HermitianMatEigen}(3)可知存在一个正交矩阵$Q$使得$Q^{-1}AQ=\operatorname{diag}\{\seq{\lambda}{n}\}=\varLambda$。对任意的$x\in\mathbb{R}^{n}$,因为$Q$为正交矩阵,$Q$可逆,所以关于$y$的非齐次线性方程组$Qy=x$有唯一解,于是对于这个存在且唯一的$y$,有:
	\begin{gather*}
		\frac{x^TAx}{x^Tx}=\frac{y^TQ^TAQy}{y^TQ^TQy}=\frac{y^T\varLambda y}{y^Ty}=\frac{\sum\limits_{i=1}^{n}\lambda_iy_i^2}{\sum\limits_{i=1}^ny_i^2}\leqslant\lambda_1\frac{\sum\limits_{i=1}^{n}y_i^2}{\sum\limits_{i=1}^ny_i^2}=\lambda_1 \\
		\frac{x^TAx}{x^Tx}=\frac{y^TQ^TAQy}{y^TQ^TQy}=\frac{y^T\varLambda y}{y^Ty}=\frac{\sum\limits_{i=1}^{n}\lambda_iy_i^2}{\sum\limits_{i=1}^ny_i^2}\geqslant\lambda_n\frac{\sum\limits_{i=1}^{n}y_i^2}{\sum\limits_{i=1}^ny_i^2}=\lambda_n
	\end{gather*}
	当$y$为$(1,0,0,\dots,0)^T$时第一式取等号,当$y$为$(0,0,\dots,0,1)^T$时第二式取等号,此时$x$分别为$\varphi_1$和$\varphi_n$。
\end{proof}
