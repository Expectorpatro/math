\section{矩阵的分解}

\subsection{QR分解}
\begin{theorem}\label{theo:QR}
	设$A\in M_{n}(\mathbb{R}^{})$为可逆矩阵则存在正交矩阵$Q$和上三角矩阵$R$使得$A=QR$。当$R$的对角元素都为正数时,分解是唯一的。
\end{theorem}
\begin{proof}
	\textbf{(1)存在性:}下面给出两种存在性的证明。\par
	\textbf{Givens变换:}由\cref{theo:Givens}和\cref{prop:OrthogonalUnitaryMatrix}(4)可知存在正交矩阵$Q$使得$QA=R$,其中$R$是上三角矩阵,由\cref{prop:OrthogonalUnitaryMatrix}(1)可得$A=Q^{-1}R$。\par
	\textbf{Householder变换:}由\cref{theo:Householder}可知存在正交矩阵$H_1$使得$H_1A$的第一列仅有第一个元素非$0$。将$H_1A$分块为:
	\begin{equation*}
		H_1A=
		\begin{pmatrix}
			a & \beta^T \\
			\mathbf{0} & A_1
		\end{pmatrix}
	\end{equation*}
	由\cref{theo:Householder}可知存在正交矩阵$H_2'$使得$H_2A_1$的第一列仅有第一个元素非$0$,构造正交矩阵:
	\begin{equation*}
		H_2=
		\begin{pmatrix}
			1 & \mathbf{0} \\
			\mathbf{0} & H_2'
		\end{pmatrix}
	\end{equation*}
	即可得$H_2H_1$使得$A$的前两列变为上三角的状态。依次重复可得到矩阵序列$\{H_n\}$与$\{A_n\}$,由\cref{prop:OrthogonalUnitaryMatrix}(4)可知存在正交矩阵$Q$使得$QA=R$,其中$R$是上三角矩阵,所以$A=Q^{-1}R$。\par
	\textbf{(2)唯一性:}\info{正定矩阵的楚列斯基分解}
\end{proof}
\begin{note}
	计算得到$A=QR$后,因为$A$可逆,所以$R$主对角线上的元素都不为$0$。设$R=(r_{ij})$,令:
	\begin{equation*}
		D=\operatorname{diag}\left\{\frac{r_{11}}{|r_{11}|},\frac{r_{22}}{|r_{22}|},\dots,\frac{r_{nn}}{|r_{nn}|}\right\}
	\end{equation*}
	则$Q'=QD$仍为正交矩阵,$R'=D^{-1}R$为对角元是$|r_{ii}|$的上三角矩阵,那么$A=Q'R'$即为唯一的$QR$分解。
\end{note}

\subsection{SVD分解}
\begin{theorem}\label{theo:AATPositiveSemidefinite}
	设$A\in M_{m\times n}(\mathbb{C})$,则$AA^H,A^HA$是半正定矩阵。
\end{theorem}
\begin{proof}
	设$\lambda_i,\;i=1,2,\dots,n$是矩阵$A^HA$的特征值,$\xi_i$是对应的特征向量,则:
	\begin{align*}
		A^HA\xi_i=\lambda_i\xi_i\rightarrow
		\xi_i^HA^HA\xi_i=\lambda_i\xi_i^H\xi_i\rightarrow
		||A\xi_i||^2=\lambda_i||\xi_i||^2
	\end{align*}
	由于左式非负,所以右式非负,而$||\xi_i||^2$非负,因此$\lambda_i$非负,由\cref{theo:PositiveSemidefinite}(3)的第五条可知$AA^T$是半正定矩阵。
\end{proof}
\begin{theorem}\label{theo:SVD}
	设$A\in M_{m\times n}(\mathbb{C})$,$\operatorname{rank}(A)=r$,则存在两个正交矩阵$P\in M_{m}(\mathbb{C}),\;Q\in M_{n}(\mathbb{C})$使得:
	\begin{equation*}
		A=P
		\begin{pmatrix}
			\varLambda & \mathbf{0} \\
			\mathbf{0} & \mathbf{0}
		\end{pmatrix}Q^H
	\end{equation*}
	其中$\varLambda=\operatorname{diag}\{\lambda_1,\lambda_2,\dots,\lambda_r\}$,$\lambda_i>0$,$\lambda_i^2$为$A^HA$的正特征值。
\end{theorem}
\begin{proof}
	由\cref{prop:MatrixRank}(8)可知$\operatorname{rank}(A^HA)=\operatorname{rank}(A)$。于是$A^HA$确实有$r$个正特征值。因为$A^HA$是一个Hermitian矩阵,由\cref{prop:HermitianMatEigen}可知存在正交矩阵$Q\in M_{n}(\mathbb{C})$使得:
	\begin{equation*}
		Q^HA^HAQ=
		\begin{pmatrix}
			\varLambda^2 & \mathbf{0} \\
			\mathbf{0} & \mathbf{0}
		\end{pmatrix}
	\end{equation*}
	记$B=AQ$,则:
	\begin{equation*}
		B^HB=
		\begin{pmatrix}
			\varLambda^2 & \mathbf{0} \\
			\mathbf{0} & \mathbf{0}
		\end{pmatrix}
	\end{equation*}
	这表明$B$的列向量相互正交,且前$r$个列向量的长度分别为$\lambda_1,\lambda_2,\dots,\lambda_r$,后$n-r$个列向量为零向量,于是存在一个正交矩阵$P\in M_{m}(\mathbb{C})$使得:
	\begin{equation*}
		B=P
		\begin{pmatrix}
			\varLambda & \mathbf{0} \\
			\mathbf{0} & \mathbf{0}
		\end{pmatrix}
	\end{equation*}
	因为$B=AQ$,所以:
	\begin{equation*}
		A=P
		\begin{pmatrix}
			\varLambda & \mathbf{0} \\
			\mathbf{0} & \mathbf{0}
		\end{pmatrix}Q^{-1}
		=P
		\begin{pmatrix}
			\varLambda & \mathbf{0} \\
			\mathbf{0} & \mathbf{0}
		\end{pmatrix}Q^H\qedhere
	\end{equation*}
\end{proof}
\begin{definition}
	设$A\in M_{m\times n}(\mathbb{C})$,$\operatorname{rank}(A)=r$,$A^HA$的正特征值为$\lambda_i,\;i=1,2,\dots,r$,称$\delta_i=\sqrt{\lambda_i}$为矩阵$A$的\gls{SingularValue}。
\end{definition}