\section{矩阵的分解}

\subsection{LU分解}
\begin{definition}
	若下三角矩阵$L\in M_{n}(K)$的主对角线元素均为$1$,其余非零元素仅出现在主对角线下方的某一列(或某一行)中,则称$L$为\gls{GaussMatrix},也称为\gls{GaussTransformation}。
\end{definition}
\begin{lemma}\label{lem:LU}
	设$A\in M_n(K)$。在不进行任何行交换的Gauss消元过程中,前$k$个主元都不为$0$的充要条件是$A$的前$k$阶主子阵$A_1,A_2,\dots,A_k$均可逆。
\end{lemma}
\begin{proof}
	由\cref{prop:Determinant}(7)(12)和\cref{prop:InvertibleMatrix}(3.a)立即可得。
\end{proof}
\begin{theorem}\label{theo:LU}
	设$A\in M_{n}(K)$为可逆矩阵,则存在一个单位下三角矩阵$L$和一个上三角矩阵$U$,使得:
	\begin{equation*}
		PA = LU
	\end{equation*}
	如果约定消去法每步选取第一个非零元为主元,则上述分解是唯一的。不对$A$作行交换,即$P=I_n$时,$A$存在$LU$分解当且仅当所有顺序主子阵$A_1,\dots,A_n$都可逆。
\end{theorem}
\begin{proof}
	\textbf{(1)存在性:} 对$n$进行数学归纳证明。对于$n=1$时,结论显然成立。假设对于任意$(n-1)\times(n-1)$的可逆矩阵都存在$P,L,U$使$P A = L U$成立,下面证明对$n\times n$矩阵也成立。由于$A$可逆,$A$的第一列中必定存在非零元素,设$a_{p1}$是第一个非零元素(位于第$p$行)。通过左乘首行交换矩阵交换第$1$行和第$p$行(这对应一个置换矩阵$P_1$),可以使$A$的$(1,1)$位置成为非零的主元。记变换后的矩阵仍为$A$。接下来,对$i=2,3,\dots,n$,令$l_{i1}=\dfrac{a_{i1}}{a_{11}}$,并用第1行的$l_{i1}$倍消去第$i$行的第1列元素,对应的初等矩阵记为$L_{1i}$。经过这一系列初等行变换,可得到一个新的矩阵$A^{(1)}$:
	\[ A^{(1)} = L_{1n}L_{1(n-1)} \cdots L_{12}P_1 A = 
	\begin{pmatrix}
		a_{11} & u^{\top} \\
		\mathbf{0} & A_1
	\end{pmatrix} \] 
	其中$A_1$是$(n-1)\times(n-1)$矩阵,$\mathbf{0}$表示$(n-1)\times 1$的零向量。由\cref{prop:Rank}(2)可知$A_1$也是可逆的。根据归纳假设,对$A_1$存在置换矩阵$P_2$、单位下三角矩阵$L'$和上三角矩阵$U'$使$P_2 A_1 = L' U'$。构造出分块矩阵:
	\[ 
	P = \begin{pmatrix}
		1 & \mathbf{0} \\
		\mathbf{0} & P_2
	\end{pmatrix}, \qquad 
	L = \begin{pmatrix}
		1 & \mathbf{0} \\
		l & L'
	\end{pmatrix}, \qquad 
	U = \begin{pmatrix}
		a_{11} & u^{\top} \\
		\mathbf{0} & U'
	\end{pmatrix}
	\] 
	其中$l=(l_{21},l_{31},\dots,l_{n1})^{\top}$。则由上述构造和归纳假设可验证:
	\begin{align*}
		&PA = \begin{pmatrix} 1 & \mathbf{0} \\ \mathbf{0} & P_2 \end{pmatrix}
		\begin{pmatrix} a_{11} & u^{\top} \\ \mathbf{0} & A_1 \end{pmatrix}
		= \begin{pmatrix} a_{11} & u^{\top} \\ \mathbf{0} & P_2 A_1 \end{pmatrix} \\
		=&\begin{pmatrix} a_{11} & u^{\top} \\ \mathbf{0} & L' U' \end{pmatrix} 
		= \begin{pmatrix} 1 & \mathbf{0} \\ l & L' \end{pmatrix}
		\begin{pmatrix} a_{11} & u^{\top} \\ \mathbf{0} & U' \end{pmatrix} = L U
	\end{align*} 
	从而完成$PA=LU$的构造。\par
	\textbf{(2)唯一性:}由主元的选取规则可知$P$是不变的,假设还有另一组分解$PA=\tilde{L}\tilde{U}$,其中$\tilde{L}$为单位下三角矩阵、$\tilde{U}$为上三角矩阵,而且满足与$L,U$相同的主元选取规则。因为$A$可逆,由\cref{prop:InvertibleMatrix}(3.a)和\cref{prop:Determinant}(11)可知$L,\tilde{L}$都是可逆的下三角矩阵且$U,\tilde{U}$都是可逆的上三角矩阵,根据$PA=LU=\tilde{L}\tilde{U}$可得:
	\[ L^{-1}\tilde{L} = U \tilde{U}^{-1}\] 
	由\cref{prop:InvertibleMatrix}(18)可知$L^{-1}$和$\tilde{U}^{-1}$分别为单位下三角矩阵和上三角矩阵,根据\cref{prop:MatrixMultiplication}(6)可知上式左侧为一个下三角矩阵,右侧为一个上三角矩阵,因此这个等式两侧实际上都是对角矩阵。因为$\tilde{L}$和$L^{-1}$是单位下三角矩阵,于是$L^{-1}\tilde{L}=I_n$,即$\tilde{L}=L$且$\tilde{U}=U$,所以分解是唯一的。\par
	不对$A$作行交换时的结论由上述分析和\cref{lem:LU}立即可得。
\end{proof}
\begin{method}
	设$PA=(a_{ij})$,因为$L^{-1}PA=U$,由\cref{alg:gauss}和\cref{prop:ElementaryMatrix}(2)可知:
	\begin{equation*}
		L^{-1}=
		\begin{pmatrix}
			1 & & & &  \\
			-l_{21} & 1 & & &  \\
			-l_{31}& -l_{32} & 1 & &  \\
			\vdots & \vdots & \ddots &  &  \\
			-l_{n1} & -l_{n2} & \cdots &-l_{n(n-1)} & 1 \\
		\end{pmatrix},\quad
		l_{ij}=\frac{a_{ij}}{a_{kk}},\;i=2,3,\dots,n,\;j=1,2,\dots,n-1
	\end{equation*}
	根据\cref{prop:InvertibleMatrix}(18)可得:
	\begin{equation*}
		L=
		\begin{pmatrix}
			1 & & & &  \\
			l_{21} & 1 & & &  \\
			l_{31}& l_{32} & 1 & &  \\
			\vdots & \vdots & \ddots &  &  \\
			l_{n1} & l_{n2} & \cdots & l_{n(n-1)} & 1 \\
		\end{pmatrix}
	\end{equation*}
	由\cref{alg:gauss}可知对第$k$列进行消元后,第$k$行的前$k-1$个数都是$0$,不会在后续过程中再使用到它们,于是$L$矩阵中的元素可以直接存储在$A$的下三角区域(除了主对角线),$U$矩阵即为消元后所剩下的上三角矩阵。
	\begin{algorithm}[H]
		\caption{In-place LU Factorization (without pivoting)}
		\begin{algorithmic}[1]
			\Require $A=(a_{ij})\in M_{n}(K)$
			\Ensure The lower triangular part of $A$ (excluding the diagonal) stores $L$,
			and the upper triangular part (including the diagonal) stores $U$
			
			\For{$k = 1$ \textbf{to} $n-1$}
				\For{$i = k+1$ \textbf{to} $n$}
					\State $a_{ik} \gets a_{ik} / a_{kk}$ \Comment{$l_{ik}$}
					\For{$j = k+1$ \textbf{to} $n$}
						\State $a_{ij} \gets a_{ij} - a_{ik} \cdot a_{kj}$
					\EndFor
				\EndFor
			\EndFor
		\end{algorithmic}
	\end{algorithm}
	该方法的计算复杂度为:
	\begin{align*}
		&\sum_{k=1}^{n-1}[2(n-k)\times(n-k)+(n-k)]=\sum_{k=1}^{n-1}[2n^2-4nk+2k^2+n-k] \\
		=&(n-1)2n^2+(n-1)n-(4n+1)\sum_{k=1}^{n-1}k+2\sum_{k=1}^{n-1}k^2 \\
		=&(n-1)2n^2+(n-1)n-(4n+1)\frac{(n-1+1)(n-1)}{2}+2\frac{(n-1)(n-1+1)(2n-2+1)}{6} \\
		=&\frac{2}{3}n^3-\frac{1}{2}n^2-\frac{1}{6}n=\frac{2}{3}n^3+\operatorname{O}(n^2)
	\end{align*}
\end{method}
\subsubsection{应用}
\begin{method}
	设矩阵$A\in M_{n}(K)$,计算$A$的行列式只需对$A$作LU分解,令$U=(u_{ij})$,由\cref{prop:Determinant}(11)(12)即可得到$\det A=\prod\limits_{i=1}^{n}u_{ii}$。为了避免$n$较大时出现数值溢出的情况,实际上我们一般计算的是$\ln|\det A|=\sum\limits_{i=1}^{n}\ln|u_{ii}|$,记录下符号再通过对数值作变换即可恢复原始数值。
\end{method}
\begin{method}
	设可逆矩阵$A\in M_{n}(K)$,使用LU分解和\cref{prop:InvertibleMatrix}(15)即可计算得到$A^{-1}$。
\end{method}
\begin{method}
	对于关于可逆矩阵$A\in M_{n}(K)$的线性方程组$Ax=b$,若将$A$分解为$A=LU$,即一个下三角矩阵$L$和一个上三角矩阵$U$的乘积,那么上述线性方程组可由如下方法得到:
	\begin{enumerate}
		\item 用前代法解$Ly=b$;
		\item 用回代法解$Ux=y$。
	\end{enumerate}\par
	关于数域$K$上可逆$n$阶上三角矩阵或下三角矩阵:
	\begin{equation*}
		L=
		\begin{pmatrix}
			l_{11} & & & &  \\
			l_{21} & l_{22} & & &  \\
			l_{31} & l_{32} & l_{33} & &  \\
			\vdots & \vdots & \vdots & \ddots &  \\
			l_{n1} & l_{n2} & l_{n3} & \cdots & l_{nn}
		\end{pmatrix},\quad
		U=
		\begin{pmatrix}
			u_{11} & u_{12} & u_{13} & \cdots & u_{1n} \\
			& u_{22} & u_{23} & \cdots & u_{2n} \\
			&       & u_{33} & \cdots & u_{3n} \\
			&  &  & \ddots & \vdots \\
			&       &       &  & u_{nn}
		\end{pmatrix}
	\end{equation*}
	的线性方程组$Lx=b$或$Ux=b$,有下述两种求解算法:
	\begin{algorithm}
		\caption{Forward Substitution for Lower Triangular Systems}
		\begin{algorithmic}[1]
			\Require Lower triangular matrix $L=(l_{ij})\in M_{n}(\mathbb{K})$, right-hand side vector $b\in\mathbb{K}^n$
			\Ensure Solution vector $x\in\mathbb{K}^n$ satisfying $Lx=b$
			
			\For{$i=1$ \textbf{to} $n$}
				\State $x_i \gets \dfrac{b_i - \sum\limits_{j=1}^{i-1} l_{ij}x_j}{l_{ii}}$
			\EndFor
		\end{algorithmic}
	\end{algorithm}
	\begin{algorithm}
		\caption{Backward Substitution for Upper Triangular Systems}
		\begin{algorithmic}[1]
			\Require Upper triangular matrix $U=(u_{ij})\in M_{n}(K)$, right-hand side vector $b\in\mathbb{K}^n$
			\Ensure Solution vector $x\in\mathbb{K}^n$ satisfying $Ux=b$
			
			\For{$i=n$ \textbf{downto} $1$}
				\State $x_i \gets \dfrac{b_i - \sum\limits_{j=i+1}^{n} u_{ij}x_j}{u_{ii}}$
			\EndFor
		\end{algorithmic}
	\end{algorithm}\par
	上述两种算法的计算复杂度分别为:
	\begin{gather*}
		\sum_{i=1}^{n}(1+i-1+i-1-1+1)=\sum_{i=1}^{n}(2i-1)=\frac{(1+2n-1)n}{2}=n^2 \\
		\sum_{i=1}^{n}(1+n-i+n-i-1+1)=\sum_{i=1}^{n}(2n-2i+1)=2n^2-n^2=n^2
	\end{gather*}
\end{method}

\subsection{Cholesky分解}
\begin{theorem}\label{theo:Cholesky}
	设$A\in M_{n}(\mathbb{R}^{})$是对称正定矩阵,则存在唯一一个可逆的下三角矩阵$L$使得$A=LL^{\top}$,其中$L$的对角线元素均为正数。
\end{theorem}
\begin{proof}
	由\cref{theo:PositiveDefinite}(7)、\cref{prop:InvertibleMatrix}(3.a)和\cref{theo:LU}可知$A$存在唯一的LU分解$A=LU$。令$U=(u_{ij})$,由\cref{theo:PositiveDefinite}(6)、\cref{prop:Determinant}(11)(12)和\cref{prop:InvertibleMatrix}(3.a)可知存在$D=\operatorname{diag}\{u_{11},u_{22},\dots,u_{nn}\}$的逆矩阵$D^{-1}$。令$\tilde{U}=D^{-1}U$,于是由\cref{prop:Transpose}(4)可得:
	\begin{equation*}
		\tilde{U}^{\top}DL^{\top}=A^{\top}=A=LD\tilde{U}
	\end{equation*}
	由\cref{prop:Determinant}(11)(12)和\cref{prop:InvertibleMatrix}(3.a)可知$\tilde{U}$可逆,于是:
	\begin{equation*}
		L^{\top}\tilde{U}^{-1}=D^{-1}(\tilde{U}^{\top})^{-1}LD
	\end{equation*}
	根据\cref{prop:InvertibleMatrix}(18)和\cref{prop:MatrixMultiplication}(6)可知上式左侧为一个单位上三角矩阵,上式右侧为一个下三角矩阵,所以有:
	\begin{equation*}
		L^{\top}\tilde{U}^{-1}=D^{-1}(\tilde{U}^{\top})^{-1}LD=I_n
	\end{equation*}
	于是$L^{\top}=\tilde{U}$,即:
	\begin{equation*}
		A=LDL^{\top}
	\end{equation*}
	因为$A$正定,所以$D$主对角线上的元素都为正数。令$\tilde{L}=L\operatorname{diag}\{\sqrt{u_{11}},\sqrt{u_{22}},\dots,\sqrt{u_{nn}}\}$,则$A=\tilde{L}\tilde{L}^{\top}$。由\cref{prop:InvertibleMatrix}(3.a)可知$\tilde{L}$可逆。
\end{proof}
\begin{method}
	为求$L=(l_{ij})$使得$A=(a_{ij})=LL^{\top}$,将等式两边展开:
	\begin{equation*}
		a_{ij}=\sum_{k=1}^{\min(i,j)}l_{ik}l_{jk}
	\end{equation*}
	当$i=j$时有:
	\begin{equation*}
		a_{ii}=\sum_{k=1}^{i}l_{ik}^2=l_{ii}^2+\sum_{k=1}^{i-1}l_{ik}^2
	\end{equation*}
	故有:
	\begin{equation*}
		l_{ii}=\sqrt{a_{ii}-\sum_{k=1}^{i-1}l_{ik}^2}
	\end{equation*}
	当$i>j$时有:
	\begin{equation*}
		a_{ij}=\sum_{k=1}^{j}l_{ik}l_{jk}=\sum_{k=1}^{j-1}l_{ik}l_{jk}+l_{ij}l_{jj}
	\end{equation*}
	由此得:
	\begin{equation*}
		l_{ij}=\frac{1}{l_{jj}}\left(a_{ij}-\sum_{k=1}^{j-1}l_{ik}l_{jk}\right)
	\end{equation*}
	根据上面的公式我们可以逐列求解$L$。
	\begin{algorithm}[H]
		\caption{Cholesky Factorization (Comparison Method)}
		\begin{algorithmic}[1]
			\Require $A=(a_{ij})\in M_{n}(\mathbb{R})$, symmetric and positive definite
			\Ensure The lower triangular part of $A$ stores $L$
			\For{$j = 1$ \textbf{to} $n$}
				\State $l_{jj}\gets\sqrt{a_{jj}-\sum\limits_{k=1}^{j-1}l_{jk}^2}$
				\For{$i = j+1$ \textbf{to} $n$}
						\State $l_{ij}\gets\dfrac{1}{l_{jj}}\left(a_{ij}-\sum\limits_{k=1}^{j-1}l_{ik}l_{jk}\right)$
				\EndFor
			\EndFor
		\end{algorithmic}
	\end{algorithm}
	在程序实现中,可直接在$A$的下三角部分存储$L$的元素,上三角部分不再使用。\par
	上述方法的计算复杂度(不考虑开方运算)为:
	\begin{align*}
		&\sum_{j=1}^{n}\left[1+j-1+j-2+\sum_{i=j+1}^{n}(2+j-1+j-2)\right]=\sum_{j=1}^{n}\left[2j-2+\sum_{i=j+1}^{n}(2j-1)\right] \\
		=&\sum_{j=1}^{n}(2j-2)+\sum_{j=1}^{n}(n-j)(2j-1)=\sum_{j=1}^{n}[-2j^2+(2n+1)j-n]+\sum_{j=1}^{n}(2j-2) \\
		=&\frac{1}{3}n^3-\frac{1}{2}n^2-\frac{11}{6}n=\frac{1}{3}n^3+\operatorname{O}(n^2)
	\end{align*}
	由\info{计算机求根号,牛顿公式}可知即使算上开方运算,也只是再加上$\operatorname{O}(n)$次运算,所以Cholesky分解计算量约为LU分解的一半。
\end{method}


\subsection{QR分解}
\begin{theorem}\label{theo:QR}
	设$A\in M_{n}(\mathbb{R}^{})$为可逆矩阵则存在正交矩阵$Q$和上三角矩阵$R$使得$A=QR$。当$R$的对角元素都为正数时,分解是唯一的。
\end{theorem}
\begin{proof}
	\textbf{(1)存在性:}下面给出两种存在性的证明。\par
	\textbf{Givens变换:}由\cref{theo:Givens}、\cref{prop:Givens}(1)和\cref{prop:OrthogonalUnitaryMatrix}(4)可知存在正交矩阵$Q$使得$QA=R$,其中$R$是上三角矩阵,由\cref{prop:OrthogonalUnitaryMatrix}(1)可得$A=Q^{-1}R$。\par
	\textbf{Householder变换:}由\cref{theo:Householder}可知存在正交矩阵$H_1$使得$H_1A$的第一列仅有第一个元素非$0$。将$H_1A$分块为:
	\begin{equation*}
		H_1A=
		\begin{pmatrix}
			a & \beta^{\top} \\
			\mathbf{0} & A_1
		\end{pmatrix}
	\end{equation*}
	由\cref{theo:Householder}可知存在正交矩阵$H_2'$使得$H_2A_1$的第一列仅有第一个元素非$0$,构造正交矩阵:
	\begin{equation*}
		H_2=
		\begin{pmatrix}
			1 & \mathbf{0} \\
			\mathbf{0} & H_2'
		\end{pmatrix}
	\end{equation*}
	即可得$H_2H_1$使得$A$的前两列变为上三角的状态。依次重复可得到矩阵序列$\{H_n\}$与$\{A_n\}$,由\cref{prop:OrthogonalUnitaryMatrix}(4)可知存在正交矩阵$Q$使得$QA=R$,其中$R$是上三角矩阵,所以$A=Q^{-1}R$。\par
	\textbf{(2)唯一性:}如果$A=QR=Q'R'$是两种QR分解,其中$Q,Q'$都是正交矩阵,$R,R'$都是上三角矩阵。由\cref{prop:OrthogonalUnitaryMatrix}(1)可得$Q'^{-1}=Q'^{\top}$,根据$R$的对角元素都为正数、\cref{prop:Determinant}(12)和\cref{prop:InvertibleMatrix}(3.a)可得$Q'^{\top}Q=R'R^{-1}$。由\cref{prop:OrthogonalUnitaryMatrix}(4)可知$Q'^{\top}Q$是正交矩阵,根据\cref{prop:InvertibleMatrix}(18)和\cref{prop:MatrixMultiplication}(6)可知$R'R^{-1}$是上三角矩阵,由\cref{prop:OrthogonalUnitaryMatrix}(6)可知$Q'^{\top}Q$为对角矩阵且主对角线元素模为$1$,也即存在一个对角矩阵$D=(\pm1,\pm1,\dots,\pm1)$使得$R'R^{-1}=D$,所以$R'=DR$。当$R$的对角元素均为正值时,上述$D$只能取单位矩阵,因此$Q'=Q,\;R'=R$,唯一性得证。
\end{proof}
\begin{note}
	计算得到$A=QR$后,因为$A$可逆,所以$R$主对角线上的元素都不为$0$。设$R=(r_{ij})$,令:
	\begin{equation*}
		D=\operatorname{diag}\left\{\frac{r_{11}}{|r_{11}|},\frac{r_{22}}{|r_{22}|},\dots,\frac{r_{nn}}{|r_{nn}|}\right\}
	\end{equation*}
	则$Q'=QD$仍为正交矩阵,$R'=D^{-1}R$为对角元是$|r_{ii}|$的上三角矩阵,那么$A=Q'R'$即为唯一的$QR$分解。
\end{note}

\subsection{SVD分解}
\begin{theorem}\label{theo:AATPositiveSemidefinite}
	设$A\in M_{m\times n}(\mathbb{C})$,则$AA^H,A^HA$是半正定矩阵。
\end{theorem}
\begin{proof}
	设$\lambda_i,\;i=1,2,\dots,n$是矩阵$A^HA$的特征值,$\xi_i$是对应的特征向量,则:
	\begin{align*}
		A^HA\xi_i=\lambda_i\xi_i\rightarrow
		\xi_i^HA^HA\xi_i=\lambda_i\xi_i^H\xi_i\rightarrow
		||A\xi_i||^2=\lambda_i||\xi_i||^2
	\end{align*}
	由于左式非负,所以右式非负,而$||\xi_i||^2$非负,因此$\lambda_i$非负,由\cref{theo:PositiveSemidefinite}(3.5)可知$AA^{\top}$是半正定矩阵。
\end{proof}
\begin{theorem}\label{theo:SVD}
	设$A\in M_{m\times n}(\mathbb{C})$,$\operatorname{rank}(A)=r$,则存在两个正交矩阵$P\in M_{m}(\mathbb{C}),\;Q\in M_{n}(\mathbb{C})$使得:
	\begin{equation*}
		A=P
		\begin{pmatrix}
			\varLambda & \mathbf{0} \\
			\mathbf{0} & \mathbf{0}
		\end{pmatrix}Q^H
	\end{equation*}
	其中$\varLambda=\operatorname{diag}\{\lambda_1,\lambda_2,\dots,\lambda_r\}$,$\lambda_i>0$,$\lambda_i^2$为$A^HA$的正特征值。
\end{theorem}
\begin{proof}
	由\cref{prop:MatrixRank}(8)可知$\operatorname{rank}(A^HA)=\operatorname{rank}(A)$。于是$A^HA$确实有$r$个正特征值。因为$A^HA$是一个Hermitian矩阵,由\cref{prop:HermitianMatEigen}可知存在正交矩阵$Q\in M_{n}(\mathbb{C})$使得:
	\begin{equation*}
		Q^HA^HAQ=
		\begin{pmatrix}
			\varLambda^2 & \mathbf{0} \\
			\mathbf{0} & \mathbf{0}
		\end{pmatrix}
	\end{equation*}
	记$B=AQ$,则:
	\begin{equation*}
		B^HB=
		\begin{pmatrix}
			\varLambda^2 & \mathbf{0} \\
			\mathbf{0} & \mathbf{0}
		\end{pmatrix}
	\end{equation*}
	这表明$B$的列向量相互正交,且前$r$个列向量的长度分别为$\lambda_1,\lambda_2,\dots,\lambda_r$,后$n-r$个列向量为零向量,于是存在一个正交矩阵$P\in M_{m}(\mathbb{C})$使得:
	\begin{equation*}
		B=P
		\begin{pmatrix}
			\varLambda & \mathbf{0} \\
			\mathbf{0} & \mathbf{0}
		\end{pmatrix}
	\end{equation*}
	因为$B=AQ$,所以:
	\begin{equation*}
		A=P
		\begin{pmatrix}
			\varLambda & \mathbf{0} \\
			\mathbf{0} & \mathbf{0}
		\end{pmatrix}Q^{-1}
		=P
		\begin{pmatrix}
			\varLambda & \mathbf{0} \\
			\mathbf{0} & \mathbf{0}
		\end{pmatrix}Q^H\qedhere
	\end{equation*}
\end{proof}
\begin{definition}
	设$A\in M_{m\times n}(\mathbb{C})$,$\operatorname{rank}(A)=r$,$A^HA$的正特征值为$\lambda_i,\;i=1,2,\dots,r$,称$\delta_i=\sqrt{\lambda_i}$为矩阵$A$的\gls{SingularValue}。
\end{definition}