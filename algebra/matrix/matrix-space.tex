\section{矩阵空间}
\begin{definition}
	由$m\times n$个数排成$m$行、$n$列的一张表称为一个$m\times n$\gls{Matrix},其中的每一个数称为这个矩阵的一个元素,第$i$行与第$j$列交叉位置的元素称为矩阵的$(i,j)$元,记作$A(i;j)$。一个$m\times n$矩阵可以简单地记作$A_{m\times n}$。如果矩阵$A$的$(i,j)$元是$a_{ij}$,那么可以记作$A=(a_{ij})$。如果一个矩阵的行数和列数相同,则称它为\gls{SquareMatrix},$n$行$n$列的方阵也成为$n$阶矩阵。对于两个矩阵$A$和$B$,如果它们的行数都等于$m$且列数都等于$n$,同时还有$A(i;j)=B(i;j),i=1,2,\dots,m,\;j=1,2,\dots,n$,那么称$A$和$B$相等,记作$A=B$。
\end{definition}
\begin{definition}
	称主对角线元素都为$1$其他位置元素都为$0$的$n$阶方阵为$n$阶\gls{IdentitMatrix},记为$I_n$。
\end{definition}

\subsection{矩阵的运算}
\subsubsection{加减法与数量乘法}
\begin{definition}
	将数域$K$上所有$m\times n$矩阵组成的集合记作$M_{m\times n}(K)$,当$m=n$时,$M_{m\times m}(K)$可以简记作$M_m(K)$。在$M_{s\times m}(K)$中定义如下运算:
	\begin{enumerate}
		\item \textbf{加法:} 
		\begin{equation*}
			\forall\;A=(a_{ij}),B=(b_{ij})\in M_{s\times m}(K),\;A+B=(a_{ij}+b_{ij})
		\end{equation*}
		\item \textbf{纯量乘法:}
		\begin{equation*}
			\forall\;k\in K,\;\forall\;A=(a_{ij}),\;kA=(ka_{ij})
		\end{equation*}
	\end{enumerate}
	那么$M_{m\times n}(K)$构成一个线性空间。
\end{definition}
\begin{proof}
	首先证明如上定义的加法和纯量乘法对$M_{m\times n}(K)$是封闭的。由数域中加法和乘法的封闭性,$a_{ij}+b_{ij}\in K,\;ka_{ij}\in K,\;i=1,2,\dots,m,\;j=1,2,\dots,n$,所以如上定义的加法与纯量乘法对$M_{m\times n}(K)$是封闭的。\par
	接下来证明如上定义的加法和纯量乘法满足线性空间中的8条运算法则:
	\begin{enumerate}
		\item 因为数域内的数满足加法交换律与加法结合律,所以$M_{m\times n}(K)$上的加法满足线性空间运算法则(1)(2);
		\item 取一个元素全为$0$的$m\times n$矩阵,将其记作$\mathbf{0}$,显然对$\forall\;A\in M_{m\times n}(K)$,有$A+\mathbf{0}=A$,因此$M_{m\times n}(K)$中存在零元且它就是$\mathbf{0}$,称其为\gls{ZeroMatrix}。因此,$M_{m\times n}(K)$上的加法满足线性空间运算法则(3);
		\item 对任意的$A\in M_{m\times n}(K)$,取$-A=(-a_{ij})$,则有$A+(-A)=(a_{ij}-a_{ij})=\mathbf{0}$。由$A$的任意性,$M_{m\times n}(K)$中的每个元素都具有负元,将$A$的负元就记作$-A$。因此,$M_{m\times n}(K)$上的加法满足线性空间运算法则(4);
		\item 因为数域内的数满足乘法结合律和乘法分配律,同时它们乘$1$的积是自身,所以$M_{m\times n}$上的纯量乘法满足线性空间运算法则(5)(6)(7)(8)。
	\end{enumerate}
	证明完毕。
\end{proof}
\begin{definition}
	定义$M_{m\times n}(K)$上矩阵的减法如下:设$A,B\in M_{m\times n}(K)$,则:
	\begin{equation*}
		A-B\coloneq A+(-B)
	\end{equation*}
\end{definition}
\subsubsection{乘法}
\begin{definition}
	设$A=(a_{ij})_{s\times n}\in M_{s\times n}(K),\;B=(b_{ij})_{n\times m}\in M_{n\times m}(K)$,令$C=(c_{ij})_{s\times m}\in M_{s\times m}(K)$,其中:
	\begin{equation*}
		c_{ij}=\sum_{k=1}^{n}a_{ik}b_{kj},\;i=1,2,\dots,s,\;j=1,2,\dots,m
	\end{equation*}
	则称$C$为矩阵$A$与$B$的乘积,记作$C=AB$。
\end{definition}
\begin{property}\label{prop:MatrixMultiplication}
	矩阵乘法具有如下性质:
	\begin{enumerate}
		\item 设$A=(a_{ij})\in M_{m\times n}(K),\;B=(b_{ij})\in M_{n\times p}(K)$,则$AB$有如下四种理解:
		\begin{gather*}
			A=(\alpha_1^T;\alpha_2^T;\dots;\alpha_m^T),\;B=(\seq{\beta}{p})\longrightarrow AB=(a_i^Tb_j) \\
			A=(\seq{\alpha}{n})\longrightarrow AB=\left(\sum_{i=1}^{n}b_{i1}\alpha_i,\sum_{i=1}^{n}b_{i2}\alpha_i,\dots,\sum_{i=1}^{n}b_{ip}\alpha_i\right) \\
			B=(\beta_1^T;\beta_2^T;\beta_n^T)\longrightarrow AB=\left(\sum_{j=1}^{n}a_{1i}\beta_i^T;\sum_{j=1}^{n}a_{2i}\beta_i^T;\dots;\sum_{j=1}^{n}a_{mi}\beta_i^T\right) \\
			A=(\seq{\alpha}{n}),\;B=(\beta_1^T;\beta_2^T;\beta_n^T)\longrightarrow AB=\sum_{i=1}^{n}\alpha_i\beta_i^T
		\end{gather*}
		\item \textbf{结合律:}设$A\in M_{m\times n}(K),\;B\in M_{n\times p}(K),\;C\in M_{p\times q}(K)$,则$(AB)C=A(BC)$;
		\item \textbf{分配律:}设$A\in M_{m\times n}(K),\;B\in M_{n\times p}(K),\;C\in M_{n\times p}(K),\;D\in M_{p\times q}(K)$,则$A(B+C)=AB+AC,\;(B+C)D=BD+CD$;
		\item 设$A\in M_{m\times n}(K),\;B\in M_{n\times p}(K),\;k\in K$,则$k(AB)=(kA)B=A(kB)$;
		\item 设$A\in M_{m\times n}(K)$,则$AI_n=I_mA=A$。
	\end{enumerate}
\end{property}
\begin{proof}
	证明过于机械,略去。
\end{proof}
\begin{definition}
	根据\cref{prop:MatrixMultiplication}(2),定义$m$阶方阵的非负幂整数幂如下:
	\begin{equation*}
		A^0=I_m,\quad A^n=\underbrace{A\cdot A \cdots A}_{n\text{个}A},\;n\in\mathbb{N}^+
	\end{equation*}
\end{definition}
\subsubsection{转置}
\begin{definition}
	设$A\in M_{m\times n}(K)$,定义$A$的\gls{Transpose}$A^T\in M_{n\times m}(K)$,它的第$i$行是$A$的第$i$列,第$j$列是$A$的第$j$行。若$K=\mathbb{C}^{}$,称$A^H=\overline{A}^T$为$A$的Hermitian转置或\gls{ConjugateTranspose},它是对$A$的每个元素先取复共轭再经过转置后得到的矩阵。
\end{definition}
\begin{property}\label{prop:Transpose}
	转置与共轭转置具有如下性质:
	\begin{enumerate}
		\item $A^H=\overline{A}^T=\overline{A^T}$;
		\item $(A^H)^H=A$,$(A^T)^T=A$;
		\item $(A+B)^H=A^H+B^H$,$(A+B)^T=A^T+B^T$;
		\item $(AB)^H=B^HA^H$,$(AB)^T=B^TA^T$。
	\end{enumerate}
\end{property}
\begin{proof}
	证明过于机械,略去。
\end{proof}
\begin{definition}
	若$A^T=A$,则称$A$为\gls{SymmetricMatrix}。若$A^H=A$,则称$A$为\gls{HermitianMatrix}。
\end{definition}
\subsubsection{初等变换}
\begin{definition}
	称以下变换为矩阵的\gls{ElementaryRowOperation}:
	\begin{enumerate}
		\item 把一行的倍数加到另一行上;
		\item 互换两行的位置;
		\item 用一个非零数乘某一行。
	\end{enumerate}
	称以下变换为矩阵的\gls{ElementaryColumnOperation}:
	\begin{enumerate}
		\item 把一列的倍数加到另一列上;
		\item 互换两列的位置;
		\item 用一个非零数乘某一列。
	\end{enumerate}
\end{definition}
\begin{definition}
	将由单位矩阵经过一次初等行(列)变换后得到的矩阵称为\gls{ElemetaryMatrix}。
\end{definition}
\begin{property}\label{prop:ElementaryMatrix}
	$n$阶初等矩阵具有如下性质:
	\begin{enumerate}
		\item 初等矩阵有且仅有三种类型,分别记为:
		\begin{gather*}
			P[j,i(k)]:\text{将$I_n$的第$j$行加上第$i$行的$k$倍,或将$I_n$的第$i$列加上第$j$列的$k$倍} \\
			P[i,j]:\text{交换$I_n$的第$i$行和第$j$行,或交换$I_n$的第$i$列和第$j$列} \\
			P[i(c)]:\text{将$I_n$的第$i$行乘非零常数$c$,或将$I_n$的第$i$列乘非零常数$c$}
		\end{gather*}
		\item 用初等矩阵左乘一个矩阵,就相当于对矩阵作从单位矩阵得到该初等矩阵的初等行变换;用初等矩阵右乘一个矩阵,就相当于对矩阵作从单位矩阵得到该初等矩阵的初等列变换;
		\item 第二种初等矩阵$P[i,j]$可以由第一种和第三种表示,即矩阵的第二种初等变换可以由一系列其它两种初等变换得到。
	\end{enumerate}
\end{property}
\begin{proof}
	(1)(2)证明过于机械,略去。\par
	(3)只需注意到$P[i,j]=P[i(-1)]P[i,j(-1)]P[j,i(1)]P[i,j(-1)]$。由(2)即可得到矩阵的第二种初等变换可以由一系列其它两种初等变换得到。
\end{proof}
\subsubsection{迹}
\begin{definition}
	$A=(a_{ij})\in M_{n}(K)$的主对角线上的元素之和称为$A$的\gls{Trace},记作$\operatorname{tr}(A)$,即:
	\begin{equation*}
		\operatorname{tr}(A)=\sum_{i=1}^{n}a_{ii}
	\end{equation*}
\end{definition}
\begin{property}\label{prop:Trace}
	设$A,B\in M_{n}(K),\;k\in K$,矩阵的迹具有如下性质:
	\begin{enumerate}
		\item $\operatorname{tr}(A+B)=\operatorname{tr}(A)+\operatorname{tr}(B)$;
		\item $\operatorname{tr}(kA)=k\operatorname{tr}(A)$;
		\item $\operatorname{tr}(AB)=\operatorname{tr}(BA)$。
	\end{enumerate}
\end{property}
\begin{proof}
	(1)(2)是显然的;\par
	(3)显然:
	\begin{gather*}
		\operatorname{tr}(AB)=\sum_{i=1}^{n}\sum_{j=1}^{n}a_{ij}b_{ji}=\sum_{j=1}^{n}\sum_{i=1}^{n}b_{ji}a_{ij}=\operatorname{tr}(BA)\qedhere
	\end{gather*}
\end{proof}

\subsection{矩阵的行列式}
\subsubsection{排列}
\begin{definition}
	$n$个不同的正整数的一个全排列称为一个\gls{NPermutation}。
\end{definition}
\begin{definition}
	在$n$元排列$a_1a_2\cdots a_n$中,从左到右任取一对数$a_ia_j(i<j)$,若$a_i<a_j$,则称这一对数构成一个\gls{NaturalOrder};若$a_i>a_j$,则称这一对数构成一个\gls{Inversion}。
\end{definition}
\begin{definition}
	一个$n$元排列$a_1a_2\cdots a_n$中逆序的总数称为\gls{NumberOfInversion},记作$\tau(a_1a_2\cdots a_n)$。逆序数为奇数的排列称为\gls{OddPermutation},逆序数为偶数的排列称为\gls{EvenPermutation}。
\end{definition}
\begin{definition}
	将一个排列$a_1a_2\cdots a_n$中第$i$个位置的元素$a_i$与第$j$个位置的元素$a_j$交换位置的运算称为\gls{Transposition},记作$(a_i,a_j)$。
\end{definition}
\begin{property}\label{prop:Transposition}
	$n$元排列的对换具有如下性质:
	\begin{enumerate}
		\item 对换改变$n$元排列的奇偶性;
		\item 任一$n$元排列可以经过一系列对换变为逆序数为$0$的排列,且所作的对换的次数与这个$n$元排列具有相同的奇偶性;
		\item 在所有由正整数$\seq{a}{n}(n>1)$构成的$n$元排列中,偶排列数等于奇排列数;
		\item 设$c_1c_2\cdots c_kd_1d_2\cdots d_{n-k}$是由$1,2,\dots,n$构成的一个$n$元排列,则:
		\begin{equation*}
			(-1)^{\tau(c_1c_2\cdots c_kd_1d_2\cdots d_{n-k})}=(-1)^{\tau(c_1c_2\cdots c_k)+\tau(d_1d_2\cdots d_{n-k})}\cdot(-1)^{c_1+c_2+\cdots+c_k}\cdot(-1)^{\frac{k(k+1)}{2}}
		\end{equation*}
	\end{enumerate}
\end{property}
\begin{proof}
	(1)任取一个$n$元排列$a_1a_2\cdots a_n$,若对换$(a_i,a_j)$的两个数$a_i,a_j$相邻,即:
	\begin{equation*}
		a_1a_2\cdots a_ia_j\cdots a_n\longrightarrow a_1a_2\cdots a_ja_i\cdots a_n
	\end{equation*}
	这个对换只改变了$a_i$与$a_j$构成的数对的顺逆序关系,于是对换前后排列的奇偶性相反。\par
	对于一般情况:
	\begin{equation*}
		a_1a_2\cdots a_ik_1k_2\cdots k_sa_j\cdots a_n\longrightarrow a_1a_2\cdots a_jk_2\cdots k_sa_i\cdots a_n
	\end{equation*}
	这个对换只需作$2s+1$次相邻数的对换即可达到:
	\begin{equation*}
		(a_i,k_1),\;(a_i,k_2),\;\dots,(a_i,k_s),\;(a_i,a_j),\;(k_s,a_j),\;(k_{s-1},a_j)\;\dots,(k_1,a_j)
	\end{equation*}
	所以对换前后排列的奇偶性相反。\par
	综上,对换改变$n$元排列的奇偶性。\par
	(2)满足逆序数为$0$的排列是一个偶排列,于是由(1)立即可得出结论。\par
	(3)将所有奇排列构成的集合记作$A_n$,偶排列构成的集合记作$B_n$。由(1)可知$f:(a_1,a_2)$是一个$A_n$与$B_n$之间的双射,所以奇排列数等于偶排列数。\par
	(4)设$c_1c_2\cdots c_k$经过$s$次对换得到$a_1a_2\cdots a_k$,其中$\tau(a_1a_2\cdots a_k)=0$,由(2)可得:
	\begin{align*}
		&(-1)^{\tau(c_1c_2\cdots c_kd_1d_2\cdots d_{n-k})}=(-1)^s(-1)^{\tau(a_1a_2\cdots a_kd_1d_2\cdots d_{n-k})} \\
		=&(-1)^{\tau(c_1c_2\cdots c_k)}(-1)^{\tau(d_1d_2\cdots d_{n-k})}(-1)^{(a_1-1)+(a_2-1)+\cdots+(a_k-1)} \\
		=&(-1)^{\tau(c_1c_2\cdots c_k)+\tau(d_1d_2\cdots d_{n-k})}\cdot(-1)^{c_1+c_2+\cdots+c_k}\cdot(-1)^{\frac{k(k+1)}{2}}\qedhere
	\end{align*}
\end{proof}
\subsubsection{行列式的定义与性质}
\begin{definition}
	定义$A=(a_{ij})\in M_{n}(K)$的\gls{Determinant}$\det A$为:
	\begin{equation*}
		\begin{vmatrix}
			a_{11} & a_{12} & \cdots & a_{1n} \\
			a_{21} & a_{22} & \cdots & a_{2n} \\
			\vdots & \vdots & \ddots & \vdots \\
			a_{n1} & a_{n2} & \cdots & a_{nn}
		\end{vmatrix}=
		\sum_{j_1j_2\cdots j_n}^{}(-1)^{\tau(j_1j_2\cdots j_n)}a_{1j_1}a_{2j_2}\cdots a_{nj_n}
	\end{equation*}
	其中$j_1j_2\cdots j_n$是由$1$到$n$构成的$n$元排列,$\det A$也记作$|A|$。
\end{definition} 
\begin{definition}
	设$A\in M_{n}(K)$。在$A$中任意取定$k(1\leqslant k<n)$行($i_1,i_2,\dots,i_k$)、$k$列($j_1,j_2,\dots,j_k$),位于这些行和列交叉处的$k^2$个元素按原来的次序组成的$k$阶矩阵的行列式称为$A$的一个\gls{KMinor},记作:
	\begin{equation*}
		A\left\{ \begin{array}{l}
			i_1,i_2,\dots,i_k \\
			j_1,j_2,\dots,j_k
		\end{array} \right\}
	\end{equation*}
	其余元素按原来的次序组成的$n-k$阶矩阵的行列式称为上式的\gls{Minor},称:
	\begin{equation*}
		(-1)^{(i_1+i_2+\cdots+i_k)+(j_1+j_2+\cdots+j_k)}A\left\{ \begin{array}{l}
			i_1',i_2',\dots,i_{n-k}' \\
			j_1',j_2',\dots,j_{n-k}'
		\end{array} \right\}
	\end{equation*}
	为其\gls{Cofactor},其中:
	\begin{gather*}
		\{i_1',i_2',\dots,i_{n-k}'\}=\{1,2,\dots,n\}\backslash\{i_1,i_2,\dots,i_k\} \\
		\{j_1',j_2',\dots,j_{n-k}'\}=\{1,2,\dots,n\}\backslash\{j_1,j_2,\dots,j_k\}
	\end{gather*}
	特别的,$A$的$(i,j)$元的余子式记作$M_{ij}$,代数余子式记作$A_{ij}$。若选取的行和列是前$k$行与前$k$列,则称这些行和列交叉处的$k^2$个元素按原来的次序组成的$k$阶矩阵的行列式为$A$的一个\gls{PrincipalKMinor}。
\end{definition}
\begin{property}\label{prop:Determinant}
	矩阵$A=(a_{ij})\in M_{n}(K)$的行列式$\det A$具有如下性质:
	\begin{enumerate}
		\item $\det A$有等价表达式:
		\begin{gather*}
			\sum_{k_1k_2\cdots k_n}^{}(-1)^{\tau(i_1i_2\cdots i_n)+\tau(k_1k_2\cdots k_n)}a_{i_1k_1}a_{i_2k_2}\cdots a_{i_nk_n} \\
			\sum_{i_1i_2\cdots i_n}^{}(-1)^{\tau(i_1i_2\cdots i_n)+\tau(k_1k_2\cdots k_n)}a_{i_1k_1}a_{i_2k_2}\cdots a_{i_nk_n} \\
			\sum_{i_1i_2\cdots i_n}^{}(-1)^{\tau(i_1i_2\cdots i_n)}a_{i_11}a_{i_22}\cdots a_{i_nn}
		\end{gather*}
		\item $\det A^T=\det A,\;|A^H|=\overline{|A|}$;
		\item 行列式一行(列)的公因子可以提出去:
		\begin{gather*}
			\begin{vmatrix}
				a_{11} & a_{12} & \cdots & a_{1n} \\
				\vdots & \vdots & \ddots & \vdots \\
				ka_{i1} & ka_{i2} & \cdots & ka_{in} \\
				\vdots & \vdots & \ddots & \vdots \\
				a_{n1} & a_{n2} & \cdots & a_{nn}
			\end{vmatrix}=k
			\begin{vmatrix}
				a_{11} & a_{12} & \cdots & a_{1n} \\
				\vdots & \vdots & \ddots & \vdots \\
				a_{i1} & a_{i2} & \cdots & a_{in} \\
				\vdots & \vdots & \ddots & \vdots \\
				a_{n1} & a_{n2} & \cdots & a_{nn}
			\end{vmatrix} \\
			\begin{vmatrix}
				a_{11} & \cdots & ka_{1j} & \cdots & a_{1n} \\
				a_{21} & \cdots & ka_{2j} & \cdots & a_{2n} \\
				\vdots & \ddots & \vdots & \ddots & \vdots \\
				a_{n1} & \cdots & ka_{nj} & \cdots & a_{nn}
			\end{vmatrix}=k
			\begin{vmatrix}
				a_{11} & \cdots & a_{1j} & \cdots & a_{1n} \\
				a_{21} & \cdots & a_{2j} & \cdots & a_{2n} \\
				\vdots & \ddots & \vdots & \ddots & \vdots \\
				a_{n1} & \cdots & a_{nj} & \cdots & a_{nn}
			\end{vmatrix}
		\end{gather*}
		\item 行列式的行(列)可以作拆分:
		\begin{gather*}
			\begin{vmatrix}
				a_{11} & a_{12} & \cdots & a_{1n} \\
				\vdots & \vdots & \ddots & \vdots \\
				a_{i1}+a_{i1}' & a_{i2}+a_{i2}' & \cdots & a_{in}+a_{in}' \\
				\vdots & \vdots & \ddots & \vdots \\
				a_{n1} & a_{n2} & \cdots & a_{nn}
			\end{vmatrix}=
			\begin{vmatrix}
				a_{11} & a_{12} & \cdots & a_{1n} \\
				\vdots & \vdots & \ddots & \vdots \\
				a_{i1} & a_{i2} & \cdots & a_{in} \\
				\vdots & \vdots & \ddots & \vdots \\
				a_{n1} & a_{n2} & \cdots & a_{nn}
			\end{vmatrix}+
			\begin{vmatrix}
				a_{11} & a_{12} & \cdots & a_{1n} \\
				\vdots & \vdots & \ddots & \vdots \\
				a_{i1}' & a_{i2}' & \cdots & a_{in}' \\
				\vdots & \vdots & \ddots & \vdots \\
				a_{n1} & a_{n2} & \cdots & a_{nn}
			\end{vmatrix} \\
			\begin{aligned}
				&\begin{vmatrix}
					a_{11} & \cdots & a_{1j}+a_{1j}' & \cdots & a_{1n} \\
					a_{21} & \cdots & a_{2j}+a_{2j}' & \cdots & a_{2n} \\
					\vdots & \ddots & \vdots & \ddots & \vdots \\
					a_{n1} & \cdots & a_{nj}+a_{nj}' & \cdots & a_{nn}
				\end{vmatrix} \\
				=&
				\begin{vmatrix}
					a_{11} & \cdots & a_{1j} & \cdots & a_{1n} \\
					a_{21} & \cdots & a_{2j} & \cdots & a_{2n} \\
					\vdots & \ddots & \vdots & \ddots & \vdots \\
					a_{n1} & \cdots & a_{nj} & \cdots & a_{nn}
				\end{vmatrix}+
				\begin{vmatrix}
					a_{11} & \cdots & a_{1j}' & \cdots & a_{1n} \\
					a_{21} & \cdots & a_{2j}' & \cdots & a_{2n} \\
					\vdots & \ddots & \vdots & \ddots & \vdots \\
					a_{n1} & \cdots & a_{nj}' & \cdots & a_{nn}
				\end{vmatrix}
			\end{aligned}
		\end{gather*}
		\item 行列式两行(列)互换,行列式反号:
		\begin{gather*}
			\begin{vmatrix}
				a_{11} & a_{12} & \cdots & a_{1n} \\
				\vdots & \vdots & \ddots & \vdots \\
				a_{i1} & a_{i2} & \cdots & a_{in} \\
				\vdots & \vdots & \ddots & \vdots \\
				a_{j1} & a_{j2} & \cdots & a_{jn} \\
				\vdots & \vdots & \ddots & \vdots \\
				a_{n1} & a_{n2} & \cdots & a_{nn}
			\end{vmatrix}=-
			\begin{vmatrix}
				a_{11} & a_{12} & \cdots & a_{1n} \\
				\vdots & \vdots & \ddots & \vdots \\
				a_{j1} & a_{j2} & \cdots & a_{jn} \\
				\vdots & \vdots & \ddots & \vdots \\
				a_{i1} & a_{i2} & \cdots & a_{in} \\
				\vdots & \vdots & \ddots & \vdots \\
				a_{n1} & a_{n2} & \cdots & a_{nn}
			\end{vmatrix} \\
			\begin{vmatrix}
				a_{11} & \cdots & a_{1i} & \cdots & a_{1j} & \cdots & a_{1n} \\
				a_{21} & \cdots & a_{2i} & \cdots & a_{2j} & \cdots & a_{2n} \\
				\vdots & \ddots & \vdots & \ddots & \vdots & \ddots & \vdots \\
				a_{n1} & \cdots & a_{ni} & \cdots & a_{nj} & \cdots & a_{nn}
			\end{vmatrix}=-
			\begin{vmatrix}
				a_{11} & \cdots & a_{1j} & \cdots & a_{1i} & \cdots & a_{1n} \\
				a_{21} & \cdots & a_{2j} & \cdots & a_{2i} & \cdots & a_{2n} \\
				\vdots & \ddots & \vdots & \ddots & \vdots & \ddots & \vdots \\
				a_{n1} & \cdots & a_{nj} & \cdots & a_{ni} & \cdots & a_{nn}
			\end{vmatrix}
		\end{gather*}
		\item 行列式两行(列)成比例则值为$0$:
		\begin{equation*}
			\begin{vmatrix}
			a_{11} & a_{12} & \cdots & a_{1n} \\
			\vdots & \vdots & \ddots & \vdots \\
			a_{i1} & a_{i2} & \cdots & a_{in} \\
			\vdots & \vdots & \ddots & \vdots \\
			ka_{i1} & ka_{i2} & \cdots & ka_{in} \\
			\vdots & \vdots & \ddots & \vdots \\
			a_{n1} & a_{n2} & \cdots & a_{nn}
			\end{vmatrix}=
			\begin{vmatrix}
			a_{11} & \cdots & a_{1j} & \cdots & ka_{1j} & \cdots & a_{1n} \\
			a_{21} & \cdots & a_{2j} & \cdots & ka_{2j} & \cdots & a_{2n} \\
			\vdots & \ddots & \vdots & \ddots & \vdots & \ddots & \vdots \\
			a_{n1} & \cdots & a_{nj} & \cdots & ka_{nj} & \cdots & a_{nn}
			\end{vmatrix}=0
		\end{equation*}
		\item 把一行(列)的倍数加到另一行(列)上行列式的值不变:
		\begin{gather*}
			\begin{vmatrix}
				a_{11} & a_{12} & \cdots & a_{1n} \\
				\vdots & \vdots & \ddots & \vdots \\
				a_{i1} & a_{i2} & \cdots & a_{in} \\
				\vdots & \vdots & \ddots & \vdots \\
				a_{j1}+ka_{i1} & a_{j2}+ka_{i2} & \cdots & a_{jn}+ka_{in} \\
				\vdots & \vdots & \ddots & \vdots \\
				a_{n1} & a_{n2} & \cdots & a_{nn}
			\end{vmatrix}=
			\begin{vmatrix}
				a_{11} & a_{12} & \cdots & a_{1n} \\
				\vdots & \vdots & \ddots & \vdots \\
				a_{i1} & a_{i2} & \cdots & a_{in} \\
				\vdots & \vdots & \ddots & \vdots \\
				a_{j1} & a_{j2} & \cdots & a_{jn} \\
				\vdots & \vdots & \ddots & \vdots \\
				a_{n1} & a_{n2} & \cdots & a_{nn}
			\end{vmatrix} \\
			\begin{vmatrix}
				a_{11} & \cdots & a_{1i} & \cdots & a_{1j}+ka_{1i} & \cdots & a_{1n} \\
				a_{21} & \cdots & a_{2i} & \cdots & a_{2j}+ka_{2i} & \cdots & a_{2n} \\
				\vdots & \ddots & \vdots & \ddots & \vdots & \ddots & \vdots \\
				a_{n1} & \cdots & a_{ni} & \cdots & a_{nj}+ka_{ni}  & \cdots & a_{nn}
			\end{vmatrix}=
			\begin{vmatrix}
				a_{11} & \cdots & a_{1i} & \cdots & a_{1j} & \cdots & a_{1n} \\
				a_{21} & \cdots & a_{2i} & \cdots & a_{2j} & \cdots & a_{2n} \\
				\vdots & \ddots & \vdots & \ddots & \vdots & \ddots & \vdots \\
				a_{n1} & \cdots & a_{ni} & \cdots & a_{nj} & \cdots & a_{nn}
			\end{vmatrix}
		\end{gather*}
		\item $A$的行列式等于它的第$i$行(第$j$列)元素与自己代数余子式的乘积之和,即:
		\begin{equation*}
			\det A=\sum_{j=1}^{n}a_{ij}A_{ij}=\sum_{i=1}^{n}a_{ij}A_{ij}
		\end{equation*}
		\item $A$的第$i$行(列)元素与第$j(j\ne i)$行(列)相应元素的代数余子式之和等于$0$,即:
		\begin{equation*}
			\sum_{k=1}^{n}a_{ik}A_{jk}=\sum_{k=1}^{n}a_{ki}A_{kj}=0
		\end{equation*}
		\item \textbf{Laplace Theorem:} 取定$A$的第$i_1,i_2,\dots,i_k(i_1<i_2<\cdots<i_k)$行,则这$k$行元素构成的所有$k$阶子式与它们自己的代数余子式的乘积之和等于$\det A$,即:
		\begin{align*}
			\det A&=\sum_{1\leqslant j_1<j_2<\cdots<j_k\leqslant n}A\left\{ \begin{array}{l}
				i_1,i_2,\dots,i_k \\
				j_1,j_2,\ \dots,j_k
			\end{array} \right\} \\
			&\quad(-1)^{(i_1+i_2+\cdots+i_k)+(j_1+j_2+\cdots+j_k)}A\left\{ \begin{array}{l}
				i_1',i_2',\dots,i_{n-k}' \\
				j_1',j_2',\ \dots,j_{n-k}'
			\end{array} \right\}
		\end{align*}
		\item 设$B\in M_{n}(K)$,则$\det(AB)=\det A\det B$;
		\item 若$A$是上三角矩阵\info{特殊矩阵},则$\det A=\prod\limits_{i=1}^na_{ii}$;
		\item 称下述行列式为\textbf{Vandermonde行列式}:
		\begin{equation*}
			\begin{vmatrix}
				1 & 1 & 1 & \cdots & 1 \\
				x_1 & x_2 & x_3 & \cdots & x_n \\
				x_1^2 & x_2^2 & x_3^2 & \cdots & x_n^2 \\
				\vdots & \vdots & \vdots & \ddots & \vdots \\
				x_1^{n-2} & x_2^{n-2} & x_3^{n-2} & \cdots & x_n^{n-2} \\
				x_1^{n-1} & x_2^{n-1} & x_3^{n-1} & \cdots & x_n^{n-1}
			\end{vmatrix}=\prod_{1\leqslant i<j\leqslant n}(x_j-x_i)
		\end{equation*}
	\end{enumerate}
\end{property}
\begin{proof}
	(1)对排列$a_{i_1k_1}a_{i_2k_2}\cdots a_{i_nk_n}$进行考察,设其进行了$s$次对换得到$a_{1j_1}a_{2j_2}\cdots a_{nj_n}$。由\cref{prop:Transposition}(2)可知:
	\begin{equation*}
		(-1)^{\tau(i_1i_2\cdots i_n)}(-1)^{s}=(-1)^{\tau(12\cdots n)}=1,\quad
		(-1)^{\tau(k_1k_2\cdots k_n)}(-1)^{s}=(-1)^{\tau(j_1j_2\cdots j_n)}
	\end{equation*}
	所以:
	\begin{equation*}
		(-1)^{\tau(i_1i_2\cdots i_n)}(-1)^{s}(-1)^{\tau(k_1k_2\cdots k_n)}(-1)^{s}=(-1)^{\tau(j_1j_2\cdots j_n)}
	\end{equation*}
	即:
	\begin{equation*}
		(-1)^{\tau(i_1i_2\cdots i_n)+\tau(k_1k_2\cdots k_n)}=(-1)^{\tau(j_1j_2\cdots j_n)}
	\end{equation*}
	于是:
	\begin{equation*}
		\sum_{j_1j_2\cdots j_n}^{}(-1)^{\tau(j_1j_2\cdots j_n)}a_{1j_1}a_{2j_2}\cdots a_{nj_n}=\sum_{k_1k_2\cdots k_n}^{}(-1)^{\tau(i_1i_2\cdots i_n)+\tau(k_1k_2\cdots k_n)}a_{i_1k_1}a_{i_2k_2}\cdots a_{nk_n}
	\end{equation*}\par
	第二式同理可得,第三式可由第二式推出。\par
	(2)利用(1)将行列式分别按行顺序与列顺序展开即可得到。共轭转置的情况由\info{乘积的共轭等于共轭的乘积}即可得出。\par
	(3)由定义可得:
	\begin{equation*}
		\sum_{j_1j_2\cdots j_n}^{}(-1)^{\tau(j_1j_2\cdots j_n)}a_{1j_1}a_{2j_2}\cdots ka_{ij_i}\cdots a_{nj_n}=k\sum_{j_1j_2\cdots j_n}^{}(-1)^{\tau(j_1j_2\cdots j_n)}a_{1j_1}a_{2j_2}\cdots a_{ij_i}\cdots a_{nj_n}
	\end{equation*}
	列的结果由(2)和行的结果即可得到。\par
	(4)由定义可得:
	\begin{align*}
		&\sum_{j_1j_2\cdots j_n}^{}(-1)^{\tau(j_1j_2\cdots j_n)}a_{1j_1}a_{2j_2}\cdots (a_{ij_i}+a_{ij_i}')\cdots a_{nj_n} \\
		=&\sum_{j_1j_2\cdots j_n}^{}(-1)^{\tau(j_1j_2\cdots j_n)}a_{1j_1}a_{2j_2}\cdots a_{ij_i}\cdots a_{nj_n} \\
		&+\sum_{j_1j_2\cdots j_n}^{}(-1)^{\tau(j_1j_2\cdots j_n)}a_{1j_1}a_{2j_2}\cdots a_{ij_i}'\cdots a_{nj_n}
	\end{align*}
	列的结果由(2)和行的结果即可得到。\par
	(5)由定义和\cref{prop:Transposition}(1)可得:
	\begin{align*}
		&\sum_{j_1j_2\cdots j_n}^{}(-1)^{\tau(j_1j_2\cdots j_i\cdots j_k\cdots j_n)}a_{1j_1}a_{2j_2}\cdots a_{ij_i}\cdots a_{kj_k}\cdots a_{nj_n} \\
		=&\sum_{j_1j_2\cdots j_n}^{}(-1)^{\tau(j_1j_2\cdots j_k\cdots j_i\cdots j_n)}a_{1j_1}a_{2j_2}\cdots a_{kj_k}\cdots a_{ij_i}\cdots a_{nj_n} \\
		=&(-1)\sum_{j_1j_2\cdots j_n}^{}(-1)^{\tau(j_1j_2\cdots j_i\cdots j_k\cdots j_n)}a_{1j_1}a_{2j_2}\cdots a_{ij_i}\cdots a_{kj_k}\cdots a_{nj_n}
	\end{align*}\par
	(6)由(3)(5)可得。\par
	(7)由(3)(6)可得。\par
	(8)由(1)可得:
	\begin{align*}
		\det A&=\sum_{k_1k_2\cdots k_{i-1}jk_{i+1}\cdots k_n}^{}(-1)^{\tau(k_1k_2\cdots k_{i-1}jk_{i+1}\cdots k_n)}a_{1k_1}a_{2k_2}\cdots a_{(i-1)k_{i-1}}a_{ij}a_{(i+1)k_{i+1}}\cdots a_{nk_n} \\
		&=\sum_{jk_1k_2\cdots k_{i-1}k_{i+1}\cdots k_n}^{}(-1)^{\tau[i12\cdots(i-1)(i+1)\cdots n]+\tau(jk_1k_2\cdots k_{i-1}k_{i+1}\cdots k_n)} \\
		&\quad a_{ij}a_{1k_1}a_{2k_2}\cdots a_{(i-1)k_{i-1}}a_{(i+1)k_{i+1}}\cdots a_{nk_n} \\
		&=\sum_{jk_1k_2\cdots k_{i-1}k_{i+1}\cdots k_n}^{}(-1)^{i-1}(-1)^{j-1}(-1)^{\tau(k_1k_2\cdots k_{i-1}k_{i+1}\cdots k_n)} \\
		&\quad a_{ij}a_{1k_1}a_{2k_2}\cdots a_{(i-1)k_{i-1}}a_{(i+1)k_{i+1}}\cdots a_{nk_n} \\
		&=\sum_{j=1}^{n}(-1)^{i-1}(-1)^{j-1}a_{ij}\sum_{k_1k_2\cdots k_{i-1}k_{i+1}\cdots k_n}(-1)^{\tau(k_1k_2\cdots k_{i-1}k_{i+1}\cdots k_n)} \\
		&\quad a_{1k_1}a_{2k_2}\cdots a_{(i-1)k_{i-1}}a_{(i+1)k_{i+1}}\cdots a_{nk_n} \\
		&=\sum_{j=1}^{n}(-1)^{i+j}a_{ij}M_{ij}=\sum_{j=1}^{n}a_{ij}A_{ij}
	\end{align*}
	列的结果由(2)和行的结果即可得到。\par
	(9)由(8)(6)(2)即可得到。\par
	(10)给定$A$的一个行指标$i_1i_2\cdots i_ki_1'i_2'\cdots i_{n-k}'$,由(1)可得:
	\begin{align*}
		\det A&=\sum_{\mu_1\mu_2\cdots\mu_k\nu_1\nu_2\cdots\nu_{n-k}}(-1)^{\tau(i_1i_2\cdots i_ki_1'i_2'\cdots i_{n-k}')+\tau(\mu_1\mu_2\cdots\mu_k\nu_1\nu_2\cdots\nu_{n-k})} \\
		&\quad a_{i_1\mu_1}a_{i_2\mu_2}\cdots a_{i_k\mu_k}a_{i_1'\nu_1}a_{i_2'\nu_2}\cdots a_{i_{n-k}'\nu_{n-k}}
	\end{align*}
	将这$n!$项进行分组:任意取定$\seq{j}{k}$列,其中$\tau(j_1j_2\cdots j_k)=0$,对应于选定结果的$n$元排列形如:
	\begin{equation*}
		\mu_1\mu_2\cdots\mu_k\nu_1\nu_2\cdots\nu_{n-k}
	\end{equation*}
	其中$\mu_1\mu_2\cdots\mu_k$是$\seq{j}{k}$形成的排列,$\nu_1\nu_2\cdots\nu_{n-k}$是$\{1,2,\dots,n\}\backslash\{\seq{j}{k}\}=\{j_1',j_2',\dots,j_{n-k}'\}$形成的排列。根据\cref{prop:Transposition}(4)可得:
	\begin{align*}
		\det A&=\sum_{1\leqslant j_1<\cdots<j_k\leqslant n}^{}\sum_{\mu_1\mu_2\cdots\mu_k}^{}\sum_{\nu_1\nu_2\cdots\nu_{n-k}}^{}(-1)^{\tau(i_1i_2\cdots i_ki_1'i_2'\cdots i_{n-k}')+\tau(\mu_1\mu_2\cdots\mu_k\nu_1\nu_2\cdots\nu_{n-k})} \\
		&\quad a_{i_1\mu_1}a_{i_2\mu_2}\cdots a_{i_k\mu_k}a_{i_1'\nu_1}a_{i_2'\nu_2}\cdots a_{i_{n-k}'\nu_{n-k}} \\
		&=\sum_{1\leqslant j_1<\cdots<j_k\leqslant n}^{}\sum_{\mu_1\mu_2\cdots\mu_k}^{}\sum_{\nu_1\nu_2\cdots\nu_{n-k}}^{}(-1)^{(i_1-1)+(i_2-1)+\cdots+(i_{k}-1)}(-1)^{\tau(\mu_1\mu_2\cdots\mu_k)+\tau(\nu_1\nu_2\cdots\nu_{n-k})} \\
		&\quad(-1)^{j_1+j_2+\cdots+j_k}(-1)^{\frac{k(k+1)}{2}}a_{i_1\mu_1}a_{i_2\mu_2}\cdots a_{i_k\mu_k}a_{i_1'\nu_1}a_{i_2'\nu_2}\cdots a_{i_{n-k}'\nu_{n-k}} \\
		&=\sum_{1\leqslant j_1<\cdots<j_k\leqslant n}^{}\sum_{\mu_1\mu_2\cdots\mu_k}^{}\sum_{\nu_1\nu_2\cdots\nu_{n-k}}^{}(-1)^{i_1+i_2+\cdots+i_k}(-1)^{\tau(\mu_1\mu_2\cdots\mu_k)+\tau(\nu_1\nu_2\cdots\nu_{n-k})} \\
		&\quad(-1)^{j_1+j_2+\cdots+j_k}a_{i_1\mu_1}a_{i_2\mu_2}\cdots a_{i_k\mu_k}a_{i_1'\nu_1}a_{i_2'\nu_2}\cdots a_{i_{n-k}'\nu_{n-k}} \\
		&=\sum_{1\leqslant j_1<\cdots<j_k\leqslant n}^{}(-1)^{(i_1+i_2+\cdots+i_k)+(j_1+j_2+\cdots+j_k)} \\
		&\quad\sum_{\mu_1\mu_2\cdots\mu_k}^{}(-1)^{\tau(\mu_1\mu_2\cdots\mu_k)}a_{i_1\mu_1}a_{i_2\mu_2}\cdots a_{i_k\mu_k} \\
		&\quad\sum_{\nu_1\nu_2\cdots\nu_{n-k}}^{}(-1)^{\tau(\nu_1\nu_2\cdots\nu_{n-k})}a_{i_1'\nu_1}a_{i_2'\nu_2}\cdots a_{i_{n-k}'\nu_{n-k}} \\
		&=\sum_{1\leqslant j_1<j_2<\cdots<j_k\leqslant n}(-1)^{(i_1+i_2+\cdots+i_k)+(j_1+j_2+\cdots+j_k)} \\
		&\quad A\left\{ \begin{array}{l}
			i_1,i_2,\dots,i_k \\
			j_1,j_2,\ \dots,j_k
		\end{array} \right\}A\left\{ \begin{array}{l}
			i_1',i_2',\dots,i_{n-k}' \\
			j_1',j_2',\ \dots,j_{n-k}'
		\end{array} \right\}
	\end{align*}\par
	(11)由(10)可得:
	\begin{equation*}
		\begin{vmatrix}
			A & \mathbf{0} \\
			-I_n & B
		\end{vmatrix}=|A||B|
	\end{equation*}
	由(10)和(3),利用矩阵的初等行变换将$A$矩阵化为$\mathbf{0}$又可得到:
	\begin{align*}
		&\begin{vmatrix}
			A & \mathbf{0} \\
			-I_n & B
		\end{vmatrix}=
		\begin{vmatrix}
		\mathbf{0} & AB \\
		-I_n & B
		\end{vmatrix} \\
		=&|AB|(-1)^{(1+2+\cdots+n)+[(n+1)+(n+2)+\cdots+2n]}|-I_n| \\
		=&|AB|(-1)^{(2n+1)n}(-1)^n|I_n|=|AB|
	\end{align*}
	所以$|A||B|=|AB|$。\par
	(12)由行列式的定义立即可得。\par
	(13)当$n=2$时结论显然成立。\par
	假设对$n-1$阶Vandermonde行列式结论成立,对于$n$阶Vandermonde行列式从最后一行到第二行将上一行的$-x_1$倍加到下一行上,由(7)(3)(8)和归纳假设可得:
	\begin{align*}
		&
		\begin{vmatrix}
			1 & 1 & 1 & \cdots & 1 \\
			x_1 & x_2 & x_3 & \cdots & x_n \\
			x_1^2 & x_2^2 & x_3^2 & \cdots & x_n^2 \\
			\vdots & \vdots & \vdots & \ddots & \vdots \\
			x_1^{n-2} & x_2^{n-2} & x_3^{n-2} & \cdots & x_n^{n-2} \\
			x_1^{n-1} & x_2^{n-1} & x_3^{n-1} & \cdots & x_n^{n-1}
		\end{vmatrix} \\
		=&
		\begin{vmatrix}
			1 & 1 & 1 & \cdots & 1 \\
			0 & x_2-x_1 & x_3-x_1 & \cdots & x_n-x_1 \\
			0 & x_2^2-x_2x_1 & x_3^2-x_3x_1 & \cdots & x_n^2-x_nx_1 \\
			\vdots & \vdots & \vdots & \ddots & \vdots \\
			0 & x_2^{n-2}-x_2^{n-3}x_1 & x_3^{n-2}-x_3^{n-3}x_1 & \cdots & x_n^{n-2}-x_n^{n-3}x_1 \\
			0 & x_2^{n-1}-x_2^{n-2}x_1 & x_3^{n-1}-x_3^{n-2}x_1 & \cdots & x_n^{n-1}-x_n^{n-2}x_1
		\end{vmatrix} \\
		=&\prod_{i=2}^{n}(x_i-x_1)
		\begin{vmatrix}
			1 & \frac{1}{x_2-x_1} & \frac{1}{x_3-x_1} & \cdots & \frac{1}{x_n-x_1} \\
			0 & 1 & 1 & \cdots & 1 \\
			0 & x_2& x_3 & \cdots & x_n \\
			\vdots & \vdots & \vdots & \ddots & \vdots \\
			0 & x_2^{n-3} & x_3^{n-3}& \cdots & x_n^{n-3}\\
			0 & x_2^{n-2} & x_3^{n-2} & \cdots & x_n^{n-2}
		\end{vmatrix} 
		=\prod_{i=2}^{n}(x_i-x_1)\prod_{2\leqslant i<j\leqslant n}(x_j-x_i) \\
		=&\prod_{1\leqslant i<j\leqslant n}^{}(x_j-x_i)\qedhere
	\end{align*}
\end{proof}


\subsection{矩阵的秩}
\begin{definition}
	设$K$是一个数域,$n$是给定的正整数,令:
	\begin{equation*}
		K^n=\{(\seq{a}{n}):a_i\in K,\;i=1,2,\dots,n\}
	\end{equation*}
	称$K^n$为\gls{NDimensionalVectorSpace},其中的元素为\gls{NDimensionalVector},将$(\seq{a}{n})$写成一行称为\gls{RowVector},写成一列称为\gls{ColumnVector}。若$n$维向量$(\seq{a}{n})$与$\seq{b}{n}$满足$a_1=b_1,\;a_2=b_2,\dots\;a_n=b_n$,则称二者相等。在$K^n$中定义如下运算:
	\begin{enumerate}
		\item \textbf{加法:} 
		\begin{equation*}
			(\seq{a}{n})+(\seq{b}{n})\coloneq(a_1+b_1,a_2+b_2,\dots,a_n+b_n)
		\end{equation*}
		\item \textbf{纯量乘法:}
		\begin{equation*}
			\forall\;k\in K,\; k(\seq{a}{n})=(\seq{ka}{n})
		\end{equation*}
	\end{enumerate}
	那么$M_{m\times n}(K)$构成一个线性空间\footnote{证明略去}。
\end{definition}
根据$n$维向量空间的定义,可以将矩阵$A\in M_{m\times n}(K)$的列向量组视为$K^m$中的元素,行向量组也可视为$K^n$中的元素。接下来我们来讨论矩阵的秩。
\begin{definition}
	矩阵$A$的列向量组的秩称为$A$的\textbf{列秩},行向量组的秩称为$A$的\textbf{行秩}。
\end{definition}
\begin{lemma}\label{lem:REFRankColumnRow}
	阶梯形矩阵$J$的行秩与等于列秩且都等于非零行数,$J$的主元所在的行构成行向量组的一个极大线性无关组,主元所在列构成列向量组的一个极大线性无关组。
\end{lemma}
\begin{lemma}\label{lem:ElementaryRowColumnTransRank}
	矩阵的初等行变换不改变行秩,初等列变换不改变列秩。
\end{lemma}
\begin{proof}
	证明三种变换前后的向量组是等价的,由\cref{prop:Rank}(3)即可得出结论。列变换的情况可由转置与行变换的结论得到。
\end{proof}
\begin{lemma}\label{lem:ElementaryRowSameColumn}
	矩阵的初等行变换不改变矩阵列向量组之间的线性相关性:
	\begin{enumerate}
		\item 设矩阵$A$经过初等行变换变成矩阵$B$,则$A$的列向量组线性相关当且仅当$B$的列向量组线性相关;
		\item 设矩阵$A$经过初等行变换变成矩阵$B$,若$B$的第$\seq{j}{r}$列构成$B$的列向量组的一个极大线性无关组,则$A$的第$\seq{j}{r}$列也构成$A$的列向量组的一个极大线性无关组\footnote{与\cref{lem:REFRankColumnRow}联合起来提供了求矩阵列向量组的极大线性无关组的方法。}。
	\end{enumerate}
\end{lemma}
\begin{proof}
	(1)将矩阵$A,B$看作齐次线性方程组的矩阵,由\info{初等行变换不改变线性方程组的解}可知$Ax=\mathbf{0}$和$Bx=\mathbf{0}$同解,于是$Ax=\mathbf{0}$有非零解当且仅当$Bx=\mathbf{0}$有非零解,即$A$的列向量组线性相关当且仅当$B$的列向量组线性相关。\par
	(2)$A$的第$\seq{j}{r}$列经过初等行变换构成$B$的第$\seq{j}{r}$列,由(1)可知它们线性无关。任取其它列第$l$列,则$A$的第$\seq{j}{r},l$列经过初等行变换构成$B$的第$\seq{j}{r},l$列,因为$B$的第$\seq{j}{r}$列构成$B$的列向量组的一个极大线性无关组,所以$B$的第$\seq{j}{r},l$列线性相关,由(1)可知$A$的第$\seq{j}{r},l$列也线性相关,所以$A$的第$\seq{j}{r},l$列构成$A$的一个极大线性无关组。
\end{proof}
\begin{property}\label{prop:MatrixRank}
	矩阵的秩具有如下性质:
	\begin{enumerate}
		\item 任意矩阵的行秩都等于列秩,所以将矩阵$A$的行秩和列秩统称为矩阵$A$的秩,记为$\operatorname{rank}(A)$;
		\item 矩阵的初等变换不改变矩阵的秩;
		\item 非零矩阵的秩等于它的不为$0$的子式的最高阶数;
		\item 矩阵$A$的不为$0$的$\operatorname{rank}(A)$阶子式所在的行(列)构成$A$的行(列)向量组的一个极大线性无关组;
		\item $\operatorname{rank}(A+B)\leqslant\operatorname{rank}(A)+\operatorname{rank}(B)$;
		\item 若$k\ne0$,则$\operatorname{rank}(kA)=\operatorname{rank}(A)$;
		\item 转置与Hermitian转置不改变矩阵的秩;
		\item 设$A\in M_{m\times n}(K)$,则有:
		\begin{equation*}
			\operatorname{rank}(AA^T)=\operatorname{rank}(A^TA)=\operatorname{rank}(A)
		\end{equation*}
		若$K=\mathbb{C}$,则有:
		\begin{equation*}
			\operatorname{rank}(AA^H)=\operatorname{rank}(A^HA)=\operatorname{rank}(A)
		\end{equation*}
		\item $\operatorname{rank}(AB)\leqslant\min\{\operatorname{rank}(A),\operatorname{rank}(B)\}$;
		\item 设$A\in M_{m\times n}(K),\;B\in M_{s\times t}(K)$,则:
		\begin{equation*}
			\operatorname{rank}\left[
			´\begin{pmatrix}
				A & \mathbf{0} \\
				\mathbf{0} & B
			\end{pmatrix}
			\right]=\operatorname{rank}(A)+\operatorname{rank}(B)
		\end{equation*}
		\item 设$A\in M_{m\times n}(K),\;B\in M_{s\times t}(K),\;C\in M_{m\times t}(K)$,则:
		\begin{equation*}
			\operatorname{rank}\left[
			´\begin{pmatrix}
				A & C\\
				\mathbf{0} & B
			\end{pmatrix}
			\right]\geqslant\operatorname{rank}(A)+\operatorname{rank}(B)
		\end{equation*}
		当$A$和$B$都行满秩或列满秩时等号成立;
	\end{enumerate}
\end{property}
\begin{proof}
	(1)任取矩阵$A$,记$A$的阶梯形矩阵为$J$。由\cref{lem:ElementaryRowColumnTransRank}可知则$A$的行秩等于$J$的行秩,由\cref{lem:REFRankColumnRow}可知$J$的行秩等于$J$的列秩,由\cref{lem:ElementaryRowSameColumn}(2)可知$J$的列秩等于$A$的列秩,于是$A$的行秩等于$A$的列秩。由$A$的任意性,结论成立。\par
	(2)由\cref{lem:ElementaryRowColumnTransRank}和(1)立即得到。\par
	(3)设矩阵$A\in M_{m\times n}(K),\;\operatorname{rank}(A)=r$,由(1)可得$A$有$r$行、$r$列线性无关,将其对应的$r^2$个元素按原本的顺序排成的矩阵记为$A_1$,由(2)、\cref{lem:REFRankColumnRow}可知$|A_1|\ne0$,所以$A$存在一个$r$阶子式。\par
	设$s>r$且$s\leqslant\min\{m,n\}$,任取$A$的一个$s$阶子式:
	\begin{equation*}
		A\left\{ \begin{array}{l}
			i_1,i_2,\dots,i_s \\
			j_1,j_2,\ \dots,j_s
		\end{array}\right\}
	\end{equation*}
	因为$\operatorname{rank}(A)=r$,所以$A$的列向量组的极大线性无关组由$r$个向量组成,而$A$的第$\seq{j}{s}$列可以由$A$的列向量组的极大线性无关组表出,且$s>r$,根据\cref{prop:LinearlyDependent}(7)可得$A$的第$\seq{j}{s}$列线性相关。由\cref{prop:Determinant}(7)可知该$m$阶子式为$0$。\par
	综上,$A$的不为$0$的子式的最高阶数为$\operatorname{rank}(A)$。\par
	(4)由\cref{prop:Determinant}(7)可知该子式对应的矩阵的行(列)向量组线性无关,从而其延伸组也线性无关,即对应于$A$的行(列)线性无关。因为该向量组的向量个数等于$\operatorname{rank}(A)$,由\cref{prop:Rank}(5)可知它是$A$的行(列)向量组的一个极大线性无关组。\par
	(5)因为$A+B$的列向量组可由$A$的列向量组与$B$的列向量组线性表出,所以$A+B$的极大线性无关组可以由$A$的极大线性无关组与$B$的极大线性无关组线性表出,由\cref{prop:LinearlyDependent}(8)即可得出结论。\par
	(6)显然。\par
	(7)转置由(1)可得,Hermitian转置只需注意到若虚部线性无关则乘上$-1$也线性无关。\par
	(8)只需要证明Hermitian转置的情况,转置是Hermitian转置在实数域上的特例。\par
	由\cref{prop:HomogeneousSLESolution}(3)\info{这里已经不可避免的用到线性方程组的结论了}可知只需证明方程$A^HAx=\mathbf{0}$与$Ax=\mathbf{0}$同解。注意到$Ax=\mathbf{0}$则必然有$A^HAx=\mathbf{0}$,而若$A^HAx=\mathbf{0}$,则必有$x^HA^HAx=||Ax||=0$,所以$Ax=\mathbf{0}$。于是:
	\begin{equation*}
		n-\operatorname{rank}(A^HA)=n-\operatorname{rank}(A)
	\end{equation*}
	所以:
	\begin{equation*}
		\operatorname{rank}(A^HA)=\operatorname{rank}(A)
	\end{equation*}
	同理由(7)可得:
	\begin{equation*}
		\operatorname{rank}(AA^H)=\operatorname{rank}(A^H)=\operatorname{rank}(A)
	\end{equation*}
	于是有:
	\begin{equation*}
		\operatorname{rank}(AA^H)=\operatorname{rank}(A^HA)=\operatorname{rank}(A)
	\end{equation*}\par
	(9)由\cref{prop:MatrixMultiplication}(1)的第二种理解方式,$AB$的列向量组可以由$A$的列向量组线性表出,由\cref{prop:Rank}(2)可得$\operatorname{rank}(AB)\leqslant\operatorname{rank}(A)$。由(7)同理可得$\operatorname{rank}(AB)=\operatorname{rank}(B^TA^T)\leqslant\operatorname{rank}(B^T)=\operatorname{rank}(B)$,所以结论成立。\par
	(10)将矩阵$\begin{pmatrix}
		A & \mathbf{0} \\
		\mathbf{0} & B
	\end{pmatrix}$通过初等变换化作阶梯形矩阵,由(2)和\cref{lem:REFRankColumnRow}即可得到结论。\par
	(11)根据(3)可知$A$有一个$\operatorname{rank}(A)$阶非零子式,$B$有一个$\operatorname{rank}(B)$阶非零子式,将它们分别记为$A_1,B_1$。由\cref{prop:Determinant}(10)可知$\begin{pmatrix}
		A & C \\
		\mathbf{0} & B
	\end{pmatrix}$有一个$\operatorname{rank}(A)+\operatorname{rank}(B)$阶子式:
	\begin{equation*}
		\begin{vmatrix}
			A_1 & C_1 \\
			\mathbf{0} & B_1
		\end{vmatrix}=|A_1||B_1|\ne0
	\end{equation*}
	于是由(3)可得:
	\begin{equation*}
		\operatorname{rank}\left[
		\begin{pmatrix}
			A & C \\
			\mathbf{0} & B
		\end{pmatrix}
		\right]\geqslant\operatorname{rank}(A)+\operatorname{rank}(B)
	\end{equation*}
	类似(10)可得到当$A$和$B$都行满秩或列满秩时等号成立。
\end{proof}

\subsection{矩阵的逆}
\begin{definition}
	对于数域$K$上的矩阵$A$,如果存在数域$K$上的矩阵$B$使得:
	\begin{equation*}
		AB=BA=I
	\end{equation*}
	则称$A$是\gls{InvertibleMatrix}或\gls{NonSingularMatrix},$B$是$A$的逆矩阵,记为$A^{-1}$。不存在逆矩阵的方阵被称为\gls{SingularMatrix}。
\end{definition}
\begin{definition}
	设$A=(a_{ij})\in M_{n}(K)$,称:
	\begin{equation*}
		A^*=
		\begin{pmatrix}
			A_{11} & A_{21} & \cdots & A_{n1} \\
			A_{12} & A_{22} & \cdots & A_{n2} \\
			\vdots & \vdots & \ddots &\vdots \\
			A_{1n} & A_{n2} & \cdots & A_{nn} 
		\end{pmatrix}
	\end{equation*}
	为$A$的\gls{AdjointMatrix},其中$A_{ij}$为$a_{ij}$的代数余子式。
\end{definition}
\begin{property}\label{prop:InvertibleMatrix}
	关于矩阵的逆有如下结论:
	\begin{enumerate}
		\item 若矩阵$A$可逆,则逆矩阵是唯一的;
		\item 可逆矩阵是方阵;
		\item 设$A\in M_{n}(K)$,则$A$可逆的充要条件为:
		\begin{enumerate}
			\item $\det A\ne0$;
			\item $\operatorname{rank}(A)=n$;
			\item $A$的行(列)向量组线性无关;
			\item $A$的行(列)向量组为$K^n$的一个基;
		\end{enumerate}
		若$A$可逆,有:
		\begin{equation*}
			A^{-1}=\frac{1}{|A|}A^*
		\end{equation*}
		\item 可逆矩阵经过初等行变换化成的简化行阶梯形矩阵一定是单位矩阵;
		\item 设$A$是一个可逆矩阵,则线性方程组$Ax=\mathbf{0}$只有零解;
		\item $\det A^{-1}=(\det A)^{-1}$;
		\item 若$A,B\in M_{n}(K)$且$AB=I_n$,则$A,B$都可逆,且$A^{-1}=B,\;B^{-1}=A$;
		\item 单位矩阵可逆;
		\item 初等矩阵可逆且逆矩阵仍为初等矩阵;
		\item 若$A$可逆,则$A^{-1}$可逆且$(A^{-1})^{-1}=A$;
		\item 若$A,B\in M_{n}(K)$都可逆,则$AB$也可逆,且$(AB)^{-1}=B^{-1}A^{-1}$;
		\item 若$A$可逆,则$A^T$可逆且$(A^T)^{-1}=(A^{-1})^T$,$A^H$也可逆且$(A^H)^{-1}=(A^{-1})^H$;
		\item 若对称矩阵$A$可逆,则$A^{-1}$仍是对称矩阵;若Hermitian矩阵$A$可逆,则$A^{-1}$仍是Hermitian矩阵;
		\item 设$A\in M_{n}(K)$,则$A$可逆的充要条件为它可以表示为一些初等矩阵的乘积;
		\item 求解逆矩阵的\textbf{初等变换法}:$(A, I)\overrightarrow{初等行变换}(I,A^{-1})$;
		\item 设$A\in M_{n}(K)$可逆,$B\in M_{m\times n}(K)$,则$\operatorname{rank}(B)=\operatorname{rank}(BA)=\operatorname{rank}(AB^T)$,即用一个可逆矩阵左(右)乘一个矩阵不会改变该矩阵的秩;
		\item 设$A=
		\begin{pmatrix}
			A_{11} & A_{12} \\
			A_{21} & A_{22}
		\end{pmatrix}$可逆。若$A_{11}$可逆,则:
		\begin{equation*}
			A^{-1}=
			\begin{pmatrix}
				A_{11}^{-1}+A_{11}^{-1}A_{12}(A_{22}-A_{21}A_{11}^{-1}A_{12})^{-1}A_{21}A_{11}^{-1} & -A_{11}^{-1}A_{12}(A_{22}-A_{21}A_{11}^{-1}A_{12})^{-1} \\
				-(A_{22}-A_{21}A_{11}^{-1}A_{12})^{-1}A_{21}A_{11}^{-1} & (A_{22}-A_{21}A_{11}^{-1}A_{12})^{-1}
			\end{pmatrix}
		\end{equation*}
		若$A_{22}$可逆,则:
		\begin{equation*}
			A^{-1}=
			\begin{pmatrix}
				(A_{11}-A_{12}A_{22}^{-1}A_{21})^{-1} & -(A_{11}-A_{12}A_{22}^{-1}A_{21})^{-1}A_{12}A_{22}^{-1} \\
				-A_{22}^{-1}A_{21}(A_{11}-A_{12}A_{22}^{-1}A_{21})^{-1} & A_{22}^{-1}+A_{22}^{-1}A_{21}(A_{11}-A_{12}A_{22}^{-1}A_{21})^{-1}A_{12}A_{22}^{-1}
			\end{pmatrix}
		\end{equation*}
	\end{enumerate}
\end{property}
\begin{proof}
	(1)设$A$有逆矩阵$B_1,B_2$且$B_1\ne B_2$,由\cref{prop:MatrixMultiplication}(5)(3)可得:
	\begin{equation*}
		B_1=B_1I=B_1AB_2=(B_1A)B_2=IB_2=B_2
	\end{equation*}
	矛盾。\par
	(2)由可逆矩阵的定义即可得到。\par
	(3)\textbf{a:}当$A$可逆时有$AA^{-1}=I_n$,所以由\cref{prop:Determinant}(11)可得$|AA^{-1}|=|A||A^{-1}|=1$,于是$\det A\ne0$。当$\det A\ne0$时,由\cref{prop:Determinant}(8)(9)可知$AA^*=A^*A=|A|E$,所以$A^{-1}=|A|^{-1}A^*$,$A$可逆。\par
	\textbf{b:}由(a)和\cref{prop:MatrixRank}(3)立即可得。\par
	\textbf{c:}由(b)和矩阵秩的定义立即可得。\par
	\textbf{d:}由(c)、\cref{prop:nDimensionalLinearSpace}和\info{$K^n$的维数为$n$}立即可得。\par
	(4)由(3.b)、\cref{lem:REFRankColumnRow}和\cref{prop:MatrixRank}(2)立即可得。\par
	(5)由(3.b)立即可得。\par
	(6)当$A$可逆时有$AA^{-1}=I_n$,所以由\cref{prop:Determinant}(11)可得$|AA^{-1}|=|A||A^{-1}|=1$,即$\det A^{-1}=(\det A)^{-1}$。\par
	(7)由\cref{prop:Determinant}(11)可得$|AB|=|A||B|=|I_n|=1$,所以$|A|,|B|\ne0$,由(3.a)可知$A,B$都可逆。注意到$A^{-1}AB=B=A^{-1}$,同理可得$A=B^{-1}$。\par
	(8)显然。\par
	(9)只需取相反的初等矩阵即可。\par
	(10)由可逆矩阵的定义立即可得。\par
	(11)代入定义验证即可得到。\par
	(12)由\cref{prop:Transpose}(4)可得$(AA^{-1})^T=(A^{-1})^TA^T=I,\;(AA^{-1})^H=(A^{-1})^HA^H=I$,根据(7)结论成立。\par
	(13)由(12)立即可得。\par
	(14)由(4)(7)(10)(9)(11)可得必要性,由(9)(7)可得充分性。\par
	(15)由(4)可知存在一系列单位矩阵$\seq{P}{n}$使得$P_nP_{n-1}\cdots P_1A=I$,根据(7)可得$P_nP_{n-1}\cdots P_1=A^{-1}$,即$P_nP_{n-1}\cdots P_1I=A^{-1}$,由\cref{prop:ElementaryMatrix}(2)可得即可得出结论。\par
	(16)由(14)和\cref{prop:MatrixRank}(2)立即可得。\par
	(17)若$A_{11}$可逆,则:
	\begin{equation*}
		\begin{pmatrix}
			I & \mathbf{0} \\
			-A_{21}A_{11}^{-1} & I
		\end{pmatrix}
		\begin{pmatrix}
			A_{11} & A_{12} \\
			A_{21} & A_{22}
		\end{pmatrix}
		\begin{pmatrix}
			I & -A_{11}^{-1}A_{12} \\
			\mathbf{0} & I
		\end{pmatrix}=
		\begin{pmatrix}
			A_{11} & \mathbf{0} \\
			\mathbf{0} & A_{22}-A_{21}A_{11}^{-1}A_{12}
		\end{pmatrix}
	\end{equation*}
	由\cref{prop:Determinant}(10)(11)和(3.a)可得$A_{22}-A_{21}A_{11}^{-1}A_{12}$和上式左边除了$A$以外的两个矩阵可逆。由(11)即可得到:
	\begin{gather*}
		\begin{pmatrix}
			I & -A_{11}^{-1}A_{12} \\
			\mathbf{0} & I
		\end{pmatrix}^{-1}A^{-1}
		\begin{pmatrix}
			I & \mathbf{0} \\
			-A_{21}A_{11}^{-1} & I
		\end{pmatrix}^{-1}=
		\begin{pmatrix}
			A_{11}^{-1} & \mathbf{0} \\
			\mathbf{0} & (A_{22}-A_{21}A_{11}^{-1}A_{12})^{-1}
		\end{pmatrix} \\
		A^{-1}=
		\begin{pmatrix}
			I & -A_{11}^{-1}A_{12} \\
			\mathbf{0} & I
		\end{pmatrix}
		\begin{pmatrix}
			A_{11}^{-1} & \mathbf{0} \\
			\mathbf{0} & (A_{22}-A_{21}A_{11}^{-1}A_{12})^{-1}
		\end{pmatrix}
		\begin{pmatrix}
			I & \mathbf{0} \\
			-A_{21}A_{11}^{-1} & I
		\end{pmatrix} \\
		A^{-1}=
		\begin{pmatrix}
			A_{11}^{-1}+A_{11}^{-1}A_{12}(A_{22}-A_{21}A_{11}^{-1}A_{12})^{-1}A_{21}A_{11}^{-1} & -A_{11}^{-1}A_{12}(A_{22}-A_{21}A_{11}^{-1}A_{12})^{-1} \\
			-(A_{22}-A_{21}A_{11}^{-1}A_{12})^{-1}A_{21}A_{11}^{-1} & (A_{22}-A_{21}A_{11}^{-1}A_{12})^{-1}
		\end{pmatrix}
	\end{gather*}
	$A_{22}$可逆的情况类似可得。
\end{proof}

\subsection{正交矩阵与Euclid空间}
\begin{definition}
	在$\mathbb{R}^{n}$中,对任意的$\alpha=(\seq{a}{n}),\;\beta=(\seq{b}{n})$定义:
	\begin{equation*}
		(\alpha,\beta)\coloneq\sum_{i=1}^{n}a_ib_i
	\end{equation*}
	则$(\alpha,\beta)$是一个内积,称之为标准内积。将有了标准内积后的$\mathbb{R}^{n}$称为\gls{EuclidSpace}。
\end{definition}
\begin{proof}
	需要证明上述定义满足内积的定义,由于过于简单所以略去。
\end{proof}
\begin{definition}
	在Euclid空间$\mathbb{R}^{n}$中,定义向量$\alpha$的长度$|\alpha|$为:
	\begin{equation*}
		|\alpha|=\coloneq\sqrt{(\alpha,\alpha)}
	\end{equation*}
	长度为$1$的向量被称为\gls{UnitVector}。把非零向量$\alpha$乘以$\frac{1}{|\alpha|}$的操作称为将$\alpha$\gls{Normalization}。
\end{definition}
\begin{definition}
	若$A\in M_{n}(\mathbb{R}^{})$满足$A^TA=I_n$,则称$A$是\gls{OrthogonalMatrix}。若$A\in M_{n}(\mathbb{C}^{})$满足$A^HA=I_n$,则称$A$是\gls{UnitaryMatrix}。
\end{definition}
\begin{property}\label{prop:OrthogonalUnitaryMatrix}
	正交矩阵与酉矩阵具有如下性质:
	\begin{enumerate}
		\item 若$A$是一个正交(酉)矩阵,则$A$可逆且$A^T=A^{-1}$($A^H=A^{-1}$);
		\item 若$A$是一个正交(酉)矩阵,则$\det A=\pm1$($|det A|=1$);
		\item 单位矩阵是正交(酉)矩阵;
		\item 若$A,B$是正交(酉)矩阵,则$AB$也是正交(酉)矩阵;
		\item 若$A$是正交(酉)矩阵,则$A^{-1}$也是正交(酉)矩阵;
	\end{enumerate}
\end{property}
\begin{proof}
	(1)由定义立即可得。\par
	(2)由\cref{prop:Determinant}(2)(11)可得:
	\begin{equation*}
		|AA^T|=|A|^2=1,\quad|AA^H|=|A||A^H|=|A|\overline{|A|}=1
	\end{equation*}\par
	(3)显然。\par
	(4)根据\cref{prop:Transpose}(4)代入验证即可。\par
	(5)由\cref{prop:Transpose}(2)和正交矩阵、酉矩阵的定义即可得到。
\end{proof}