\section{矩阵空间}

\begin{definition}
	由$s\cdot m$个数排成$s$行、$m$列的一张表称为一个$s\times m$\gls{Matrix},通常用大写英文字母表示,其中的每一个数称为这个矩阵的一个元素,第$i$行与第$j$列交叉位置的元素称为矩阵的$(i,j)$元,记作$A(i;j)$。一个$s\times m$矩阵可以简单地记作$A_{s\times m}$。如果矩阵$A$的$(i,j)$元是$a_{ij}$,那么可以记作$A=(a_{ij})$。如果一个矩阵的行数和列数相同,则称它为方阵,$n$行$n$列的方阵也成为$n$阶矩阵。对于两个矩阵$A$和$B$,如果它们的行数都等于$s$且列数都等于$m$,同时还有$A(i;j)=B(i;j),i=1,2,\dots,s,\;j=1,2,\dots,m$,那么称$A$和$B$相等,记作$A=B$。
\end{definition}

\subsection{矩阵的运算}
\subsubsection{加减法与数量乘法}
\begin{definition}
	将数域$K$上所有$s\times m$矩阵组成的集合记作$M_{s\times m}(K)$,当$s=m$时,$M_{s\times s}(K)$可以简记作$M_s(K)$。在$M_{s\times m}(K)$中定义如下运算:
	\begin{enumerate}
		\item \textbf{加法:} 
		\begin{equation*}
			\forall\;A=(a_{ij}),B=(b_{ij})\in M_{s\times m}(K),\;A+B=(a_{ij}+b_{ij})
		\end{equation*}
		\item \textbf{纯量乘法:}
		\begin{equation*}
			\forall\;k\in K,\;\forall\;A=(a_{ij}),\;kA=(ka_{ij})
		\end{equation*}
	\end{enumerate}
	那么$M_{s\times m}(K)$构成一个线性空间。
\end{definition}
\begin{proof}
	首先证明如上定义的加法和纯量乘法对$M_{s\times m}(K)$是封闭的。由数域中加法和乘法的封闭性,$a_{ij}+b_{ij}\in K,\;ka_{ij}\in K\;i=1,2,\dots,s,\;j=1,2,\dots,m$,所以如上定义的加法与纯量乘法对$M_{s\times m}(K)$是封闭的。\par
	接下来证明如上定义的加法和纯量乘法满足线性空间中的8条运算法则:
	\begin{enumerate}
		\item 因为数域内的数满足加法交换律与加法结合律,所以$M_{s\times m}(K)$上的加法满足线性空间运算法则(1)(2);
		\item 取一个元素全为$0$的$s\times m$矩阵,将其记作$\mathbf{0}$,显然对$\forall\;A\in M_{s\times m}(K)$,有$A+\mathbf{0}=A$,因此$M_{s\times m}(K)$中存在零元且它就是元素全为$0$的$s\times m$矩阵,称其为\gls{ZeroMatrix},就记作$\mathbf{0}$。因此,$M_{s\times m}(K)$上的加法满足线性空间运算法则(3);
		\item 对$\forall\;A\in M_{s\times m}(K)$,取$-A=(-a_{ij})$,则有$A+(-A)=(a_{ij}-a_{ij})=\mathbf{0}$。由$A$的任意性,$M_{s\times m}(K)$中的每个元素都具有负元,将$\forall\;A\in M_{s\times m}(K)$的负元就记作$-A$。因此,$M_{s\times m}(K)$上的加法满足线性空间运算法则(4);
		\item 因为数域内的数满足乘法结合律和乘法分配律,同时它们乘$1$的积是自身,所以$M_{s\times n}$上的纯量乘法满足线性空间运算法则(5)(6)(7)(8)。
	\end{enumerate}
	证明完毕。
\end{proof}
\begin{definition}
	定义$M_{s\times m}(K)$上矩阵的减法如下:设$A,B\in M_{s\times m}(K)$,则:
	\begin{equation*}
		A-B\overset{def}{=}A+(-B)
	\end{equation*}
\end{definition}
\subsubsection{乘法}
\begin{definition}
	设$A=(a_{ij})_{s\times n},\;B=(b_{ij})_{n\times m}$,令$C=(c_{ij})_{s\times m}$,其中:
	\begin{equation*}
		c_{ij}=\sum_{k=1}^{n}a_{ik}b_{kj},\;i=1,2,\dots,s,\;j=1,2,\dots,m
	\end{equation*}
	则矩阵$C$称作矩阵$A$与$B$的乘积,记作$C=AB$。
\end{definition}
\subsubsection{初等变换}
\begin{definition}
	称以下变换为矩阵的\gls{ElementaryRowOperation}:
	\begin{enumerate}
		\item 把一行的倍数加到另一行上;
		\item 互换两行的位置;
		\item 用一个非零数乘某一行。
	\end{enumerate}
	称以下变换为矩阵的\gls{ElementaryColumnOperation}:
	\begin{enumerate}
		\item 把一列的倍数加到另一列上;
		\item 互换两列的位置;
		\item 用一个非零数乘某一列。
	\end{enumerate}
\end{definition}
\subsection{矩阵的行列式}

\subsection{矩阵的秩}
\begin{definition}
	矩阵$A$的列向量组的秩称为$A$的\textbf{列秩},行向量组的秩称为$A$的\textbf{行秩}
\end{definition}
\begin{lemma}\label{lem:REFRankColumnRow}
	阶梯形矩阵$J$的行秩与等于列秩且都等于非零行数,$J$的主元所在的行构成行向量组的一个极大线性无关组,主元所在列构成列向量组的一个极大线性无关组。
\end{lemma}
\begin{lemma}\label{lem:ElementaryRowColumnTransRank}
	矩阵的初等行变换不改变行秩,初等列变换不改变列秩。
\end{lemma}
\begin{proof}
	证明三种变换前后的向量组是等价的,由\cref{prop:Rank}(3)即可得出结论。列变换的情况可由转置与行变换的结论得到。
\end{proof}
\begin{lemma}\label{lem:ElementaryRowSameColumn}
	矩阵的初等行变换不改变矩阵列向量组之间的线性相关性:
	\begin{enumerate}
		\item 设矩阵$A$经过初等行变换变成矩阵$B$,则$A$的列向量组线性相关当且仅当$B$的列向量组线性相关;
		\item 设矩阵$A$经过初等行变换变成矩阵$B$,若$B$的第$\seq{j}{r}$列构成$B$的列向量组的一个极大线性无关组,则$A$的第$\seq{j}{r}$也构成$A$的一个极大线性无关组\footnote{与\cref{lem:REFRankColumnRow}联合起来提供了求矩阵列向量组的极大线性无关组的方法。};
		\item 初等行变换不改变列秩。
	\end{enumerate}
\end{lemma}
\begin{proof}
	(1)将矩阵$A,B$看作齐次线性方程组的矩阵,则$Ax=\mathbf{0}$和$Bx=\mathbf{0}$同解,于是$Ax=\mathbf{0}$有非零解当且仅当$Bx=\mathbf{0}$有非零解,即$A$的列向量组线性相关当且仅当$B$的列向量组线性相关。\par
	(2)$A$的第$\seq{j}{r}$列经过初等行变换构成$B$的第$\seq{j}{r}$列,由(1)可知它们线性无关。任取其它列第$l$列,则$A$的第$\seq{j}{r},l$列经过初等行变换构成$B$的第$\seq{j}{r},l$列,因为$B$的第$\seq{j}{r}$列构成$B$的列向量组的一个极大线性无关组,所以$B$的第$\seq{j}{r},l$列线性相关,由(1)可知$A$的第$\seq{j}{r},l$列也线性相关,所以$A$的第$\seq{j}{r},l$列构成$A$的一个极大线性无关组。\par
	(3)由(2)直接得到。
\end{proof}
\begin{theorem}\label{theo:RowRank=ColumnRank}
	任意矩阵的行秩都等于列秩。
\end{theorem}
\begin{proof}
	任取矩阵$A$,记$A$的阶梯形矩阵为$J$。由\cref{lem:ElementaryRowColumnTransRank}可知则$A$的行秩等于$J$的行秩,由\cref{lem:REFRankColumnRow}可知$J$的行秩等于$J$的列秩,由\cref{lem:ElementaryRowSameColumn}(3)可知$J$的列秩等于$A$的列秩,于是$A$的行秩等于$A$的列秩。由$A$的任意性,结论成立。
\end{proof}
\begin{definition}
	矩阵$A$的行秩和列秩统称为矩阵$A$的秩,记为$\operatorname{rank}(A)$。
\end{definition}
\begin{corollary}
	矩阵的初等变换不改变矩阵的秩。
\end{corollary}
\begin{proof}
	由\cref{lem:ElementaryRowColumnTransRank}和\cref{theo:RowRank=ColumnRank}立即得到。
\end{proof}

