\section{矩阵空间}

\begin{definition}
	由$s\times m$个数排成$s$行、$m$列的一张表称为一个$s\times m$\gls{Matrix},通常用大写英文字母表示,其中的每一个数称为这个矩阵的一个元素,第$i$行与第$j$列交叉位置的元素称为矩阵的$(i,j)$元,记作$A(i;j)$。一个$s\times m$矩阵可以简单地记作$A_{s\times m}$。如果矩阵$A$的$(i,j)$元是$a_{ij}$,那么可以记作$A=(a_{ij})$。如果一个矩阵的行数和列数相同,则称它为方阵,$n$行$n$列的方阵也成为$n$阶矩阵。对于两个矩阵$A$和$B$,如果它们的行数都等于$s$且列数都等于$m$,同时还有$A(i;j)=B(i;j),i=1,2,\dots,s,\;j=1,2,\dots,m$,那么称$A$和$B$相等,记作$A=B$。
\end{definition}

\subsection{矩阵的运算}
\subsubsection{加减法与数量乘法}
\begin{definition}
	将数域$K$上所有$s\times m$矩阵组成的集合记作$M_{s\times m}(K)$,当$s=m$时,$M_{s\times s}(K)$可以简记作$M_s(K)$。在$M_{s\times m}(K)$中定义如下运算:
	\begin{enumerate}
		\item \textbf{加法:} 
		\begin{equation*}
			\forall\;A=(a_{ij}),B=(b_{ij})\in M_{s\times m}(K),\;A+B=(a_{ij}+b_{ij})
		\end{equation*}
		\item \textbf{纯量乘法:}
		\begin{equation*}
			\forall\;k\in K,\;\forall\;A=(a_{ij}),\;kA=(ka_{ij})
		\end{equation*}
	\end{enumerate}
	那么$M_{s\times m}(K)$构成一个线性空间。
\end{definition}
\begin{proof}
	首先证明如上定义的加法和纯量乘法对$M_{s\times m}(K)$是封闭的。由数域中加法和乘法的封闭性,$a_{ij}+b_{ij}\in K,\;ka_{ij}\in K\;i=1,2,\dots,s,\;j=1,2,\dots,m$,所以如上定义的加法与纯量乘法对$M_{s\times m}(K)$是封闭的。\par
	接下来证明如上定义的加法和纯量乘法满足线性空间中的8条运算法则:
	\begin{enumerate}
		\item 因为数域内的数满足加法交换律与加法结合律,所以$M_{s\times m}(K)$上的加法满足线性空间运算法则(1)(2);
		\item 取一个元素全为$0$的$s\times m$矩阵,将其记作$\mathbf{0}$,显然对$\forall\;A\in M_{s\times m}(K)$,有$A+\mathbf{0}=A$,因此$M_{s\times m}(K)$中存在零元且它就是元素全为$0$的$s\times m$矩阵,称其为\gls{ZeroMatrix},就记作$\mathbf{0}$。因此,$M_{s\times m}(K)$上的加法满足线性空间运算法则(3);
		\item 对$\forall\;A\in M_{s\times m}(K)$,取$-A=(-a_{ij})$,则有$A+(-A)=(a_{ij}-a_{ij})=\mathbf{0}$。由$A$的任意性,$M_{s\times m}(K)$中的每个元素都具有负元,将$\forall\;A\in M_{s\times m}(K)$的负元就记作$-A$。因此,$M_{s\times m}(K)$上的加法满足线性空间运算法则(4);
		\item 因为数域内的数满足乘法结合律和乘法分配律,同时它们乘$1$的积是自身,所以$M_{s\times n}$上的纯量乘法满足线性空间运算法则(5)(6)(7)(8)。
	\end{enumerate}
	证明完毕。
\end{proof}
\begin{definition}
	定义$M_{s\times m}(K)$上矩阵的减法如下:设$A,B\in M_{s\times m}(K)$,则:
	\begin{equation*}
		A-B\coloneq A+(-B)
	\end{equation*}
\end{definition}
\subsubsection{乘法}
\begin{definition}
	设$A=(a_{ij})_{s\times n},\;B=(b_{ij})_{n\times m}$,令$C=(c_{ij})_{s\times m}$,其中:
	\begin{equation*}
		c_{ij}=\sum_{k=1}^{n}a_{ik}b_{kj},\;i=1,2,\dots,s,\;j=1,2,\dots,m
	\end{equation*}
	则矩阵$C$称作矩阵$A$与$B$的乘积,记作$C=AB$。
\end{definition}
\subsubsection{转置}
\begin{definition}
	设$A\in M_{m\times n}(K)$,定义$A$的转置$A^T\in M_{n\times m}(K)$,它的第$i$行是$A$的第$i$列,第$j$列是$A$的第$j$行。
\end{definition}
\subsubsection{初等变换}
\begin{definition}
	称以下变换为矩阵的\gls{ElementaryRowOperation}:
	\begin{enumerate}
		\item 把一行的倍数加到另一行上;
		\item 互换两行的位置;
		\item 用一个非零数乘某一行。
	\end{enumerate}
	称以下变换为矩阵的\gls{ElementaryColumnOperation}:
	\begin{enumerate}
		\item 把一列的倍数加到另一列上;
		\item 互换两列的位置;
		\item 用一个非零数乘某一列。
	\end{enumerate}
\end{definition}
\subsection{矩阵的行列式}
\subsubsection{排列}
\begin{definition}
	$n$个不同的正整数的一个全排列称为一个\textbf{$n$元排列}。
\end{definition}
\begin{definition}
	在$n$元排列$a_1a_2\cdots a_n$中,从左到右任取一对数$a_ia_j(i<j)$,若$a_i<a_j$,则称这一对数构成一个\textbf{顺序};若$a_i>a_j$,则称这一对数构成一个\textbf{逆序}。
\end{definition}
\begin{definition}
	一个$n$元排列$a_1a_2\cdots a_n$中逆序的总数称为\textbf{逆序数},记作$\tau(a_1a_2\cdots a_n)$。逆序数为奇数的排列称为\textbf{奇排列},逆序数为偶数的排列称为\textbf{偶排列}。
\end{definition}
\begin{definition}
	将一个排列$a_1a_2\cdots a_n$中第$i$个位置的元素$a_i$与第$j$个位置的元素$a_j$交换位置的运算称为\textbf{对换},记作$(a_i,a_j)$。
\end{definition}
\begin{property}\label{prop:Transposition}
	$n$元排列的对换具有如下性质:
	\begin{enumerate}
		\item 对换改变$n$元排列的奇偶性;
		\item 任一由$1,2,\dots,n$构成的$n$元排列与排列$12\cdots n$可以经过一系列对换互变,且所作的对换的次数与这个$n$元排列具有相同的奇偶性。
	\end{enumerate}
\end{property}
\begin{proof}
	(1)任取一个$n$元排列$a_1a_2\cdots a_n$,若对换$(a_i,a_j)$的两个数$a_i,a_j$相邻,即:
	\begin{equation*}
		a_1a_2\cdots a_ia_j\cdots a_n\longrightarrow a_1a_2\cdots a_ja_i\cdots a_n
	\end{equation*}
	这个对换只改变了$a_i$与$a_j$构成的数对的顺逆序关系,于是对换前后排列的奇偶性相反。\par
	对于一般情况:
	\begin{equation*}
		a_1a_2\cdots a_ik_1k_2\cdots k_sa_j\cdots a_n\longrightarrow a_1a_2\cdots a_jk_2\cdots k_sa_i\cdots a_n
	\end{equation*}
	这个对换只需作$2s+1$次相邻数的对换即可达到:
	\begin{equation*}
		(a_i,k_1),\;(a_i,k_2),\;\dots,(a_i,k_s),\;(a_i,a_j),\;(k_s,a_j),\;(k_{s-1},a_j)\;\dots,(k_1,a_j)	
	\end{equation*}
	所以对换前后排列的奇偶性相反。\par
	综上,对换改变$n$元排列的奇偶性。\par
	(2)排列$12\cdots n$是一个偶排列,于是由(1)立即可得出结论。
\end{proof}
\subsubsection{行列式的定义与性质}
\begin{definition}
	定义$A=(a_{ij})\in M_{n}(K)$的\textbf{行列式}$\det A$为:
	\begin{equation*}
		\begin{vmatrix}
			a_{11} & a_{12} & \cdots & a_{1n} \\
			a_{21} & a_{22} & \cdots & a_{2n} \\
			\vdots & \vdots & \ddots & \vdots \\
			a_{n1} & a_{n2} & \cdots & a_{nn}
		\end{vmatrix}=
		\sum_{j_1j_2\cdots j_n}^{}(-1)^{\tau(j_1j_2\cdots j_n)}a_{1j_1}a_{2j_2}\cdots a_{nj_n}
	\end{equation*}
	其中$j_1j_2\cdots j_n$是由$1$到$n$构成的$n$元排列,$\det A$也记作$|A|$。
\end{definition} 
\begin{definition}
	设$A\in M_{n}(K)$。在$A$中任意取定$k(1\leqslant k<n)$行($i_1,i_2,\dots,i_k$)、$k$列($j_1,j_2,\dots,j_k$),位于这些行和列交叉处的$k^2$个元素按原来的次序组成的$k$阶矩阵的行列式称为$A$的一个\textbf{$k$阶子式},记作:
	\begin{equation*}
		A\left\{ \begin{array}{l}
			i_1,i_2,\dots,i_k \\
			j_1,j_2,\dots,j_k
		\end{array} \right\}
	\end{equation*}
	其余元素按原来的次序组成的$n-k$阶矩阵的行列式称为上式的\textbf{余子式},称:
	\begin{equation*}
		(-1)^{(i_1+i_2+\cdots+i_k)+(j_1+j_2+\cdots+j_k)}A\left\{ \begin{array}{l}
			i_1',i_2',\dots,i_{n-k}' \\
			j_1',j_2',\dots,j_{n-k}'
		\end{array} \right\}
	\end{equation*}
	为其\textbf{代数余子式},其中:
	\begin{gather*}
		\{i_1',i_2',\dots,i_{n-k}'\}=\{1,2,\dots,n\}\backslash\{i_1,i_2,\dots,i_k\} \\
		\{j_1',j_2',\dots,j_{n-k}'\}=\{1,2,\dots,n\}\backslash\{j_1,j_2,\dots,j_k\}
	\end{gather*}
	特别的,$A$的$(i,j)$元的余子式记作$M_{ij}$,代数余子式记作$A_{ij}$。
\end{definition}
\begin{property}\label{prop:Determinant}
	矩阵$A=(a_{ij})\in M_{n}(K)$的行列式$\det A$具有如下性质:
	\begin{enumerate}
		\item $\det A$有一个等价表达式:
		\begin{equation*}
			\sum_{k_1k_2\cdots k_n}^{}(-1)^{\tau(i_1i_2\cdots i_n)+\tau(k_1k_2\cdots k_n)}a_{i_1k_1}a_{i_2k_2}\cdots a_{nk_n}
		\end{equation*}
		\item $\det A^T=\det A$;
		\item 行列式一行(列)的公因子可以提出去:
		\begin{gather*}
			\begin{vmatrix}
				a_{11} & a_{12} & \cdots & a_{1n} \\
				\vdots & \vdots & \ddots & \vdots \\
				ka_{i1} & ka_{i2} & \cdots & ka_{in} \\
				\vdots & \vdots & \ddots & \vdots \\
				a_{n1} & a_{n2} & \cdots & a_{nn}
			\end{vmatrix}=k
			\begin{vmatrix}
				a_{11} & a_{12} & \cdots & a_{1n} \\
				\vdots & \vdots & \ddots & \vdots \\
				a_{i1} & a_{i2} & \cdots & a_{in} \\
				\vdots & \vdots & \ddots & \vdots \\
				a_{n1} & a_{n2} & \cdots & a_{nn}
			\end{vmatrix} \\
			\begin{vmatrix}
				a_{11} & \cdots & ka_{1j} & \cdots & a_{1n} \\
				a_{21} & \cdots & ka_{2j} & \cdots & a_{2n} \\
				\vdots & \ddots & \vdots & \ddots & \vdots \\
				a_{n1} & \cdots & ka_{nj} & \cdots & a_{nn}
			\end{vmatrix}=k
			\begin{vmatrix}
				a_{11} & \cdots & a_{1j} & \cdots & a_{1n} \\
				a_{21} & \cdots & a_{2j} & \cdots & a_{2n} \\
				\vdots & \ddots & \vdots & \ddots & \vdots \\
				a_{n1} & \cdots & a_{nj} & \cdots & a_{nn}
			\end{vmatrix}
		\end{gather*}
		\item 行列式的行(列)可以作拆分:
		\begin{gather*}
			\begin{vmatrix}
				a_{11} & a_{12} & \cdots & a_{1n} \\
				\vdots & \vdots & \ddots & \vdots \\
				a_{i1}+a_{i1}' & a_{i2}+a_{i2}' & \cdots & a_{in}+a_{in}' \\
				\vdots & \vdots & \ddots & \vdots \\
				a_{n1} & a_{n2} & \cdots & a_{nn}
			\end{vmatrix}=
			\begin{vmatrix}
				a_{11} & a_{12} & \cdots & a_{1n} \\
				\vdots & \vdots & \ddots & \vdots \\
				a_{i1} & a_{i2} & \cdots & a_{in} \\
				\vdots & \vdots & \ddots & \vdots \\
				a_{n1} & a_{n2} & \cdots & a_{nn}
			\end{vmatrix}+
			\begin{vmatrix}
				a_{11} & a_{12} & \cdots & a_{1n} \\
				\vdots & \vdots & \ddots & \vdots \\
				a_{i1}' & a_{i2}' & \cdots & a_{in}' \\
				\vdots & \vdots & \ddots & \vdots \\
				a_{n1} & a_{n2} & \cdots & a_{nn}
			\end{vmatrix} \\
			\begin{aligned}
				&\begin{vmatrix}
					a_{11} & \cdots & a_{1j}+a_{1j}' & \cdots & a_{1n} \\
					a_{21} & \cdots & a_{2j}+a_{2j}' & \cdots & a_{2n} \\
					\vdots & \ddots & \vdots & \ddots & \vdots \\
					a_{n1} & \cdots & a_{nj}+a_{nj}' & \cdots & a_{nn}
				\end{vmatrix} \\
				=&
				\begin{vmatrix}
					a_{11} & \cdots & a_{1j} & \cdots & a_{1n} \\
					a_{21} & \cdots & a_{2j} & \cdots & a_{2n} \\
					\vdots & \ddots & \vdots & \ddots & \vdots \\
					a_{n1} & \cdots & a_{nj} & \cdots & a_{nn}
				\end{vmatrix}+
				\begin{vmatrix}
					a_{11} & \cdots & a_{1j}' & \cdots & a_{1n} \\
					a_{21} & \cdots & a_{2j}' & \cdots & a_{2n} \\
					\vdots & \ddots & \vdots & \ddots & \vdots \\
					a_{n1} & \cdots & a_{nj}' & \cdots & a_{nn}
				\end{vmatrix}
			\end{aligned}
		\end{gather*}
		\item 行列式两行(列)互换,行列式反号:
		\begin{gather*}
			\begin{vmatrix}
				a_{11} & a_{12} & \cdots & a_{1n} \\
				\vdots & \vdots & \ddots & \vdots \\
				a_{i1} & a_{i2} & \cdots & a_{in} \\
				\vdots & \vdots & \ddots & \vdots \\
				a_{j1} & a_{j2} & \cdots & a_{jn} \\
				\vdots & \vdots & \ddots & \vdots \\
				a_{n1} & a_{n2} & \cdots & a_{nn}
			\end{vmatrix}=-
			\begin{vmatrix}
				a_{11} & a_{12} & \cdots & a_{1n} \\
				\vdots & \vdots & \ddots & \vdots \\
				a_{j1} & a_{j2} & \cdots & a_{jn} \\
				\vdots & \vdots & \ddots & \vdots \\
				a_{i1} & a_{i2} & \cdots & a_{in} \\
				\vdots & \vdots & \ddots & \vdots \\
				a_{n1} & a_{n2} & \cdots & a_{nn}
			\end{vmatrix} \\
			\begin{vmatrix}
				a_{11} & \cdots & a_{1i} & \cdots & a_{1j} & \cdots & a_{1n} \\
				a_{21} & \cdots & a_{2i} & \cdots & a_{2j} & \cdots & a_{2n} \\
				\vdots & \ddots & \vdots & \ddots & \vdots & \ddots & \vdots \\
				a_{n1} & \cdots & a_{ni} & \cdots & a_{nj} & \cdots & a_{nn}
			\end{vmatrix}=-
			\begin{vmatrix}
				a_{11} & \cdots & a_{1j} & \cdots & a_{1i} & \cdots & a_{1n} \\
				a_{21} & \cdots & a_{2j} & \cdots & a_{2i} & \cdots & a_{2n} \\
				\vdots & \ddots & \vdots & \ddots & \vdots & \ddots & \vdots \\
				a_{n1} & \cdots & a_{nj} & \cdots & a_{ni} & \cdots & a_{nn}
			\end{vmatrix}
		\end{gather*}
		\item 行列式两行(列)相同则值为$0$:
		\begin{equation*}
			\begin{vmatrix}
				a_{11} & a_{12} & \cdots & a_{1n} \\
				\vdots & \vdots & \ddots & \vdots \\
				a_{i1} & a_{i2} & \cdots & a_{in} \\
				\vdots & \vdots & \ddots & \vdots \\
				a_{i1} & a_{i2} & \cdots & a_{in} \\
				\vdots & \vdots & \ddots & \vdots \\
				a_{n1} & a_{n2} & \cdots & a_{nn}
			\end{vmatrix}=
			\begin{vmatrix}
			a_{11} & \cdots & a_{1j} & \cdots & a_{1j} & \cdots & a_{1n} \\
			a_{21} & \cdots & a_{2j} & \cdots & a_{2j} & \cdots & a_{2n} \\
			\vdots & \ddots & \vdots & \ddots & \vdots & \ddots & \vdots \\
			a_{n1} & \cdots & a_{nj} & \cdots & a_{nj} & \cdots & a_{nn}
			\end{vmatrix}=0
		\end{equation*}
		\item 行列式两行(列)成比例则值为$0$:
		\begin{equation*}
			\begin{vmatrix}
			a_{11} & a_{12} & \cdots & a_{1n} \\
			\vdots & \vdots & \ddots & \vdots \\
			a_{i1} & a_{i2} & \cdots & a_{in} \\
			\vdots & \vdots & \ddots & \vdots \\
			ka_{i1} & ka_{i2} & \cdots & ka_{in} \\
			\vdots & \vdots & \ddots & \vdots \\
			a_{n1} & a_{n2} & \cdots & a_{nn}
			\end{vmatrix}=
			\begin{vmatrix}
			a_{11} & \cdots & a_{1j} & \cdots & ka_{1j} & \cdots & a_{1n} \\
			a_{21} & \cdots & a_{2j} & \cdots & ka_{2j} & \cdots & a_{2n} \\
			\vdots & \ddots & \vdots & \ddots & \vdots & \ddots & \vdots \\
			a_{n1} & \cdots & a_{nj} & \cdots & ka_{nj} & \cdots & a_{nn}
			\end{vmatrix}=0
		\end{equation*}
		\item 把一行(列)的倍数加到另一行(列)上行列式的值不变:
		\begin{gather*}
			\begin{vmatrix}
				a_{11} & a_{12} & \cdots & a_{1n} \\
				\vdots & \vdots & \ddots & \vdots \\
				a_{i1} & a_{i2} & \cdots & a_{in} \\
				\vdots & \vdots & \ddots & \vdots \\
				a_{j1}+ka_{i1} & a_{j2}+ka_{i2} & \cdots & a_{jn}+ka_{in} \\
				\vdots & \vdots & \ddots & \vdots \\
				a_{n1} & a_{n2} & \cdots & a_{nn}
			\end{vmatrix}=
			\begin{vmatrix}
				a_{11} & a_{12} & \cdots & a_{1n} \\
				\vdots & \vdots & \ddots & \vdots \\
				a_{i1} & a_{i2} & \cdots & a_{in} \\
				\vdots & \vdots & \ddots & \vdots \\
				a_{j1} & a_{j2} & \cdots & a_{jn} \\
				\vdots & \vdots & \ddots & \vdots \\
				a_{n1} & a_{n2} & \cdots & a_{nn}
			\end{vmatrix} \\
			\begin{vmatrix}
				a_{11} & \cdots & a_{1i} & \cdots & a_{1j}+ka_{1i} & \cdots & a_{1n} \\
				a_{21} & \cdots & a_{2i} & \cdots & a_{2j}+ka_{2i} & \cdots & a_{2n} \\
				\vdots & \ddots & \vdots & \ddots & \vdots & \ddots & \vdots \\
				a_{n1} & \cdots & a_{ni} & \cdots & a_{nj}+ka_{ni}  & \cdots & a_{nn}
			\end{vmatrix}=
			\begin{vmatrix}
				a_{11} & \cdots & a_{1i} & \cdots & a_{1j} & \cdots & a_{1n} \\
				a_{21} & \cdots & a_{2i} & \cdots & a_{2j} & \cdots & a_{2n} \\
				\vdots & \ddots & \vdots & \ddots & \vdots & \ddots & \vdots \\
				a_{n1} & \cdots & a_{ni} & \cdots & a_{nj} & \cdots & a_{nn}
			\end{vmatrix}
		\end{gather*}
		\item $A$的行列式等于它的第$i$行(第$j$列)元素与自己代数余子式的乘积之和,即:
		\begin{equation*}
			\det A=\sum_{j=1}^{n}a_{ij}A_{ij}=\sum_{i=1}^{n}a_{ij}A_{ij}
		\end{equation*}
		\item $A$的第$i$行(列)元素与第$j(j\ne i)$行(列)相应元素的代数余子式之和等于$0$,即:
		\begin{equation*}
			\sum_{k=1}^{n}a_{ik}A_{jk}=\sum_{k=1}^{n}a_{ki}A_{kj}=0
		\end{equation*}
		\item \textbf{Laplace Theorem:} 取定$A$的第$i_1,i_2,\dots,i_k(i_1<i_2<\cdots<i_k)$行,则这$k$行元素构成的所有$k$阶子式与它们自己的代数余子式的乘积之和等于$\det A$,即:
		\begin{align*}
			\det A&=\sum_{1\leqslant j_1<j_2<\cdots<j_k\leqslant n}A\left\{ \begin{array}{l}
				i_1,i_2,\dots,i_k \\
				j_1,j_2,\ \dots,j_k
			\end{array} \right\} \\
			&\quad(-1)^{(i_1+i_2+\cdots+i_k)+(j_1+j_2+\cdots+j_k)}A\left\{ \begin{array}{l}
				i_1',i_2',\dots,i_{n-k}' \\
				j_1',j_2',\ \dots,j_{n-k}'
			\end{array} \right\}
		\end{align*}
		\item 若$A$是上三角矩阵\info{特殊矩阵},则$\det A=\prod\limits_{i=1}^na_{ii}$;
		\item 称下述行列式为\textbf{Vandermonde行列式}:
		\begin{equation*}
			\begin{vmatrix}
				1 & 1 & 1 & \cdots & 1 \\
				x_1 & x_2 & x_3 & \cdots & x_n \\
				x_1^2 & x_2^2 & x_3^2 & \cdots & x_n^2 \\
				\vdots & \vdots & \vdots & \ddots & \vdots \\
				x_1^{n-2} & x_2^{n-2} & x_3^{n-2} & \cdots & x_n^{n-2} \\
				x_1^{n-1} & x_2^{n-1} & x_3^{n-1} & \cdots & x_n^{n-1}
			\end{vmatrix}=\prod_{1\leqslant i<j\leqslant n}(a_j-a_i)
		\end{equation*}
	\end{enumerate}
\end{property}
\begin{proof}
	(1)对排列$a_{i_1k_1}a_{i_2k_2}\cdots a_{nk_n}$进行考察,设其进行了$s$次对换得到$a_{1j_1}a_{2j_2}\cdots a_{nj_n}$。由\cref{prop:Transposition}(2)可知:
	\begin{equation*}
		(-1)^{\tau(i_1i_2\cdots i_n)}(-1)^{s}=(-1)^{\tau(12\cdots n)}=1,\quad
		(-1)^{\tau(k_1k_2\cdots k_n)}(-1)^{s}=(-1)^{\tau(j_1j_2\cdots j_n)}
	\end{equation*}
	所以:
	\begin{equation*}
		(-1)^{\tau(i_1i_2\cdots i_n)}(-1)^{s}(-1)^{\tau(k_1k_2\cdots k_n)}(-1)^{s}=(-1)^{\tau(j_1j_2\cdots j_n)}
	\end{equation*}
	即:
	\begin{equation*}
		(-1)^{\tau(i_1i_2\cdots i_n)+\tau(k_1k_2\cdots k_n)}=(-1)^{\tau(j_1j_2\cdots j_n)}
	\end{equation*}
	于是:
	\begin{equation*}
		\sum_{j_1j_2\cdots j_n}^{}(-1)^{\tau(j_1j_2\cdots j_n)}a_{1j_1}a_{2j_2}\cdots a_{nj_n}=\sum_{k_1k_2\cdots k_n}^{}(-1)^{\tau(i_1i_2\cdots i_n)+\tau(k_1k_2\cdots k_n)}a_{i_1k_1}a_{i_2k_2}\cdots a_{nk_n}
	\end{equation*}\par
	(2)利用(1)将行列式分别按行顺序与列顺序展开即可得到。\par
	(3)由定义可得:
	\begin{equation*}
		\sum_{j_1j_2\cdots j_n}^{}(-1)^{\tau(j_1j_2\cdots j_n)}a_{1j_1}a_{2j_2}\cdots ka_{ij_i}\cdots a_{nj_n}=k\sum_{j_1j_2\cdots j_n}^{}(-1)^{\tau(j_1j_2\cdots j_n)}a_{1j_1}a_{2j_2}\cdots a_{ij_i}\cdots a_{nj_n}
	\end{equation*}
	列的结果由(2)和行的结果即可得到。\par
	(4)由定义可得:
	\begin{align*}
		&\sum_{j_1j_2\cdots j_n}^{}(-1)^{\tau(j_1j_2\cdots j_n)}a_{1j_1}a_{2j_2}\cdots (a_{ij_i}+a_{ij_i}')\cdots a_{nj_n} \\
		=&\sum_{j_1j_2\cdots j_n}^{}(-1)^{\tau(j_1j_2\cdots j_n)}a_{1j_1}a_{2j_2}\cdots a_{ij_i}\cdots a_{nj_n} \\
		&+\sum_{j_1j_2\cdots j_n}^{}(-1)^{\tau(j_1j_2\cdots j_n)}a_{1j_1}a_{2j_2}\cdots a_{ij_i}'\cdots a_{nj_n}
	\end{align*}
	列的结果由(2)和行的结果即可得到。\par
	(5)由定义和\cref{prop:Transposition}(1)可得:
	\begin{align*}
		&\sum_{j_1j_2\cdots j_n}^{}(-1)^{\tau(j_1j_2\cdots j_i\cdots j_k\cdots j_n)}a_{1j_1}a_{2j_2}\cdots a_{ij_i}\cdots a_{kj_k}\cdots a_{nj_n} \\
		=&\sum_{j_1j_2\cdots j_n}^{}(-1)^{\tau(j_1j_2\cdots j_k\cdots j_i\cdots j_n)}a_{1j_1}a_{2j_2}\cdots a_{kj_k}\cdots a_{ij_i}\cdots a_{nj_n} \\
		=&(-1)\sum_{j_1j_2\cdots j_n}^{}(-1)^{\tau(j_1j_2\cdots j_i\cdots j_k\cdots j_n)}a_{1j_1}a_{2j_2}\cdots a_{ij_i}\cdots a_{kj_k}\cdots a_{nj_n}
	\end{align*}\par
	(6)由(5)可得。\par
	(7)由(3)(6)可得。\par
	(8)由(3)(7)可得。\par
	(9)由定义可得:
	\begin{align*}
		\det A&=\sum_{k_1k_2\cdots k_{i-1}jk_{i+1}\cdots k_n}^{}(-1)^{\tau(k_1k_2\cdots k_{i-1}jk_{i+1}\cdots k_n)}a_{1k_1}a_{2k_2}\cdots a_{(i-1)k_{i-1}}a_{ij}a_{(i+1)k_{i+1}}\cdots a_{nk_n} \\
		&=\sum_{jk_1k_2\cdots k_{i-1}k_{i+1}\cdots k_n}^{}(-1)^{\tau[i12\cdots(i-1)(i+1)\cdots n]+\tau(jk_1k_2\cdots k_{i-1}k_{i+1}\cdots k_n)} \\
		&\quad a_{ij}a_{1k_1}a_{2k_2}\cdots a_{(i-1)k_{i-1}}a_{(i+1)k_{i+1}}\cdots a_{nk_n} \\
		&=\sum_{jk_1k_2\cdots k_{i-1}k_{i+1}\cdots k_n}^{}(-1)^{i-1}(-1)^{j-1}(-1)^{\tau(k_1k_2\cdots k_{i-1}k_{i+1}\cdots k_n)} \\
		&\quad a_{ij}a_{1k_1}a_{2k_2}\cdots a_{(i-1)k_{i-1}}a_{(i+1)k_{i+1}}\cdots a_{nk_n} \\
		&=\sum_{j=1}^{n}(-1)^{i-1}(-1)^{j-1}a_{ij}\sum_{k_1k_2\cdots k_{i-1}k_{i+1}\cdots k_n}(-1)^{\tau(k_1k_2\cdots k_{i-1}k_{i+1}\cdots k_n)} \\
		&\quad a_{1k_1}a_{2k_2}\cdots a_{(i-1)k_{i-1}}a_{(i+1)k_{i+1}}\cdots a_{nk_n} \\
		&=\sum_{j=1}^{n}(-1)^{i+j}a_{ij}M_{ij}=\sum_{j=1}^{n}a_{ij}A_{ij}
	\end{align*}
	列的结果由(2)和行的结果即可得到。\par
	(10)由(6)即可得到。\par
	(11)给定$A$的一个行指标$i_1i_2\cdots i_ki_1'i_2'\cdots i_{n-k}'$,由(1)可得:
	\begin{equation*}
		\det A=\sum_{\mu_1\mu_2\cdots\mu_k\nu_1\nu_2\cdots\nu_{n-k}}(-1)^{\tau(i_1i_2\cdots i_ki_1'i_2'\cdots i_{n-k}')+\tau(\mu_1\mu_2\cdots\mu_k\nu_1\nu_2\cdots\nu_{n-k})}a_{i_1\mu_1}a_{i_2\mu_2}\cdots a_{i_k\mu_k}a_{i_1'\nu_1}a_{i_2'\nu_2}\cdots a_{i_{n-k}'\nu_{n-k}}
	\end{equation*}
	\info{Laplace定理的证明}\par
	(12)由行列式的定义立即可得。\par
	(13)\info{Vandermonde行列式记得证明}
\end{proof}


\subsection{矩阵的秩}
\begin{definition}
	矩阵$A$的列向量组的秩称为$A$的\textbf{列秩},行向量组的秩称为$A$的\textbf{行秩}。
\end{definition}
\begin{lemma}\label{lem:REFRankColumnRow}
	阶梯形矩阵$J$的行秩与等于列秩且都等于非零行数,$J$的主元所在的行构成行向量组的一个极大线性无关组,主元所在列构成列向量组的一个极大线性无关组。
\end{lemma}
\begin{lemma}\label{lem:ElementaryRowColumnTransRank}
	矩阵的初等行变换不改变行秩,初等列变换不改变列秩。
\end{lemma}
\begin{proof}
	证明三种变换前后的向量组是等价的,由\cref{prop:Rank}(3)即可得出结论。列变换的情况可由转置与行变换的结论得到。
\end{proof}
\begin{lemma}\label{lem:ElementaryRowSameColumn}
	矩阵的初等行变换不改变矩阵列向量组之间的线性相关性:
	\begin{enumerate}
		\item 设矩阵$A$经过初等行变换变成矩阵$B$,则$A$的列向量组线性相关当且仅当$B$的列向量组线性相关;
		\item 设矩阵$A$经过初等行变换变成矩阵$B$,若$B$的第$\seq{j}{r}$列构成$B$的列向量组的一个极大线性无关组,则$A$的第$\seq{j}{r}$也构成$A$的一个极大线性无关组\footnote{与\cref{lem:REFRankColumnRow}联合起来提供了求矩阵列向量组的极大线性无关组的方法。};
		\item 初等行变换不改变列秩。
	\end{enumerate}
\end{lemma}
\begin{proof}
	(1)将矩阵$A,B$看作齐次线性方程组的矩阵,则$Ax=\mathbf{0}$和$Bx=\mathbf{0}$同解,于是$Ax=\mathbf{0}$有非零解当且仅当$Bx=\mathbf{0}$有非零解,即$A$的列向量组线性相关当且仅当$B$的列向量组线性相关。\par
	(2)$A$的第$\seq{j}{r}$列经过初等行变换构成$B$的第$\seq{j}{r}$列,由(1)可知它们线性无关。任取其它列第$l$列,则$A$的第$\seq{j}{r},l$列经过初等行变换构成$B$的第$\seq{j}{r},l$列,因为$B$的第$\seq{j}{r}$列构成$B$的列向量组的一个极大线性无关组,所以$B$的第$\seq{j}{r},l$列线性相关,由(1)可知$A$的第$\seq{j}{r},l$列也线性相关,所以$A$的第$\seq{j}{r},l$列构成$A$的一个极大线性无关组。\par
	(3)由(2)直接得到。
\end{proof}
\begin{theorem}\label{theo:RowRank=ColumnRank}
	任意矩阵的行秩都等于列秩。
\end{theorem}
\begin{proof}
	任取矩阵$A$,记$A$的阶梯形矩阵为$J$。由\cref{lem:ElementaryRowColumnTransRank}可知则$A$的行秩等于$J$的行秩,由\cref{lem:REFRankColumnRow}可知$J$的行秩等于$J$的列秩,由\cref{lem:ElementaryRowSameColumn}(3)可知$J$的列秩等于$A$的列秩,于是$A$的行秩等于$A$的列秩。由$A$的任意性,结论成立。
\end{proof}
\begin{definition}
	矩阵$A$的行秩和列秩统称为矩阵$A$的秩,记为$\operatorname{rank}(A)$。
\end{definition}
\begin{corollary}
	矩阵的初等变换不改变矩阵的秩。
\end{corollary}
\begin{proof}
	由\cref{lem:ElementaryRowColumnTransRank}和\cref{theo:RowRank=ColumnRank}立即得到。
\end{proof}

