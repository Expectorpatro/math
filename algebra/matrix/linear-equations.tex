\section{线性方程组}
\begin{definition}
	设 $x_1, x_2, \dots, x_n$ 为 $n$ 个未知数,若一个方程具有如下形式:
	\[
	a_1 x_1 + a_2 x_2 + \dots + a_n x_n = b
	\]
	其中,$a_1, a_2, \dots, a_n$ 为\gls{Coefficient},$b$为\gls{ConstantTerm},则称该方程为\gls{LinearEquation}。
	由$m$个形如上式的方程组成的方程组:
	\[
	\begin{cases}
		a_{11} x_1 + a_{12} x_2 + \dots + a_{1n} x_n = b_1 \\
		a_{21} x_1 + a_{22} x_2 + \dots + a_{2n} x_n = b_2 \\
		\quad \vdots \\
		a_{m1} x_1 + a_{m2} x_2 + \dots + a_{mn} x_n = b_m
	\end{cases}
	\]
	被称为$n$元\gls{SLE}。由矩阵乘法的定义,该方程组也可以写作矩阵形式:
	\[
	Ax=b
	\]
	其中:
	\[
	A =
	\begin{pmatrix}
		a_{11} & a_{12} & \dots & a_{1n} \\
		a_{21} & a_{22} & \dots & a_{2n} \\
		\vdots & \vdots & \ddots & \vdots \\
		a_{m1} & a_{m2} & \dots & a_{mn}
	\end{pmatrix}, \quad
	x =
	\begin{pmatrix}
		x_1 \\ x_2 \\ \vdots \\ x_n
	\end{pmatrix}, \quad
	b =
	\begin{pmatrix}
		b_1 \\ b_2 \\ \vdots \\ b_m
	\end{pmatrix}
	\]
\end{definition}
\begin{definition}
	给定线性方程组$Ax=b$,称如下矩阵:
	\[
	\begin{pmatrix}
		a_{11} & a_{12} & \dots & a_{1n} & b_1 \\
		a_{21} & a_{22} & \dots & a_{2n} & b_2 \\
		\vdots & \vdots & \ddots & \vdots & \vdots \\
		a_{m1} & a_{m2} & \dots & a_{mn} & b_m
	\end{pmatrix}.
	\]
	为该线性方程组的\gls{AugmentedMatrix},记为$[A|b]$。
\end{definition}
\begin{definition}
	一个矩阵被称为\gls{REF},如果它满足以下条件:
	\begin{enumerate}
		\item 所有零行(全为零的行)位于非零行的下方;
		\item 若某一行非零,则该行的首个非零元素(称为\gls{Pivot})位于该行之前所有行的主元右侧。
	\end{enumerate}
	一个矩阵被称为\gls{RREF},如果满足以下条件:
	\begin{enumerate}
		\item 它是阶梯形矩阵;
		\item 每个非零行的主元都是$1$;
		\item 每个主元所在列的其他元素均为$0$。
	\end{enumerate}
\end{definition}
\begin{theorem}
	任意一个矩阵都可以经过一系列初等行变换化成行阶梯形矩阵,进而可以经过一系列初等行变换化成简化行阶梯形矩阵。
\end{theorem}
\begin{definition}
	设增广矩阵化简后变为阶梯形矩阵,称每一行主元所在列所对应的未知数为\gls{PivotVariable},同时称非主元所在列对应的未知数为\gls{FreeVariable}。
\end{definition}
\subsection{初等方法}
\begin{theorem}\label{theo:SolutionOfSLE1}	
	数域$K$上的$n$元线性方程组的解的情况只有三种可能:
	\begin{enumerate}
		\item \textbf{无解:}增广矩阵化成的阶梯形方程出现$0=d$且$d\ne0$;
		\item 有解:
		\begin{enumerate}
			\item \textbf{唯一解:}阶梯形矩阵的非零行数$r$等于未知量个数$n$;
			\item \textbf{无穷多解:}阶梯形矩阵的非零行数$r$小于未知量个数$n$;
		\end{enumerate}
	\end{enumerate}
	这导致:
	\begin{enumerate}
		\item 数域$K$上$n$元齐次线性方程组有非零解的充分必要条件为:系数矩阵经过初等行变换化成的阶梯形矩阵中非零行数$r<n$;
		\item 数域$K$上$n$元齐次线性方程组的方程数$m$若小于未知量数$n$,则一定有非零解。
	\end{enumerate}
\end{theorem}
\subsection{秩与子空间}
\begin{theorem}\label{theo:SolutionOfSLE2}
	数域$K$上$n$元线性方程组$Ax=b$(即$\sum\limits_{i=1}^{n}\alpha_ix_i=b$,其中$\alpha_i$为$A$的列向量)有解的充分必要条件为:
	\begin{enumerate}
		\item $b\in<\seq{\alpha}{n}>$;
		\item $\operatorname{rank}(A)=\operatorname{rank}([A|b])$;
	\end{enumerate}
	进一步可得唯一解与无穷多解的判别方法:
	\begin{enumerate}
		\item \textbf{唯一解:}$\operatorname{rank}(A)=n$;
		\item \textbf{无穷多解:}$\operatorname{rank}(A)<n$。
	\end{enumerate}
	这导致齐次线性方程组有非零解的充分必要条件为$\operatorname{rank}(A)<n$。
\end{theorem}
\begin{proof}
	(1)显然。\par
	(2)由\cref{prop:SpanSubspace}(4)可得$Ax=b$有解$\iff b\in<\seq{\alpha}{n}>\iff<\seq{\alpha}{n},\beta>=<\seq{\alpha}{n}>\iff\dim<\seq{\alpha}{n},\beta>=\dim<\seq{\alpha}{n}>\iff\operatorname{rank}(A)=\operatorname{rank}([A|b])$。\par
	(3)若$\operatorname{rank}(A)=n$,则阶梯形矩阵的非零行数$r=n$,由\cref{theo:SolutionOfSLE1}可得此时有唯一解。\par
	(4)与(3)类似。
\end{proof}
\subsection{解的结构}
\subsubsection{齐次线性方程组}
\begin{property}\label{prop:HomogeneousSLESolution}
	数域$K$上$n$元齐次线性方程组$Ax=\mathbf{0}$的解具有如下性质:
	\begin{enumerate}
		\item 若$\alpha,\beta$是解,对任意的$c_1,c_2\in K$,$k_1\alpha+k_2\beta$也是解;
		\item 解空间$W$构成$K^n$的一个子空间;
		\item 解空间$W$满足$\dim(W)=n-\operatorname{rank}(A)$。
	\end{enumerate}
\end{property}
\begin{proof}
	(1)$A(k_1\alpha+k_2\beta)=k_1A\alpha+k_2A\beta=\mathbf{0}$。\par
	(2)由(1)立即可得。\par
	(3)设$A$的列向量组为$\seq{\alpha}{n}$,$A$的行数为$m$。定义线性映射$\mathcal{T}:\alpha\longrightarrow A\alpha$,则$\mathcal{T}$是$K^n$到$\mathbb{K}^{m}$的一个线性映射。于是有:
	\begin{gather*}
		\operatorname{Ker}(\mathcal{T})=\{\alpha\in K^n:\mathcal{T}\alpha=\mathbf{0}\}=\{\alpha\in K^n:A\alpha=\mathbf{0}\}=W \\
		\operatorname{Im}(\mathcal{T})=\{A\alpha:\alpha\in K^n\}=<\seq{\alpha}{n}>
	\end{gather*}
	所以由\cref{prop:SpanSubspace}(4)可得:
	\begin{gather*}
		\dim(\operatorname{Ker}\mathcal{T})=\dim(W) \\
		\operatorname{rank}(A)=\operatorname{rank}\{\seq{\alpha}{n}\}=\dim<\seq{\alpha}{n}>=\dim(\operatorname{Im}\mathcal{T})
	\end{gather*}
	由\cref{prop:LinearMapping}(10)即可得到:
	\begin{equation*}
		\dim(K^n)=\dim(\operatorname{Ker}\mathcal{T})+\dim(\operatorname{Im}\mathcal{T})=\dim(W)+\operatorname{rank}(A)
	\end{equation*}
	即$n=\dim(W)+\operatorname{rank}(A)$。
\end{proof}
\begin{definition}
	设数域$K$上$n$元齐次线性方程组$Ax=\mathbf{0}$有非零解,称它的解空间$W$的一组基为\gls{FundamentalSolutionSet}。
\end{definition}
\subsubsection{非齐次线性方程组}
\begin{property}\label{prop:InhomogeneousSLESolution}
	数域$K$上$n$数域$K$上$n$元非齐次线性方程组$Ax=b$的解具有如下性质:
	\begin{enumerate}
		\item 若$\alpha,\beta$是解,则$\alpha-\beta$为$Ax=\mathbf{0}$的解;
		\item 设$W$为$Ax=\mathbf{0}$的解空间,若$\alpha$是$Ax=b$的解,则对任意的$\beta\in W$,$\alpha+\beta$也是$Ax=b$的解;
		\item 设$W$为$Ax=\mathbf{0}$的解空间,则$Ax=b$的解集$U$可以表示为:
		\begin{equation*}
			U=\{\alpha+\beta:\beta\in W\}
		\end{equation*}
		其中$\alpha$为$Ax=b$的任意一个解;
		\item $Ax=b$的解唯一当且仅当$Ax=\mathbf{0}$的解空间为零空间。
	\end{enumerate}
\end{property}
\begin{proof}
	(1)$A(\alpha-\beta)=A\alpha-A\beta=b-b=\mathbf{0}$。\par
	(2)$A(\alpha+\beta)=A\alpha+A\beta=b+\mathbf{0}=b$。\par
	(3)由(1)(2)可得。\par
	(4)由(3)立即可得。
\end{proof}
\begin{algorithm}
	\caption{Gaussian Elimination}
	\label{alg:gauss}
	\begin{algorithmic}[1]
		\Require Augmented matrix $[A|\mathbf{b}] \in \mathbb{R}^{m \times (n+1)}$
		\Ensure Solution status and expression
		\State \textbf{Step 1: Forward Elimination}
		\For{$j = 1$ to $n$}
		\State Find $p$ such that $|a_{pj}|$ is maximum for $j \leq p \leq m$ \Comment{Pivoting}
		\If{$|a_{pj}| < \varepsilon$}
		\State \textbf{continue} \Comment{Skip zero column}
		\EndIf
		\State Swap row $p$ and row $j$
		\For{$i = j+1$ to $m$}
		\For{$k = j$ to $n+1$}
		\State $a_{ik} \gets a_{ik} - a_{ij} / a_{jj} \cdot a_{jk}$
		\EndFor
		\State $a_{ij} \gets 0$ \Comment{Explicitly zero to avoid error}
		\EndFor
		\EndFor
		
		\State \textbf{Step 2: Inconsistency Check}
		\For{$i = 1$ to $m$}
		\If{All $a_{ij} = 0$ for $j = 1$ to $n$ \textbf{and} $a_{i(n+1)} \neq 0$}
		\State \Return ``No solution (inconsistent row)''
		\EndIf
		\EndFor
		
		\State \textbf{Step 3: Identify Pivot and Free Variables}
		\State $\mathcal{P} \gets$ set of index of pivot columns in row echelon form, $\mathcal{F} \gets \{1,\dots,n\}\backslash\mathcal{P}$, $r \gets |\mathcal{P}|$
		\If{$r = n$}
		\State \textbf{Back substitution: unique solution}
		\State Initialize $\mathbf{x} \gets (0,\dots,0)^T$
		\For{$i = n$ downto $1$}
		\State $x_i \gets \left(a_{i(n+1)} - \sum\limits_{k=i+1}^{n} a_{ik} x_k\right) / a_{ii}$
		\EndFor
		\State \Return $\mathbf{x} = (x_1,\dots,x_n)^T$
		\EndIf
	\end{algorithmic}
\end{algorithm}
\begin{algorithm}
	\caption{Gaussian Elimination (Part 2): General Solution via RREF}
	\begin{algorithmic}[1]
		\Require Row echelon form (REF) matrix $[A|\mathbf{b}]$ with $r < n$
		\Ensure General solution $\mathbf{x} = \mathbf{x}_p + \sum t_j \mathbf{v}_j$
		
		\State \textbf{Construct general solution: infinite solutions}
		\State \textbf{Step 4: Transform to Reduced Row Echelon Form (RREF)}
		\For{$j = r$ downto $1$}
		\State Let $i$ be the row where pivot in column $\mathcal{P}[j]$ appears
		\State Divide entire row $i$ by $a_{i\mathcal{P}[j]}$ to make pivot = 1
		\For{$k = 1$ to $i-1$}
		\For{$l = \mathcal{P}[j]$ to $n+1$}
		\State $a_{kl} \gets a_{kl} - a_{k\mathcal{P}[j]} \cdot a_{il}$
		\EndFor
		\EndFor
		\EndFor
		
		\State \textbf{Step 5: Compute Particular Solution $\mathbf{x}_p$}
		\State Initialize $\mathbf{x}_p = (0, 0, \dots, 0)^T$
		\For{$j = 1$ to $r$}
		\State Let $i$ be the row where pivot in column $\mathcal{P}[j]$ appears
		\State $\mathbf{x}_{p\mathcal{P}[j]}\gets a_{i(n+1)}$
		\EndFor
		
		\State \textbf{Step 6: Compute Basis Vectors $\{\mathbf{v}_j\}$}
		\For{$j = 1$ to $n-r$}
		\State Initialize $\mathbf{v}_j = (0, 0, \dots, 0)^T$
		\State $\mathbf{v}_{j\mathcal{F}[j]}\gets1$
		\State Let $i$ be the first row such that the pivot column index is greater than $\mathcal{F}[j]$
		\For{$k=1$ to $i-1$}
		\State $\mathbf{v}_{jk}\gets-a_{k\mathcal{F}[j]}$
		\EndFor
		\State Store $\mathbf{v}_j$
		\EndFor
		
		\State \Return General solution: $\mathbf{x} = \mathbf{x}_p + \sum\limits_{j=1}^{n-r} c_j \mathbf{v}_j$, where $c_j\in\mathbb{R}^{}$
	\end{algorithmic}
\end{algorithm}
