\section{矩阵的等价关系}

\subsection{相抵}
\begin{definition}
	$A,B\in M_{s\times m}(K)$,如果满足下述条件中的任意一个:
	\begin{enumerate}
		\item $A$能够通过初等行变换和初等列变换变成$B$;
		\item 存在数域$K$上的$s$阶初等矩阵$P_1,P_2,\dots,P_t$与$m$阶初等矩阵$Q_1,Q_2,\dots,Q_n$使得:
		\begin{equation*}
			P_t\cdots P_2P_1AQ_1Q_2\cdots Q_n=B
		\end{equation*}
		\item 存在数域$K$上的$s$阶可逆矩阵$P$与$m$阶可逆矩阵$Q$使得:
		\begin{equation*}
			PAQ=B
		\end{equation*}
	\end{enumerate}
	则称$A$与$B$\gls{Equivalent}。
\end{definition}
上述三个条件显然是等价的。
\begin{theorem}
	相抵是$M_{s\times m}(K)$上的一个等价关系。在相抵关系下,矩阵$A$的等价类称为$A$的\textbf{相抵类}。
\end{theorem}
\begin{proof}
	证明是显然的。
\end{proof}
\begin{theorem}
	设$A\in M_{s\times m}(K)$,且$\operatorname{rank}(A)=r$。如果$r>0$,那么$A$相抵于如下形式的矩阵:
	\begin{equation*}
		\begin{pmatrix}
			I_r & \mathbf{0} \\
			\mathbf{0} & \mathbf{0}
		\end{pmatrix}
	\end{equation*}
	称该矩阵为$A$的\textbf{相抵标准形}。如果$r=0$,则$A$相抵于零矩阵,此时称零矩阵为$A$的\textbf{相抵标准形}。
\end{theorem}
\begin{proof}
	一个矩阵通过初等行变换一定可以变成一个简化行阶梯型矩阵,再由初等列变换即可得到上述矩阵。
\end{proof}
\begin{theorem}[相抵的完全不变量]
	$A,B\in M_{s\times m}(K)$,$A$与$B$相抵当且仅当它们的秩相同。
\end{theorem}
\begin{proof}
	\textbf{(1)必要性:}初等行变换和初等列变换不改变矩阵的秩。\par
	\textbf{(2)充分性:}若$A,B$的秩相同,则它们的相抵标准形相同。因为相抵是一个等价关系,由等价关系的对称性与传递性即可得到$A$与$B$相抵。
\end{proof}

\subsection{相似}
\begin{definition}
	$A,B\in M_{n}(K)$。如果存在可逆矩阵$P\in M_{n}(K)$,使得:
	\begin{equation*}
		P^{-1}AP=B
	\end{equation*}
	则称$A$与$B$\gls{Similar}。
\end{definition}
\begin{theorem}
	相似是$M_{n}(K)$上的一个等价关系。在相似关系下,矩阵$A$的等价类称为$A$的\textbf{相似类}。
\end{theorem}
\begin{proof}
	证明是显然的。
\end{proof}
\begin{property}[相似的不变量]\label{prop:Similar}
	相似的矩阵具有相同的行列式值、秩、迹、特征多项式、特征值(包括重数相同)。
\end{property}
\begin{proof}
	设$A,B\in M_{n}(K)$且$A$与$B$相似,于是存在可逆矩阵$P\in M_{n}(K)$使得$P^{-1}AP=B$。\par
	(1)$|A|=|P^{-1}AP|=|P^{-1}|\;|B|\;|P|=|P^{-1}|\;|P|\;|B|=|B|$。\par
	(2)初等行变换与初等列变换不改变矩阵的秩。\par
	(3)由\cref{prop:Trace}(3)可得$\operatorname{tr}(A)=\operatorname{tr}(P^{-1}BP)=\operatorname{tr}(BPP^{-1})=\operatorname{tr}(B)$。\par
	(4)(5)参考\cref{theo:SameEigenvalue}。
\end{proof}

\subsection{合同}
\begin{definition}
	$A,B\in M_{n}(K)$。如果存在可逆矩阵$C\in M_{n}(K)$,使得:
	\begin{equation*}
		C^TAC=B
	\end{equation*}
	则称$A$与$B$\gls{Congruent},记作$A\cong B$。如果对称矩阵$A$合同于一个对角矩阵,那么称这个对角矩阵为$A$的一个\textbf{合同标准形}。
\end{definition}
\begin{theorem}
	合同是$M_{n}(K)$上的一个等价关系。在合同关系下,矩阵$A$的等价类称为$A$的\textbf{合同类}。
\end{theorem}
\begin{proof}
	证明是显然的。
\end{proof}
\begin{definition}
	对$n$阶矩阵的行作初等行变换,再对该矩阵的同样标号的列作相同的初等列变换,这种变换被称为\textbf{成对初等行、列变换}。
\end{definition}
\begin{lemma}\label{lem:CTAC}
	$A,B\in M_{n}(K)$,则$A$合同于$B$当且仅当$A$经过一系列成对初等行、列变换可以变成$B$,此时对$I$作其中的初等列变换即可得到可逆矩阵$C$,使得$C^TAC=B$。
\end{lemma}
\begin{proof}
	由可逆矩阵的初等矩阵分解,可得:
	\begin{gather*}
		\begin{aligned}
			A\cong B
			&\iff\text{存在数域$K$上的可逆矩阵$C$,使得}C^TAC=B \\
			&\iff\text{存在数域$K$上的初等矩阵$P_1,P_2,\dots,P_t$使得}
		\end{aligned}\\
		C=P_1P_2\cdots P_t \\
		P_t^T\cdots P_2^TP_1^TAP_1P_2\cdots P_t=B\qedhere
	\end{gather*}
\end{proof}
\begin{theorem}\label{theo:AllCongruent}
	数域$K$上的任一对称矩阵都合同于一个对角矩阵。
\end{theorem}
\begin{proof}
	对数域$K$上对称矩阵的阶数$n$作数学归纳法,。\par
	当$n=1$时,因为矩阵合同于自身,同时一阶矩阵都是对角矩阵,所以结论成立。\par
	假设$n-1$阶对称矩阵都合同于对角矩阵,考虑$n$阶矩阵$A=(a_{ij})$。\par
	\textbf{情形一:$a_{11}\ne 0$}\par
	把$A$写成分块矩阵的形式,然后对$A$作初等行变换与初等列变换可得:
	\begin{equation*}
		\begin{pmatrix}
			a_{11} & A_1 \\
			A_1^T & A_2
		\end{pmatrix}
		\longrightarrow
		\begin{pmatrix}
			a_{11} & A_1 \\
			\mathbf{0} & A_2-a_{11}^{-1}A_1^TA_1
		\end{pmatrix}
		\longrightarrow
		\begin{pmatrix}
			a_{11} & \mathbf{0} \\
			\mathbf{0} & A_2-a_{11}^{-1}A_1^TA_1
		\end{pmatrix}
	\end{equation*}
	于是有:
	\begin{equation*}
		\begin{pmatrix}
			1 & \mathbf{0} \\
			-a_{11}^{-1}A_1^T & I_{n-1}
		\end{pmatrix}
		\begin{pmatrix}
			a_{11} & A_1 \\
			A_1^T & A_2
		\end{pmatrix}
		\begin{pmatrix}
			1 & -a_{11}^{-1}A_1 \\
			\mathbf{0} & I_{n-1}
		\end{pmatrix}
		=
		\begin{pmatrix}
			a_{11} & \mathbf{0} \\
			\mathbf{0} & A_2-a_{11}^{-1}A_1^TA_1
		\end{pmatrix}
	\end{equation*}
	因为$A$是一个对称矩阵,所以$A_2$是一个对称矩阵,于是:
	\begin{equation*}
		(A_2-a_{11}^{-1}A_1^TA_1)^T=A_2^T-a_{11}^{-1}A_1^T(A_1^T)^T=A_2-a_{11}^{-1}A_1^TA_1
	\end{equation*}
	所以$A_2-a_{11}^{-1}A_1'A_1$是$n-1$阶对称矩阵。由归纳假设可知存在可逆矩阵$C\in M_{n-1}(K)$使得$C^T(A_2-a_{11}^{-1}A_1'A_1)C=D$,其中$D$是一个对角矩阵,即:
	\begin{equation*}
		\begin{pmatrix}
			1 & \mathbf{0} \\
			\mathbf{0} & C^T
		\end{pmatrix}
		\begin{pmatrix}
			a_{11} & \mathbf{0} \\
			\mathbf{0} & A_2-a_{11}^{-1}A_1^TA_1
		\end{pmatrix}
		\begin{pmatrix}
		1 & \mathbf{0} \\
		\mathbf{0} & C
		\end{pmatrix}
		=
		\begin{pmatrix}
			a_{11} & \mathbf{0} \\
			\mathbf{0} & D
		\end{pmatrix}
	\end{equation*}
	于是有:
	\begin{equation*}
		\begin{pmatrix}
			1 & \mathbf{0} \\
			\mathbf{0} & C^T
		\end{pmatrix}
		\begin{pmatrix}
			1 & \mathbf{0} \\
			-a_{11}^{-1}A_1 & I_{n-1}
		\end{pmatrix}
		\begin{pmatrix}
			a_{11} & A_1 \\
			A_1^T & A_2
		\end{pmatrix}
		\begin{pmatrix}
			1 & -a_{11}^{-1}A_1 \\
			\mathbf{0} & I_{n-1}
		\end{pmatrix}
		\begin{pmatrix}
			1 & \mathbf{0} \\
			\mathbf{0} & C
		\end{pmatrix}
		=
		\begin{pmatrix}
			a_{11} & \mathbf{0} \\
			\mathbf{0} & D
		\end{pmatrix}
	\end{equation*}
	因为:
	\begin{equation*}
		\left[
		\begin{pmatrix}
			1 & -a_{11}^{-1}A_1 \\
			\mathbf{0} & I_{n-1}
		\end{pmatrix}
		\begin{pmatrix}
			1 & \mathbf{0} \\
			\mathbf{0} & C
		\end{pmatrix}
		\right]^T
		=
		\begin{pmatrix}
			1 & \mathbf{0} \\
			\mathbf{0} & C
		\end{pmatrix}^T
		\begin{pmatrix}
			1 & -a_{11}^{-1}A_1 \\
			\mathbf{0} & I_{n-1}
		\end{pmatrix}^T
		=
		\begin{pmatrix}
			1 & \mathbf{0} \\
			\mathbf{0} & C^T
		\end{pmatrix}
		\begin{pmatrix}
			1 & \mathbf{0} \\
			-a_{11}^{-1}A_1 & I_{n-1}
		\end{pmatrix}
	\end{equation*}
	并且:
	\begin{equation*}
		\begin{pmatrix}
			1 & -a_{11}^{-1}A_1 \\
			\mathbf{0} & I_{n-1}
		\end{pmatrix}
		\begin{pmatrix}
			1 & \mathbf{0} \\
			\mathbf{0} & C
		\end{pmatrix}
	\end{equation*}
	是一个可逆矩阵,所以$A$合同于对角矩阵:
	\begin{equation*}
		\begin{pmatrix}
			a_{11} & \mathbf{0} \\
			\mathbf{0} & D
		\end{pmatrix}
	\end{equation*}\par
	\textbf{情形二:$a_{11}=0,\;\text{存在$i\ne 1$使得$a_{ii}\ne0$}$}\par
	把$A$的第$1,i$行呼唤,再把所得矩阵的第$1,i$列呼唤,得到的矩阵$B$的$(1,1)$元即为$a_{ii}\ne0$。根据情形一的讨论,$B$合同于一个对角矩阵。因为$B$是由$A$作成对初等行、列变换得到的,由\cref{lem:CTAC}可得$A\cong B$。由合同的传递性,$A$也合同于一个对角矩阵。\par
	\textbf{情形三:$a_{ii}=0,\;\forall\;i=1,2,\dots,n,\;\text{存在$a_{ij}\ne 0,\;i\ne j$}$}\par
	把$A$的第$j$行加到第$i$行上,再把所得矩阵的第$j$列加到第$i$列上,得到的矩阵$E$的$(i,i)$元即为$2a_{ij}\ne0$。由情形二的讨论,$E$合同于一个对角矩阵。因为$E$是由$A$作成对初等行、列变换得到的,由\cref{lem:CTAC}可得$A\cong E$。由合同的传递性,$A$也合同于一个对角矩阵。\par
	\textbf{情形四:$A=\mathbf{0}$}\par
	因为$\mathbf{0}$是一个对角矩阵,所以结论显然成立。
\end{proof}
\begin{theorem}\label{theo:CongruentRank}
	设对角矩阵$B$是对称矩阵$A$的合同标准形,则$B$对角线上不为$0$的元素的个数等于$A$的秩。
\end{theorem}
\begin{proof}
	因为$A\cong B$,所以存在可逆矩阵$C$使得$C^TAC=B$,于是$\operatorname{rank}(A)=\operatorname{rank}(B)$。
\end{proof}
\subsubsection{实对称矩阵的合同规范形}
\begin{theorem}\label{theo:Congruent1-10}
	对于任意的对称矩阵$A\in M_{n}(\mathbb{R})$,$A$都合同于对角矩阵\\$\operatorname{diag}\{1,1,\dots,1,-1,-1,\dots,-1,0,0,\dots,0\}$,系数为$1$的平方项个数称为$A$的\gls{PositiveInertiaIndex},系数为$-1$的平方项个数称为$A$的\gls{NegativeInertiaIndex},这个对角矩阵称为$A$的\textbf{合同规范形}。
\end{theorem}
\begin{proof}
	任取矩阵$A\in M_{n}(\mathbb{R})$,由\cref{theo:AllCongruent}可得$A$合同一个对角矩阵$B$。对$B$作成对初等行、列变换可将$B$对角线上的元素重新排列,使得正值在前,负值在中间,零值在最后,如此得到对角矩阵$C$,$C$可写作:
	\begin{equation*}
		C=
		\left(
		\begin{array}{*{11}c}
			c_1 & 0 & \cdots & 0 & 0 & 0 & \cdots & 0 & 0 & \cdots & 0 \\
			0 & c_2 & \cdots & 0 & 0 & 0 & \cdots & 0 & 0 & \cdots & 0 \\
			\vdots & \vdots & \ddots & \vdots & \vdots & \vdots & \ddots & \vdots & \vdots & \ddots & \vdots \\
			0 & 0 & \cdots & c_p & 0 & 0 & \cdots & 0 & 0 & \cdots & 0 \\
			0 & 0 & \cdots & 0 & -c_{p+1} & 0 & \cdots & 0 & 0 & \cdots & 0 \\
			0 & 0 & \cdots & 0 & 0 & -c_{p+2} & \cdots & 0 & 0 & \cdots & 0 \\
			\vdots & \vdots & \ddots & \vdots & \vdots & \vdots & \ddots & \vdots & \vdots & \ddots & \vdots \\
			0 & 0 & \cdots & 0 & 0 & 0 & \cdots & -c_r & 0 & \cdots & 0 \\
			0 & 0 & \cdots & 0 & 0 & 0 & \cdots & 0 & 0 & \cdots & 0 \\
			\vdots & \vdots & \ddots & \vdots & \vdots & \vdots & \ddots & \vdots & \vdots & \ddots & \vdots \\
			0 & 0 & \cdots & 0 & 0 & 0 & \cdots & 0 & 0 & \cdots & 0 \\
		\end{array}
		\right)
	\end{equation*}
	其中$c_1,c_2,\dots,c_r>0$。再对$C$作成对初等行、列变换,即先对第$i$行除$\sqrt{c_i}$,再对第$i$列除$\sqrt{c_i},\;i=1,2,\dots,n$,即可得到对角矩阵$D$:
	\begin{equation*}
		D=
		\left(
		\begin{array}{*{11}c}
			1 & 0 & \cdots & 0 & 0 & 0 & \cdots & 0 & 0 & \cdots & 0 \\
			0 & 1 & \cdots & 0 & 0 & 0 & \cdots & 0 & 0 & \cdots & 0 \\
			\vdots & \vdots & \ddots & \vdots & \vdots & \vdots & \ddots & \vdots & \vdots & \ddots & \vdots \\
			0 & 0 & \cdots & 1 & 0 & 0 & \cdots & 0 & 0 & \cdots & 0 \\
			0 & 0 & \cdots & 0 & -1 & 0 & \cdots & 0 & 0 & \cdots & 0 \\
			0 & 0 & \cdots & 0 & 0 & -1 & \cdots & 0 & 0 & \cdots & 0 \\
			\vdots & \vdots & \ddots & \vdots & \vdots & \vdots & \ddots & \vdots & \vdots & \ddots & \vdots \\
			0 & 0 & \cdots & 0 & 0 & 0 & \cdots & -1 & 0 & \cdots & 0 \\
			0 & 0 & \cdots & 0 & 0 & 0 & \cdots & 0 & 0 & \cdots & 0 \\
			\vdots & \vdots & \ddots & \vdots & \vdots & \vdots & \ddots & \vdots & \vdots & \ddots & \vdots \\
			0 & 0 & \cdots & 0 & 0 & 0 & \cdots & 0 & 0 & \cdots & 0 \\
		\end{array}
		\right)
	\end{equation*}
	由\cref{lem:CTAC}可得,$D\cong C$,$C\cong B$,又因为$A\cong B$,由合同的传递性与对称性即可得$A\cong D$。由$A$的任意性结论得证。
\end{proof}
\subsubsection{复对称矩阵的合同规范形}
\begin{theorem}\label{theo:Congruent10}
	对于任意的$A\in M_{n}(\mathbb{C})$,$A$都合同于对角矩阵$\operatorname{diag}\{1,1,\dots,1,0,0,\dots,0\}$,这个对角矩阵称为$A$的合同规范形。
\end{theorem}
\begin{proof}
	任取矩阵$A\in M_{n}(\mathbb{C})$,由\cref{theo:AllCongruent}可得$A\cong B=\operatorname{diag}\{b_1,b_2,\dots,b_r,0,0,\dots,0\}$,其中$r$是矩阵$B$的秩,$b_1,b_2,\dots,b_r\ne0$。设$b_j=r_j\cos\theta_j+ir_j\sin\theta_j,\;\theta_j\in[0,2\pi),\;j=1,2,\dots,r$。因为:
	\begin{equation*}
		\left[\sqrt{r_j}\left(\cos\frac{\theta_j}{2}+i\sin\frac{\theta_j}{2}\right)\right]^2=b_j
	\end{equation*}
	将$\sqrt{r_j}\left(\cos\dfrac{\theta_j}{2}+i\sin\dfrac{\theta_j}{2}\right)$记作$\sqrt{b_j}$,作成对初等行、列变换,即先对第$j$行除$\sqrt{b_j}$,再对第$j$列除$\sqrt{b_j}$,则可得到矩阵$C=\operatorname{diag}\{1,1,\dots,1,0,0\dots,0\}$,其中$1$的个数为$r$。由\cref{lem:CTAC}可得,$B\cong C$。因为$A\cong B$,由合同的传递性,$A\cong C$。由$A$的任意性,结论成立。
\end{proof}