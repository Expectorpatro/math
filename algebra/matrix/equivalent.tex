\section{相抵的应用}

\subsection{广义逆}
\begin{definition}
	设$A\in M_{m\times n}(K)$,一切满足方程组:
	\begin{equation*}
		AXA=A
	\end{equation*}
	的矩阵$X$都被称为是$A$的\gls{GeneralizedInverse},记为$A^-$。
\end{definition}
\begin{theorem}\label{theo:ExistenceOfGeneralizedInverse}
	设非零矩阵$A\in M_{m\times n}(K)$,$\operatorname{rank}(A)=r$且:
	\begin{equation*}
		A=P
		\begin{pmatrix}
			I_r & \mathbf{0} \\
			\mathbf{0} & \mathbf{0}
		\end{pmatrix}
		Q
	\end{equation*}
	其中$P,Q$分别为数域$K$上的$m$阶可逆矩阵和$n$阶可逆矩阵,则矩阵方程:
	\begin{equation*}
		AXA=A
	\end{equation*}
	一定有解,且其通解可表示为:
	\begin{equation*}
		X=Q^{-1}
		\begin{pmatrix}
			I_r & B \\
			C & D
		\end{pmatrix}
		P^{-1}
	\end{equation*}
	其中$B,C,D$分别为数域$K$上任意的$r\times (m-r),\;(n-r)\times r,\;(n-r)\times(m-r)$矩阵。
\end{theorem}
\begin{proof}
	若$X$是上述矩阵方程的一个解,则:
	\begin{gather*}
		P
		\begin{pmatrix}
			I_r & \mathbf{0} \\
			\mathbf{0} & \mathbf{0}
		\end{pmatrix}
		Q
		X
		P
		\begin{pmatrix}
			I_r & \mathbf{0} \\
			\mathbf{0} & \mathbf{0}
		\end{pmatrix}
		Q
		=
		P
		\begin{pmatrix}
			I_r & \mathbf{0} \\
			\mathbf{0} & \mathbf{0}
		\end{pmatrix}
		Q  \\
		\begin{pmatrix}
			I_r & \mathbf{0} \\
			\mathbf{0} & \mathbf{0}
		\end{pmatrix}
		Q
		X
		P
		\begin{pmatrix}
			I_r & \mathbf{0} \\
			\mathbf{0} & \mathbf{0}
		\end{pmatrix}
		=
		\begin{pmatrix}
			I_r & \mathbf{0} \\
			\mathbf{0} & \mathbf{0}
		\end{pmatrix}
	\end{gather*}
	将$QXP$写作如下分块矩阵的形式:
	\begin{equation*}
		QXP=
		\begin{pmatrix}
			H & B \\
			C & D
		\end{pmatrix}
	\end{equation*}
	其中$H,B,C,D$分别为数域$K$上任意的$r\times r,\;r\times (m-r),\;(n-r)\times r,\;(n-r)\times(m-r)$矩阵。于是:
	\begin{gather*}
		\begin{pmatrix}
			I_r & \mathbf{0} \\
			\mathbf{0} & \mathbf{0}
		\end{pmatrix}
		\begin{pmatrix}
			H & B \\
			C & D
		\end{pmatrix}
		\begin{pmatrix}
			I_r & \mathbf{0} \\
			\mathbf{0} & \mathbf{0}
		\end{pmatrix}
		=
		\begin{pmatrix}
			I_r & \mathbf{0} \\
			\mathbf{0} & \mathbf{0}
		\end{pmatrix} \\
		\begin{pmatrix}
			H & \mathbf{0} \\
			\mathbf{0} & \mathbf{0}
		\end{pmatrix}
		=
		\begin{pmatrix}
			I_r & \mathbf{0} \\
			\mathbf{0} & \mathbf{0}
		\end{pmatrix}
	\end{gather*}
	所以$H=I_r$,因此:
	\begin{equation*}
		X=Q^{-1}
		\begin{pmatrix}
			I_r & B \\
			C & D
		\end{pmatrix}
		P^{-1}\qedhere
	\end{equation*}
\end{proof}
\begin{property}\label{prop:A-}
	设$A,\in M_{m\times n}(K),\;B\in M_{m\times q}(K),\;C\in M_{p\times n}(K)$,则广义逆$A^-$具有如下性质:
	\begin{enumerate}
		\item $A^-$唯一的充分必要条件为$A$可逆,此时$A^-=A^{-1}$;
		\item $\operatorname{rank}(A^-)\geqslant\operatorname{rank}(A)=\operatorname{rank}(AA^-)=\operatorname{rank}(A^-A)$;
		\item 若$\mathcal{M}(B)\subset\mathcal{M}(A),\mathcal{M}(C)\subset\mathcal{M}(A^T)$,则$C^TA^-B$与$A^-$的选择无关\info{证明有问题};
		\item $A(A^TA)^-A^T$与$(A^TA)^-$的选择无关;
		\item $A(A^TA)^-A^TA=A,\;A^TA(A^TA)^-A^T=A^T$;
		\item 若$A$对称,则$[(A)^-]^T=(A)^-$;
		\item 若存在$\alpha$使得$c=A^T\alpha$,则$c^T(A^TA)^-A^TA=c^T$。
	\end{enumerate}
\end{property}
\begin{proof}
	(1)\textbf{充分性:}若$A$可逆,则$r=n$,由$A^-$的通解公式,显然此时$A^-$唯一。\par
	\textbf{必要性:}若$A^-$唯一,则$r=n$,显然此时$A$可逆。\par
	(2)由$A^-$的通解公式,$\operatorname{rank}(A^-)\geqslant r=\operatorname{rank}(A)$。因为:
	\begin{gather*}
		AA^-=P
		\begin{pmatrix}
			I_r & \mathbf{0} \\
			\mathbf{0} & \mathbf{0}
		\end{pmatrix}
		QQ^{-1}
		\begin{pmatrix}
			I_r & B \\
			C & D
		\end{pmatrix}
		P^{-1}
		=P
		\begin{pmatrix}
			I_r & B \\
			\mathbf{0} & \mathbf{0}
		\end{pmatrix}
		p^{-1} \\
		A^-A=Q^{-1}
		\begin{pmatrix}
			I_r & B \\
			C & D
		\end{pmatrix}
		P^{-1}P
		\begin{pmatrix}
			I_r & \mathbf{0} \\
			\mathbf{0} & \mathbf{0}
		\end{pmatrix}
		Q
		=Q^{-1}
		\begin{pmatrix}
			I_r & \mathbf{0} \\
			C & \mathbf{0}
		\end{pmatrix}
		Q
	\end{gather*}
	显然,$\operatorname{rank}(AA^-)=\operatorname{rank}(A^-A)=\operatorname{rank}(A)=r$。\par
	(3)由已知条件,存在矩阵$D_1,D_2$使得$B=AD_1,\;C=A^TD_2$,于是:
	\begin{equation*}
		C^TA^-B=D_2^TAA^-AD_1=D_2^TAD_1
	\end{equation*}\par
	(4)由\cref{prop:OrthogonalProjectionMat}(2)可知$A(A^TA)^-A^T$是向$\mathcal{M}(A)$的正交投影阵,由\cref{prop:ProjectionTransformation}(7)和\cref{theo:LinearTransformationMatrix}可得$A(A^TA)^-A$是唯一的,与$(A^TA)^-$的选择无关。\par
	(5)设$B=A(A^TA)^-A^TA-A$,则:
	\begin{align*}
		B^TB
		&=\{A^TA[(A^TA)^-]^TA^T-A^T\}[A(A^TA)^-A^TA-A] \\
		&=A^TA[(A^TA)^-]^TA^TA(A^TA)^-A^TA-A^TA[(A^TA)^-]^TA^TA \\
		&\quad-A^TA(A^TA)^-A^TA+A^TA \\
		&=A^TA[(A^TA)^-]^TA^TA-A^TA[(A^TA)^-]^TA^TA-A^TA+A^TA=\mathbf{0}
	\end{align*}
	所以$B=\mathbf{0}$(考虑$B^TB$主对角线上的元素),于是$A(A^TA)^-A^TA=A$。\par
	设$C=A^TA(A^TA)^-A^T-A^T$,则:
	\begin{align*}
		CC^T
		&=[A^TA(A^TA)^-A^T-A^T]\{A[(A^TA)^-]^TA^TA-A\} \\
		&=A^TA(A^TA)^-A^TA[(A^TA)^-]^TA^TA-A^TA(A^TA)^-A^TA \\
		&\quad-A^TA[(A^TA)^-]^TA^TA+A^TA \\
		&=A^TA[(A^TA)^-]^TA^TA-A^TA-A^TA[(A^TA)^-]^TA^TA+A^TA=\mathbf{0}
	\end{align*}
	所以$C=\mathbf{0}$,于是$A^TA(A^TA)^-A^T=A^T$。\par
	(6)此时有:
	\begin{equation*}
		AXA=A\iff A^TX^TA^T=A^T\iff AX^TA=A
	\end{equation*}\par
	(7)由(5)可得:
	\begin{equation*}
		c^T(A^TA)^-A^TA=\alpha^TA(A^TA)^-A^TA=\alpha^TA=c^T\qedhere
	\end{equation*}
\end{proof}

\subsection{Moore-Penrose广义逆}
\begin{definition}
	设$A\in M_{m\times n}(\mathbb{C})$。若$X\in M_{n\times m}(\mathbb{C})$满足:
	\begin{equation*}
		\begin{cases}
			AXA=A \\
			XAX=X \\
			(AX)^H=AX \\
			(XA)^H=XA
		\end{cases}
	\end{equation*}
	则称$X$为$A$的Moore-Penrose广义逆,记作$A^+$,上述方程组被称为$A$的Penrose方程组。
\end{definition}
\subsubsection{满秩分解导出的广义逆}
\begin{theorem}
	设$A\in M_{m\times n}(\mathbb{C})$,则$A$的Penrose方程组一定有唯一解。对$A$进行满秩分解,设$A=BC$,其中$B,C$分别为列满秩矩阵与行满秩矩阵,则$A$的Penrose方程组的解可表示为:
	\begin{equation*}
		X=C^H(CC^H)^{-1}(B^HB)^{-1}B^H
	\end{equation*}
\end{theorem}
\begin{proof}
	由\cref{theo:RankAAHA}可知$(B^HB)^{-1},(CC^H)^{-1}$存在,将上述$X$代入$A$的Penrose方程组可得:
	\begin{align*}
		&\begin{aligned}
			XAX
			&=C^H(CC^H)^{-1}(B^HB)^{-1}B^HBCC^H(CC^H)^{-1}(B^HB)^{-1}B^H \\
			&=C^H(CC^H)^{-1}(B^HB)^{-1}B^H=X
		\end{aligned} \\
		&AXA=BCC^H(CC^H)^{-1}(B^HB)^{-1}B^HBC=BC=A \\
		&\begin{aligned}
			(AX)^H
			&=X^HA^H=B[(B^HB)^{-1}]^H[(CC^H)^{-1}]^HCC^HB^H \\
			&=B[(B^HB)^{-1}]^H[(CC^H)^{-1}]^HCC^HB^H \\
			&=B[(B^HB)^H]^{-1}[(CC^H)^H]^{-1}CC^HB^H \\
			&=B(B^HB)^{-1}(CC^H)^{-1}CC^HB^H \\
			&=B(B^HB)^{-1}B^H \\
			&=B(CC^H)(CC^H)^{-1}(B^HB)^{-1}B^H=AX
		\end{aligned} \\
		&\begin{aligned}
			(XA)^H
			&=A^HX^H=C^HB^HB[(B^HB)^{-1}]^H[(CC^H)^{-1}]^HC \\
			&=C^H(CC^H)^{-1}C=C^H(CC^H)^{-1}(B^HB)^{-1}(B^HB)C=XA
		\end{aligned}
	\end{align*}
	于是$X$与$A$的Penrose方程组相容,所以$X$是解。
\end{proof}
\subsubsection{奇异值分解导出的广义逆}
\begin{theorem}\label{theo:A+SVD}
	设$A\in M_{m\times n}(\mathbb{C})$,则有:
	\begin{equation*}
		A^+=Q
		\begin{pmatrix}
			\varLambda^{-1} & \mathbf{0} \\
			\mathbf{0} & \mathbf{0}
		\end{pmatrix}
		P^H
	\end{equation*}
	其中$P,Q,\varLambda$为$A$的奇异值分解中相关矩阵。
\end{theorem}
\begin{proof}
	将之代入到$A$的Penrose方程组中可得:
	\begin{gather*}
		\begin{aligned}
			AQ
			\begin{pmatrix}
				\varLambda^{-1} & \mathbf{0} \\
				\mathbf{0} & \mathbf{0}
			\end{pmatrix}
			P^HA
			&=P
			\begin{pmatrix}
				\varLambda & \mathbf{0} \\
				\mathbf{0} & \mathbf{0}
			\end{pmatrix}Q^HQ
			\begin{pmatrix}
				\varLambda^{-1} & \mathbf{0} \\
				\mathbf{0} & \mathbf{0}
			\end{pmatrix}
			P^HP
			\begin{pmatrix}
				\varLambda & \mathbf{0} \\
				\mathbf{0} & \mathbf{0}
			\end{pmatrix}Q^H \\
			&=P\begin{pmatrix}
				\varLambda & \mathbf{0} \\
				\mathbf{0} & \mathbf{0}
			\end{pmatrix}Q^H
			=A
		\end{aligned} \\
		\begin{aligned}
			Q
			\begin{pmatrix}
				\varLambda^{-1} & \mathbf{0} \\
				\mathbf{0} & \mathbf{0}
			\end{pmatrix}
			P^HA
			Q
			\begin{pmatrix}
				\varLambda^{-1} & \mathbf{0} \\
				\mathbf{0} & \mathbf{0}
			\end{pmatrix}
			P^H
			&=Q
			\begin{pmatrix}
				\varLambda^{-1} & \mathbf{0} \\
				\mathbf{0} & \mathbf{0}
			\end{pmatrix}
			P^HP
			\begin{pmatrix}
				\varLambda & \mathbf{0} \\
				\mathbf{0} & \mathbf{0}
			\end{pmatrix}
			Q^HQ
			\begin{pmatrix}
				\varLambda^{-1} & \mathbf{0} \\
				\mathbf{0} & \mathbf{0}
			\end{pmatrix}
			P^H \\
			&=Q
			\begin{pmatrix}
				\varLambda^{-1} & \mathbf{0} \\
				\mathbf{0} & \mathbf{0}
			\end{pmatrix}
			P^H
		\end{aligned} \\
		AQ
		\begin{pmatrix}
			\varLambda^{-1} & \mathbf{0} \\
			\mathbf{0} & \mathbf{0}
		\end{pmatrix}
		P^H=Q
		\begin{pmatrix}
			\varLambda^{-1} & \mathbf{0} \\
			\mathbf{0} & \mathbf{0}
		\end{pmatrix}
		P^HA=I
	\end{gather*}
	因为$I$是Hermitian矩阵,于是$Q
	\begin{pmatrix}
		\varLambda^{-1} & \mathbf{0} \\
		\mathbf{0} & \mathbf{0}
	\end{pmatrix}
	P^H$与$A$的Penrose方程组相容,所以它是解。
\end{proof}
\subsubsection{Moore-Penrose广义逆的性质}
\begin{property}\label{prop:A+}
	设$A\in M_{m\times n}(\mathbb{C})$,则$A$的Moore-Penrose广义逆$A^+$具有如下性质:
	\begin{enumerate}
		\item $A^+$是唯一的;
		\item $(A^+)^+=A$;
		\item $(A^+)^H=(A^H)^+$;
		\item $\operatorname{rank}(A^+)=\operatorname{rank}(A)$;
		\item 若$A$是一个Hermitian矩阵,则:
		\begin{equation*}
			A^+=Q
			\begin{pmatrix}
				\varLambda^{-1} & \mathbf{0} \\
				\mathbf{0} & \mathbf{0}
			\end{pmatrix}Q^H
		\end{equation*}
		其中$\varLambda$为$A$的非零特征值构成的对角矩阵,$Q$是一个正交矩阵;
		\item 若$\alpha$是一个非零向量,则$\alpha^+=\frac{\alpha^H}{||\alpha||^2}$;
		\item $I-A^+A\geqslant \mathbf{0}$;
		\item $(A^HA)^+=A^+(A^H)^+$;
		\item $A^+=(A^HA)^+A^H=A^H(AA^H)^+$。
	\end{enumerate}
\end{property}
\begin{proof}
	(1)设$X_1,X_2$都是$A$的Penrose方程组的解,则:
	\begin{align*}
		X_1
		&=X_1AX_1=X_1(AX_2A)X_1=X_1(AX_2)(AX_1) \\
		&=X_1(AX_2)^H(AX_1)^H=X_1(AX_1AX_2)^H=X_1X_2^H(AX_1A)^H \\
		&=X_1X_2^HA^H=X_1(AX_2)^H=X_1AX_2=X_1(AX_2A)X_2 \\
		&=(X_1A)(X_2A)X_2=(X_1A)^H(X_2A)^HX_2=(X_2AX_1A)^HX_2 \\
		&=(X_2A)^HX_2=X_2AX_2=X_2
	\end{align*}
	所以Penrose方程组的解是唯一的。\par
	(2)由Penrose方程的对称性可直接得到。\par
	(3)由$A^+$的奇异值分解表示(\cref{theo:A+SVD})可得:
	\begin{align*}
		(A^+)^H&=\left[Q
		\begin{pmatrix}
			\varLambda^{-1} & \mathbf{0} \\
			\mathbf{0} & \mathbf{0}
		\end{pmatrix}P^H\right]^H
		=P
		\begin{pmatrix}
			\varLambda^{-1} & \mathbf{0} \\
			\mathbf{0} & \mathbf{0}
		\end{pmatrix}^HQ^H \\
		&=P
		\begin{pmatrix}
			(\varLambda^{-1})^H & \mathbf{0} \\
			\mathbf{0} & \mathbf{0}
		\end{pmatrix}Q^H
		=P
		\begin{pmatrix}
			(\varLambda^H)^{-1} & \mathbf{0} \\
			\mathbf{0} & \mathbf{0}
		\end{pmatrix}Q^H
	\end{align*}
	将其代入$A^H$的Penrose方程组可得:
	\begin{gather*}
		\begin{aligned}
			A^H(A^+)^HA^H
			&=Q
			\begin{pmatrix}
				\varLambda^H & \mathbf{0} \\
				\mathbf{0} & \mathbf{0}
			\end{pmatrix}
			P^HP
			\begin{pmatrix}
				(\varLambda^H)^{-1} & \mathbf{0} \\
				\mathbf{0} & \mathbf{0}
			\end{pmatrix}
			Q^HQ
			\begin{pmatrix}
				\varLambda^H & \mathbf{0} \\
				\mathbf{0} & \mathbf{0}
			\end{pmatrix}
			P^H \\
			&=Q
			\begin{pmatrix}
				\varLambda^H & \mathbf{0} \\
				\mathbf{0} & \mathbf{0}
			\end{pmatrix}
			P^H
			=A^H
		\end{aligned} \\
		\begin{aligned}
			(A^+)^HA^H(A^+)^H
			&=P
			\begin{pmatrix}
				(\varLambda^H)^{-1} & \mathbf{0} \\
				\mathbf{0} & \mathbf{0}
			\end{pmatrix}
			Q^H\begin{pmatrix}
				\varLambda^H & \mathbf{0} \\
				\mathbf{0} & \mathbf{0}
			\end{pmatrix}
			P^HP
			\begin{pmatrix}
				(\varLambda^H)^{-1} & \mathbf{0} \\
				\mathbf{0} & \mathbf{0}
			\end{pmatrix}Q^H \\
			&=P
			\begin{pmatrix}
				(\varLambda^H)^{-1} & \mathbf{0} \\
				\mathbf{0} & \mathbf{0}
			\end{pmatrix}Q^H
			=(A^+)^H
		\end{aligned} \\
		[A^H(A^+)^H]^H=[(A^+)^HA^H]^H=A^+A=I
	\end{gather*}
	因为$I$是Hermitian矩阵,于是$(A^+)^H$与$A^H$的Penrose方程组相容,所以$(A^+)^H=(A^H)^+$。\par
	(4)由$A^+$的奇异值分解表示(\cref{theo:A+SVD})显然可得$\operatorname{rank}(A^+)=\operatorname{rank}(\varLambda)$,而$\operatorname{rank}(\varLambda)=\operatorname{rank}(A)$,所以有$\operatorname{rank}(A^+)=\operatorname{rank}(A)$。\par
	(5)因为$A$是一个Hermitian矩阵,由\cref{prop:HermitianMatEigen}(3)可知存在正交矩阵$Q$使得:
	\begin{equation*}
		A=Q
		\begin{pmatrix}
			\varLambda & \mathbf{0} \\
			\mathbf{0} & \mathbf{0}
		\end{pmatrix}Q^H
	\end{equation*}
	将$Q
	\begin{pmatrix}
		\varLambda^{-1} & \mathbf{0} \\
		\mathbf{0} & \mathbf{0}
	\end{pmatrix}Q^H$代入$A$的Penrose方程组可得:
	\begin{gather*}
		\begin{aligned}
			AQ
			\begin{pmatrix}
				\varLambda^{-1} & \mathbf{0} \\
				\mathbf{0} & \mathbf{0}
			\end{pmatrix}Q^HA
			&=Q
			\begin{pmatrix}
				\varLambda & \mathbf{0} \\
				\mathbf{0} & \mathbf{0}
			\end{pmatrix}
			Q^HQ
			\begin{pmatrix}
				\varLambda^{-1} & \mathbf{0} \\
				\mathbf{0} & \mathbf{0}
			\end{pmatrix}
			Q^HQ
			\begin{pmatrix}
				\varLambda & \mathbf{0} \\
				\mathbf{0} & \mathbf{0}
			\end{pmatrix}
			Q^H \\
			&=Q
			\begin{pmatrix}
				\varLambda & \mathbf{0} \\
				\mathbf{0} & \mathbf{0}
			\end{pmatrix}
			Q^H
			=A
		\end{aligned} \\
		\begin{aligned}
			Q
			\begin{pmatrix}
				\varLambda^{-1} & \mathbf{0} \\
				\mathbf{0} & \mathbf{0}
			\end{pmatrix}
			Q^HAQ
			\begin{pmatrix}
				\varLambda^{-1} & \mathbf{0} \\
				\mathbf{0} & \mathbf{0}
			\end{pmatrix}
			Q^H
			&=Q
			\begin{pmatrix}
				\varLambda^{-1} & \mathbf{0} \\
				\mathbf{0} & \mathbf{0}
			\end{pmatrix}
			Q^HQ
			\begin{pmatrix}
				\varLambda & \mathbf{0} \\
				\mathbf{0} & \mathbf{0}
			\end{pmatrix}
			Q^HQ
			\begin{pmatrix}
				\varLambda^{-1} & \mathbf{0} \\
				\mathbf{0} & \mathbf{0}
			\end{pmatrix}
			Q^H \\
			&=Q
			\begin{pmatrix}
				\varLambda^{-1} & \mathbf{0} \\
				\mathbf{0} & \mathbf{0}
			\end{pmatrix}
			Q^H
		\end{aligned} \\
		\left[AQ
		\begin{pmatrix}
			\varLambda^{-1} & \mathbf{0} \\
			\mathbf{0} & \mathbf{0}
		\end{pmatrix}
		Q^H\right]^H
		=
		\left[Q
		\begin{pmatrix}
			\varLambda^{-1} & \mathbf{0} \\
			\mathbf{0} & \mathbf{0}
		\end{pmatrix}
		Q^HA\right]^H
		=Q
		\begin{pmatrix}
			I & \mathbf{0} \\
			\mathbf{0} & \mathbf{0}
		\end{pmatrix}
		Q^H
	\end{gather*}
	因为$Q
	\begin{pmatrix}
		I & \mathbf{0} \\
		\mathbf{0} & \mathbf{0}
	\end{pmatrix}
	Q^H$是Hermitian矩阵,于是$Q
	\begin{pmatrix}
		\varLambda^{-1} & \mathbf{0} \\
		\mathbf{0} & \mathbf{0}
	\end{pmatrix}Q^H$与$A$的Penrose方程组相容,所以$Q
	\begin{pmatrix}
	\varLambda^{-1} & \mathbf{0} \\
	\mathbf{0} & \mathbf{0}
	\end{pmatrix}Q^H=A^+$。\par
	(6)将$\frac{\alpha^H}{||\alpha||^2}$代入$\alpha$的Penrose方程组可得:
	\begin{gather*}
		\alpha\frac{\alpha^H}{||\alpha||^2}\alpha=\alpha  \\
		\frac{\alpha^H}{||\alpha||^2}\alpha\frac{\alpha^H}{||\alpha||^2}=\frac{\alpha^H}{||\alpha||^2} \\
		\left(\alpha\frac{\alpha^H}{||\alpha||^2}\right)^H=\left(\frac{\alpha^H}{||\alpha||^2}\alpha\right)^H=1
	\end{gather*}
	显然$\dfrac{\alpha^H}{||\alpha||^2}=\alpha^+$。\par
	(7)由$A^+$的奇异值分解表示(\cref{theo:A+SVD})可得:
	\begin{align*}
		I-A^+A&=I-Q
		\begin{pmatrix}
			\varLambda^{-1} & \mathbf{0} \\
			\mathbf{0} & \mathbf{0}
		\end{pmatrix}
		P^HP
		\begin{pmatrix}
			\varLambda & \mathbf{0} \\
			\mathbf{0} & \mathbf{0}
		\end{pmatrix}Q^H
		=I-Q
		\begin{pmatrix}
			I & \mathbf{0} \\
			\mathbf{0} & \mathbf{0}
		\end{pmatrix}Q^H \\
		&=I-\begin{pmatrix}
			I & \mathbf{0} \\
			\mathbf{0} & \mathbf{0}
		\end{pmatrix}
		=\begin{pmatrix}
			\mathbf{0} & \mathbf{0} \\
			\mathbf{0} & I
		\end{pmatrix}\cong
		\begin{pmatrix}
			I & \mathbf{0} \\
			\mathbf{0} & \mathbf{0}
		\end{pmatrix}
	\end{align*}
	由\cref{theo:PositiveSemidefinite}(3)的第三条可知$I-A^+A\geqslant\mathbf{0}$。\par
	(8)由(3)可得:
	\begin{align*}
		A^+(A^H)^+&=A^+(A^+)^H=Q
		\begin{pmatrix}
			\varLambda^{-1} & \mathbf{0} \\
			\mathbf{0} & \mathbf{0}
		\end{pmatrix}
		P^HP
		\begin{pmatrix}
			(\varLambda^H)^{-1} & \mathbf{0} \\
			\mathbf{0} & \mathbf{0}
		\end{pmatrix}Q^HQ^H \\
		&=Q\begin{pmatrix}
			\varLambda^{-1}(\varLambda^H)^{-1} & \mathbf{0} \\
			\mathbf{0} & \mathbf{0}
		\end{pmatrix}Q^H
		=\begin{pmatrix}
			\varLambda^{-1}(\varLambda^{H})^{-1} & \mathbf{0} \\
			\mathbf{0} & \mathbf{0}
		\end{pmatrix}
	\end{align*}
	由$A$的奇异值分解(\cref{theo:SVD})可得:
	\begin{align*}
		A^HA&=\left[P
		\begin{pmatrix}
			\varLambda & \mathbf{0} \\
			\mathbf{0} & \mathbf{0}
		\end{pmatrix}Q^H\right]^HP
		\begin{pmatrix}
			\varLambda & \mathbf{0} \\
			\mathbf{0} & \mathbf{0}
		\end{pmatrix}Q^H \\
		=&Q
		\begin{pmatrix}
			\varLambda^H & \mathbf{0} \\
			\mathbf{0} & \mathbf{0}
		\end{pmatrix}
		P^HP
		\begin{pmatrix}
			\varLambda & \mathbf{0} \\
			\mathbf{0} & \mathbf{0}
		\end{pmatrix}Q^H
		=\begin{pmatrix}
			\varLambda^H\varLambda & \mathbf{0} \\
			\mathbf{0} & \mathbf{0}
		\end{pmatrix}
	\end{align*}
	将$A^+(A^H)^+$代入$A^HA$的Penrose方程组中即可验证得到$(A^HA)^+=A^+(A^H)^+$。\par
	(9)由(8)、(3)和$A^+$的奇异值分解表示(\cref{theo:A+SVD})可得:
	\begin{gather*}
		\begin{aligned}
			(A^HA)^+A^H
			&=A^+(A^H)^+A^H
			=A^+(A^+)^HA^H \\
			&=Q
			\begin{pmatrix}
				\varLambda^{-1} & \mathbf{0} \\
				\mathbf{0} & \mathbf{0}
			\end{pmatrix}
			P^HP
			\begin{pmatrix}
				(\varLambda^H)^{-1} & \mathbf{0} \\
				\mathbf{0} & \mathbf{0}
			\end{pmatrix}
			Q^HQ
			\begin{pmatrix}
				\varLambda^H & \mathbf{0} \\
				\mathbf{0} & \mathbf{0}
			\end{pmatrix}
			P^H \\
			&=Q
			\begin{pmatrix}
				\varLambda^{-1} & \mathbf{0} \\
				\mathbf{0} & \mathbf{0}
			\end{pmatrix}
			P^H=A^+
		\end{aligned} \\
		\begin{aligned}
			A^H(AA^H)^+
			&=A^H(A^H)^+A^+
			=A^H(A^+)^HA^+ \\
			&=Q
			\begin{pmatrix}
				\varLambda^H & \mathbf{0} \\
				\mathbf{0} & \mathbf{0}
			\end{pmatrix}
			P^HP
			\begin{pmatrix}
				(\varLambda^H)^{-1} & \mathbf{0} \\
				\mathbf{0} & \mathbf{0}
			\end{pmatrix}
			Q^HQ
			\begin{pmatrix}
				\varLambda^{-1} & \mathbf{0} \\
				\mathbf{0} & \mathbf{0}
			\end{pmatrix}
			P^H \\
			&=Q
			\begin{pmatrix}
				\varLambda^{-1} & \mathbf{0} \\
				\mathbf{0} & \mathbf{0}
			\end{pmatrix}
			P^H=A^+
		\end{aligned}\qedhere
	\end{gather*}
\end{proof}

\subsection{线性方程组的解}
\begin{theorem}\label{theo:ConsistentLinearEqCondition}
	数域$K$上$n$元非齐次线性方程组$Ax=\beta$有解的充分必要条件为对$A$的任一广义逆$A^-$都有:
	\begin{equation*}
		\beta=AA^-\beta
	\end{equation*}
\end{theorem}
\begin{proof}
	\textbf{(1)必要性:}若$Ax=\beta$有解,取其一个解$\alpha$,于是对$A$的任一广义逆有:
	\begin{equation*}
		\beta=A\alpha=AA^-A\alpha=AA^-\beta
	\end{equation*}
	\textbf{(2)充分性:}若此时对$A$的任一广义逆$A^-$有$\beta=AA^-\beta$,则方程组可化为:
	\begin{equation*}
		Ax=AA^-\beta
	\end{equation*}
	容易看出$A^-\beta$就是$Ax=\beta$的一个解。
\end{proof}
\subsubsection{齐次方程组解的结构}
\begin{theorem}\label{theo:HomogeneousLinearEq'sGeneralSolution}
	若数域$K$上$n$元齐次线性方程组$Ax=\mathbf{0}$有解,则它的通解为:
	\begin{equation*}
		x=(I_n-A^-A)y
	\end{equation*}
	其中$A^-$是$A$的任意一个给定的广义逆,$y$取遍$K^n$中的列向量。
\end{theorem}
\begin{proof}
	任取$y\in K^n$,有:
	\begin{equation*}
		A(I_n-A^-A)y=Ay-AA^-Ay=Ay-Ay=\mathbf{0}
	\end{equation*}
	所以对任意的$y\in K^n$,$(I_n-A^-A)y$都是$Ax=\mathbf{0}$的解。\par
	若$\eta$是$Ax=\mathbf{0}$的一个解,则:
	\begin{equation*}
		(I_n-A^-A)\eta=\eta-A^-A\eta=\eta-A^-\mathbf{0}=\eta
	\end{equation*}
	所以$Ax=\mathbf{0}$的任意一个解$x$都可以表示为$(I_n-A^-A)x$的形式。\par
	综上,$Ax=\mathbf{0}$的通解为$x=(I_n-A^-A)y$。
\end{proof}
\subsubsection{非齐次方程组解的结构}
\begin{theorem}[结构1]
	\label{theo:InhomogeneousLinearEq'sGeneralSolution1}
	若数域$K$上$n$元非齐次线性方程组$Ax=\beta$有解,则它的通解为:
	\begin{equation*}
		x=A^-\beta+(I_n-A^-A)y
	\end{equation*}
	其中$A^-$是$A$的任意一个给定的广义逆,$y$取遍$K^n$中的列向量。
\end{theorem}
\begin{proof}
	由\cref{theo:ConsistentLinearEqCondition}的充分性可知对于给定的这一$A^-$,$A^-\beta$为$Ax=\beta$的一个特解,而由\cref{theo:HomogeneousLinearEq'sGeneralSolution}可知齐次线性方程组$Ax=\mathbf{0}$的通解为$(I_n-A^-A)y$,由\cref{prop:InhomogeneousSLESolution}(3)可得$Ax=\beta$的通解为$x=A^-\beta+(I_n-A^-A)y$。
\end{proof}
\begin{theorem}[结构2]
	\label{theo:InhomogeneousLinearEq'sGeneralSolution2}
	若数域$K$上$n$元非齐次线性方程组$Ax=\beta$有解,则它的通解为:
	\begin{equation*}
		x=A^-\beta
	\end{equation*}
	$A^-$取遍$A$的所有广义逆。
\end{theorem}
\begin{proof}
	由\cref{theo:ConsistentLinearEqCondition}的充分性可知对于任意的$A^-$,$A^-\beta$都是$Ax=\beta$的解。\par
	对于$Ax=\beta$的任意一个解$y$,由\cref{theo:InhomogeneousLinearEq'sGeneralSolution1}可知存在$A$的一个广义逆$G$和$K^n$上的一个列向量$z$,使得:
	\begin{equation*}
		y=G\beta+(I_n-GA)z
	\end{equation*}
	因为$\beta\ne\mathbf{0}$,所以$\beta^H\beta\ne0$,于是存在数域$K$上的矩阵$B=z(\beta^H\beta)^{-1}\beta^H$使得$B\beta=z$,于是:
	\begin{equation*}
		y=G\beta+(I_n-GA)B\beta=[G+(I_n-GA)B]\beta
	\end{equation*}
	因为:
	\begin{align*}
		A[G+(I_n-GA)B]A
		&=AGA+A(I_n-GA)BA \\
		&=A+ABA-AGABA \\
		&=A+ABA-ABA=A
	\end{align*}
	所以$G+(I_n-GA)B$是$A$的一个广义逆,即$Ax=\beta$的任一解可以表示为$A^-\beta$。
\end{proof}
\begin{theorem}
	在数域$K$上相容线性方程组$Ax=\beta$的解集中,$x_0=A^+\beta$为长度最小者。
\end{theorem}
\begin{proof}
	由\cref{theo:InhomogeneousLinearEq'sGeneralSolution1}可知,$Ax=\beta$的通解可以表示为:
	\begin{equation*}
		x=A^+\beta+(I-A^+A)y
	\end{equation*}
	于是:
	\begin{align*}
		||x||
		&=[A^+\beta+(I-A^+A)y]^H[A^+\beta+(I-A^+A)y] \\
		&=||x_0||+\beta^H(A^+)^H(I-A^+A)y \\
		&\quad+y^H(I-A^+A)^HA^+\beta+y^H(I-A^+A)^H(I-A^+A)y \\
		&=||x_0||+2\beta^H(A^+)^H(I-A^+A)y+||(I-A^+A)y||
	\end{align*}
	由\cref{prop:A+}(9)可得:
	\begin{align*}
		(A^+)^H(I-A^+A)
		&=(A^+)^H-(A^+)^HA^+A=(A^H)^+-(A^H)^+A^+A \\
		&=(A^H)^+-[A(A^H)]^+A=\mathbf{0}
	\end{align*}
	于是有$2\beta^H(A^+)^H(I-A^+A)y=0$。因为$||(I-A^+A)y||\geqslant0$,等号成立当且仅当$(I-A^+A)y=\mathbf{0}$,所以$x=A^+\beta=x_0$时长度最小。
\end{proof}