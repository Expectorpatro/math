\section{矩阵的向量空间}

\begin{definition}
	设$A=(\seq{\alpha}{n})\in M_{m\times n}(K)$,将:
	\begin{equation*}
		\left\{\sum_{i=1}^{n}k_i\alpha_i:k_i\in K\right\}\overset{def}{=}\mathcal{M}(A)
	\end{equation*}
\end{definition}

\begin{theorem}\label{theo:VectorSpaceAAAT}
	设$A\in M_{m\times n}(K)$,则:
	\begin{equation*}
		\mathcal{M}(A)=\mathcal{M}(AA^T)
	\end{equation*}
\end{theorem}
\begin{proof}
	由定义,显然$\mathcal{M}(AA^T)\subset\mathcal{M}(A)$。对于任意的$x\perp\mathcal{M}(AA^T)$,有$x^TAA^T=\mathbf{0}$,于是$||A^Tx||^2=x^TAA^Tx=0$,即$A^Tx=\mathbf{0}$,于是$x\perp\mathcal{M}(A)$。\info{回头改证明,同时注意数域问题}
\end{proof}

\begin{theorem}\label{theo:RankAAHA}
	设$A\in M_{m\times n}(\mathbb{C})$,则有:
	\begin{equation*}
		\operatorname{rank}(AA^H)=\operatorname{rank}(A^HA)=\operatorname{rank}(A)
	\end{equation*}
\end{theorem}
\begin{proof}
	\info{线性方程组解链接}
	只需证明方程$A^HAx=\mathbf{0}$与$Ax=\mathbf{0}$同解。注意到$Ax=\mathbf{0}$则必然有$A^HAx=\mathbf{0}$,而若$A^HAx=\mathbf{0}$,则必有$x^HA^HAx=||Ax||=0$,所以$Ax=\mathbf{0}$。\info{链接方程组秩与维数的公式}
	\begin{equation*}
		n-\operatorname{rank}(A^HA)=n-\operatorname{rank}(A)
	\end{equation*}
	所以:
	\begin{equation*}
		\operatorname{rank}(A^HA)=\operatorname{rank}(A)
	\end{equation*}
	同理可得:
	\begin{equation*}
		\operatorname{rank}(AA^H)=\operatorname{rank}(A^H)=\operatorname{rank}(A)
	\end{equation*}
	于是有:
	\begin{equation*}
		\operatorname{rank}(AA^H)=\operatorname{rank}(A^HA)=\operatorname{rank}(A)\qedhere
	\end{equation*}
\end{proof}