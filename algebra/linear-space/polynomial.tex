\section{多项式}

\subsection{环}
\begin{definition}
	设$X$是一个非空集合。如果$X$上有一个运算,即$f:(\alpha,\beta)\rightarrow\gamma(\alpha,\beta,\gamma\in X)$,将该运算称为\textbf{加法},把$\gamma$称为$\alpha$与$\beta$的\textbf{和},记作$\alpha+\beta=\gamma$;同时$X$有另一个运算,即$g:(\alpha,\beta)\rightarrow\delta(\alpha,\beta,\delta\in X)$,将该运算称为\textbf{乘法},把$\delta$称为$k$与$\alpha$的\textbf{积},记作$\alpha\cdot\beta=\delta$。若上述两个运算还满足以下$6$条运算法则:
	\begin{enumerate}
		\item $\forall\;\alpha,\beta\in X,\;\alpha+\beta=\beta+\alpha$;
		\item $\forall\;\alpha,\beta,\gamma\in X,\;(\alpha+\beta)+\gamma=\alpha+(\beta+\gamma)$;
		\item $X$中有一个元素,记作$\mathbf{0}$,称为$X$的\textbf{零元},它使得:
		\begin{equation*}
			\forall\;\alpha\in X,\;\alpha+\mathbf{0}=\alpha
		\end{equation*}
		\item 对于任意的$\alpha\in X$,存在与之对应的$\beta\in X$,称为$\alpha$的\textbf{逆元},记作$-\alpha$,它使得:
		\begin{equation*}
			\alpha+\beta=\mathbf{0}
		\end{equation*}
		\item $\forall\;\alpha,\beta,\gamma\in X,\;(\alpha\beta)\gamma=\alpha(\beta\gamma)$;
		\item $\forall\;\alpha,\beta,\gamma\in X,\;(\alpha+\beta)\gamma=\alpha\gamma+\beta\gamma,\;\alpha(\beta+\gamma)=\alpha\gamma+\alpha\gamma$。
	\end{enumerate}
	那么称$X$是一个\gls{Xing}。
\end{definition}
\begin{property}\label{prop:RingAlgebra}
	环$X$具有如下性质:
	\begin{enumerate}
		\item $X$中的零元是唯一的;
		\item $X$中每个元素的逆元是唯一的;
		\item $\forall\;\alpha\in X,\;\mathbf{0}\alpha=\alpha\mathbf{0}=\mathbf{0}$;
		\item $\forall\;\alpha\in X,\;-(-\alpha)=\alpha$;
		\item 若$X$中有一个元素$e$满足:
		\begin{equation*}
			\forall\;\alpha\in X,\;\alpha e=e\alpha=\alpha
		\end{equation*}
		则称$e$是$X$的\textbf{单位元},单位元是唯一的。
	\end{enumerate}
\end{property}
\begin{proof}
	(1)假设$X$中有两个零元$\mathbf{0}_1,\mathbf{0}_2$且$\mathbf{0}_1\ne\mathbf{0}_2$,由环运算法则(3)可得:
	\begin{equation*}
		\mathbf{0}_1+\mathbf{0}_2=\mathbf{0}_1,\quad\mathbf{0}_2+\mathbf{0}_1=\mathbf{0}_2
	\end{equation*}
	而由环运算法则(1)可得:
	\begin{equation*}
		\mathbf{0}_1+\mathbf{0}_2=\mathbf{0}_2+\mathbf{0}_1
	\end{equation*}
	于是$\mathbf{0}_1=\mathbf{0}_2$,产生矛盾,所以$X$中的零元是唯一的。\par
	(2)任取$X$中的一个元素$\alpha$,假设它有两个负元$\beta_1,\beta_2$。由环运算法则(4)(1)(3)(2)可得:
	\begin{gather*}
		(\beta_1+\alpha)+\beta_2=\mathbf{0}+\beta_2=\beta_2 \\
		(\beta_1+\alpha)+\beta_2=\beta_1+(\alpha+\beta_2)=\beta_1+\mathbf{0}=\beta_1
	\end{gather*}
	所以$\beta_1=\beta_2$,产生矛盾。由$\alpha$的任意性,$X$中每个元素的负元都是唯一的。\par
	(3)由环运算法则(6)(3)可得:
	\begin{equation*}
		\mathbf{0}\alpha+\mathbf{0}\alpha=(\mathbf{0}+\mathbf{0})\alpha=\mathbf{0}\alpha
	\end{equation*}
	两边同时加上$-(\mathbf{0}\alpha)$可得:
	\begin{equation*}
		\mathbf{0}\alpha+\mathbf{0}\alpha+[-(\mathbf{0}\alpha)]=\mathbf{0}\alpha+[-(\mathbf{0}\alpha)]
	\end{equation*}
	由环运算法则(2)(4)(3)可得:
	\begin{equation*}
		\mathbf{0}\alpha=\mathbf{\mathbf{0}}
	\end{equation*}
	由环运算法则(6)中的左分配律同理可得$\alpha\mathbf{0}=\mathbf{0}$。\par
	(4)由环运算法则(4)可得$-\alpha+[-(-\alpha)]=\mathbf{0}$,在两边同时加上$\alpha$由(4)(1)(3)即可得出结论。\par
	(5)假设$X$中有两个单位元$e_1,e_2$,由单位元的定义可得:
	\begin{equation*}
		e_1e_2=e_1,\quad e_1e_2=e_2
	\end{equation*}
	于是$e_1=e_2$,产生矛盾,所以$X$中的单位元是唯一的。
\end{proof}
\begin{definition}
	设$X$是一个环。由\cref{prop:RingAlgebra}(2),定义$f:(\alpha,\beta)\rightarrow\alpha+(-\beta)\in X(\alpha,\beta\in X)$,将该运算称为\textbf{减法},把$\alpha+(-\beta)$称为$\alpha$与$\beta$的\textbf{差},记作$\alpha-\beta$。
\end{definition}
\begin{definition}
	$X$是一个环。对于$\alpha\in X$,若存在$\beta\in X$使得$\alpha\beta=\mathbf{0}$($\beta\alpha=\mathbf{0}$),则称$\alpha$是一个\textbf{左零因子}(\textbf{右零因子})。根据\cref{prop:RingAlgebra}(3),称$X$中的零元$\mathbf{0}$是\textbf{平凡的零因子},称左零因子和右零因子为\textbf{非平凡的零因子}。
\end{definition}
\begin{definition}
	若环$X$还满足乘法交换律,则称$X$为\textbf{交换环}。
\end{definition}
\begin{definition}
	有单位元的无非平凡零因子的交换环称为\textbf{整环}。
\end{definition}
\subsubsection{子环}
\begin{definition}
	如果环$X$的一个非空子集$E$对于$X$上的加法和乘法也构成一个环,则称$E$是$X$的一个\textbf{子环}。
\end{definition}
\begin{theorem}
	环$X$的一个非空子集$E$是$X$的子环的充分必要条件是$E$对于$X$的减法和乘法都封闭,即:
	\begin{equation*}
		\forall\;\alpha,\beta\in E,\;\alpha-\beta,\alpha\beta\in E
	\end{equation*}
\end{theorem}
\begin{proof}
	\textbf{(1)必要性:}因为$E$是$X$的子环,所以$E$对$X$上的加法和乘法封闭,于是只需要证明$E$对$X$的减法封闭。因为$E$是环,所以$E$中存在零元$\mathbf{0}_E$,由环运算法则(3)可得$\mathbf{0}_E+\mathbf{0}_E=\mathbf{0}_E$。因为$\mathbf{0}_E\in E\subseteq X$,由环运算法则(4)可知$\mathbf{0}_E$存在$X$中的逆元$-\mathbf{0}_E$,于是根据环运算法则(2)(4)(3)可得:
	\begin{gather*}
		\mathbf{0}_E+\mathbf{0}_E+(-\mathbf{0}_E)=\mathbf{0}_E+(-\mathbf{0}_E) \\
		\mathbf{0}_E+\mathbf{0}_X=\mathbf{0}_X
	\end{gather*}
	即$\mathbf{0}_E=\mathbf{0}_X$。\par
	任取$a,b\in E$,对于$X$的减法,由减法的定义可知$a-b=a+(-b)_X$。因为$\mathbf{0}_E=\mathbf{0}_X$,所以$b+(-b)_E=\mathbf{0}_E=\mathbf{0}_X$,由\cref{prop:RingAlgebra}(2)可知$(-b)_E=(-b)_X$,所以$a+(-b)_X=a+(-b)_E$。因为$E$对$X$上的加法封闭,所以$a-b\in E$,即$E$对$X$的减法封闭。\par
	\textbf{(2)充分性:}因为$E$对于$X$的减法封闭且$E\ne\varnothing$,所以存在$c\in E$且$c-c=\mathbf{0}_X\in E$,由定义验证可知$\mathbf{0}_X$即为$E$中的零元$\mathbf{0}_E$。\par
	因为$E$对于$X$的减法封闭且$E\ne\varnothing$,所以对于任意的$a\in E$,$\mathbf{0}_X-a=-a\in E$,即$E$中的元素都存在逆元,其在$E$中的逆元就等于在$X$中的逆元。\par
	任取$a,b\in E$,由之前的证明可得$-b\in E$。对于$X$上的加法和减法有$a+b=a-(-b)\in E$,于是$X$上的加法可以成为$E$上的加法,同理$X$上的乘法可以成为$E$上的乘法。\par
	综上,充分性得证。
\end{proof}

\subsection{一元多项式}
\begin{definition}
	数域$K$上的\textbf{一元多项式}指的是如下形式的表达式:
	\begin{equation*}
		a_nx^n+a_{n-1}x^{n-1}+\cdots+a_1x+a_0
	\end{equation*}
	其中$x$是一个符号,$n\in\mathbb{N}^+$,$a_i\in K,\;i=0,1,\dots,n$。规定$x^0=1$,称$x$为\textbf{不定元},$a_i$为\textbf{系数},$a_ix^i$为\textbf{i次项},$a_0$也称为\textbf{常数项}。两个这种形式的表达式相等当且仅当它们在除去系数为$0$的项后含有完全相同的项。系数全为$0$的一元多项式称为\textbf{零多项式},记作$0$。
\end{definition}
\begin{definition}
	将数域$K$上所有一元多项式组成的集合记作$K[x]$。设$m>n$,对于任意的$f(x)=\sum\limits_{i=0}^{m}a_ix_i,g(x)=\sum\limits_{i=0}^{n}b_ix_i\in K[x]$,在$K[x]$中定义如下运算:
	\begin{enumerate}
		\item \textbf{加法:}$f(x)+g(x)=\sum\limits_{i=0}^{m}(a_i+b_i)x_i$;
		\item \textbf{纯量乘法:}$f(x)g(x)=\sum\limits_{k=0}^{m+n}\left(\sum_{i+j=k}^{}a_ib_j\right)x^k$;
	\end{enumerate}
	那么$K[x]$构成一个环。
\end{definition}
\begin{proof}
	由数域中加法和乘法的封闭性可知如上定义的加法和乘法对$K[x]$是封闭的。\par
	接下来证明如上定义的加法和乘法满足环中的$6$条运算法则:
	\begin{enumerate}
		\item 因为数域内的数满足加法交换律与加法结合律,所以$K[x]$上的加法满足环运算法则(1)(2);
		\item 对任意的$f(x)\in K[x]$,有$f(x)+0=f(x)$,因此$K[x]$中存在零元且它就是$0$,$K[x]$上的加法满足环运算法则(3);
		\item 对任意的$f(x)=\sum\limits_{i=0}^{m}a_ix_i\in K[x]$,取$-f(x)=\sum\limits_{i=0}^{m}(-a_i)x_i$,则有$f(x)+[-f(x)]=0$。由$f(x)$的任意性,$K[x]$中的每个元素都具有负元。因此,$K[x]$上的加法满足环运算法则(4);
		\item 因为数域内的数满足乘法结合律和乘法分配律,所以$K[x]$上的乘法满足环运算法则(5)(6)。
	\end{enumerate}
	证明完毕。
\end{proof}
\begin{property}
	$K[x]$具有如下性质:
	\begin{enumerate}
		\item $K[x]$是一个整环;
	\end{enumerate}
\end{property}