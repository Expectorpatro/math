\section{多项式}

\subsection{环}
\begin{definition}
	设$R$是一个非空集合。如果$XR$上有一个运算,即$f:(\alpha,\beta)\rightarrow\gamma(\alpha,\beta,\gamma\in R)$,将该运算称为\textbf{加法},把$\gamma$称为$\alpha$与$\beta$的\textbf{和},记作$\alpha+\beta=\gamma$;同时$R$有另一个运算,即$g:(\alpha,\beta)\rightarrow\delta(\alpha,\beta,\delta\in R)$,将该运算称为\textbf{乘法},把$\delta$称为$k$与$\alpha$的\textbf{积},记作$\alpha\cdot\beta=\delta$。若上述两个运算还满足以下$6$条运算法则:
	\begin{enumerate}
		\item $\forall\;\alpha,\beta\in R,\;\alpha+\beta=\beta+\alpha$;
		\item $\forall\;\alpha,\beta,\gamma\in R,\;(\alpha+\beta)+\gamma=\alpha+(\beta+\gamma)$;
		\item $R$中有一个元素,记作$\mathbf{0}$,称为$X$的\textbf{零元},它使得:
		\begin{equation*}
			\forall\;\alpha\in R,\;\alpha+\mathbf{0}=\alpha
		\end{equation*}
		\item 对于任意的$\alpha\in R$,存在与之对应的$\beta\in R$,称为$\alpha$的\textbf{逆元},记作$-\alpha$,它使得:
		\begin{equation*}
			\alpha+\beta=\mathbf{0}
		\end{equation*}
		\item $\forall\;\alpha,\beta,\gamma\in R,\;(\alpha\beta)\gamma=\alpha(\beta\gamma)$;
		\item $\forall\;\alpha,\beta,\gamma\in R,\;(\alpha+\beta)\gamma=\alpha\gamma+\beta\gamma,\;\alpha(\beta+\gamma)=\alpha\gamma+\alpha\gamma$。
	\end{enumerate}
	那么称$R$是一个\gls{Ring}。
\end{definition}
\begin{definition}
	$R$是一个环。对于$\alpha\in R$,若存在$\beta\in R$使得$\alpha\beta=\mathbf{0}$($\beta\alpha=\mathbf{0}$),则称$\alpha$是一个\textbf{左零因子}(\textbf{右零因子})。
\end{definition}
\begin{definition}
	有如下几种特殊的环:
	\begin{enumerate}
		\item 若环$R$还满足乘法交换律,则称$R$为\textbf{交换环};
		\item 若环$R$中有一个元素$e$满足:
		\begin{equation*}
			\forall\;\alpha\in R,\;\alpha e=e\alpha=\alpha
		\end{equation*}
		则称$e$是$R$的\textbf{单位元},称$R$是\textbf{有单位元的环};
		\item 若环$R$中没有零因子,则称$R$是\textbf{无零因子的环};
		\item 有单位元的无零因子的交换环称为\textbf{整环}。
	\end{enumerate}
\end{definition}
\begin{property}
	环$R$具有如下性质:
	\begin{enumerate}
		\item $R$中的零元是唯一的;
		\item $R$中每个元素的逆元是唯一的;
		\item $\forall\;\alpha\in R,\;\mathbf{0}\alpha=\alpha\mathbf{0}=\mathbf{0}$;
		\item $\forall\;\alpha\in R,\;-(-\alpha)=\alpha$;
		\item 若$R$是有单位元的环,则单位元是唯一的;
	\end{enumerate}
\end{property}
\begin{proof}
	(1)假设$R$中有两个零元$\mathbf{0}_1,\mathbf{0}_2$且$\mathbf{0}_1\ne\mathbf{0}_2$,由环运算法则(3)可得:
	\begin{equation*}
		\mathbf{0}_1+\mathbf{0}_2=\mathbf{0}_1,\quad\mathbf{0}_2+\mathbf{0}_1=\mathbf{0}_2
	\end{equation*}
	而由环运算法则(1)可得:
	\begin{equation*}
		\mathbf{0}_1+\mathbf{0}_2=\mathbf{0}_2+\mathbf{0}_1
	\end{equation*}
	于是$\mathbf{0}_1=\mathbf{0}_2$,产生矛盾,所以$R$中的零元是唯一的。\par
	(2)任取$R$中的一个元素$\alpha$,假设它有两个负元$\beta_1,\beta_2$。由环运算法则(4)(3)(2)可得:
	\begin{gather*}
		(\beta_1+\alpha)+\beta_2=\mathbf{0}+\beta_2=\beta_2 \\
		(\beta_1+\alpha)+\beta_2=\beta_1+(\alpha+\beta_2)=\beta_1+\mathbf{0}=\beta_1
	\end{gather*}
	所以$\beta_1=\beta_2$,产生矛盾。由$\alpha$的任意性,$R$中每个元素的负元都是唯一的。\par
	(3)由环运算法则(6)可得:
	\begin{equation*}
		\mathbf{0}\alpha+\mathbf{0}\alpha=(\mathbf{0}+\mathbf{0})\alpha=\mathbf{0}\alpha
	\end{equation*}
	两边同时加上$-(\mathbf{0}\alpha)$可得:
	\begin{equation*}
		\mathbf{0}\alpha+\mathbf{0}\alpha+[-(\mathbf{0}\alpha)]=\mathbf{0}\alpha+[-(\mathbf{0}\alpha)]
	\end{equation*}
	由环运算法则(2)(4)和(3)可得:
	\begin{equation*}
		\mathbf{0}\alpha=\mathbf{\mathbf{0}}
	\end{equation*}
	由环运算法则(6)中的左分配律同理可得$\alpha\mathbf{0}=\mathbf{0}$。\par
	(4)由环运算法则(4)可得$-\alpha+[-(-\alpha)]=\mathbf{0}$,在两边同时加上$\alpha$即可得出结论。\par
	(5)假设$R$中有两个单位元$e_1,e_2$,由单位元的定义可得:
	\begin{equation*}
		e_1e_2=e_1,\quad e_1e_2=e_2
	\end{equation*}
	于是$e_1=e_2$,产生矛盾,所以$R$中的单位元是唯一的。
\end{proof}
\begin{definition}
	在环$R$中定义减法:
	\begin{equation*}
		\forall\;\alpha,\beta\in R,\;\alpha-\beta=\alpha+(-\beta)
	\end{equation*}
\end{definition}
\subsubsection{子环}
\begin{definition}
	如果环$R$的一个非空子集$R_1$对于$R$上的加法和乘法也构成一个环,则称$R_1$是$R$的一个\textbf{子环}。
\end{definition}
\begin{theorem}
	环$R$的一个非空子集$R_1$是$R$的子环的充分必要条件是$R_1$对于$R$的减法和乘法都封闭,即:
	\begin{equation*}
		\forall\;\alpha,\beta\in R_1,\;\alpha-\beta,\alpha\beta\in R_1
	\end{equation*}
\end{theorem}
\begin{proof}
	\textbf{(1)必要性:}因为$R_1$是$R$的子环,所以$R_1$对$R$上的加法和乘法封闭,于是只需要证明对任意的$\alpha\in R_1$,它在$R$中的负元$-\alpha\in R_1$即可。
\end{proof}