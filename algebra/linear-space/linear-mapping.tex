\section{线性映射}

\subsection{线性映射的定义与基本性质}
\begin{definition}
	设$X,Y$是域$F$上的线性空间,$X$到$Y$上的一个映射$\mathcal{T}$如果对任意的$\alpha,\beta\in X$和任意的$k_1,k_2\in F$,有:
	\begin{equation*}
		\mathcal{T}(k_1\alpha+k_2\beta)=k_1\mathcal{T}\alpha+k_2\mathcal{T}\beta
	\end{equation*}
	则称$\mathcal{T}$是$X$到$Y$的一个\gls{LinearMapping}。若$Y=X$,则称$\mathcal{T}$为$X$上的\gls{LinearTransformation};若$Y=F$,则称$\mathcal{T}$为$X$上的\gls{LinearFunction}。
\end{definition}
\subsubsection{线性映射空间与线性映射的运算}
\begin{definition}
	设$X,Y$是域$F$上的线性空间,将$X$到$Y$的所有线性映射组成的集合记为$\operatorname{Hom}(X,Y)$。设$\mathcal{T}_1,\mathcal{T}_2\in\operatorname{Hom}(X,Y),\;k\in F$,定义线性映射的加法与纯量乘法如下:
	\begin{gather*}
		(\mathcal{T}_1+\mathcal{T}_2)\alpha=\mathcal{T}_1\alpha+\mathcal{T}_2\alpha,\;\forall\;\alpha\in X \\
		(k\mathcal{T}_1)\alpha=k\mathcal{T}_1\alpha,\;\forall\;\alpha\in X
	\end{gather*}
	容易验证$\operatorname{Hom}(X,Y)$成为域$F$上的一个线性空间。$X=Y$时将$\operatorname{Hom}(X,Y)$简记为$\operatorname{Hom}(X)$。
\end{definition}
\begin{definition}
	设$X,Y$是域$F$上的线性空间,$\mathcal{T}_1,\mathcal{T}_2\in\operatorname{Hom}(X,Y)$,定义线性映射的减法如下:
	\begin{equation*}
		\mathcal{T}_1-\mathcal{T}_2=\mathcal{T}_1+(-\mathcal{T}_2)
	\end{equation*}
\end{definition}
\begin{definition}
	设$X,Y$是域$F$上的线性空间,$\mathcal{T}\in\operatorname{Hom}(X,Y)$,若对任意的$\alpha\in X$,有$\mathcal{T}\alpha=\mathbf{0}_Y$,则称$\mathcal{T}$为$X$到$Y$的\gls{ZeroMapping},记作$\mathcal{O}$。
\end{definition}
\begin{definition}
	设$X,Y,Z$是域$F$上的线性空间,$\mathcal{T}_1\in\operatorname{Hom}(X,Y),\mathcal{T}_2\in\operatorname{Hom}(Z,Y)$,定义线性映射乘法如下:
	\begin{equation*}
		\forall\;\alpha\in X,\;(\mathcal{T}_2\mathcal{T}_1)\alpha=\mathcal{T}_2(\mathcal{T}_1\alpha)
	\end{equation*}
\end{definition}
\begin{definition}
	设$X,Y$是域$F$上的线性空间,$\mathcal{T}_1\in\operatorname{Hom}(X,Y)$,若存在$\mathcal{T}_2\in\operatorname{Hom}(Y,X)$使得$\mathcal{T}_1\mathcal{T}_2=\mathcal{I}_Y,\;\mathcal{T}_2\mathcal{T}_1=\mathcal{I}_X$,则称$\mathcal{T}_1$是可逆的,$\mathcal{T}_2$是$\mathcal{T}_1$的逆映射。
\end{definition}
\begin{theorem}
	设$X,Y,Z$是域$F$上的线性空间,$\mathcal{T}_1\in\operatorname{Hom}(X,Y),\mathcal{T}_2\in\operatorname{Hom}(Z,Y)$,则$\mathcal{T}_2\mathcal{T}_1\in\operatorname{Hom}(X,Z)$。
\end{theorem}
\begin{proof}
	只需注意到对任意的$\alpha,\beta\in X,\;k_1,k_2\in F$,有:
	\begin{align*}
		(\mathcal{T}_2\mathcal{T}_1)(k_1\alpha+k_2\beta)
		&=\mathcal{T}_2[\mathcal{T}_1(k_1\alpha+k_2\beta)]
		=\mathcal{T}_2(k_1\mathcal{T}_1\alpha+k_2\mathcal{T}_1\beta) \\
		&=k_1\mathcal{T}_2(\mathcal{T}_1\alpha)+k_2\mathcal{T}_2(\mathcal{T}_1\beta)
		=k_1(\mathcal{T}_2\mathcal{T}_1)\alpha+k_2(\mathcal{T}_2\mathcal{T}_1)\beta\qedhere
	\end{align*}
\end{proof}
\begin{definition}
	设$X,Y$是域$F$上的线性空间,$\mathcal{T}$是$X$到$Y$上的一个线性映射,分别称:
	\begin{equation*}
		\{\alpha\in X:\mathcal{T}\alpha=\mathbf{0}_Y\},\quad
		\{\mathcal{T}\alpha:\alpha\in X\}
	\end{equation*}
	为$\mathcal{T}$的\gls{Kernel}与\gls{Image},将它们分别记作$\operatorname{Ker}\mathcal{T}$和$\operatorname{Im}\mathcal{T}$。
\end{definition}
\subsubsection{线性空间的性质}
\begin{property}\label{prop:LinearMapping}
	设$X,Y$是域$F$上的线性空间,$\mathcal{T}$是$X$到$Y$上的线性映射,则:
	\begin{enumerate}
		\item 若$\mathcal{T}$可逆,则$\mathcal{T}$是$X$到$Y$上的同构映射;
		\item $\mathcal{T}\mathbf{0}_X=\mathbf{0}_Y$;
		\item 对于任意的$\alpha\in X$,有$\mathcal{T}(-\alpha)=-\mathcal{T}\alpha$;
		\item 对于任意的$\seq{\alpha}{n}\in X,\;\seq{k}{n}\in F$,有:
		\begin{equation*}
			\mathcal{T}\left(\sum_{i=1}^{n}k_i\alpha_i\right)=\sum_{i=1}^{n}k_i\mathcal{T}\alpha_i
		\end{equation*}
		这表明,如果$X$是有限维的,那么只要知道$X$的一组基在$\mathcal{T}$下的象,那么$X$中所有向量在$\mathcal{T}$下的象就都确定了;
		\item 若$\seq{\alpha}{n}\in X$线性相关,则$\seq{\mathcal{T}\alpha}{n}$线性相关;
		\item $\operatorname{Ker}\mathcal{T}$和$\operatorname{Im}\mathcal{T}$分别是$X$和$Y$的子空间;
		\item $\mathcal{T}$是单射当且仅当$\operatorname{Ker}\mathcal{T}=\mathbf{0}_X$;
		\item $\mathcal{T}$是满射当且仅当$\operatorname{Im}\mathcal{T}=Y$。
		\item $X/\operatorname{Ker}\mathcal{T}$与$\operatorname{Im}\mathcal{T}$在映射:
		\begin{equation*}
			\sigma:\alpha+\operatorname{Ker}\mathcal{T}\longrightarrow\mathcal{T}\alpha
		\end{equation*}
		下同构;
		\item 若$X$是有限维的,则$\operatorname{Ker}\mathcal{T}$和$\operatorname{Im}\mathcal{T}$都是有限维的,且有:
		\begin{equation*}
			\dim(X)=\dim(\operatorname{Ker}\mathcal{T})+\dim(\operatorname{Im}\mathcal{T})
		\end{equation*}
		\item 若$\dim(X)=\dim Y=n<+\infty$,则$\mathcal{T}$是单射当且仅当$\mathcal{T}$是满射;
		\item 设$\seq{\alpha}{n}\in X$,则有$\mathcal{T}<\seq{\alpha}{n}>=<\seq{\mathcal{T}\alpha}{n}>$;
	\end{enumerate}
\end{property}
\begin{proof}
	(1)(2)(3)(4)(5)(8)证明都是显然的,只需参考\cref{prop:IsomorphicOfLinearSpace}即可,这是因为线性映射只比同构映射少了双射这一条件,所以同构映射不涉及双射条件的性质对于线性映射也成立。\par
	(6)任取$\alpha,\beta\in\operatorname{Ker}\mathcal{T}$和$k_1,k_2\in F$,则有:
	\begin{equation*}
		\mathcal{T}(k_1\alpha+k_2\beta)=k_1\mathcal{T}\alpha+k_2\mathcal{T}\beta=\mathbf{0}
	\end{equation*}
	于是$k_1\alpha+k_2\beta\in\operatorname{Ker}\mathcal{T}$,所以$\operatorname{Ker}\mathcal{T}$是$X$的子空间。\par
	任取$\mathcal{T}\alpha,\mathcal{T}\beta\in\operatorname{Im}\mathcal{T}$和$k_1,k_2\in F$,则有:
	\begin{equation*}
		k_1\mathcal{T}\alpha+k_2\mathcal{T}\beta=\mathcal{T}(k_1\alpha+k_2\beta)
	\end{equation*}
	因为$X$是一个线性空间,所以$k_1\alpha+k_2\beta\in X$,于是$\mathcal{T}(k_1\alpha+k_2\beta)\in\operatorname{Im}\mathcal{T}$,因此$\operatorname{Im}\mathcal{T}$是$Y$的子空间。\par
	(7)\textbf{充分性:}假设此时$\mathcal{T}$不是单射,则存在$\mathcal{T}\alpha,\mathcal{T}\beta\in\mathcal{T}$使得$\mathcal{T}\alpha=\mathcal{T}\beta$且$\alpha\ne\beta$,而此时$\mathcal{T}\alpha-\mathcal{T}\beta=\mathcal{T}(\alpha-\beta)=\mathbf{0}_Y$,由已知条件可得$\alpha-\beta=\mathbf{0}_X$,即$\alpha=\beta$,矛盾。\par
	\textbf{必要性:}由(2)可知$\mathcal{T}\mathbf{0}_X=\mathbf{0}_Y$,因为$\mathcal{T}$是一个单射,所以$\operatorname{Ker}\mathcal{T}=\mathbf{0}_X$。\par
	(9)先证明$\sigma$是一个映射。若$\alpha+\operatorname{Ker}\mathcal{T}=\beta+\operatorname{Ker}\mathcal{T}$,则$\alpha-\beta\in\operatorname{Ker}\mathcal{T}$,即$\mathcal{T}(\alpha-\beta)=\mathcal{T}\alpha-\mathcal{T}\beta=\mathbf{0}_Y$,于是$\mathcal{T}\alpha=\mathcal{T}\beta$,所以$\sigma$是一个映射。\par
	任取$\alpha+\operatorname{Ker}\mathcal{T},\beta+\operatorname{Ker}\mathcal{T}\in X/\operatorname{Ker}\mathcal{T}$和$k_1,k_2\in F$,则有:
	\begin{align*}
		\sigma[k_1(\alpha+\operatorname{Ker}\mathcal{T})+k_2(\beta+\operatorname{Ker}\mathcal{T})]
		&=\sigma(k_1\alpha+k_2\beta+\operatorname{Ker}\mathcal{T})
		=\mathcal{T}(k_1\alpha+k_2\beta) \\
		&=k_1\mathcal{T}\alpha+k_2\mathcal{T}\beta
		=k_1\sigma(\alpha+\operatorname{Ker}\mathcal{T})+k_2(\beta+\operatorname{Ker}\mathcal{T})
	\end{align*}
	所以$\sigma$是一个线性映射。\par
	显然$\sigma$是一个满射。\par
	若存在$\alpha+\operatorname{Ker}\mathcal{T},\beta+\operatorname{Ker}\mathcal{T}\in X$满足$\alpha+\operatorname{Ker}\mathcal{T}\ne\beta+\operatorname{Ker}\mathcal{T}$且$\mathcal{T}\alpha=\mathcal{T}\beta$,则此时有$\mathcal{T}(\alpha-\beta)=\mathcal{T}\alpha-\mathcal{T}\beta=\mathbf{0}_Y$,所以$\alpha-\beta\in\operatorname{Ker}\mathcal{T}$,即$\alpha+\operatorname{Ker}\mathcal{T}=\beta+\operatorname{Ker}\mathcal{T}$,矛盾,因此$\sigma$是个单射。\par
	综上,$\sigma$是一个双射且是一个线性映射,于是$X/\operatorname{Ker}\mathcal{T}$与$\operatorname{Im}\mathcal{T}$在$\sigma$下同构。\par
	(10)因为$X$是有限维的,由\cref{theo:QuotientDim}可知$X/\operatorname{Ker}\mathcal{T}$和$\operatorname{Ker}\mathcal{T}$都是有限维的。由\cref{theo:IsomorphicDim}和(9)可知$\dim(X/\operatorname{Ker}\mathcal{T})=\dim(\operatorname{Im}\mathcal{T})$,于是$\operatorname{Im}\mathcal{T}$也是有限维的。由\cref{theo:QuotientDim}可得:
	\begin{equation*}
		\dim(\operatorname{Im}\mathcal{T})=\dim(X/\operatorname{Ker}\mathcal{T})=\dim(X)-\dim(\operatorname{Ker}\mathcal{T})
	\end{equation*}\par
	(11)由(7)(10)(6)和\cref{theo:DimSubspace}可得:
	\begin{align*}
		\mathcal{T}\text{是单射}
		&\iff\operatorname{Ker}\mathcal{T}=\mathbf{0}_X	\iff\dim(\operatorname{Ker}\mathcal{T})=0 \\
		&\iff\dim Y=\dim(X)=\dim(\operatorname{Im}\mathcal{T})
		\iff\mathcal{T}\text{是满射}
	\end{align*}\par
	(12)由(4)可得;
	\begin{align*}
		<\seq{\mathcal{T}\alpha}{n}>&=\left\{\sum_{i=1}^{n}c_i\mathcal{T}\alpha_i:c_i\in F\right\}=\left\{\mathcal{T}\left(\sum_{i=1}^{n}c_i\alpha_i\right):c_i\in F\right\} \\
		&=\mathcal{T}<\seq{\alpha}{n}>\qedhere
	\end{align*}
\end{proof}
\begin{definition}
	设$X,Y$是域$F$上的线性空间,$X$是有限维的,$\mathcal{T}\in\operatorname{Hom}(X,Y)$,由\cref{prop:LinearMapping}(10)可知$\operatorname{Ker}(\mathcal{T}),\operatorname{Im}(\mathcal{T})$都是有限维的,称$\dim(\operatorname{Ker}\mathcal{T})$为$\mathcal{T}$的\gls{Nullity},称$\dim(\operatorname{Im}\mathcal{T})$为$\mathcal{T}$的秩,记为$\operatorname{rank}(\mathcal{T})$。
\end{definition}
\begin{definition}
	设$X,Y$是域$F$上的线性空间,$\mathcal{T}\in\operatorname{Hom}(X,Y)$,称$Y/\operatorname{Im}\mathcal{T}$为$\mathcal{T}$的\gls{Cokernel}。
\end{definition}
\begin{theorem}
	设$X,Y$是域$F$上的线性空间,$\mathcal{T}\in\operatorname{Hom}(X,Y)$,则$\mathcal{T}$是满射当且仅当$\operatorname{Coker}\mathcal{T}=\mathbf{0}$。
\end{theorem}
\begin{proof}
	$\mathcal{T}$是满射$\iff\operatorname{Im}\mathcal{T}=Y\iff Y/\operatorname{Im}\mathcal{T}=\mathbf{0}$。这里的$\mathbf{0}$实际上是商空间的零元,也即$Y$。
\end{proof}

\subsection{线性映射的矩阵表示}
\begin{definition}
	设$X,Y$分别为域$F$上的$m$维、$n$维线性空间,$\mathcal{T}$是$X$到$Y$的一个线性映射。由\cref{prop:LinearMapping}(4)可知$\mathcal{T}$被它在$X$的一组基上的作用所决定。取$X$的一组基$\seq{\alpha}{m}$和$Y$的一组基$\seq{\beta}{n}$,则:
	\begin{equation*}
		(\seq{\mathcal{T}\alpha}{m})=(\seq{\beta}{n})
		\begin{pmatrix}
			a_{11} & a_{12} & \cdots & a_{1m} \\
			a_{21} & a_{22} & \cdots & a_{2m} \\
			\vdots & \vdots & \ddots & \vdots \\
			a_{n1} & a_{n2} & \cdots & a_{nm} \\
		\end{pmatrix}
	\end{equation*}
	将$(\seq{\mathcal{T}\alpha}{m})$记作$\mathcal{T}(\seq{\alpha}{m})$,将上式右端矩阵记为$A$,称$A$为$\mathcal{T}$在$X$的基$\seq{\alpha}{m}$和$Y$的基$\seq{\beta}{n}$下的矩阵。若$X=Y$,取$(\seq{\beta}{n})$为$(\seq{\alpha}{m})$,称$A$为$\mathcal{T}$在基$\seq{\alpha}{m}$下的矩阵。
\end{definition}
\begin{theorem}\label{theo:LinearTransformationMatrix}
	设$X,Y$分别为域$F$上的$m$维、$n$维线性空间,则映射$\sigma:\mathcal{T}\in\operatorname{Hom}(X,Y)\longrightarrow\mathcal{T}$在$X$的一组基和$Y$的一组基下的矩阵$A$是$\operatorname{Hom}(X,Y)$到$M_{n\times m}(F)$的一个同构映射,于是有:
	\begin{equation*}
		\operatorname{Hom}(X,Y)\cong M_{n\times m}(F),\quad\dim[\operatorname{Hom}(X,Y)]=\dim(X)\dim(Y)=mn
	\end{equation*}
\end{theorem}
\begin{proof}
	任取$X$的一组基$\seq{\alpha}{m}$和$Y$的一组基$\seq{\beta}{n}$。\par
	任意的$\mathcal{T}\in\operatorname{Hom}(X,Y)$都存在对应的$A\in M_{n\times m}(F)$使得:
	\begin{equation*}
		(\seq{\mathcal{T}\alpha}{m})=(\seq{\beta}{n})A
	\end{equation*}
	因为$\seq{\beta}{n}$是$Y$的一组基,由\cref{prop:LinearlyRepresentation}(1)可知$T\alpha_i$由$\seq{\beta}{n}$表出的方式唯一,所以$A$是唯一的。\par
	任取$B=(b_{ij})\in M_{n\times m}(F)$,令:
	\begin{equation*}
		\mathcal{T}:\alpha=\sum_{i=1}^{m}a_i\alpha_i\in X\longrightarrow\sum_{i=1}^{m}a_i\sum_{j=1}^{n}b_{ji}\beta_j\in Y
	\end{equation*}
	因为$\seq{\alpha}{m}$是$X$的一组基,由\cref{prop:LinearlyRepresentation}(1)可知$\alpha$由$\seq{\alpha}{m}$表出的方式唯一,所以$\mathcal{T}$是一个映射。显然$\mathcal{T}$是一个线性映射。因为:
	\begin{equation*}
		(\seq{\mathcal{T}\alpha}{m})=(\seq{\beta}{n})B
	\end{equation*}
	所以$B$是$\mathcal{T}$在$X$的基$\seq{\alpha}{m}$和$Y$的基$\seq{\beta}{n}$下的矩阵。因为$\mathcal{T}$满足:
	\begin{equation*}
		\mathcal{T}\alpha_i=\sum_{j=1}^{n}b_{ji}\beta_j,\quad i=1,2,\dots,m
	\end{equation*}
	且$\seq{\alpha}{m}$是$X$的一组基,由\cref{prop:LinearMapping}(4)可知$B$所对应的$\mathcal{T}$是唯一的。\par
	综上,存在$\operatorname{Hom}(X,Y)$到$M_{n\times m}(K)$上的一个双射$\sigma$。\par
	任取$\mathcal{T}_1,\mathcal{T}_2\in\operatorname{Hom}(X,Y),\;\sigma(\mathcal{T}_1)=C,\;\sigma(\mathcal{T}_2)=D$,则对任意的$k_1,k_2\in F$有:
	\begin{align*}
		&[\seq{(k_1\mathcal{T}_1+k_2\mathcal{T}_2)\alpha}{m}] \\
		=&(k_1\mathcal{T}_1\alpha_1+k_2\mathcal{T}_2\alpha_1,k_1\mathcal{T}_1\alpha_2+k_2\mathcal{T}_2\alpha_2,\dots,k_1\mathcal{T}_1\alpha_m+k_2\mathcal{T}_2\alpha_m) \\
		=&(k_1\mathcal{T}_1\alpha_1,k_1\mathcal{T}_1\alpha_2,\dots,k_1\mathcal{T}_1\alpha_m)+(k_2\mathcal{T}_2\alpha_1,k_2\mathcal{T}_2\alpha_2,\dots,k_2\mathcal{T}_2\alpha_m) \\
		=&k_1(\mathcal{T}_1\alpha_1,\mathcal{T}_1\alpha_2,\dots,\mathcal{T}_1\alpha_m)+k_2(\mathcal{T}_2\alpha_1,\mathcal{T}_2\alpha_2,\dots,\mathcal{T}_2\alpha_m) \\
		=&k_1(\seq{\beta}{n})C+k_2(\seq{\beta}{n})D=(\seq{\beta}{n})(k_1C+k_2D)
	\end{align*}
	即:
	\begin{equation*}
		\sigma(k_1\mathcal{T}_1+k_2\mathcal{T}_2)=k_1C+k_2D=k_1\sigma(\mathcal{T}_1)+k_2\sigma(\mathcal{T}_2)
	\end{equation*}
	所以$\sigma$是一个线性映射。\par
	综上,映射$\sigma:\mathcal{T}\in\operatorname{Hom}(X,Y)\longrightarrow\mathcal{T}$在$X$的一组基和$Y$的一组基下的矩阵$A$是$\operatorname{Hom}(X,Y)$到$M_{n\times m}(F)$的一个同构映射,$\operatorname{Hom}(X,Y)\cong M_{n\times m}(F)$,由\cref{theo:IsomorphicDim}可得$\dim[\operatorname{Hom}(X,Y)]=mn=\dim(X)\dim(Y)$。
\end{proof}
\subsubsection{向量在线性映射下象的坐标}
\begin{theorem}\label{theo:LinearMappingCoordinate}
	设$X,Y$分别为域$F$上的$m$维、$n$维线性空间,$\mathcal{T}$是$X$到$Y$的一个线性映射,$\mathcal{T}$在$X$的一组基$\seq{\alpha}{m}$和$Y$的一组基$\seq{\beta}{n}$下的矩阵为$A$,向量$\alpha$在$\seq{\alpha}{m}$下的坐标为$x$,则$\mathcal{T}\alpha$在$\seq{\beta}{n}$下的坐标为$Ax$。
\end{theorem}
\begin{proof}
	显然:
	\begin{equation*}
		\mathcal{T}\alpha=\mathcal{T}[(\seq{\alpha}{m})x]=[\seq{\mathcal{T}\alpha}{m}]x=(\seq{\beta}{n})Ax\qedhere
	\end{equation*}
\end{proof}

