\section{线性变换}

\subsection{线性变换的定义与基本性质}
\begin{definition}
	设$X,Y$是域$F$上的线性空间,$X$到$Y$上的一个映射$\mathcal{T}$如果对任意的$\alpha,\beta\in X$和任意的$k_1,k_2\in F$,有:
	\begin{equation*}
		\mathcal{T}(k_1\alpha+k_2\beta)=k_1\mathcal{T}\alpha+k_2\mathcal{T}\beta
	\end{equation*}
	则称$\mathcal{T}$是$X$到$Y$的一个\gls{LinearMapping}。若$Y=X$,则称$\mathcal{T}$为$X$上的\gls{LinearTransformation};若$Y=F$,则称$\mathcal{T}$为$X$上的\gls{LinearFunction}。
\end{definition}
\subsubsection{线性映射空间与线性映射的运算}
\begin{definition}
	设$X,Y$是域$F$上的线性空间,将$X$到$Y$的所有线性映射组成的集合记为$\operatorname{Hom}(X,Y)$。设$\mathcal{T}_1,\mathcal{T}_2\in\operatorname{Hom}(X,Y),\;k\in F$,定义线性映射的加法与纯量乘法如下:
	\begin{gather*}
		(\mathcal{T}_1+\mathcal{T}_2)\alpha=\mathcal{T}_1\alpha+\mathcal{T}_2\alpha,\;\forall\;\alpha\in X \\
		(k\mathcal{T}_1)=k\mathcal{T}_1\alpha,\;\forall\;\alpha\in X
	\end{gather*}
	容易验证$\operatorname{Hom}(X,Y)$成为域$F$上的一个线性空间。$X=Y$时将$\operatorname{Hom}(X,Y)$简记为$\operatorname{Hom}(X)$。
\end{definition}
\begin{definition}
	设$X,Y$是域$F$上的线性空间,$\mathcal{T}_1,\mathcal{T}_2\in\operatorname{Hom}(X,Y)$,定义线性映射的减法如下:
	\begin{equation*}
		\mathcal{T}_1-\mathcal{T}_2=\mathcal{T}_1+(-\mathcal{T}_2)
	\end{equation*}
\end{definition}
\begin{definition}
	设$X,Y,Z$是域$F$上的线性空间,$\mathcal{T}_1\in\operatorname{Hom}(X,Y),\mathcal{T}_2\in\operatorname{Hom}(Z,Y)$,定义线性映射乘法如下:
	\begin{equation*}
		\forall\;\alpha\in X,\;(\mathcal{T}_2\mathcal{T}_1)\alpha=\mathcal{T}_2(\mathcal{T}_1\alpha)
	\end{equation*}
\end{definition}
\begin{theorem}
	设$X,Y,Z$是域$F$上的线性空间,$\mathcal{T}_1\in\operatorname{Hom}(X,Y),\mathcal{T}_2\in\operatorname{Hom}(Z,Y)$,则$\mathcal{T}_2\mathcal{T}_1\in\operatorname{Hom}(X,Z)$。
\end{theorem}
\begin{proof}
	只需注意到对任意的$\alpha,\beta\in X,\;k_1,k_2\in F$,有:
	\begin{align*}
		(\mathcal{T}_2\mathcal{T}_1)(k_1\alpha+k_2\beta)
		&=\mathcal{T}_2[\mathcal{T}_1(k_1\alpha+k_2\beta)]
		=\mathcal{T}_2(k_1\mathcal{T}_1\alpha+k_2\mathcal{T}_1\beta) \\
		&=k_1\mathcal{T}_2(\mathcal{T}_1\alpha)+k_2\mathcal{T}_2(\mathcal{T}_1\beta)
		=k_1(\mathcal{T}_2\mathcal{T}_1)\alpha+k_2(\mathcal{T}_2\mathcal{T}_1)\beta\qedhere
	\end{align*}
\end{proof}
\begin{definition}
	设$X$是域$F$上的线性空间,$\mathcal{T}\in\operatorname{Hom}(X)$,定义$\mathcal{T}$的正整数指数幂如下:
	\begin{equation*}
		\mathcal{T}^n=\underbrace{\mathcal{T} \cdot \mathcal{T} \cdots \mathcal{T}}_{n\text{个}\mathcal{T}}
		,\;n\in \mathbb{N}^+
	\end{equation*}
	若$\mathcal{T}$可逆,还可以定义$\mathcal{T}$的负整数指数幂如下:
	\begin{equation*}
		\mathcal{T}^{-n}=(\mathcal{T}^{-1})^n,\;n\in\mathbb{N}^+
	\end{equation*}
\end{definition}
\begin{definition}
	设$X,Y$是域$F$上的线性空间,$\mathcal{T}$是$X$到$Y$上的一个线性映射,分别称:
	\begin{equation*}
		\{\alpha\in X:\mathcal{T}\alpha=\mathbf{0}_Y\},\quad
		\{\mathcal{T}\alpha:\alpha\in X\}
	\end{equation*}
	为$\mathcal{T}$的\gls{Kernel}与\gls{Image},将它们分别记作$\operatorname{Ker}\mathcal{T}$和$\operatorname{Im}\mathcal{T}$。
\end{definition}
\begin{property}\label{prop:LinearMapping}
	设$X,Y$是域$F$上的线性空间,$\mathcal{T}$是$X$到$Y$上的线性映射,则:
	\begin{enumerate}
		\item 若$\mathcal{T}$可逆,则$\mathcal{T}$是$X$到$Y$上的同构映射;
		\item $\mathcal{T}\mathbf{0}_X=\mathbf{0}_Y$;
		\item 对于任意的$\alpha\in X$,有$\mathcal{T}(-\alpha)=-\mathcal{T}\alpha$;
		\item 对于任意的$\seq{\alpha}{n}\in X,\;\seq{k}{n}\in F$,有:
		\begin{equation*}
			\mathcal{T}\left(\sum_{i=1}^{n}k_i\alpha_i\right)=\sum_{i=1}^{n}k_i\mathcal{T}\alpha_i
		\end{equation*}
		这表明,如果$X$是有限维的,那么只要知道$X$的一个基在$\mathcal{T}$下的象,那么$X$中所有向量在$\mathcal{T}$下的象就都确定了。
		\item 若$\seq{\alpha}{n}\in X$线性相关,则$\seq{\mathcal{T}\alpha}{n}$线性相关。
		\item $\operatorname{Ker}\mathcal{T}$和$\operatorname{Im}\mathcal{T}$分别是$X$和$Y$的子空间;
		\item $\mathcal{T}$是单射当且仅当$\operatorname{Ker}\mathcal{T}=\mathbf{0}_X$;
		\item $\mathcal{T}$是满射当且仅当$\operatorname{Im}\mathcal{T}=Y$。
		\item $X/\operatorname{Ker}\mathcal{T}$与$\operatorname{Im}\mathcal{T}$在映射:
		\begin{equation*}
			\sigma:\alpha+\operatorname{Ker}\mathcal{T}\longrightarrow\mathcal{T}\alpha
		\end{equation*}
		下同构;
		\item 若$X$是有限维的,则$\operatorname{Ker}\mathcal{T}$和$\operatorname{Im}\mathcal{T}$都是有限维的,且有:
		\begin{equation*}
			\dim X=\dim(\operatorname{Ker}\mathcal{T})+\dim(\operatorname{Im}\mathcal{T})
		\end{equation*}
		\item 若$\dim X=\dim Y=n<+\infty$,则$\mathcal{T}$是单射当且仅当$\mathcal{T}$是满射。
	\end{enumerate}
\end{property}
\begin{proof}
	(1)(2)(3)(4)(5)(8)证明都是显然的,只需参考\cref{prop:IsomorphicOfLinearSpace}即可,这是因为线性映射只比同构映射少了双射这一条件,所以同构映射不涉及双射条件的性质对于线性映射也成立。\par
	(6)任取$\alpha,\beta\in\operatorname{Ker}\mathcal{T}$和$k_1,k_2\in F$,则有:
	\begin{equation*}
		\mathcal{T}(k_1\alpha+k_2\beta)=k_1\mathcal{T}\alpha+k_2\mathcal{T}\beta=\mathbf{0}
	\end{equation*}
	于是$k_1\alpha+k_2\beta\in\operatorname{Ker}\mathcal{T}$,所以$\operatorname{Ker}\mathcal{T}$是$X$的子空间。\par
	任取$\mathcal{T}\alpha,\mathcal{T}\beta\in\operatorname{Im}\mathcal{T}$和$k_1,k_2\in F$,则有:
	\begin{equation*}
		k_1\mathcal{T}\alpha+k_2\mathcal{T}\beta=\mathcal{T}(k_1\alpha+k_2\beta)
	\end{equation*}
	因为$X$是一个线性空间,所以$k_1\alpha+k_2\beta\in X$,于是$\mathcal{T}(k_1\alpha+k_2\beta)\in\operatorname{Im}\mathcal{T}$,因此$\operatorname{Im}\mathcal{T}$是$Y$的子空间。\par
	(7)\textbf{充分性:}假设此时$\mathcal{T}$不是单射,则存在$\mathcal{T}\alpha,\mathcal{T}\beta\in\mathcal{T}$使得$\mathcal{T}\alpha=\mathcal{T}\beta$且$\alpha\ne\beta$,而此时$\mathcal{T}\alpha-\mathcal{T}\beta=\mathcal{T}(\alpha-\beta)=\mathbf{0}_Y$,由已知条件可得$\alpha-\beta=\mathbf{0}_X$,即$\alpha=\beta$,矛盾。\par
	\textbf{必要性:}由(2)可知$\mathcal{T}\mathbf{0}_X=\mathbf{0}_Y$,因为$\mathcal{T}$是一个单射,所以$\operatorname{Ker}\mathcal{T}=\mathbf{0}_X$。\par
	(9)先证明$\sigma$是一个映射。若$\alpha+\operatorname{Ker}\mathcal{T}=\beta+\operatorname{Ker}\mathcal{T}$,则$\alpha-\beta\in\operatorname{Ker}\mathcal{T}$,即$\mathcal{T}(\alpha-\beta)=\mathcal{T}\alpha-\mathcal{T}\beta=\mathbf{0}_Y$,于是$\mathcal{T}\alpha=\mathcal{T}\beta$,所以$\sigma$是一个映射。\par
	任取$\alpha+\operatorname{Ker}\mathcal{T},\beta+\operatorname{Ker}\mathcal{T}\in X/\operatorname{Ker}\mathcal{T}$和$k_1,k_2\in F$,则有:
	\begin{align*}
		\sigma[k_1(\alpha+\operatorname{Ker}\mathcal{T})+k_2(\beta+\operatorname{Ker}\mathcal{T})]
		&=\sigma(k_1\alpha+k_2\beta+\operatorname{Ker}\mathcal{T})
		=\mathcal{T}(k_1\alpha+k_2\beta) \\
		&=k_1\mathcal{T}\alpha+k_2\mathcal{T}\beta
		=k_1\sigma(\alpha+\operatorname{Ker}\mathcal{T})+k_2(\beta+\operatorname{Ker}\mathcal{T})
	\end{align*}
	所以$\sigma$是一个线性映射。\par
	显然$\sigma$是一个满射。\par
	若存在$\alpha+\operatorname{Ker}\mathcal{T},\beta+\operatorname{Ker}\mathcal{T}\in X$满足$\alpha+\operatorname{Ker}\mathcal{T}\ne\beta+\operatorname{Ker}\mathcal{T}$且$\mathcal{T}\alpha=\mathcal{T}\beta$,则此时有$\mathcal{T}(\alpha-\beta)=\mathcal{T}\alpha-\mathcal{T}\beta=\mathbf{0}_Y$,所以$\alpha-\beta\in\operatorname{Ker}\mathcal{T}$,即$\alpha+\operatorname{Ker}\mathcal{T}=\beta+\operatorname{Ker}\mathcal{T}$,矛盾,因此$\sigma$是个单射。\par
	综上,$\sigma$是一个双射且是一个线性映射,于是$X/\operatorname{Ker}\mathcal{T}$与$\operatorname{Im}\mathcal{T}$在$\sigma$下同构。\par
	(10)因为$X$是有限维的,由\cref{theo:QuotientDim}可知$X/\operatorname{Ker}\mathcal{T}$和$\operatorname{Ker}\mathcal{T}$都是有限维的。由\cref{theo:IsomorphicDim}和(9)可知$\dim(X/\operatorname{Ker}\mathcal{T})=\dim(\operatorname{Im}\mathcal{T})$,于是$\operatorname{Im}\mathcal{T}$也是有限维的。由\cref{theo:QuotientDim}可得:
	\begin{equation*}
		\dim(\operatorname{Im}\mathcal{T})=\dim(X/\operatorname{Ker}\mathcal{T})=\dim X-\dim(\operatorname{Ker}\mathcal{T})
	\end{equation*}\par
	(11)由(7)(10)(6)和\cref{theo:DimSubspace}可得:
	\begin{align*}
		\mathcal{T}\text{是单射}
		&\Leftrightarrow\operatorname{Ker}\mathcal{T}=\mathbf{0}_X	\Leftrightarrow\dim(\operatorname{Ker}\mathcal{T})=0 \\
		&\Leftrightarrow\dim Y=\dim X=\dim(\operatorname{Im}\mathcal{T})
		\Leftrightarrow\mathcal{T}\text{是满射}\qedhere
	\end{align*}
\end{proof}
\begin{definition}
	设$X,Y$是域$F$上的线性空间,$X$是有限维的,$\mathcal{T}\in\operatorname{Hom}(X,Y)$,由\cref{prop:LinearMapping}(10)可知$\operatorname{Ker}(\mathcal{T}),\operatorname{Im}(\mathcal{T})$都是有限维的,称$\dim(\operatorname{Ker}\mathcal{T})$为$\mathcal{T}$的\gls{Nullity},称$\dim(\operatorname{Im}\mathcal{T})$为$\mathcal{T}$的秩,记为$\operatorname{rank}(\mathcal{T})$。
\end{definition}
\begin{definition}
	设$X,Y$是域$F$上的线性空间,$\mathcal{T}\in\operatorname{Hom}(X,Y)$,称$Y/\operatorname{Im}\mathcal{T}$为$\mathcal{T}$的\gls{Cokernel}。
\end{definition}
\begin{theorem}
	设$X,Y$是域$F$上的线性空间,$\mathcal{T}\in\operatorname{Hom}(X,Y)$,则$\mathcal{T}$是满射当且仅当$\operatorname{Coker}\mathcal{T}=\mathbf{0}$。
\end{theorem}
\begin{proof}
	$\mathcal{T}$是满射$\Leftrightarrow\operatorname{Im}\mathcal{T}=Y\Leftrightarrow Y/\operatorname{Im}\mathcal{T}=\mathbf{0}$。这里的$\mathbf{0}$实际上是商空间的零元,也即$Y$。
\end{proof}

\subsection{线性映射的矩阵表示}
\begin{definition}
	设$X,Y$分别为域$F$上的$m$维、$n$维线性空间,$\mathcal{T}$是$X$到$Y$的一个线性映射。由\cref{prop:LinearMapping}(4)可知$\mathcal{T}$被它在$X$的一个基上的作用所决定。取$X$的一组基$\seq{\alpha}{m}$和$Y$的一组基$\seq{\beta}{n}$,则:
	\begin{equation*}
		(\seq{\mathcal{T}\alpha}{m})=(\seq{\beta}{n})
		\begin{pmatrix}
			a_{11} & a_{12} & \cdots & a_{1m} \\
			a_{21} & a_{22} & \cdots & a_{2m} \\
			\vdots & \vdots & \ddots & \vdots \\
			a_{n1} & a_{n2} & \cdots & a_{nm} \\
		\end{pmatrix}
	\end{equation*}
	将$(\seq{\mathcal{T}\alpha}{m})$记作$\mathcal{T}(\seq{\alpha}{m})$,将上式右端矩阵记为$A$,称$A$为$\mathcal{T}$在$X$的基$\seq{\alpha}{m}$和$Y$的基$\seq{\beta}{n}$下的矩阵。若$X=Y$,取$(\seq{\beta}{n})$为$(\seq{\alpha}{m})$,称$A$为$\mathcal{T}$在基$\seq{\alpha}{m}$下的矩阵。
\end{definition}
\begin{theorem}
	设$X$是域$F$上的一个$n$维线性空间,$\mathcal{T}\in\operatorname{Hom}(X)$,它在$X$的一个基$\seq{\alpha}{n}$下的矩阵为$A$,则$\operatorname{rank}(A)=\operatorname{rank}(\mathcal{T})$。
\end{theorem}
\begin{proof}
	由\cref{prop:LinearMapping}(4)可知任意的$\mathcal{T}\alpha\in X$都可以由$\seq{\mathcal{T}\alpha}{n}$线性表出,于是$\operatorname{Im}(\mathcal{T})\subseteq<\seq{\mathcal{T}\alpha}{n}>$。由\cref{prop:LinearMapping}(6)以及$\seq{\mathcal{T}\alpha}{n}\in\operatorname{Im}(\mathcal{T})$,所以$<\seq{\mathcal{T}\alpha}{n}>\subseteq\operatorname{Im}(\mathcal{T})$。于是$\operatorname{Im}(\mathcal{T})=<\seq{\mathcal{T}\alpha}{n}>$,即:
	\begin{equation*}
		\operatorname{rank}(\mathcal{T})=\dim(\operatorname{Im}\mathcal{T})=\dim<\seq{\mathcal{T}\alpha}{n}>
	\end{equation*}
	因为$\dim(X)=n$,由\cref{theo:IsomorphicDim}可知$X\cong F^n$
\end{proof}

\subsection{常见线性映射}
\begin{definition}
	设$X,Y$是域$F$上的线性空间,$\mathcal{T}_1\in\operatorname{Hom}(X,Y),\;\mathcal{T}_2\in\operatorname{Hom}(X)$,则:
	\begin{enumerate}
		\item 若对任意的$\alpha\in X$,有$\mathcal{T}_1\alpha=\mathbf{0}_Y$,则称$\mathcal{T}_1$为$X$到$Y$的\gls{ZeroMapping},记作$\mathcal{O}$;
		\item 若对任意的$\alpha\in X$,有$\mathcal{T}_2\alpha=\alpha$,则称$\mathcal{T}_2$为$X$到$Y$的\gls{IdentityTransformation},记作$\mathcal{I}$;
		\item 给定$k\in F$,若对任意的$\alpha\in X$,有$\mathcal{T}_2\alpha=k\alpha$,则称$\mathcal{T}_2$为$X$到$Y$的\gls{ScalarTransformation},记作$\mathcal{K}$;
		\item 若$\mathcal{T}_2^2=\mathcal{I}$,则称$\mathcal{T}_2$为\gls{Involution};
		\item 若$\mathcal{T}_2^2=\mathcal{T}_2$,则称$\mathcal{T}_2$为\gls{IdempotentTransformation};
		\item 若存在$n\in\mathbb{N}^+$使得$\mathcal{T}_2^n=\mathcal{O}$,则称$\mathcal{T}_2$为\gls{NilpotentTransformation},使得$\mathcal{T}_2^n=\mathcal{O}$成立的最小正整数$n$被称为是$\mathcal{T}_2$的\gls{NilpotentIndex};
		\item 若$E$和$W$是$X$的子空间,且有$X=E\oplus W$,若对任意的$\alpha=\alpha_1+\alpha_2\in X,\;\alpha_1\in E,\;\alpha_2\in W$,有:
		\begin{equation*}
			\mathcal{T}_2\alpha=\alpha_1
		\end{equation*}
		则称$\mathcal{T}_2$为平行于$W$在$E$上的\gls{ProjectionTransformation},记为$\mathcal{P}_E$。
	\end{enumerate}
\end{definition}
\begin{definition}
	设$X$是域$F$上的线性空间,$\mathcal{T}_1,\mathcal{T}_2\in \operatorname{Hom}(X)$。若$\mathcal{T}_1\mathcal{T}_2=\mathcal{T}_2\mathcal{T}_1=\mathcal{O}$,则称$\mathcal{T}_1$和$\mathcal{T}_2$\textbf{正交}。
\end{definition}

\subsubsection{投影变换}
\begin{property}\label{prop:ProjectionTransformation}
	设$X$是域$F$上的线性空间,$E$和$W$是$X$的子空间,且有$X=E\oplus W$,$\mathcal{P}_E$是平行于$W$在$E$上的投影变换,$\mathcal{P}_W$是平行于$E$在$W$上的投影变换,则:
	\begin{enumerate}
		\item $\mathcal{P}_E$是线性变换;
		\item $\mathcal{P}_E$是幂等变换;
		\item 幂等变换$\mathcal{T}\in\operatorname{Hom}(X)$是平行于$\operatorname{Ker}\mathcal{T}$在$\operatorname{Im}\mathcal{T}$上的投影变换,此时$X=\operatorname{Im}\mathcal{T}\oplus\operatorname{Ker}\mathcal{T}$;
		\item $\mathcal{P}_E$与$\mathcal{P}_W$正交;
		\item $\mathcal{P}_E+\mathcal{P}_W=\mathcal{I}$;
		\item 若$\mathcal{T}_1,\mathcal{T}_2\in\operatorname{Hom}(X)$,$\mathcal{T}_1$和$\mathcal{T}_2$是正交的幂等变换,且$\mathcal{T}_1+\mathcal{T}_2=\mathcal{I}$,则$X=\operatorname{Im}\mathcal{T}_1\oplus\operatorname{Im}\mathcal{T}_2$,并且有$\mathcal{T}_1$是平行于$\operatorname{Im}\mathcal{T}_2$在$\operatorname{Im}\mathcal{T}_1$上的投影,$\mathcal{T}_2$是平行于$\operatorname{Im}\mathcal{T}_1$在$\operatorname{Im}\mathcal{T}_2$上的投影;
	\end{enumerate}
\end{property}
\begin{proof}
	(1)任取$\alpha,\beta\in X$和$k_1,k_2\in F$,其中$\alpha=\alpha_1+\alpha_2,\;\beta=\beta_1+\beta_2,\;\alpha_1,\beta_1\in E,\;\alpha_2,\beta_2\in W$,于是有:
	\begin{align*}
		\mathcal{P}_E(k_1\alpha+k_2\beta)
		=\mathcal{P}_E[(k_1\alpha_1+k_2\beta_1)+(k_1\alpha_2+k_2\beta_2)]
		=k_1\alpha_1+k_2\beta_1
		=k_1\mathcal{P}_E\alpha+k_2\mathcal{P}_E\beta
	\end{align*}
	所以$\mathcal{P}_E$是线性变换。\par
	(2)任取$\alpha=\alpha_1+\alpha_2\in X,\;\alpha_1\in E,\;\alpha_2\in W$,则:
	\begin{equation*}
		\mathcal{P}_E(\mathcal{P}_E\alpha)=\mathcal{P}_E(\alpha_1)=\alpha_1=\mathcal{P}_E\alpha
	\end{equation*}
	所以$\mathcal{P}_E$是幂等变换。\par
	(3)任取$\alpha\in X$,则$\mathcal{T}\alpha\in\operatorname{Im}\mathcal{T}$。因为:
	\begin{equation*}
		\mathcal{T}(\alpha-\mathcal{T}\alpha)=\mathcal{T}\alpha-\mathcal{T}^2\alpha=\mathcal{T}\alpha-\mathcal{T}\alpha=\mathbf{0}
	\end{equation*}
	所以$\alpha-\mathcal{T}\alpha\in\operatorname{Ker}\mathcal{T}$。因为$\alpha=\mathcal{T}\alpha+\alpha-\mathcal{T}\alpha$,所以$X=\operatorname{Im}\mathcal{T}+\operatorname{Ker}\mathcal{T}$。\par
	任取$\beta\in\operatorname{Im}\mathcal{T}\cap\operatorname{Ker}\mathcal{T}$,则存在$\gamma\in X$使得$\mathcal{T}\gamma=\beta$,且有$\mathcal{T}\beta=\mathbf{0}$,于是:
	\begin{equation*}
		\mathbf{0}=\mathcal{T}\beta=\mathcal{T}(\mathcal{T}\gamma)=\mathcal{T}^2\gamma=\mathcal{T}\gamma=\beta
	\end{equation*}
	所以$\operatorname{Im}\mathcal{T}\cap\operatorname{Ker}\mathcal{T}=\mathbf{0}$,由\cref{theo:DirectSum}(3)可知$X=\operatorname{Im}\mathcal{T}\oplus\operatorname{Ker}\mathcal{T}$。\par
	记$\operatorname{Im}\mathcal{T}=E,\;\operatorname{Ker}\mathcal{T}=W$,对上述$\alpha=\mathcal{T}\alpha+\alpha-\mathcal{T}\alpha$,有$\mathcal{P}_E\alpha=\mathcal{T}\alpha$。由$\alpha$的任意性,$\mathcal{T}=\mathcal{P}_E$。\par
	(4)任取$\alpha=\alpha_1+\alpha_2\in X,\;\alpha_1\in E,\;\alpha_2\in W$,则:
	\begin{equation*}
		\mathcal{P}_E(\mathcal{P}_W\alpha)=\mathcal{P}_E\alpha_2=\mathbf{0},\;
		\mathcal{P}_W(\mathcal{P}_E\alpha)=\mathcal{P}_W\alpha_2=\mathbf{0}
	\end{equation*}
	所以$\mathcal{P}_E\mathcal{P}_W=\mathcal{P}_W\mathcal{P}_E=\mathcal{O}$,即$\mathcal{P}_E$与$\mathcal{P}_W$正交。\par
	(5)任取$\alpha=\alpha_1+\alpha_2\in X,\;\alpha_1\in E,\;\alpha_2\in W$,则:
	\begin{equation*}
		(\mathcal{P}_E+\mathcal{P}_W)\alpha=\mathcal{P}_E\alpha+\mathcal{P}_W\alpha=\alpha_1+\alpha_2=\alpha
	\end{equation*}
	于是$\mathcal{P}_E+\mathcal{P}_W=\mathcal{I}$。\par
	(6)任取$\alpha\in X$,则$\alpha=(\mathcal{T}_1+\mathcal{T}_2)\alpha=\mathcal{T}_1\alpha+\mathcal{T}_2\alpha$,所以$X=\operatorname{Im}\mathcal{T}_1+\operatorname{Im}\mathcal{T}_2$。\par
	任取$\beta\in\operatorname{Im}\mathcal{T}_1\cap\operatorname{Im}\mathcal{T}_2$,则存在$\gamma,\delta\in X$使得$\mathcal{T}_1\gamma=\beta,\;\mathcal{T}_2\delta=\beta$,于是有$\beta=\mathcal{T}_1\gamma=\mathcal{T}_1^2\gamma=\mathcal{T}_1(\mathcal{T}_1\gamma)=\mathcal{T}_1\beta$。因为$\mathcal{T}_1$与$\mathcal{T}_2$正交,所以:
	\begin{equation*}
		\mathcal{T}_1\beta=\mathcal{T}_1(\mathcal{T}_2\delta)=(\mathcal{T}_1\mathcal{T}_2)\delta=\mathbf{0}
	\end{equation*}
	于是$\beta=\mathbf{0}$,即$\operatorname{Im}\mathcal{T}_1\cap\operatorname{Im}\mathcal{T}_2=\mathbf{0}$。由\cref{theo:DirectSum}可得$X=\operatorname{Im}\mathcal{T}_1\oplus\operatorname{Im}\mathcal{T}_2$。\par
	任取$\varepsilon=\mathcal{T}_1\varepsilon_1+\mathcal{T}_2\varepsilon_2\in X$,其中$\varepsilon_1,\varepsilon_2\in X$。因为$\mathcal{T}_1$与$\mathcal{T}_2$正交、$\mathcal{T}_1$是幂等变换,所以显然有:
	\begin{equation*}
		\mathcal{T}_1\varepsilon=\mathcal{T}_1(\mathcal{T}_1\varepsilon_1+\mathcal{T}_2\varepsilon_2)=\mathcal{T}_1^2\varepsilon_1+(\mathcal{T}_1\mathcal{T}_2)\varepsilon_2=\mathcal{T}_1\varepsilon_1
	\end{equation*}
	于是$\mathcal{T}_1$是平行于$\operatorname{Im}\mathcal{T}_2$在$\operatorname{Im}\mathcal{T}_1$上的投影。$\mathcal{T}_2$同理。
\end{proof}
\begin{corollary}
	设$X$是域$F$上的线性空间,由\cref{prop:ProjectionTransformation}(3)可得到如下推论:
	\begin{enumerate}
		\item 若$X=E\oplus W$,$\mathcal{P}_E$为平行于$W$在$E$上的投影变换,则:
		\begin{equation*}
			E=\operatorname{Im}\mathcal{P}_E,\;W=\operatorname{Ker}\mathcal{P}_E
		\end{equation*}
		\item $X$的任一子空间$E$是平行于$E$的一个补空间在$E$上的投影变换的象;
		\item $X$的任一子空间$E$是平行于$E$在$E$的一个补空间上的投影变换的核。
	\end{enumerate}
\end{corollary}
\begin{proof}
	(1)任取$\alpha=\alpha_1+\alpha_2\in X,\;\alpha_1\in E,\;\alpha_2\in W$,则$\mathcal{P}_E\alpha=\alpha_1\in E$,所以$\operatorname{Im}\mathcal{P}_E\subset E$。任取$\beta\in E$,有$\mathcal{P}_E\beta=\beta\in\operatorname{Im}\mathcal{P}_E$,所以$E\subset\operatorname{Im}\mathcal{P}_E$。因此$E=\operatorname{Im}\mathcal{P}_E$。\par
	任取$\gamma\in W$,有$\mathcal{P}_E\gamma=\mathbf{0}$,所以$\gamma\in\operatorname{Ker}\mathcal{P}_E$,于是$W\subset\operatorname{Ker}\mathcal{P}_E$。任取$\delta=\delta_1+\delta_2\in\operatorname{Ker}\mathcal{P}_E,\;\delta_1\in E,\;\delta_2\in W$,则$\mathcal{P}_E\delta=\delta_1=\mathbf{0}$,所以$\delta=\delta_2\in W$,于是$\operatorname{Ker}\mathcal{P}_E\subset W$。因此$W=\operatorname{Ker}\mathcal{P}_E$。\par
	(2)由\cref{theo:ExistenceOfComplement}可知$E$必定存在一个补空间$W$,于是$X=E\oplus W$,由(1)即可得到$E=\operatorname{Im}\mathcal{P}_E$。\par
	(3)与(2)类似可得。
	
\end{proof}