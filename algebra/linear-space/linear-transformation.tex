\section{线性变换}
\begin{definition}
	设$X$是域$F$上的线性空间,$\mathcal{T}\in\operatorname{Hom}(X)$,定义$\mathcal{T}$的正整数指数幂如下:
	\begin{equation*}
		\mathcal{T}^n=\underbrace{\mathcal{T} \cdot \mathcal{T} \cdots \mathcal{T}}_{n\text{个}\mathcal{T}}
		,\;n\in \mathbb{N}^+
	\end{equation*}
	若$\mathcal{T}$可逆,还可以定义$\mathcal{T}$的负整数指数幂如下:
	\begin{equation*}
		\mathcal{T}^{-n}=(\mathcal{T}^{-1})^n,\;n\in\mathbb{N}^+
	\end{equation*}
\end{definition}
\begin{definition}
	设$X$是域$F$上的线性空间,$\mathcal{T}\in\operatorname{Hom}(X)$。
	\begin{enumerate}
		\item 若对任意的$\alpha\in X$,有$\mathcal{T}\alpha=\alpha$,则称$\mathcal{T}$为$X$到$Y$的\gls{IdentityTransformation},记作$\mathcal{I}$;
		\item 给定$k\in F$,若对任意的$\alpha\in X$,有$\mathcal{T}\alpha=k\alpha$,则称$\mathcal{T}$为$X$上的\gls{ScalarTransformation},记作$\mathcal{K}$;
		\item 若$E$和$W$是$X$的子空间,且有$X=E\oplus W$,若对任意的$\alpha=\alpha_1+\alpha_2\in X,\;\alpha_1\in E,\;\alpha_2\in W$,有:
		\begin{equation*}
			\mathcal{T}\alpha=\alpha_1
		\end{equation*}
		则称$\mathcal{T}$为平行于$W$在$E$上的\gls{ProjectionTransformation},记为$\mathcal{P}_E$。
		\item 若$\mathcal{T}^2=\mathcal{I}$,则称$\mathcal{T}$为\gls{Involution};
		\item 若$\mathcal{T}^2=\mathcal{T}$,则称$\mathcal{T}$为\gls{IdempotentTransformation};
		\item 若存在$n\in\mathbb{N}^+$使得$\mathcal{T}^n=\mathcal{O}$,则称$\mathcal{T}$为\gls{NilpotentTransformation},使得$\mathcal{T}^n=\mathcal{O}$成立的最小正整数$n$被称为是$\mathcal{T}$的\gls{NilpotentIndex}。
	\end{enumerate}
\end{definition}
\begin{definition}
	设$X$是域$F$上的线性空间,$\mathcal{T}_1,\mathcal{T}_2\in \operatorname{Hom}(X)$。若$\mathcal{T}_1\mathcal{T}_2=\mathcal{T}_2\mathcal{T}_1=\mathcal{O}$,则称$\mathcal{T}_1$和$\mathcal{T}_2$\textbf{正交}。
\end{definition}

\subsection{线性变换的性质}
\subsubsection{线性变换保持的矩阵运算}
\begin{theorem}
	线性变换对其对应的矩阵保持加法、纯量乘法和乘法的运算。
\end{theorem}
\begin{proof}
	设$X$分别为域$F$上的$n$维线性空间,$\sigma:\mathcal{T}\in\operatorname{Hom}(X)\longrightarrow\mathcal{T}$在$X$的一组基$\seq{\alpha}{m}$下的矩阵。加法和纯量乘法的结论由\cref{theo:LinearTransformationMatrix}可得。取$\mathcal{T}_1,\mathcal{T}_2\in\operatorname{Hom}(X),\;\sigma(\mathcal{T}_1)=A,\;\sigma(\mathcal{T}_2)=B$,于是有:
	\begin{align*}
		&[\seq{(\mathcal{T}_1\mathcal{T}_2)\alpha}{n}]
		=[\mathcal{T}_1(\mathcal{T}_2\alpha_1),\mathcal{T}_1(\mathcal{T}_2\alpha_2),\dots,\mathcal{T}_1(\mathcal{T}_2\alpha_n)] \\
		=&\mathcal{T}_1(\seq{\mathcal{T}_2\alpha}{n}) =\mathcal{T}_1[(\seq{\alpha}{n})B] =(\seq{\mathcal{T}_1\alpha}{n})B \\
		&=(\seq{\alpha}{n})AB
	\end{align*}
	于是$\sigma(\mathcal{T}_1\mathcal{T}_2)=AB$,即线性变换对其对应的矩阵保持乘法运算。
\end{proof}
\subsubsection{线性变换在不同基下的矩阵的关系}
\begin{theorem}
	设$X$是域$F$上的$n$维线性空间,$\seq{\alpha}{n}$和$\seq{\beta}{n}$是$X$的两组基,$P$为$\seq{\alpha}{n}$到$\seq{\beta}{n}$的过渡矩阵,$\mathcal{T}\in\operatorname{Hom}(X)$,则$\mathcal{T}$在$\seq{\alpha}{n}$下的矩阵$A$与在$\seq{\beta}{n}$下的矩阵$B$满足:
	\begin{equation*}
		B=P^{-1}AP
	\end{equation*}
\end{theorem}
\begin{proof}
	由已知条件可得:
	\begin{gather*}
		(\seq{\mathcal{T}\alpha}{n})=(\seq{\alpha}{n})A \\
		(\seq{\mathcal{T}\beta}{n})=(\seq{\beta}{n})B \\
		(\seq{\beta}{n})=(\seq{\alpha}{n})P
	\end{gather*}
	设$P$的列向量为$\seq{P}{n}$,所以:
	\begin{align*}
		&(\seq{\mathcal{T}\beta}{n}) \\
		=&\{\mathcal{T}[(\seq{\alpha}{n})P_1],\mathcal{T}[(\seq{\alpha}{n})P_2],\dots,\mathcal{T}[(\seq{\alpha}{n})P_n]\} \\
		=&\{[\mathcal{T}(\seq{\alpha}{n})]P_1,[\mathcal{T}(\seq{\alpha}{n})]P_2,\dots,[\mathcal{T}(\seq{\alpha}{n})]P_n\} \\
		=&(\seq{\mathcal{T}\alpha}{n})P 
		=(\seq{\alpha}{n})AP
		=(\seq{\beta}{n})P^{-1}AP
	\end{align*}
	由\cref{theo:LinearTransformationMatrix}可知$\mathcal{T}$在$\seq{\beta}{n}$下的矩阵是唯一的,所以$B=P^{-1}AP$。
\end{proof}
\begin{theorem}
	设$X$是域$F$上的一个$n$维线性空间,$\mathcal{T}\in\operatorname{Hom}(X)$在$X$的一组基$\seq{\alpha}{n}$下的矩阵为$A$,则$\operatorname{rank}(A)=\operatorname{rank}(\mathcal{T})$。
\end{theorem}
\begin{proof}
	由\cref{prop:LinearMapping}(12)可得:
	\begin{align*}
		\operatorname{rank}(\mathcal{T})&=\dim(\operatorname{Im}\mathcal{T})=\dim(\mathcal{T}X)=\dim(\mathcal{T}<\seq{\alpha}{n}>) \\
		&=\dim<\seq{\mathcal{T}\alpha}{n}>
	\end{align*}
	因为$\dim(X)=n$,由\cref{theo:IsomorphicDim}可知$X\cong F^n$,映射:
	\begin{equation*}
		\sigma:\sum_{i=1}^{n}b_i\beta_i\longrightarrow(\seq{b}{n})^T
	\end{equation*}
	是$X$到$F^n$的一个同构映射。注意到:
	\begin{equation*}
		(\seq{\mathcal{T}\alpha}{n})=(\seq{\alpha}{n})A
	\end{equation*}
	$T\alpha_i$在基$\seq{\alpha}{n}$下的坐标就是$A$的第$i$列,所以$\sigma(\mathcal{T}\alpha_i)=A_i$,其中$A_i$表示$A$的第$i$列,于是由\cref{prop:LinearMapping}(12)可得$\sigma<\seq{\mathcal{T}\alpha}{n}>=<\seq{A}{n}>$,所以由\cref{prop:IsomorphicOfLinearSpace}(7)可得:
	\begin{align*}
		\operatorname{rank}(\mathcal{T})&=\dim<\seq{\mathcal{T}\alpha}{n}>=\dim(\sigma<\seq{\mathcal{T}\alpha}{n}>) \\
		&=\dim<\seq{A}{n}>=\operatorname{rank}(A)\qedhere
	\end{align*}
\end{proof}
由上述两个定理,我们可以给出如下定义:
\begin{definition}
	设$X$是域$F$上的$n$维线性空间,$\seq{\alpha}{n}$是$X$的一组基,$\mathcal{T}\in\operatorname{Hom}(X)$,$A$是$\mathcal{T}$在$\seq{\alpha}{n}$下的矩阵$A$。定义:
	\begin{gather*}
		\operatorname{rank}(\mathcal{T})=\operatorname{rank}(A),\quad\det(\mathcal{T})=\det(A),\quad\operatorname{tr}(\mathcal{T})=\operatorname{tr}(A)
	\end{gather*}
	称$A$的特征多项式为$\mathcal{T}$的特征多项式。
\end{definition}

\subsection{线性变换的特征值与特征向量}
\begin{definition}
	设$X$是域$F$上的一个线性空间,$\mathcal{T}\in\operatorname{Hom}(X)$。若$X$中存在非零向量$\xi$,$F$中存在元素$\lambda$,使得:
	\begin{equation*}
		\mathcal{T}\xi=\lambda\xi
	\end{equation*}
	则称$\lambda$是$\mathcal{T}$的一个特征值,$\xi$是$\mathcal{T}$属于特征值$\lambda$的一个特征向量。
\end{definition}
\begin{theorem}\label{theo:LinearlyTransformationEigen}
	设$X$是域$F$上的一个$n$维线性空间,$\seq{\alpha}{n}$是$X$的一组基,$\mathcal{T}\in\operatorname{Hom}(X)$,$A$是$\mathcal{T}$在$\seq{\alpha}{n}$下的矩阵,则:
	\begin{enumerate}
		\item $\lambda$是$\mathcal{T}$的特征值$\iff\lambda$是$A$的特征值;
		\item $\xi$是$\mathcal{T}$属于$\lambda$的特征向量$\iff \xi$在$\seq{\alpha}{n}$下的坐标是$A$属于$\lambda$的特征向量。
	\end{enumerate}
\end{theorem}
\begin{proof}
	设$\xi$在$\seq{\alpha}{n}$下的坐标为$x$,由\cref{theo:LinearMappingCoordinate}可知若$\mathcal{T}\xi=\lambda\xi$则有$Ax=\lambda x$。当$Ax=\lambda x$时,$\mathcal{T}\xi$与$\lambda\xi$坐标相同,即二者相等。于是有
	\begin{equation*}
		\mathcal{T}\xi=\lambda\xi\iff Ax=\lambda x\qedhere
	\end{equation*}
\end{proof}
\begin{property}\label{prop:EigenvectorLinearlyTransformation}
	设$X$是域$F$上的一个线性空间,$\mathcal{T}\in\operatorname{Hom}(X)$,$\lambda$是$\mathcal{T}$的一个特征值,则:
	\begin{enumerate}
		\item $\mathcal{T}$属于特征值$\lambda$的特征向量构成$X$的一个子空间,称该子空间为$\mathcal{T}$属于特征值$\lambda$的特征子空间,记为$X_{\lambda}$;
		\item $X_{\lambda}=\operatorname{Ker}(\lambda\mathcal{I}-\mathcal{T})$;
		\item 若$\dim(X)=n$,则$\dim(X_{\lambda})=n-\operatorname{rank}(\lambda\mathcal{I}-\mathcal{T})$;
		\item $\mathcal{T}$属于不同特征值的特征向量是线性无关的。
	\end{enumerate}
\end{property}
\begin{proof}
	(1)任取$\mathcal{T}$属于$\lambda$的两个特征向量$\xi_1,\xi_2$以及域$F$上的任意两个元素$k_1,k_2$,则:
	\begin{equation*}
		\mathcal{T}(k_1\xi_1+k_2\xi_2)=k_1\mathcal{T}\xi_1+k_2\mathcal{T}\xi_2=k_1\lambda\xi_1+k_2\lambda\xi_2=\lambda(k_1\xi_1+k_1\xi_2)
	\end{equation*}
	所以$k_1\xi_1+k_2\xi_2$也是$\mathcal{T}$属于$\lambda$的特征向量,由\cref{theo:Subspace}可知$\mathcal{T}$属于$\lambda$的特征向量构成$X$的一个子空间。\par
	(2)$\;\xi\in X_{\lambda}\iff\mathcal{T}\xi=\lambda\xi\iff(\mathcal{T}-\lambda\mathcal{I})\xi=\mathbf{0}\iff\xi\in\operatorname{Ker}(\lambda\mathcal{I}-\mathcal{T})$。\par
	(3)由\cref{theo:IsomorphicDim},设$\sigma$是$X$到$F^n$的同构映射,它将$X$中的元素映射到在基$\seq{\alpha}{n}$下的坐标,记$\mathcal{T}$在$\seq{\alpha}{n}$下的矩阵为$A$。由\cref{theo:LinearlyTransformationEigen}可知$\sigma(X_{\lambda})$等于齐次线性方程组$(\lambda I_n-A)x=\mathbf{0}$的解空间$W$,根据\cref{prop:HomogeneousSLESolution}(3)、\cref{prop:IsomorphicOfLinearSpace}(7)和\cref{theo:LinearTransformationMatrix}可得:
	\begin{equation*}
		\dim(X_{\lambda})=\dim[\sigma(X_\lambda)]=\dim(W)=n-\operatorname{rank}(\lambda I_n-A)=n-\operatorname{rank}(\lambda\mathcal{I}-\mathcal{T})
	\end{equation*}\par
	(4)设$\sigma$是$X$中的向量到它在基$\seq{\xi}{n}$下的坐标的同构映射。由\cref{prop:Eigenvector}(2)、\cref{theo:LinearlyTransformationEigen}和\cref{prop:IsomorphicOfLinearSpace}(4)立即可得。
\end{proof}
\begin{definition}
	设$X$是域$F$上的一个线性空间,$\mathcal{T}\in\operatorname{Hom}(X)$,$\lambda$是$\mathcal{T}$的一个特征值,称$\lambda$作为$\mathcal{T}$的特征多项式的根的重数称为$\lambda$的代数重数,将$X_{\lambda}$的维数称为$\lambda$的几何重数。
\end{definition}
\subsubsection{线性变换的对角化}
\begin{definition}
	设$X$是域$F$上的一个$n$维线性空间,$\mathcal{T}\in\operatorname{Hom}(X)$,若$X$中存在一组基使得$\mathcal{T}$在这组基下的矩阵是对角矩阵,则称$\mathcal{T}$可对角化。
\end{definition}
\begin{theorem}
	设$X$是域$F$上的一个$n$维线性空间,$\mathcal{T}\in\operatorname{Hom}(X)$,$\mathcal{T}$可对角化当且仅当:
	\begin{enumerate}
		\item $X$中存在一组基使得$\mathcal{T}$在这组基下的矩阵可对角化;
		\item $\mathcal{T}$有$n$个线性无关的特征向量$\seq{\xi}{n}$;
		\item $X$中存在由$\mathcal{T}$的特征向量构成的一组基;
		\item $\mathcal{T}$属于不同特征值的特征子空间的维数之和为$n$;
		\item $X$可表为$\mathcal{T}$属于不同特征值的特征子空间的直和;
		\item $\mathcal{T}$的特征多项式在$F[\lambda]$中可分解为:
		\begin{equation*}
			(\lambda-\lambda_1)^{r_1}(\lambda-\lambda_2)^{r_2}\cdots(\lambda-\lambda_m)^{r_m}
		\end{equation*}
		其中$\seq{\lambda}{m}$两两不等且$\mathcal{T}$的每个特征值$\lambda_i$的几何重数等于它的代数重数,$i=1,2,\dots,m$。
	\end{enumerate}
	此时$(\seq{\mathcal{T}\xi}{n})=(\seq{\xi}{n})\operatorname{diag}\{\seq{\lambda}{n}\}$,$\lambda_i$为$\xi_i$所属的特征值,$A=\operatorname{diag}\{\seq{\lambda}{n}\}$被称为$\mathcal{T}$的\textbf{标准形}。除了主对角线上元素的排列顺序外,$A$是由$\mathcal{T}$唯一决定的。
\end{theorem}
\begin{proof}
	(1)必要性显然。设$\mathcal{T}$在基$\seq{\alpha}{n}$下的矩阵$A$可对角化,即存在可逆矩阵$P$与对角矩阵$\varLambda$使得$A=P\varLambda P^{-1}$,设$P$的列为$\seq{P}{n}$,于是有:
	\begin{gather*}
		(\seq{\mathcal{T}\alpha}{n})=(\seq{\alpha}{n})A \\
		(\seq{\mathcal{T}\alpha}{n})=(\seq{\alpha}{n})P\varLambda P^{-1} \\
		(\seq{\mathcal{T}\alpha}{n})P=(\seq{\alpha}{n})P\varLambda \\
		\begin{aligned}
			&[\seq{(\seq{\mathcal{T}\alpha}{n})P}{n}] \\
			=&[\seq{(\seq{\alpha}{n})P}{n}]\varLambda
		\end{aligned}\\
		\begin{aligned}
			&\{\mathcal{T}[(\seq{\alpha}{n})P_1],\mathcal{T}[(\seq{\alpha}{n})P_2],\dots,\mathcal{T}[(\seq{\alpha}{n})P_n]\} \\
			=&[\seq{(\seq{\alpha}{n})P}{n}]\varLambda
		\end{aligned}
	\end{gather*}
	由\cref{theo:BasisTransInvertibleMat}可知$\seq{(\seq{\alpha}{n})P}{n}$还是$X$的一组基,于是$\mathcal{T}$在这组基下的矩阵是对角矩阵$\varLambda$,$\mathcal{T}$可对角化,充分性得证。\par
	(2)由$\mathcal{T}$可对角化的定义可得:
	\begin{align*}
		&\mathcal{T}\text{可对角化}\iff\mathcal{T}\text{在$X$的基}\seq{\xi}{n}\text{下的矩阵为}\operatorname{diag}\{\seq{\lambda}{n}\} \\
		\iff&(\seq{\mathcal{T}\xi}{n})=(\seq{\xi}{n})\operatorname{diag}\{\seq{\lambda}{n}\} \\
		\iff&(\seq{\mathcal{T}\xi}{n})=(\lambda_1\xi_1,\lambda_2\xi_2,\dots,\lambda_n\xi_n) \\
		\iff&\mathcal{T}\xi_i=\lambda_i\xi_i\iff\mathcal{T}\text{有$n$个线性无关的特征向量}\seq{\xi}{n}
	\end{align*}\par
	(3)由(2)和\cref{prop:nDimensionalLinearSpace}(2)立即可得。\par
	(4)由(1)可得$\mathcal{T}$可对角化当且仅当存在$X$的一组基$\seq{\xi}{n}$使得$\mathcal{T}$在这组基下的矩阵$A$可对角化,设$\sigma$是$X$中的向量到它在基$\seq{\xi}{n}$下的坐标的同构映射。由\cref{theo:DiagCondition}(2)可知$A$的特征子空间维数之和为$n$,根据\cref{prop:IsomorphicOfLinearSpace}(7)可知$\mathcal{T}$属于不同特征值的特征子空间的维数之和为$n$。\par
	(5)设$\mathcal{T}$全部的不同的特征值为$\seq{\lambda}{m}$。由(4)、\cref{prop:EigenvectorLinearlyTransformation}(4)和\cref{prop:nDimensionalLinearSpace}(2)可得:
	\begin{align*}
		\mathcal{T}\text{可对角化}
		\iff&\sum_{i=1}^{m}\dim(X_{\lambda_i})=n \\
		\iff&X_{\lambda_i},i=1,2,\dots,m\text{的基合起来是$n$个线性无关的向量} \\
		\iff&X_{\lambda_i},i=1,2,\dots,m\text{的基合起来是$X$的一组基} \\
		\iff&X=X_{\lambda_1}\oplus X_{\lambda_2}\oplus\cdots\oplus X_{\lambda_m}
	\end{align*}\par
	(6)由(1)和\cref{theo:DiagCondition}(3)立即可得。
\end{proof}

\subsection{不变子空间}
\begin{definition}
	设$X$是域$F$上的线性空间,$\mathcal{T}$是$X$上的一个线性变换,$E$是$X$的一个子空间。若对于任意的$\alpha\in E$都有$\mathcal{T}\alpha\in E$,则称$E$是$X$的一个\gls{InvariantSubspace},简称为$\mathcal{T}$-子空间。
\end{definition}
\begin{property}
	设$X$是域$F$上的线性空间,$\mathcal{T}$是$X$上的一个线性变换,则:
	\begin{enumerate}
		\item $X$、$X$的零子空间是$\mathcal{T}$-子空间,称二者为\textbf{平凡的}$\mathcal{T}$-子空间;
		\item $\operatorname{Ker}(\mathcal{T}),\operatorname{Im}(\mathcal{T})$和$\mathcal{T}$的特征子空间是$\mathcal{T}$-子空间;
		\item 
	\end{enumerate}
\end{property}
\begin{proof}
	(1)(2)显然
\end{proof}


\subsection{投影变换}
\begin{property}\label{prop:ProjectionTransformation}
	设$X$是域$F$上的线性空间,$E$和$W$是$X$的子空间,且有$X=E\oplus W$,$\mathcal{P}_E$是平行于$W$在$E$上的投影变换,$\mathcal{P}_W$是平行于$E$在$W$上的投影变换,则:
	\begin{enumerate}
		\item $\mathcal{P}_E$是线性变换;
		\item $\mathcal{P}_E$是幂等变换;
		\item 幂等变换$\mathcal{T}\in\operatorname{Hom}(X)$是平行于$\operatorname{Ker}\mathcal{T}$在$\operatorname{Im}\mathcal{T}$上的投影变换,此时$X=\operatorname{Im}\mathcal{T}\oplus\operatorname{Ker}\mathcal{T}$;
		\item $\mathcal{P}_E$与$\mathcal{P}_W$正交;
		\item $\mathcal{P}_E+\mathcal{P}_W=\mathcal{I}$;
		\item 若$\mathcal{T}_1,\mathcal{T}_2\in\operatorname{Hom}(X)$,$\mathcal{T}_1$和$\mathcal{T}_2$是正交的幂等变换,且$\mathcal{T}_1+\mathcal{T}_2=\mathcal{I}$,则$X=\operatorname{Im}\mathcal{T}_1\oplus\operatorname{Im}\mathcal{T}_2$,并且有$\mathcal{T}_1$是平行于$\operatorname{Im}\mathcal{T}_2$在$\operatorname{Im}\mathcal{T}_1$上的投影,$\mathcal{T}_2$是平行于$\operatorname{Im}\mathcal{T}_1$在$\operatorname{Im}\mathcal{T}_2$上的投影;
		\item 平行于$W$在$E$上的投影变换是唯一的。
	\end{enumerate}
\end{property}
\begin{proof}
	(1)任取$\alpha,\beta\in X$和$k_1,k_2\in F$,其中$\alpha=\alpha_1+\alpha_2,\;\beta=\beta_1+\beta_2,\;\alpha_1,\beta_1\in E,\;\alpha_2,\beta_2\in W$,于是有:
	\begin{align*}
		\mathcal{P}_E(k_1\alpha+k_2\beta)
		=\mathcal{P}_E[(k_1\alpha_1+k_2\beta_1)+(k_1\alpha_2+k_2\beta_2)]
		=k_1\alpha_1+k_2\beta_1
		=k_1\mathcal{P}_E\alpha+k_2\mathcal{P}_E\beta
	\end{align*}
	所以$\mathcal{P}_E$是线性变换。\par
	(2)任取$\alpha=\alpha_1+\alpha_2\in X,\;\alpha_1\in E,\;\alpha_2\in W$,则:
	\begin{equation*}
		\mathcal{P}_E(\mathcal{P}_E\alpha)=\mathcal{P}_E(\alpha_1)=\alpha_1=\mathcal{P}_E\alpha
	\end{equation*}
	所以$\mathcal{P}_E$是幂等变换。\par
	(3)任取$\alpha\in X$,则$\mathcal{T}\alpha\in\operatorname{Im}\mathcal{T}$。因为:
	\begin{equation*}
		\mathcal{T}(\alpha-\mathcal{T}\alpha)=\mathcal{T}\alpha-\mathcal{T}^2\alpha=\mathcal{T}\alpha-\mathcal{T}\alpha=\mathbf{0}
	\end{equation*}
	所以$\alpha-\mathcal{T}\alpha\in\operatorname{Ker}\mathcal{T}$。因为$\alpha=\mathcal{T}\alpha+\alpha-\mathcal{T}\alpha$,所以$X=\operatorname{Im}\mathcal{T}+\operatorname{Ker}\mathcal{T}$。\par
	任取$\beta\in\operatorname{Im}\mathcal{T}\cap\operatorname{Ker}\mathcal{T}$,则存在$\gamma\in X$使得$\mathcal{T}\gamma=\beta$,且有$\mathcal{T}\beta=\mathbf{0}$,于是:
	\begin{equation*}
		\mathbf{0}=\mathcal{T}\beta=\mathcal{T}(\mathcal{T}\gamma)=\mathcal{T}^2\gamma=\mathcal{T}\gamma=\beta
	\end{equation*}
	所以$\operatorname{Im}\mathcal{T}\cap\operatorname{Ker}\mathcal{T}=\mathbf{0}$,由\cref{theo:DirectSum}(3)可知$X=\operatorname{Im}\mathcal{T}\oplus\operatorname{Ker}\mathcal{T}$。\par
	记$\operatorname{Im}\mathcal{T}=E,\;\operatorname{Ker}\mathcal{T}=W$,对上述$\alpha=\mathcal{T}\alpha+\alpha-\mathcal{T}\alpha$,有$\mathcal{P}_E\alpha=\mathcal{T}\alpha$。由$\alpha$的任意性,$\mathcal{T}=\mathcal{P}_E$。\par
	(4)任取$\alpha=\alpha_1+\alpha_2\in X,\;\alpha_1\in E,\;\alpha_2\in W$,则:
	\begin{equation*}
		\mathcal{P}_E(\mathcal{P}_W\alpha)=\mathcal{P}_E\alpha_2=\mathbf{0},\;
		\mathcal{P}_W(\mathcal{P}_E\alpha)=\mathcal{P}_W\alpha_2=\mathbf{0}
	\end{equation*}
	所以$\mathcal{P}_E\mathcal{P}_W=\mathcal{P}_W\mathcal{P}_E=\mathcal{O}$,即$\mathcal{P}_E$与$\mathcal{P}_W$正交。\par
	(5)任取$\alpha=\alpha_1+\alpha_2\in X,\;\alpha_1\in E,\;\alpha_2\in W$,则:
	\begin{equation*}
		(\mathcal{P}_E+\mathcal{P}_W)\alpha=\mathcal{P}_E\alpha+\mathcal{P}_W\alpha=\alpha_1+\alpha_2=\alpha
	\end{equation*}
	于是$\mathcal{P}_E+\mathcal{P}_W=\mathcal{I}$。\par
	(6)任取$\alpha\in X$,则$\alpha=(\mathcal{T}_1+\mathcal{T}_2)\alpha=\mathcal{T}_1\alpha+\mathcal{T}_2\alpha$,所以$X=\operatorname{Im}\mathcal{T}_1+\operatorname{Im}\mathcal{T}_2$。\par
	任取$\beta\in\operatorname{Im}\mathcal{T}_1\cap\operatorname{Im}\mathcal{T}_2$,则存在$\gamma,\delta\in X$使得$\mathcal{T}_1\gamma=\beta,\;\mathcal{T}_2\delta=\beta$,于是有$\beta=\mathcal{T}_1\gamma=\mathcal{T}_1^2\gamma=\mathcal{T}_1(\mathcal{T}_1\gamma)=\mathcal{T}_1\beta$。因为$\mathcal{T}_1$与$\mathcal{T}_2$正交,所以:
	\begin{equation*}
		\mathcal{T}_1\beta=\mathcal{T}_1(\mathcal{T}_2\delta)=(\mathcal{T}_1\mathcal{T}_2)\delta=\mathbf{0}
	\end{equation*}
	于是$\beta=\mathbf{0}$,即$\operatorname{Im}\mathcal{T}_1\cap\operatorname{Im}\mathcal{T}_2=\mathbf{0}$。由\cref{theo:DirectSum}可得$X=\operatorname{Im}\mathcal{T}_1\oplus\operatorname{Im}\mathcal{T}_2$。\par
	任取$\varepsilon=\mathcal{T}_1\varepsilon_1+\mathcal{T}_2\varepsilon_2\in X$,其中$\varepsilon_1,\varepsilon_2\in X$。因为$\mathcal{T}_1$与$\mathcal{T}_2$正交、$\mathcal{T}_1$是幂等变换,所以显然有:
	\begin{equation*}
		\mathcal{T}_1\varepsilon=\mathcal{T}_1(\mathcal{T}_1\varepsilon_1+\mathcal{T}_2\varepsilon_2)=\mathcal{T}_1^2\varepsilon_1+(\mathcal{T}_1\mathcal{T}_2)\varepsilon_2=\mathcal{T}_1\varepsilon_1
	\end{equation*}
	于是$\mathcal{T}_1$是平行于$\operatorname{Im}\mathcal{T}_2$在$\operatorname{Im}\mathcal{T}_1$上的投影。$\mathcal{T}_2$同理。\par
	(7)若存在另一平行于$W$在$E$上的投影变换,则它与$\mathcal{P}_E$的作用完全相同,于是二者相等,唯一性得证。
\end{proof}
\begin{corollary}
	设$X$是域$F$上的线性空间,由\cref{prop:ProjectionTransformation}(3)可得到如下推论:
	\begin{enumerate}
		\item 若$X=E\oplus W$,$\mathcal{P}_E$为平行于$W$在$E$上的投影变换,则:
		\begin{equation*}
			E=\operatorname{Im}\mathcal{P}_E,\;W=\operatorname{Ker}\mathcal{P}_E
		\end{equation*}
		\item $X$的任一子空间$E$是平行于$E$的一个补空间在$E$上的投影变换的象;
		\item $X$的任一子空间$E$是平行于$E$在$E$的一个补空间上的投影变换的核。
	\end{enumerate}
\end{corollary}
\begin{proof}
	(1)任取$\alpha=\alpha_1+\alpha_2\in X,\;\alpha_1\in E,\;\alpha_2\in W$,则$\mathcal{P}_E\alpha=\alpha_1\in E$,所以$\operatorname{Im}\mathcal{P}_E\subset E$。任取$\beta\in E$,有$\mathcal{P}_E\beta=\beta\in\operatorname{Im}\mathcal{P}_E$,所以$E\subset\operatorname{Im}\mathcal{P}_E$。因此$E=\operatorname{Im}\mathcal{P}_E$。\par
	任取$\gamma\in W$,有$\mathcal{P}_E\gamma=\mathbf{0}$,所以$\gamma\in\operatorname{Ker}\mathcal{P}_E$,于是$W\subset\operatorname{Ker}\mathcal{P}_E$。任取$\delta=\delta_1+\delta_2\in\operatorname{Ker}\mathcal{P}_E,\;\delta_1\in E,\;\delta_2\in W$,则$\mathcal{P}_E\delta=\delta_1=\mathbf{0}$,所以$\delta=\delta_2\in W$,于是$\operatorname{Ker}\mathcal{P}_E\subset W$。因此$W=\operatorname{Ker}\mathcal{P}_E$。\par
	(2)由\cref{theo:ExistenceOfComplement}可知$E$必定存在一个补空间$W$,于是$X=E\oplus W$,由(1)即可得到$E=\operatorname{Im}\mathcal{P}_E$。\par
	(3)与(2)类似可得。
\end{proof}