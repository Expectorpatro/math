\section{线性空间}

\subsection{线性空间的概念与基本性质}

\begin{definition}
	设$S$是一个非空集合,$S\times S$是$S$与自身的一个Cartesian product(定义可见\cref{CartesianProduct}),则$f:S\times S\rightarrow S$称为$S$上一个\gls{BinaryAlgebraicOperation},简称为$S$上的一个运算。
\end{definition}
\subsubsection{线性空间的定义}
\begin{definition}
	设$X$是一个非空集合,$F$是一个域\info{域的概念}。如果$X$上有一个运算,即$f:(\alpha,\beta)\rightarrow\gamma(\alpha,\beta,\gamma\in X)$,将该运算称为\gls{Addition},把$\gamma$称为$\alpha$与$\beta$的\gls{Sum},记作$\alpha+\beta=\gamma$;同时$F$与$X$有一个运算,即$g:(k,\alpha)\rightarrow\delta(k\in F,\;\alpha,\delta\in X)$,将该运算称为\gls{ScalarMultiplication},把$\delta$称为$k$与$\alpha$的\gls{ScalarMultiple},记作$k\alpha=\delta$。若上述两个运算还满足以下$8$条运算法则:
	\begin{enumerate}
		\item $\forall\;\alpha,\beta\in X,\;\alpha+\beta=\beta+\alpha$;
		\item $\forall\;\alpha,\beta,\gamma\in X,\;(\alpha+\beta)+\gamma=\alpha+(\beta+\gamma)$;
		\item $X$中有一个元素,记作$\mathbf{0}$,称为$X$的\gls{ZeroVector},它使得:
		\begin{equation*}
			\forall\;\alpha\in X,\;\alpha+\mathbf{0}=\alpha
		\end{equation*}
		\item 对于任意的$\alpha\in X$,存在与之对应的$\beta\in X$,称为$\alpha$的\gls{AdditiveInverse},记作$-\alpha$,它使得:
		\begin{equation*}
			\alpha+\beta=\mathbf{0}
		\end{equation*}
		\item $F$中的\gls{MultiplicativeIdentity}$1$满足$\forall\;\alpha\in X,\;1\alpha=\alpha$;
		\item $\forall\;\alpha\in X,\;k,l\in F,\;(kl)\alpha=k(l\alpha)$;
		\item $\forall\;\alpha\in X,\;k,l\in F,\;(k+l)\alpha=k\alpha+l\alpha$;
		\item $\forall\;\alpha,\beta\in X,\;k\in F,\;k(\alpha+\beta)=k\alpha+k\beta$。
	\end{enumerate}
	那么称$X$是域$F$上的一个\gls{LinearSpace},称$X$中的元素为\gls{Vector}。
\end{definition}
\subsubsection{线性空间的基本性质}
\begin{property}\label{prop:LinearSpace}
	域$F$上的线性空间$X$具有如下性质:
	\begin{enumerate}
		\item $X$中的零元是唯一的;
		\item $X$中每个元素的负元是唯一的;
		\item $\forall\;\alpha\in X,\;0\alpha=\mathbf{0}$;
		\item $\forall\;k\in F,\;k\mathbf{0}=\mathbf{0}$;
		\item 设$k\in F,\;\alpha\in X$。如果$k\alpha=\mathbf{0}$,那么$k=0$或$\alpha=\mathbf{0}$。
		\item $\forall\;\alpha\in X,\;(-1)\alpha=-\alpha$;
	\end{enumerate}
\end{property}
\begin{proof}
	(1)假设$X$中有两个零元$\mathbf{0}_1,\mathbf{0}_2$且$\mathbf{0}_1\ne\mathbf{0}_2$,由线性空间运算法则(3)可得:
	\begin{equation*}
		\mathbf{0}_1+\mathbf{0}_2=\mathbf{0}_1,\;\mathbf{0}_2+\mathbf{0}_1=\mathbf{0}_2
	\end{equation*}
	而由线性空间运算法则(1)可得:
	\begin{equation*}
		\mathbf{0}_1+\mathbf{0}_2=\mathbf{0}_2+\mathbf{0}_1
	\end{equation*}
	于是$\mathbf{0}_1=\mathbf{0}_2$,产生矛盾,所以$X$中的零元是唯一的。\par
	(2)任取$X$中的一个元素$\alpha$,假设它有两个负元$\beta_1,\beta_2$。由线性空间运算法则(4)(3)(2)可得:
	\begin{gather*}
		(\beta_1+\alpha)+\beta_2=\mathbf{0}+\beta_2=\beta_2 \\
		(\beta_1+\alpha)+\beta_2=\beta_1+(\alpha+\beta_2)=\beta_1+\mathbf{0}=\beta_1
	\end{gather*}
	所以$\beta_1=\beta_2$,产生矛盾。由$\alpha$的任意性,$X$中每个元素的负元都是唯一的。\par
	(3)由线性空间运算法则(7)可得:
	\begin{equation*}
		0\alpha+0\alpha=(0+0)\alpha=0\alpha
	\end{equation*}
	两边同时加上$-0\alpha$可得:
	\begin{equation*}
		0\alpha+0\alpha+(-0\alpha)=0\alpha+(-0\alpha)
	\end{equation*}
	由线性空间运算法则(2)(4)和(3)可得:
	\begin{equation*}
		0\alpha=\mathbf{0}
	\end{equation*}\par
	(4)由线性空间运算法则(8)和(3)可得:
	\begin{equation*}
		k\mathbf{0}+k\mathbf{0}=k(\mathbf{0}+\mathbf{0})=k\mathbf{0}
	\end{equation*}
	两边加上$-k\mathbf{0}$再由线性空间运算法则(2)(4)和(3)可得:
	\begin{equation*}
		k\mathbf{0}=\mathbf{0}
	\end{equation*}\par
	(5)如果$k\ne0$,依次由线性空间运算法则(5)、(6)和线性空间基本性质(4)可得:
	\begin{equation*}
		\alpha=1\alpha=(k^{-1}k)\alpha=k^{-1}(k\alpha)=k^{-1}\mathbf{0}=\mathbf{0}
	\end{equation*}\par
	(6)由线性空间运算法则(5)与(7)以及线性空间基本性质(3)可得:
	\begin{equation*}
		\alpha+(-1)\alpha=1\alpha+(-1)\alpha=(1-1)\alpha=0\alpha=\mathbf{0}
	\end{equation*}
	再由负元的定义,$(-1)\alpha=-\alpha$。
\end{proof}
\begin{definition}
	设$X$是域$F$上的线性空间。由\cref{prop:LinearSpace}(2),定义$f:(\alpha,\beta)\rightarrow\alpha+(-\beta)\in X(\alpha,\beta\in X)$,将该运算称为\gls{Subtraction},把$\alpha+(-\beta)$称为$\alpha$与$\beta$的\gls{Difference},记作$\alpha-\beta=\alpha+(-\beta)$。
\end{definition}

\subsection{线性相关与线性无关}
\begin{definition}
	$X$是域$F$上的线性空间。按照一定顺序写出的有限多个向量(允许有相同的向量)称为$X$的一个\gls{SetOfVectors},如$\seq{\alpha}{n}$。
\end{definition}
\begin{definition}
	$X$是域$F$上的线性空间。对于$X$中的一组向量$\seq{\alpha}{n}$和$F$中的一组元素$\seq{k}{n}$,作纯量乘法和加法得到:
	\begin{equation*}
		\sum_{i=1}^{n}k_i\alpha_i\in X
	\end{equation*}
	称该向量为$\seq{\alpha}{n}$的一个\gls{LinearCombination}。
\end{definition}
\begin{definition}
	$X$是域$F$上的线性空间。若$\beta\in X$可以表示成向量组$\seq{\alpha}{n}$的一个线性组合,则称$\beta$可以由$\seq{\alpha}{n}$\textbf{线性表出}。若$\beta\in X$可以由向量集$W\subseteq X$中有限多个向量构成的向量组线性表出,则称$\beta$可以由向量集$W$线性表出。
\end{definition}
\begin{definition}
	$X$是域$F$上的线性空间。按照如下方式定义$X$中对象的\gls{LinearlyDependent}与\gls{LinearlyIndependent}:
	\begin{table}[H]
		\centering
		\begin{tabular}{>{\centering\arraybackslash}p{4cm}|>{\centering\arraybackslash}p{5cm}|>{\centering\arraybackslash}p{6cm}}
			\toprule
			\textbf{研究对象} & \textbf{线性相关} & \textbf{线性无关} \\
			\midrule
			$X$中的向量组$\seq{\alpha}{n}$ & $F$中有不全为$0$的元素$\seq{k}{n}$使得$\sum\limits_{i=1}^{n}k_i\alpha_i=\mathbf{0}$ & 从$\sum\limits_{i=1}^{n}k_i\alpha_i=\mathbf{0}$可以推出$k_1=k_2=\cdots=k_n=0$ \\
			\hline
			空集 & & 定义空集是线性无关的\\
			\hline
			$X$的非空有限子集 & 给这个子集的元素一种编号所得的向量组线性相关 & 给这个子集的元素一种编号所得的向量组线性无关 \\
			\hline
			$X$的无限子集$W$ & $W$有一个有限子集线性相关 & $W$的任一有限子集都线性无关 \\
			\bottomrule
		\end{tabular}
	\end{table}
\end{definition}
\begin{property}\label{prop:LinearlyDependent}
	设$X$是域$F$上的线性空间,则:
	\begin{enumerate}
		\item 如果$X$中向量组的一个部分组线性相关,那么这个向量组线性相关;
		\item 包含$\mathbf{0}$的向量组是线性相关的;
		\item 元素个数大于$1$的向量集$W$线性相关当且仅当$W$中至少有一个向量可以由其余向量中的有限多个线性表出,从而$W$线性无关当且仅当$W$中的每一个向量都不能由其余向量中的有限多个线性表出。
	\end{enumerate}
\end{property}
\begin{proof}
	(1)取$X$中的向量组$\seq{\alpha}{n}$,其部分组$\alpha_{i_1},\alpha_{i_2},\dots,\alpha_{i_m},\;i_m,m\leqslant n$线性相关,即$F$中存在不全为$0$的一组元素$k_{i_1},k_{i_2},\dots,k_{i_m}$使得:
	\begin{equation*}
		k_{i_1}\alpha_{i_1}+k_{i_2}\alpha_{i_2}+\cdots+k_{i_m}a_{i_m}=\mathbf{0}
	\end{equation*}
	在:
	\begin{equation*}
		l_1\alpha_1+l_2\alpha_2+\cdots+l_n\alpha_n
	\end{equation*}
	中取$l_{i_j}=k_{i_j},\;j=1,2,\dots,m$,其余系数为$0$,则$l_1,l_2,\dots,l_n$不全为$0$同时上式值为$\mathbf{0}$,即向量组$\seq{\alpha}{n}$线性相关。\par
	(2)由\cref{prop:LinearSpace}(4)可得$\mathbf{0}$作为向量组的部分组是线性相关的,由(1)即可得出结果。\par
	(3)由定义直接得到。
\end{proof}
\begin{property}\label{prop:LinearlyRepresentation}
	设$X$是域$F$上的线性空间,则:
	\begin{enumerate}
		\item 向量$\beta\in X$可以由向量集$W\subseteq X$线性表出,则表示方法唯一的充分必要条件是$W$线性无关;
		\item 若向量组$\seq{\alpha}{n}$线性无关,则向量$\beta$可以由向量组$\seq{\alpha}{n}$线性表出的充分必要条件为$\seq{\alpha}{n},\beta$线性相关。
	\end{enumerate}
\end{property}
\begin{proof}
	(1)\textbf{充分性:}因为$\beta$可以由向量集$W$线性表出,假设此时$\beta$有以下两种表出方式:
	\begin{gather*}
		\beta=k_1\alpha_1+\cdots+k_r\alpha_r+k_{r+1}u_1+\cdots+k_{r+s}u_s \\
		\beta=l_1\alpha_1+\cdots+l_r\alpha_r+l_{r+1}v_1+\cdots+l_{r+t}v_t
	\end{gather*}
	其中$\seq{\alpha}{r},\seq{u}{s},\seq{v}{t}\in W,\;k_1,\dots,k_{r+s},l_1,\dots.l_{r+t}\in F,\;r,s,t\geqslant0$。二式作差可得:
	\begin{equation*}
		\mathbf{0}=(k_1-l_1)\alpha_1+\cdots+(k_r-l_r)\alpha_r+k_{r+1}u_1+\cdots+k_{r+s}u_s-l_{r+1}v_1-\cdots-l_{r+t}v_t
	\end{equation*}
	因为$W$线性无关,所以向量组$\seq{\alpha}{r},\seq{u}{s},\seq{v}{t}$线性无关,于是:
	\begin{equation*}
		k_1-l_1=0,\cdots,k_r-l_r=0,k_{r+1}=0,\cdots,k_{r+s}=0,l_{r+1}=0,\dots,l_{r+t}=0
	\end{equation*}
	所以两个表出方式完全相同,$\beta$由$W$线性表出的表示方法唯一。\par
	\textbf{必要性:}如果$W$线性相关,则$W$有一个有限子集$\{\seq{\alpha}{n}\}$线性相关,于是$F$中有不全为$0$的元素$\seq{k}{n}$使得:
	\begin{equation*}
		k_1\alpha_1+k_2\alpha_2+\cdots+k_n\alpha_n=\mathbf{0}
	\end{equation*}
	由于$\beta$可以由$W$线性表出,所以:
	\begin{equation*}
		\beta=l_1\alpha_1+\cdots+l_n\alpha_n+l_{n+1}v_1+\cdots+l_{n+s}v_s
	\end{equation*}
	其中$l_i\in F,\;i=1,2,\dots,n;\;v_j\in X,\;j=1,2,\dots,s$。将上两式相加可得:
	\begin{equation*}
		\beta=(l_1+k_1)\alpha_1+\cdots+(l_n+k_n)\alpha_n+l_{n+1}v_1+\cdots+l_{n+s}v_s
	\end{equation*}
	因为$\seq{k}{n}$不全为$0$,所以有序元素组:
	\begin{equation*}
		(l_1,\dots,l_n,l_{n+1},\dots,l_{n+s})\ne(l_1+k_1,\dots,l_n+k_n,l_{n+1},\dots,l_{n+s})
	\end{equation*}
	于是$\beta$由$W$线性表出的方式不唯一,矛盾,所以$W$线性无关。\par
	(2)\textbf{充分性:}因为$\seq{\alpha}{n},\beta$线性相关,所以域$F$中存在不全为$0$的元素$\seq{k}{n},l$使得:
	\begin{equation*}
		k_1\alpha_1+k_2\alpha_2+\cdots+k_n\alpha_n+l\beta=\mathbf{0}
	\end{equation*}
	若$l=0$,则域$F$中存在不全为$0$的元素$\seq{k}{n}$使得:
	\begin{equation*}
		k_1\alpha_1+k_2\alpha_2+\cdots+k_n\alpha_n=\mathbf{0}
	\end{equation*}
	这与向量组$\seq{\alpha}{n}$线性无关矛盾,所以$l\ne0$。于是:
	\begin{equation*}
		\beta=-\frac{k_1}{l}\alpha_1-\frac{k_2}{l}\alpha_2-\cdots-\frac{k_n}{l}\alpha_n
	\end{equation*}
	即向量$\beta$可以由向量组$\seq{\alpha}{n}$线性表出。\par
	\textbf{必要性:}因为$\beta$可以由向量组$\seq{\alpha}{n}$线性表出,所以域$F$中存在不全为$0$的元素$\seq{k}{n},l$使得:
	\begin{equation*}
		\beta=k_1\alpha_1+k_2\alpha_2+\cdots+k_n\alpha_n
	\end{equation*}
	移项即可得到:
	\begin{equation*}
		k_1\alpha_1+k_2\alpha_2+\cdots+k_n\alpha_n-\beta=\mathbf{0}
	\end{equation*}
	因为$-1\ne0$,所以$\seq{\alpha}{n},\beta$线性相关。
\end{proof}
\subsubsection{极大线性无关组与秩}
\begin{definition}
	设$X$是域$F$上的一个线性空间。向量组$\seq{\alpha}{n}$中的一个部分组若满足下述条件:
	\begin{enumerate}
		\item 本身线性无关;
		\item 若该部分组不等于向量组,则从向量组的其余向量中任取一个向量添加进该部分组都将使部分组线性相关。若该部分组等于向量组,则跳过此条件。
	\end{enumerate}
	则称该部分组是向量组的一个\gls{MaximalLinearlyIndependentSystem}。
\end{definition}
\begin{definition}
	如果向量组$\seq{\alpha}{s}$的每一个向量都可以由向量组$\seq{\beta}{r}$线性表出,那么称向量组$\seq{\alpha}{s}$可以由向量组$\seq{\beta}{r}$线性表出。如果向量组$\seq{\alpha}{s}$和向量组$\seq{\beta}{r}$可以互相线性表出,则称两个向量组\gls{EquivalentVectorSet},记作:
	\begin{equation*}
		\{\seq{\alpha}{s}\}\cong\{\seq{\beta}{r}\}
	\end{equation*}
\end{definition}
\begin{theorem}
	$X$是域$K$上的一个线性空间。$X$中向量组的等价是$X$中向量组的一个等价关系。
\end{theorem}
\begin{proof}
	(1)反身性与(2)对称性显然成立。\par
	(3)传递性:若向量组$\seq{\alpha}{s}$可以由向量组$\seq{\beta}{r}$线性表出,且向量组$\seq{\beta}{r}$可以由向量组$\seq{\gamma}{t}$线性表出,于是有:
	\begin{equation*}
		\alpha_i=\sum_{j=1}^{r}k_{j_i}\beta_j,\;i=1,2,\dots,s\quad
		\beta_j=\sum_{l=1}^{t}q_{l_j}\gamma_l,\;j=1,2,\dots,r
	\end{equation*}
	于是:
	\begin{equation*}
		\alpha_i=\sum_{j=1}^{r}k_{j_i}\left(\sum_{l=1}^{t}q_{l_j}\gamma_l\right)=\sum_{j=1}^{r}\sum_{l=1}^{t}k_{j_i}q_{l_j}\gamma_l=\sum_{l=1}^{t}\left(\sum_{j=1}^{r}k_{j_i}q_{l_j}\right)\gamma_l,\;i=1,2,\dots,s
	\end{equation*}
	即向量组$\seq{\alpha}{s}$可以由向量组$\seq{\gamma}{t}$线性表出。传递性得证。
\end{proof}
\begin{theorem}\label{theo:LinearlyIndependentNum}
	设$X$是域$F$上的一个线性空间,则:
	\begin{enumerate}
		\item 若向量组$\seq{\beta}{r}$可以由向量组$\seq{\alpha}{s}$线性表出,同时$r>s$,那么向量组$\seq{\beta}{r}$线性相关;
		\item 若向量组$\seq{\beta}{r}$可以由向量组$\seq{\alpha}{s}$线性表出,同时向量组$\seq{\beta}{r}$线性无关,就有$r\leqslant s$;
		\item 等价的线性无关的向量组所含向量的个数相同。
	\end{enumerate}
\end{theorem}
\begin{proof}
	(1)考虑方程:
	\begin{equation*}
		k_1\beta_1+k_2\beta_2+\cdots+k_r\beta_r=\mathbf{0}
	\end{equation*}
	因为向量组$\seq{\beta}{r}$可以由向量组$\seq{\alpha}{s}$线性表出,所以:
	\begin{equation*}
		\begin{cases}
			\beta_1=a_{11}\alpha_1+a_{12}\alpha_2+\cdots+a_{1s}\alpha_s \\
			\beta_2=a_{21}\alpha_1+a_{22}\alpha_2+\cdots+a_{2s}\alpha_s \\
			\vdots \\
			\beta_r=a_{r1}\alpha_1+a_{r2}\alpha_2+\cdots+a_{rs}\alpha_s
		\end{cases}
	\end{equation*}
	则有:
	\begin{align*}
		k_1\beta_1+k_2\beta_2+\cdots+k_r\beta_r
		&=k_1(a_{11}\alpha_1+a_{12}\alpha_2+\cdots+a_{1s}\alpha_s) \\
		&\quad+k_2(a_{21}\alpha_1+a_{22}\alpha_2+\cdots+a_{2s}\alpha_s)+\cdots \\
		&\quad+k_r(a_{r1}\alpha_1+a_{r2}\alpha_2+\cdots+a_{rs}\alpha_s) \\
		&=(k_1a_{11}+k_2a_{21}+\cdots+k_ra_{r1})\alpha_1 \\
		&\quad+(k_1a_{12}+k_2a_{22}+\cdots+k_ra_{r2})\alpha_2+\cdots \\
		&\quad+(k_1a_{1s}+k_2a_{2s}+\cdots+k_ra_{rs})\alpha_s \\
		&=\mathbf{0}
	\end{align*}
	将上式看作$\seq{k}{r}$的线性方程,考虑如下齐次线性方程组:
	\begin{equation*}
		\begin{pmatrix}
			a_{11} & a_{12} & \cdots & a_{1s} \\
			a_{21} & a_{22} & \cdots & a_{2s} \\
			\vdots & \vdots & \ddots & \vdots \\
			a_{r1} & a_{r2} & \cdots & a_{rs}
		\end{pmatrix}
		\begin{pmatrix}
			k_1 \\
			k_2 \\
			\vdots \\
			k_r
		\end{pmatrix}
		=
		\begin{pmatrix}
			0 \\
			0 \\
			\vdots \\
			0
		\end{pmatrix}
	\end{equation*}
	因为$s<r$,由\cref{theo:SolutionOfSLE1}可知上述齐次线性方程组必有非零解。取它的一个非零解$(\seq{k}{r})$,由\cref{prop:LinearSpace}(3)和线性空间运算法则(3)即可得:
	\begin{equation*}
		k_1\beta_1+k_2\beta_2+\cdots+k_r\beta_r=0\alpha_1+0\alpha_2+\cdots+0\alpha_s=\mathbf{0}
	\end{equation*}
	于是向量组$\seq{\beta}{r}$线性相关。\par
	(2)是(1)的逆否命题。\par
	(3)可由(2)直接得到。
\end{proof}
\begin{property}\label{prop:MaximalLinearlyIndependentSystem}
	设$X$是域$F$上的一个线性空间,则:
	\begin{enumerate}
		\item 一个向量组与它的任意一个极大线性无关组等价;
		\item 一个向量组的任意两个极大线性无关组等价;
		\item 一个向量组的任意两个极大线性无关组所含向量的个数相同。
	\end{enumerate}
\end{property}
\begin{proof}
	(1)设$\seq{\alpha}{s}$是一个向量组,任取它的一个极大线性无关组$\alpha_{i_1},\alpha_{i_2},\dots,\alpha_{i_n}$。显然$\alpha_{i_1},\alpha_{i_2},\dots,\alpha_{i_n}$可以由$\seq{\alpha}{s}$线性表出。由极大线性无关组的定义与\cref{prop:LinearlyRepresentation}(2)的充分性可直接得到$\seq{\alpha}{s}$可以由$\alpha_{i_1},\alpha_{i_2},\dots,\alpha_{i_n}$线性表出。\par
	(2)由(1)以及向量组等价的对称性与传递性可直接推出。\par
	(3)由\cref{theo:LinearlyIndependentNum}(3)直接得到。
\end{proof}
\begin{definition}
	向量组的极大线性无关组所含向量的个数称为这个向量组的\gls{Rank}。把向量组$\seq{\alpha}{s}$的秩记作$\operatorname{rank}\{\seq{\alpha}{s}\}$。全由零向量组成的向量组的秩规定为$0$。
\end{definition}
\begin{property}\label{prop:Rank}
	设$X$是域$F$上的一个线性空间,则:
	\begin{enumerate}
		\item 向量组$\seq{\alpha}{s}$线性无关的充分必要条件是它的秩等于它所含向量的个数。
		\item 如果向量组$\seq{\alpha}{s}$可以由向量组$\seq{\beta}{r}$线性表出,那么:
		\begin{equation*}
			\operatorname{rank}\{\seq{\alpha}{s}\}\leqslant\operatorname{rank}\{\seq{\beta}{r}\}
		\end{equation*}
		\item 等价的向量组具有相等的秩。
	\end{enumerate}
\end{property}
\begin{proof}
	(1)$\;\seq{\alpha}{s}$线性无关$\iff$极大线性无关组就是自身$\iff$秩等于所含向量的个数。\par
	(2)任取向量组$\seq{\alpha}{s}$的一个极大线性无关组$\alpha_{i_1},\alpha_{i_2},\dots,\alpha_{i_n}$,再取向量组$\seq{\beta}{r}$的一个极大线性无关组$\beta_{j_1},\beta_{j_2},\dots,\beta_{j_m}$。因为向量组$\seq{\alpha}{s}$可以由向量组$\seq{\beta}{r}$线性表出,所以$\alpha_{i_1},\alpha_{i_2},\dots,\alpha_{i_n}$可以由$\beta_{j_1},\beta_{j_2},\dots,\beta_{j_m}$线性表出,因为$\alpha_{i_1},\alpha_{i_2},\dots,\alpha_{i_n}$线性无关,由\cref{theo:LinearlyIndependentNum}(2)可得,$n\leqslant m$,即$\operatorname{rank}\{\seq{\alpha}{s}\}\leqslant\operatorname{rank}\{\seq{\beta}{r}\}$。\par
	(3)由(2)可直接推得。
\end{proof}
\subsubsection{基与维数}
\begin{definition}
	设$X$是域$F$上的一个线性空间。$X$中的向量集$S$若满足下述条件:
	\begin{enumerate}
		\item 本身线性无关;
		\item $X$中的每一个向量都可以由$S$中有限多个向量线性表出。
	\end{enumerate}
	则称$S$是向量组的一个\gls{Basis}。
\end{definition}
\begin{theorem}\label{theo:ExistenceOfBasis}
	任一域$F$上的任一线性空间$X$都有一组基。
\end{theorem}
该定理的证明不提供,涉及Zorn引理。
\begin{definition}
	$X$是域$F$上的线性空间。如果$X$的一组基是由有限多个向量组成的,那么称$X$是\gls{FiniteDimensional};如果$X$有一组基含有无穷多个向量,则称$X$是\gls{InfiniteDimensional}。
\end{definition}
\begin{theorem}
	如果域$F$上的线性空间$X$是有限维的,那么$X$的任意两个基所含向量的个数相同。
\end{theorem}
\begin{proof}
	由有限维线性空间的定义,$X$存在一组基只有有限多个向量,记为$\seq{\alpha}{n}$。再取$X$的一组基$S$,从中取出$n+1$个向量$\seq{\beta}{n+1}$(考虑$S$可能含有无数个向量的情况)。因为$\seq{\alpha}{n}$是$X$的基,所以$\seq{\beta}{n+1}$可以由$\seq{\alpha}{n}$线性表出。因为$\seq{\alpha}{n}$线性无关,所以$\seq{\beta}{n+1}$线性相关,而此时$S$也应线性相关,矛盾,所以$S$中的元素小于$n+1$个。设$S=\{\seq{\beta}{m}\}$,则$S$与$\seq{\alpha}{n}$都线性无关且二者等价,由\cref{theo:LinearlyIndependentNum}(3),它们具有相同的秩,即所含向量个数相同。由$S$的任意性与向量组等价的传递性、对称性,$X$的任意两个基所含向量的个数相同。
\end{proof}
\begin{corollary}
	如果域$F$上的线性空间$X$是无限维的,那么$X$的任意一组基都含有无穷多个向量。
\end{corollary}
\begin{proof}
	如果$X$有一组基由有限多个向量组成,那么$X$也是一个有限维线性空间,而有限维线性空间所有基所含向量的个数都相同,那么$X$就不可能有一组基含有无穷多个向量,与无穷维线性空间的定义矛盾。
\end{proof}
\begin{definition}
	$X$是域$F$上的线性空间。如果$X$是有限维的,那么把$X$的基所含向量的个数称为$X$的\gls{Dimension},记作$\operatorname{dim}V$;如果$X$是无限维的,那么记$\dim V=+\infty$。
\end{definition}
\begin{property}\label{prop:nDimensionalLinearSpace}
	设$X$是域$F$上的$n$维线性空间,则:
	\begin{enumerate}
		\item $X$中任意$n+1$个向量都线性相关。
		\item 任意$n$个线性无关的向量都是$X$的一组基;
		\item 如果$X$中任一向量都可以由$\seq{\alpha}{n}$线性表出,则$\seq{\alpha}{n}$是$X$的一组基;
		\item $X$中任意一个线性无关的向量组都可以扩充成$X$的一组基。
	\end{enumerate}
\end{property}
\begin{proof}
	(1)任意$n+1$个向量都可以由一组基线性表出,而每一组基所含向量个数都是$n$,由\cref{theo:LinearlyIndependentNum}(1),这$n+1$个向量线性相关。\par
	(2)任取$X$中$n$个线性无关的向量$\seq{\alpha}{n}$,对任意的$\beta\in X$,由上一个定理,向量组$\seq{\alpha}{n},\beta$线性相关。由\cref{prop:LinearlyRepresentation}(2),$\beta$可以由$\seq{\alpha}{n}$线性表出。由基的定义,$\seq{\alpha}{n}$是$X$的一组基。由$\seq{\alpha}{n}$的任意性,命题成立。\par
	(3)取$X$的一组基$\seq{\beta}{n}$,则$\seq{\beta}{n}$可以由$\seq{\alpha}{n}$线性表出。由\cref{prop:Rank}(2),$n=\operatorname{rank}\{\seq{\beta}{n}\}\leqslant\operatorname{rank}\{\seq{\alpha}{n}\}\leqslant n$,所以$\operatorname{rank}\{\seq{\alpha}{n}\}=n$,$\seq{\alpha}{n}$线性无关。由(2),$\seq{\alpha}{n}$是$X$的一组基。\par
	(4)任取$X$中一个线性无关的向量组$\seq{\alpha}{r}$,若$r=n$,由(2)可知,$\seq{\alpha}{r}$是$X$的一组基;若$r<n$,则$X$中必定存在一个元素不能由$\seq{\alpha}{r}$线性表出,将其记为$\alpha_{r+1}$,否则的话$\seq{\alpha}{r}$就是$X$的一组基,进而$X$的维数应是$r$,矛盾。不断重复上述过程即可得到一个向量组$\seq{\alpha}{n}$,这就是$X$的一组基。
\end{proof}
\subsubsection{坐标与坐标变换}
\begin{definition}
	$X$是域$F$上的$n$维线性空间,$\seq{\alpha}{n}$是$X$的一组基,由\cref{prop:LinearlyRepresentation}(1)可知$X$中任一向量$\alpha$由$\seq{\alpha}{n}$线性表出的方式唯一:
	\begin{equation*}
		\alpha=a_1\alpha_1+a_2\alpha_2+\cdots+a_n\alpha_n
	\end{equation*}
	把系数构成的$n$元有序数组写成列向量的形式,得到$(\seq{a}{n})^T$,该列向量被称为$\alpha$在基$\seq{\alpha}{n}$下的\gls{Coordinate}。
\end{definition}
接下来我们要讨论的是$X$中某一向量在不同基下的坐标之间有什么关系,首先我们需要定义什么叫不同的基。
\begin{definition}
	$X$是域$F$上的$n$维线性空间。$X$中的两个向量组$\seq{\alpha}{n}$与$\seq{\beta}{n}$如果满足$\alpha_i=\beta_i,\;i=1,2,\dots,n$,那么称这两个向量组\textbf{相等}。
\end{definition}
\begin{definition}
	$X$是域$F$上的$n$维线性空间。给定$V$的两个基:
	\begin{equation*}
		\seq{\alpha}{n}\quad\seq{\beta}{n}
	\end{equation*}
	因为$\seq{\alpha}{n}$是$V$的一组基,所以有:
	\begin{equation*}
		\begin{cases}
			\beta_1=a_{11}\alpha_1+a_{12}\alpha_2+\cdots+a_{1n}\alpha_n \\
			\beta_2=a_{21}\alpha_1+a_{22}\alpha_2+\cdots+a_{2n}\alpha_n \\
			\cdots\cdots\cdots \\
			\beta_n=a_{n1}\alpha_1+a_{n2}\alpha_2+\cdots+a_{nn}\alpha_n
		\end{cases}
	\end{equation*}
	模仿矩阵乘法的定义将上式写作:
	\begin{equation*}
		(\seq{\beta}{n})=(\seq{\alpha}{n})
		\begin{pmatrix}
			a_{11} & a_{21} & \cdots & a_{n1} \\
			a_{12} & a_{22} & \cdots & a_{n2} \\
			\vdots & \vdots & \ddots & \vdots \\
			a_{1n} & a_{2n} & \cdots & a_{nn}
		\end{pmatrix}
	\end{equation*}
	将上式右端的矩阵记作$A$,称它是基$\seq{\alpha}{n}$到基$\seq{\beta}{n}$的\gls{TransitionMatrix}。于是上式可以写作:
	\begin{equation*}
		(\seq{\beta}{n})=(\seq{\alpha}{n})A
	\end{equation*}
	由于这种写法是模仿矩阵乘法的定义,所以矩阵乘法所满足的运算法则对于这种写法也成立。
\end{definition}
\begin{theorem}\label{theo:BasisTransInvertibleMat}
	$X$是域$F$上的$n$维线性空间,$\seq{\alpha}{n}$是$X$的一组基,且向量组$\seq{\beta}{n}$满足:
	\begin{equation*}
		(\seq{\beta}{n})=(\seq{\alpha}{n})A
	\end{equation*}
	则$\seq{\beta}{n}$是$X$的一组基当且仅当$A$是可逆矩阵。
\end{theorem}
\begin{proof}
	由\info{齐次方程组的解与逆矩阵}可得:
	\begin{align*}
		\seq{\beta}{n}\text{是$X$的一组基}
		&\iff\text{由}\sum_{i=1}^{n}k_i\beta_i=\mathbf{0}
		\text{可推出}k_1=k_2=\cdots=k_n=0 \\
		&\iff\text{由}(\seq{\alpha}{n})A
		\begin{pmatrix}
			k_1 \\
			k_2 \\
			\vdots \\
			k_n
		\end{pmatrix}=\mathbf{0}
		\text{可推出}k_1=k_2=\cdots=k_n=0 \\
		&\iff\text{由}A
		\begin{pmatrix}
			k_1 \\
			k_2 \\
			\vdots \\
			k_n
		\end{pmatrix}=\mathbf{0}
		\text{可推出}k_1=k_2=\cdots=k_n=0 \\
		&\iff\text{齐次线性方程组}Ax=\mathbf{0}\text{只有零解} \\
		&\iff A\text{可逆}\qedhere
	\end{align*}
\end{proof}
\begin{theorem}
	$X$是域$F$上的$n$维线性空间,$\seq{\alpha}{n}$和$\seq{\beta}{n}$是$X$上的两个基,$A$是由基$\seq{\alpha}{n}$到基$\seq{\beta}{n}$的过渡矩阵。若$X$中向量$\alpha$在这两个基下的坐标分别为:
	\begin{equation*}
		x=(\seq{x}{n})^T,\;y=(\seq{y}{n})^T
	\end{equation*}
	则有:
	\begin{equation*}
		y=A^{-1}x
	\end{equation*}
\end{theorem}
\begin{proof}
	因为:
	\begin{align*}
		\alpha
		&=(\seq{\alpha}{n})(\seq{x}{n})^T \\
		&=(\seq{\beta}{n})(\seq{y}{n})^T \\
		&=(\seq{\alpha}{n})A(\seq{y}{n})^T
	\end{align*}
	所以:
	\begin{equation*}
		(\seq{x}{n})^T=A(\seq{y}{n})^T
	\end{equation*}
	即$x=Ay$。因为$\seq{\alpha}{n}$和$\seq{\beta}{n}$是$X$上的两个基,所以$A$可逆,于是$y=A^{-1}x$。
\end{proof}

\subsection{子空间}
\subsubsection{子空间的定义及其判别}
\begin{definition}
	$X$是域$F$上的一个线性空间,$E$是$X$的一个非空子集。若$E$对于$X$上的加法和纯量乘法也构成域$F$上的一个线性空间,则称$E$是$X$的一个\gls{LinearSubspace},简称为\gls{Subspace}。
\end{definition}
\begin{theorem}\label{theo:Subspace}
	$X$是域$F$上的一个线性空间,$E$是$X$的一个非空子集。$E$是$X$的一个子空间的充分必要条件是$E$对$X$中的加法和纯量乘法封闭,即:
	\begin{gather*}
		\alpha,\beta\in E\Rightarrow\alpha+\beta\in E \\
		k\in F,\alpha\in E\Rightarrow k\alpha\in E
	\end{gather*}
\end{theorem}
\begin{proof}
	\textbf{(1)必要性:}因为$E$是$X$的子空间,由子空间的定义,$E$中的加法和纯量乘法就是$X$中的加法和纯量乘法,由加法和纯量乘法的定义,$E$对$X$中的加法和纯量乘法封闭。\par
	\textbf{(2)充分性:}由条件可知此时已经对$E$定义了加法和纯量乘法,且就是$X$中的加法和纯量乘法,还需要证明的是$E$对于它自身的加法和纯量乘法满足线性空间的八条运算法则。对$\forall\;\alpha,\beta,\gamma\in E,\;\forall\;k,l\in F$:
	\begin{enumerate}
		\item 因为$\alpha,\beta,\gamma\in E$,所以$\alpha,\beta,\gamma\in X$,对于$X$上的加法,有$\alpha+\beta=\beta+\alpha,\;(\alpha+\beta)+\gamma=\alpha+(\beta+\gamma)$,而$X$上的加法就是$E$上的加法,所以$E$上的加法满足线性空间运算法则(1)(2);同理,$E$上的纯量乘法满足线性空间运算法则(5)(6)(7)(8);
		\item 因为$E$不是空集,所以存在$\delta\in E$。因为$\delta\in X$,所以由$X$中的纯量乘法可得$0\delta=\mathbf{0}_X\in E$。对任意的$\alpha\in E$,有$\alpha\in X$,根据$X$中的加法有$\alpha+\mathbf{0}_X=\alpha$,于是$\mathbf{0}_X$是$E$中的零元,即$E$满足线性空间运算法则(3);
		\item 因为$E$对纯量乘法封闭,所以$(-1)\alpha\in E$。因为$\alpha+(-1)\alpha=[1+(-1)]\alpha=0\alpha=\mathbf{0}$(第一步到第二步由$1$,第三步到第四步因为$\alpha\in X$),所以$(-1)\alpha$是$\alpha$的负元,即$E$满足线性空间运算法则(4)。
	\end{enumerate}
\end{proof}
\subsubsection{子空间的性质}
\begin{theorem}\label{theo:DimSubspace}
	设$X$是域$F$上的一个线性空间,$E$是$X$的任意一个子空间,则有$\dim(E)\leqslant\dim(X)$。$X$是有限维时等号成立当且仅当$E=X$。
\end{theorem}
\begin{proof}
	由\cref{prop:nDimensionalLinearSpace}(4)可得,$E$的一组基可以扩充成$X$的一组基,所以$\dim(E)\leqslant\dim(X)$。当$E=X$时,显然$\dim(E)=\dim(X)$。当$\dim(E)=\dim(X)$时,由\cref{prop:nDimensionalLinearSpace}(2)可知$E$的一组基就是$X$的一组基,所以$X$中的任一向量可以由$E$的基线性表出,于是$X\subseteq E$,从而$E=X$。
\end{proof}
\subsubsection{张成的子空间}
\begin{definition}
	设$X$是域$F$上的一个线性空间,$\seq{\alpha}{n}\in X$,称:
	\begin{equation*}
		W=\left\{\sum_{i=1}^{n}k_i\alpha_i:k_i\in F\right\}
	\end{equation*}
	为向量组$\seq{\alpha}{n}$\textbf{张成的线性子空间},记作$<\seq{\alpha}{n}>$。
\end{definition}
\begin{property}\label{prop:SpanSubspace}
	设$X$是域$F$上的一个线性空间,$\seq{\alpha}{n}\in X$。$<\seq{\alpha}{n}>$具有如下性质:
	\begin{enumerate}
		\item $<\seq{\alpha}{n}>$是$X$的一个子空间;
		\item $<\seq{\alpha}{n}>$是$X$中包含$\seq{\alpha}{n}$的最小的子空间;
		\item $\seq{\alpha}{n}$的极大线性无关组是$<\seq{\alpha}{n}>$的基;
		\item $\operatorname{dim}<\seq{\alpha}{n}>=\operatorname{rank}\{\seq{\alpha}{n}\}$;
		\item 若$\seq{\beta}{m}\in X$,则有:
		\begin{equation*}
			<\seq{\alpha}{n}>=<\seq{\beta}{m}>\iff\{\seq{\alpha}{n}\}\cong\{\seq{\beta}{m}\}
		\end{equation*}
		\item 若$\seq{\alpha}{n}$是$X$的一组基,则$X=<\seq{\alpha}{n}>$。
	\end{enumerate}
\end{property}
\begin{proof}
	(1)显然$<\seq{\alpha}{n}>$中的元素对$X$中的加法与纯量乘法封闭。\par
	(2)由子空间对加法与纯量乘法的封闭性,包含$\seq{\alpha}{n}$的子空间必然包含$<\seq{\alpha}{n}>$。\par
	(3)显然。\par
	(4)由(3)直接得到。\par
	(5)充分性显然,必要性由反证法可得。\par
	(6)显然。
\end{proof}
\subsubsection{子空间的交与和}
\begin{theorem}\label{theo:CapSubspace}
	设$X$是域$F$上的一个线性空间,$I$是一个指标集,对任意的$i\in I$有$X_i$是$X$的子空间,则
	\begin{equation*}
		\underset{i\in I}{\overset{}{\cap}}X_i=\{\alpha:\alpha\in X_i,\;\forall\;i\in I\}
	\end{equation*}
	也是$X$的子空间。
\end{theorem}
\begin{proof}
	任取$\alpha,\beta\in\underset{i\in I}{\overset{}{\cap}}X_i$和$k_1,k_2\in F$。对任意的$i\in I$,因为$X_i$是$X$的子空间,$\alpha,\beta\in X_i$,所以$k_1\alpha+k_2\beta\in X_i$,于是$k_1\alpha+k_2\beta\in\underset{i\in I}{\overset{}{\cap}}X_i$。由\cref{theo:Subspace}可知$\underset{i\in I}{\overset{}{\cap}}X_i$是$X$的子空间。
\end{proof}
\begin{definition}
	设$X$是域$F$上的一个线性空间,$\seq{X}{n}$是$X$的子空间。定义:
	\begin{equation*}
		\sum_{i=1}^{n}X_i=\left\{\sum_{i=1}^{n}\alpha_i:\alpha_i\in X_i\right\}
	\end{equation*}
\end{definition}
\begin{theorem}\label{theo:DierectSumSubspace}
	设$X$是域$F$上的一个线性空间,$\seq{X}{n}$是$X$的子空间,$n\in\mathbb{N}^+$,则$X_1+X_2+\cdots+X_n$也是$X$的子空间。
\end{theorem}
\begin{proof}
	任取$\alpha,\beta\in X_1+X_2+\cdots+X_n$和$k_1,k_2\in F$,则:
	\begin{equation*}
		\alpha=\gamma_1+\gamma_2+\cdots+\gamma_n,\quad
		\beta=\delta_1+\delta_2+\cdots+\delta_n
	\end{equation*}
	其中$\gamma_i,\delta_i\in X_i,\;i=1,2,\dots,n$,于是:
	\begin{equation*}
		k_1\alpha+k_2\beta=\sum_{i=1}^{n}(k_i\gamma_i+k_2\delta_i)
	\end{equation*}
	因为$X_i$是$X$的子空间,所以$k_i\gamma_i+k_2\delta_i\in X_i,\;i=1,2,\dots,n$,于是$k_1\alpha+k_2\beta\in X_1+X_2+\cdots+X_n$。由\cref{theo:Subspace}可得$X_1+X_2+\cdots+X_n$是$X$的子空间。
\end{proof}
\begin{lemma}\label{lem:SubspaceSumalpha}
	设$X$是域$F$上的一个线性空间,$\seq{\alpha}{m},\seq{\beta}{n}\in X$,则:
	\begin{equation*}
		<\seq{\alpha}{m}>+<\seq{\beta}{n}>=<\seq{\alpha}{m},\seq{\beta}{n}>
	\end{equation*}
\end{lemma}
\begin{proof}
	由子空间与子空间和的定义可直接得到。
\end{proof}
\begin{theorem}\label{theo:DimOfSubspace}
	设$X$是域$F$上的一个线性空间,$X_1,X_2$为$X$的有限维子空间,则$X_1\cap X_2,\;X_1+X_2$也是有限维的,并且有:
	\begin{equation*}
		\dim(X_1)+\dim(X_2)=\dim(X_1+X_2)+\dim(X_1\cap X_2)
	\end{equation*}
\end{theorem}
\begin{proof}
	显然$X_1\cap X_2$是有限维的,设$X_1,X_2,X_1\cap X_2$的维数分别为$n_1,n_2,m$。取$X_1\cap X_2$的一组基$\seq{\alpha}{m}$,把它分别扩充为$X_1$和$X_2$的一组基$\seq{\alpha}{m},\seq{\beta}{n_1-m}$与$\seq{\alpha}{m},\seq{\gamma}{n_2-m}$。由\cref{prop:SpanSubspace}(6)和\cref{lem:SubspaceSumalpha}可得:
	\begin{align*}
		X_1+X_2
		&=<\seq{\alpha}{m},\seq{\beta}{n_1-m}>+<\seq{\alpha}{m},\seq{\gamma}{n_2-m}> \\
		&=<\seq{\alpha}{m},\seq{\beta}{n_1-m},\seq{\gamma}{n_2-m}>
	\end{align*}
	根据\cref{prop:SpanSubspace}(4)可得$\operatorname{dim}(X_1+X_2)<n_1+n_2-m$,即$X_1+X_2$是有限维的。\par
	设:
	\begin{equation*}
		k_1\alpha_1+\cdots+k_m\alpha_m+l_1\beta_1+\cdots+l_{n_1-m}\beta_{n_1-m}+q_1\gamma_1+\cdots+q_{n_2-m}\gamma_{n_2-m}=\mathbf{0}
	\end{equation*}
	于是:
	\begin{equation*}
		k_1\alpha_1+\cdots+k_m\alpha_m+l_1\beta_1+\cdots+l_{n_1-m}\beta_{n_1-m}=-(q_1\gamma_1+\cdots+q_{n_2-m}\gamma_{n_2-m})
	\end{equation*}
	左边属于$X_1$,右边属于$X_2$,所以它们属于$X_1\cap X_2$。于是右边的向量也可以由$\seq{\alpha}{m}$线性表出,即:
	\begin{equation*}
		(q_1\gamma_1+\cdots+q_{n_2-m}\gamma_{n_2-m})=p_1\alpha_1+\cdots+p_m\alpha_m
	\end{equation*}
	因为$\seq{\alpha}{m},\seq{\gamma}{n_2-m}$是$X_2$的一组基,所以:
	\begin{equation*}
		q_1=q_2=\cdots=q_{n_2-m}=p_1=p_2=\cdots=p_m=0
	\end{equation*}
	又因为$\seq{\alpha}{m},\seq{\beta}{n_1-m}$是$X_1$的一组基,所以:
	\begin{equation*}
		k_1=k_2=\cdots=k_m=l_1=l_2=\cdots=l_{n_1-m}=0
	\end{equation*}
	于是$\seq{\alpha}{m},\seq{\beta}{n_1-m},\seq{\gamma}{n_2-m}$线性无关。由基的定义,它们是$X$的一组基,于是有:
	\begin{align*}
		\dim(X_1+X_2)
		&=m+n_1-m+n_2-m \\
		&=\dim(X_1\cap X_2)+\dim(X_1)-\dim(X_1\cap X_2)+\dim(X_2)-\dim(X_1\cap X_2) \\
		&=\dim(X_1)+\dim(X_2)-\dim(X_1\cap X_2)
	\end{align*}
	即:
	\begin{equation*}
		\dim(X_1)+\dim(X_2)=\dim(X_1+X_2)+\dim(X_1\cap X_2)\qedhere
	\end{equation*}
\end{proof}
\subsubsection{子空间的直和}
\begin{definition}
	设$X$是域$F$上的线性空间,$\seq{X}{n}$是$X$的子空间。如果$X_1+X_2+\cdots+X_n$中的每个向量$\alpha$都能唯一地表示为:
	\begin{equation*}
		\alpha=\sum_{i=1}^{n}\alpha_i,\;\alpha_i\in X_i
	\end{equation*}
	则称和$X_1+X_2+\cdots+X_n$为\gls{DirectSum},记为$X_1\oplus X_2\oplus\cdots\oplus X_n$,也可以写作$\oplus_{i=1}^nX_i$。若$X=X_1\oplus X_2$,则称$X_2$为$X_1$的\gls{Complement}。
\end{definition}
\begin{theorem}\label{theo:DirectSum}
	设$X$是域$F$上的线性空间,$\seq{X}{n}$是$X$的有限维子空间,则下列命题等价:
	\begin{enumerate}
		\item 和$X_1+X_2+\cdots+X_n$是直和;
		\item 和$X_1+X_2+\cdots+X_n$中零向量的表示方法唯一;
		\item $X_i\cap\left(\sum\limits_{j\ne i}X_j\right)=\mathbf{0},\;i,j=1,2,\dots,n$;
		\item $\dim(X_1+X_2+\cdots+X_n)=\dim(X_1)+\dim(X_2)+\cdots+\dim(X_n)$;
		\item $X_i,\;i=1,2,\dots,n$的基合起来是和$X_1+X_2+\cdots+X_n$的一组基。
	\end{enumerate}
	前三条在$\seq{X}{n}$是无限维子空间时也成立。
\end{theorem}
\begin{proof}
	$1\Rightarrow2$:由直和的定义是显然的。\par
	$2\Rightarrow3$:任取$\alpha\in X_i\cap\left(\sum\limits_{j\ne i}X_j\right)$。因为$\alpha\in X_i$,$X_i$是一个子空间,所以$-\alpha\in X_i$。因为$\alpha\in\left(\sum\limits_{j\ne i}X_j\right)$,所以$\alpha$可表示为:
	\begin{equation*}
		\alpha=\sum_{j\ne i}\alpha_j,\;j=1,2,\dots,i-1,i+1,\dots n
	\end{equation*}
	于是:
	\begin{equation*}
		\mathbf{0}=\alpha+(-\alpha)=-\alpha+\sum_{j\ne i}\alpha_j
	\end{equation*}
	因为$\mathbf{0}=\mathbf{0}+\mathbf{0}+\cdots+\mathbf{0}$,所以$-\alpha=\mathbf{0}$,$\alpha=\mathbf{0}$。由$\alpha$的任意性,$X_i\cap\left(\sum\limits_{j\ne i}X_j\right)=\mathbf{0},\;i,j=1,2,\dots,n$。\par
	$3\Rightarrow1$:假设和$X_1+X_2+\cdots+X_n$不是直和,则存在$\alpha\in X_1+X_2+\cdots+X_n$有两种表示方式,设:
	\begin{equation*}
		\alpha=\sum_{i=1}^{n}\alpha_i=\sum_{i=1}^{n}\beta_i,\quad
		\alpha_i,\beta_i\in X_i,\;\alpha_i\ne\beta_i
	\end{equation*}
	则:
	\begin{equation*}
		\alpha_i-\beta_i=\sum_{j\ne i}(\beta_j-\alpha_j),\quad i,j=1,2,\dots,n
	\end{equation*}
	注意到左边属于$X_i$,右边属于$\sum\limits_{j\ne i}X_j$,所以:
	\begin{equation*}
		\alpha_i-\beta_i\in X_i\cap\left(\sum\limits_{j\ne i}X_j\right)=\mathbf{0}
	\end{equation*}
	于是$\alpha_i=\beta_i$,矛盾,因此和$X_1+X_2+\cdots+X_n$是直和。\par
	$1\Rightarrow4$:因为和$X_1+X_2+\cdots+X_n$是直和,由(3)可得$X_i\cap\left(\sum\limits_{j\ne i}X_j\right)=\mathbf{0},\;i,j=1,2,\dots,n$。由\cref{theo:DimOfSubspace}可得:
	\begin{align*}
		\dim\left(\sum_{i=1}^{n}X_i\right)
		&=\dim\left(X_1+\sum_{i=2}^{n}X_i\right)
		=\dim(X_1)+\dim\left(\sum_{i=2}^{n}X_i\right)-\dim\left(X_1\cap\sum_{i=2}^{n}X_i\right) \\
		&=\dim(X_1)+\dim\left(\sum_{i=2}^{n}X_i\right)
	\end{align*}
	注意到:
	\begin{equation*}
		X_2\cap\sum_{i=3}^{n}X_i\subset X_2\cap\left(X_1+\sum_{i=3}^{n}X_i\right)=\mathbf{0}
	\end{equation*}
	所以:
	\begin{equation*}
		\dim\left(\sum_{i=2}^{n}X_i\right)
		=\dim(X_2)+\dim\left(\sum_{i=3}^{n}X_i\right)
	\end{equation*}
	由数学归纳法可得:
	\begin{equation*}
		\dim(X_1+X_2+\cdots+X_n)=\dim(X_1)+\dim(X_2)+\cdots+\dim(X_n)
	\end{equation*}\par
	$4\Rightarrow5$:在$X_i$中取一组基$\alpha_{i1},\alpha_{i2},\dots,\alpha_{ir_i},\;i=1,2,\dots,n$,由\cref{lem:SubspaceSumalpha}和\cref{prop:SpanSubspace}(6)可得:
	\begin{align*}
		\sum_{i=1}^{n}X_i
		&=<\alpha_{11},\alpha_{12},\dots,\alpha_{1r_1}>+<\alpha_{21},\alpha_{22},\dots,\alpha_{2r_2}>+\cdots+<\alpha_{n1},\alpha_{n2},\dots,\alpha_{nr_n}> \\
		&=<\alpha_{11},\alpha_{12},\dots,\alpha_{1r_1},\alpha_{21},\alpha_{22},\dots,\alpha_{2r_2},\dots,\alpha_{n1},\alpha_{n2},\dots,\alpha_{nr_n}>
	\end{align*}
	因为:
	\begin{equation*}
		\dim(X_1+X_2+\cdots+X_n)=\dim(X_1)+\dim(X_2)+\cdots+\dim(X_n)=\sum_{i=1}^{n}r_i
	\end{equation*}
	所以$\alpha_{ij},\;i=1,2,\dots,n,\;j=1,2,\dots,r_i$线性无关,否则的话,由\cref{prop:SpanSubspace}(4)可得$\dim(X_1+X_2+\cdots+X_n)<\sum\limits_{i=1}^{n}r_i$,矛盾。由\cref{prop:nDimensionalLinearSpace}(2)可知$\alpha_{ij},\;i=1,2,\dots,n,\;j=1,2,\dots,r_i$是$X_1+X_2+\cdots+X_n$的一组基,即$X_i,\;i=1,2,\dots,n$的基合起来是和$X_1+X_2+\cdots+X_n$的一组基。\par
	$5\Rightarrow1$:在$X_i$中取一组基$\alpha_{i1},\alpha_{i2},\dots,\alpha_{ir_i},\;i=1,2,\dots,n$,则$\alpha_{ij},\;i=1,2,\dots,n,\;j=1,2,\dots,r_i$是$X_1+X_2+\cdots+X_n$的一组基。设:
	\begin{equation*}
		\mathbf{0}=\alpha_1+\alpha_2+\cdots+\alpha_n,\;\alpha_i\in X_i
	\end{equation*}
	则有:
	\begin{equation*}
		\alpha_1+\alpha_2+\cdots+\alpha_n=\sum_{i=1}^{n}\sum_{j=1}^{r_i}k_{ij}\alpha_{ij}=\mathbf{0}
	\end{equation*}
	$k_{ij}\in F$。因为$\alpha_{ij},\;i=1,2,\dots,n,\;j=1,2,\dots,r_i$是$X_1+X_2+\cdots+X_n$的一组基,所以$k_{ij}=0,\;i=1,2,\dots,n,\;j=1,2,\dots,r_i$,于是$\alpha_i=\mathbf{0}$,即和$X_1+X_2+\cdots+X_n$中零向量的表示方法唯一,由(2)可得和$X_1+X_2+\cdots+X_n$是直和。
\end{proof}
\begin{theorem}\label{theo:ExistenceOfComplement}
	设$X$是域$F$上的线性空间,则$X$的任一子空间$E$都有补空间。
\end{theorem}
\begin{proof}
	\textbf{(1)$\;\dim(X)=n<+\infty$:}取$E$的一组基$\seq{\alpha}{m}$,把它扩充为$X$的一组基$\seq{\alpha}{m},\seq{\beta}{n-m}$,由\cref{lem:SubspaceSumalpha}和\cref{prop:SpanSubspace}(6)可知:
	\begin{align*}
		X
		&=<\seq{\alpha}{m},\seq{\beta}{n-m}> \\
		&=<\seq{\alpha}{m}>+<\seq{\beta}{n-m}>=E+W
	\end{align*}
	其中$W=<\seq{\beta}{n-m}>$。因为$\seq{\alpha}{m},\seq{\beta}{n-m}$线性无关,由\cref{prop:LinearlyDependent}(1)可知$\seq{\beta}{n-m}$线性无关,所以$\seq{\beta}{n-m}$是$W$的一组基。于是$E$的一组基和$W$的一组基合起来就是$X$的一组基。由\cref{theo:DirectSum}(5)可知$X=E\oplus W$,于是$W$是$E$的补空间。\par
	\textbf{(2)$\;\dim(X)=+\infty$:}
	考虑商空间$X/E$,设其一组基为$\alpha_i+E,\;i\in I$,其中$I$是一个指标集。由\cref{cor:A+WALinearIndependent}可知$\alpha_i,\;i\in I$线性无关。令:
	\begin{equation*}
		W=\left\{\sum_{i=1}^{n}k_i\alpha_{i}:k_i\in F,\;n\in\mathbb{N}^+\right\}
	\end{equation*}
	显然$\alpha_i,\;i\in I$是$W$的一组基。下面证明$W$是$E$的一个补空间。\par
	任取$\alpha\in X$,因为$\alpha_i+E,\;i\in I$是$X/E$的一组基,所以:
	\begin{equation*}
		\alpha+E=\sum_{i=1}^{m}l_i\alpha_{i}+E,\;l_j\in F
	\end{equation*}
	即:
	\begin{equation*}
		\alpha-\sum_{i=1}^{m}l_i\alpha_{i}\in E
	\end{equation*}
	于是$\alpha$可以表示为$E$和$W$中两个元素的和。由$\alpha$的任意性,$X=E+W$。\par
	任取$\beta\in W\cap E$,因为$\beta\in W$,所以:
	\begin{equation*}
		\beta=\sum_{i=1}^{t}p_ia_{i},\;p_i\in F
	\end{equation*}
	又因为$\beta\in E$,所以:
	\begin{equation*}
		E=\beta+E=\sum_{i=1}^{t}p_ia_{r_i}+E=\sum_{i=1}^{t}p_i(a_{i}+E)
	\end{equation*}
	因为$\alpha_i+E,\;i\in I$线性无关,所以$p_i=0,\;i=1,2,\dots,t$($E$是$X/E$的零元),于是$\beta=\mathbf{0}$。由\cref{theo:DirectSum}(3)可知$X=E\oplus W$,$W$是$E$的补空间。\par
	综上,$X$的任一子空间$E$都有补空间。
\end{proof}

\subsection{线性空间的同构}
\begin{definition}
	设$X,Y$为域$F$上的线性空间。如果存在$X$到$Y$的一个双射$\sigma$,使得对于任意的$\alpha,\beta\in X,\;k_1,k_2\in F$,有:
	\begin{equation*}
		\sigma(k_1\alpha+k_2\beta)=k_1\sigma(\alpha)+k_2\sigma(\beta)
	\end{equation*}
	则称$\sigma$是$X$到$Y$的一个\gls{Isomorphism},此时称$X$与$Y$\gls{Isomorphic},记作$X\cong Y$。
\end{definition}
\begin{property}\label{prop:IsomorphicOfLinearSpace}
	设$X,Y$为域$F$上的线性空间,且$X,Y$同构,$\sigma$是$X$到$Y$的同构映射,则:
	\begin{enumerate}
		\item $\sigma(\mathbf{0}_X)=\mathbf{0}_Y$;
		\item 对于任意的$\alpha\in X$,有$\sigma(-\alpha)=-\sigma(\alpha)$;
		\item 对于任意的$\seq{\alpha}{n}\in X,\;\seq{k}{n}\in F$,有:
		\begin{equation*}
			\sigma\left(\sum_{i=1}^{n}k_i\alpha_i\right)=\sum_{i=1}^{n}k_i\sigma(\alpha_i)
		\end{equation*}
		\item $\seq{\alpha}{n}\in X$线性相关当且仅当$\sigma(\alpha_1),\sigma(\alpha_2),\dots,\sigma(\alpha_n)$线性相关;
		\item 如果$\seq{\alpha}{n}$是$X$的一组基,则$\sigma(\alpha_1),\sigma(\alpha_2),\dots,\sigma(\alpha_n)$是$Y$的一组基;
		\item 若$\seq{\alpha}{n}$是$X$的一组基,则$\alpha\in X$在基$\seq{\alpha}{n}$下的坐标和$\sigma(\alpha)\in Y$在基$\sigma(\alpha_1),\sigma(\alpha_2),\dots,\sigma(\alpha_n)$下的坐标相同。
		\item 若$E$是$X$的一个子空间,则$\sigma(E)$是$Y$的一个子空间。若$\dim(E)=n<+\infty$,则$\dim \sigma(E)=n$;
		\item 线性空间的同构是一个等价关系,其等价类被称为同构类。
	\end{enumerate}
\end{property}
\begin{proof}
	(1)任取$\alpha\in X$,由\cref{prop:LinearSpace}(3)和同构映射的定义可得$\sigma(\mathbf{0}_X)=\sigma(0\alpha)=0\sigma(\alpha)=\mathbf{0}_Y$。\par
	(2)由\cref{prop:LinearSpace}(6)可知$\sigma(-\alpha)=\sigma[(-1)\alpha]=(-1)\sigma(\alpha)=-\sigma(\alpha)$。\par
	(3)由同构映射的定义直接可得。\par
	(4)设:
	\begin{equation*}
		\sum_{i=1}^{n}k_i\alpha_i=\mathbf{0}_X,\;k_i\in F
	\end{equation*}
	由(3)和(1)可得:
	\begin{equation*}
		\sum_{i=1}^{n}k_i\sigma(\alpha_i)=\sigma(\mathbf{0}_X)=\mathbf{0}_Y
	\end{equation*}
	若$k_i$不全为$0$,即$\seq{\alpha}{n}\in X$或$\sigma(\alpha_1),\sigma(\alpha_2),\dots,\sigma(\alpha_n)$线性相关,显然此时另一个也线性相关。\par
	(5)由(4)可得$\sigma(\alpha_1),\sigma(\alpha_2),\dots,\sigma(\alpha_n)$线性无关。任取$\beta\in Y$,因为$\sigma$是一个满射,则存在$\alpha\in X$使得$\sigma(\alpha)=\beta$。因为$\seq{\alpha}{n}$是$X$的一组基,所以存在$\seq{k}{n}\in F$使得:
	\begin{equation*}
		\alpha=\sum_{i=1}^{n}k_i\alpha_i
	\end{equation*}
	由(3)可得:
	\begin{equation*}
		\beta=\sigma(\alpha)=\sigma\left(\sum_{i=1}^{n}k_i\alpha_i\right)=\sum_{i=1}^{n}k_i\sigma(\alpha_i)
	\end{equation*}
	于是$\beta$可由$\sigma(\alpha_i),\;i=1,2,\dots,n$线性表出。由$\beta$的任意性,$\sigma(\alpha_i),\;i=1,2,\dots,n$是$Y$的一组基。\par
	(6)由(3)(5)直接得到。\par
	(7)任取$\alpha,\beta\in\sigma(E),\;k_1,k_2\in F$,考虑$k_1\alpha+k_2\beta$。因为$\sigma$是$X$到$Y$的一个双射,所以对于$\alpha,\beta$,存在$a,b\in X$满足$\sigma(a)=\alpha,\;\sigma(b)=\beta$。因为$E$是$X$的一个子空间,所以$k_1a+k_2b\in E$,于是$\sigma(k_1a+k_2b)=k_1\alpha+k_2\beta\in\sigma(E)$,所以$\sigma(E)$是$Y$的一个子空间。由(5)可直接得到有限维情况下$E$与$\sigma(E)$之间的维数关系。\par
	(8)反身性由恒等映射保证,对称性由双射保证,传递性由复合映射可直接得到。
\end{proof}
\begin{theorem}\label{theo:IsomorphicDim}
	设$X,Y$为域$F$上的有限维线性空间,则$X$与$Y$同构的充分必要条件为它们的维数相同,于是维数是有限维线性空间同构类的完全不变量。
\end{theorem}
\begin{proof}
	\textbf{(1)必要性:}由\cref{prop:IsomorphicOfLinearSpace}(5)直接得到。\par
	\textbf{(2)充分性:}设二者维数都是$n$,取$X$的一组基$\seq{\alpha}{n}$和$Y$的一组基$\seq{\beta}{n}$。令:
	\begin{equation*}
		\sigma:\sum_{i=1}^{n}k_i\alpha_i\longrightarrow\sum_{i=1}^{n}k_i\beta_i
	\end{equation*}
	显然它是一个线性映射并且是一个双射,于是$X$与$Y$同构。
\end{proof}

\subsection{商空间}
\subsubsection{商空间的定义}
\begin{theorem}\label{theo:QuotientRelationship}
	设$X$是域$F$上的一个线性空间,$L$是$E$的子空间。对于$\forall\;x,y\in X$,若$x-y\in L$,则称$x,y$是等价的,记为$x\sim y$。该关系是一个等价关系。对于该关系的任意等价类$\alpha$,任取$x\in\alpha$,$\alpha$可由$\{x+y:y\in L\}$来表示,简记为$x+L$。
\end{theorem}
\begin{proof}
	任取$x,y,z\in X$。\par
	(1)因为$x-x=\mathbf{0}\in L$,所以该关系满足自反性。\par
	(2)若$x\sim y$,即$x-y\in L$,因为$L$是一个子空间,所以$y-x=(-1)(x-y)\in L$,于是$y\sim x$,该关系满足对称性。\par
	(3)若$x\sim y,\;y\sim z$,则有$x-y\in L,\;y-z\in L$,因为$L$是一个线性空间,所以$x-z=(x-y)+(y-z)\in L$,即$x\sim z$。该关系满足传递性。\par
	综上,该关系是一个等价关系。\par
	下证明表示方法的正确性。\par
	对任意的$a\in\alpha$,因为$x\in\alpha$,所以$x-a\in L$,于是存在$z\in L$使得$x-a=-z$,即$a=x+z$。
\end{proof}
\begin{theorem}
	设$X$是域$F$上的一个线性空间,$L$是$X$的子空间,$\hat{X}$为$X$中所有等价类构成的集合,确定等价类的关系为\cref{theo:QuotientRelationship}中的关系。对任意的$\alpha,\beta\in\hat{X},\;k\in F$,在$\hat{X}$中定义线性运算如下:
	\begin{gather*}
		\alpha+\beta=x+y+L,\;x\in\alpha,\;y\in\beta \\
		k\alpha=kx+L,\;x\in\alpha
	\end{gather*}
	则$\hat{X}$成为一个线性空间,称其为$X$关于$L$的\gls{QuotientSpace},记作$X/L$。
\end{theorem}
\begin{proof}
	先证明上述线性运算与$x,y$的选择无关。\par
	(1)任取$x'\in\alpha,\;y'\in\beta$,满足$x\ne x',\;y\ne y'$。于是有:
	\begin{equation*}
		x'+y'+L=x+y+(x'-x)+(y'-y)+L
	\end{equation*}
	因为$x,x'\in\alpha,\;y,y'\in\beta$,所以$x'-x,y'-y\in L$,于是$x'+y'+L=x+y+L$。\par
	(2)任取$x'\in\alpha$,满足$x\ne x'$。于是有:
	\begin{equation*}
		kx'+L=kx+k(x'-x)+L
	\end{equation*}
	因为$x,x'\in\alpha$,所以$x'-x\in L$。因为$L$是线性空间,所以$k(x'-x)\in L$,于是$kx'+L=kx+L$。\par
	下证明$\hat{X}$是一个线性空间。\par
	\begin{enumerate}
		\item 任取$\hat{X}$中的两个元素$\alpha=x+L,\;\beta=y+L$。因为$x,y\in X$,$X$是一个线性空间,所以$x+y=y+x$,于是$\alpha+\beta=x+y+L=y+x+L=\beta+\alpha$。由$\alpha,\beta$的任意性,$\hat{X}$上的加法满足线性空间运算法则(1);
		\item 任取$\hat{X}$中的三个元素$\alpha=x+L,\;\beta=y+L,\;\gamma=z+L$。因为$x,y,z\in X$,$X$是一个线性空间,所以$(x+y)+z=x+(y+z)$,于是$(\alpha+\beta)+\gamma=(x+y)+z+L=x+(y+z)+L=\alpha+(\beta+\gamma)$。由$\alpha,\beta,\gamma$的任意性,$\hat{X}$上的加法满足线性空间运算法则(2);
		\item 任取$\alpha=x+L\in\hat{X}$,则$\alpha+\mathbf{0}+L=x+\mathbf{0}+L=x+L=\alpha$,于是$\mathbf{0}+L$是$\hat{X}$中的零元。$\hat{X}$上的加法满足线性空间运算法则(3);
		\item 任取$\alpha=x+L\in\hat{X}$,则$\alpha+-x+L=x+(-x)+L=\mathbf{0}+L$,于是$-x+L$是$\alpha$的负元。由$\alpha$的任意性,$\hat{X}$上的加法满足线性空间运算法则(4);
		\item 任取$\alpha=x+L\in\hat{X}$。因为$x\in X$,$X$是一个线性空间,所以$1x=x$,于是$1\alpha=1x+L=x+L=\alpha$。由$\alpha$的任意性,$\hat{X}$上的纯量乘法满足线性空间运算法则(5);
		\item 任取$\alpha=x+L\in\hat{X},\;k,l\in F$。因为$x\in X$,$X$是一个线性空间,所以$(kl)x=k(lx)$,于是$(kl)\alpha=(kl)x+L=k(lx)+L=k(l\alpha)$。由$\alpha,k,l$的任意性,$\hat{X}$上的纯量乘法满足线性空间运算法则(6);
		\item 任取$\alpha=x+L\in\hat{X},\;k,l\in F$。因为$x\in X$,$X$是一个线性空间,所以$(k+l)x=kx+lx$,于是$(k+l)\alpha=(k+l)x+L=kx+lx+L=k\alpha+l\alpha$。由$\alpha,k,l$的任意性,$\hat{X}$上的纯量乘法满足线性空间运算法则(7);
		\item 任取$\hat{X}$中的两个元素$\alpha=x+L,\;\beta=y+L$。因为$x,y\in X$,$X$是一个线性空间,所以$k(x+y)=ky+kx$,于是$k(\alpha+\beta)=k(x+y)+L=kx+ky+L=k\beta+k\alpha$。由$\alpha,\beta,k$的任意性,$\hat{X}$上的加法满足线性空间运算法则(8)。
	\end{enumerate}
	综上,$\hat{X}$是一个线性空间。
\end{proof}
\subsubsection{商空间的性质}
\begin{theorem}\label{theo:QuotientDim}
	设$X$是域$F$上的一个有限维线性空间,$E$是$X$的一个子空间,则:
	\begin{equation*}
		\dim(X/E)=\dim(X)-\dim(E)
	\end{equation*}
\end{theorem}
\begin{proof}
	设$\dim(X)=n,\;\dim(E)=m$,取$E$的一组基$\seq{\alpha}{m}$,把它扩充为$X$的一组基$\seq{\alpha}{n}$。任取$\beta+W\in X/E$,因为$\beta\in X$,于是:
	\begin{equation*}
		\beta+E=\sum_{i=1}^{n}k_i\alpha_i+E=\sum_{i=1}^{m}k_i\alpha_i+\sum_{i=m+1}^{n}k_i\alpha_i+E=\sum_{i=m+1}^{n}k_i\alpha_i+E
	\end{equation*}
	其中$k_i\in F,\;i=1,2,\dots,n$。上式表明,对任意的$\beta+W\in X/E$,都可以用$\alpha_i+W,\;i=m+1,m+2,\dots,n$的线性组合表示。下证明它们线性无关。\par
	设:
	\begin{equation*}
		\sum_{i=m+1}^{n}k_i\alpha_i+E=E
	\end{equation*}
	则$\sum\limits_{i=m+1}^{n}k_i\alpha_i\in E$,于是存在$l_j\in F,\;j=1,2,\dots,m$使得:
	\begin{equation*}
		\sum_{j=1}^{m}l_j\alpha_j=\sum_{i=m+1}^{n}k_i\alpha_i
	\end{equation*}
	因为$\seq{\alpha}{n}$线性无关,所以$k_i=l_j=0,\;i=m+1,m+2,\dots,n,\;j=1,2,\dots,m$,因此$\alpha_i+W,\;i=m+1,m+2,\dots,n$线性无关,即它们是$X/E$的一组基,$\dim(X/E)=n-m=\dim(X)-\dim(E)$。
\end{proof}
\begin{definition}
	设$X$是域$F$上的一个线性空间,$E$是$X$的一个子空间。若$X/E$是有限维的,则称$\dim(X/E)$是$E$在$X$中的\gls{Codimension},记作$\operatorname{codim}E$。
\end{definition}
\subsubsection{标准映射}
\begin{definition}
	设$X$是域$F$上的一个线性空间,$E$是$X$的一个子空间,定义映射:
	\begin{equation*}
		\pi:\alpha\longrightarrow\alpha+E,\;\forall\;\alpha\in X
	\end{equation*}
	称之为\gls{CanonicalMapping}。
\end{definition}
\begin{property}\label{prop:CanonicalMap}
	设$X$是域$F$上的一个线性空间,$E$是$X$的一个子空间,$\pi$是$X$到$X/E$的标准映射,则$\pi$具有如下性质:
	\begin{enumerate}
		\item $X/E$中的一个元素$\alpha+E$在$\pi$下的原像是:
		\begin{equation*}
			\{\alpha+\beta:\beta\in E\}
		\end{equation*}
		\item $\pi$是一个满射。当且仅当$E$是零空间时,$\pi$是一个双射;
		\item $\pi$是一个线性映射;
		\item 若$\seq{\alpha}{n}\in X$线性相关,则$\pi(\alpha_1),\pi(\alpha_2),\dots,\pi(\alpha_n)$线性相关。
	\end{enumerate}
\end{property}
\begin{proof}
	(1)显然:
	\begin{align*}
		\gamma\in\pi^{-1}(\alpha+E)
		&\iff
		\gamma+E=\alpha+E
		\iff
		\gamma-\alpha\in E \\
		&\iff
		\exists\;\beta\in E,\;\gamma=\alpha+\beta
		\iff
		\gamma\in\{\alpha+\beta:\beta\in E\}
	\end{align*}\par
	(2)满射是显然的结论。当$E$是零空间时,$\pi$是$X$上的恒等变换,恒等变换显然是双射。当$\pi$是双射时,$\pi$是一个单射,即对任何的$\alpha+E\in X/E$,有且仅有$\alpha\in X$使得$\pi(\alpha)=\alpha+E$,也即不存在$\beta\in X,\;\beta\ne\alpha$,使得$\beta-\alpha\in E$,此时$E$只能为零空间。\par
	(3)$\;\forall\;\alpha,\beta\in X,\;k_1,k_2\in F$,有:
	\begin{equation*}
		\pi(k_1\alpha+k_2\beta)=k_1\alpha+k_2\beta+E=k_1\alpha+E+k_2\alpha+E=k_1\pi(\alpha)+k_2\pi(\beta)
	\end{equation*}
	于是$\pi$是一个线性映射。\par
	(4)因为$\seq{\alpha}{n}$线性相关,所以存在不全为零的$\seq{k}{n}\in F$使得:
	\begin{equation*}
		\sum_{i=1}^{n}k_i\alpha_i=\mathbf{0}
	\end{equation*}
	由(3)可得:
	\begin{equation*}
		\pi\left(\sum_{i=1}^{n}k_i\alpha_i\right)=\sum_{i=1}^{n}k_i\pi(\alpha_i)=\sum_{i=1}^{n}k_i\alpha_i+E=E
	\end{equation*}
	于是$\pi(\alpha_i),\;i=1,2,\dots,n$线性相关。
\end{proof}
\begin{corollary}\label{cor:A+WALinearIndependent}
	设$X$是域$F$上的一个线性空间,$E$是$X$的一个子空间。若$\{\alpha_i+W:i\in I\}$是$X/E$的一组基,$I$是一个指标集,则$\alpha_i,\;i\in I$线性无关。
\end{corollary}
\begin{proof}
	就$X/E$是无限维的给出证明,有限维情况类似。\par
	若此时$\alpha_i,\;i\in I$线性相关,则存在$\alpha_{i_1},\alpha_{i_2},\dots,\alpha_{i_n}$线性相关,其中$n\in \mathbb{N}^+$。由\cref{prop:CanonicalMap}(4)可得$\alpha_{i_j}+W,\;j=1,2,\dots,n$线性相关,于是$\{\alpha_i+W:i\in I\}$线性相关,矛盾。
\end{proof}
