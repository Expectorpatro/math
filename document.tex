%\documentclass[UTF8]{ctexart}
%%%%%%%词汇表%%%%%%%%%%%%
%\usepackage{glossaries}
%\makeglossaries
%%%%%%%%%%%%%%%%%%%%%%%
%\begin{document}
%	\section{词汇表举例}
%	\newglossaryentry{gloss}{
%		name=glossary,
%		description={A vocabulary with annotations for a particular subject},
%		plural=glossaries
%	}
%	
%	\Glspl{gloss} are import for technical documents.
%	
%	\newglossaryentry{sec}{
%		name=分节,
%		description={把文章分成章节},
%	}
%	\gls{sec}对于长文档很重要。
%	%产生词汇表
%	\printglossaries
%\end{document}
\documentclass[11pt, a4paper, twoside, openany]{ctexbook}
\usepackage{glossaries-cn}    % 用于生成术语表

\NewTerm[no acronym] {FT} {Fourier Transform} {傅里叶变换}
\NewTerm[phrase] {AFT} {} {傅里叶逆变换}
\NewTerm {DTFT} {Discrete-Time Fourier Transform} {离散时间傅里叶变换}
\NewTerm {DFT} {Discrete Fourier Transform} {离散傅里叶变换}
\NewTerm {FFT} {Fast Fourier Transformation} {快速傅里叶变换}
\NewSymbol {f} {f} {原函数}
\NewSymbol {ff} {\hat{f}} {原函数$\glsintoc{f}$的傅里叶变换}
\NewSymbol {e} {\mathrm{e}} {自然对数的底}
\NewSymbol {i} {\mathrm{i}} {虚数}
\NewSymbol[hide] {pi} {\pi} {圆周率}
\begin{document}
	\gls{FT}是一种分析信号的方法,它可分析信号的成分,也可用这些成分合成信号。许多波形可作为信号的成分,比如正弦
	波、方波、锯齿波等,\gls{FT}用正弦波作为信号的成分。\gls{FT}分为\gls{DFT}、\gls{DTFT}以及\gls{FFT}。
    \gls{DTFT}是\gls{FT}在时域和频域上都呈离散的形式,\gls{DTFT}是\gls{FT}的一种。是\gls{DFT}之后进行采样的离散
	行数,\gls{FFT}是\gls{DFT}的快速算法,它是根据\gls{DFT}的奇、偶、虚、实等特性,对\gls{DFT}的算法进行
	改进获得的。\gls{FT}如公式\eqref{eqn:Fourier Transform}所示,\gls{AFT}如公式\eqref{eqn:Inverse Transform}所示。\begin{equation}\label{eqn:Fourier Transform}
		\gls{ff} = \int_{-\infty}^{+\infty}\gls{f}(x)\gls{e}^{-2\gls{pi}\gls{i}x\xi}{\rm d}x
		\end{equation}
		\begin{equation}\label{eqn:Inverse Transform}
		gls{f} = \int_{-\infty}^{+\infty}\gls{ff}(\xi)\gls{e}^{2\gls{pi}\gls{i}\xi x} {\rm d}\xi
		\end{equation}
		\PrintTermList
\end{document}