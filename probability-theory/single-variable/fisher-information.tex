\section{Fisher信息量}

\begin{definition}
	设$\mathbf{X}$是测度空间$(X,\mathcal{F},\mu)$上的一个随机向量,其分布由$n$维参数$\theta=(\seq{\theta}{n})$决定,$\mathbf{X}$的概率函数为$f(\mathbf{X};\theta)$。若$f(\mathbf{X};\theta)$满足如下正则条件:
	\begin{enumerate}
		\item $f(\mathbf{X};\theta)$关于$\theta$的偏导数a.e.存在;
		\item 对$f(\mathbf{X};\theta)$在$X$上的积分关于$\theta$任一分量求导时都可以交换求导与积分的顺序;
		\item $f(\mathbf{X};\theta)$的定义域与$\theta$无关。
	\end{enumerate}
	则称:
	\begin{equation*}
		[I(\theta)]_{ij}=\operatorname{E}\left[\left(\frac{\partial\ln f(\mathbf{X};\theta)}{\partial\theta_i}\right)\left(\frac{\partial\ln f(\mathbf{X};\theta)}{\partial\theta_j}\right)\right]
	\end{equation*}
	为$\mathbf{X}$的\gls{FIM}。
\end{definition}
\begin{note}
	Fisher信息量(即一维情况的信息矩阵)用来表明随机变量$\mathbf{X}$携带的关于参数$\theta$的信息。如果它比较大,表示平均下来$\theta$的微小变化会给$\mathbf{X}$的分布带来较大的变化,即$\mathbf{X}$的分布很依赖$\theta$的具体取值,所以携带了较多关于$\theta$的信息。
\end{note}
\begin{property}\label{prop:FIM}
	设$\mathbf{X}$是测度空间$(X,\mathcal{F},\mu)$上的一个随机向量,其Fisher信息矩阵具有如下性质:
	\begin{enumerate}
		\item Fisher信息矩阵可以看作协方差矩阵:
		\begin{equation*}
			I(\theta)=\operatorname{Cov}\left[\frac{\partial\ln f(\mathbf{X};\theta)}{\partial\theta}\right]
		\end{equation*}
		\item 若$\ln f(\mathbf{X};\theta)$有关于$\theta$的所有二阶导数,且对该二阶导数在X上的积分关于$\theta$任一分量求导时都可以交换求导与积分的顺序,则:
		\begin{equation*}
			[I(\theta)]_{ij}=-\operatorname{E}\left[\frac{\partial^2\ln f(\mathbf{X};\theta)}{\partial\theta_i\partial\theta_j}\right]
		\end{equation*}
	\end{enumerate}
\end{property}
\begin{proof}
	(1)由正则条件可得:
	\begin{align*}
		\operatorname{E}\left[\frac{\partial\ln f(\mathbf{X};\theta)}{\partial\theta_i}\right]
		&=\int_{X}\frac{\dfrac{\partial f(\mathbf{X};\theta)}{\partial\theta_i}}{f(\mathbf{X};\theta)}f(\mathbf{X};\theta)\dif\mu
		=\int_{X}\frac{\partial f(\mathbf{X};\theta)}{\partial\theta_i}\dif\mu \\
		&=\frac{\partial}{\partial\theta_i}\int_{X}f(\mathbf{X};\theta)\dif\mu
		=\frac{\partial1}{\partial\theta_i}=0
	\end{align*}\par
	(2)由(1)和正则条件可得:
	\begin{gather*}
		\frac{\partial}{\partial\theta_j}\operatorname{E}\left[\frac{\partial\ln f(\mathbf{X};\theta)}{\partial\theta_i}\right]=0 \\
		\frac{\partial}{\partial\theta_j}\int_{X}\frac{\partial\ln f(\mathbf{X};\theta)}{\partial\theta_i}f(\mathbf{X};\theta)\dif\mu=0 \\
		\int_{X}\frac{\partial}{\partial\theta_j}\left[\frac{\partial\ln f(\mathbf{X};\theta)}{\partial\theta_i}f(\mathbf{X};\theta)\right]\dif\mu=0 \\
		\int_{X}\left[\frac{\partial^2\ln f(\mathbf{X};\theta)}{\partial\theta_i\partial\theta_j}f(\mathbf{X};\theta)+\frac{\partial\ln f(\mathbf{X};\theta)}{\partial\theta_i}\frac{\partial f(\mathbf{X};\theta)}{\partial\theta_j}\right]\dif\mu=0 \\
		\int_{X}\left[\frac{\partial^2\ln f(\mathbf{X};\theta)}{\partial\theta_i\partial\theta_j}f(\mathbf{X};\theta)+\frac{\partial\ln f(\mathbf{X};\theta)}{\partial\theta_i}\frac{\partial \ln f(\mathbf{X};\theta)}{\partial\theta_j}f(\mathbf{X};\theta)\right]\dif\mu=0 \\
		\operatorname{E}\left[\left(\frac{\partial\ln f(\mathbf{X};\theta)}{\partial\theta_i}\right)\left(\frac{\partial\ln f(\mathbf{X};\theta)}{\partial\theta_j}\right)\right]=-\operatorname{E}\left[\frac{\partial^2\ln f(\mathbf{X};\theta)}{\partial\theta_i\partial\theta_j}\right]
	\end{gather*}
	其中倒数第三行到倒数第二行是因为:
	\begin{equation*}
		\frac{\partial\ln f(\mathbf{X};\theta)}{\partial\theta_i}f(\mathbf{X};\theta)=\frac{\dfrac{\partial f(\mathbf{X};\theta)}{\partial\theta_i}}{f(\mathbf{X};\theta)}f(\mathbf{X};\theta)=\dfrac{\partial f(\mathbf{X};\theta)}{\partial\theta_i}
	\end{equation*}
\end{proof}