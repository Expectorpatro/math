\section{方差}

\begin{definition}
	设$f$是概率空间$(X,\mathscr{F},P)$上积分存在的随机变量。若$[f-\operatorname{E}(f)]^2$的积分存在,则称:
	\begin{equation*}
		\operatorname{Var}(f)\coloneq\operatorname{E}\{[f-\operatorname{E}(f)]^2\}
	\end{equation*}
	为$f$的\gls{Variance}。若$[f-\operatorname{E}(f)]^2$可积,则称$f$的方差是有限的。
\end{definition}
\begin{property}\label{prop:Variance}
	设$f,g$是概率空间$(X,\mathscr{F},P)$上积分存在的随机变量。方差具有如下性质:
	\begin{enumerate}
		\item 若$f\in L_2(X)$,则$\operatorname{Var}(f)=\operatorname{E}(f^2)-[\operatorname{E}(f)]^2$;
		\item 若$f\in L_2(X)$,则$\operatorname{Var}(f)=\operatorname{E}[\operatorname{Var}(f|g)]+\operatorname{Var}[\operatorname{E}(f|g)]$;
		\item 若$\operatorname{E}(f)+\operatorname{E}(g)$有意义\info{需要考虑协方差,期望的线性运算那里},则$\operatorname{Var}(f\pm g)=\operatorname{Var}(f)\pm2\operatorname{Cov}(f,g)+\operatorname{Var}(g)$;
	\end{enumerate}
\end{property}
\begin{proof}
	(1)因为$f\in L_2(X)$,所以$\operatorname{E}(|f|^2)<+\infty$,由\cref{theo:LtLs}和\cref{prop:MeasurableIntegral}(4)可知$\operatorname{E}(f)=\mu\in\mathbb{R}^{}$。根据方差的定义和\cref{prop:MeasurableIntegral}(5)可得:
	\begin{align*}
		\operatorname{Var}(f)
		=\operatorname{E}[(f-\mu)^2]
		=\operatorname{E}(f^2-2\mu f+\mu^2)
		=\operatorname{E}(f^2)-2\mu^2+\mu^2
		=\operatorname{E}(f^2)-\mu^2
	\end{align*}\par
	(2)由(1)和\cref{prop:ConditionalExpectation}(3)可得:
	\begin{align*}
		\operatorname{E}[\operatorname{Var}(f|g)]
		&=\operatorname{E}\{\operatorname{E}(f^2|g)-[\operatorname{E}(f|g)]^2\} \\
		&=\operatorname{E}[\operatorname{E}(f^2|g)]-\operatorname{E}\{[\operatorname{E}(f|g)]^2\} \\
		&=\operatorname{E}(f^2)-\operatorname{E}\{[\operatorname{E}(f|g)]^2\} \\
		\operatorname{Var}[\operatorname{E}(f|g)]
		&=\operatorname{E}\{[\operatorname{E}(f|g)]^2\}-\{\operatorname{E}[\operatorname{E}(f|g)]\}^2 \\
		&=\operatorname{E}\{[\operatorname{E}(f|g)]^2\}-[\operatorname{E}(f)]^2
	\end{align*}
	于是:
	\begin{equation*}
		\operatorname{E}[\operatorname{Var}(f|g)]+\operatorname{Var}[\operatorname{E}(f|g)]=\operatorname{E}(f^2)-[\operatorname{E}(f)]^2=\operatorname{Var}(f)
	\end{equation*}\par
	(3)由方差的定义和\cref{prop:MeasurableIntegral}(5)可得:
	\begin{align*}
		\operatorname{Var}(f\pm g)
		&=\operatorname{E}[f\pm g-\operatorname{E}(f\pm g)]^2 \\
		&=\operatorname{E}\{[f-\operatorname{E}(f)\pm[g-\operatorname{E}(g)]]\}^2 \\
		&=\operatorname{E}\{[f-\operatorname{E}(f)]^2\pm 2[f-\operatorname{E}(f)][g-\operatorname{E}(g)]+[g-\operatorname{E}(g)]^2\} \\
		&=\operatorname{Var}(f)\pm2\operatorname{Cov}(f,g)+\operatorname{Var}(g)
	\end{align*}
\end{proof}

