\section{期望}

\begin{definition}
	设$f$是概率空间$(X,\mathscr{F},P)$上的随机变量。若$f$的积分存在,则称$f$的\gls{MathematicalExpectation}(简称为期望)存在,并将:
	\begin{equation*}
		\operatorname{E}(f)\coloneq\int_{X}f(x)\dif P
	\end{equation*}
	称为$f$的期望。若$f$可积,则称$f$的期望是有限的。
\end{definition}
\begin{property}
	设$f$是概率空间$(X,\mathscr{F},P)$上积分存在的随机变量,分布函数为$F$。期望有如下性质:
	\begin{enumerate}
		\item 对任何$(\mathbb{R}^{},\mathcal{B})$上的可测函数$g$,$g\circ f$是$(X,\mathscr{F},P)$上的可测函数,只要:
		\begin{equation*}
			\operatorname{E}(g\circ f),\quad\int_{\mathbb{R}^{}}g\dif Pf^{-1}
		\end{equation*}
		之一有意义,则二者一定相等。特别的,当$f$为连续型随机变量时,上式化作:
		\begin{equation*}
			\operatorname{E}(g\circ f),\quad\int_{\mathbb{R}^{}}g(x)p(x)\dif\lambda,\quad p(x)=\frac{\dif Pf^{-1}}{\dif\lambda}
		\end{equation*}
		其中$\lambda$为L测度;当$f$为离散型随机变量时,令$D=\{x_n\}$,含义与随机变量分类时的含义相同,则上式化作:
		\begin{equation*}
				\operatorname{E}(g\circ f),\quad\sum_{n=1}^{+\infty}g(x_n)p(x_n)
		\end{equation*}
	\end{enumerate}
\end{property}
\begin{proof}
	由\cref{prop:MeasurableMapping}(2)可知$g\circ f$是$(X,\mathscr{F},P)$到$(\mathbb{R}^{},\mathcal{B})$上的可测函数。注意到:
	\begin{equation*}
		\operatorname{E}(g\circ f)=\int_{X}g[f(x)]\dif P=\int_{f^{-1}(\mathbb{R}^{})}g[f(x)]\dif P
	\end{equation*}
	由\cref{theo:IntBySubstitution}即可得出一般结论。根据\cref{lem:IntChangeOfMeasure}可得出连续型随机变量时的结论。对于离散型随机变量,使用典型方法证明即可。\info{离散型得再看看}
\end{proof}
\begin{note}
	很多教材在上述定理中写的并不是$g$对pushforward measure$Pf^{-1}$进行积分,而是对$f$的分布函数$F$进行积分。$F$是一个测度吗?$F$与$Pf^{-1}$等价吗?显然并不是。由\cref{theo:Quasi-distributionMeasure}和\cref{prop:Semiring}(2)我们可以看到$F$可以引出半环$\mathscr{A}=\{(a,b]:a,b\in\mathbb{R}^{}\}$上的测度$\mu$,注意到:
	\begin{equation*}
		\mathbb{R}^{}=\underset{n=1}{\overset{+\infty}{\bigcup}}\Big[(n-1,n]\cup(-n,-n+1]\Big]
	\end{equation*}
	于是$\mathscr{A}$满足\cref{theo:SemiringMeasureExtension}中的条件,于是$\mu$在$\mathcal{B}=\sigma(\{(a,b]:a,b\in\mathbb{R}^{}\})$(\cref{prop:BorelSigmaField}(1.d))上存在唯一的扩张$\nu$。在上述半环上,由$\mu$的定义可知:
	\begin{equation*}
		\mu\Bigl((a,b]\Bigr)=
		\begin{cases}
			F(b)-F(a),&a<b \\
			0,&a\geqslant b
		\end{cases}
	\end{equation*}
	因为$f$是一个可测函数,由\cref{prop:MeasurableFunction}(1)可知$\{a<f\leqslant b\},\{f\leqslant b\},\{f\leqslant a\}\in\mathscr{F}$。根据\cref{prop:Measure}(3)(可减性)可得:
	\begin{equation*}
		Pf^{-1}\Bigl((a,b]\Bigr)=P(\{a<f\leqslant b\})=
		\begin{cases}
			P(\{f\leqslant b\}\backslash\{f\leqslant a\})=F(b)-F(a),&a<b \\
			P(\varnothing)=0,&a\geqslant b
		\end{cases}
	\end{equation*}
	所以$Pf^{-1}$就是$\mu$在$\mathcal{B}$上的扩张$\nu$。由此可以看出$F$可唯一确定$Pf^{-1}$,所以可使用$F$来代表$Pf^{-1}$。
\end{note}

\subsection{条件期望}
\begin{lemma}\label{lem:ConditionalExpectation}
	设$f$为概率空间$(X,\mathscr{F},P)$上积分存在的随机变量。对任意的$A\in\mathscr{F}$,令:
	\begin{equation*}
		\varphi(A)=\int_{A}f(x)\dif P
	\end{equation*}
	则:
	\begin{enumerate}
		\item $\varphi$是$\mathscr{F}$上的符号测度;
		\item $\varphi\ll P$。
	\end{enumerate}
\end{lemma}
\begin{proof}
	(1)因为$P(\varnothing)=0$,由\cref{prop:MeasurableIntegral}(1)可得$\varphi(\varnothing)=0$。根据\cref{theo:MeasurableCountableIntegral}可知对互不相交的$\{A_n\}\subseteq\mathscr{F}$有:
	\begin{equation*}
		\varphi\left(\underset{n=1}{\overset{+\infty}{\cup}}A_n\right)=\int_{\underset{n=1}{\overset{+\infty}{\cup}}A_n}f(x)\dif P=\sum_{n=1}^{+\infty}\left[\int_{A_n}f(x)\dif P\right]=\sum_{n=1}^{+\infty}\varphi(A_n)
	\end{equation*}
	所以$\varphi$是$\mathscr{F}$上的符号测度。\par
	(2)由\cref{prop:MeasurableIntegral}(1)直接可得。
\end{proof}
\begin{definition}
	设$f$为概率空间$(X,\mathscr{F},P)$上积分存在的随机变量,$\mathscr{A}$是$\mathscr{F}$的一个子$\sigma$域,对任意的$A\in\mathscr{F}$,令:
	\begin{equation*}
		\varphi(A)=\int_{A}f(x)\dif P
	\end{equation*}
	因为概率是$\sigma$有限测度,由\cref{prop:Measure}(4)、\cref{lem:ConditionalExpectation}、\cref{prop:SignedMeasure}(7)和\cref{theo:RandonNikodym}可知存在$(X,\mathscr{A},P)$上a.s.于$X$的意义下唯一的积分存在的可测函数$\operatorname{E}(f|\mathscr{A})$满足:
	\begin{equation*}
		\forall\;A\in\mathscr{A},\;\varphi(A)=\int_{A}\operatorname{E}(f|\mathscr{A})\dif P
	\end{equation*}
	若对任意的$A\in\mathscr{A}$有:
	\begin{equation*}
		\int_{A}\operatorname{E}(f|\mathscr{A})\dif P=\int_{A}f(x)\dif P
	\end{equation*}
	则称$\operatorname{E}(f|\mathscr{A})$为$f$关于$\mathscr{A}$的\gls{ConditionalExpectation}。当$\mathscr{A}=\sigma(g)$,$g$是$(X,\mathscr{F})$到可测空间 $(Y,\mathscr{B})$的可测映射,则称$\operatorname{E}(f|g)\coloneq\operatorname{E}[f|\sigma(g)]$为$f$关于$g$的条件期望(由\cref{lem:PreimageSigmaField}可知$\sigma(g)$为$\mathscr{F}$的子$\sigma$域)。分别称:
	\begin{equation*}
		P(A|\mathscr{A})\coloneq\operatorname{E}(I_A|\mathscr{A}),\quad P(A|g)\coloneq P[A|\sigma(g)]
	\end{equation*}
	为事件$A\in\mathscr{F}$关于$\mathscr{A}$的\gls{ConditionalProbability}和事件$A\in\mathscr{F}$关于$g$的条件概率(显然$I_A$在概率空间上积分存在)。
\end{definition}
\begin{lemma}\label{lem:IndependentExpectation}
	设$f$是概率空间$(X,\mathscr{F},P)$上积分存在的随机变量。若$f$与$\mathscr{A}\subseteq\mathscr{F}$独立,则对任意的$A\in\mathscr{A}$有:
	\begin{equation*}
		\operatorname{E}(fI_A)=\operatorname{E}(f)\cdot P(A)
	\end{equation*}
\end{lemma}
\begin{proof}
	使用典型方法进行证明。\par
	\textbf{(1)非负简单函数:}设$f$为非负简单函数,由\cref{prop:SigmaField}(4)和\cref{prop:NonnegativeSimpleIntegral}(4)可得:
	\begin{align*}
		\operatorname{E}(fI_A)&=\int_{X}f(x)I_A(x)\dif P=\int_{A}f(x)\dif P=\sum_{i=1}^{n}c_iP(A\cap E_i) \\
		&=\sum_{i=1}^{n}c_iP(E_i)P(A)=\operatorname{E}(f)\cdot P(A)
	\end{align*}\par
	\textbf{(2)非负可测函数:}设$f$为非负可测函数,根据\cref{prop:MeasurableFunction}(8),取非负简单函数列$\{f_n\}$满足$f_n\uparrow f$,由\cref{prop:Independent}(8)可知 $f_n$与$\mathscr{A}$独立。显然有$f_nI_A\uparrow fI_A$,根据\cref{prop:SimpleFunction}(3)(2)可知$f_nI_A$为非负简单函数,所以由\cref{prop:NonnegativeMeasurableIntegral}(4)、非负简单函数时的结论和极限的线性性质可得:
	\begin{align*}
		\operatorname{E}(fI_A)&=\int_{X}f(x)I_A(x)\dif P=\lim_{n\to+\infty}\left[\int_{X}f_n(x)I_A(x)\dif P\right] \\
		&=\lim_{n\to+\infty}[\operatorname{E}(f_n)\cdot P(A)]=P(A)\lim_{n\to+\infty}\operatorname{E}(f_n)=P(A)\cdot\operatorname{E}(f)
	\end{align*}\par
	\textbf{(3)一般可测函数:}设$f$为一般可测函数,因为$f$积分存在,根据\cref{prop:NonnegativeMeasurableIntegral}(6)可知$(fI_A)^+=f^+I_A,(fI_A)^-=f^-I_A$中至少一个积分有限,所以$fI_A$积分存在。由\cref{prop:Independent}(7)和非负可测函数时的情况可得:
	\begin{align*}
		\operatorname{E}(fI_A)&=\int_{X}f(x)I_A(x)\dif P=\int_{X}f^+(x)I_A(x)\dif P-\int_{X}f^-(x)I_A(x)\dif P \\
		&=P(A)\int_{X}f^+(x)\dif P-P(A)\int_{X}f^-(x)\dif P=P(A)\left[\int_{X}f^+(x)\dif P-\int_{X}f^-(x)\dif P\right] \\
		&=P(A)\int_{X}f(x)\dif\mu=P(A)\cdot\operatorname{E}(f)\qedhere
	\end{align*}
\end{proof}
\begin{property}\label{prop:ConditionalExpectation}
	设$f,g$是概率空间$(X,\mathscr{F},P)$上积分存在的随机变量,$\mathscr{A},\mathscr{B}$是$\mathscr{F}$的子$\sigma$域,则:
	\begin{enumerate}
		\item 若$f\in L_1(X)$,则$f$关于$\mathscr{A}$可测的充要条件为$\operatorname{E}(f|\mathscr{A})=f\;$a.s.于$(X,\mathscr{A},P)$;
		\item 若$f$与$\mathscr{A}$独立,则$\operatorname{E}(f|\mathscr{A})=\operatorname{E}(f)\;$a.s.于$(X,\mathscr{A},P)$;
		\item 若$\mathscr{A}\subseteq\mathscr{B}$,则$\operatorname{E}[\operatorname{E}(f|\mathscr{A})|\mathscr{B}]=\operatorname{E}(f|\mathscr{A})\;$a.s.于$(X,\mathscr{B},P)$,$\operatorname{E}[\operatorname{E}(f|\mathscr{B})|\mathscr{A}]=\operatorname{E}(f|\mathscr{A})\;$a.s.于$(X,\mathscr{A},P)$,$\operatorname{E}[\operatorname{E}(f|\mathscr{A})]=\operatorname{E}(f)$;
 		\item 若$f,g\in L_1(X)$或$f,g$为非负可测函数,$f\leqslant g\;$a.s.于$(X,\mathscr{A},P)$,则$\operatorname{E}(f|\mathscr{A})\leqslant\operatorname{E}(g|\mathscr{A})\;$a.s.于$(X,\mathscr{A},P)$;
		\item 对任意的$\alpha,\beta\in\mathbb{R}^{}$,若$\alpha\operatorname{E}(f)+\beta\operatorname{E}(g)$有意义,则:
		\begin{equation*}
			\operatorname{E}(\alpha f+\beta g|\mathscr{A})=\alpha\operatorname{E}(f|\mathscr{A})+b\operatorname{E}(g|\mathscr{A})
		\end{equation*}
		a.s.于$(X,\mathscr{A},P)$;
		\item 若$0\leqslant f_n\uparrow f\;$a.s.于$(X,\mathscr{F},P)$,则$0\leqslant\operatorname{E}(f_n|\mathscr{A})\uparrow \operatorname{E}(f|\mathscr{A})\;$a.s.于$(X,\mathscr{A},P)$;
		\item 若$f_n\geqslant0\;$a.s.于$(X,\mathscr{F},P)$,则$\operatorname{E}\left(\varliminf\limits_{n\to+\infty}f_n|\mathscr{A}\right)\leqslant\varliminf\limits_{n\to+\infty}\operatorname{E}(f_n|\mathscr{A})\;$a.s.于$(X,\mathscr{A},P)$;
		\item 若$|f_n|\leqslant g\in L_1$且$f_n\overset{\text{a.s.}}{\longrightarrow}f$,则$\lim_{n\to+\infty}\operatorname{E}(f_n|\mathscr{A})=\operatorname{E}(f|\mathscr{A})\;$a.s.于$(X,\mathscr{A},P)$;
	\end{enumerate}
\end{property}
\begin{proof}
	(1)必要性由条件期望的定义和\cref{prop:MeasurableIntegral}(11)即可得到,充分性由条件期望的定义和\cref{prop:MeasurableFunction}(11)即可得到。\par
	(2)因为$\sigma[\operatorname{E}(f)]=\{\varnothing,X\}\subseteq\mathscr{A}$,所以$\operatorname{E}(f)$作为$(X,\mathscr{A},P)$上的函数是可测的,由一般可测函数积分的定义可知$\operatorname{E}(f)$积分存在。分积分有限、积分为正无穷、积分为负无穷三种情况讨论,由积分的性质可得到$\int_{A}\operatorname{E}(f)\dif P=\operatorname{E}(f)P(A)$\info{回头再看看}。对于任意的$A\in\mathscr{A}$,因为$f$与$\mathscr{A}$独立,由\cref{lem:IndependentExpectation}、\cref{prop:MeasurableIntegral}(6)、\cref{prop:SigmaField}(4)和\cref{theo:MeasurableCountableIntegral}可得:
	\begin{equation*}
		\int_{A}\operatorname{E}(f)\dif\mu=\operatorname{E}(f)\cdot P(A)=\operatorname{E}(fI_A)=\int_{X}f(x)I_A(x)\dif\mu=\int_{A}f(x)\dif\mu
	\end{equation*}
	根据$\operatorname{E}(f|\mathscr{A})$的定义可知结论成立。\par
	(3)由$\operatorname{E}(f|\mathscr{A})$的定义可知$\operatorname{E}(f|\mathscr{A})$是$\mathscr{A}$可测的,因为$\mathscr{A}\subseteq\mathscr{B}$,所以$\operatorname{E}(f|\mathscr{A})$是$\mathscr{B}$可测的。由(1)可得$\operatorname{E}[\operatorname{E}(f|\mathscr{A})|\mathscr{B}]=\operatorname{E}(f|\mathscr{A})\;$a.s.于$(X,\mathscr{B},P)$。由$\operatorname{E}[\operatorname{E}(f|\mathscr{B})|\mathscr{A}]$的定义可知它是$\mathscr{A}$可测的,于是对任意的$A\in\mathscr{A}$根据条件期望的定义可得:
	\begin{equation*}
		\int_{A}\operatorname{E}[\operatorname{E}(f|\mathscr{B})|\mathscr{A}]\dif P=\int_{A}\operatorname{E}(f|\mathscr{B})\dif P=\int_{A}f(x)\dif\mu
	\end{equation*}
	即$\operatorname{E}(f|\mathscr{A})=\operatorname{E}[\operatorname{E}(f|\mathscr{B})|\mathscr{A}]\;$a.s.于$(X,\mathscr{A},P)$。\par
	因为$\{\varnothing,X\}$是任意$\sigma$域的子$\sigma$域,于是有:
	\begin{equation*}
		\operatorname{E}[\operatorname{E}(f|\mathscr{A})|\{\varnothing,X\}]=\operatorname{E}(f|\{\varnothing,X\})\;\text{a.s.于}(X,\{\varnothing,X\},P)
	\end{equation*}
	因为任意随机变量与$\{\varnothing,X\}$独立,根据(2)有:
	\begin{equation*}
		\operatorname{E}[\operatorname{E}(f|\mathscr{A})|\{\varnothing,X\}]=\operatorname{E}(f|\mathscr{A})\;\text{a.s.于}(X,\{\varnothing,X\},P)
	\end{equation*}
	由\cref{prop:MeasurableIntegral}(7)可知:
	\begin{align*}
		\operatorname{E}[\operatorname{E}(f|\mathscr{A})]&=\int_{X}\operatorname{E}(f|\mathscr{A})\dif P=\int_{X}\operatorname{E}[\operatorname{E}(f|\mathscr{A})|\{\varnothing,X\}]\dif P \\
		&=\int_{X}\operatorname{E}(f|\{\varnothing,X\})\dif P=\int_{X}f(x)\dif P=\operatorname{E}(f)
	\end{align*}\par
	(4)\textbf{$\;L_1$:}由\cref{prop:MeasurableIntegral}(7)可得:
	\begin{equation*}
		\int_{A}\operatorname{E}(f|\mathscr{A})\dif P=\int_{A}f(x)\dif P\leqslant\int_{A}g(x)\dif P=\int_{A}\operatorname{E}(g|\mathscr{A})\dif P
	\end{equation*}\par
	根据\cref{prop:MeasurableIntegral}(4)(10)即可得出结论。\par
	\textbf{非负可测:}定义截断函数:
	\begin{equation*}
		\forall\;n\in\mathbb{N}^+,\;f_n=f\wedge n,g_n=g\wedge n
	\end{equation*}
	根据\cref{prop:MeasurableFunction}(6)可知$f_n$和$g_n$是可测函数,于是有$0\leqslant f_n\leqslant g_n\leqslant n\;$a.s.于$(X,\mathscr{A},P)$且$f_n\uparrow f,\;g_n\uparrow g$,$f_n,g_n\in L_1$。由$L_1$时的情形可得$\operatorname{E}(f_n|\mathscr{A})\leqslant\operatorname{E}(g_n|\mathscr{A})\;$a.s.于$(X,\mathscr{A},P)$且$\operatorname{E}(f_n|\mathscr{A})\uparrow\;$a.s.于$(X,\mathscr{A},P)$。由(1)可知$\operatorname{E}(0|\mathscr{A})=0\;$a.s.于$(X,\mathscr{A},P)$,由\cref{prop:Measure}(4)(次有限可加性)和测度的非负性可得$0\leqslant\operatorname{E}(f_n|\mathscr{A})\;$a.s.于$(X,\mathscr{A},P)$。根据\cref{prop:MeasurableFunction}(6)可知$\lim\limits_{n\to+\infty}\operatorname{E}(f_n|\mathscr{A})$是可测函数,由极限的不等式性它也是非负函数a.s.于$(X,\mathscr{A},P)$。对任意的$A\in\mathscr{A}$,由\cref{theo:LeviTheorem}可得:
	\begin{equation*}
		\int_{A}\left[\lim_{n\to+\infty}\operatorname{E}(f_n|\mathscr{A})\right]\dif P=\lim_{n\to+\infty}\left[\int_{A}\operatorname{E}(f_n|\mathscr{A})\dif P\right]=\lim_{n\to+\infty}\left[\int_{A}f_n(x)\dif P\right]=\int_{A}f(x)\dif P
	\end{equation*}
	根据条件期望的定义可知:
	\begin{equation*}
		\lim_{n\to+\infty}\operatorname{E}(f_n|\mathscr{A})=\operatorname{E}(f|\mathscr{A})\;\text{a.s.于}(X,\mathscr{A},P)
	\end{equation*}
	同理可得:
	\begin{equation*}
		\lim_{n\to+\infty}\operatorname{E}(g_n|\mathscr{A})=\operatorname{E}(g|\mathscr{A})\;\text{a.s.于}(X,\mathscr{A},P)
	\end{equation*}
	由\cref{prop:Measure}(4)(次有限可加性)、测度的非负性和极限的不等式性即可得出结论。\par
	(5)根据\cref{prop:MeasurableIntegral}(3)(5),对任意的$A\in\mathscr{A}$有:
	\begin{align*}
		\int_{A}\operatorname{E}(\alpha f+\beta g|\mathscr{A})\dif P&=\int_{A}[\alpha f(x)+\beta g(x)]\dif P=\alpha\int_{A}f(x)\dif P+\beta\int_{A}g(x)\dif P \\
		&=\alpha\int_{A}\operatorname{E}(f|\mathscr{A})\dif P+\beta\int_{A}\operatorname{E}(g|\mathscr{A})\dif P \\
		&=\int_{A}\alpha\operatorname{E}(f|\mathscr{A})\dif P+\int_{A}\beta\operatorname{E}(g|\mathscr{A})\dif P \\
		&=\int_{A}[\alpha\operatorname{E}(f|\mathscr{A})+\beta\operatorname{E}(g|\mathscr{A})]\dif P
	\end{align*}
	由条件期望的定义可得$\operatorname{E}(\alpha f+\beta g|\mathscr{A})=\alpha\operatorname{E}(f|\mathscr{A})+\beta\operatorname{E}(g|\mathscr{A})\;$a.s.于$(X,\mathscr{A},P)$。\par
	(6)由(4)非负可测的情况立即可得。\par
	(7)仿照\cref{theo:FatouLemma}即可得到。\par
	(8)由(7),仿照\cref{cor:FatouLemma}和\cref{theo:DominatedConvergenceTheorem}即可得到。
\end{proof}