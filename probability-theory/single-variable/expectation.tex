\section{期望}

\begin{definition}
	设$f$是概率空间$(X,\mathscr{F},P)$上的随机变量。若$f$的积分存在,则称$f$的\gls{MathematicalExpectation}(简称为期望)存在,并将:
	\begin{equation*}
		\operatorname{E}(f)\coloneq\int_{X}f(x)\dif P
	\end{equation*}
	称为$f$的期望。若$f$可积,则称$f$的期望是有限的。
\end{definition}
\begin{property}
	设$f$是概率空间$(X,\mathscr{F},P)$上积分存在的随机变量,分布函数为$F$。期望有如下性质:
	\begin{enumerate}
		\item 对任何$(\mathbb{R}^{},\mathcal{B})$上的可测函数$g$,$g\circ f$是$(X,\mathscr{F},P)$上的可测函数,只要:
		\begin{equation*}
			\operatorname{E}(g\circ f),\quad\int_{\mathbb{R}^{}}g\dif Pf^{-1}
		\end{equation*}
		之一有意义,则二者一定相等。特别的,当$f$为连续型随机变量时,上式化作:
		\begin{equation*}
			\operatorname{E}(g\circ f),\quad\int_{\mathbb{R}^{}}g(x)p(x)\dif\lambda
		\end{equation*}
		其中$\lambda$为L测度;当$f$为离散型随机变量时,令$D=\{x_n\}$,含义与随机变量分类时的含义相同,则上式化作:
		\begin{equation*}
				\operatorname{E}(g\circ f),\quad\sum_{n=1}^{+\infty}g(x_n)p(x_n)
		\end{equation*}
	\end{enumerate}
\end{property}
\begin{proof}
	由\cref{prop:MeasurableMapping}(2)可知$g\circ f$是$(X,\mathscr{F},P)$到$(\mathbb{R}^{},\mathcal{B})$上的可测函数。注意到:
	\begin{equation*}
		\operatorname{E}(g\circ f)=\int_{X}g[f(x)]\dif P=\int_{f^{-1}(\mathbb{R}^{})}g[f(x)]\dif P
	\end{equation*}
	由\cref{theo:IntBySubstitution}即可得出结论。
\end{proof}
\begin{note}
	很多教材在上述定理中写的并不是$g$对pushforward measure$Pf^{-1}$进行积分,而是对$f$的分布函数$F$进行积分。$F$是一个测度吗?$F$与$Pf^{-1}$等价吗?显然并不是。由\cref{theo:Quasi-distributionMeasure}我们可以看到$F$可以引出半环$\mathscr{A}=\{(a,b]:a,b\in\mathbb{R}^{}\}$上的测度$\mu$,注意到:
	\begin{equation*}
		\mathbb{R}^{}=\underset{n=1}{\overset{+\infty}{\bigcup}}\Big[(n-1,n]\cup(-n,-n+1]\Big]
	\end{equation*}
	于是$\mathscr{A}$满足\cref{theo:SemiringMeasureExtension}中的条件,于是$\mu$在$\mathcal{B}=\sigma(\{(a,b]:a,b\in\mathbb{R}^{}\})$(\cref{prop:BorelSigmaField}(1.d))上存在唯一的扩张$\nu$。在上述半环上,由$\mu$的定义可知:
	\begin{equation*}
		\mu\Bigl((a,b]\Bigr)=
		\begin{cases}
			F(b)-F(a),&a<b \\
			0,&a\geqslant b
		\end{cases}
	\end{equation*}
	因为$f$是一个可测函数,由\cref{prop:MeasurableFunction}(1)可知$\{a<f\leqslant b\},\{f\leqslant b\},\{f\leqslant a\}\in\mathscr{F}$。根据\cref{prop:Measure}(3)(可减性)可得:
	\begin{equation*}
		Pf^{-1}\Bigl((a,b]\Bigr)=P(\{a<f\leqslant b\})=
		\begin{cases}
			P(\{f\leqslant b\}\backslash\{f\leqslant a\})=F(b)-F(a),&a<b \\
			P(\varnothing)=0,&a\geqslant b
		\end{cases}
	\end{equation*}
	所以$Pf^{-1}$就是$\mu$在$\mathcal{B}$上的扩张$\nu$。由此可以看出$F$可唯一确定$Pf^{-1}$,所以可使用$F$来代表$Pf^{-1}$。
\end{note}

\subsection{条件期望}
\begin{lemma}\label{lem:ConditionalExpectation}
	设$f$为概率空间$(X,\mathscr{F},P)$上积分存在的随机变量。对任意的$A\in\mathscr{F}$,令:
	\begin{equation*}
		\varphi(A)=\int_{A}f(x)\dif P
	\end{equation*}
	则:
	\begin{enumerate}
		\item $\varphi$是$\mathscr{F}$上的符号测度;
		\item $\varphi\ll P$。
	\end{enumerate}
\end{lemma}
\begin{proof}
	(1)因为$P(\varnothing)=0$,由\cref{prop:MeasurableIntegral}(1)可得$\varphi(\varnothing)=0$。根据\cref{theo:MeasurableCountableIntegral}可知对互不相交的$\{A_n\}\subseteq\mathscr{F}$有:
	\begin{equation*}
		\varphi\left(\underset{n=1}{\overset{+\infty}{\cup}}A_n\right)=\int_{\underset{n=1}{\overset{+\infty}{\cup}}A_n}f(x)\dif P=\sum_{n=1}^{+\infty}\left[\int_{A_n}f(x)\dif P\right]=\sum_{n=1}^{+\infty}\varphi(A_n)
	\end{equation*}
	所以$\varphi$是$\mathscr{F}$上的符号测度。\par
	(2)由\cref{prop:MeasurableIntegral}(1)直接可得。
\end{proof}
\begin{definition}
	设$f$为概率空间$(X,\mathscr{F},P)$上积分存在的随机变量,$\mathscr{A}$是$\mathscr{F}$的一个子$\sigma$域,对任意的$A\in\mathscr{F}$,令:
	\begin{equation*}
		\varphi(A)=\int_{A}f(x)\dif P
	\end{equation*}
	因为概率是$\sigma$有限测度,由\cref{prop:Measure}(4)、\cref{lem:ConditionalExpectation}、\cref{prop:SignedMeasure}(7)和\cref{theo:RandonNikodym}可知存在$(X,\mathscr{A},P)$上a.s.意义下唯一的可测函数$\operatorname{E}(f|\mathscr{A})$满足:
	\begin{equation*}
		\forall\;A\in\mathscr{A},\;\varphi(A)=\int_{A}\operatorname{E}(f|\mathscr{A})\dif P
	\end{equation*}
	若:
	\begin{enumerate}
		\item $\operatorname{E}(f|\mathscr{A})$在$(X,\mathscr{A},P)$上积分存在;
		\item 对任意的$A\in\mathscr{A}$有:
		\begin{equation*}
			\int_{A}\operatorname{E}(f|\mathscr{A})\dif P=\int_{A}f(x)\dif P
		\end{equation*}
	\end{enumerate}
	则称:
	\begin{equation*}
		P(A|\mathscr{A})\coloneq\operatorname{E}(I_A|\mathscr{A})
	\end{equation*}
	为事件$A\in\mathscr{F}$关于$\mathscr{A}$的子$\sigma$域的\gls{ConditionalProbability}。若$g$是$(X,\mathscr{A},P)$到可测空间$(Y,\mathscr{B})$的可测映射,则分别称:
	\begin{equation*}
		\operatorname{E}(f|g)\coloneq\operatorname{E}(f|\sigma(g)),\quad P(A|g)\coloneq P(A|\sigma(g))
	\end{equation*}
	为$f$关于$g$的\gls{ConditionalExpectation}和事件$A\in\mathscr{F}$关于$g$的条件概率。
\end{definition}
\begin{lemma}
	
\end{lemma}
\begin{theorem}
	设$f,g$是概率空间$(X,\mathscr{F},P)$上积分存在的随机变量,$\mathscr{A},\mathscr{B}$是$\mathscr{F}$的子$\sigma$域,则:
	\begin{enumerate}
		\item 若$f$关于$\mathscr{A}$可测,则$\operatorname{E}(f|\mathscr{A})=f\;$a.s.于$\mathscr{A}$;
		\item 若$f$与$\mathscr{A}$独立,则$\operatorname{E}(f|\mathscr{A})=\operatorname{E}(f)\;$a.s.于$\mathscr{A}$;
		\item 若$\mathscr{A}\subseteq\mathscr{B}$,则$\operatorname{E}[\operatorname{E}(f|\mathscr{A})|\mathscr{B}]=\operatorname{E}(f|\mathscr{A})=\operatorname{E}[\operatorname{E}(f|\mathscr{B})|\mathscr{A}]\;$a.s.于;
		\item 若$f\leqslant g\;$a.s.于,则$\operatorname{E}(f|\mathscr{A})\leqslant\operatorname{E}(g|\mathscr{A})\;$a.s.于
		\item 对任意的$\alpha,\beta\in\mathbb{R}^{}$,若$\alpha\operatorname{E}(f)+\beta\operatorname{E}(g)$有意义,则:
		\begin{equation*}
			\operatorname{E}(\alpha f+\beta g|\mathscr{A})=\alpha\operatorname{E}(f|\mathscr{A})+b\operatorname{E}(g|\mathscr{A})
		\end{equation*}
		a.e.于
		\item 若$0$
	\end{enumerate}
\end{theorem}
\begin{proof}
	(1)由\cref{prop:MeasurableIntegral}(11)即可得到。\par
	(2)
\end{proof}