\section{矩}

\begin{definition}
	设$f$是概率空间$(X,\mathscr{F},P)$上的随机变量,$n\in\mathbb{N}^+$。若$\operatorname{E}(|f|^n)<+\infty$,则称$f$的$n$阶\gls{Moment}存在并将:
	\begin{equation*}
		\mu_n=\operatorname{E}(f^n),\quad\nu_n=\operatorname{E}\{[f-\operatorname{E}(f)]^n\}
	\end{equation*}
	称为$f$的$n$阶\gls{RawMoment}和$n$阶\gls{CentralMoment}。
\end{definition}
\begin{property}\label{prop:Moment}
	设$f$是概率空间$(X,\mathscr{F},P)$上的随机变量,$n\in\mathbb{N}^+$。$f$的矩具有如下性质:
	\begin{enumerate}
		\item 若$f$的$n$阶矩存在,则$f$具有所有不超过$n$阶的矩;
		\item $f$的中心矩$\nu_n$与原点矩$\mu_n$之间存在如下关系:
		\begin{equation*}
			\nu_n=\sum_{i=0}^{n}\binom{n}{i}\mu_i(-\mu_1)^{n-i}
		\end{equation*}
		\item 若$f$的$n$阶矩存在,则$\mu_n,\nu_n\in\mathbb{R}^{}$。
	\end{enumerate}
\end{property}
\begin{proof}
	(1)设$f$的$n$阶矩存在,则$\operatorname{E}(|f|^n)<+\infty$,由\cref{theo:LtLs}可知对任意的非负数$i\leqslant n$有$\operatorname{E}(|f|^i)<+\infty$,于是$f$具有所有不超过$n$阶的矩。\par
	(2)由(1)和\cref{prop:MeasurableIntegral}(4)可知$f^i,\;i=0,1,\dots,n$在$X$上可积,即$\mu_i=\operatorname{E}(f)<+\infty$。由中心矩的定义和\cref{prop:MeasurableIntegral}(5)可得:
	\begin{equation*}
		\nu_n
		=\operatorname{E}\{[f-\operatorname{E}(f)]^n\}
		=\operatorname{E}\left[\sum_{i=0}^{n}\binom{n}{i}f^i(-\mu_1)^{n-i}\right]
		=\sum_{i=0}^{n}\binom{n}{i}\mu_i(-\mu_1)^{n-i}
	\end{equation*}\par
	(3)因为$f$的$n$阶矩存在,所以$\operatorname{E}(|f|^n)<+\infty$,即$|f|^n=|f^n|$在$X$上可积,由\cref{prop:MeasurableIntegral}(4)可知$f^n$在$X$上可积,于是$\mu_n=\operatorname{E}(f^n)\in\mathbb{R}^{}$。结合(1)(2)可得$\nu_n\in\mathbb{R}^{}$。综上,$f$的$n$阶中心矩与$n$阶原点矩在$\mathbb{R}^{}$上。
\end{proof}
\begin{note}
	需要注意随机变量的期望是可以取无穷的,而矩则必须是有限值。
\end{note}