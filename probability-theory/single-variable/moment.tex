\section{矩}

\subsection{原点矩}
\begin{definition}
	设$X$是一个随机变量,$n\in\mathbb{N}^+$。若数学期望:
	\begin{equation*}
		\mu_n=\operatorname{E}(X^n)
	\end{equation*}
	存在,则称$\mu_n$为$X$的$n$阶\gls{RawMoment}。
\end{definition}
\begin{definition}
	设$\mathbf{X}$是一个$n$维随机向量,$\seq{\alpha}{n}\in\mathbb{N}$。若数学期望:
	\begin{equation*}
		\mu_{\seq{\alpha}{n}}=\operatorname{E}\left(\prod_{i=1}^{n}\mathbf{X}_i^{\alpha_i}\right)
	\end{equation*}
	存在,则称$\mu_{\seq{\alpha}{n}}$为$\mathbf{X}$的阶数为$(\seq{\alpha}{n})$的原点矩。
\end{definition}
\begin{theorem}\label{theo:LowerRawMoment}
	设$X$是一个随机变量,$m\in\mathbb{N}^+$。若$X$的$m$阶原点矩$\mu_{m}$存在,则$X$具有所有不超过$m$阶的原点矩。
\end{theorem}
\begin{proof}
	取任意的$n<m$且$n\in\mathbb{N}^+$,则显然:
	\begin{equation*}
		|x^n|\leqslant
		\begin{cases}
			1, & |x|\leqslant1 \\
			|x^m|, & |x|>1
		\end{cases}
	\end{equation*}
	于是$|x^n|\leqslant1+|x^m|$。因为$X$有$m$阶中心矩,所以:
	\begin{equation*}
		\int_{-\infty}^{+\infty}|x^m|p(x)\dif x<+\infty
	\end{equation*}
	于是:
	\begin{equation*}
		\int_{-\infty}^{+\infty}|x|^np(x)\dif x
		\leqslant\int_{-\infty}^{+\infty}(|x|^m+1)p(x)\dif x
		=\int_{-\infty}^{+\infty}|x|^mp(x)\dif x+1<+\infty
	\end{equation*}
	所以$X$具有$n$阶原点矩。由$n$的任意性,结论成立。
\end{proof}

\subsection{中心矩}
\begin{definition}
	设$X$是一个随机变量,$\mu_1=\operatorname{E}(X),\;n\in\mathbb{N}^+$。若数学期望:
	\begin{equation*}
		\nu_n=\operatorname{E}[(X-\mu_1)^n]
	\end{equation*}
	存在,则称$\nu_n$为$X$的$n$阶\gls{CentralMoment}。
\end{definition}
\begin{definition}
设$\mathbf{X}$是一个$n$维随机向量,$\mu=\operatorname{E}(\mathbf{X}),\;\seq{\alpha}{n}\in\mathbb{N}$。若数学期望:
\begin{equation*}
	\nu_{\seq{\alpha}{n}}=\operatorname{E}\left[\prod_{i=1}^{n}(\mathbf{X}_i-\mu_i)^{\alpha_i}\right]
\end{equation*}
存在,则称$\nu_{\seq{\alpha}{n}}$为$\mathbf{X}$的阶数为$(\seq{\alpha}{n})$的中心矩。
\end{definition}
\begin{theorem}\label{theo:Moment}
	随机变量$X$的中心矩$\nu_n$与原点矩$\mu_n$之间存在如下关系:
	\begin{equation*}
		\nu_n=\sum_{i=0}^{n}\binom{n}{i}\mu_i(-\mu_1)^{n-i}
	\end{equation*}
\end{theorem}
\begin{proof}
	由中心矩的定义可得:
	\begin{equation*}
		\nu_n
		=\operatorname{E}[(X-\mu_1)^n]
		=\operatorname{E}\left[\sum_{i=0}^{n}\binom{n}{i}X^i(-\mu_1)^{n-i}\right]
		=\sum_{i=0}^{n}\binom{n}{i}\mu_i(-\mu_1)^{n-i}\qedhere
	\end{equation*}
\end{proof}