\chapter{概率初步}

\section{随机变量的分类}
设$f$是概率空间$(X,\mathscr{F},P)$上的随机变量。由\cref{prop:MeasurableMapping}(3)可知$(\mathbb{R}^{},\mathcal{B},Pf^{-1})$是一个概率空间。因为$Pf^{-1}$是概率测度,所以$Pf^{-1}$是$\sigma$有限测度,而由\cref{prop:LSMeasure}(6)可知L测度$\lambda$也是$\sigma$有限测度,根据\cref{theo:LebesgueDecomposition}可知$Pf^{-1}$有如下分解式:
\begin{equation*}
	Pf^{-1}=\mu_1+\mu_s,\quad\mu_1\ll\lambda,\quad\mu_s\perp\lambda
\end{equation*}
其中$\mu_1$和$\mu_s$是$\sigma$有限的符号测度。由\cref{prop:BorelSigmaField}(2)可知$\mathbb{R}^{}$上的单点集都在$\mathcal{B}$中,于是记:
\begin{gather*}
	D=\{x\in\mathbb{R}^{}:\mu_s(\{x\})>0\}
\end{gather*}
则$D$是有限集或可列集,否则对任意满足$\underset{n=1}{\overset{+\infty}{\cup}}A_n=\mathbb{R}^{}$的互不相交的$\{A_n\}\subseteq\mathcal{B}$,必然有一个$A_n$中有不可列个正测度的单点集(不然的话由\info{可列个可列是可列}即可得到$D$是可列集)。设$A_n$中所有正测度的点构成的集合为$E$,由\cref{theo:JordanDecomposition}可知$\mu_s$可以分解为测度$\varphi^+$与有限测度$\varphi^-$的差或有限测度$\varphi^+$与测度$\varphi^-$的差\info{检查符号测度部分可以负无穷的所有情况}。取:
\begin{equation*}
	\forall\;k\in\mathbb{N}^+,\;E_k=\left\{x\in E:\mu_s(x)\geqslant\frac{1}{k}\right\}
\end{equation*}
因为$E$中的元素不可列,所以必然存在一个$E_k$含有不可列个元素,否则由\info{可列个可列是可列}即可得出$E$中的元素是可列的。根据\cref{prop:BorelSigmaField}(2)可知单点集在$\mathcal{B}$中,因为测度具有可列可加性,可取$E$的一个子集,在选定的子集最前面加入$E_k$中的点,加一个点该子集关于$\varphi^+$的测度就增加$\dfrac{1}{k}$,这个增加的过程可以无限的进行下去,于是$A_n$中存在在$\mathscr{F}$中的可列子集使得其关于$\varphi^+$的测度为$+\infty$。\par
在第一种情况中,由\cref{prop:Measure}(3)(单调性)和$\varphi^-$的有限性可得$\mu_s(A_n)=\varphi^+(A_n)-\varphi^-(A_n)=+\infty$,那么$\mu_s$就不是$\sigma$有限的符号测度,矛盾。\par
第二种情况时,$\mathscr{F}$中不可能存在关于$\varphi^+$的测度为$+\infty$的集合。\par
综上,$D$至多可列。\par
因为$D$至多可列,根据\cref{prop:BorelSigmaField}(2)可得单点集在$\mathcal{B}$中,所以$D\in\mathcal{B}$,于是令:
\begin{equation*}
	\forall\;A\in\mathcal{B},\;\mu_2(A)=\mu_s(A\cap D),\quad\mu_3=\mu_s-\mu_2
\end{equation*}
由$D$的定义和\cref{prop:Measure}(3)(单调性)可得$\mu_2$是一个非负集函数。因为$\mu_s$是符号测度,所以$\mu_2(\varnothing)=\mu_s(\varnothing)=0$,由\cref{prop:SetOperation}(4)可知$\mu_2$满足可列可加性,于是$\mu_2$是测度。因为$\mu_s$是$\sigma$有限测度,则$\mu_2$显然也是$\sigma$有限测度。\info{思考$\mu_3$是不是有限测度}\par
综上,$Pf^{-1}$有分解:
\begin{equation*}
	Pf^{-1}=\mu_1+\mu_2+\mu_3
\end{equation*}\par
若对任意的$A\in\mathcal{B}$有$\mu_2(A)=\mu_3(A)=0$,称$f$为\textbf{连续型随机变量}。因为$\mu_1\ll\lambda$,所以此时有$Pf^{-1}\ll\lambda$,因为$\lambda$是$\sigma$有限测度,$Pf^{-1}$是概率测度,由\cref{theo:RandonNikodym}可知存在$(\mathbb{R}^{},\mathcal{B},\lambda)$上在a.e.于$\mathbb{R}^{}$的意义下唯一的可测函数$p(x)$满足:
\begin{equation*}
	\dfrac{\dif Pf^{-1}}{\dif \lambda}(x)=p(x)
\end{equation*}
称$p(x)$为$f$的\textbf{概率密度函数}。在没有明确随机变量的场合,也将概率测度$P$对L测度$\lambda$的Randon-Nikodym导数称为概率密度函数。\par
若对任意的$A\in\mathcal{B}$有$\mu_1(A)=\mu_3(A)=0$,称$f$为\textbf{离散型随机变量}。记$D=\{x_n\},\;p(x_n)=Pf^{-1}(\{x_n\}),\;n\in\mathbb{N}^+$,则:
\begin{equation*}
	\{p(x_n):x_n\in D\}
\end{equation*}
完全确定了$f$的概率分布,称上式为$f$的\textbf{概率分布列}(简称为分布列)。在没有明确随机变量的场合,也将满足条件的$\mu_2$称为概率分布列。\par
若对任意的$A\in\mathcal{B}$有$\mu_1(A)=\mu_2(A)=0$,称$f$为\textbf{奇异型随机变量}。\par
统称概率分布列与概率密度函数为\textbf{概率函数}。\par
\begin{theorem}
	任何随机变量的分布都是离散型、连续型、奇异型随机变量分布的混合。
\end{theorem}

\section{独立性}
\begin{definition}
	设$(X,\mathscr{F},P)$是一个概率空间,$T$是一个指标集。
	\begin{enumerate}
		\item 若$(X,\mathscr{F},P)$上的集族$\{A_n:n\in T\}$对任意的$m\in\mathbb{N}^+$和任意的$\{\seq{n}{m}\}\subseteq T$,有:
		\begin{equation*}
			P\left(\underset{i=1}{\overset{m}{\cap}}A_{n_i}\right)=\prod_{i=1}^{m}P(A_{n_i})
		\end{equation*}
		则称$\{A_n:n\in T\}$\gls{MutuallyIndependent};
		\item 设$\{\mathscr{A}_n\subseteq\mathscr{F}:n\in T\}$是$(X,\mathscr{F},P)$上由集族构成的集族,若对每个$n\in T$任取一个$A_n\in\mathscr{A}_n$所构成的集族$\{A_n:n\in T\}$相互独立,则称$\{\mathscr{A}_n\subseteq\mathscr{F}:n\in T\}$相互独立;
		\item 设$\{f_n:n\in T\}$是由$(X,\mathscr{F},P)$上的随机变量构成的随机变量族,若$\{\sigma(f_n):n\in T\}$相互独立,则称$\{f_n:n\in T\}$相互独立。若$\sigma(f)$与$\mathscr{A}\subseteq\mathscr{F}$独立,也称为$f$与$\mathscr{A}$独立。
	\end{enumerate}
	若$(X,\mathscr{F},P)$上的集族$\{A_n\}$对任意的$i\ne j$有$P(A_i\cap A_j)=P(A_i)P(A_j)$,则称$\{A_n\}$\gls{PairwiseIndependent}。可见两个集合之间的相互独立性与成对独立性是等价的,于是将两个集合相互独立或成对独立简称为它们独立。
\end{definition}
\begin{property}\label{prop:Independent}
	设$(X,\mathscr{F},P)$是一个概率空间。独立性具有如下性质:
	\begin{enumerate}
		\item 若$\{A_n\}\subseteq\mathscr{F}$相互独立,则对其中任意个元素取补集后得到的$\{B_n\}$也相互独立;
		\item $\{A_n\}\subseteq\mathscr{F}$相互独立当且仅当$\{I_{A_n}\}$相互独立;
		\item $\{\mathscr{A}_n\subseteq\mathscr{F}\}$相互独立则$\{(X,\mathscr{A}_n)\}$上任意的随机变量族也相互独立;
		\item 设$\{f_n\}$是$(X,\mathscr{F},P)$上相互独立的随机变量族,$\{g_n\}$是一族$(\mathbb{R}^{},\mathcal{B})$上的Borel函数,则$\{g_n\circ f_n\}$是相互独立的随机变量族;
		\item 若由$\pi$系构成的集族$\{\mathscr{A}_n\subseteq\mathscr{F}\}$相互独立,则将其中任意个元素修改为由自身生成的$\sigma$域后得到的$\{\mathscr{B}_n\}$也相互独立;
		\item 设$f$为$(X,\mathscr{F})$上的随机变量,$f$与$\mathscr{A}\subseteq\mathscr{F}$独立,$g$是$(\mathbb{R}^{},\mathcal{B})$上的Borel函数,则$g\circ f$与$\mathscr{A}$独立;
		\item 若$f$与$\mathscr{A}\subseteq\mathscr{F}$独立,则$f^+,f^-$与$\mathscr{A}$独立;
		\item 若$f$与$\mathscr{A}\subseteq\mathscr{F}$独立,当$f$是非负可测函数时,\cref{prop:MeasurableFunction}(8)中满足$\varphi_n\uparrow f$的非负简单函数列$\{\varphi_n\}$与$\mathscr{A}$独立;
	\end{enumerate}
\end{property}
\begin{proof}
	(1)当$\{A_n\}=\{\seq{A}{n}\}$时,考虑$\{A_1^c,A_2,\dots,A_n\}$的情况。对任意的$m\in\{1,2,\dots,n\}$和任意的$\{\seq{n}{m}\}\subseteq\{1,2,\dots,n\}$,若$1\notin\{\seq{n}{m}\}$,自然有:
	\begin{equation*}
		P\left(\underset{i=1}{\overset{m}{\cap}}A_{n_i}\right)=\prod_{i=1}^{m}P(A_{n_i})
	\end{equation*}
	若$1\in\{\seq{n}{m}\}$,修改数值使得$n_1=1$,由\cref{prop:SetOperation}(6)和\cref{prop:Measure}(2)可知::
	\begin{align*}
		P\left[A_1^c\cap\left(\underset{i=2}{\overset{m}{\cap}}A_{n_i}\right)\right]&=P\left\{\underset{i=2}{\overset{m}{\cap}}A_{n_i}\backslash\left[A_1\cap\left(\underset{i=2}{\overset{m}{\cap}}A_{n_i}\right)\right]\right\}=P\left(\underset{i=2}{\overset{m}{\cap}}A_{n_i}\right)-P\left(\underset{i=1}{\overset{m}{\cap}}A_{n_i}\right) \\
		&=\prod_{i=2}^{m}P(A_{n_i})-\prod_{i=1}^{m}P(A_{n_i})=[1-P(A_1)]\prod_{i=2}^{m}P(A_{n_i}) \\
		&=P(A_1^c)\prod_{i=2}^{m}P(A_{n_i})
	\end{align*}
	于是$\{A_1^c,A_2,\dots,A_n\}$相互独立,该结论显然可推广至对$\{A_n\}$中任意个元素取补集后得到的$\{B_n\}$也相互独立。\par
	当$\{A_n\}$含有无穷个元素时,由相互独立的定义和有限个元素时的情况即可得出结论。\par
	(2)注意到$\{\sigma(I_{A_n})\}=\Big\{\{\varnothing,A_n,A_n^c,X\}\Big\}$,所以充分性显然成立,下证必要性。对每个$n\in T$任取一个$B_n\in\sigma(I_{A_n})$从而构成集族$\{B_n\}$。对任意的$m\in\mathbb{N}^+$且$m$小于等于$\{B_n\}$中元素的个数,从$\{B_n\}$中任取$m$个元素构成集族$\{C_n\}$。若$\varnothing\in\{C_n\}$,则:
	\begin{equation*}
		P\left(\underset{i=1}{\overset{m}{\cap}}C_i\right)=P(\varnothing)=0,\quad\prod_{i=1}^{m}P(C_i)=0
	\end{equation*}
	二者相等。其它情况由$X=\varnothing^c$和(1)即可得出。\par
	(3)由随机变量的定义立即可得。\par
	(4)由\cref{prop:MeasurableMapping}(2)的证明过程可知$\sigma(g_n\circ f_n)\subseteq\sigma(f_n)$,而$\{\sigma(f_n)\}$相互独立,所以$\{g_n\circ f_n\}$相互独立。\par
	(5)当$\{\mathscr{A}_n\}=\{\seq{\mathscr{A}}{n}\},\;\{\mathscr{B}_n\}=\{\sigma(\mathscr{A}_1),\mathscr{A}_2,\dots,\mathscr{A}_n\}$时,记:
	\begin{equation*}
		\mathscr{B}=\left\{A\in\mathscr{F}:\textbf{对任意的}B=\underset{i=2}{\overset{m}{\cap}}A_{n_i},A_{n_i}\in\mathscr{A}_{n_i},\;i=2,\dots,m,\;m\leqslant n\textbf{有}P(A\cap B)=P(A)P(B)\right\}
	\end{equation*}
	注意到$P(X\cap B)=P(B)=P(X)P(B)$,所以$X\in\mathscr{B}$。若$C,D\in\mathscr{B}$且$D\subseteq C$,由\cref{prop:SetOperation}(4)和\cref{prop:Measure}(2)可得:
	\begin{align*}
		P[(C\backslash D)\cap B]&=P[(C\cap B)\backslash(D\cap B)]=P(C\cap B)-P(D\cap B) \\
		&=P(C)P(B)-P(D)P(B)=[P(C)-P(D)]P(B)=P(C\backslash D)P(B)
	\end{align*}
	所以$C\backslash D\in\mathscr{B}$。取$\mathscr{B}$中单调递增的集合序列$\{B_m\}$,由\cref{prop:SigmaField}(2)可知$\{B_m\cap B\}$也是$\mathscr{F}$中单调递增的集合序列,由\cref{prop:SetOperation}(4)和\cref{prop:Measure}(3)(下连续性)可得:
	\begin{align*}
		P\left[\left(\underset{m=1}{\overset{+\infty}{\cup}}B_m\right)\cap B\right]&=P\left[\underset{m=1}{\overset{+\infty}{\cup}}(B_m\cap B)\right]=\lim_{m\to+\infty}P\left[B_m\cap B\right] \\
		&=\lim_{m\to+\infty}[P(B_m)P(B)]=P(B)\lim_{m\to+\infty}P(B_m) \\
		&=P(B)P\left(\underset{m=1}{\overset{+\infty}{\cup}}B_m\right)
	\end{align*}
	于是$\underset{m=1}{\overset{+\infty}{\cup}}B_m\in\mathscr{B}$。\par
	综上,$\mathscr{B}$是一个$\lambda$系,由相互独立性的定义可知$\mathscr{A}_1\subseteq\mathscr{B}$,根据\cref{cor:SigmaPi=LambdaPi}可得$\sigma(\mathscr{A}_1)\subseteq\mathscr{B}$,所以有$\sigma(\mathscr{A}_1),\mathscr{A}_2,\dots,\mathscr{A}_n$相互独立。显然可将该结论推广为:将$\seq{\mathscr{A}}{n}$中任意个元素修改为由自身生成的$\sigma$域后得到的$\seq{\mathscr{B}}{n}$也相互独立。\par
	当$\{\mathscr{A}_n\}$含有无穷个元素时,由相互独立的定义和有限个元素时的情况即可得出结论。\par
	(6)由\cref{prop:MeasurableMapping}(2)的证明过程可知$\sigma(g\circ f)\subseteq\sigma(f)$,而$\sigma(f)$与$\mathscr{A}$相互独立,所以$g\circ f$与$\mathscr{A}$相互独立。\par
	(7)$\;f^+$可视作$f$关于函数$h$的复合:
	\begin{equation*}
		h(x)=
		\begin{cases}
			x,&x>0 \\
			0,&x\leqslant0
		\end{cases}
	\end{equation*}
	显然$h$是一个$\mathbb{R}^{}$上的实值连续函数,由\cref{prop:MeasurableFunction}(10)即可得到$h$是$(\mathbb{R}^{},\mathcal{B})$上的Borel函数,根据(6)即可得出$f^+$与$\mathscr{A}$独立。$f^-$的情况同理可得。\par
	(8)对任意的$A\in\mathcal{B}$和任意的$n\in\mathbb{N}^+$,根据\cref{prop:MeasurableFunction}(8)可知$\varphi_n^{-1}(A)$是有限个$\sigma(f)$中集合的并集,由\cref{prop:SigmaField}(3)可知$\varphi_n^{-1}(A)\in\sigma(f)$,根据$A$的任意性可得$\sigma(\varphi_n)\subseteq\sigma(f)$。因为$\sigma(f)$与$\mathscr{A}$独立,所以有$\sigma(\varphi_n)$与$\mathscr{A}$独立,即$\varphi_n$与$\mathscr{A}$独立。由$n$的任意性即可得出结论。
\end{proof}

\section{期望}

\begin{definition}
	设$X$是一个随机变量,具有分布函数$F(x)$。若$X$的积分有限,则称:
	\begin{equation*}
		\operatorname{E}(X)=\int_{-\infty}^{+\infty}x\dif F(x)
	\end{equation*}
	为$X$的\gls{MathematicalExpectation}。
\end{definition}

\section{方差}

\begin{definition}
	设$f$是概率空间$(X,\mathscr{F},P)$上积分存在的随机变量。若$[f-\operatorname{E}(f)]^2$的积分存在,则称:
	\begin{equation*}
		\operatorname{Var}(f)\coloneq\operatorname{E}\{[f-\operatorname{E}(f)]^2\}
	\end{equation*}
	为$f$的\gls{Variance}。若$[f-\operatorname{E}(f)]^2$可积,则称$f$的方差是有限的。
\end{definition}
\begin{property}\label{prop:Variance}
	设$f,g$是概率空间$(X,\mathscr{F},P)$上积分存在的随机变量。方差具有如下性质:
	\begin{enumerate}
		\item 若$f\in L_2(X)$,则$\operatorname{Var}(f)=\operatorname{E}(f^2)-[\operatorname{E}(f)]^2$;
		\item $\operatorname{Var}(f)=\operatorname{E}[\operatorname{Var}(f|g)]+\operatorname{Var}[\operatorname{E}(f|g)]$;
		\item $\operatorname{Var}(f\pm g)=\operatorname{Var}(f)\pm\operatorname{Cov}(f,g)+\operatorname{Var}(g)$,若$f,g$不相关,则有$\operatorname{Var}(f\pm g)=\operatorname{Var}(f)+\operatorname{Var}(g)$;
	\end{enumerate}
\end{property}
\begin{proof}
	(1)因为$f\in L_2(X)$,所以$\operatorname{E}(|f|^2)<+\infty$,由\cref{theo:LtLs}和\cref{prop:MeasurableIntegral}(4)可知$\operatorname{E}(f)=\mu\in\mathbb{R}^{}$。根据方差的定义和\cref{prop:MeasurableIntegral}(5)可得:
	\begin{align*}
		\operatorname{Var}(f)
		=\operatorname{E}[(f-\mu)^2]
		=\operatorname{E}(f^2-2\mu f+\mu^2)
		=\operatorname{E}(f^2)-2\mu^2+\mu^2
		=\operatorname{E}(f^2)-\mu^2
	\end{align*}\par
	(2)由(1)可得:
	\begin{align*}
		\operatorname{E}[\operatorname{Var}(f|g)]
		&=\operatorname{E}\{\operatorname{E}(f^2|g)-[\operatorname{E}(f|g)]^2\} \\
		&=\operatorname{E}[\operatorname{E}(f^2|g)]-\operatorname{E}\{[\operatorname{E}(f|g)]^2\} \\
		&=\operatorname{E}(f^2)-\operatorname{E}\{[\operatorname{E}(f|g)]^2\} \\
		\operatorname{Var}[\operatorname{E}(f|g)]
		&=\operatorname{E}\{[\operatorname{E}(f|g)]^2\}-\{\operatorname{E}[\operatorname{E}(f|g)]\}^2 \\
		&=\operatorname{E}\{[\operatorname{E}(f|g)]^2\}-[\operatorname{E}(f)]^2
	\end{align*}
	于是:
	\begin{equation*}
		\operatorname{E}[\operatorname{Var}(f|g)]+\operatorname{Var}[\operatorname{E}(f|g)]=\operatorname{E}(f^2)-[\operatorname{E}(f)]^2=\operatorname{Var}(f)
	\end{equation*}\par
	(3)由方差的定义可得:
	\begin{align*}
		\operatorname{Var}(f\pm g)
		&=\operatorname{E}[f\pm g-\operatorname{E}(f\pm g)]^2 \\
		&=\operatorname{E}\{[f-\operatorname{E}(f)\pm[g-\operatorname{E}(g)]]\}^2 \\
		&=\operatorname{E}\{[f-\operatorname{E}(f)]^2\pm 2[f-\operatorname{E}(f)][g-\operatorname{E}(g)]+[g-\operatorname{E}(g)]^2\} \\
		&=\operatorname{Var}(f)\pm2\operatorname{Cov}(f,g)+\operatorname{Var}(g)
	\end{align*}
\end{proof}


\section{矩}

\begin{definition}
	设$f$是概率空间$(X,\mathscr{F},P)$上的随机变量,$n\in\mathbb{N}^+$。若$\operatorname{E}(|f|^n)<+\infty$,则称$f$的$n$阶\gls{Moment}存在并将:
	\begin{equation*}
		\mu_n=\operatorname{E}(f^n),\quad\nu_n=\operatorname{E}\{[f-\operatorname{E}(f)]^n\}
	\end{equation*}
	称为$f$的$n$阶\gls{RawMoment}和$n$阶\gls{CentralMoment}。
\end{definition}
\begin{property}\label{prop:Moment}
	设$f$是概率空间$(X,\mathscr{F},P)$上的随机变量,$n\in\mathbb{N}^+$。$f$的矩具有如下性质:
	\begin{enumerate}
		\item 若$f$的$n$阶矩存在,则$f$具有所有不超过$n$阶的矩;
		\item $f$的中心矩$\nu_n$与原点矩$\mu_n$之间存在如下关系:
		\begin{equation*}
			\nu_n=\sum_{i=0}^{n}\binom{n}{i}\mu_i(-\mu_1)^{n-i}
		\end{equation*}
		\item 若$f$的$n$阶矩存在,则$\mu_n,\nu_n\in\mathbb{R}^{}$。
	\end{enumerate}
\end{property}
\begin{proof}
	(1)设$f$的$n$阶矩存在,则$\operatorname{E}(|f|^n)<+\infty$,由\cref{theo:LtLs}可知对任意的非负数$i\leqslant n$有$\operatorname{E}(|f|^i)<+\infty$,于是$f$具有所有不超过$n$阶的矩。\par
	(2)由(1)和\cref{prop:MeasurableIntegral}(4)可知$f^i,\;i=0,1,\dots,n$在$X$上可积,即$\mu_i=\operatorname{E}(f)<+\infty$。由中心矩的定义和\cref{prop:MeasurableIntegral}(5)可得:
	\begin{equation*}
		\nu_n
		=\operatorname{E}\{[f-\operatorname{E}(f)]^n\}
		=\operatorname{E}\left[\sum_{i=0}^{n}\binom{n}{i}f^i(-\mu_1)^{n-i}\right]
		=\sum_{i=0}^{n}\binom{n}{i}\mu_i(-\mu_1)^{n-i}
	\end{equation*}\par
	(3)因为$f$的$n$阶矩存在,所以$\operatorname{E}(|f|^n)<+\infty$,即$|f|^n=|f^n|$在$X$上可积,由\cref{prop:MeasurableIntegral}(4)可知$f^n$在$X$上可积,于是$\mu_n=\operatorname{E}(f^n)\in\mathbb{R}^{}$。结合(1)(2)可得$\nu_n\in\mathbb{R}^{}$。综上,$f$的$n$阶中心矩与$n$阶原点矩在$\mathbb{R}^{}$上。
\end{proof}
\begin{note}
	需要注意随机变量的期望是可以取无穷的,而矩则必须是有限值。
\end{note}
\section{协方差}

\begin{definition}
	随机向量$\mathbf{X}$与随机向量$\mathbf{Y}$的\gls{Covariance}矩阵定义为:
	\begin{equation*}
		\operatorname{Cov}(\mathbf{X},\mathbf{Y})=\operatorname{E}\Bigl[\Bigl(\mathbf{X}-\operatorname{E}(\mathbf{X})\Bigr)\Bigl(\mathbf{Y}-\operatorname{E}(\mathbf{Y})\Bigr)^T\Bigr]
	\end{equation*}
	若$\mathbf{X}=\mathbf{Y}$,则可将$\operatorname{Cov}(\mathbf{X},\mathbf{Y})$简写为$\operatorname{Cov}(\mathbf{X})$。
\end{definition}
\begin{definition}
	设$X,Y$是两个随机变量,则:
	\begin{enumerate}
		\item 若$\operatorname{Cov}(X,Y)>0$,称$X,Y$\gls{PositivelyCorrelated};
		\item 若$\operatorname{Cov}(X,Y)<0$,称$X,Y$\gls{NegativelyCorrelated};
		\item 若$\operatorname{Cov}(X,Y)=0$,称$X,Y$\gls{Uncorrelated}。
	\end{enumerate}
\end{definition}
\begin{property}\label{prop:CovMat}
	协方差矩阵具有如下性质:
	\begin{enumerate}
		\item $X$是一个$n$维随机向量,则$\operatorname{tr}[\operatorname{Cov}(\mathbf{X})]=\sum\limits_{i=1}^{n}\operatorname{Var}(\mathbf{X}_i)$;
		\item $X$是一个$n$维随机向量,则$\operatorname{Cov}(\mathbf{X})$是半正定的对称矩阵;
		\item 设$A$和$B$分别为$p\times n$和$q\times m$非随机矩阵,$\mathbf{X}$和$\mathbf{Y}$分别为$n$维、$m$维随机向量,则:
		\begin{equation*}
			\operatorname{Cov}(A\mathbf{X},B\mathbf{Y})=A\operatorname{Cov}(\mathbf{X},\mathbf{Y})B^T
		\end{equation*}
		\item 若$\mathbf{X}$是一个常数向量,$\mathbf{Y}$是一个随机向量,则$\operatorname{Cov}(\mathbf{X},\mathbf{Y})=\mathbf{0}$;
		\item 设$\mathbf{X},\mathbf{Y},\mathbf{Z}$为随机向量,则:
		\begin{gather*}
			\operatorname{Cov}(\mathbf{X}+\mathbf{Y},\mathbf{Z})=\operatorname{Cov}(\mathbf{X},\mathbf{Z})+\operatorname{Cov}(\mathbf{Y},\mathbf{Z}) \\
			\operatorname{Cov}(\mathbf{X},\mathbf{Y}+\mathbf{Z})=\operatorname{Cov}(\mathbf{X},\mathbf{Y})+\operatorname{Cov}(\mathbf{X},\mathbf{Z})
		\end{gather*}
		\item $\operatorname{Cov}(\mathbf{X})=\operatorname{E}(\mathbf{X}\mathbf{X}^T)-[\operatorname{E}(\mathbf{X})][\operatorname{E}(\mathbf{X})]^T$。
	\end{enumerate}
\end{property}
\begin{proof}
	(1)$\;\operatorname{Cov}(\mathbf{X})$在$(i,i)$位置上的元素为:
	\begin{equation*}
		\operatorname{E}\Bigl[\Bigl(\mathbf{X}_i-\operatorname{E}(\mathbf{X}_i)\Bigr)\Bigl(\mathbf{X}_i-\operatorname{E}(\mathbf{X}_i)\Bigr)^T\Bigr]=\operatorname{E}\Bigl[\Bigl(\mathbf{X}_i-\operatorname{E}(\mathbf{X}_i)\Bigr)^2\Bigr]=\operatorname{Var}(\mathbf{X}_i)
	\end{equation*}
	所以$\operatorname{tr}[\operatorname{Cov}(\mathbf{X})]=\sum\limits_{i=1}^{n}\operatorname{Var}(\mathbf{X}_i)$。\par
	(2)因为:
	\begin{align*}
	\operatorname{Cov}(\mathbf{X})_{(i,j)}
	&=\operatorname{E}\Bigl[\Bigl(\mathbf{X}_i-\operatorname{E}(\mathbf{X}_i)\Bigr)\Bigl(\mathbf{X}_j-\operatorname{E}(\mathbf{X}_j)\Bigr)^T\Bigr] \\
	&=\operatorname{E}\Bigl[\Bigl(\mathbf{X}_j-\operatorname{E}(\mathbf{X}_j)\Bigr)\Bigl(\mathbf{X}_i-\operatorname{E}(\mathbf{X}_i)\Bigr)^T\Bigr] \\
	&=\operatorname{Cov}(\mathbf{X})_{(j,i)}
	\end{align*}
	所以$\operatorname{Cov}(\mathbf{X})$是一个对称矩阵。\par
	取$n$维非随机向量$c$,设$Y=c^T\mathbf{X}$则有:
	\begin{align*}
		\operatorname{Var}(Y)
		&=\operatorname{Var}(c^T\mathbf{X}) \\
		&=\operatorname{E}\Bigl[\Bigl(c^T\mathbf{X}-\operatorname{E}(c^T\mathbf{X})\Bigr)\Bigl(c^T\mathbf{X}-\operatorname{E}(c^T\mathbf{X})\Bigr)\Bigr] \\
		&=\operatorname{E}\Bigl[\Bigl(c^T\mathbf{X}-c^T\operatorname{E}(\mathbf{X})\Bigr)\Bigl(c^T\mathbf{X}-c^T\operatorname{E}(\mathbf{X})\Bigr)^T\Bigr] \\
		&=\operatorname{E}\Bigl\{c^T\Bigl(\mathbf{X}-\operatorname{E}(\mathbf{X})\Bigr)\Bigl[c^T\Bigl(\mathbf{X}-\operatorname{E}(\mathbf{X})\Bigr)\Bigr]^T\Bigr\} \\
		&=c^T\operatorname{E}\Bigl[\Bigl(\mathbf{X}-\operatorname{E}(\mathbf{X})\Bigr)\Bigl(\mathbf{X}-\operatorname{E}(\mathbf{X})\Bigr)^T\Bigr]c \\
		&=c^T\operatorname{Cov}(\mathbf{X})c\geqslant0
	\end{align*}
	由$c$的任意性,$\operatorname{Cov}(\mathbf{X})$是半正定的。\par
	(3)类似于(2)中的推导,有:
	\begin{align*}
		\operatorname{Cov}(A\mathbf{X},B\mathbf{Y})
		&=\operatorname{E}\Bigl[\Bigl(A\mathbf{X}-\operatorname{E}(A\mathbf{X})\Bigr)\Bigl(B\mathbf{Y}-\operatorname{E}(B\mathbf{Y})\Bigr)^T\Bigr] \\
		&=A\operatorname{E}\Bigl[\Bigl(\mathbf{X}-\operatorname{E}(\mathbf{X})\Bigr)\Bigl(\mathbf{Y}-\operatorname{E}(\mathbf{Y})\Bigr)^T\Bigr]B^T \\
		&=A\operatorname{Cov}(\mathbf{X},\mathbf{Y})B^T
	\end{align*}\par
	(4)由协方差的定义直接可得;\par
	(5)由\info{期望的线性性}可得:
	\begin{gather*}
		\begin{aligned}
			\operatorname{Cov}(\mathbf{X}+\mathbf{Y},\mathbf{Z})
			&=\operatorname{E}\left[\Bigl(\mathbf{X}+\mathbf{Y}-\operatorname{E}(\mathbf{X}+\mathbf{Y})\Bigr)\Bigl(\mathbf{Z}-\operatorname{E}(\mathbf{Z})\Bigr)^T\right] \\
			&=\operatorname{E}\left[\Bigl(\mathbf{X}+\mathbf{Y}-\operatorname{E}(\mathbf{X})-\operatorname{E}(\mathbf{Y})\Bigr)\Bigl(\mathbf{Z}-\operatorname{E}(\mathbf{Z})\Bigr)^T\right] \\
			&=\operatorname{E}\left[\Bigl(\mathbf{X}-\operatorname{E}(\mathbf{X})\Bigr)\Bigl(\mathbf{Z}-\operatorname{E}(\mathbf{Z})\Bigr)^T+\Bigl(\mathbf{Y}-\operatorname{E}(\mathbf{Y})\Bigr)\Bigl(\mathbf{Z}-\operatorname{E}(\mathbf{Z})\Bigr)^T\right] \\
			&=\operatorname{E}\Bigl[\Bigl(\mathbf{X}-\operatorname{E}(\mathbf{X})\Bigr)\Bigl(\mathbf{Z}-\operatorname{E}(\mathbf{Z})\Bigr)^T\Bigr]+\operatorname{E}\Bigl[\Bigl(\mathbf{Y}-\operatorname{E}(\mathbf{Y})\Bigr)\Bigl(\mathbf{Z}-\operatorname{E}(\mathbf{Z})\Bigr)^T\Bigr] \\
			&=\operatorname{Cov}(\mathbf{X},\mathbf{Z})+\operatorname{Cov}(\mathbf{Y},\mathbf{Z})
		\end{aligned} \\
		\begin{aligned}
			\operatorname{Cov}(\mathbf{X},\mathbf{Y}+\mathbf{Z})
			&=\operatorname{E}\left[\Bigl(\mathbf{X}-\operatorname{E}(\mathbf{X})\Bigr)\Bigl(\mathbf{Y}+\mathbf{Z}-\operatorname{E}(\mathbf{Y}+\mathbf{Z})\Bigr)^T\right] \\
			&=\operatorname{E}\left[\Bigl(\mathbf{X}-\operatorname{E}(\mathbf{X})\Bigr)\Bigl(\mathbf{Y}+\mathbf{Z}-\operatorname{E}(\mathbf{Y})-\operatorname{E}(\mathbf{Z})\Bigr)^T\right] \\
			&=\operatorname{E}\left[\Bigl(\mathbf{X}-\operatorname{E}(\mathbf{X})\Bigr)\Bigl(\mathbf{Y}-\operatorname{E}(\mathbf{Y})\Bigr)^T+\Bigl(\mathbf{X}-\operatorname{E}(\mathbf{X})\Bigr)\Bigl(\mathbf{Z}-\operatorname{E}(\mathbf{Z})\Bigr)^T\right] \\
			&=\operatorname{E}\left[\Bigl(\mathbf{X}-\operatorname{E}(\mathbf{X})\Bigr)\Bigl(\mathbf{Y}-\operatorname{E}(\mathbf{Y})\Bigr)^T\right]+\operatorname{E}\left[\Bigl(\mathbf{X}-\operatorname{E}(\mathbf{X})\Bigr)\Bigl(\mathbf{Z}-\operatorname{E}(\mathbf{Z})\Bigr)^T\right] \\
			&=\operatorname{Cov}(\mathbf{X},\mathbf{Y})+\operatorname{Cov}(\mathbf{X},\mathbf{Z})
		\end{aligned}
	\end{gather*}\par
	(6)显然:
	\begin{align*}
		\operatorname{Cov}(\mathbf{X})
		&=\operatorname{E}\Bigl[\Bigl(\mathbf{X}-\operatorname{E}(\mathbf{X})\Bigr)\Bigl(\mathbf{X}-\operatorname{E}(\mathbf{X})\Bigr)^T\Bigr]
		=\operatorname{E}\Bigl[\Bigl(\mathbf{X}-\operatorname{E}(\mathbf{X})\Bigr)\mathbf{X}^T-\Bigl(\mathbf{X}-\operatorname{E}(\mathbf{X})\Bigr)\operatorname{E}(\mathbf{X})^T\Bigr] \\
		&=\operatorname{E}\Bigl[\Bigl(\mathbf{X}-\operatorname{E}(\mathbf{X})\Bigr)\mathbf{X}^T\Bigr]-\operatorname{E}[\mathbf{X}-\operatorname{E}(\mathbf{X})]\operatorname{E}(\mathbf{X})^T =\operatorname{E}(\mathbf{X}\mathbf{X}^T)-[\operatorname{E}(\mathbf{X})][\operatorname{E}(\mathbf{X})]^T\qedhere
	\end{align*}
\end{proof}

\section{二次型}

\begin{definition}
	$\mathbf{X}$是一个$n$维随机向量,$A=(a_{ij})$为$n$阶非随机实对称阵,则随机变量:
	\begin{equation*}
		\mathbf{X}^TA\mathbf{X}=\sum_{i=1}^{n}\sum_{j=1}^{n}a_{ij}\mathbf{X}_i\mathbf{X}_j
	\end{equation*}
	称为$\mathbf{X}$的二次型。
\end{definition}
\subsubsection{随机变量二次型的均值}
\begin{theorem}\label{theo:ERVQuadraticForm}
	$\mathbf{X}$是一个$n$维随机向量,$\operatorname{E}(\mathbf{X})=\mu,\;\operatorname{Cov}(\mathbf{X})=\Sigma$,则:
	\begin{equation*}
		\operatorname{E}(\mathbf{X}^TA\mathbf{X})=\mu^TA\mu+\operatorname{tr}(A\Sigma)
	\end{equation*}	
\end{theorem}
\begin{proof}
	\begin{align*}
		\operatorname{E}(\mathbf{X}^TA\mathbf{X})
		&=\operatorname{E}[(\mathbf{X}-\mu+\mu)^TA(\mathbf{X}-\mu+\mu)] \\
		&=\operatorname{E}[(\mathbf{X}-\mu)^TA(\mathbf{X}-\mu)]+\operatorname{E}[(\mathbf{X}-\mu)^TA\mu]+\operatorname{E}[\mu^TA(\mathbf{X}-\mu)]+\operatorname{E}(\mu^TA\mu) \\
		&=\operatorname{E}\{\operatorname{tr}[(\mathbf{X}-\mu)^TA(\mathbf{X}-\mu)]\}+\mu^TA\mu \\
		&=\operatorname{E}\{\operatorname{tr}[A(\mathbf{X}-\mu)(\mathbf{X}-\mu)^T]\}+\mu^TA\mu \\
		&=\operatorname{tr}\operatorname{E}[A(\mathbf{X}-\mu)(\mathbf{X}-\mu)^T]+\mu^TA\mu \\
		&=\operatorname{tr}\{A\operatorname{E}[(\mathbf{X}-\mu)(\mathbf{X}-\mu)^T]\}+\mu^TA\mu \\
		&=\operatorname{tr}(A\Sigma)+\mu^TA\mu
	\end{align*}
	第二行到第三行利用到了$\operatorname{E}(\mathbf{X})=\mu$以及$(\mathbf{X}-\mu)^TA(\mathbf{X}-\mu)=\operatorname{tr}[(\mathbf{X}-\mu)^TA(\mathbf{X}-\mu)]$,后式成立是因为$(\mathbf{X}-\mu)^TA(\mathbf{X}-\mu)$是一个标量,标量的迹自然等于自身。第三行到第四行使用到了\cref{prop:Trace}(3)。
\end{proof}
\subsubsection{独立随机变量二次型的方差}
\begin{theorem}\label{theo:VRVQuadraticForm}
	设随机变量$X_i,\;i=1,2,\dots,n$相互独立,$\operatorname{E}(X_i)=\mu_i,\;\operatorname{Var}(X_i)=\sigma^2,\;\nu_k^{(i)}=\operatorname{E}[(X_i-\mu_i)^k]$,$\mathbf{X}=(\seq{X}{n})^T,\;\mu=(\seq{\mu}{n})^T$,$A=(a_{ij})$为$n$阶非随机实对称阵,$a=(a_{11},a_{22},\dots,a_{nn})^T$,$b=(\nu_3^{(1)}a_{11},\nu_3^{(2)}a_{22},\dots,\nu_3^{(n)}a_{nn})^T$,则:
	\begin{equation*}
		\operatorname{Var}(\mathbf{X}^TA\mathbf{X})=\sum_{i=1}^{n}a_{ii}^2\nu_4^{(i)}+\sigma^4[2\operatorname{tr}(A^2)-3a^Ta]+4\sigma^2\mu^TA^2\mu+4\mu^TAb
	\end{equation*}
\end{theorem}
\begin{proof}
	由\cref{prop:Variance}(1)可得:
	\begin{equation*}
		\operatorname{Var}(\mathbf{X}^TA\mathbf{X})=\operatorname{E}[(\mathbf{X}^TA\mathbf{X})^2]-[\operatorname{E}(\mathbf{X}^TA\mathbf{X})]^2 
	\end{equation*}
	由题设可知:
	\begin{equation*}
		\operatorname{E}(\mathbf{X})=\mu,\;\operatorname{Var}(\mathbf{X})=\sigma^2I
	\end{equation*}
	根据\cref{theo:ERVQuadraticForm}可得:
	\begin{align*}
		[\operatorname{E}(\mathbf{X}^TA\mathbf{X})]^2&=[\operatorname{tr}(A\sigma^2I)+\mu^TA\mu]^2=[\sigma^2\operatorname{tr}(A)+\mu^TA\mu]^2 \\
		&=\sigma^4[\operatorname{tr}(A)]^2+2\sigma^2\operatorname{tr}(A)\mu^TA\mu+(\mu^TA\mu)^2
	\end{align*}
	同时:
	\begin{align*}
		(\mathbf{X}^TA\mathbf{X})^2
		&=[(\mathbf{X}-\mu+\mu)^TA(\mathbf{X}-\mu+\mu)]^2 \\
		&=[(\mathbf{X}-\mu)^TA(\mathbf{X}-\mu)+2\mu^TA(\mathbf{X}-\mu)+\mu^TA\mu]^2 \\
		&=[(\mathbf{X}-\mu)^TA(\mathbf{X}-\mu)]^2+4[\mu^TA(\mathbf{X}-\mu)]^2+(\mu^TA\mu)^2 \\
		&\quad+4(\mathbf{X}-\mu)^TA(\mathbf{X}-\mu)\mu^TA(\mathbf{X}-\mu)+2(\mathbf{X}-\mu)^TA(\mathbf{X}-\mu)\mu^TA\mu \\
		&\quad+4\mu^TA(\mathbf{X}-\mu)\mu^TA\mu
	\end{align*}
	令$\mathbf{Y}=\mathbf{X}-\mu$,则有$\operatorname{E}(\mathbf{Y})=\mathbf{0}$,再由\cref{theo:ERVQuadraticForm}可得:
	\begin{align*}
		\operatorname{E}[(\mathbf{X}^TA\mathbf{X})^2]
		&=\operatorname{E}[(\mathbf{Y}^TA\mathbf{Y})^2]+4\operatorname{E}[(\mu^TA\mathbf{Y})^2]+(\mu^TA\mu)^2 \\
		&\quad+4\operatorname{E}(\mathbf{Y}^TA\mathbf{Y}\mu^TA\mathbf{Y})+2\mu^TA\mu\sigma^2\operatorname{tr}(A)
	\end{align*}
	考虑:
	\begin{align*}
		\operatorname{E}[(\mathbf{Y}^TA\mathbf{Y})^2]
		&=\operatorname{E}\left(\sum_{i=1}^{n}\sum_{j=1}^{n}\sum_{k=1}^{n}\sum_{l=1}^{n}a_{ij}a_{kl}\mathbf{Y}_i\mathbf{Y}_j\mathbf{Y}_k\mathbf{Y}_l\right) \\
		&=\sum_{i=1}^{n}\sum_{j=1}^{n}\sum_{k=1}^{n}\sum_{l=1}^{n}a_{ij}a_{kl}\operatorname{E}(\mathbf{Y}_i\mathbf{Y}_j\mathbf{Y}_k\mathbf{Y}_l)
	\end{align*}
	作分类讨论:
	\begin{enumerate}
		\item $i,j,k,l$互不相同,则$\operatorname{E}(\mathbf{Y}_i\mathbf{Y}_j\mathbf{Y}_k\mathbf{Y}_l)=E(\mathbf{Y}_i)E(\mathbf{Y}_j)E(\mathbf{Y}_k)E(\mathbf{Y}_l)=0$;
		\item $i,j,k,l$中存在某两个值相同:
		\begin{itemize}
			\item 此时另外两个不同,则$\operatorname{E}(\mathbf{Y}_i\mathbf{Y}_j\mathbf{Y}_k\mathbf{Y}_l)=0$;
			\item 此时另外两个也相同(即$i=j,k=l$或$i=k,j=l$或$i=l,j=k$),则$\operatorname{E}(\mathbf{Y}_i\mathbf{Y}_j\mathbf{Y}_k\mathbf{Y}_l)=\sigma^4$。
		\end{itemize}
		\item $i,j,k,l$中存在某三个值相同,则$\operatorname{E}(\mathbf{Y}_i\mathbf{Y}_j\mathbf{Y}_k\mathbf{Y}_l)=0$;
		\item $i,j,k,l$相同,则$\operatorname{E}(\mathbf{Y}_i\mathbf{Y}_j\mathbf{Y}_k\mathbf{Y}_l)=\nu_4^{(i)}$。
	\end{enumerate}
	于是:
	\begin{align*}
		\operatorname{E}[(\mathbf{Y}^TA\mathbf{Y})^2]
		&=\sum_{i=1}^{n}\sum_{j=1}^{n}\sum_{k=1}^{n}\sum_{l=1}^{n}a_{ij}a_{kl}\operatorname{E}(\mathbf{Y}_i\mathbf{Y}_j\mathbf{Y}_k\mathbf{Y}_l) \\
		&=\sum_{i=1}^{n}a_{ii}^2\nu_4^{(i)}+\sigma^4\left(\sum_{i\ne k}a_{ii}a_{kk}+\sum_{i\ne j}a_{ij}^2+\sum_{i\ne j}a_{ij}a_{ji}\right) \\
		&=\sum_{i=1}^{n}a_{ii}^2\nu_4^{(i)}+\sigma^4\left(\sum_{i\ne k}a_{ii}a_{kk}+2\sum_{i\ne j}a_{ij}^2\right)
	\end{align*}
	因为:
	\begin{gather*}
		\sum_{i\ne k}a_{ii}a_{kk}=[\operatorname{tr}(A)]^2-a^Ta \\
		\sum_{i\ne j}a_{ij}^2=\operatorname{tr}(AA^T)-a^Ta=\operatorname{tr}(A^2)-a^Ta
	\end{gather*}
	所以:
	\begin{equation*}
		\operatorname{E}[(\mathbf{Y}^TA\mathbf{Y})^2]=\sum_{i=1}^{n}a_{ii}^2\nu_4^{(i)}+\sigma^4\{[\operatorname{tr}(A)]^2+2\operatorname{tr}(A^2)-3a^Ta\}
	\end{equation*}
	由\cref{theo:ERVQuadraticForm}和\cref{prop:Trace}(3)可得:
	\begin{align*}
		\operatorname{E}[(\mu^TA\mathbf{Y})^2]
		&=\operatorname{E}(\mu^TA\mathbf{Y}\mu^TA\mathbf{Y})
		=\operatorname{E}(\mathbf{Y}^TA\mu\mu^TA\mathbf{Y})
		=\operatorname{tr}(A\mu\mu^TA\sigma^2I) \\
		&=\sigma^2\operatorname{tr}(A\mu\mu^TA)
		=\sigma^2\operatorname{tr}(\mu^TA^2\mu)
		=\sigma^2\mu^TA^2\mu
	\end{align*}
	注意到:
	\begin{align*}
		\operatorname{E}(\mathbf{Y}^TA\mathbf{Y}\mu^TA\mathbf{Y})
		&=\operatorname{E}\left(\sum_{i=1}^{n}\sum_{j=1}^{n}a_{ij}\mathbf{Y}_i\mathbf{Y}_j\sum_{k=1}^{n}\sum_{l=1}^{n}a_{kl}\mu_k\mathbf{Y}_l\right) \\
		&=\operatorname{E}\left(\sum_{i=1}^{n}\sum_{j=1}^{n}\sum_{k=1}^{n}\sum_{l=1}^{n}a_{ij}a_{kl}\mu_k\mathbf{Y}_i\mathbf{Y}_j\mathbf{Y}_l\right) \\
		&=\sum_{i=1}^{n}\sum_{j=1}^{n}\sum_{k=1}^{n}\sum_{l=1}^{n}a_{ij}a_{kl}\mu_k\operatorname{E}(\mathbf{Y}_i\mathbf{Y}_j\mathbf{Y}_l)
	\end{align*}
	和之前的讨论类似,可以得到:
	\begin{equation*}
		\operatorname{E}(\mathbf{Y}_i\mathbf{Y}_j\mathbf{Y}_l)=
		\begin{cases}
			\nu_3^{(i)},\;&i=j=l \\
			0,\;&\text{其他情况}
		\end{cases}
	\end{equation*}
	于是有:
	\begin{equation*}
		\operatorname{E}(\mathbf{Y}^TA\mathbf{Y}\mu^TA\mathbf{Y})
		=\sum_{i=1}^{n}\sum_{k=1}^{n}a_{ii}\nu_3^{(i)}a_{ki}\mu_k
	\end{equation*}
	令$b=(\nu_3^{(1)}a_{11},\nu_3^{(2)}a_{22},\dots,\nu_3^{(n)}a_{nn})^T$,则:
	\begin{equation*}
		\operatorname{E}(\mathbf{Y}^TA\mathbf{Y}\mu^TA\mathbf{Y})
		=\sum_{i=1}^{n}\sum_{k=1}^{n}a_{ii}\nu_3^{(i)}a_{ki}\mu_k=\mu^TAb
	\end{equation*}
	将以上求得的期望值全部代入,即可得到:
	\begin{align*}
		\operatorname{E}[(\mathbf{X}^TA\mathbf{X})^2]
		&=\operatorname{E}[(\mathbf{Y}^TA\mathbf{Y})^2]+4\operatorname{E}[(\mu^TA\mathbf{Y})^2]+(\mu^TA\mu)^2 \\
		&\quad+4\operatorname{E}(\mathbf{Y}^TA\mathbf{Y}\mu^TA\mathbf{Y})+2\mu^TA\mu\sigma^2\operatorname{tr}(A) \\
		&=\sum_{i=1}^{n}a_{ii}^2\nu_4^{(i)}+\sigma^4\{[\operatorname{tr}(A)]^2+2\operatorname{tr}(A^2)-3a^Ta\} \\
		&\quad+4\sigma^2\mu^TA^2\mu+(\mu^TA\mu)^2+4\mu^TAb+2\mu^TA\mu\sigma^2\operatorname{tr}(A)
	\end{align*}
	于是:
	\begin{align*}
		\operatorname{Var}(\mathbf{X}^TA\mathbf{X})
		&=\operatorname{E}[(\mathbf{X}^TA\mathbf{X})^2]-[\operatorname{E}(\mathbf{X}^TA\mathbf{X})]^2 \\
		&=\sum_{i=1}^{n}a_{ii}^2\nu_4^{(i)}+\sigma^4\{[\operatorname{tr}(A)]^2+2\operatorname{tr}(A^2)-3a^Ta\} \\
		&\quad+4\sigma^2\mu^TA^2\mu+(\mu^TA\mu)^2+4\mu^TAb+2\mu^TA\mu\sigma^2\operatorname{tr}(A) \\
		&\quad-\sigma^4[\operatorname{tr}(A)]^2-2\sigma^2\operatorname{tr}(A)\mu^TA\mu-(\mu^TA\mu)^2 \\
		&=\sum_{i=1}^{n}a_{ii}^2\nu_4^{(i)}+\sigma^4[2\operatorname{tr}(A^2)-3a^Ta]+4\sigma^2\mu^TA^2\mu+4\mu^TAb\qedhere
	\end{align*}
\end{proof}










\section{矩母函数}

\begin{definition}
	设$X$是一个随机变量。称:
	\begin{equation*}
		M_X(t)=\operatorname{E}(e^{tX})
	\end{equation*}
	为$X$的\gls{m.g.f.},其中$t\in\mathbb{R}$。
\end{definition}
\begin{definition}
	设$\mathbf{X}$是一个$n$维随机向量。称:
	\begin{equation*}
		M_\mathbf{X}(t)=\operatorname{E}(e^{t^T\mathbf{X}})
	\end{equation*}
	为$\mathbf{X}$的矩母函数,其中$t\in\mathbb{R}^{n}$。
\end{definition}
\begin{property}\label{prop:m.g.f.}
	设$\mathbf{X}$是一个$n$维随机向量,则其矩母函数$M_\mathbf{X}(t)$具有如下性质:
	\begin{enumerate}
		\item $M_\mathbf{X}(\mathbf{0})=1$;
		\item $M_\mathbf{X}(t)\geqslant e^{t^T\mu}$,其中$\mu$是$\mathbf{X}$的均值向量;
		\item 矩母函数与概率分布之间存在一个双射,即$M_\mathbf{X}(t)=M_\mathbf{Y}(t)$当且仅当$\mathbf{X}$与$\mathbf{Y}$具有相同的概率分布;
		\item 设$m$维随机向量$\seq{\mathbf{X}}{n}$彼此独立,$\alpha_i$为常数,$\beta_i$为$m$维常数向量,则$\mathbf{Y}=\sum\limits_{i=1}^{n}(\alpha_i\mathbf{X}_i+\beta_i)$的特征函数为:
		\begin{equation*}
			M_\mathbf{Y}(t)=\prod_{i=1}^ne^{t^T\beta_i}M_{\mathbf{X}_i}(\alpha_it)
		\end{equation*}
		\item $M_X^{(n)}(0)=\mu_n$,其中$X$是一个随机变量,$\mu_n$是$X$的$n$阶原点矩;
		\item $M_\mathbf{X}(t)$有如下幂级数展开:
		\begin{equation*}
			M_\mathbf{X}(t)=\sum_{(\seq{m}{n})\in\mathbb{N}^n}\mu_{\seq{m}{n}}\prod_{i=1}^{n}\frac{t_i^{m_i}}{m_i!}
		\end{equation*}
	\end{enumerate}
\end{property}
\begin{proof}
	(1)$M_\mathbf{X}(\mathbf{0})=\operatorname{E}(e^0)=1$。\par
	(2)由Jensen不等式直接可得。\info{Jensen不等式链接}\par
	(3)\par
	(4)由矩母函数定义可得:
	\begin{equation*}
		M_\mathbf{Y}(t)=\operatorname{E}(e^{t^T\mathbf{Y}})
		=\operatorname{E}\left(\exp\left\{t^T\sum_{i=1}^{n}(\alpha_i\mathbf{X}_i+\beta_i)\right\}\right)=\operatorname{E}\left(\prod_{i=1}^{n}e^{\alpha_it^T\mathbf{X}_i}\right)\prod_{i=1}^ne^{t^T\beta_i}
	\end{equation*}
	因为$\mathbf{X}_i$互相独立,所以$\alpha_i\mathbf{X}_i$也相互独立,于是有:
	\begin{equation*}
		M_\mathbf{Y}(t)=\operatorname{E}\left(\prod_{i=1}^{n}e^{\alpha_it^T\mathbf{X}_i}\right)\prod_{i=1}^ne^{t^T\beta_i}=\prod_{i=1}^{n}\operatorname{E}\left(e^{\alpha_it^T\mathbf{X}_i}\right)\prod_{i=1}^ne^{t^T\beta_i}=\prod_{i=1}^ne^{t^T\beta_i}M_{\mathbf{X}_i}(\alpha_it)
	\end{equation*}\par
	(5)将$e^{tX}$展开为幂级数:
	\begin{equation*}
		M_X(t)=\operatorname{E}(e^{tX})=\operatorname{E}\left(\sum_{n=0}^{+\infty}\frac{t^nX^n}{n!}\right)
	\end{equation*}
	于是:
	\begin{equation*}
		M_X^{(n)}(t)=\operatorname{E}\left(X^n+\sum_{m=n+1}^{+\infty}\frac{t^mX^m}{m!}\right)=\mu_n+\sum_{m=1}^{+\infty}\frac{t^m}{m!}\mu_m
	\end{equation*}
	所以:
	\begin{equation*}
		M_X^{(n)}(0)=\operatorname{E}(X^n)=\mu_n
	\end{equation*}\par
	(6)由\info{期望的线性性质,Lebesgue积分}可得:
	\begin{align*}
		M_\mathbf{X}(t)&=\operatorname{E}(e^{t^T\mathbf{X}})
		=\operatorname{E}\left(\exp\left\{\sum_{i=1}^{n}t_i\mathbf{X}_i\right\}\right)
		=\operatorname{E}\left[\sum_{m=0}^{+\infty}\frac{1}{m!}\left(\sum_{i=1}^{n}t_i\mathbf{X}_i\right)^m\right] \\
		&=\sum_{m=0}^{+\infty}\frac{1}{m!}\operatorname{E}\left[\left(\sum_{i=1}^{n}t_i\mathbf{X}_i\right)^m\right]
		=\sum_{m=0}^{+\infty}\frac{1}{m!}\operatorname{E}\left(\sum_{\sum\limits_{i=1}^{n}m_i=m}\frac{m!}{m_1!m_2!\cdots m_n!}\prod_{i=1}^{n}(t_i\mathbf{X}_i)^{m_i}\right) \\
		&=\sum_{m=0}^{+\infty}\frac{1}{m!}\sum_{\sum\limits_{i=1}^{n}m_i=m}\frac{m!}{m_1!m_2!\cdots m_n!}\operatorname{E}\left[\prod_{i=1}^{n}(t_i\mathbf{X}_i)^{m_i}\right] \\
		&=\sum_{m=0}^{+\infty}\sum_{\sum\limits_{i=1}^{n}m_i=m}\frac{1}{m_1!m_2!\cdots m_n!}\operatorname{E}\left(\prod_{i=1}^{n}\mathbf{X}_i^{m_i}\right)\prod_{i=1}^{n}t_i^{m_i} \\
		&=\sum_{(\seq{m}{n})\in\mathbb{N}^n}\mu_{\seq{m}{n}}\prod_{i=1}^{n}\frac{t_i^{m_i}}{m_i!}\qedhere
	\end{align*}
\end{proof}
\section{累积量生成函数}

\begin{definition}
	设$X$是一个随机变量。称$K_X(t)=\log M_X(t)$为$X$的\gls{c.g.f.},其中$t\in\mathbb{R}$。
\end{definition}
\begin{definition}
	设$\mathbf{X}$是一个$n$维随机向量。称$K_\mathbf{X}(t)=\log M_\mathbf{X}(t)$为$\mathbf{X}$的累积量生成函数,其中$t\in\mathbb{R}^{n}$。
\end{definition}
\begin{definition}
	设$\mathbf{X}$是一个$n$维随机向量。因为:
	\begin{align*}
		M_X(t)&=\sum_{(\seq{m}{n})\in\mathbb{N}^n}\frac{1}{m_1!m_2!\cdots m_n!}\prod_{i=1}^{n}t_i^{m_i}\mu_{\seq{m}{n}} \\
		&=1+\sum_{\substack{(\seq{m}{n})\in\mathbb{N}^n \\ (\seq{m}{n}\ne\mathbf{0})}}\frac{1}{m_1!m_2!\cdots m_n!}\prod_{i=1}^{n}t_i^{m_i}\mu_{\seq{m}{n}}
	\end{align*}
	由对数函数的幂级数展开可得:
	\begin{align*}
		K_\mathbf{X}(t)
		&=\log\left(1+\sum_{\substack{(\seq{m}{n})\in\mathbb{N}^n \\ (\seq{m}{n}\ne\mathbf{0})}}\frac{1}{m_1!m_2!\cdots m_n!}\prod_{i=1}^{n}t_i^{m_i}\mu_{\seq{m}{n}}\right) \\
		&=\sum_{j=1}^{+\infty}(-1)^{j+1}\frac{1}{j}\left(\sum_{\substack{(\seq{m}{n})\in\mathbb{N}^n \\ (\seq{m}{n}\ne\mathbf{0})}}\frac{1}{m_1!m_2!\cdots m_n!}\prod_{i=1}^{n}t_i^{m_i}\mu_{\seq{m}{n}}\right)^j
	\end{align*}
\end{definition}
\begin{property}
	
\end{property}
\section{特征函数}


\begin{definition}
	设$X$是一个随机变量。称:
	\begin{equation*}
		\varphi_X(t)=\operatorname{E}(e^{itX})
	\end{equation*}
	为$X$的\gls{c.f.},其中$t\in\mathbb{R}$。
\end{definition}
\begin{definition}
	设$\mathbf{X}$是一个$n$维随机向量。称:
	\begin{equation*}
		\varphi_\mathbf{X}(t)=\operatorname{E}(e^{it^T\mathbf{X}})
	\end{equation*}
	为$\mathbf{X}$的特征函数,其中$t\in\mathbb{R}^{n}$。
\end{definition}
\begin{definition}
	设$\mathbf{X}$是一个$m\times n$随机矩阵。称:
	\begin{equation*}
		\varphi_\mathbf{X}(t)=\operatorname{E}\Bigl[\exp\Bigl(i\operatorname{tr}(t^T\mathbf{X})\Bigr)\Bigr]
	\end{equation*}
	为$\mathbf{X}$的特征函数,其中$t\in M_{m\times n}(\mathbb{R})$。
\end{definition}
\begin{property}\label{prop:CharacteristicFunction}
	设$X,Y,\seq{X}{n}$是随机变量,$\seq{\alpha}{n},\;\beta_1,\beta_2,\dots,\beta_n$为常数,则:
	\begin{enumerate}
		\item $X$的特征函数$\varphi_X(t)$存在;
		\item $|\varphi_X(t)|\leqslant\varphi_X(0)=1$;
		\item $\varphi_X(-t)=\overline{\varphi_X(t)}$;
		\item 若$\seq{X}{n}$相互独立,则$Y=\sum\limits_{k=1}^n(\alpha_kX_k+\beta_k)$的特征函数为:
		\begin{equation*}
			\varphi_{Y}(t)=\prod_{k=1}^ne^{it\beta_k}\varphi_{X_k}(\alpha_kt)
		\end{equation*}
		\item $\seq{X}{n}$相互独立的充分必要条件为:
		\begin{equation*}
			\varphi_{X_1,\dots,X_n}(t_1,t_2,\dots,t_n)=\prod_{i=1}^n\varphi_{X_i}(t_i)
		\end{equation*}
		\item 特征函数与概率分布之间存在一个双射,即$\varphi_X(t)=\varphi_Y(t)$当且仅当$X$与$Y$具有相同的概率分布。
		\item 若$\operatorname{E}(X^n)$存在,则$\varphi_X^{(n)}(t)$存在,且对$1\leqslant k\leqslant n$有:
		\begin{equation*}
			\operatorname{E}(X^k)=i^{-k}\varphi_X^{(k)}(0)
		\end{equation*}
		特别的:
		\begin{equation*}
			\operatorname{E}(X)=-i\varphi_X'(0),\;
			\operatorname{Var}(X)=-\varphi_X''(0)+[\varphi_X'(0)]^2
		\end{equation*}
		\item 若$\varphi_X(t)$在$t=0$处最高有$n$阶导数,如果$n$为奇数,则$X$具有所有不超过$n-1$阶的原点矩;若$n$为偶数,则$X$具有所有不超过$n$阶的原点矩;
		\item $\varphi_X(t)$在$\mathbb{R}$上一致连续;
		\item $\varphi_X(t)$是半正定的,即对任意的$n\in\mathbb{N}^+$及任意的$t=(t_1,t_2,\dots,t_n)^T\in\mathbb{R}^{n}$和任意的$c=(c_1,c_2,\dots,c_n)^T\in\mathbb{C}^{n}$,令$A=[\varphi_X(t_i-t_j)]\in M_{n}(\mathbb{C})$,则有:
		\begin{equation*}
			c^TA\overline{c}=\sum_{i=1}^{n}\sum_{j=1}^{n}c_i\overline{c_j}\varphi_X(t_i-t_j)\geqslant0
		\end{equation*}
	\end{enumerate}
\end{property}
\begin{proof}
	(1)因为:
	\begin{equation*}
		e^{itX}=\cos(tX)+i\sin(tX)
	\end{equation*}
	所以$|e^{itX}|=1$,于是:\info{链接Lebesgue积分性质}
	\begin{equation*}
		\Bigl|\operatorname{E}(e^{itX})\Bigr|=\Bigl|\int_{-\infty}^{+\infty}e^{itx}p(x)\dif x\Bigr|\leqslant\int_{-\infty}^{+\infty}|e^{itx}|p(x)\dif x=\int_{-\infty}^{+\infty}p(x)\dif x=1
	\end{equation*}
	所以$\varphi_X(t)$存在。\par
	(2)可以发现:
	\begin{equation*}
		\varphi_X(0)=\int_{-\infty}^{+\infty}p(x)\dif x=1
	\end{equation*}
	再由(1)的证明过程即可得出结论。\par
	(3)因为:
	\begin{equation*}
		\varphi_X(t)=\operatorname{E}(e^{itX})=\operatorname{E}[\cos(tX)+i\sin(tX)]=\operatorname{E}[\cos(tX)]+i\operatorname{E}[\sin(tX)]
	\end{equation*}
	所以:
	\begin{equation*}
		\overline{\varphi_X(t)}=\operatorname{E}[\cos(tX)]-i\operatorname{E}[\sin(tX)]=\operatorname{E}[\cos(-tX)]+i\operatorname{E}[\sin(-tX)]=\varphi_X(-t)
	\end{equation*}\par
	(4)因为$X_k$相互独立,所以$e^{it(\alpha_kX_k+\beta_k)}$之间也相互独立,$k=1,2,\dots,n$,于是有:
	\begin{align*}
		\varphi_Y(t)
		&=\operatorname{E}\left[\exp\left(it\sum_{k=1}^{n}(\alpha_kX_k+\beta_k)\right)\right]
		=\operatorname{E}\left(\prod_{k=1}^ne^{it(\alpha_kX_k+\beta_k)}\right) \\
		&=\prod_{k=1}^n\operatorname{E}[e^{it(\alpha_kX_k+\beta_k)}]
		=\prod_{k=1}^ne^{it\beta_k}\operatorname{E}(e^{it\alpha_kX_k})
		=\prod_{k=1}^ne^{it\beta_k}\varphi_{X_k}(\alpha_kt)
	\end{align*}\par
	(5)\textbf{必要性:}因为$X_k$相互独立,所以$e^{it_kX_k}$相互独立,$k=1,2,\dots,n$。由随机向量特征函数的定义可得:
	\begin{align*}
		\varphi_{X_1,\dots,X_n}(t_1,t_2,\dots,t_n)
		&=\operatorname{E}\left[\exp\left(i\sum_{k=1}^{n}t_kX_k\right)\right]
		=\operatorname{E}\left(\prod_{k=1}^ne^{it_kX_k}\right) \\
		&=\prod_{k=1}^n\operatorname{E}(e^{it_kX_k})
		=\prod_{k=1}^n\varphi_{X_k}(t_k)
	\end{align*}
	\textbf{充分性:}因为:
	\begin{gather*}
		\begin{aligned}
			\varphi_{X_1,\dots,X_n}(t_1,t_2,\dots,t_n)
			&=\operatorname{E}\left[\exp\left(i\sum_{k=1}^{n}t_kX_k\right)\right] \\
			&=\int_{-\infty}^{+\infty}\cdots\int_{-\infty}^{+\infty}\exp\left(i\sum_{k=1}^{n}t_kx_k\right)p(x_1,\dots,x_n)\dif x_1\cdots\dif x_n
		\end{aligned} \\
		\begin{aligned}
			\prod_{i=1}^n\varphi_{X_i}(t_i)
			&=\prod_{k=1}^n\operatorname{E}(e^{it_kX_k}) \\
			&=\prod_{k=1}^n\int_{-\infty}^{+\infty}e^{it_kx_k}p(x_k)\dif x_k \\
			&=\int_{-\infty}^{+\infty}\cdots\int_{-\infty}^{+\infty}\exp\left(i\sum_{k=1}^{n}t_kx_k\right)p(x_1)p(x_2)\cdots p(x_n)\dif x_1\dif x_2\cdots\dif x_n
		\end{aligned}
	\end{gather*}
	若两式相等,则有:
	\begin{equation*}
		p(x_1,x_2,\dots,x_n)=p(x_1)p(x_2)\cdots p(x_n)
	\end{equation*}
	由\info{链接独立性条件}可得$X_k,\;k=1,2,\dots,n$相互独立。\par
	(6)\par
	(7)因为$\operatorname{E}(X^n)$存在,所以:
	\begin{equation*}
		\int_{-\infty}^{+\infty}|x|^np(x)\dif x<+\infty
	\end{equation*}
	于是:
	\begin{equation*}
		\left|\int_{-\infty}^{+\infty}i^nx^ne^{itx}p(x)\dif x\right|\leqslant\int_{-\infty}^{+\infty}|x|^np(x)\dif x<+\infty
	\end{equation*}
	所以:
	\begin{equation*}
		\varphi_X^{(n)}(t)=\int_{-\infty}^{+\infty}i^nx^ne^{itx}p(x)\dif x
	\end{equation*}
	存在。由\cref{prop:Moment}(1)可知对$1\leqslant k\leqslant n$有$\operatorname{E}(X^k)$存在,于是:
	\begin{equation*}
		\varphi_X^{(k)}(0)=\int_{-\infty}^{+\infty}i^kx^kp(x)\dif x=i^k\int_{-\infty}^{+\infty}x^kp(x)\dif x=i^k\operatorname{E}(X^k)
	\end{equation*}
	也存在。\par
	(8)注意到:
	\begin{equation*}
		\varphi_X^{(n)}(t)=\int_{-\infty}^{+\infty}i^nx^ne^{itx}p(x)\dif x
	\end{equation*}
	因为$\varphi_X(t)$在$t=0$处最高具有$n$阶导数,于是:
	\begin{equation*}
		|\varphi_X^{(n)}(0)|=\left|\int_{-\infty}^{+\infty}i^nx^np(x)\dif x\right|=\left|\int_{-\infty}^{+\infty}x^{n}p(x)\dif x\right|<+\infty
	\end{equation*}
	当$n=2k+1,\;k\in\mathbb{N}$时,有:
	\begin{equation*}
		\int_{-\infty}^{+\infty}|x|^{n}p(x)\dif x>|\varphi_X^{(n)}(0)|=\Bigl|\int_{-\infty}^{+\infty}x^np(x)\dif x\Bigr|
	\end{equation*}
	所以$\operatorname{E}(X^n)$不一定存在。\info{需要证明对小于的都存在}
	当$n=2k,\;k\in\mathbb{N}^+$时,有:
	\begin{equation*}
		|\varphi_X^{(n)}(0)|=\left|\int_{-\infty}^{+\infty}x^{n}p(x)\dif x\right|=\int_{-\infty}^{+\infty}|x|^np(x)\dif x<+\infty
	\end{equation*}
	存在,于是$\operatorname{E}(X^n)$存在。由\cref{prop:Moment}(1)可知,此时$X$具有所有不超过$n$阶的原点矩。\par
	(9)对任意的$t,h\in \mathbb{R}$和$a>0$,有:
	\begin{align*}
		|\varphi(t+h)-\varphi(t)|
		&=\left|\int_{-\infty}^{+\infty}[e^{i(t+h)x}-e^{itx}]p(x)\dif x\right| \\
		&=\left|\int_{-\infty}^{+\infty}(e^{ihx}-1)e^{itx}p(x)\dif x\right| \\
		&\leqslant\int_{-\infty}^{+\infty}|(e^{ihx}-1)e^{itx}|p(x)\dif x \\
		&=\int_{-\infty}^{+\infty}|e^{ihx}-1||e^{itx}|p(x)\dif x \\
		&=\int_{-\infty}^{+\infty}|e^{ihx}-1|p(x)\dif x \\
		&=\int_{-a}^{a}|e^{ihx}-1|p(x)\dif x+\int_{|x|\geqslant a}|e^{ihx}-1|p(x)\dif x \\
		&\leqslant\int_{-a}^{a}|e^{ihx}-1|p(x)\dif x+\int_{|x|\geqslant a}(|e^{ihx}|+1)p(x)\dif x \\
		&=\int_{-a}^{a}|e^{ihx}-1|p(x)\dif x+2\int_{|x|\geqslant a}p(x)\dif x
	\end{align*}
	对于任意的$\varepsilon>0$,可以先选定一个充分大的$a$,使得:
	\begin{equation*}
		2\int_{|x|\geqslant a}p(x)\dif x<\frac{\varepsilon}{2}
	\end{equation*}
	对任意的$x\in[-a,a]$,只要取$\delta=\dfrac{\varepsilon}{2a}$,则当$|h|<\delta$时,就有:
	\begin{align*}
		|e^{ihx}-1|
		&=\Bigl|e^{ihx}-e^{i\frac{hx}{2}}e^{i\frac{-hx}{2}}\Bigr|=\Bigl|e^{i\frac{hx}{2}}(e^{i\frac{hx}{2}}-e^{i\frac{-hx}{2}})\Bigr| \\
		&=\Bigl|e^{i\frac{hx}{2}}\Bigr|\;\Bigl|e^{i\frac{hx}{2}}-e^{i\frac{-hx}{2}}\Bigr| \\
		&=\Bigl|e^{i\frac{hx}{2}}-e^{i\frac{-hx}{2}}\Bigr| \\
		&=\Bigl|\cos\frac{hx}{2}+i\sin\frac{hx}{2}-\cos\frac{-hx}{2}-i\sin\frac{-hx}{2}\Bigr| \\
		&=\Bigl|2i\sin\frac{hx}{2}\Bigr|
		=2\Bigl|\sin\frac{hx}{2}\Bigr|\leqslant2\Bigl|\frac{hx}{2}\Bigr|\leqslant ha<\frac{\varepsilon}{2}
	\end{align*}
	于是对任意的$t\in\mathbb{R}$,有:
	\begin{equation*}
		|\varphi(t+h)-\varphi(t)|<\int_{-a}^{a}\frac{\varepsilon}{2}p(x)\dif x+2\int_{|x|\geqslant a}p(x)\dif x<\frac{\varepsilon}{2}\int_{-\infty}^{+\infty}p(x)\dif x+\frac{\varepsilon}{2}=\varepsilon
	\end{equation*}
	即$\varphi_X(t)$在$\mathbb{R}$上一致连续。\par
	(10)显然:
	\begin{align*}
		\sum_{i=1}^{n}\sum_{j=1}^{n}c_i\overline{c}_j\varphi_X(t_i-t_j)
		&=\sum_{k=1}^{n}\sum_{j=1}^{n}c_k\overline{c}_j\int_{-\infty}^{+\infty}e^{i(t_k-t_j)x}p(x)\dif x \\
		&=\int_{-\infty}^{+\infty}\sum_{k=1}^{n}\sum_{j=1}^{n}c_k\overline{c}_je^{i(t_k-t_j)x}p(x)\dif x \\
		&=\int_{-\infty}^{+\infty}\left(\sum_{k=1}^{n}c_ke^{it_kx}\right)\left(\sum_{j=1}^{n}\overline{c}_je^{-it_jx}\right)p(x) \dif x \\
		&=\int_{-\infty}^{+\infty}\left(\sum_{k=1}^{n}c_ke^{it_kx}\right)\left(\sum_{j=1}^{n}\overline{c_ke^{it_kx}}\right)p(x) \dif x \\
		&=\int_{-\infty}^{+\infty}\Bigl|\sum_{k=1}^{n}c_ke^{it_kx}\Bigr|^2p(x) \dif x\qedhere
	\end{align*}
\end{proof}
\section{Fisher信息量}

\begin{definition}
	设$\mathbf{X}$是测度空间$(X,\mathcal{F},\mu)$上的一个随机向量,其分布由$n$维参数$\theta=(\seq{\theta}{n})$决定,$\mathbf{X}$的概率函数为$f(\mathbf{X};\theta)$。若$f(\mathbf{X};\theta)$满足如下正则条件:
	\begin{enumerate}
		\item $f(\mathbf{X};\theta)$关于$\theta$的偏导数a.e.存在;
		\item 对$f(\mathbf{X};\theta)$在$X$上的积分关于$\theta$任一分量求导时都可以交换求导与积分的顺序;
		\item $f(\mathbf{X};\theta)$的定义域与$\theta$无关。
	\end{enumerate}
	则称:
	\begin{equation*}
		[I(\theta)]_{ij}=\operatorname{E}\left[\left(\frac{\partial\ln f(\mathbf{X};\theta)}{\partial\theta_i}\right)\left(\frac{\partial\ln f(\mathbf{X};\theta)}{\partial\theta_j}\right)\right]
	\end{equation*}
	为$\mathbf{X}$的\gls{FIM}。
\end{definition}
\begin{note}
	Fisher信息量(即一维情况的信息矩阵)用来表明随机变量$\mathbf{X}$携带的关于参数$\theta$的信息。如果它比较大,表示平均下来$\theta$的微小变化会给$\mathbf{X}$的分布带来较大的变化,即$\mathbf{X}$的分布很依赖$\theta$的具体取值,所以携带了较多关于$\theta$的信息。
\end{note}
\begin{property}\label{prop:FIM}
	设$\mathbf{X}$是测度空间$(X,\mathcal{F},\mu)$上的一个随机向量,其Fisher信息矩阵具有如下性质:
	\begin{enumerate}
		\item Fisher信息矩阵可以看作协方差矩阵:
		\begin{equation*}
			I(\theta)=\operatorname{Cov}\left[\frac{\partial\ln f(\mathbf{X};\theta)}{\partial\theta}\right]
		\end{equation*}
		\item 若$\ln f(\mathbf{X};\theta)$有关于$\theta$的所有二阶导数,且对该二阶导数在X上的积分关于$\theta$任一分量求导时都可以交换求导与积分的顺序,则:
		\begin{equation*}
			[I(\theta)]_{ij}=-\operatorname{E}\left[\frac{\partial^2\ln f(\mathbf{X};\theta)}{\partial\theta_i\partial\theta_j}\right]
		\end{equation*}
	\end{enumerate}
\end{property}
\begin{proof}
	(1)由正则条件可得:
	\begin{align*}
		\operatorname{E}\left[\frac{\partial\ln f(\mathbf{X};\theta)}{\partial\theta_i}\right]
		&=\int_{X}\frac{\dfrac{\partial f(\mathbf{X};\theta)}{\partial\theta_i}}{f(\mathbf{X};\theta)}f(\mathbf{X};\theta)\dif\mu
		=\int_{X}\frac{\partial f(\mathbf{X};\theta)}{\partial\theta_i}\dif\mu \\
		&=\frac{\partial}{\partial\theta_i}\int_{X}f(\mathbf{X};\theta)\dif\mu
		=\frac{\partial1}{\partial\theta_i}=0
	\end{align*}\par
	(2)由(1)和正则条件可得:
	\begin{gather*}
		\frac{\partial}{\partial\theta_j}\operatorname{E}\left[\frac{\partial\ln f(\mathbf{X};\theta)}{\partial\theta_i}\right]=0 \\
		\frac{\partial}{\partial\theta_j}\int_{X}\frac{\partial\ln f(\mathbf{X};\theta)}{\partial\theta_i}f(\mathbf{X};\theta)\dif\mu=0 \\
		\int_{X}\frac{\partial}{\partial\theta_j}\left[\frac{\partial\ln f(\mathbf{X};\theta)}{\partial\theta_i}f(\mathbf{X};\theta)\right]\dif\mu=0 \\
		\int_{X}\left[\frac{\partial^2\ln f(\mathbf{X};\theta)}{\partial\theta_i\partial\theta_j}f(\mathbf{X};\theta)+\frac{\partial\ln f(\mathbf{X};\theta)}{\partial\theta_i}\frac{\partial f(\mathbf{X};\theta)}{\partial\theta_j}\right]\dif\mu=0 \\
		\int_{X}\left[\frac{\partial^2\ln f(\mathbf{X};\theta)}{\partial\theta_i\partial\theta_j}f(\mathbf{X};\theta)+\frac{\partial\ln f(\mathbf{X};\theta)}{\partial\theta_i}\frac{\partial \ln f(\mathbf{X};\theta)}{\partial\theta_j}f(\mathbf{X};\theta)\right]\dif\mu=0 \\
		\operatorname{E}\left[\left(\frac{\partial\ln f(\mathbf{X};\theta)}{\partial\theta_i}\right)\left(\frac{\partial\ln f(\mathbf{X};\theta)}{\partial\theta_j}\right)\right]=-\operatorname{E}\left[\frac{\partial^2\ln f(\mathbf{X};\theta)}{\partial\theta_i\partial\theta_j}\right]
	\end{gather*}
	其中倒数第三行到倒数第二行是因为:
	\begin{equation*}
		\frac{\partial\ln f(\mathbf{X};\theta)}{\partial\theta_i}f(\mathbf{X};\theta)=\frac{\dfrac{\partial f(\mathbf{X};\theta)}{\partial\theta_i}}{f(\mathbf{X};\theta)}f(\mathbf{X};\theta)=\dfrac{\partial f(\mathbf{X};\theta)}{\partial\theta_i}
	\end{equation*}
\end{proof}
\section{统计距离}

\subsection{Mahalanobis距离}
\begin{definition}
	设$x,y$是均值为$\boldsymbol{\mu}$和协方差矩阵为$\Sigma$的随机向量$\mathbf{X}$的两个实现值,$\mathbf{X}$的分布记为$D$。称:
	\begin{equation*}
		d^2_m(x,D)=(x-\boldsymbol{\mu})^T\Sigma^{-1}(x-\boldsymbol{\mu})
	\end{equation*}
	为$x$与$D$之间的\textbf{Mahalanobis距离}。称:
	\begin{equation*}
		d^2_m(x,y)=(x-y)^T\Sigma^{-1}(x-y)
	\end{equation*}
	为$x$与$y$之间的Mahalanobis距离。
\end{definition}