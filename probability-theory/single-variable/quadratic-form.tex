\section{二次型}

\begin{definition}
	$\mathbf{X}$是一个$n$维随机向量,$A=(a_{ij})$为$n$阶非随机实对称阵,则随机变量:
	\begin{equation*}
		\mathbf{X}^TA\mathbf{X}=\sum_{i=1}^{n}\sum_{j=1}^{n}a_{ij}\mathbf{X}_i\mathbf{X}_j
	\end{equation*}
	称为$\mathbf{X}$的二次型。
\end{definition}
\subsubsection{随机变量二次型的均值}
\begin{theorem}\label{theo:ERVQuadraticForm}
	$\mathbf{X}$是一个$n$维随机向量,$\operatorname{E}(\mathbf{X})=\mu,\;\operatorname{Cov}(\mathbf{X})=\Sigma$,则:
	\begin{equation*}
		\operatorname{E}(\mathbf{X}^TA\mathbf{X})=\mu^TA\mu+\operatorname{tr}(A\Sigma)
	\end{equation*}	
\end{theorem}
\begin{proof}
	\begin{align*}
		\operatorname{E}(\mathbf{X}^TA\mathbf{X})
		&=\operatorname{E}[(\mathbf{X}-\mu+\mu)^TA(\mathbf{X}-\mu+\mu)] \\
		&=\operatorname{E}[(\mathbf{X}-\mu)^TA(\mathbf{X}-\mu)]+\operatorname{E}[(\mathbf{X}-\mu)^TA\mu]+\operatorname{E}[\mu^TA(\mathbf{X}-\mu)]+\operatorname{E}(\mu^TA\mu) \\
		&=\operatorname{E}\{\operatorname{tr}[(\mathbf{X}-\mu)^TA(\mathbf{X}-\mu)]\}+\mu^TA\mu \\
		&=\operatorname{E}\{\operatorname{tr}[A(\mathbf{X}-\mu)(\mathbf{X}-\mu)^T]\}+\mu^TA\mu \\
		&=\operatorname{tr}\operatorname{E}[A(\mathbf{X}-\mu)(\mathbf{X}-\mu)^T]+\mu^TA\mu \\
		&=\operatorname{tr}\{A\operatorname{E}[(\mathbf{X}-\mu)(\mathbf{X}-\mu)^T]\}+\mu^TA\mu \\
		&=\operatorname{tr}(A\Sigma)+\mu^TA\mu
	\end{align*}
	第二行到第三行利用到了$\operatorname{E}(\mathbf{X})=\mu$以及$(\mathbf{X}-\mu)^TA(\mathbf{X}-\mu)=\operatorname{tr}[(\mathbf{X}-\mu)^TA(\mathbf{X}-\mu)]$,后式成立是因为$(\mathbf{X}-\mu)^TA(\mathbf{X}-\mu)$是一个标量,标量的迹自然等于自身。第三行到第四行使用到了\cref{prop:Trace}(3)。
\end{proof}
\subsubsection{独立随机变量二次型的方差}
\begin{theorem}\label{theo:VRVQuadraticForm}
	设随机变量$X_i,\;i=1,2,\dots,n$相互独立,$\operatorname{E}(X_i)=\mu_i,\;\operatorname{Var}(X_i)=\sigma^2,\;\nu_k^{(i)}=\operatorname{E}[(X_i-\mu_i)^k]$,$\mathbf{X}=(\seq{X}{n})^T,\;\mu=(\seq{\mu}{n})^T$,$A=(a_{ij})$为$n$阶非随机实对称阵,$a=(a_{11},a_{22},\dots,a_{nn})^T$,$b=(\nu_3^{(1)}a_{11},\nu_3^{(2)}a_{22},\dots,\nu_3^{(n)}a_{nn})^T$,则:
	\begin{equation*}
		\operatorname{Var}(\mathbf{X}^TA\mathbf{X})=\sum_{i=1}^{n}a_{ii}^2\nu_4^{(i)}+\sigma^4[2\operatorname{tr}(A^2)-3a^Ta]+4\sigma^2\mu^TA^2\mu+4\mu^TAb
	\end{equation*}
\end{theorem}
\begin{proof}
	由\cref{prop:Variance}(1)可得:
	\begin{equation*}
		\operatorname{Var}(\mathbf{X}^TA\mathbf{X})=\operatorname{E}[(\mathbf{X}^TA\mathbf{X})^2]-[\operatorname{E}(\mathbf{X}^TA\mathbf{X})]^2 
	\end{equation*}
	由题设可知:
	\begin{equation*}
		\operatorname{E}(\mathbf{X})=\mu,\;\operatorname{Var}(\mathbf{X})=\sigma^2I
	\end{equation*}
	根据\cref{theo:ERVQuadraticForm}可得:
	\begin{align*}
		[\operatorname{E}(\mathbf{X}^TA\mathbf{X})]^2&=[\operatorname{tr}(A\sigma^2I)+\mu^TA\mu]^2=[\sigma^2\operatorname{tr}(A)+\mu^TA\mu]^2 \\
		&=\sigma^4[\operatorname{tr}(A)]^2+2\sigma^2\operatorname{tr}(A)\mu^TA\mu+(\mu^TA\mu)^2
	\end{align*}
	同时:
	\begin{align*}
		(\mathbf{X}^TA\mathbf{X})^2
		&=[(\mathbf{X}-\mu+\mu)^TA(\mathbf{X}-\mu+\mu)]^2 \\
		&=[(\mathbf{X}-\mu)^TA(\mathbf{X}-\mu)+2\mu^TA(\mathbf{X}-\mu)+\mu^TA\mu]^2 \\
		&=[(\mathbf{X}-\mu)^TA(\mathbf{X}-\mu)]^2+4[\mu^TA(\mathbf{X}-\mu)]^2+(\mu^TA\mu)^2 \\
		&\quad+4(\mathbf{X}-\mu)^TA(\mathbf{X}-\mu)\mu^TA(\mathbf{X}-\mu)+2(\mathbf{X}-\mu)^TA(\mathbf{X}-\mu)\mu^TA\mu \\
		&\quad+4\mu^TA(\mathbf{X}-\mu)\mu^TA\mu
	\end{align*}
	令$\mathbf{Y}=\mathbf{X}-\mu$,则有$\operatorname{E}(\mathbf{Y})=\mathbf{0}$,再由\cref{theo:ERVQuadraticForm}可得:
	\begin{align*}
		\operatorname{E}[(\mathbf{X}^TA\mathbf{X})^2]
		&=\operatorname{E}[(\mathbf{Y}^TA\mathbf{Y})^2]+4\operatorname{E}[(\mu^TA\mathbf{Y})^2]+(\mu^TA\mu)^2 \\
		&\quad+4\operatorname{E}(\mathbf{Y}^TA\mathbf{Y}\mu^TA\mathbf{Y})+2\mu^TA\mu\sigma^2\operatorname{tr}(A)
	\end{align*}
	考虑:
	\begin{align*}
		\operatorname{E}[(\mathbf{Y}^TA\mathbf{Y})^2]
		&=\operatorname{E}\left(\sum_{i=1}^{n}\sum_{j=1}^{n}\sum_{k=1}^{n}\sum_{l=1}^{n}a_{ij}a_{kl}\mathbf{Y}_i\mathbf{Y}_j\mathbf{Y}_k\mathbf{Y}_l\right) \\
		&=\sum_{i=1}^{n}\sum_{j=1}^{n}\sum_{k=1}^{n}\sum_{l=1}^{n}a_{ij}a_{kl}\operatorname{E}(\mathbf{Y}_i\mathbf{Y}_j\mathbf{Y}_k\mathbf{Y}_l)
	\end{align*}
	作分类讨论:
	\begin{enumerate}
		\item $i,j,k,l$互不相同,则$\operatorname{E}(\mathbf{Y}_i\mathbf{Y}_j\mathbf{Y}_k\mathbf{Y}_l)=E(\mathbf{Y}_i)E(\mathbf{Y}_j)E(\mathbf{Y}_k)E(\mathbf{Y}_l)=0$;
		\item $i,j,k,l$中存在某两个值相同:
		\begin{itemize}
			\item 此时另外两个不同,则$\operatorname{E}(\mathbf{Y}_i\mathbf{Y}_j\mathbf{Y}_k\mathbf{Y}_l)=0$;
			\item 此时另外两个也相同(即$i=j,k=l$或$i=k,j=l$或$i=l,j=k$),则$\operatorname{E}(\mathbf{Y}_i\mathbf{Y}_j\mathbf{Y}_k\mathbf{Y}_l)=\sigma^4$。
		\end{itemize}
		\item $i,j,k,l$中存在某三个值相同,则$\operatorname{E}(\mathbf{Y}_i\mathbf{Y}_j\mathbf{Y}_k\mathbf{Y}_l)=0$;
		\item $i,j,k,l$相同,则$\operatorname{E}(\mathbf{Y}_i\mathbf{Y}_j\mathbf{Y}_k\mathbf{Y}_l)=\nu_4^{(i)}$。
	\end{enumerate}
	于是:
	\begin{align*}
		\operatorname{E}[(\mathbf{Y}^TA\mathbf{Y})^2]
		&=\sum_{i=1}^{n}\sum_{j=1}^{n}\sum_{k=1}^{n}\sum_{l=1}^{n}a_{ij}a_{kl}\operatorname{E}(\mathbf{Y}_i\mathbf{Y}_j\mathbf{Y}_k\mathbf{Y}_l) \\
		&=\sum_{i=1}^{n}a_{ii}^2\nu_4^{(i)}+\sigma^4\left(\sum_{i\ne k}a_{ii}a_{kk}+\sum_{i\ne j}a_{ij}^2+\sum_{i\ne j}a_{ij}a_{ji}\right) \\
		&=\sum_{i=1}^{n}a_{ii}^2\nu_4^{(i)}+\sigma^4\left(\sum_{i\ne k}a_{ii}a_{kk}+2\sum_{i\ne j}a_{ij}^2\right)
	\end{align*}
	因为:
	\begin{gather*}
		\sum_{i\ne k}a_{ii}a_{kk}=[\operatorname{tr}(A)]^2-a^Ta \\
		\sum_{i\ne j}a_{ij}^2=\operatorname{tr}(AA^T)-a^Ta=\operatorname{tr}(A^2)-a^Ta
	\end{gather*}
	所以:
	\begin{equation*}
		\operatorname{E}[(\mathbf{Y}^TA\mathbf{Y})^2]=\sum_{i=1}^{n}a_{ii}^2\nu_4^{(i)}+\sigma^4\{[\operatorname{tr}(A)]^2+2\operatorname{tr}(A^2)-3a^Ta\}
	\end{equation*}
	由\cref{theo:ERVQuadraticForm}和\cref{prop:Trace}(3)可得:
	\begin{align*}
		\operatorname{E}[(\mu^TA\mathbf{Y})^2]
		&=\operatorname{E}(\mu^TA\mathbf{Y}\mu^TA\mathbf{Y})
		=\operatorname{E}(\mathbf{Y}^TA\mu\mu^TA\mathbf{Y})
		=\operatorname{tr}(A\mu\mu^TA\sigma^2I) \\
		&=\sigma^2\operatorname{tr}(A\mu\mu^TA)
		=\sigma^2\operatorname{tr}(\mu^TA^2\mu)
		=\sigma^2\mu^TA^2\mu
	\end{align*}
	注意到:
	\begin{align*}
		\operatorname{E}(\mathbf{Y}^TA\mathbf{Y}\mu^TA\mathbf{Y})
		&=\operatorname{E}\left(\sum_{i=1}^{n}\sum_{j=1}^{n}a_{ij}\mathbf{Y}_i\mathbf{Y}_j\sum_{k=1}^{n}\sum_{l=1}^{n}a_{kl}\mu_k\mathbf{Y}_l\right) \\
		&=\operatorname{E}\left(\sum_{i=1}^{n}\sum_{j=1}^{n}\sum_{k=1}^{n}\sum_{l=1}^{n}a_{ij}a_{kl}\mu_k\mathbf{Y}_i\mathbf{Y}_j\mathbf{Y}_l\right) \\
		&=\sum_{i=1}^{n}\sum_{j=1}^{n}\sum_{k=1}^{n}\sum_{l=1}^{n}a_{ij}a_{kl}\mu_k\operatorname{E}(\mathbf{Y}_i\mathbf{Y}_j\mathbf{Y}_l)
	\end{align*}
	和之前的讨论类似,可以得到:
	\begin{equation*}
		\operatorname{E}(\mathbf{Y}_i\mathbf{Y}_j\mathbf{Y}_l)=
		\begin{cases}
			\nu_3^{(i)},\;&i=j=l \\
			0,\;&\text{其他情况}
		\end{cases}
	\end{equation*}
	于是有:
	\begin{equation*}
		\operatorname{E}(\mathbf{Y}^TA\mathbf{Y}\mu^TA\mathbf{Y})
		=\sum_{i=1}^{n}\sum_{k=1}^{n}a_{ii}\nu_3^{(i)}a_{ki}\mu_k
	\end{equation*}
	令$b=(\nu_3^{(1)}a_{11},\nu_3^{(2)}a_{22},\dots,\nu_3^{(n)}a_{nn})^T$,则:
	\begin{equation*}
		\operatorname{E}(\mathbf{Y}^TA\mathbf{Y}\mu^TA\mathbf{Y})
		=\sum_{i=1}^{n}\sum_{k=1}^{n}a_{ii}\nu_3^{(i)}a_{ki}\mu_k=\mu^TAb
	\end{equation*}
	将以上求得的期望值全部代入,即可得到:
	\begin{align*}
		\operatorname{E}[(\mathbf{X}^TA\mathbf{X})^2]
		&=\operatorname{E}[(\mathbf{Y}^TA\mathbf{Y})^2]+4\operatorname{E}[(\mu^TA\mathbf{Y})^2]+(\mu^TA\mu)^2 \\
		&\quad+4\operatorname{E}(\mathbf{Y}^TA\mathbf{Y}\mu^TA\mathbf{Y})+2\mu^TA\mu\sigma^2\operatorname{tr}(A) \\
		&=\sum_{i=1}^{n}a_{ii}^2\nu_4^{(i)}+\sigma^4\{[\operatorname{tr}(A)]^2+2\operatorname{tr}(A^2)-3a^Ta\} \\
		&\quad+4\sigma^2\mu^TA^2\mu+(\mu^TA\mu)^2+4\mu^TAb+2\mu^TA\mu\sigma^2\operatorname{tr}(A)
	\end{align*}
	于是:
	\begin{align*}
		\operatorname{Var}(\mathbf{X}^TA\mathbf{X})
		&=\operatorname{E}[(\mathbf{X}^TA\mathbf{X})^2]-[\operatorname{E}(\mathbf{X}^TA\mathbf{X})]^2 \\
		&=\sum_{i=1}^{n}a_{ii}^2\nu_4^{(i)}+\sigma^4\{[\operatorname{tr}(A)]^2+2\operatorname{tr}(A^2)-3a^Ta\} \\
		&\quad+4\sigma^2\mu^TA^2\mu+(\mu^TA\mu)^2+4\mu^TAb+2\mu^TA\mu\sigma^2\operatorname{tr}(A) \\
		&\quad-\sigma^4[\operatorname{tr}(A)]^2-2\sigma^2\operatorname{tr}(A)\mu^TA\mu-(\mu^TA\mu)^2 \\
		&=\sum_{i=1}^{n}a_{ii}^2\nu_4^{(i)}+\sigma^4[2\operatorname{tr}(A^2)-3a^Ta]+4\sigma^2\mu^TA^2\mu+4\mu^TAb\qedhere
	\end{align*}
\end{proof}









