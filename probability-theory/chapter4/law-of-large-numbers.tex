\section{大数定律}

\subsection{弱大数定律}
\begin{definition}
	设$\{X_n\}$是一个随机变量序列,若
	\begin{equation*}
		\frac{1}{n}\sum_{i=1}^{n}X_i\overset{P}{\longrightarrow}\frac{1}{n}\sum_{i=1}^{n}\operatorname{E}(X_i)
	\end{equation*}
	则称$\{X_n\}$服从\gls{WeakLawOfLargeNumbers}。
\end{definition}
\begin{theorem}[Markov's Weak Law of Large Numbers]
	设$\{X_n\}$是一个随机变量序列,若:
	\begin{equation*}
		\lim_{n\to+\infty}\left[\frac{1}{n^2}\operatorname{Var}\left(\sum_{i=1}^{n}X_i\right)\right]=0
	\end{equation*}
	则$\{X_n\}$服从弱大数定律。
\end{theorem}
\begin{proof}
	由\cref{ineq:Chebyshev}可知对任意的$\varepsilon>0$有:
	\begin{equation*}
		P\left(\left|\frac{1}{n}\sum_{i=1}^{n}X_i-\frac{1}{n}\sum_{i=1}^{n}\operatorname{E}(X_i)\right|\geqslant\varepsilon\right)\leqslant\frac{\operatorname{Var}\left(\dfrac{1}{n}\sum\limits_{i=1}^{n}X_i\right)}{\varepsilon^2}=\frac{1}{n^2\varepsilon^2}\operatorname{Var}\left(\sum_{i=1}^{n}X_i\right)\to0\qedhere
	\end{equation*}
\end{proof}
\begin{corollary}
	由Markov's Weak Law of Large Numbers可以推出:
	\begin{enumerate}
		\item (Chebyshev's Weak Law of Large Numbers)设$\{X_n\}$为一列互不相关的随机变量,若对于任意的$n\in\mathbb{N}^+$,有$\operatorname{Var}(X_i)\leqslant c<+\infty$,则$\{X_n\}$服从弱大数定律。
		\item (Bernoulli's Weak Law of Large Numbers)设$\{X_n\}$为一列独立同两点分布$b(1,p)$的随机变量,则$\{X_n\}$服从弱大数定律。
	\end{enumerate}
\end{corollary}
\begin{proof}
	(1)因为$\{X_n\}$互不相关,所以\info{以后写不相关的和的方差,需要注意从乘积期望开始写,还得链接方差线性变换的性质}:
	\begin{equation*}
		\operatorname{Var}\left(\frac{1}{n}\sum_{i=1}^{n}X_i\right)=\frac{1}{n^2}\sum_{i=1}^{n}\operatorname{Var}(X_i)\leqslant\frac{1}{n^2}nc=\frac{c}{n}\to0
	\end{equation*}\par
	(2)由两点分布的性质或者(1)直接可得。
\end{proof}
%\begin{theorem}[khinchin's Weak Law of Large Numbers]\label{theo:WeakLawOfLargeNumbers}
%%	设$\{Xm_n\}$是一列独立同分布的随机变量,若$X_n$的数学期望有限,则$\{X_n\}$服从大数定律。
%\end{theorem}
\begin{proof}
	\info{需要写完特征函数}
\end{proof}

\begin{theorem}[Strong Law of Large Numbers]
	\label{theo:StrongLawOfLargeNumbers}
	设$\{X_n\}$为一列独立同分布的随机变量,服从的分布的均值为$\mu$且满足$\operatorname{E}(|X_i|)<+\infty$,则有:
	\begin{equation*}
		P\left(\lim_{n\to+\infty}\bar{X}_n\ne\mu\right)=0
	\end{equation*}
	即$\bar{X}_n\overset{\text{a.e.}}{\longrightarrow}\mu$。
\end{theorem}
