\section{随机变量的分类}

设$f$是概率空间$(X,\mathscr{F},P)$上的随机变量。由\cref{prop:MeasurableMapping}(4)可知$(\mathbb{R}^{},\mathcal{B},Pf^{-1})$是一个概率空间。因为$Pf^{-1}$是概率测度,所以$Pf^{-1}$是$\sigma$有限测度,而由\cref{prop:LSMeasure}(1)可知L测度$\lambda$也是$\sigma$有限测度,根据\cref{theo:LebesgueDecomposition}可知$Pf^{-1}$有如下分解式:
\begin{equation*}
	Pf^{-1}=\mu_1+\mu_s,\quad\mu_1\ll\lambda,\quad\mu_s\perp\lambda
\end{equation*}
其中$\mu_1$和$\mu_s$是$\sigma$有限的符号测度。由\cref{prop:BorelSigmaField}(1)可知$\mathbb{R}^{}$上的单点集都在$\mathcal{B}$中,于是记:
\begin{gather*}
	D=\{x\in\mathbb{R}^{}:\mu_s(\{x\})>0\}
\end{gather*}
则$D$是有限集或可列集,否则对任意满足$\underset{n=1}{\overset{+\infty}{\cup}}A_n=\mathbb{R}^{}$的互不相交的$\{A_n\}\subseteq\mathcal{B}$,必然有一个$A_n$中有不可列个正测度的单点集(不然的话由\cref{theo:CountableUnionCountable}即可得到$D$是可列集)。设$A_n$中所有正测度的点构成的集合为$E$,由\cref{theo:JordanDecomposition}可知$\mu_s$可以分解为测度$\varphi^+$与有限测度$\varphi^-$的差或有限测度$\varphi^+$与测度$\varphi^-$的差\info{检查符号测度部分可以负无穷的所有情况}。取:
\begin{equation*}
	\forall\;k\in\mathbb{N}^+,\;E_k=\left\{x\in E:\mu_s(x)\geqslant\frac{1}{k}\right\}
\end{equation*}
因为$E$中的元素不可列,所以必然存在一个$E_k$含有不可列个元素,否则由\cref{theo:CountableUnionCountable}即可得出$E$中的元素是可列的。根据\cref{prop:BorelSigmaField}(1)可知单点集在$\mathcal{B}$中,因为测度具有可列可加性,可取$E$的一个子集,在选定的子集最前面加入$E_k$中的点,加一个点该子集关于$\varphi^+$的测度就增加$\dfrac{1}{k}$,这个增加的过程可以无限的进行下去,于是$A_n$中存在在$\mathscr{F}$中的可列子集使得其关于$\varphi^+$的测度为$+\infty$。\par
在第一种情况中,由\cref{prop:Measure}(3)(单调性)和$\varphi^-$的有限性可得$\mu_s(A_n)=\varphi^+(A_n)-\varphi^-(A_n)=+\infty$,那么$\mu_s$就不是$\sigma$有限的符号测度,矛盾。\par
第二种情况时,$\mathscr{F}$中不可能存在关于$\varphi^+$的测度为$+\infty$的集合。\par
综上,$D$至多可列。\par
因为$D$至多可列,根据\cref{prop:BorelSigmaField}(1)可得单点集在$\mathcal{B}$中,所以$D\in\mathcal{B}$,于是令:
\begin{equation*}
	\forall\;A\in\mathcal{B},\;\mu_2(A)=\mu_s(A\cap D),\quad\mu_3=\mu_s-\mu_2
\end{equation*}
由$D$的定义和\cref{prop:Measure}(3)(单调性)可得$\mu_2$是一个非负集函数。因为$\mu_s$是符号测度,所以$\mu_2(\varnothing)=\mu_s(\varnothing)=0$,由\cref{prop:SetOperation}(4)可知$\mu_2$满足可列可加性,于是$\mu_2$是测度。因为$\mu_s$是$\sigma$有限测度,则$\mu_2$显然也是$\sigma$有限测度。\info{思考$\mu_3$是不是有限测度}\par
综上,$Pf^{-1}$有分解:
\begin{equation*}
	Pf^{-1}=\mu_1+\mu_2+\mu_3
\end{equation*}\par
若对任意的$A\in\mathcal{B}$有$\mu_2(A)=\mu_3(A)=0$,称$f$为\textbf{连续型随机变量}。因为$\mu_1\ll\lambda$,所以此时有$Pf^{-1}\ll\lambda$,因为$\lambda$是$\sigma$有限测度,$Pf^{-1}$是概率测度,由\cref{theo:RandonNikodym}可知存在$(\mathbb{R}^{},\mathcal{B},\lambda)$上在a.e.于$\mathbb{R}^{}$的意义下唯一的可测函数$p(x)$满足:
\begin{equation*}
	\dfrac{\dif Pf^{-1}}{\dif \lambda}(x)=p(x)
\end{equation*}
称$p(x)$为$f$的\textbf{概率密度函数}。在没有明确随机变量的场合,也将概率测度$P$对L测度$\lambda$的Randon-Nikodym导数称为概率密度函数。\par
若对任意的$A\in\mathcal{B}$有$\mu_1(A)=\mu_3(A)=0$,称$f$为\textbf{离散型随机变量}\footnote{从这里开始我们使用和初等概率论一样的记号,用$P(f=x_n)$来表示$P(\{f=x_n\})$。}。记点列$D=\{x_n\},\;P(f=x_n)=Pf^{-1}(\{x_n\}),\;n\in\mathbb{N}^+$,则:
\begin{equation*}
	\{P(f=x_n):x_n\in D\}
\end{equation*}
完全确定了$f$的概率分布,称上式为$f$的\textbf{概率分布列}(简称为分布列)。在没有明确随机变量的场合,也将满足条件的$\mu_2$称为概率分布列。\par
若对任意的$A\in\mathcal{B}$有$\mu_1(A)=\mu_2(A)=0$,称$f$为\textbf{奇异型随机变量}。\par
统称概率分布列与概率密度函数为\textbf{概率函数}。\par
\begin{theorem}
	任何随机变量都是离散型、连续型、奇异型随机变量的混合。
\end{theorem}
\begin{definition}
	称一族概率测度$\mathscr{P}$为\textbf{分布族}。
\end{definition}
\begin{property}\label{prop:ContinuousRV}
	设$X$是一个连续型随机变量,则:
	\begin{enumerate}
		\item 对于任意的$a\in\mathbb{R}^{}$,$P(X\leqslant a)=P(X<a),\;P(X\geqslant a)=P(X>a)$;
		\item $(\mathbb{R}^{},\mathcal{B}_{\mathbb{R}^{}})$上的Borel函数$p$是某个连续型随机变量的概率密度函数的充要条件为:
		\begin{enumerate}
			\item $p(x)\geqslant0\;$a.e.于$(\mathbb{R}^{},\mathcal{B}_{\mathbb{R}^{}},\lambda)$;
			\item $\int_{\mathbb{R}^{}}p(x)\dif\lambda=1$。
		\end{enumerate}
	\end{enumerate}
\end{property}
\begin{proof}
	(1)由\cref{prop:MeasurableFunction}(1)(2)、\cref{prop:Measure}(1)和连续型随机变量的定义即可得到。
\end{proof}