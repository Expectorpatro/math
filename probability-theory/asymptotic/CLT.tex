\section{中心极限定理}

\begin{theorem}\label{theo:CLT}
	设$\{X_n\}$为一列独立同分布的随机变量,其方差有限,则有:
	\begin{equation*}
		\frac{\sqrt{n}[\overline{X}_n-\operatorname{E}(X_1)]}{\sigma}\overset{d}{\longrightarrow}\operatorname{N}(0,1)
	\end{equation*}
	其中$\overline{X}_n=\dfrac{1}{n}\sum\limits_{i=1}^{n}X_i$。
\end{theorem}
\begin{note}
	中心极限定理的结论能否被写作:
	\begin{equation*}
		\overline{X}_n\overset{d}{\longrightarrow}\operatorname{N}\left(\operatorname{E}(X_1),\frac{\sigma^2}{n}\right)
	\end{equation*}
	答案是不能,无法推得。
\end{note}
\begin{definition}
	设$\{X_n\}$为一列独立同分布的随机变量,其方差有限,根据\cref{theo:CLT}有:
	\begin{equation*}
		\frac{\sqrt{n}[\overline{X}_n-\operatorname{E}(X_1)]}{\sigma}\overset{d}{\longrightarrow}\operatorname{N}(0,1)
	\end{equation*}
	将$\operatorname{N}\left(\operatorname{E}(X_1),\dfrac{\sigma^2}{n}\right)$称为$\overline{X}_n$的\gls{AsymptoticApproximateDistribution},记作:
	\begin{equation*}
		\overline{X}_n\overset{a}{\longrightarrow}\operatorname{N}\left(\operatorname{E}(X_1),\frac{\sigma^2}{n}\right)
	\end{equation*}
\end{definition}