\section{可测映射与可测函数}
\subsection{可测映射}
\begin{definition}
	称$X$和其上的一个$\sigma$域$\mathscr{A}$为\gls{MeasurableSpace},记为$(X,\mathscr{A})$。
\end{definition}
\begin{definition}
	设$(X,\mathscr{A})$和$(Y,\mathscr{B})$为可测空间,$f$是一个$X$到$Y$的映射。如果$f^{-1}(\mathscr{B})\subset\mathscr{A}$,则称$f$为从$(X,\mathscr{A})$到$(Y,\mathscr{B})$的\gls{MeasurableMap},称$\sigma(f)=f^{-1}(\mathscr{B})$为使映射$f$可测的最小$\sigma$域。
\end{definition}
\begin{lemma}
	设$X,Y$为两个集合,$f$为一个$X$到$Y$的映射,$\mathscr{A}\subset Y$,则:
	\begin{equation*}
		\sigma[f^{-1}(\mathscr{A})]=f^{-1}[\sigma(\mathscr{A})]
	\end{equation*}
\end{lemma}
\begin{proof}
	先证$f^{-1}[\sigma(\mathscr{A})]$是一个$\sigma$域。\par
	(1)因为$\sigma(\mathscr{A})$是一个$\sigma$域,所以$Y\in\sigma(\mathscr{A})$,于是$X\in f^{-1}[\sigma(\mathscr{A})]$。\par
	(2)任取$A\in f^{-1}[\sigma(\mathscr{A})]$,设$f(A)=B$。由\cref{theo:PropertyOfPreimage}(3)可得,$A^c=[f^{-1}(B)]^c=f^{-1}(B^c)$。因为$\sigma(\mathscr{A})$是一个$\sigma$域,$B\in\sigma(\mathscr{A})$,所以$B^c\in\sigma(\mathscr{A})$,所以$A^c\in f^{-1}[\sigma(\mathscr{A})]$。由$A$的任意性,$f^{-1}[\sigma(\mathscr{A})]$对补的运算封闭。\par
	(3)任取集合序列$\{A_n\}\subset f^{-1}[\sigma(\mathscr{A})]$,$f(A_n)=B_n\in\sigma(\mathscr{A})$,由\cref{theo:PropertyOfPreimage}(4)可得:
	\begin{equation*}
		\underset{n=1}{\overset{+\infty}{\cup}}A_n=\underset{n=1}{\overset{+\infty}{\cup}}f^{-1}(B_n)=f^{-1}\left(\underset{n=1}{\overset{+\infty}{\cup}}B_n\right)
	\end{equation*}
	因为$\sigma(\mathscr{A})$是一个$\sigma$域,$B_n\in\sigma(\mathscr{A}),\;\forall\;n\in\mathbb{N}^+$,所以:
	\begin{equation*}
		\underset{n=1}{\overset{+\infty}{\cup}}B_n\in\sigma(\mathscr{A})
	\end{equation*}
	于是$\underset{n=1}{\overset{+\infty}{\cup}}A_n\in f^{-1}[\sigma(\mathscr{A})]$。由$\{A_n\}$的任意性,$f^{-1}[\sigma(\mathscr{A})]$对可列并的运算封闭。\par
	综上,$f^{-1}[\sigma(\mathscr{A})]$是一个$\sigma$域。由生成的定义,$\mathscr{A}\subset\sigma(\mathscr{A})$,由\cref{theo:PropertyOfPreimage}(2)可得$f^{-1}(\mathscr{A})\subset f^{-1}[\sigma(\mathscr{A})]$,即$f^{-1}[\sigma(\mathscr{A})]$是一个包含$f^{-1}(\mathscr{A})$的$\sigma$域,所以$\sigma[f^{-1}(\mathscr{A})]\subset f^{-1}[\sigma(\mathscr{A})]$。\par
	令:
	\begin{equation*}
		\mathscr{B}=\{B\subset Y:f^{-1}(B)\in \sigma[f^{-1}(\mathscr{A})]\}
	\end{equation*}
	下证$\mathscr{B}$是一个$\sigma$域。\par
	(1)因为$\sigma[f^{-1}(\mathscr{A})]$是一个$\sigma$域,所以$X\in\sigma[f^{-1}(\mathscr{A})]$。由\cref{theo:PropertyOfPreimage}(1)可得$f^{-1}(Y)=X\in\sigma[f^{-1}(\mathscr{A})]$,所以$Y\in\mathscr{B}$。\par
	(2)任取$B\in\mathscr{B}$。由\cref{theo:PropertyOfPreimage}(3)可得$f^{-1}(B^c)=[f^{-1}(B)]^c$。因为$\sigma[f^{-1}(\mathscr{A})]$是一个$\sigma$域,$f^{-1}(B)\in\sigma[f^{-1}(\mathscr{A})]$,所以$f^{-1}(B^c)=[f^{-1}(B)]^c\in\sigma[f^{-1}(\mathscr{A})]$,即$B^c\in\mathscr{B}$。由$B$的任意性,$\mathscr{B}$对补的运算封闭。\par
	(3)任取$\{B_n\}\subset\mathscr{B}$。由\cref{theo:PropertyOfPreimage}(4)可得:
	\begin{equation*}
		f^{-1}\left(\underset{n=1}{\overset{+\infty}{\cup}}B_n\right)=\underset{n=1}{\overset{+\infty}{\cup}}f^{-1}(B_i)
	\end{equation*}
	因为$B_n\in\mathscr{B},\;\forall\;n\in\mathbb{N}^+$,所以$f^{-1}(B_n)\in\sigma[f^{-1}(\mathscr{A})],\;\forall\;n\in\mathbb{N}^+$。因为$\sigma[f^{-1}(\mathscr{A})]$是一个$\sigma$域,所以:
	\begin{equation*}
		f^{-1}\left(\underset{n=1}{\overset{+\infty}{\cup}}B_n\right)=\underset{n=1}{\overset{+\infty}{\cup}}f^{-1}(B_n)\in\sigma[f^{-1}(\mathscr{A})]
	\end{equation*}
	于是$\underset{n=1}{\overset{+\infty}{\cup}}B_n\in\mathscr{B}$。由$\{B_n\}$的任意性,$\mathscr{B}$对可列并的运算封闭。\par
	综上,$\mathscr{B}$是一个$\sigma$域。因为$f^{-1}(\mathscr{A})\subset\sigma[f^{-1}(\mathscr{A})]$,所以$\mathscr{A}\subset\mathscr{B}$。由生成的定义,$\sigma(\mathscr{A})\subset\mathscr{B}$。任取$C\in f^{-1}[\sigma(\mathscr{A})]$,则存在$A\in\sigma(\mathscr{A})$使得$C=f^{-1}(A)$,因为$\sigma(\mathscr{A})\subset\mathscr{B}$,所以$A\in\mathscr{B}$,于是$C=f^{-1}(A)\in\sigma[f^{-1}(\mathscr{A})]$。由$C$的任意性,$f^{-1}[\sigma(\mathscr{A})]\subset\sigma[f^{-1}(\mathscr{A})]$。\par
	综上,$f^{-1}[\sigma(\mathscr{A})]=\sigma[f^{-1}(\mathscr{A})]$。
\end{proof}
\begin{theorem}\label{theo:SigmaMeasurable}
	设$\mathscr{B}$是$Y$上的任一集族,$(X,\mathscr{A}),(Y,\sigma(\mathscr{B}))$是两个可测空间,则映射$f$为一个$(X,\mathscr{A})$到$(Y,\sigma(\mathscr{B}))$的可测映射的充分必要条件为:
	\begin{equation*}
		f^{-1}(\mathscr{B})\subset\mathscr{A}
	\end{equation*}
\end{theorem}
\begin{proof}
	由可测映射的定义:
	\begin{equation*}
		\text{$f$为一个$(X,\mathscr{A})$到$(Y,\sigma(\mathscr{B}))$的可测映射}
		\Leftrightarrow f^{-1}[\sigma(\mathscr{B})]\subset\mathscr{A}
		\Leftrightarrow
		\sigma[f^{-1}(\mathscr{B})]\subset\mathscr{A}
	\end{equation*}\par
	\textbf{(1)必要性:}若$\sigma[f^{-1}(\mathscr{B})]\subset\mathscr{A}$,则$f^{-1}(\mathscr{B})\subset\sigma[f^{-1}(\mathscr{B})]\subset\mathscr{A}$。\par
	\textbf{(2)充分性:}若$f^{-1}(\mathscr{B})\subset\mathscr{A}$,因为$\mathscr{A}$是一个$\sigma$域,由生成的定义可知$\sigma[f^{-1}(\mathscr{B})]\subset\mathscr{A}$。
\end{proof}
\begin{theorem}
	设$g$是可测空间$(X,\mathscr{A})$到可测空间$(Y,\mathscr{B})$的可测映射,$f$是可测空间$(Y,\mathscr{B})$到可测空间$(Z,\mathscr{C})$的可测映射,则$f\circ g$是$(X,\mathscr{A})$到$(Z,\mathscr{C})$的可测映射。
\end{theorem}
\begin{proof}
	因为$g$是可测空间$(X,\mathscr{A})$到可测空间$(Y,\mathscr{B})$的可测映射,所以$g^{-1}(\mathscr{B})\subset\mathscr{A}$。因为$f$是可测空间$(Y,\mathscr{B})$到可测空间$(Z,\mathscr{C})$的可测映射,所以$f^{-1}(\mathscr{C})\subset\mathscr{B}$。由\cref{theo:PropertyOfPreimage}(2)可得:
	\begin{equation*}
		(f\circ g)^{-1}(\mathscr{C})=g^{-1}[f^{-1}(\mathscr{C})]\subset g^{-1}(\mathscr{B})\subset\mathscr{A}\qedhere
	\end{equation*}
\end{proof}

\subsection{可测函数}
\subsubsection{可测函数的定义}
\begin{definition}
	从可测空间$(X,\mathscr{A})$到可测空间$(\overline{\mathbb{R}},\mathcal{B}_{\overline{\mathbb{R}}})$的可测映射称为$(X,\mathscr{A})$上的\gls{MeasurableFunction}。特别的,从可测空间$(X,\mathscr{A})$到可测空间$(\mathbb{R},\mathcal{B})$的可测映射称为$(X,\mathscr{A})$上的有限值可测函数或\gls{r.v.}。
\end{definition}
\begin{theorem}\label{theo:MeasurableFunction}
	$f$是可测空间$(X,\mathscr{A})$上的可测函数的充要条件为:
	\begin{enumerate}
		\item 对任意的$a\in\mathbb{R}$,$\{f<a\}\in\mathscr{A}$。
		\item 对任意的$a\in\mathbb{R}$,$\{f\leqslant a\}\in\mathscr{A}$。
		\item 对任意的$a\in\mathbb{R}$,$\{f>a\}\in\mathscr{A}$。
		\item 对任意的$a\in\mathbb{R}$,$\{f\geqslant a\}\in\mathscr{A}$。
	\end{enumerate}
\end{theorem}
\begin{proof}
	(1)由\cref{theo:BorelRwqEquivDef}中$\mathcal{B}_{\overline{\mathbb{R}}}$的等价定义(1)以及\cref{theo:SigmaMeasurable}可得:
	\begin{align*}
		f\text{是可测函数}
		&\Leftrightarrow f^{-1}(\mathcal{B}_{\overline{\mathbb{R}}})\subset\mathscr{A} \\
		&\Leftrightarrow\forall\;\alpha\in\mathbb{R},\;f^{-1}\Bigl([-\infty,\alpha)\Bigr)\in\mathscr{A} \\
		&\Leftrightarrow\forall\;\alpha\in\mathbb{R},\;\{f<a\}\in\mathscr{A}
	\end{align*}\par
	(2)(3)(4)同理,只需在(1)的第一行到第二行的过程中使用\cref{theo:BorelRwqEquivDef}涉及到的对应的$\mathcal{B}_{\overline{\mathbb{R}}}$的等价定义。
\end{proof}
\begin{corollary}\label{cor:f<g<=g=g}
	设$f(x)$和$g(x)$为可测空间$(X,\mathscr{A})$上的可测函数,则$\{f<g\},\{f\leqslant g\},\{f=g\}\in\mathscr{A}$。
\end{corollary}
\begin{proof}
	因为:
	\begin{equation*}
		\{f<g\}=\underset{r\in\mathbb{Q}}{\overset{}{\cup}}[\{f<r\}\cap\{g>r\}]
	\end{equation*}
	而:
	\begin{equation*}
		\{f<r\}\cap\{g>r\}=\Bigl(\{f\geqslant r\}\cup\{g\leqslant r\}\Bigr)^c
	\end{equation*}
	由\cref{theo:MeasurableFunction}可知右式在$\mathscr{A}$中,于是有:
	\begin{equation*}
		\{f<g\}\in\mathscr{A}
	\end{equation*}
	由$f(x)$与$g(x)$的对称性,$\{g<f\}\in\mathscr{A}$,那么就有$\{f\leqslant g\}=\{g<f\}^c\in\mathscr{A}$。因为:
	\begin{align*}
		\{f=g\}
		&=\{f\leqslant g\}\backslash\{f<g\} \\
		&=\{f\leqslant g\}\cap\{f<g\}^c \\
		&=\Bigl(\{f\leqslant g\}^c\cup\{f<g\}\Bigr)^c
	\end{align*}
	所以$\{f=g\}\in\mathscr{A}$。
\end{proof}
\subsubsection{可测函数的运算}
\begin{theorem}\label{theo:IMeasurable}
	可测空间$(X,\mathscr{A})$上集合$A\in\mathscr{A}$的指示函数$I_A$是可测函数。
\end{theorem}
\begin{lemma}
	$(X,\mathscr{A})$是一个可测空间。对任何$a,b\in\overline{\mathbb{R}}$和任何的$A,B\in\mathscr{A}$,只要$a+b$有意义,那么$aI_A+bI_B$是可测函数。
\end{lemma}
\begin{theorem}\label{theo:MeasurableFunctionOperate}
	若$f,g$是可测空间$(X,\mathscr{A})$上的可测函数,则:
	\begin{enumerate}
		\item 对任意的$\alpha,\beta\in\overline{\mathbb{R}}$,若对任意的$x\in X$,$\alpha f+\beta g$有意义,则$\alpha f+\beta g$是可测函数;
		\item $fg$是可测函数;
		\item 若对任意的$x\in X$,有$g(x)\ne0$,则$f/g$是可测函数。
	\end{enumerate}
\end{theorem}
\begin{theorem}\label{theo:MeasurableFunctionSeqMeasurable}
	$\{f_n(x)\}$是可测空间$(X,\mathscr{A})$上的一列可测函数,则:
	\begin{equation*}
		\inf_nf_n(x),\;\sup_nf_n(x),\;
		\varliminf_{n\to+\infty}f_n(x),\;\varlimsup_{n\to+\infty}f_n(x)
	\end{equation*}
	也是可测函数。
\end{theorem}
\subsubsection{可测函数与简单函数}
\begin{definition}
	对于空间$X$,如果存在有限个互不相交的集合$\{A_i\subset X:i=1,2,\dots,n\}$满足:
	\begin{equation*}
		\underset{i=1}{\overset{n}{\cup}}A_i=X
	\end{equation*}
	则称$\{A_i\subset X:i=1,2,\dots,n\}$为$X$的一个\textbf{有限分割}。如果$A_i\in \mathscr{A},\;\forall\;i=1,2,\dots,n$,则称$\{A_i\subset X:i=1,2,\dots,n\}$为可测空间$(X,\mathscr{A})$的一个\textbf{有限可测分割}。当$\{A_n\in\mathscr{A}\}$是一个互不相交可列集合且满足:
	\begin{equation*}
		\underset{n=1}{\overset{+\infty}{\cup}}A_n=X
	\end{equation*}
	时,称$\{A_n\}$为可测空间$(X,\mathcal{A})$的一个\textbf{可列可测分割}。
\end{definition}
\begin{definition}
	对于可测空间$(X,\mathscr{A})$上的函数$\varphi:X\rightarrow \mathbb{R}^{}$,如果存在有限可测分割$\{A_i\in \mathscr{A}:i=1,2,\dots,n\}$和$\{a_i\in\mathbb{R}^{}:i=1,2,\dots,n\}$使得:
	\begin{equation*}
		\varphi(x)=\sum_{i=1}^{n}a_iI_{A_i}(x)
	\end{equation*}
	其中$I_{A_i}(x)$为表示$x$是否在$A_i$中的指示函数,则称$\varphi(x)$为\gls{SimpleFunction}。
\end{definition}
\begin{property}\label{prop:SimpleFunction}
	简单函数具有如下性质:
	\begin{enumerate}
		\item 简单函数是可测函数;
		\item 设$\varphi(x),\psi(x)$为简单函数,可分别表示为:
		\begin{equation*}
			\varphi(x)=\sum_{i=1}^{m}c_iI_{E_i}(x),\quad
			\psi(x)=\sum_{j=1}^{n}d_jI_{F_j}(x)
		\end{equation*}
		则对任意的$\alpha,\beta\in\mathbb{R}^{}$,$\alpha\varphi(x)+\beta\psi(x)$也是简单函数,且可以表示为:
		\begin{equation*}
		\alpha\varphi(x)+\beta\psi(x)=\sum_{i=1}^{m}\sum_{j=1}^{n}(\alpha c_i+\beta d_j)I_{E_i\cap F_j}(x)
		\end{equation*}
		\item 如果$f$是可测空间$(X,\mathscr{A})$上的可测函数,则$f$为简单函数的充分必要条件为它的值域是有限个实数组成的集合。
	\end{enumerate}
\end{property}
\begin{proof}
	(1)由可测函数的定义立即可得。\par
	(2)因为:
	\begin{align*}
		\alpha\varphi(x)+\beta\psi(x)
		&=\alpha\sum_{i=1}^{m}c_iI_{E_i}(x)+\beta\sum_{j=1}^{n}d_jI_{F_j}(x) \\
		&=\alpha\sum_{i=1}^{m}c_i\sum_{j=1}^{n}I_{E_i\cap F_j}(x)+\beta\sum_{j=1}^{n}d_j\sum_{i=1}^{m}I_{E_i\cap F_j}(x) \\
		&=\sum_{i=1}^{m}\sum_{j=1}^{n}\alpha c_iI_{E_i\cap F_j}(x)+\sum_{j=1}^{n}\sum_{i=1}^{m}\beta d_jI_{E_i\cap F_j}(x) \\
		&=\sum_{i=1}^{m}\sum_{j=1}^{n}(\alpha c_i+\beta d_j)I_{E_i\cap F_j}(x)
	\end{align*}
	并且有$\{E_i\cap F_j:i=1,2,\dots,m,\;j=1,2,\dots,n\}$是$X$的有限可测分割,由此可知结论成立。\par
	(3)必要性由简单函数的定义即可立即得到。设$f$的值域为$\{a_i:i=1,2,\dots,n\}$,因为$f$是可测函数,由\cref{theo:MeasurableFunction}可得$\{f=a_i\}\in\mathscr{A}$,于是$f$可表为:
	\begin{equation*}
		f(x)=\sum_{i=1}^{n}a_iI_{\{f=a_i\}}(x),\quad X=\underset{i=1}{\overset{n}{\cup}}\{f=a_i\},\quad\{f=a_i\}\cap\{f=a_j\}=\varnothing,\;\forall\;i\ne j
	\end{equation*}
	所以$f$为简单函数,充分性得证。
\end{proof}
\begin{definition}
	设$f(x)$是可测空间$(X,\mathscr{A})$上的可测函数,令:
	\begin{gather*}
		f^+(x)=\max\{f(x),0\}=
		\begin{cases}
			f(x),&f(x)\geqslant 0 \\
			0,&f(x)<0
		\end{cases} \\
		f^-(x)=-\min\{f(x),0\}=
		\begin{cases}
			-f(x),&f(x)\leqslant 0 \\
			0,&f(x)>0
		\end{cases}
	\end{gather*}
	分别称$f^+(x)$和$f^-(x)$为$f(x)$的正部和负部。
\end{definition}
\begin{theorem}\label{theo:Measurablef^+f^-}
	设$f(x)$是可测空间$(X,\mathscr{A})$上的可测函数,则$f^+(x)$和$f^-(x)$也是可测函数。
\end{theorem}
\begin{proof}
	由可测函数的定义,$g(x)=0$是一个可测函数。因为$\max\{f(x),0\}=\sup\{f(x),g(x)\}$,由\cref{theo:MeasurableFunctionSeqMeasurable}可知此时$f^+(x)$是可测函数。同理,$f^-(x)$也是可测函数。
\end{proof}
\begin{theorem}\label{theo:MeasurableFunctionSimpleFunction}
	(1)若$f(x)$是可测空间$(X,\mathscr{A})$上的非负可测函数,则存在简单函数列$\{\varphi_n(x)\}$,使得对任意$x\in X$,$\varphi_n(x)\leqslant\varphi_{n+1}(x),\;n\in\mathbb{N}^+$,且$\lim\limits_{n\to+\infty}\varphi_n(x)=f(x)$。\par
	(2)若$f(x)$是可测空间$(X,\mathscr{A})$上的可测函数,则存在可测简单函数列$\{\varphi_n(x)\}$,使得对任意$x\in X$,$\lim\limits_{n\to+\infty}\varphi_n(x)=f(x)$。若$f(x)$有界,则上述收敛可以是一致收敛。
\end{theorem}
\begin{proof}
	(1)对任意的$n\in\mathbb{N}^+$,将$[0,n]$分为$n2^n$份,令:
	\begin{gather*}
		E_{nj}=\left\{x\in X:\frac{j-1}{2^n}\leqslant f(x)<\frac{j}{2^n}\right\},\quad j=1,2,\dots,n2^n \\
		E_n=\{x:f(x)\geqslant n\},\quad n=1,2,\dots
	\end{gather*}
	作函数列:
	\begin{equation*}
		\varphi_n(x)=
		\begin{cases}
			\dfrac{j-1}{2^n},&x\in E_{nj} \\
			n,&x\in E_n
		\end{cases}
		j=1,2,\dots,n2^n;\;n=1,2,\dots
	\end{equation*}
	则$\varphi_n(x)$是简单函数,并且有:
	\begin{equation*}
		\varphi_n(x)\leqslant\varphi_{n+1}(x)\leqslant f(x),\quad\forall\;n\in\mathbb{N}^+
	\end{equation*}
	设$x\in X$,若$f(x)<+\infty$,则当$n>f(x)$时有:
	\begin{equation*}
		0\leqslant f(x)-\varphi_n(x)\leqslant 2^{-n}
	\end{equation*}
	若$f(x)=+\infty$,则$\varphi_n(x)=n,\;n=1,2,\dots$,因此$\lim\limits_{n\to+\infty}\varphi_n(x)=f(x)$。\par
	(2)$\;f(x)=f^+(x)-f^-(x)$,若$f(x)$是可测函数,由\cref{theo:Measurablef^+f^-}可知$f^+(x),f^-(x)$也是可测函数。由(1),存在可测简单函数列$\{\varphi_n^+(x)\}$和$\{\varphi_n^-(x)\}$,使得对任意的$x\in X$,有:
	\begin{equation*}
		\lim\limits_{n\to+\infty}\varphi_n^+(x)=f^+(x),\;
		\lim\limits_{n\to+\infty}\varphi_n^-(x)=f^-(x)
	\end{equation*}
	令$\varphi_n(x)=\varphi_n^+(x)-\varphi_n^-(x)$,由\cref{theo:MeasurableFunctionOperate}可知$\{\varphi_n(x)\}$是可测简单函数列,且对任意的$x\in X$,$\lim\limits_{n\to+\infty}\varphi_n(x)=f(x)$。\par
	若$f(x)$有界,设$\sup\limits_{x\in E}\{|f(x)|\}=M$,则由(1)的证明过程,当$n>M$时有:
	\begin{gather*}
		\sup_{x\in X}\{|f^+(x)-\varphi_n^+(x)|\}\leqslant\frac{1}{2^n} \\
		\sup_{x\in X}\{|f^-(x)-\varphi_n^-(x)|\}\leqslant\frac{1}{2^n}
	\end{gather*}
	因此:
	\begin{align*}
		\sup_{x\in X}\{|f(x)-\varphi_n(x)|\}
		&=\sup_{x\in X}\{|f^+(x)-f^-(x)-\varphi_n^+(x)+\varphi_n^-(x)|\} \\
		&\leqslant\sup_{x\in X}\{|f^+(x)-\varphi_n^+(x)|+|f^-(x)-\varphi_n^-(x)|\} \\
		&\leqslant\sup_{x\in X}\{|f^+(x)-\varphi_n^+(x)|\}+\sup_{x\in X}\{|f^-(x)-\varphi_n^-(x)|\} \\
		&\leqslant\frac{1}{2^{n-1}}
	\end{align*}
	所以$\{\varphi_n(x)\}$在$X$上一致收敛于$f(x)$。
\end{proof}

\subsection{可测函数的收敛性}
\begin{lemma}\label{lem:|f-g|Measurable}
	若$f,g$是测度空间$(X,\mathscr{F},\mu)$上的可测函数,则对于任意的$\varepsilon>0$,$\{|f-g|\geqslant\varepsilon\}\in \mathscr{F}$。
\end{lemma}
\begin{proof}
	因为:
	\begin{equation*}
		\{|f-g|\geqslant\varepsilon\}=\{f-g\geqslant\varepsilon\}\cup\{f-g\leqslant-\varepsilon\}=\{f\geqslant g+\varepsilon\}\cup\{f\leqslant g-\varepsilon\}
	\end{equation*}
	因为$g$是可测函数,由\cref{theo:MeasurableFunction}可得$g+\varepsilon$和$g-\varepsilon$也是可测函数。由\cref{cor:f<g<=g=g}可知$\{f\geqslant g+\varepsilon\},\{f\leqslant g-\varepsilon\}\in\mathscr{F}$。因为$\mathscr{F}$是一个$\sigma$域,所以:
	\begin{equation*}
		\{|f-g|\geqslant\varepsilon\}=\{f\geqslant g+\varepsilon\}\cup\{f\leqslant g-\varepsilon\}\in \mathscr{F}\qedhere
	\end{equation*}
\end{proof}
\subsubsection{几乎处处收敛}
\begin{definition}
	设$\{f_n\}$和$f$是测度空间$(X,\mathscr{F},\mu)$上的可测函数,如果:
	\begin{equation*}
		\mu\left(\left\{\lim_{n\to+\infty}f_n\ne f\right\}\right)=0
	\end{equation*}
	则称可测函数列$\{f_n\}\;$a.e.以$f$为极限,记为$f_n\overset{\text{a.e.}}{\longrightarrow}f$。若此时还有$f$有限a.e.于$X$,则称$\{f_n\}\;$a.e.收敛到$f$。若$(X,\mathscr{F},P)$是概率空间,称$\{f_n\}\;$a.e.收敛到$f$为$\{f_n\}$几乎必然收敛到$f$,记作$f_n\overset{\text{a.s.}}{\longrightarrow}f$
\end{definition}
\begin{theorem}\label{theo:EquiConditiona.e.}
	设$\{f_n\}$和$f$是测度空间$(X,\mathscr{F},\mu)$上的可测函数,$\{f_n\}\;$a.e.收敛于$f$的充分必要条件为对任意的$\varepsilon>0$,有\footnote{从今往后,对任何的$\varepsilon>0$和$n\in\mathbb{N}^+$,$f_n(x)-f(x)$没有定义的$x\in X$也计入集合$\{|f_n-f|\geqslant\varepsilon\}$}:
	\begin{equation*}
		\mu\left(\varlimsup_{n\to+\infty}\{|f_n-f|\geqslant\varepsilon\}\right)=0
	\end{equation*}
\end{theorem}
\begin{proof}
	若$\lim\limits_{n\to+\infty}f_n(x)\ne f(x)$,则有:
	\begin{equation*}
		\exists\;k\in\mathbb{N}^+,\;\forall\;N\in\mathbb{N}^+,\;\exists\;n\geqslant N,\;|f_n(x)-f(x)|\geqslant\frac{1}{k}
	\end{equation*}
	于是:
	\begin{equation*}
		\left\{\lim_{n\to+\infty}f_n\ne f\right\}=\underset{k=1}{\overset{+\infty}{\cup}}\underset{m=1}{\overset{+\infty}{\cap}}\underset{n=m}{\overset{+\infty}{\cup}}\left\{|f_n-f|\geqslant\frac{1}{k}\right\}
	\end{equation*}\par
	\textbf{(1)充分性:}此时由半环上测度的次可列可加性(\cref{theo:MeasureOfSemiring})可得:
	\begin{align*}
		\mu\left(\varlimsup_{n\to+\infty}\{|f_n-f|\geqslant\varepsilon\}\right)
		&=\mu\left(\underset{k=1}{\overset{+\infty}{\cup}}\underset{m=1}{\overset{+\infty}{\cap}}\underset{n=m}{\overset{+\infty}{\cup}}\left\{|f_n-f|\geqslant\frac{1}{k}\right\}\right) \\
		&\leqslant\sum_{k=1}^{+\infty}\mu\left(\underset{m=1}{\overset{+\infty}{\cap}}\underset{n=m}{\overset{+\infty}{\cup}}\left\{|f_n-f|\geqslant\frac{1}{k}\right\}\right)=0
	\end{align*}\par
	\textbf{(2)必要性:}对任意取定的$\varepsilon>0$,若:
	\begin{equation*}
		x\in\varlimsup_{n\to+\infty}\{|f_n-f|\geqslant\varepsilon\}
	\end{equation*}
	则:
	\begin{equation*}
		\forall\;m\in\mathbb{N}^+,\;\exists\;n\geqslant m,\;|f_n(x)-f(x)|\geqslant \varepsilon
	\end{equation*}
	即:
	\begin{equation*}
		\lim_{n\to+\infty}f_n(x)\ne f(x)
	\end{equation*}
	于是对这个$\varepsilon$,有:
	\begin{equation*}
		\varlimsup_{n\to+\infty}\{|f_n-f|\geqslant\varepsilon\}\subset\left\{\lim_{n\to+\infty}f_n\ne f\right\}
	\end{equation*}
	因为$f_n,f$都是可测函数,由\cref{lem:|f-g|Measurable}可知$\{|f_n-f|\geqslant\varepsilon\}\in\mathscr{F}$。因为$\mathscr{F}$是一个$\sigma$域,对可列并封闭,所以:
	\begin{equation*}
		\varliminf_{n\to+\infty}\{|f_n-f|\geqslant\varepsilon\}=\underset{m=1}{\overset{+\infty}{\cap}}\underset{n=m}{\overset{+\infty}{\cup}}\{|f_n-f|\geqslant\varepsilon\}\in \mathscr{F}
	\end{equation*}
	由测度的单调性可得:
	\begin{equation*}
		\mu\left(\varlimsup_{n\to+\infty}\{|f_n-f|\geqslant\varepsilon\}\right)=0\qedhere
	\end{equation*}
\end{proof}
\subsubsection{几乎一致收敛}
\begin{definition}
	设$\{f_n\}$和$f$是测度空间$(X,\mathscr{F},\mu)$上的可测函数。如果对任意的$\varepsilon>0$,存在$A\in \mathscr{F}$使得$\mu(A)<\varepsilon$且:
	\begin{equation*}
		\lim_{n\to+\infty}\sup_{x\notin A}|f_n(x)-f(x)|=0
	\end{equation*}
	则称$\{f_n\}$几乎一致收敛到$f$,记为$f_n\overset{\text{a.u.}}{\longrightarrow}f$。
\end{definition}
\begin{theorem}\label{theo:EquiConditiona.u.}
	设$\{f_n\}$和$f$是测度空间$(X,\mathscr{F},\mu)$上的可测函数,$f_n\overset{\text{a.u.}}{\longrightarrow}f$的充分必要条件为对任意的$\varepsilon$有:
	\begin{equation*}
		\lim_{m\to+\infty}\mu\left(\underset{n=m}{\overset{+\infty}{\cup}}\{|f_n-f|\geqslant\varepsilon\}\right)=0
	\end{equation*}
\end{theorem}
\begin{proof}
	\textbf{(1)必要性:}因为$\{f_n\}\overset{\text{a.u.}}{\longrightarrow}f$,所以:
	\begin{equation*}
		\forall\;\delta>0,\;\exists\;A\in \mathscr{F},\;\mu(A)<\delta,\;\forall\;\varepsilon>0,\;\exists\;m\in \mathbb{N}^+,\;\forall\;n>m,\;\sup_{x\notin A}|f_n(x)-f(x)|<\varepsilon
	\end{equation*}
	即:
	\begin{equation*}
		\forall\;\delta>0,\;\exists\;A\in \mathscr{F},\;\mu(A)<\delta,\;\forall\;\varepsilon>0,\;\exists\;m\in\mathbb{N}^+,\;A^c\subset\underset{n=m}{\overset{+\infty}{\cap}}\{|f_n-f|<\varepsilon\}
	\end{equation*}
	于是:
	\begin{equation*}
		\forall\;\delta>0,\;\exists\;A\in \mathscr{F},\;\mu(A)<\delta\;\forall\;\varepsilon>0,\;\exists\;m\in\mathbb{N}^+,\;\underset{n=m}{\overset{+\infty}{\cup}}\{|f_n-f|\geqslant\varepsilon\}\subset A
	\end{equation*}
	因为$f_n$和$f$都是可测函数,由\cref{lem:|f-g|Measurable}可知$\{|f_n-f|\geqslant\varepsilon\}\in \mathscr{F}$。因为$\mathscr{F}$是一个$\sigma$域,对可列并封闭,所以:
	\begin{equation*}
		\underset{n=m}{\overset{+\infty}{\cup}}\{|f_n-f|\geqslant\varepsilon\}\in \mathscr{F}
	\end{equation*}
	由测度的单调性可得:
	\begin{equation*}
		\mu\left(\underset{n=m}{\overset{+\infty}{\cup}}\{|f_n-f|\geqslant\varepsilon\}\right)<\mu(A)<\delta
	\end{equation*}
	所以:
	\begin{equation*}
		\forall\;\delta>0,\;\forall\;\varepsilon>0,\;\exists\;m\in\mathbb{N}^+,\;\mu\left(\underset{n=m}{\overset{+\infty}{\cup}}\{|f_n-f|\geqslant\varepsilon\}\right)<\delta
	\end{equation*}
	即对任意的$\varepsilon>0$,有(上式前面的两个任意用集合语言来描述就是两个交运算,这是可以交换顺序的,所以有了如下结论):
	\begin{equation*}
		\lim_{m\to+\infty}\mu\left(\underset{n=m}{\overset{+\infty}{\cup}}\{|f_n-f|\geqslant\varepsilon\}\right)=0
	\end{equation*}\par
	\textbf{(2)充分性:}对任意的$\delta>0$,由所给条件,对$k\in\mathbb{N}^+$有:
	\begin{equation*}
		\lim_{m\to+\infty}\mu\left(\underset{n=m}{\overset{+\infty}{\cup}}\left\{|f_n-f|\geqslant\frac{1}{k}\right\}\right)=0	
	\end{equation*}
	于是存在$\{m_k\}$使得:
	\begin{equation*}
		\mu\left(\underset{n=m_k}{\overset{+\infty}{\cup}}\left\{|f_n-f|\geqslant\frac{1}{k}\right\}\right)<\frac{\delta}{2^k}
	\end{equation*}
	取:
	\begin{equation*}
		A=\underset{k=1}{\overset{+\infty}{\cup}}\underset{n=m_k}{\overset{+\infty}{\cup}}\left\{|f_n-f|\geqslant\frac{1}{k}\right\}
	\end{equation*}
	由\cref{lem:|f-g|Measurable}可得$A\in \mathscr{F}$,于是根据半环上测度的次可列可加性(\cref{theo:MeasureOfSemiring})可得:
	\begin{align*}
		\mu(A)&=\mu\left(\underset{k=1}{\overset{+\infty}{\cup}}\underset{n=m_k}{\overset{+\infty}{\cup}}\left\{|f_n-f|\geqslant\frac{1}{k}\right\}\right) \\
		&\leqslant\sum_{k=1}^{+\infty}\mu\left(\underset{n=m_k}{\overset{+\infty}{\cup}}\left\{|f_n-f|\geqslant\frac{1}{k}\right\}\right)<\delta
	\end{align*}
	注意到:
	\begin{align*}
		A^c&=\left(\underset{k=1}{\overset{+\infty}{\cup}}\underset{n=m_k}{\overset{+\infty}{\cup}}\left\{|f_n-f|\geqslant\frac{1}{k}\right\}\right)^c=\underset{k=1}{\overset{+\infty}{\cap}}\left(\underset{n=m_k}{\overset{+\infty}{\cup}}\left\{|f_n-f|\geqslant\frac{1}{k}\right\}\right)^c \\
		&=\underset{k=1}{\overset{+\infty}{\cap}}\underset{n=m_k}{\overset{+\infty}{\cap}}\left\{|f_n-f|<\frac{1}{k}\right\}
	\end{align*}
	所以若$x\notin A$,则对任意的$k\in \mathbb{N}^+$,当$n\geqslant m_k$时就有:
	\begin{equation*}
		|f_n(x)-f(x)|<\frac{1}{k}
	\end{equation*}
	由上确界的不等式性,此时即:
	\begin{equation*}
		\forall\;k\in\mathbb{N}^+,\;\exists\;m_k\in\mathbb{N}^+,\;\forall\;n\geqslant m_k,\;\sup_{x\notin A}|f_n(x)-f(x)|\leqslant\frac{1}{k}
	\end{equation*}
	也即:
	\begin{equation*}
		\lim_{n\to+\infty}\sup_{x\notin A}|f_n(x)-f(x)|=0
	\end{equation*}
	所以$f_n\overset{\text{a.u.}}{\longrightarrow}f$。
\end{proof}
\subsubsection{依测度收敛}
\begin{definition}
	设$\{f_n\}$和$f$时测度空间$(X,\mathscr{F},\mu)$上的可测函数。如果对任意的$\varepsilon>0$都有:
	\begin{equation*}
		\lim_{n\to+\infty}\mu(\{|f_n-f|\geqslant\varepsilon\})=0
	\end{equation*}
	则称可测函数列$\{f_n\}$\gls{ConvergentInMeasure}到$f$,记为$f_n\overset{\mu}{\longrightarrow}f$。若$(X,\mathscr{F},P)$是概率空间,称$f_n\overset{P}{\longrightarrow}f$为$\{f_n\}$\gls{ConvergentInProbability}到$f$。
\end{definition}
\subsubsection{依分布收敛}
\begin{theorem}
	设$X=\mathbb{R},\;\mathscr{A}=\{(a,b]:a,b\in\mathbb{R}\}$,$F$为$\mathbb{R}^{}$上非降右连续实值函数。对任意的$a,b\in\mathbb{R}$,令:
	\begin{equation*}
		\mu\Bigl((a,b]\Bigr)=
		\begin{cases}
			F(b)-F(a),&a<b \\
			0,&a\geqslant b
		\end{cases}
	\end{equation*}
	则$\mu$是$\mathscr{A}$上的测度。
\end{theorem}
\begin{definition}
	称$\mathbb{R}^{}$上非降右连续实值函数$F$为\gls{Quasi-DistributionFunction}。若$F$还满足:
	\begin{equation*}
		\lim_{x\to-\infty}F(x)=0,\;\lim_{x\to+\infty}F(x)=1
	\end{equation*}
	则称$F$为\gls{d.f.}。
\end{definition}
\begin{theorem}
	设$f$是概率空间$(X,\mathscr{F},P)$上的随机变量,对任意的$x\in\mathbb{R}^{}$,令:
	\begin{equation*}
		F(x)=P(f\leqslant x)
	\end{equation*}
	则$F$是一个分布函数。
\end{theorem}
\begin{proof}
	(1)由条件可得:
	\begin{equation*}
		F(x)=P(f\leqslant x)=P(\{f\leqslant x\})
	\end{equation*}
	是一个非降的实值函数。\par
	(2)注意到:
	\begin{equation*}
		\{f\leqslant x\}=\lim_{n\to+\infty}\left\{f\leqslant x+\frac{1}{n}\right\}
	\end{equation*}
	考虑集族:
	\begin{equation*}
		\left\{X_n=\left\{f\leqslant x+\frac{1}{n}\right\}\right\}
	\end{equation*}
	则$\{X_n\}$是一个单调不增的集族,有:
	\begin{equation*}
		\lim_{n\to+\infty}X_n=\underset{n=1}{\overset{+\infty}{\cap}}X_n=\underset{n=1}{\overset{+\infty}{\cap}}\left\{f\leqslant x+\frac{1}{n}\right\}
	\end{equation*}
	由半环上测度的上连续性(\cref{theo:MeasureOfSemiring})可得:
	\begin{align*}
		F(x)&=P(\{f\leqslant x\})
		=P\left(\lim_{n\to+\infty}\left\{f\leqslant x+\frac{1}{n}\right\}\right)
		=P\left(\lim_{n\to+\infty}X_n\right) \\
		&=\lim_{n\to+\infty}P(X_n)
		=\lim_{n\to+\infty}P\left(\left\{f\leqslant x+\frac{1}{n}\right\}\right)
		=\lim_{h\to+0}F(x+h)
	\end{align*}
	所以$F(x)$是右连续的。\par
	(3)注意到$f$是一个实值函数,由半环上测度的上连续性与下连续性(\cref{theo:MeasureOfSemiring})可得:
	\begin{gather*}
		\lim_{x\to-\infty}F(x)=\lim_{x\to-\infty}P(f\leqslant x)=P(f\leqslant-\infty)=P(\varnothing)=0 \\
		\lim_{x\to+\infty}F(x)=\lim_{x\to+\infty}P(f\leqslant x)=P(f\leqslant+\infty)=P(X)=1\qedhere
	\end{gather*}
\end{proof}
\begin{definition}
	设$f$是概率空间$(X,\mathscr{F},P)$上的随机变量,$F$是一个分布函数。若对任意的$x\in\mathbb{R}$,都有:
	\begin{equation*}
		F(x)=P(f\leqslant x)
	\end{equation*}
	则称r.v.$\;f$的分布函数是$F$,也说成$f$服从$F$,记为$f\sim F$。若$f$是从概率空间$(X,\mathscr{F},P)$到可测空间$(Y,\mathscr{A})$的可测映射,称:
	\begin{equation*}
		P(f^{-1}A),\;\forall\;A\in \mathscr{A}
	\end{equation*}
	为$f$的\gls{ProbabilityDistribution}。
\end{definition}
\begin{definition}
	设$\{f_n\sim F_n\}$是概率空间$(X,\mathscr{F},P)$上的随机变量序列,$F$是一个分布函数。若对于$F$的每一个连续点$x$都有:
	\begin{equation*}
		\lim_{n\to+\infty}F_n(x)=F(x)
	\end{equation*}
	则称$\{f_n\}$\gls{ConvergentInDistribution},记为$f_n\overset{d}{\longrightarrow}F$。若此时概率空间$(X,\mathscr{F},P)$上的随机变量$f\sim F$,则称随机变量序列$\{f_n\}$依分布收敛到$f$,记为$f_n\overset{d}{\longrightarrow f}$。
\end{definition}

\subsubsection{收敛性之间的关系}
\begin{theorem}\label{theo:a.e.a.u.mu.d}
	设$\{f_n\}$和$f$为测度空间$(X,\mathscr{F},\mu)$上的可测函数,则:
	\begin{enumerate}
		\item $f_n\overset{\text{a.u.}}{\longrightarrow}f$可推出$f_n\overset{\textbf{a.e.}}{\longrightarrow}f$和$f_n\overset{\mu}{\longrightarrow}f$;
		\item 若$\mu(X)<+\infty$,则$f_n\overset{\text{a.u.}}{\longrightarrow}f$等价于$f_n\overset{\text{a.e.}}{\longrightarrow}f$;
		\item $f_n\overset{\mu}{\longrightarrow}f$当且仅当对$\{f_n\}$的任一子列,存在该子列的子列$\{f_{n_k}\}$使得$f_{n_k}\overset{\text{a.u.}}{\longrightarrow}f$;
		\item $f_n\overset{P}{\longrightarrow}f$可推出$f_n\overset{d}{\longrightarrow}f$。
	\end{enumerate}
\end{theorem}
\begin{proof}
	(1)对任意的$n\in\mathbb{N}^+$和任意的$\varepsilon>0$,由\cref{cor:f<g<=g=g}可得$\{|f_n-f|\geqslant\varepsilon\}$是可测集,因为$\mathscr{F}$是一个$\sigma$域,对可列并、可列交封闭,所以:
	\begin{equation*}
		\underset{m=n}{\overset{+\infty}{\cup}}\{|f_m-f|\geqslant\varepsilon\}\in \mathscr{F},\;\underset{n=1}{\overset{+\infty}{\cap}}\underset{m=n}{\overset{+\infty}{\cup}}\{|f_m-f|\geqslant\varepsilon\}\in \mathscr{F}
	\end{equation*}
	因为:
	\begin{gather*}
		\{|f_n-f|\geqslant\varepsilon\}\subset\underset{m=n}{\overset{+\infty}{\cup}}\{|f_m-f|\geqslant\varepsilon\} \\
		\underset{n=1}{\overset{+\infty}{\cap}}\underset{m=n}{\overset{+\infty}{\cup}}\{|f_m-f|\geqslant\varepsilon\}\subset\underset{m=n}{\overset{+\infty}{\cup}}\{|f_m-f|\geqslant\varepsilon\}
	\end{gather*}
	由测度的单调性可得:
	\begin{gather*}
		\mu(\{|f_n-f|\geqslant\varepsilon\})\leqslant\mu\left(\underset{m=n}{\overset{+\infty}{\cup}}\{|f_m-f|\geqslant\varepsilon\}\right) \\
		\mu\left(\underset{n=1}{\overset{+\infty}{\cap}}\underset{m=n}{\overset{+\infty}{\cup}}\{|f_m-f|\geqslant\varepsilon\}\right)\leqslant\mu\left(\underset{m=n}{\overset{+\infty}{\cup}}\{|f_m-f|\geqslant\varepsilon\}\right) 
	\end{gather*}
	因为$f_n\overset{\text{a.u.}}{\longrightarrow}f$,由极限的不等式性和测度的非负性可得:
	\begin{gather*}
		0\leqslant\lim_{n\to+\infty}\mu(\{|f_n-f|\geqslant\varepsilon\})\leqslant\lim_{n\to+\infty}\mu\left(\underset{m=n}{\overset{+\infty}{\cup}}\{|f_m-f|\geqslant\varepsilon\}\right)=0 \\
		0\leqslant\mu\left(\underset{n=1}{\overset{+\infty}{\cap}}\underset{m=n}{\overset{+\infty}{\cup}}\{|f_m-f|\geqslant\varepsilon\}\right)\leqslant\lim_{n\to+\infty}\mu\left(\underset{m=n}{\overset{+\infty}{\cup}}\{|f_m-f|\geqslant\varepsilon\}\right)=0
	\end{gather*}
	所以:
	\begin{equation*}
		\lim_{n\to+\infty}\mu(\{|f_n-f|\geqslant\varepsilon\})=0,\;\mu\left(\underset{n=1}{\overset{+\infty}{\cap}}\underset{m=n}{\overset{+\infty}{\cup}}\{|f_m-f|\geqslant\varepsilon\}\right)=0
	\end{equation*}
	即$f_n\overset{\mu}{\longrightarrow}f,\;f_n\overset{a.e.}{\longrightarrow}f$。\par
	(2)设$f_n\overset{a.e.}{\longrightarrow}f$,令:
	\begin{equation*}
		\left\{X_n=\underset{m=n}{\overset{+\infty}{\cup}}\{|f_m-f|\geqslant\varepsilon\}\right\}
	\end{equation*}
	显然$\{X_n\}$是一个单调不增序列,其极限为:
	\begin{equation*}
		\lim_{n\to+\infty}X_n=\underset{n=1}{\overset{+\infty}{\cap}}\underset{m=n}{\overset{+\infty}{\cup}}\{|f_m-f|\geqslant\varepsilon\}
	\end{equation*}
	由半环上测度的上连续性可得(\cref{theo:MeasureOfSemiring}):
	\begin{equation*}
		\mu\left(\lim_{n\to+\infty}X_n\right)=\lim_{n\to+\infty}\mu(X_n)
	\end{equation*}
	于是:
	\begin{equation*}
		\mu\left(\underset{n=1}{\overset{+\infty}{\cap}}\underset{m=n}{\overset{+\infty}{\cup}}\{|f_m-f|\geqslant\varepsilon\}\right)=\lim_{n\to+\infty}\mu\left(\underset{m=n}{\overset{+\infty}{\cup}}\{|f_m-f|\geqslant\varepsilon\}\right)
	\end{equation*}
	所以:
	\begin{equation*}
		\lim_{n\to+\infty}\mu\left(\underset{m=n}{\overset{+\infty}{\cup}}\{|f_m-f|\geqslant\varepsilon\}\right)=0
	\end{equation*}
	即$f_n\overset{a.u.}{\longrightarrow}f$,再结合(1)即可得出结论。\par
	(3)\par
	(4)记$F$为$f$的分布函数,$F_n$为$f_n$的分布函数。因为$f_n\overset{P}{\longrightarrow}f$,所以对任意的$\varepsilon>0$,有:
	\begin{equation*}
		\lim_{n\to+\infty}P(|f_n-f|\geqslant\varepsilon)=0
	\end{equation*}\par
	对任意的$x\in X$、任意的$\varepsilon>0$和任意的$n\in \mathbb{N}^+$,有:
	\begin{align*}
		F_n(x)&=P(f_n\leqslant x) \\
		&\leqslant P(f_n\leqslant x,|f_n-f|<\varepsilon)+P(f_n\leqslant x,|f_n-f|\geqslant\varepsilon) \\
		&\leqslant P(f\leqslant x+\varepsilon)+P(|f_n-f|\geqslant\varepsilon)
	\end{align*}
	第二行到第三行第一式的变化是因为:
	\begin{equation*}
		|f_n-f|<\varepsilon\Leftrightarrow-\varepsilon<f_n-f<\varepsilon\rightarrow f<f_n+\varepsilon
	\end{equation*}
	而$f_n\leqslant x$,于是变为$f\leqslant x+\varepsilon$。该条件比原条件宽松,所以是小于等于号。由极限的不等式性可得:
	\begin{equation*}
		\lim_{n\to+\infty}F_n(x)\leqslant P(f\leqslant x+\varepsilon)+\lim_{n\to+\infty}P(|f_n-f|\geqslant\varepsilon)=P(f\leqslant x+\varepsilon)
	\end{equation*}
	由$\varepsilon$的任意性可得:
	\begin{equation*}
		\lim_{n\to+\infty}F_n(x)\leqslant F(x)
	\end{equation*}\par
	对任意的$x\in X$、任意的$\varepsilon>0$和任意的$n\in \mathbb{N}^+$,又有:
	\begin{align*}
		P(f\leqslant x-\varepsilon)&\leqslant P(f\leqslant x-\varepsilon,|f_n-f|<\varepsilon)+P(f\leqslant x-\varepsilon,|f_n-f|\geqslant \varepsilon) \\
		&\leqslant P(f_n\leqslant x)+P(|f_n-f|\geqslant\varepsilon)
	\end{align*}
	由极限的不等式性可得:
	\begin{equation*}
		P(f\leqslant x-\varepsilon)\leqslant\lim_{n\to+\infty}P(f_n\leqslant x)+\lim_{n\to+\infty}P(|f_n-f|\geqslant\varepsilon)=\lim_{n\to+\infty}F_n(x)
	\end{equation*}
	于是:
	\begin{equation*}
		P(f\leqslant x-0)\leqslant\lim_{n\to+\infty}F_n(x)
	\end{equation*}
	若$F(x)$在$x$处连续,就有:
	\begin{equation*}
		P(f\leqslant x-0)=F(x-0)=F(x)
	\end{equation*}
	所以:
	\begin{equation}
		\lim_{n\to+\infty}F_n(x)=F(x)
	\end{equation}
	即$f_n\overset{d}{\longrightarrow}f$。
\end{proof}
