\section{可测映射与可测函数}
\begin{note}
	在测度论中讨论的映射与函数都是定义在整个空间上的。
\end{note}
\subsection{可测映射}
\begin{definition}
	称$X$和其上的一个$\sigma$域$\mathscr{A}$为\gls{MeasurableSpace},记为$(X,\mathscr{A})$,称$\mathscr{A}$中的集合为可测集。
\end{definition}
\begin{definition}
	设$(X,\mathscr{A})$和$(Y,\mathscr{B})$为可测空间,$f$是一个$X$到$Y$的映射。如果$f^{-1}(\mathscr{B})\subseteq\mathscr{A}$,则称$f$为从$(X,\mathscr{A})$到$(Y,\mathscr{B})$的\gls{MeasurableMap},也称$f$是$\mathscr{A}$可测的。
\end{definition}
\begin{lemma}\label{lem:PreimageSigmaField}
	设$X,Y$为两个集合,$f$为一个$X$到$Y$的映射,$\mathscr{A}\subseteq Y$,则:
	\begin{equation*}
		\sigma[f^{-1}(\mathscr{A})]=f^{-1}[\sigma(\mathscr{A})]
	\end{equation*}
\end{lemma}
\begin{proof}
	先证$f^{-1}[\sigma(\mathscr{A})]$是一个$\sigma$域。\par
	(1)因为$\sigma(\mathscr{A})$是一个$\sigma$域,所以$Y\in\sigma(\mathscr{A})$,于是$X=f^{-1}(Y)\in f^{-1}[\sigma(\mathscr{A})]$。\par
	(2)任取$A\in f^{-1}[\sigma(\mathscr{A})]$,设$f(A)=B$。由\cref{theo:PropertyOfPreimage}(3)可得,$A^c=[f^{-1}(B)]^c=f^{-1}(B^c)$。因为$\sigma(\mathscr{A})$是一个$\sigma$域,$B\in\sigma(\mathscr{A})$,所以$B^c\in\sigma(\mathscr{A})$,所以$A^c\in f^{-1}[\sigma(\mathscr{A})]$。由$A$的任意性,$f^{-1}[\sigma(\mathscr{A})]$对补封闭。\par
	(3)任取集合序列$\{A_n\}\subseteq f^{-1}[\sigma(\mathscr{A})]$,$f(A_n)=B_n\in\sigma(\mathscr{A})$,由\cref{theo:PropertyOfPreimage}(4)可得:
	\begin{equation*}
		\underset{n=1}{\overset{+\infty}{\cup}}A_n=\underset{n=1}{\overset{+\infty}{\cup}}f^{-1}(B_n)=f^{-1}\left(\underset{n=1}{\overset{+\infty}{\cup}}B_n\right)
	\end{equation*}
	因为$\sigma(\mathscr{A})$是一个$\sigma$域,$B_n\in\sigma(\mathscr{A}),\;\forall\;n\in\mathbb{N}^+$,所以$\underset{n=1}{\overset{+\infty}{\cup}}B_n\in\sigma(\mathscr{A})$,于是$\underset{n=1}{\overset{+\infty}{\cup}}A_n\in f^{-1}[\sigma(\mathscr{A})]$。由$\{A_n\}$的任意性,$f^{-1}[\sigma(\mathscr{A})]$对可列并的运算封闭。\par
	综上,$f^{-1}[\sigma(\mathscr{A})]$是一个$\sigma$域。\par
	由生成的定义,$\mathscr{A}\subseteq\sigma(\mathscr{A})$,由\cref{theo:PropertyOfPreimage}(2)可得$f^{-1}(\mathscr{A})\subseteq f^{-1}[\sigma(\mathscr{A})]$,即$f^{-1}[\sigma(\mathscr{A})]$是一个包含$f^{-1}(\mathscr{A})$的$\sigma$域,所以$\sigma[f^{-1}(\mathscr{A})]\subseteq f^{-1}[\sigma(\mathscr{A})]$。\par
	令:
	\begin{equation*}
		\mathscr{B}=\{B\subseteq Y:f^{-1}(B)\in \sigma[f^{-1}(\mathscr{A})]\}
	\end{equation*}
	下证$\mathscr{B}$是一个$\sigma$域。\par
	(1)因为$\sigma[f^{-1}(\mathscr{A})]$是一个$\sigma$域,所以$X\in\sigma[f^{-1}(\mathscr{A})]$。由\cref{theo:PropertyOfPreimage}(1)可得$f^{-1}(Y)=X\in\sigma[f^{-1}(\mathscr{A})]$,所以$Y\in\mathscr{B}$。\par
	(2)任取$B\in\mathscr{B}$。由\cref{theo:PropertyOfPreimage}(3)可得$f^{-1}(B^c)=[f^{-1}(B)]^c$。因为$\sigma[f^{-1}(\mathscr{A})]$是一个$\sigma$域,$f^{-1}(B)\in\sigma[f^{-1}(\mathscr{A})]$,所以$f^{-1}(B^c)=[f^{-1}(B)]^c\in\sigma[f^{-1}(\mathscr{A})]$,即$B^c\in\mathscr{B}$。由$B$的任意性,$\mathscr{B}$对补封闭。\par
	(3)任取$\{B_n\}\subseteq\mathscr{B}$。由\cref{theo:PropertyOfPreimage}(4)可得:
	\begin{equation*}
		f^{-1}\left(\underset{n=1}{\overset{+\infty}{\cup}}B_n\right)=\underset{n=1}{\overset{+\infty}{\cup}}f^{-1}(B_n)
	\end{equation*}
	因为$B_n\in\mathscr{B},\;\forall\;n\in\mathbb{N}^+$,所以$f^{-1}(B_n)\in\sigma[f^{-1}(\mathscr{A})],\;\forall\;n\in\mathbb{N}^+$。因为$\sigma[f^{-1}(\mathscr{A})]$是一个$\sigma$域,所以:
	\begin{equation*}
		f^{-1}\left(\underset{n=1}{\overset{+\infty}{\cup}}B_n\right)=\underset{n=1}{\overset{+\infty}{\cup}}f^{-1}(B_n)\in\sigma[f^{-1}(\mathscr{A})]
	\end{equation*}
	于是$\underset{n=1}{\overset{+\infty}{\cup}}B_n\in\mathscr{B}$。由$\{B_n\}$的任意性,$\mathscr{B}$对可列并封闭。\par
	综上,$\mathscr{B}$是一个$\sigma$域。因为$f^{-1}(\mathscr{A})\subseteq\sigma[f^{-1}(\mathscr{A})]$,所以$\mathscr{A}\subseteq\mathscr{B}$。由生成的定义,$\sigma(\mathscr{A})\subseteq\mathscr{B}$。任取$C\in f^{-1}[\sigma(\mathscr{A})]$,则存在$A\in\sigma(\mathscr{A})$使得$C=f^{-1}(A)$,因为$\sigma(\mathscr{A})\subseteq\mathscr{B}$,所以$A\in\mathscr{B}$,于是$C=f^{-1}(A)\in\sigma[f^{-1}(\mathscr{A})]$。由$C$的任意性,$f^{-1}[\sigma(\mathscr{A})]\subseteq\sigma[f^{-1}(\mathscr{A})]$。\par
	综上,$f^{-1}[\sigma(\mathscr{A})]=\sigma[f^{-1}(\mathscr{A})]$。
\end{proof}
\begin{definition}
	设$(X,\mathscr{A})$和$(Y,\mathscr{B})$为可测空间,$f$是一个$X$到$Y$的映射。称$\sigma(f)=f^{-1}(\mathscr{B})$为使映射$f$可测的最小$\sigma$域。
\end{definition}
\begin{note}
	定义中说$\sigma(f)$是$\sigma$域,这是由于每个$\sigma$域都可以看作是自己生成的$\sigma$域,再利用\cref{lem:PreimageSigmaField}即可得到$\sigma(f)$是一个$\sigma$域。
\end{note}
\begin{property}\label{prop:MeasurableMapping}
	可测映射具有如下性质:
	\begin{enumerate}
		\item 设$\mathscr{B}$是$Y$上的任一集族,$(X,\mathscr{A}),(Y,\sigma(\mathscr{B}))$是两个可测空间,则映射$f$为一个$(X,\mathscr{A})$到$(Y,\sigma(\mathscr{B}))$的可测映射的充分必要条件为$f^{-1}(\mathscr{B})\subseteq\mathscr{A}$;
		\item 设$f$是可测空间$(X,\mathscr{A})$到可测空间$(Y,\mathscr{B})$的可测映射,$g$是可测空间$(Y,\mathscr{B})$到可测空间$(Z,\mathscr{C})$的可测映射,则$g\circ f$是$(X,\mathscr{A})$到$(Z,\mathscr{C})$的可测映射;
		\item 设$f$是可测空间$(X,\mathscr{A})$到可测空间$(Y,\mathscr{B})$的可测映射,$\mu$是$\mathscr{A}$上的测度,定义:
		\begin{equation*}
			\forall\;B\in\mathscr{B},\;\nu(B)=\mu[f^{-1}(B)]
		\end{equation*}
		则$(Y,\mathscr{B},\nu)$是测度空间,且:
		\begin{enumerate}
			\item 若$(X,\mathscr{A},\mu)$是概率空间,则$(Y,\mathscr{B},\nu)$也是概率空间;
			\item 若$\mu$是有限测度,则$\nu$也是有限测度;
			\item 若$\mu$是$\sigma$有限测度,则$\nu$也是$\sigma$有限测度。\info{怀疑这个结论不对}
		\end{enumerate}
	\end{enumerate}
\end{property}
\begin{proof}
	(1)由可测映射的定义:
	\begin{equation*}
		\text{$f$为一个$(X,\mathscr{A})$到$(Y,\sigma(\mathscr{B}))$的可测映射}
		\Leftrightarrow f^{-1}[\sigma(\mathscr{B})]\subseteq\mathscr{A}
		\Leftrightarrow
		\sigma[f^{-1}(\mathscr{B})]\subseteq\mathscr{A}
	\end{equation*}\par
	\textbf{必要性:}若$\sigma[f^{-1}(\mathscr{B})]\subset\mathscr{A}$,则$f^{-1}(\mathscr{B})\subseteq\sigma[f^{-1}(\mathscr{B})]\subseteq\mathscr{A}$。\par
	\textbf{充分性:}若$f^{-1}(\mathscr{B})\subseteq\mathscr{A}$,因为$\mathscr{A}$是一个$\sigma$域,由生成的定义可知$\sigma[f^{-1}(\mathscr{B})]\subseteq\mathscr{A}$。\par
	(2)因为$f$是可测空间$(X,\mathscr{A})$到可测空间$(Y,\mathscr{B})$的可测映射,所以$f^{-1}(\mathscr{B})\subseteq\mathscr{A}$。因为$g$是可测空间$(Y,\mathscr{B})$到可测空间$(Z,\mathscr{C})$的可测映射,所以$g^{-1}(\mathscr{C})\subseteq\mathscr{B}$。由\cref{theo:PropertyOfPreimage}(2)可得:
	\begin{equation*}
		(g\circ f)^{-1}(\mathscr{C})=f^{-1}[g^{-1}(\mathscr{C})]\subseteq f^{-1}(\mathscr{B})\subseteq\mathscr{A}
	\end{equation*}\par
	(3)由\cref{theo:PropertyOfPreimage}(1)可知$\nu(\varnothing)=\mu[f^{-1}(\varnothing)]=\mu(\varnothing)=0$。取$\mathscr{B}$中互不相交的$\{B_n\}$,根据映射的定义可知$\{f^{-1}(B_n)\}$也互不相交,所以由\cref{theo:PropertyOfPreimage}(4)可得:
	\begin{align*}
		\nu\left(\underset{n=1}{\overset{+\infty}{\cup}}A_n\right)&=\mu\left[f^{-1}\left(\underset{n=1}{\overset{+\infty}{\cup}}A_n\right)\right]=\mu\left[\underset{n=1}{\overset{+\infty}{\cup}}f^{-1}(A_n)\right] \\
		&=\sum_{n=1}^{+\infty}\mu[f^{-1}(A_n)]=\sum_{n=1}^{+\infty}\nu(A_n)
	\end{align*}
	于是$\nu$是$(Y,\mathscr{B})$上的测度。\par
	若$(X,\mathscr{A},\mu)$是概率空间,即$\mu(X)=1$,由\cref{theo:PropertyOfPreimage}(1)可知$\nu(Y)=\mu[f^{-1}(Y)]=\mu(X)=1$,所以$(Y,\mathscr{B},\nu)$也是一个概率空间。\par
	若$\mu$是有限测度,由$\nu$的定义即可得$\nu$也是有限测度。\par
\end{proof}
\begin{note}
	上述第三条性质是一个很重要的结论,$\nu$被称之为pushforward measure,在数理统计和积分的变量替换中有很多它的应用。
\end{note}

\subsection{可测函数}
\begin{definition}
	从可测空间$(X,\mathscr{A})$到可测空间$(\overline{\mathbb{R}},\mathcal{B}_{\overline{\mathbb{R}}})$的可测映射称为$(X,\mathscr{A})$上的\gls{MeasurableFunction}。特别的,从可测空间$(X,\mathscr{A})$到可测空间$(\mathbb{R},\mathcal{B})$的可测映射称为$(X,\mathscr{A})$上的有限值可测函数或\gls{r.v.}或Borel可测函数(简称Borel函数)。
\end{definition}
\subsubsection{简单函数}
\begin{definition}
	对于空间$X$,如果存在有限个互不相交的集合$\{A_i\subseteq X:i=1,2,\dots,n\}$满足:
	\begin{equation*}
		\underset{i=1}{\overset{n}{\cup}}A_i=X
	\end{equation*}
	则称$\{A_i\subseteq X:i=1,2,\dots,n\}$为$X$的一个\textbf{有限分割}。如果还有$A_i\in \mathscr{A},\;\forall\;i=1,2,\dots,n$,则称$\{A_i\subseteq X:i=1,2,\dots,n\}$为可测空间$(X,\mathscr{A})$的一个\textbf{有限可测分割}。当$\{A_n\in\mathscr{A}:n\in\mathbb{N}^+\}$是一个元素间互不相交的集族且满足:
	\begin{equation*}
		\underset{n=1}{\overset{+\infty}{\cup}}A_n=X,\quad A_n\in\mathscr{A},\;\forall\;n\in\mathbb{N}^+
	\end{equation*}
	时,称$\{A_n\}$为可测空间$(X,\mathscr{A})$的一个\textbf{可列可测分割}。
\end{definition}
\begin{definition}
	对于可测空间$(X,\mathscr{A})$上的函数$\varphi:X\rightarrow \mathbb{R}$,如果存在有限可测分割$\{A_i\in \mathscr{A}:i=1,2,\dots,n\}$和$\{a_i\in\mathbb{R}:i=1,2,\dots,n\}$使得:
	\begin{equation*}
		\varphi(x)=\sum_{i=1}^{n}a_iI_{A_i}(x)
	\end{equation*}
	其中$I_{A_i}(x)$为表示$x$是否在$A_i$中的指示函数,则称$\varphi(x)$为\gls{SimpleFunction}。
\end{definition}
\begin{property}\label{prop:SimpleFunction}
	简单函数具有如下性质:
	\begin{enumerate}
		\item 简单函数是可测函数;
		\item 设$\varphi,\psi$为可测空间$(X,\mathscr{A})$上的简单函数,可分别表示为:
		\begin{equation*}
			\varphi(x)=\sum_{i=1}^{m}c_iI_{E_i}(x),\quad
			\psi(x)=\sum_{j=1}^{n}d_jI_{F_j}(x)
		\end{equation*}
		则:
		\begin{enumerate}
			\item 对任意的$\alpha,\beta\in\mathbb{R}$,$\alpha\varphi+\beta\psi$是简单函数,且可以表示为:
			\begin{equation*}
				\alpha\varphi(x)+\beta\psi(x)=\sum_{i=1}^{m}\sum_{j=1}^{n}(\alpha c_i+\beta d_j)I_{E_i\cap F_j}(x)
			\end{equation*}
			\item $\varphi\psi$是简单函数,且可以表示为:
			\begin{equation*}
					\varphi(x)\psi(x)=\sum_{i=1}^{m}\sum_{j=1}^{n}c_id_jI_{E_i\cap F_j}(x)
			\end{equation*}
			\item 若$\varphi$的值域中不含$0$,则$\dfrac{\psi}{\varphi}$也是简单函数,且可以表示为:
			\begin{equation*}
				\frac{\psi}{\varphi}=\sum_{i=1}^{m}\sum_{j=1}^{n}\frac{d_j}{c_i}I_{E_i\cap F_j}(x)
			\end{equation*}
			\item $|\varphi|$是简单函数,且可以表示为:
			\begin{equation*}
				\varphi(x)=\sum_{i=1}^{m}|c_i|I_{E_i}(x)
			\end{equation*}
			\item $\max\{\varphi,\psi\},\min\{\varphi,\psi\}$是简单函数,且分别可以表示为:
			\begin{gather*}
				\max\{\varphi(x),\psi(x)\}=\sum_{i=1}^{m}\sum_{j=1}^{n}\max\{c_i,d_j\}I_{E_i\cap F_j}(x) \\
				\min\{\varphi(x),\psi(x)\}=\sum_{i=1}^{m}\sum_{j=1}^{n}\min\{c_i,d_j\}I_{E_i\cap F_j}(x)
			\end{gather*}
		\end{enumerate}
		\item 可测空间$(X,\mathscr{A})$上集合$A\in\mathscr{A}$的指示函数$I_A$是简单函数。
	\end{enumerate}
\end{property}
\begin{proof}
	(1)由可测函数的定义和\cref{prop:SigmaField}(3)即可得到。\par
	(2)\textbf{线性运算:}注意到:
	\begin{align*}
		\alpha\varphi(x)+\beta\psi(x)
		&=\alpha\sum_{i=1}^{m}c_iI_{E_i}(x)+\beta\sum_{j=1}^{n}d_jI_{F_j}(x) \\
		&=\alpha\sum_{i=1}^{m}c_i\sum_{j=1}^{n}I_{E_i\cap F_j}(x)+\beta\sum_{j=1}^{n}d_j\sum_{i=1}^{m}I_{E_i\cap F_j}(x) \\
		&=\sum_{i=1}^{m}\sum_{j=1}^{n}\alpha c_iI_{E_i\cap F_j}(x)+\sum_{j=1}^{n}\sum_{i=1}^{m}\beta d_jI_{E_i\cap F_j}(x) \\
		&=\sum_{i=1}^{m}\sum_{j=1}^{n}(\alpha c_i+\beta d_j)I_{E_i\cap F_j}(x)
	\end{align*}
	由\cref{prop:SigmaField}(2)、\cref{prop:SetOperation}(4)可得:
	\begin{equation*}
		\underset{i=1}{\overset{m}{\cup}}\underset{j=1}{\overset{n}{\cup}}(E_i\cap F_j)=\underset{i=1}{\overset{m}{\cup}}\left[E_i\cap\left(\underset{j=1}{\overset{n}{\cup}}F_j\right)\right]=\underset{i=1}{\overset{m}{\cup}}E_i=X
	\end{equation*}
	并且$\{E_i\cap F_j:i=1,2,\dots,m,\;j=1,2,\dots,n\}\subseteq\mathscr{A}$,所以它是$X$的有限可测分割,由此可知结论成立。\par
	\textbf{乘法:}注意到:
	\begin{equation*}
		\varphi(x)\psi(x)=\left[\sum_{i=1}^{m}c_iI_{E_i}(x)\right]\left[\sum_{j=1}^{n}d_jI_{F_j}(x)\right]=\sum_{i=1}^{m}\sum_{j=1}^{n}c_id_jI_{E_i\cap F_j}(x)
	\end{equation*}
	显然$\varphi\psi$是一个简单函数。\par
	\textbf{除法:}当$\varphi$的值域不含$0$时,$\dfrac{1}{\varphi}$是一个简单函数,由简单函数对乘法的封闭性即可得到:
	\begin{equation*}
		\frac{\psi}{\varphi}=\sum_{i=1}^{m}\sum_{j=1}^{n}\frac{d_j}{c_i}I_{E_i\cap F_j}(x)
	\end{equation*}\par
	\textbf{绝对值:}显然。\par
	\textbf{最大值最小值:}显然。\par
	(3)由$\sigma$域对补封闭即可得到:
	\begin{equation*}
		I_A(x)=1I_A(x)+0I_{A^c}(x)
	\end{equation*}
	符合简单函数的定义。
\end{proof}
\subsubsection{可测函数的性质}
\begin{definition}
	设$f(x)$是可测空间$(X,\mathscr{A})$上的可测函数,令:
	\begin{gather*}
		f^+(x)=\max\{f(x),0\}=
		\begin{cases}
			f(x),&f(x)\geqslant 0 \\
			0,&f(x)<0
		\end{cases} \\
		f^-(x)=-\min\{f(x),0\}=
		\begin{cases}
			-f(x),&f(x)\leqslant 0 \\
			0,&f(x)>0
		\end{cases}
	\end{gather*}
	分别称$f^+(x)$和$f^-(x)$为$f(x)$的正部和负部。
\end{definition}
下面用$\{f<a\}$表示$\{x:f(x)<a\}$,小于等于、大于、大于等于、等于号同理。
\begin{property}\label{prop:MeasurableFunction}
	设$(X,\mathscr{A})$是可测空间。可测函数具有如下性质:
	\begin{enumerate}
		\item $f$是$(X,\mathscr{A})$上的可测函数的充要条件为:
		\begin{enumerate}
			\item 对任意的$a\in\mathbb{R}$,$\{f<a\}\in\mathscr{A}$。
			\item 对任意的$a\in\mathbb{R}$,$\{f\leqslant a\}\in\mathscr{A}$。
			\item 对任意的$a\in\mathbb{R}$,$\{f>a\}\in\mathscr{A}$。
			\item 对任意的$a\in\mathbb{R}$,$\{f\geqslant a\}\in\mathscr{A}$。
		\end{enumerate}
		设$D$是$\mathbb{R}$上的可数稠密子集,对上述结论,把$\mathbb{R}$改为$D$仍然成立;
		\item 若$f$是$(X,\mathscr{A})$上的可测函数,则对任意的$a\in\overline{\mathbb{R}^{}}$,有$\{f=a\}\in\mathscr{A}$;
		\item 若$f$和$g$是$(X,\mathscr{A})$上的可测函数,则$\{f<g\},\{f\leqslant g\},\{f=g\}.\{f\ne g\}\in\mathscr{A}$,同时对于任意的$\varepsilon>0$,有$\{|f-g|\geqslant\varepsilon\},\{|f-g|>\varepsilon\}\in \mathscr{A}$。
		\item 如果$f$是$(X,\mathscr{A})$上的可测函数,则$f$为简单函数的充分必要条件为它的值域是有限个实数组成的集合;
		\item 若$f,g$是$(X,\mathscr{A})$上的可测函数,则\info{未完成}:
		\begin{enumerate}
			\item 对任意的$\alpha,\beta\in\overline{\mathbb{R}}$,若对任意的$x\in X$,$\alpha f+\beta g$有意义,则$\alpha f+\beta g$是可测函数;
			\item $fg$是可测函数;
			\item 若对任意的$x\in X$,有$g(x)\ne0$,则$f/g$是可测函数;
			\item $|f|$是可测函数;
		\end{enumerate}
		\item $\{f_n\}$是$(X,\mathscr{A})$上的一列可测函数,则:
		\begin{equation*}
			\inf_nf_n,\;\sup_nf_n,\;
			\varliminf_{n\to+\infty}f_n,\;\varlimsup_{n\to+\infty}f_n,\;\max_nf_n,\;\min_nf_n
		\end{equation*}
		也是可测函数,若$\varliminf\limits_{n\to+\infty}f_n=\varlimsup\limits_{n\to+\infty}f_n$,则$\lim\limits_{n\to+\infty}f_n$也是可测函数;
		\item 设$f$是$(X,\mathscr{A})$上的可测函数,则$f^+$和$f^-$也是可测函数;
		\item 若$f$是$(X,\mathscr{A})$上的(非负)可测函数,则存在(非负)简单函数列$\{\varphi_n\}$,使得对任意$x\in X$有$\lim\limits_{n\to+\infty}\varphi_n(x)=f(x)$($\varphi_n\uparrow f$)。若$f(x)$有界,则上述收敛可以是一致收敛;
		\item $(\mathbb{R},\mathcal{B})$上的实值单调函数是Borel函数;
		\item $(\mathbb{R},\mathcal{B})$上的实值连续函数是Borel函数;
		\item 设$f$是完全测度空间$(X,\mathscr{A},\mu)$上的可测函数,则改变$f$在$\mu$零测集上的值不改变$f$的可测性,让$f$在$\mu$零测集上无定义也不改变$f$的可测性。
	\end{enumerate}
\end{property}
\begin{proof}
	(1)由\cref{prop:BorelSigmaField}(1.c)中$\mathcal{B}_{\mathbb{R}}$的等价定义2以及\cref{prop:MeasurableMapping}(1)可得:
	\begin{align*}
		f\text{是可测函数}
		&\Leftrightarrow f^{-1}(\mathcal{B}_{\mathbb{R}})\subseteq\mathscr{A} \\
		&\Leftrightarrow\forall\;\alpha\in\mathbb{R},\;f^{-1}\Bigl((-\infty,\alpha)\Bigr)\in\mathscr{A} \\
		&\Leftrightarrow\forall\;\alpha\in\mathbb{R},\;\{f<a\}\in\mathscr{A}
	\end{align*}\par
	(1.b)(1.c)(1.d)同理,只需在上第一行到第二行的过程中使用\cref{prop:BorelSigmaField}(1.c)涉及到的对应的$\mathcal{B}_{\mathbb{R}}$的等价定义。$D$时的情况由\cref{prop:BorelSigmaField}即可得到\info{这里还没改,需要考虑Borelset改为稠密子集时的情况}。\par
	(2)实数的情况可由(1)与\cref{prop:SigmaField}(4)得到。由(1)和\cref{prop:SigmaField}(2)可得:
	\begin{equation*}
		\{f=+\infty\}=\underset{n=1}{\overset{+\infty}{\cap}}\{f>n\}\in\mathscr{A},\quad\{f=-\infty\}=\underset{n=1}{\overset{+\infty}{\cap}}\{f<-n\}\in\mathscr{A}
	\end{equation*}\par
	(3)由\info{有理数的稠密性}、(1)、\info{有理数集的可列性}和\cref{prop:SigmaField}(2)可得:
	\begin{equation*}
		\{f<g\}=\underset{r\in\mathbb{Q}}{\overset{}{\cup}}[\{f<r\}\cap\{g>r\}]\in\mathscr{A}
	\end{equation*}
	由$f$与$g$的对称性可得$\{g<f\}\in\mathscr{A}$,那么就有$\{f\leqslant g\}=\{g<f\}^c\in\mathscr{A}$。根据\cref{prop:SigmaField}(4)可得:
	\begin{align*}
		\{f=g\}=\{f\leqslant g\}\backslash\{f<g\}\in\mathscr{A}
	\end{align*}
	所以$\{f\ne g\}=\{f=g\}^c\in\mathscr{A}$。因为:
	\begin{equation*}
		\{|f-g|\geqslant\varepsilon\}=\{f-g\geqslant\varepsilon\}\cup\{f-g\leqslant-\varepsilon\}=\{f\geqslant g+\varepsilon\}\cup\{f\leqslant g-\varepsilon\}
	\end{equation*}
	因为$g$是可测函数,由(1)可得$g+\varepsilon$和$g-\varepsilon$也是可测函数,所以$\{f\geqslant g+\varepsilon\},\{f\leqslant g-\varepsilon\}\in\mathscr{A}$。由\cref{prop:SigmaField}(3)可得:
	\begin{equation*}
		\{|f-g|\geqslant\varepsilon\}=\{f\geqslant g+\varepsilon\}\cup\{f\leqslant g-\varepsilon\}\in \mathscr{A}
	\end{equation*}
	同理可得$\{|f-g|>\varepsilon\}\in\mathscr{A}$。\par
	(4)必要性由简单函数的定义即可立即得到。设$f$的值域为$\{a_i:i=1,2,\dots,n\}$,因为$f$是可测函数,由(1)和\cref{prop:SigmaField}(4)可得$\{f=a_i\}=\{f\leqslant a_i\}\backslash\{f<a_i\}\in\mathscr{A}$,于是$f$可表为:
	\begin{equation*}
		f(x)=\sum_{i=1}^{n}a_iI_{\{f=a_i\}}(x),\quad X=\underset{i=1}{\overset{n}{\cup}}\{f=a_i\},\quad\{f=a_i\}\cap\{f=a_j\}=\varnothing,\;\forall\;i\ne j
	\end{equation*}
	所以$f$为简单函数,充分性得证。\par
	(5)\textbf{ a:}\par
	(6)注意到:
	\begin{equation*}
		\left\{\inf_nf_n\geqslant a\right\}=\underset{n=1}{\overset{+\infty}{\cap}}\{f_n\geqslant a\},\quad\left\{\sup_nf_n\leqslant a\right\}=\underset{n=1}{\overset{+\infty}{\cap}}\{f_n\leqslant a\}
	\end{equation*}
	由(1)和\cref{prop:SigmaField}(2)即可得到$\inf\limits_nf_n,\sup\limits_nf_n$是可测函数。因为:
	\begin{equation*}
		\varliminf_{n\to+\infty}f_n=\sup_{n}\left(\inf_{k\geqslant n}f_k\right),\quad\varlimsup_{n\to+\infty}f_n=\inf_{n}\left(\sup_{k\geqslant n}f_k\right)
	\end{equation*}
	所以可测函数列的下极限函数与上极限函数也是可测函数,从而上下极限函数相等时极限函数也是可测函数。类似于上下确界可得:
	\begin{equation*}
		\left\{\min_nf_n\geqslant a\right\}=\underset{n=1}{\overset{+\infty}{\cap}}\{f_n\geqslant a\},\quad\left\{\max_nf_n\leqslant a\right\}=\underset{n=1}{\overset{+\infty}{\cap}}\{f_n\leqslant a\}
	\end{equation*}
	由(1)和\cref{prop:SigmaField}(2)即可得到$\max\limits_nf_n,\min\limits_nf_n$是可测函数。\par
	(7)由可测函数的定义,$g(x)=0$是一个可测函数。根据(5)可知此时$f^+(x),f^-(x)$是可测函数。\par
	(8)\textbf{非负可测函数:}对任意的$n\in\mathbb{N}^+$,将$[0,n]$分为$n2^n$份,令:
	\begin{gather*}
		E_{nj}=\left\{\frac{j-1}{2^n}\leqslant f<\frac{j}{2^n}\right\},\;j=1,2,\dots,n2^n ,\quad E_n=\{f\geqslant n\}
	\end{gather*}
	作函数列:
	\begin{equation*}
		\varphi_n(x)=
		\begin{cases}
			\dfrac{j-1}{2^n},&x\in E_{nj} \\
			n,&x\in E_n
		\end{cases}
	\end{equation*}
	由(1)和\cref{prop:SigmaField}(4)可得$\varphi_n(x)$是非负简单函数,并且有:
	\begin{equation*}
		\varphi_n(x)\leqslant\varphi_{n+1}(x)\leqslant f(x),\quad\forall\;n\in\mathbb{N}^+
	\end{equation*}
	设$x\in X$,若$f(x)<+\infty$,则当$n>f(x)$时有:
	\begin{equation*}
		0\leqslant f(x)-\varphi_n(x)\leqslant 2^{-n}
	\end{equation*}
	若$f(x)=+\infty$,则$\varphi_n(x)=n,\;n=1,2,\dots$,因此$\lim\limits_{n\to+\infty}\varphi_n(x)=f(x)$。\par
	\textbf{一般可测函数:}$f=f^+-f^-$,若$f$是可测函数,由(7)可知$f^+,f^-$也是可测函数,所以存在简单函数列$\{\varphi_n^+\}$和$\{\varphi_n^-\}$,使得对任意的$x\in X$,有:
	\begin{equation*}
		\lim\limits_{n\to+\infty}\varphi_n^+(x)=f^+(x),\;
		\lim\limits_{n\to+\infty}\varphi_n^-(x)=f^-(x)
	\end{equation*}
	令$\varphi_n=\varphi_n^+-\varphi_n^-$,由\cref{prop:SimpleFunction}(2)可知$\{\varphi_n(x)\}$是简单函数列,且对任意的$x\in X$,$\lim\limits_{n\to+\infty}\varphi_n(x)=f(x)$。\par
	若$f(x)$有界,设$\sup\limits_{x\in X}\{|f(x)|\}=M$,则由非负可测函数情况下的证明过程,当$n>M$时有:
	\begin{equation*}
		\sup_{x\in X}\{|f^+(x)-\varphi_n^+(x)|\}\leqslant\frac{1}{2^n},\quad
		\sup_{x\in X}\{|f^-(x)-\varphi_n^-(x)|\}\leqslant\frac{1}{2^n}
	\end{equation*}
	因此由\info{上确界的性质}可得:
	\begin{align*}
		\sup_{x\in X}\{|f(x)-\varphi_n(x)|\}
		&=\sup_{x\in X}\{|f^+(x)-f^-(x)-\varphi_n^+(x)+\varphi_n^-(x)|\} \\
		&\leqslant\sup_{x\in X}\{|f^+(x)-\varphi_n^+(x)|+|f^-(x)-\varphi_n^-(x)|\} \\
		&\leqslant\sup_{x\in X}\{|f^+(x)-\varphi_n^+(x)|\}+\sup_{x\in X}\{|f^-(x)-\varphi_n^-(x)|\} \\
		&\leqslant\frac{1}{2^{n-1}}
	\end{align*}
	所以$\{\varphi_n\}$在$X$上一致收敛于$f$。\par
	(9)由(1.a)和\cref{prop:BorelSigmaField}(1.c)立即可得。\par
	(10)由\cref{theo:ContinousMapO2OC2C}、\cref{prop:BorelSigmaField}(1.a)和\cref{prop:MeasurableMapping}(1)即可得到。\par
	(11)由(1)、完全测度空间的定义和\cref{prop:SigmaField}(3)(4)即可得到,在零测集上改变数值而言无非是对$\{f<a\}$并上或去掉一个可测集,让$f$在零测集上无定义则是对$\{f<a\}$去掉一个可测集。
\end{proof}
\begin{note}
	上述第11条性质是一个非常重要的结论,它告诉我们在讨论完全测度空间上函数的可测性的时候是不必计较其在一个零测集上的取值的。前面我们提到过在测度论中讨论的函数都是定义在整个空间$X$上的,现在我们将这个范围拓宽到除了$X$的一个零测集以外都有定义的函数。\par
	再重新检查一遍我们已经证明过的可测函数的性质,有哪些结论在完全测度空间中可以得到推广?性质5的a可以改为$\alpha f+\beta g$在一个零测集外有意义,c可以改为$g$在一个零测集外非零,性质9和性质10在\cref{prop:LSMeasure}(5)下可以推广为:$(\mathbb{R}^{},\mathscr{A}_{\lambda_F},\lambda_F)$上的实值函数$f$若在一个零测集外单调或连续,则$f$是Borel函数。在积分论中我们还会证明,改变可测函数在零测集上的值也不影响其积分值。多么美妙!\par
	在这里我们才看到完全测度空间的应用,我也很犹豫是否应将完全测度空间的介绍改变一下顺序,将其放在可测函数这里,我个人并不了解数学史,但我有理由相信完全测度空间的研究一定是在人们研究函数在零测集上的行为之后才开始的,任何科学研究必然具有其动机。\par
	仍然需要提醒各位的一件事情是,“完全测度空间使得所有零测集都可测”这句话是完全错误的,如果一个集合具有测度,它必然可测,不可测的集合哪儿来的测度?我们需要按照严格的数学定义进行学习。\par
	由于完全测度空间具有一系列优良性质,且根据\cref{theo:CompleteMeasureSpace}可知完全测度空间是广泛存在的,在接下来的讨论中我们提到的测度空间都有可能是完全测度空间。各位需要心中有数,哪些结论必须在完全测度空间上才有可能成立。
\end{note}
\begin{definition}
	对于测度空间$(X,\mathscr{A},\mu)$上关于$X$的元素$x$的一个命题,若它只在$\mathscr{A}$中的一个$\mu$零测集上不成立,则称这个命题\textbf{几乎处处成立},简记为a.e.(almost everywhere)。当同时存在多个空间、多个$\sigma$域或多个测度时,为了使得提到几乎处处的时候不引起歧义,应在a.e.后加入限定,如命题成立a.e.于$X$、命题成立a.e.于$(X,\mathscr{A},\mu)$、命题成立a.e.于$\mu$等等。
\end{definition}

\subsection{可测函数的收敛性}
\subsubsection{几乎处处收敛}
\begin{definition}
	设$\{f_n\}$和$f$是测度空间$(X,\mathscr{F},\mu)$上的可测函数,如果:
	\begin{equation*}
		\mu\left(\left\{\lim_{n\to+\infty}f_n\ne f\right\}\right)=0
	\end{equation*}
	则称可测函数列$\{f_n\}\;$a.e.以$f$为极限,记为$f_n\overset{\text{a.e.}}{\longrightarrow}f$。若此时还有$f$有限a.e.于$X$,则称$\{f_n\}\;$a.e.收敛到$f$。若$(X,\mathscr{F},P)$是概率空间,称$\{f_n\}\;$a.e.收敛到$f$为$\{f_n\}$几乎必然收敛到$f$,记作$f_n\overset{\text{a.s.}}{\longrightarrow}f$。
\end{definition}
\begin{note}
	接下来若没有特殊说明,$f_n\overset{a.e.}{\longrightarrow}f$指的都是$\{f_n\}\;$a.e.收敛到$f$。
\end{note}
\begin{theorem}\label{theo:EquiConditiona.e.}
	设$\{f_n\}$和$f$是测度空间$(X,\mathscr{F},\mu)$上的可测函数,$f_n\overset{a.e.}{\longrightarrow}f$的充分必要条件为对任意的$\varepsilon>0$,有\info{思考没有定义的情况}\footnote{从今往后,对任何的$\varepsilon>0$和$n\in\mathbb{N}^+$,$f_n(x)-f(x)$没有定义的$x\in X$也计入集合$\{|f_n-f|\geqslant\varepsilon\}$,在这些点上自然无法说$\lim\limits_{n\to+\infty}f_n$等于$f$。为简明起见,不对这些点的情况作特殊说明。}:
	\begin{equation*}
		\mu\left(\varlimsup_{n\to+\infty}\{|f_n-f|\geqslant\varepsilon\}\right)=0
	\end{equation*}
\end{theorem}
\begin{proof}
	若$\lim\limits_{n\to+\infty}f_n(x)\ne f(x)$,则有:
	\begin{equation*}
		\exists\;k\in\mathbb{N}^+,\;\forall\;N\in\mathbb{N}^+,\;\exists\;n\geqslant N,\;|f_n(x)-f(x)|\geqslant\frac{1}{k}
	\end{equation*}
	于是:
	\begin{equation*}
		\left\{\lim_{n\to+\infty}f_n\ne f\right\}=\underset{k=1}{\overset{+\infty}{\cup}}\underset{m=1}{\overset{+\infty}{\cap}}\underset{n=m}{\overset{+\infty}{\cup}}\left\{|f_n-f|\geqslant\frac{1}{k}\right\}
	\end{equation*}\par
	\textbf{(1)充分性:}此时由\cref{prop:Measure}(3)可得:
	\begin{align*}
		\mu\left(\left\{\lim_{n\to+\infty}f_n\ne f\right\}\right)
		&=\mu\left(\underset{k=1}{\overset{+\infty}{\cup}}\underset{m=1}{\overset{+\infty}{\cap}}\underset{n=m}{\overset{+\infty}{\cup}}\left\{|f_n-f|\geqslant\frac{1}{k}\right\}\right) \\
		&\leqslant\sum_{k=1}^{+\infty}\mu\left(\underset{m=1}{\overset{+\infty}{\cap}}\underset{n=m}{\overset{+\infty}{\cup}}\left\{|f_n-f|\geqslant\frac{1}{k}\right\}\right) \\
		&=\sum_{k=1}^{+\infty}\mu\left(\varlimsup_{n\to+\infty}\left\{|f_n-f|\geqslant\frac{1}{k}\right\}\right)=0
	\end{align*}
	由测度的非负性即可得到:
	\begin{equation*}
		\mu\left(\lim_{n\to+\infty}f_n\ne f\right)=0
	\end{equation*}\par
	\textbf{(2)必要性:}对任意取定的$\varepsilon>0$,若:
	\begin{equation*}
		x\in\varlimsup_{n\to+\infty}\{|f_n-f|\geqslant\varepsilon\}
	\end{equation*}
	则:
	\begin{equation*}
		\forall\;m\in\mathbb{N}^+,\;\exists\;n\geqslant m,\;|f_n(x)-f(x)|\geqslant \varepsilon
	\end{equation*}
	即:
	\begin{equation*}
		\lim_{n\to+\infty}f_n(x)\ne f(x)
	\end{equation*}
	于是对这个$\varepsilon$,有:
	\begin{equation*}
		\varlimsup_{n\to+\infty}\{|f_n-f|\geqslant\varepsilon\}\subseteq\left\{\lim_{n\to+\infty}f_n\ne f\right\}
	\end{equation*}
	因为$f_n,f$都是可测函数,由\cref{prop:MeasurableFunction}(3)可知$\{|f_n-f|\geqslant\varepsilon\}\in\mathscr{F}$,根据\cref{prop:SigmaField}(2)可得:
	\begin{equation*}
		\varlimsup_{n\to+\infty}\{|f_n-f|\geqslant\varepsilon\}=\underset{m=1}{\overset{+\infty}{\cap}}\underset{n=m}{\overset{+\infty}{\cup}}\{|f_n-f|\geqslant\varepsilon\}\in \mathscr{F}
	\end{equation*}
	由\cref{prop:Measure}(3)可得:
	\begin{equation*}
		\mu\left(\varlimsup_{n\to+\infty}\{|f_n-f|\geqslant\varepsilon\}\right)=0\qedhere
	\end{equation*}
\end{proof}
\subsubsection{几乎一致收敛}
\begin{definition}
	设$\{f_n\}$和$f$是测度空间$(X,\mathscr{F},\mu)$上的可测函数。如果对任意的$\varepsilon>0$,存在$A\in \mathscr{F}$使得$\mu(A)<\varepsilon$且:
	\begin{equation*}
		\lim_{n\to+\infty}\sup_{x\notin A}|f_n(x)-f(x)|=0
	\end{equation*}
	则称$\{f_n\}$几乎一致收敛到$f$,记为$f_n\overset{\text{a.u.}}{\longrightarrow}f$。
\end{definition}
\begin{theorem}\label{theo:EquiConditiona.u.}
	设$\{f_n\}$和$f$是测度空间$(X,\mathscr{F},\mu)$上的可测函数,$f_n\overset{\text{a.u.}}{\longrightarrow}f$的充分必要条件为对任意的$\varepsilon$有:
	\begin{equation*}
		\lim_{m\to+\infty}\mu\left(\underset{n=m}{\overset{+\infty}{\cup}}\{|f_n-f|\geqslant\varepsilon\}\right)=0
	\end{equation*}
\end{theorem}
\begin{proof}
	\textbf{(1)必要性:}因为$\{f_n\}\overset{\text{a.u.}}{\longrightarrow}f$,所以:
	\begin{equation*}
		\forall\;\delta>0,\;\exists\;A\in \mathscr{F},\;\mu(A)<\delta,\;\forall\;\varepsilon>0,\;\exists\;m\in \mathbb{N}^+,\;\forall\;n>m,\;\sup_{x\notin A}|f_n(x)-f(x)|<\varepsilon
	\end{equation*}
	即:
	\begin{equation*}
		\forall\;\delta>0,\;\exists\;A\in \mathscr{F},\;\mu(A)<\delta,\;\forall\;\varepsilon>0,\;\exists\;m\in\mathbb{N}^+,\;A^c\subseteq\underset{n=m}{\overset{+\infty}{\cap}}\{|f_n-f|<\varepsilon\}
	\end{equation*}
	于是由\cref{prop:SetOperation}(7)可得:
	\begin{equation*}
		\forall\;\delta>0,\;\exists\;A\in \mathscr{F},\;\mu(A)<\delta,\;\forall\;\varepsilon>0,\;\exists\;m\in\mathbb{N}^+,\;\underset{n=m}{\overset{+\infty}{\cup}}\{|f_n-f|\geqslant\varepsilon\}\subseteq A
	\end{equation*}
	因为$f_n$和$f$都是可测函数,由\cref{prop:MeasurableFunction}(3)可知$\{|f_n-f|\geqslant\varepsilon\}\in \mathscr{F}$。因为$\mathscr{F}$是一个$\sigma$域,对可列并封闭,所以:
	\begin{equation*}
		\underset{n=m}{\overset{+\infty}{\cup}}\{|f_n-f|\geqslant\varepsilon\}\in \mathscr{F}
	\end{equation*}
	由\cref{prop:Measure}(3)可得:
	\begin{equation*}
		\mu\left(\underset{n=m}{\overset{+\infty}{\cup}}\{|f_n-f|\geqslant\varepsilon\}\right)\leqslant\mu(A)<\delta
	\end{equation*}
	所以:
	\begin{equation*}
		\forall\;\delta>0,\;\forall\;\varepsilon>0,\;\exists\;m\in\mathbb{N}^+,\;\mu\left(\underset{n=m}{\overset{+\infty}{\cup}}\{|f_n-f|\geqslant\varepsilon\}\right)<\delta
	\end{equation*}
	即:
	\begin{equation*}
		\forall\;\varepsilon>0,\;\forall\;\delta>0,\;\exists\;m\in\mathbb{N}^+,\;\mu\left(\underset{n=m}{\overset{+\infty}{\cup}}\{|f_n-f|\geqslant\varepsilon\}\right)<\delta
	\end{equation*}
	取$\delta=\dfrac{1}{k}$,则:
	\begin{equation*}
		\forall\;\varepsilon>0,\;\exists\;m_k\in\mathbb{N}^+,\;\mu\left(\underset{n=m}{\overset{+\infty}{\cup}}\{|f_n-f|\geqslant\varepsilon\}\right)<\frac{1}{k}
	\end{equation*}
	令$k\to+\infty$,结合测度的非负性就有:
	\begin{equation*}
		\forall\;\varepsilon>0,\;\lim_{m\to+\infty}\mu\left(\underset{n=m}{\overset{+\infty}{\cup}}\{|f_n-f|\geqslant\varepsilon\}\right)=0
	\end{equation*}\par
	\textbf{(2)充分性:}对任意的$\delta>0$,由所给条件,对$k\in\mathbb{N}^+$有:
	\begin{equation*}
		\lim_{m\to+\infty}\mu\left(\underset{n=m}{\overset{+\infty}{\cup}}\left\{|f_n-f|\geqslant\frac{1}{k}\right\}\right)=0	
	\end{equation*}
	于是存在$\{m_k\}$使得:
	\begin{equation*}
		\mu\left(\underset{n=m_k}{\overset{+\infty}{\cup}}\left\{|f_n-f|\geqslant\frac{1}{k}\right\}\right)<\frac{\delta}{2^k}
	\end{equation*}
	取:
	\begin{equation*}
		A=\underset{k=1}{\overset{+\infty}{\cup}}\underset{n=m_k}{\overset{+\infty}{\cup}}\left\{|f_n-f|\geqslant\frac{1}{k}\right\}
	\end{equation*}
	由\cref{prop:MeasurableFunction}(3)可得$A\in \mathscr{F}$,于是根据\cref{prop:Measure}(3)可得:
	\begin{align*}
		\mu(A)&=\mu\left(\underset{k=1}{\overset{+\infty}{\cup}}\underset{n=m_k}{\overset{+\infty}{\cup}}\left\{|f_n-f|\geqslant\frac{1}{k}\right\}\right) \\
		&\leqslant\sum_{k=1}^{+\infty}\mu\left(\underset{n=m_k}{\overset{+\infty}{\cup}}\left\{|f_n-f|\geqslant\frac{1}{k}\right\}\right)<\delta
	\end{align*}
	注意到:
	\begin{align*}
		A^c&=\left(\underset{k=1}{\overset{+\infty}{\cup}}\underset{n=m_k}{\overset{+\infty}{\cup}}\left\{|f_n-f|\geqslant\frac{1}{k}\right\}\right)^c=\underset{k=1}{\overset{+\infty}{\cap}}\left(\underset{n=m_k}{\overset{+\infty}{\cup}}\left\{|f_n-f|\geqslant\frac{1}{k}\right\}\right)^c \\
		&=\underset{k=1}{\overset{+\infty}{\cap}}\underset{n=m_k}{\overset{+\infty}{\cap}}\left\{|f_n-f|<\frac{1}{k}\right\}
	\end{align*}
	所以若$x\notin A$,则对任意的$k\in \mathbb{N}^+$,当$n\geqslant m_k$时就有:
	\begin{equation*}
		|f_n(x)-f(x)|<\frac{1}{k}
	\end{equation*}
	由上确界的不等式性,此时即:
	\begin{equation*}
		\forall\;k\in\mathbb{N}^+,\;\exists\;m_k\in\mathbb{N}^+,\;\forall\;n\geqslant m_k,\;\sup_{x\notin A}|f_n(x)-f(x)|\leqslant\frac{1}{k}
	\end{equation*}
	也即:
	\begin{equation*}
		\lim_{n\to+\infty}\sup_{x\notin A}|f_n(x)-f(x)|=0
	\end{equation*}
	所以$f_n\overset{\text{a.u.}}{\longrightarrow}f$。
\end{proof}
\subsubsection{依测度收敛}
\begin{definition}
	设$\{f_n\}$和$f$时测度空间$(X,\mathscr{F},\mu)$上的可测函数。如果对任意的$\varepsilon>0$都有:
	\begin{equation*}
		\lim_{n\to+\infty}\mu(\{|f_n-f|\geqslant\varepsilon\})=0
	\end{equation*}
	则称可测函数列$\{f_n\}$\gls{ConvergentInMeasure}到$f$,记为$f_n\overset{\mu}{\longrightarrow}f$。若$(X,\mathscr{F},P)$是概率空间,称$f_n\overset{P}{\longrightarrow}f$为$\{f_n\}$\gls{ConvergentInProbability}到$f$。
\end{definition}
\subsubsection{依分布收敛}
\begin{definition}
	若准分布函数$F$还满足:
	\begin{equation*}
		\lim_{x\to-\infty}F(x)=0,\;\lim_{x\to+\infty}F(x)=1
	\end{equation*}
	则称$F$为\gls{d.f.}。
\end{definition}
\begin{theorem}
	设$f$是概率空间$(X,\mathscr{F},P)$上的随机变量,对任意的$x\in\mathbb{R}$,令:
	\begin{equation*}
		F(x)=P(\{f\leqslant x\})
	\end{equation*}
	则$F$是一个分布函数。
\end{theorem}
\begin{proof}
	(1)由\cref{prop:Measure}(3)可得$F(x)$是一个非降的实值函数。\par
	(2)注意到:
	\begin{equation*}
		\{f\leqslant x\}=\lim_{n\to+\infty}\left\{f\leqslant x+\frac{1}{n}\right\}
	\end{equation*}
	考虑集族:
	\begin{equation*}
		\left\{X_n=\left\{f\leqslant x+\frac{1}{n}\right\}\right\}
	\end{equation*}
	则$\{X_n\}$是一个单调不增的集族,有:
	\begin{equation*}
		\lim_{n\to+\infty}X_n=\underset{n=1}{\overset{+\infty}{\cap}}X_n=\underset{n=1}{\overset{+\infty}{\cap}}\left\{f\leqslant x+\frac{1}{n}\right\}
	\end{equation*}
	因为$P(X=1)$,由\cref{prop:Measure}(3)可得$P(X_1)<+\infty$,所以根据\cref{prop:Measure}(3)可得:
	\begin{align*}
		F(x)&=P(\{f\leqslant x\})
		=P\left(\lim_{n\to+\infty}\left\{f\leqslant x+\frac{1}{n}\right\}\right)
		=P\left(\lim_{n\to+\infty}X_n\right) \\
		&=\lim_{n\to+\infty}P(X_n)
		=\lim_{n\to+\infty}P\left(\left\{f\leqslant x+\frac{1}{n}\right\}\right)
		=\lim_{h\to+0}F(x+h)
	\end{align*}
	于是$F(x)$是右连续的。\par
	(3)注意到$f$是一个实值函数,由\cref{prop:Measure}(3)可得:
	\begin{gather*}
		\lim_{x\to-\infty}F(x)=\lim_{x\to-\infty}P(f\leqslant x)=P(f\leqslant-\infty)=P(\varnothing)=0 \\
		\lim_{x\to+\infty}F(x)=\lim_{x\to+\infty}P(f\leqslant x)=P(f\leqslant+\infty)=P(X)=1\qedhere
	\end{gather*}
\end{proof}
\begin{definition}
	设$f$是概率空间$(X,\mathscr{F},P)$上的随机变量,$F$是一个分布函数。若对任意的$x\in\mathbb{R}$,都有:
	\begin{equation*}
		F(x)=P(\{f\leqslant x\})
	\end{equation*}
	则称r.v.$\;f$的分布函数是$F$,也说成$f$服从$F$,记为$f\sim F$。若$f$是从概率空间$(X,\mathscr{F},P)$到可测空间$(Y,\mathscr{A})$的可测映射,称:
	\begin{equation*}
		P(f^{-1}A),\;\forall\;A\in \mathscr{A}
	\end{equation*}
	为$f$的\gls{ProbabilityDistribution},简记为$Pf^{-1}$。在没有明确可测映射的场合,也称概率测度$P$为概率分布。
\end{definition}
\begin{note}
	可以看出随机变量的分布函数为其概率分布在$\{(-\infty,a]:a\in\mathbb{R}^{}\}$上的取值。
\end{note}
\begin{definition}
	对非降实值函数$\{f_n\}$和$f$,若对$f$的每一个连续点$x$有:
	\begin{equation*}
		\lim_{n\to+\infty}f_n(x)=f(x)
	\end{equation*}
	则称$f_n$\textbf{弱收敛}于$f$,记作$f_n\overset{w}{\longrightarrow}f$。
\end{definition}
\begin{property}
	设$F$是一个分布函数,则:
	\begin{enumerate}
		\item 必存在一个概率空间$(X,\mathscr{F},P)$和其上的随机变量$f$使得$f\sim F$;
		\item 若分布函数列$F_n\overset{w}{\longrightarrow}F$,则$F^{\leftarrow}_n\overset{w}{\longrightarrow}F^{\leftarrow}$。
	\end{enumerate}
\end{property}
\begin{proof}
	(1)\info{未完成}
\end{proof}
\begin{definition}
	设$\{f_n\sim F_n\}$是概率空间$(X,\mathscr{F},P)$上的随机变量序列,$F$是一个分布函数。若$F_n\overset{w}{\longrightarrow}F$,则称$\{f_n\}$\gls{ConvergentInDistribution},记为$f_n\overset{d}{\longrightarrow}F$。若此时概率空间$(X,\mathscr{F},P)$上的随机变量$f\sim F$,则称随机变量序列$\{f_n\}$依分布收敛到$f$,记为$f_n\overset{d}{\longrightarrow}f$。
\end{definition}

\subsubsection{收敛性之间的关系}
\begin{theorem}\label{theo:a.e.a.u.mu.d}
	设$\{f_n\}$和$f$为测度空间$(X,\mathscr{F},\mu)$上的可测函数,则:
	\begin{enumerate}
		\item $f_n\overset{\text{a.u.}}{\longrightarrow}f$可推出$f_n\overset{\textbf{a.e.}}{\longrightarrow}f$和$f_n\overset{\mu}{\longrightarrow}f$;
		\item 若$\mu(X)<+\infty$,则$f_n\overset{\text{a.u.}}{\longrightarrow}f\iff f_n\overset{\text{a.e.}}{\longrightarrow}f\Rightarrow f_n\overset{\mu}{\longrightarrow}f$;
		\item $f_n\overset{\mu}{\longrightarrow}f$当且仅当对$\{f_n\}$的任一子列,存在该子列的子列$\{f_{n_k}\}$使得$f_{n_k}\overset{\text{a.u.}}{\longrightarrow}f$;
		\item $f_n\overset{P}{\longrightarrow}f$可推出$f_n\overset{d}{\longrightarrow}f$;
		\item 当$f$为常值函数时,$f_n\overset{P}{\longrightarrow}f\iff f_n\overset{d}{\longrightarrow}f$
	\end{enumerate}
\end{theorem}
\begin{proof}
	(1)对任意的$n\in\mathbb{N}^+$和任意的$\varepsilon>0$,由\cref{prop:MeasurableFunction}(3)可得$\{|f_n-f|\geqslant\varepsilon\}$是可测集,由\cref{prop:SigmaField}(2)可得:
	\begin{equation*}
		\underset{m=n}{\overset{+\infty}{\cup}}\{|f_m-f|\geqslant\varepsilon\}\in \mathscr{F},\;\underset{n=1}{\overset{+\infty}{\cap}}\underset{m=n}{\overset{+\infty}{\cup}}\{|f_m-f|\geqslant\varepsilon\}\in \mathscr{F}
	\end{equation*}
	因为:
	\begin{gather*}
		\{|f_n-f|\geqslant\varepsilon\}\subseteq\underset{m=n}{\overset{+\infty}{\cup}}\{|f_m-f|\geqslant\varepsilon\} \\
		\underset{n=1}{\overset{+\infty}{\cap}}\underset{m=n}{\overset{+\infty}{\cup}}\{|f_m-f|\geqslant\varepsilon\}\subseteq\underset{m=n}{\overset{+\infty}{\cup}}\{|f_m-f|\geqslant\varepsilon\}
	\end{gather*}
	由\cref{prop:Measure}(3)可得:
	\begin{gather*}
		\mu(\{|f_n-f|\geqslant\varepsilon\})\leqslant\mu\left(\underset{m=n}{\overset{+\infty}{\cup}}\{|f_m-f|\geqslant\varepsilon\}\right) \\
		\mu\left(\underset{n=1}{\overset{+\infty}{\cap}}\underset{m=n}{\overset{+\infty}{\cup}}\{|f_m-f|\geqslant\varepsilon\}\right)\leqslant\mu\left(\underset{m=n}{\overset{+\infty}{\cup}}\{|f_m-f|\geqslant\varepsilon\}\right) 
	\end{gather*}
	因为$f_n\overset{\text{a.u.}}{\longrightarrow}f$,由极限的不等式性和测度的非负性可得:
	\begin{gather*}
		0\leqslant\lim_{n\to+\infty}\mu(\{|f_n-f|\geqslant\varepsilon\})\leqslant\lim_{n\to+\infty}\mu\left(\underset{m=n}{\overset{+\infty}{\cup}}\{|f_m-f|\geqslant\varepsilon\}\right)=0 \\
		0\leqslant\mu\left(\underset{n=1}{\overset{+\infty}{\cap}}\underset{m=n}{\overset{+\infty}{\cup}}\{|f_m-f|\geqslant\varepsilon\}\right)\leqslant\lim_{n\to+\infty}\mu\left(\underset{m=n}{\overset{+\infty}{\cup}}\{|f_m-f|\geqslant\varepsilon\}\right)=0
	\end{gather*}
	所以:
	\begin{equation*}
		\lim_{n\to+\infty}\mu(\{|f_n-f|\geqslant\varepsilon\})=0,\;\mu\left(\underset{n=1}{\overset{+\infty}{\cap}}\underset{m=n}{\overset{+\infty}{\cup}}\{|f_m-f|\geqslant\varepsilon\}\right)=0
	\end{equation*}
	即$f_n\overset{\mu}{\longrightarrow}f,\;f_n\overset{a.e.}{\longrightarrow}f$。\par
	(2)设$f_n\overset{a.e.}{\longrightarrow}f$,令:
	\begin{equation*}
		\left\{X_n=\underset{m=n}{\overset{+\infty}{\cup}}\{|f_m-f|\geqslant\varepsilon\}\right\}
	\end{equation*}
	显然$\{X_n\}$是一个单调不增序列,其极限为:
	\begin{equation*}
		\lim_{n\to+\infty}X_n=\underset{n=1}{\overset{+\infty}{\cap}}\underset{m=n}{\overset{+\infty}{\cup}}\{|f_m-f|\geqslant\varepsilon\}
	\end{equation*}
	由\cref{prop:Measure}(3)和$\mu(X)<+\infty$可知$\mu(X_1)<+\infty$并且有:
	\begin{equation*}
		\mu\left(\lim_{n\to+\infty}X_n\right)=\lim_{n\to+\infty}\mu(X_n)
	\end{equation*}
	于是:
	\begin{equation*}
		\mu\left(\underset{n=1}{\overset{+\infty}{\cap}}\underset{m=n}{\overset{+\infty}{\cup}}\{|f_m-f|\geqslant\varepsilon\}\right)=\lim_{n\to+\infty}\mu\left(\underset{m=n}{\overset{+\infty}{\cup}}\{|f_m-f|\geqslant\varepsilon\}\right)
	\end{equation*}
	所以:
	\begin{equation*}
		\lim_{n\to+\infty}\mu\left(\underset{m=n}{\overset{+\infty}{\cup}}\{|f_m-f|\geqslant\varepsilon\}\right)=0
	\end{equation*}
	即$f_n\overset{a.u.}{\longrightarrow}f$,再结合(1)即可得出结论。\par
	(3)\textbf{必要性:}因为$f_n\overset{\mu}{\longrightarrow}f$,所以对$k\in\mathbb{N}^+$可取$n_k$使得:
	\begin{equation*}
		\mu\left(\left\{|f_{n_k}-f|\geqslant\frac{1}{k}\right\}\right)<\frac{1}{2^k}
	\end{equation*}
	由\cref{prop:Measure}(3)可得:
	\begin{equation*}
		\mu\left(\underset{k=m}{\overset{+\infty}{\cup}}\left\{|f_{n_k}-f|\geqslant\frac{1}{k}\right\}\right)\leqslant\sum_{k=m}^{+\infty}\mu\left(\left\{|f_{n_k}-f|\geqslant\frac{1}{k}\right\}\right)<\sum_{k=m}^{+\infty}\frac{1}{2^k}=\frac{1}{2^{m-1}}
	\end{equation*}
	所以:
	\begin{equation*}
		\lim_{m\to+\infty}\mu\left(\underset{k=m}{\overset{+\infty}{\cup}}\left\{|f_{n_k}-f|\geqslant\frac{1}{k}\right\}\right)=0
	\end{equation*}
	即$f_{n_k}\overset{a.u.}{\longrightarrow}f$。\par
	\textbf{充分性:}若此时$f_n\overset{\mu}{\longrightarrow}f$不成立,由\info{收敛则所有子列收敛}可知存在子列$\{f_{n_k}\}$不依测度收敛于$f$,即存在$\varepsilon_0,\delta_0>0$,对于任意的$K\in\mathbb{N}^+$,存在$m>K$使得:
	\begin{equation*}
		\mu(\{|f_{n_m}-f|\geqslant\varepsilon_0\})\geqslant\delta_0
	\end{equation*}
	那么对于$\{f_{n_k}\}$的任一子列$\{f_{n_{k_i}}\}$,对于任意的$I\in\mathbb{N}^+$,存在$j>I$使得:
	\begin{equation*}
		\mu(\{|f_{n_{k_j}}-f|\geqslant\varepsilon_0\})\geqslant\delta_0
	\end{equation*}
	由\cref{prop:Measure}(3)可得:
	\begin{equation*}
		\mu\left(\underset{i=I}{\overset{+\infty}{\cup}}\{|f_{n_{k_i}}-f|\geqslant\varepsilon_0\}\right)\geqslant\mu(\{|f_{n_{k_j}}-f|\geqslant\varepsilon_0\})\geqslant\delta_0
	\end{equation*}
	所以$\{f_{n_{k_i}}\}$不几乎一致收敛。由$\{f_{n_{k_i}}\}$的任意性可知与条件矛盾,所以此时有$f_n\overset{\mu}{\longrightarrow}f$。\par
	(4)记$F$为$f$的分布函数,$F_n$为$f_n$的分布函数。因为$f_n\overset{P}{\longrightarrow}f$,所以对任意的$\varepsilon>0$,有:
	\begin{equation*}
		\lim_{n\to+\infty}P(|f_n-f|\geqslant\varepsilon)=0
	\end{equation*}\par
	对任意的$x\in X$、任意的$\varepsilon>0$和任意的$n\in \mathbb{N}^+$,由\cref{prop:Measure}(3)可得:
	\begin{align*}
		F_n(x)&=P(f_n\leqslant x) \\
		&\leqslant P(f_n\leqslant x,|f_n-f|<\varepsilon)+P(f_n\leqslant x,|f_n-f|\geqslant\varepsilon) \\
		&\leqslant P(f\leqslant x+\varepsilon)+P(|f_n-f|\geqslant\varepsilon)
	\end{align*}
	第二行到第三行第一式的变化是因为:
	\begin{equation*}
		|f_n-f|<\varepsilon\Leftrightarrow-\varepsilon<f_n-f<\varepsilon\rightarrow f<f_n+\varepsilon
	\end{equation*}
	而$f_n\leqslant x$,于是变为$f\leqslant x+\varepsilon$。由极限的不等式性可得:
	\begin{equation*}
		\lim_{n\to+\infty}F_n(x)\leqslant P(f\leqslant x+\varepsilon)+\lim_{n\to+\infty}P(|f_n-f|\geqslant\varepsilon)=P(f\leqslant x+\varepsilon)
	\end{equation*}
	由$\varepsilon$的任意性可得:
	\begin{equation*}
		\lim_{n\to+\infty}F_n(x)\leqslant F(x)
	\end{equation*}\par
	对任意的$x\in X$、任意的$\varepsilon>0$和任意的$n\in \mathbb{N}^+$,又有:
	\begin{align*}
		P(f\leqslant x-\varepsilon)&\leqslant P(f\leqslant x-\varepsilon,|f_n-f|<\varepsilon)+P(f\leqslant x-\varepsilon,|f_n-f|\geqslant \varepsilon) \\
		&\leqslant P(f_n\leqslant x)+P(|f_n-f|\geqslant\varepsilon)
	\end{align*}
	由极限的不等式性可得:
	\begin{equation*}
		P(f\leqslant x-\varepsilon)\leqslant\lim_{n\to+\infty}P(f_n\leqslant x)+\lim_{n\to+\infty}P(|f_n-f|\geqslant\varepsilon)=\lim_{n\to+\infty}F_n(x)
	\end{equation*}
	于是:
	\begin{equation*}
		P(f\leqslant x-0)\leqslant\lim_{n\to+\infty}F_n(x)
	\end{equation*}
	若$F(x)$在$x$处连续,就有:
	\begin{equation*}
		P(f\leqslant x-0)=F(x-0)=F(x)
	\end{equation*}
	所以:
	\begin{equation}
		\lim_{n\to+\infty}F_n(x)=F(x)
	\end{equation}
	即$f_n\overset{d}{\longrightarrow}f$。\par
	(5)设$f=c$且$f_n\overset{d}{\longrightarrow}f$,对任意的$\varepsilon>0$,由\cref{prop:Measure}(3)(有限可加性)可得:
	\begin{equation*}
		P(\{|f_n-f|\geqslant\varepsilon\})=1-P(\{|f_n-f|\leqslant\varepsilon\})=1-P(\{-\varepsilon\leqslant f_n-f\leqslant\varepsilon\})=1-F_n(c+\varepsilon)+F_n(c-\varepsilon)
	\end{equation*}
	因为$f$为常值函数,所以$F$在除$c$点以外的点都连续。由依分布收敛的定义和分布函数的右连续性可得:
	\begin{equation*}
		\lim_{\varepsilon\to0+}\left[\lim_{n\to+\infty}F_n(c+\varepsilon)\right]=F(c)=1,\quad\lim_{\varepsilon\to0+}\left[\lim_{n\to+\infty}F_n(c-\varepsilon)\right]=0
	\end{equation*}
	根据\info{极限的线性性质}可得:
	\begin{equation*}
		\lim_{n\to+\infty}P(\{|f_n-f|\geqslant\varepsilon\})=0
	\end{equation*}
	即$f_n\overset{P}{\longrightarrow}f$。结合(4)即可得出结论。
\end{proof}
