\section{不定积分}

\subsection{符号测度}
\begin{definition}
	设$(X,\mathscr{F})$是一个可测空间,若从$\mathscr{F}$到$\overline{\mathbb{R}}$的集函数$\varphi$满足:
	\begin{enumerate}
		\item $\varphi(\varnothing)=0$;
		\item $\varphi$具有可列可加性。
	\end{enumerate}
	则称$\varphi$	为\gls{SignedMeasure}。若对任意的$A\in\mathscr{F}$有$|\varphi(A)|<+\infty$,则称$\varphi$是有限的;若存在$X$的可列可测分割$\{A_n\}\subseteq\mathscr{F}$满足对任意的$n\in\mathbb{N}^+$有$|\varphi(A_n)|<+\infty$,则称$\varphi$是$\sigma$有限的。
\end{definition}
\begin{property}\label{prop:SignedMeasure}
	设$(X,\mathscr{F})$是一个可测空间,其上的符号测度$\varphi$具有如下性质:
	\begin{enumerate}
		\item $\varphi$具有有限可加性;
		\item $\varphi$只可能出现以下两种情况中的一种\footnote{下面所涉及的结论都是关于第一种情况的,对于第二种情况只需取$-\varphi$即可得到相关结果。}:
		\begin{gather*}
			\forall\;A\in\mathscr{F},\;-\infty<\varphi(A)\leqslant+\infty \\
			\forall\;A\in\mathscr{F},\;-\infty\leqslant\varphi(A)<+\infty
		\end{gather*}
		\item 若$A,B\in\mathscr{F},\;B\subseteq A$且$|\varphi(A)|<+\infty$,则$|\varphi(B)|<+\infty$;
		\item 若$\{A_n\}\subseteq\mathscr{F}$两两不交且满足:
		\begin{equation*}
			\left|\varphi\left(\underset{n=1}{\overset{+\infty}{\cup}}A_n\right)\right|<+\infty
		\end{equation*}
		则有:
		\begin{equation*}
			\sum_{n=1}^{+\infty}|\varphi(A_n)|<+\infty
		\end{equation*}
		\item 对$a\in\mathbb{R}^{}$,定义$(a\varphi)(A)=a\varphi(A),\;\forall\;A\in\mathscr{F}$,则$a\varphi$也是一个符号测度;若$\psi$也是$(X,\mathscr{F})$上的符号测度,则$\varphi+\psi$也是一个符号测度\footnote{从此处开始,提及符号测度的和的时候不考虑和符号测度不存在的情况,即不考虑两个符号测度在同一集合$A\in\mathscr{F}$上取异号极限。};
		\item 设$\mu,\nu$为$(X,\mathscr{F})$上的两个测度。只要$\mu,\nu$中有一个是有限的,则可定义符号测度:
		\begin{equation*}
			\forall\;A\in\mathscr{F},\;(\mu-\nu)(A)\coloneq\mu(A)-\nu(A)
		\end{equation*}
		\item $\varphi$在$\mathscr{F}$的子$\sigma$域$\mathscr{A}$上也是一个符号测度。
	\end{enumerate}
\end{property}
\begin{proof}
	(1)由符号测度的定义显然可得。\par
	(2)设$A,B\in\mathscr{F}$且$\varphi(A)=+\infty,\varphi(B)=-\infty$,则由(1)和\cref{prop:SigmaField}(4)可得:
	\begin{equation*}
		\varphi(A\cup B)=\varphi(A)+\varphi(B\backslash A)=\varphi(B)+\varphi(A\backslash B)
	\end{equation*}
	要使得上式有意义,$\varphi(A\cup B)$必须既等于$+\infty$又等于$-\infty$,矛盾。\par
	(3)由(1)可得$\varphi(A)=\varphi(B)+\varphi(A\backslash B)$,当$|\varphi(A)|<+\infty$时,上式有意义必须满足$|\varphi(B)|<+\infty$。\par
	(4)记:
	\begin{equation*}
		A_n^+=
		\begin{cases}
			\varnothing,&\varphi(A_n)\leqslant0 \\
			A_n,&\varphi(A_n)>0
		\end{cases}
		\quad
		A_n^-=
		\begin{cases}
			A_n,&\varphi(A_n)\leqslant0 \\
			\varnothing,&\varphi(A_n)>0
		\end{cases}
	\end{equation*}
	则:
	\begin{equation*}
		\underset{n=1}{\overset{+\infty}{\cup}}A_n=\left(\underset{n=1}{\overset{+\infty}{\cup}}A_n^+\right)\cup\left(\underset{n=1}{\overset{+\infty}{\cup}}A_n^-\right)
	\end{equation*}
	由(3)可得:
	\begin{equation*}
		\left|\varphi\left(\underset{n=1}{\overset{+\infty}{\cup}}A_n^+\right)\right|<+\infty,\quad
		\left|\varphi\left(\underset{n=1}{\overset{+\infty}{\cup}}A_n^-\right)\right|<+\infty
	\end{equation*}
	因为$\{A_n\}$互不相交,由$\{A_n^+\},\{A_n^-\}$的构造方式显然二者内部互不相交且二者之间也互不相交,于是有:
	\begin{align*}
		\sum_{n=1}^{+\infty}|\varphi(A_n)|&=\sum_{n=1}^{+\infty}[|\varphi(A_n^+)|+|\varphi(A_n^-)|] =\sum_{n=1}^{+\infty}\varphi(A_n^+)+\sum_{n=1}^{+\infty}|\varphi(A_n^-)| \\
		&=\sum_{n=1}^{+\infty}\varphi(A_n^+)+\left|\sum_{n=1}^{+\infty}\varphi(A_n^-)\right|
		=\varphi\left(\underset{n=1}{\overset{+\infty}{\cup}}A_n^+\right)+\left|\varphi\left(\underset{n=1}{\overset{+\infty}{\cup}}A_n^-\right)\right|<+\infty
	\end{align*}\par
	(5)根据定义易验证得到。\par
	(6)$\;(\mu-\nu)(\varnothing)=\mu(\varnothing)-\nu(\varnothing)=0$。取互不相交的$\{A_n\}\subseteq\mathscr{F}$,由测度的定义可得:
	\begin{align*}
		(\mu-\nu)\left(\underset{n=1}{\overset{+\infty}{\cup}}A_n\right)&=\mu\left(\underset{n=1}{\overset{+\infty}{\cup}}A_n\right)-\nu\left(\underset{n=1}{\overset{+\infty}{\cup}}A_n\right)=\sum_{n=1}^{+\infty}\mu(A_n)-\sum_{n=1}^{+\infty}\nu(A_n) \\
		&=\sum_{n=1}^{+\infty}[\mu(A_n)-\nu(A_n)]=\sum_{n=1}^{+\infty}(\mu-\nu)(A)
	\end{align*}
	所以$\mu-\nu$是一个符号测度。\par
	(7)根据定义易验证得到。
\end{proof}
\subsection{Hahn, Jordan, Lebesgue分解}
\begin{definition}
	 设$(X,\mathscr{F})$是一个可测空间,$\varphi$是其上的符号测度。定义$\varphi^*$:
	\begin{equation*}
		\forall\;A\in\mathscr{F},\;\varphi^*(A)=\sup\{\varphi(B):B\subseteq A,\;B\in\mathscr{F}\}
	\end{equation*}
\end{definition}
\begin{property}\label{prop:varphi*}
	设$(X,\mathscr{F})$是一个可测空间,$\varphi$是其上的符号测度。$\varphi^*$具有如下性质:
	\begin{enumerate}
		\item $\varphi^*(\varnothing)=0$;
		\item $\varphi^*$具有单调性;
		\item $\varphi^*$是非负集函数;
		\item $\mathscr{A}=\{A\in\mathscr{F}:\varphi^*(A)=0\}$是一个$\sigma$环。
	\end{enumerate}
\end{property}
\begin{proof}
	(1)由$\varphi^*$的定义和$\varphi(\varnothing)=0$即可得出。\par
	(2)由$\varphi^* $的定义即可得出。\par
	(3)由(1)(2)立即可得。\par
	(4)由(1)可知$\mathscr{A}\ne\varnothing$。任取$A,B\in\mathscr{A}$,由\cref{prop:SigmaField}(4)和(3)(2)可得:
	\begin{equation*}
		0\leqslant\varphi^*(A\backslash B)\leqslant\varphi^*(A)=0
	\end{equation*}
	所以$\varphi^*(A\backslash B)=0,\;A\backslash B\in\mathscr{A}$。\par
	任取$\{A_n\}\subseteq\mathscr{A}$,由(1)可得:
	\begin{equation*}
		0\leqslant\varphi^*\left(\underset{n=1}{\overset{+\infty}{\cup}}A_n\right)=\sup\left\{\varphi(B):B\subseteq\underset{n=1}{\overset{+\infty}{\cup}}A_n,\;B\in\mathscr{F}\right\}
	\end{equation*}
\end{proof}
\begin{lemma}
	设$(X,\mathscr{F})$是一个可测空间,$\varphi$是其上的符号测度。
	\begin{enumerate}
		\item 若$A\in\mathscr{F}$且$\varphi(A)<+\infty$,则对任意的$\varepsilon>0$,存在$B\in\mathscr{F}$满足$B\subseteq A,\varphi(B)\geqslant0$且$\varphi^*(A\backslash B)\leqslant\varepsilon$;
		\item 若$A\in\mathscr{F}$且$\varphi(A)<0$,则存在$B\in\mathscr{F}$满足$B\subseteq A,\;\varphi(B)<0$且$\varphi^*(B)=0$。
	\end{enumerate}
\end{lemma}
\begin{proof}
	(1)若此时不满足结论,即存在$\varepsilon>0$,对于任意的$B\subseteq A$都有$\varphi(B)<0$或$\varphi^*(A\backslash B)>\varepsilon$。取$B=\varnothing$即可排除$\varphi(B)<0$,即此时必须有$\varphi^*(A\backslash B)>\varepsilon$。取$B_0=\varnothing$,根据归纳假设可得$\varphi^*(A\backslash B_0)=\varphi^*(A)>\varepsilon$,由$\varphi^*$的定义可知存在$B_1\in\mathscr{F}$满足$B_1\subseteq A\backslash B_0=A$且$\varphi(B_1)>\varepsilon$。根据归纳假设,因为$B_1\subseteq A$,所以有$\varphi^*(A\backslash B_1)>\varepsilon$,即存在$B_2\in\mathscr{F}$满足$B_2\subseteq A\backslash B_1\subseteq A$且$\varphi(B_2)>\varepsilon$。根据归纳假设,因为$B_1\cup B_2\subseteq A$,所以有$\varphi^*[A\backslash(B_1\cup B_2)]>\varepsilon$,即存在$B_3\in\mathscr{F}$满足$B_3\subseteq A\backslash(B_1\cup B_2)\subseteq A$且$\varphi(B_3)>\varepsilon$。继续操作下去,可以得到互不相交的集合序列$\{B_n\}$满足:
	\begin{equation*}
		B_n\subseteq A,\;B_n\in\mathscr{F},\;\varphi(B_n)>\varepsilon
	\end{equation*}
	令$B=\underset{n=1}{\overset{+\infty}{\cup}}B_n$,则有:
	\begin{equation*}
		B\subseteq A,\;B\in\mathscr{F},\;\varphi(B)=\sum_{n=1}^{+\infty}\varphi(B_n)=+\infty
	\end{equation*}
	由\cref{prop:SignedMeasure}(3)可得$|\varphi(A)|=+\infty$,于是只能有$\varphi(A)=-\infty$,矛盾,所以结论成立。\par
	(2)由(1)可知存在$B_1\subseteq A$满足$\varphi(B)\geqslant0$且$\varphi^*(A\backslash B)\leqslant1$。根据$\varphi^*$的定义可得:
	\begin{equation*}
		\varphi(A\backslash B_1)\leqslant\varphi^*(A\backslash B_1)\leqslant1
	\end{equation*}
	所以存在$B_2\subseteq A\backslash B_1$满足$\varphi(B)\geqslant0$且$\varphi^*[A\backslash (B_1\cup B_2)]\leqslant\frac{1}{2}$。继续操作下去,可以得到互不相交的集合序列$\{B_n\}\subseteq\mathscr{F}$满足:
	\begin{equation*}
		B_n\subseteq A,\;\varphi(B_n)\geqslant0,\;\varphi^*\left[A\Big\backslash\left(\underset{i=1}{\overset{n}{\cup}}B_n\right)\right]\leqslant\frac{1}{n}
	\end{equation*}
	令$B=\underset{n=1}{\overset{+\infty}{\cup}}B_n$,由符号测度的可列可加性以及\cref{prop:varphi*}(2)(3)可得:
	\begin{equation*}
		\varphi(B)=\sum_{n=1}^{+\infty}\varphi(B_n)\geqslant0,\quad0\leqslant\varphi^*(A\backslash B)\leqslant\varphi^*\left[A\Big\backslash\left(\underset{i=1}{\overset{n}{\cup}}B_n\right)\right]\leqslant\frac{1}{n}
	\end{equation*}
	所以$\varphi^*(A\backslash B)=0$。取$A_0=A\backslash B$,根据\cref{prop:SigmaField}(4)可得:
	\begin{equation*}
		A_0\in\mathscr{F},\;A_0\subseteq A,\;\varphi^*(A_0)=0
	\end{equation*}
	而由\cref{prop:SignedMeasure}(1)可得:
	\begin{equation*}
		\varphi(A)=\varphi(A_0)+\varphi(B)
	\end{equation*}
	因为$\varphi(A)<0,\;\varphi(B)\geqslant0$,所以$\varphi(A_0)<0$,$A_0$即满足条件。
\end{proof}
\begin{theorem}[Hahn Decomposition]\label{theo:HahnDecomposition}
	设$\varphi$是可测空间$(X,\mathscr{F})$上的符号测度,则存在$X^{\pm}\in\mathscr{F}$满足:
	\begin{equation*}
		X^+\cup X^-=X,\quad X^+\cap X^-=\varnothing
	\end{equation*}
	并且有(\cref{prop:SigmaField}(2)):
	\begin{equation*}
		\forall\;A\in\mathscr{F},\;\varphi(A\cap X^+)\geqslant0\geqslant\varphi(A\cap X^-)
	\end{equation*}
	且$X^{\pm}$在下列意义下是唯一的:若$\{X^+_1,X^-_1\},\{X^+_2,X^-_2\}$都满足上述条件,则:
	\begin{gather*}
		\forall\;A\in\mathscr{F},A\subseteq X^+_1\Delta X^+_2\Rightarrow\varphi(A)=0 \\
		\forall\;A\in\mathscr{F},A\subseteq X^-_1\Delta X^-_2\Rightarrow\varphi(A)=0
	\end{gather*}
	称$\{X^+,X^-\}$为$\varphi$的\textbf{Hahn分解}。
\end{theorem}
\begin{proof}
	令$\mathscr{A}=\{A\in\mathscr{F}:\varphi^*(A)=0\}$,记$\alpha=\inf\{\varphi(A):A\in\mathscr{A}\}$。由\cref{prop:varphi*}(1)可知$\varnothing\in\mathscr{A}$,所以$\alpha\leqslant\varphi(\varnothing)=0$。取$\{A_n\}\subseteq\mathscr{A}$满足$\lim_{n\to+\infty}\limits\varphi(A_n)=\alpha$,令$X^-=\underset{n=1}{\overset{+\infty}{\cup}}A_n$。由\cref{prop:varphi*}(4)可知$\mathscr{A}$是一个$\sigma$环,所以$X^-\in\mathscr{A}$,$X^-\in\mathscr{F}$。根据$\varphi^*$的定义、\cref{prop:SigmaField}(2)和\cref{prop:varphi*}(2)可得:
	\begin{equation*}
		\varphi(A\cap X^-)\leqslant\varphi^*(A\cap X^-)\leqslant\varphi^*(X^-)=0
	\end{equation*}
	所以$X^-$满足要求。\par
	令$X^+=X\backslash X^-$,则$X^+\cup X^-=X$。由\cref{prop:SigmaField}(4)可知$X^+\in\mathscr{F}$。若存在$A\in\mathscr{F}$使得$\varphi(A\cap X^+)$
\end{proof}
\begin{theorem}[Jordan Decomposition]
	\label{theo:JordanDecomposition}
	设$\varphi$是可测空间$(X,\mathscr{F})$上的符号测度,则存在$(X,\mathscr{F})$上的测度$\varphi^+$和有限测度$\varphi^-$使得$\varphi=\varphi^+-\varphi^-$并且有:
	\begin{equation*}
		\forall\;A\in\mathscr{F},\;\varphi^+(A)=\varphi^*(A)=\varphi(A\cap X^+),\;\varphi^-(A)=(-\varphi)^*=-\varphi(A\cap X^-)
	\end{equation*}
	称分解式$\varphi=\varphi^+-\varphi^-$为$\varphi$的\textbf{Jordan分解},分别称$\varphi^+,\varphi^-,|\varphi|\coloneq\varphi^++\varphi^-$为$\varphi$的\textbf{上变差、下变差和全变差}。
\end{theorem}
\begin{definition}
	设$\varphi$和$\mu$分别是可测空间$(X,\mathscr{F})$上的符号测度与测度,若对任何的$\mu$零测集$A$都有$\varphi(A)=0$,则称$\varphi$对$\mu$\textbf{绝对连续},记作$\varphi\ll\mu$。
\end{definition}
\begin{definition}
	设$\varphi$和$\psi$是可测空间$(X,\mathscr{F})$上的符号测度,若存在$N\in\mathscr{F}$使得$|\varphi|(N^c)=|\psi|(N)=0$,则称$\varphi$和$\psi$是\textbf{相互奇异}的,记作$\varphi\perp\psi$。
\end{definition}
\begin{theorem}[Lebesgue Decomposition]
	\label{theo:LebesgueDecomposition}
	设$\varphi$和$\psi$是可测空间$(X,\mathscr{F})$上$\sigma$有限的符号测度,则存在两个$(X,\mathscr{F})$上的$\sigma$有限符号测度$\varphi_c,\varphi_s$使得:
	\begin{equation*}
		\varphi=\varphi_c+\varphi_s,\;\varphi_c\ll\psi,\;\varphi_s\perp\psi
	\end{equation*}
\end{theorem}

\subsection{Randon-Nikodym导数}
\begin{theorem}\label{theo:RandonNikodym}
	设$\varphi$和$\mu$分别是可测空间$(X,\mathscr{F})$上的符号测度和$\sigma$有限测度。若$\varphi\ll\mu$,则存在$(X,\mathscr{F},\mu)$上在a.e.于$X$的意义下唯一的可测函数$f$满足:
	\begin{equation*}
		\forall\;A\in\mathscr{F},\;\varphi(A)=\int_{A}f(x)\dif\mu\quad\int_{X}f^-(x)\dif\mu<+\infty
	\end{equation*}
	若$\varphi$是$\sigma$有限的,则$f$有限a.e.于$X$。称$f$为$\varphi$关于$\mu$的Randon-Nikodym导数,记作$\dfrac{\dif\varphi}{\dif\mu}$。
\end{theorem}
\begin{lemma}\label{lem:IntChangeOfMeasure}
	设$\varphi$和$\mu$是可测空间$(X,\mathscr{F})$上的$\sigma$有限测度,$\varphi\ll\mu$。对任意$(X,\mathscr{F})$上的可测函数$f$和任意的$A\in\mathscr{F}$,只要:
	\begin{equation*}
		\int_{A}f(x)\dif\nu=\int_{A}f(x)\frac{\dif\nu}{\dif\mu}\dif\mu
	\end{equation*}
\end{lemma}
\begin{proof}
	content...
\end{proof}
\begin{property}
	设$(X,\mathscr{F})$是一个可测空间。Randon-Nikodym导数的计算具有如下性质:
	\begin{enumerate}
		\item 设$\varphi$是$(X,\mathscr{F})$上的符号测度,$\nu$和$\mu$是$(X,\mathscr{F})$上的$\sigma$有限测度且$\varphi\ll\nu\ll\mu$,则有:
		\begin{equation*}
			\frac{\dif\varphi}{\dif\mu}=\frac{\dif\varphi}{\dif\nu}\cdot\frac{\dif\nu}{\dif\mu}\;\text{a.e.于}X
		\end{equation*}
		\item 设$\nu$和$\mu$是$(X,\mathscr{F})$上的$\sigma$有限测度且$\nu\ll\mu$,则$\mu\ll\nu$当且仅当$\dfrac{\dif\nu}{\dif\mu}>0\;$a.e.于$(X,\mathscr{F},\mu)$,此时有:
		\begin{equation*}
			\frac{\dif\mu}{\dif\nu}=1\Big/\frac{\dif\nu}{\dif\mu}\;\text{a.e.于}X
		\end{equation*}
		\item 若$\varphi$是$(X,\mathscr{F})$上的符号测度,$\mu$是$(X,\mathscr{F})$上的$\sigma$有限测度,$\varphi\ll\mu$,则对任意的$a\in\mathbb{R}^{}$有:
		\begin{equation*}
			\frac{\dif(a\varphi)}{\dif\mu}=a\frac{\dif\varphi}{\dif\mu}\;\text{a.e.于}(X,\mathscr{F},\mu)
		\end{equation*}
		\item 若$\varphi,\psi$是$(X,\mathscr{F})$上的符号测度,$\mu$是$(X,\mathscr{F})$上的$\sigma$有限测度且$\varphi,\psi\ll\mu$,则有:
		\begin{equation*}
			\frac{\dif(\varphi+\psi)}{\dif\mu}=\frac{\dif\varphi}{\dif\mu}+\frac{\dif\psi}{\dif\mu}\;\text{a.e.于}(X,\mathscr{F},\mu)
		\end{equation*}
	\end{enumerate}
\end{property}
\begin{proof}
	(2)因为$\nu\ll\mu$且$\nu$和$\mu$是$\sigma$有限测度,根据\cref{theo:RandonNikodym}可知存在$(X,\mathscr{F},\mu)$上在a.e.于$X$的意义下唯一的可测函数$\dfrac{\dif\nu}{\dif\mu}$使得:
	\begin{equation*}
		\forall\;A\in\mathscr{F},\;\nu(A)=\int_{A}\frac{\dif\nu}{\dif\mu}\dif\mu
	\end{equation*}
	由\cref{prop:MeasurableFunction}(1)可知:
	\begin{equation*}
		\left\{\frac{\dif\nu}{\dif\mu}<0\right\}\in\mathscr{F}
	\end{equation*}
	若上述集合关于$\mu$的测度大于$0$,则:
	\begin{equation*}
		\nu\left(\left\{\frac{\dif\nu}{\dif\mu}<0\right\}\right)=\int_{\left\{\frac{\dif\nu}{\dif\mu}<0\right\}}\frac{\dif\nu}{\dif\mu}\dif\mu<0
	\end{equation*}
	与$\nu$是测度相矛盾,所以有:
	\begin{equation*}
		\mu\left(\left\{\frac{\dif\nu}{\dif\mu}<0\right\}\right)=0
	\end{equation*}\par
	\textbf{必要性:}取集合:
	\begin{equation*}
		N=\left\{\frac{\dif\nu}{\dif\mu}=0\right\}
	\end{equation*}
	因为$\dfrac{\dif\nu}{\dif\mu}$是可测函数,由\cref{prop:MeasurableFunction}(2)可知$N\in\mathscr{F}$。由简单函数的定义可知$\dfrac{\dif\nu}{\dif\mu}$限制在$N$上时为非负简单函数,根据非负简单函数积分的定义可得到:
	\begin{equation*}
		\nu(N)=\int_{N}\frac{\dif\nu}{\dif\mu}\dif\mu=0
	\end{equation*}
	因为$\mu\ll\nu$,所以$\mu(N)=0$。由\cref{prop:Measure}(1)可得:
	\begin{equation*}
		\mu\left(\left\{\dfrac{\dif\nu}{\dif\mu}\leqslant0\right\}\right)=\mu\left(\left\{\dfrac{\dif\nu}{\dif\mu}<0\right\}\right)+\mu\left(\left\{\dfrac{\dif\nu}{\dif\mu}=0\right\}\right)=0
	\end{equation*}
	所以$\dfrac{\dif\nu}{\dif\mu}>0\;$a.e.于$(X,\mathscr{F},\mu)$。\par
	\textbf{充分性:}若此时不满足$\mu\ll\nu$,即存在满足$\nu(A)=0$的$A\in\mathscr{F}$有$\mu(A)>0$。因为$\dfrac{\dif\nu}{\dif\mu}>0\;$a.e.于$(X,\mathscr{F},\mu)$,由\cref{prop:MeasurableIntegral}(7)
	
	(3)根据\cref{prop:SignedMeasure}(5)可知$a\varphi$也是一个符号测度。因为$\varphi\ll\mu$,由$a\varphi$的定义可得$a\varphi\ll\mu$,根据\cref{theo:RandonNikodym}可知存在$(X,\mathscr{F},\mu)$上在a.e.于$X$的意义下唯一的可测函数$\dfrac{\dif(a\varphi)}{\dif\mu}$使得:
	\begin{equation*}
		(a\varphi)(A)=a\varphi(A)=\int_{A}\frac{\dif(a\varphi)}{\dif\mu}\dif\mu
	\end{equation*}
	由\cref{prop:MeasurableIntegral}(5)可知:
	\begin{equation*}
		\varphi(A)=\int_{A}\frac{1}{a}\frac{\dif(a\varphi)}{\dif\mu}\dif\mu
	\end{equation*}
	根据\cref{theo:RandonNikodym}可知:
	\begin{equation*}
		\dfrac{1}{a}\frac{\dif(a\varphi)}{\dif\mu}=\dfrac{\dif\varphi}{\dif\mu}\;\text{a.e.于}(X,\mathscr{F},\mu)
	\end{equation*}
	即:
	\begin{equation*}
		\frac{\dif(a\varphi)}{\dif\mu}=a\dfrac{\dif\varphi}{\dif\mu}\;\text{a.e.于}(X,\mathscr{F},\mu)
	\end{equation*}\par
	(4)由\cref{prop:SignedMeasure}(5)可知$\varphi+\psi$也是一个符号测度。因为$\varphi,\psi\ll\mu$,根据$\varphi+\psi$的定义可得$\varphi+\psi\ll\mu$。由\cref{theo:RandonNikodym}可知存在$(X,\mathscr{F},\mu)$上在a.e.于$X$的意义下唯一的可测函数$\dfrac{\dif(\varphi+\psi)}{\dif\mu}$使得:
	\begin{equation*}
		(\varphi+\psi)(A)=\varphi(A)+\psi(A)=\int_{A}\frac{\dif(\varphi+\psi)}{\dif\mu}\dif\mu
	\end{equation*}
	同理可得:
	\begin{equation*}
		\varphi(A)=\int_{A}\frac{\dif\varphi}{\dif\mu}\dif\mu,\quad\psi(A)=\int_{A}\frac{\dif\psi}{\dif\mu}\dif\mu
	\end{equation*}
	由\cref{theo:RandonNikodym}中Randon-Nikodym导数负部的可积性和\cref{prop:MeasurableIntegral}(5)可得:
	\begin{equation*}
		\varphi(A)+\psi(A)=\int_{A}\frac{\dif\varphi}{\dif\mu}\dif\mu+\int_{A}\frac{\dif\psi}{\dif\mu}\dif\mu=\int_{A}\left(\frac{\dif\varphi}{\dif\mu}+\frac{\dif\psi}{\dif\mu}\right)\dif\mu
	\end{equation*}
	根据\cref{theo:RandonNikodym}可知:
	\begin{equation*}
		\frac{\dif(\varphi+\psi)}{\dif\mu}=\frac{\dif\varphi}{\dif\mu}+\frac{\dif\psi}{\dif\mu}\;\text{a.e.于}(X,\mathscr{F},\mu)\qedhere
	\end{equation*}
\end{proof}

\subsection{乘积空间}
\begin{theorem}
	设$(X_1,\mathscr{F}_1,\mu_1),(X_2,\mathscr{F}_2,\mu_2)$是$\sigma$有限测度空间,则:
	\begin{enumerate}
		\item 在乘积空间$(X_1\times X_2,\mathscr{F}_1\times \mathscr{F}_2)$上存在唯一的测度$\mu$使得对任意的$A_1\in\mathscr{F}_1$和任意的$A_2\in\mathscr{F}_2$有:
		\begin{equation*}
			\mu(A_1\times A_2)=\mu_1(A_1)\mu_2(A_2)
		\end{equation*}
		该测度$\mu_1\times\mu_2\coloneq\mu$是$\sigma$有限的,称之为$\mu_1$和$\mu_2$的乘积测度;
		\item 对$(X_1\times X_2,\mathscr{F}_1\times \mathscr{F}_2,\mu_1\times\mu_2)$上任何积分存在的可测函数$f$有:
		\begin{align*}
			\int_{X_1\times X_2}f\dif(\mu_1\times\mu_2)&=\int_{X_1}\mu_1(\dif x_1)\int_{X_2}f(x_1,x_2)\mu_2(\dif x_2) \\
			&=\int_{X_2}\mu_2(\dif x_2)\int_{X_1}f(x_1,x_2)\mu_1(\dif x_1)
		\end{align*}
	\end{enumerate}
\end{theorem}