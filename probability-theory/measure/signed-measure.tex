\section{不定积分}

\subsection{符号测度}
\begin{definition}
	设$(X,\mathscr{F})$是一个可测空间,若从$\mathscr{F}$到$\overline{\mathbb{R}^{}}$的集函数$\varphi$满足:
	\begin{enumerate}
		\item $\varphi(\varnothing)=0$;
		\item $\varphi$具有可列可加性。
	\end{enumerate}
	则称$\varphi$	为\gls{SignedMeasure}。若对任意的$A\in\mathscr{F}$有$|\varphi(A)|<+\infty$,则称$\varphi$是有限的;若存在$X$的可列可测分割$\{A_n\}\subseteq\mathscr{F}$满足对任意的$n\in\mathbb{N}^+$有$|\varphi(A_n)|<+\infty$,则称$\varphi$是$\sigma$有限的。
\end{definition}
\begin{property}
	设$(X,\mathscr{F})$是一个可测空间,其上的符号测度$\varphi$具有如下性质:
	\begin{enumerate}
		\item $\varphi$具有有限可加性;
		\item $\varphi$只可能出现以下两种情况中的一种\footnote{下面所涉及的结论都是关于第一种情况的,对于第二种情况只需取$-\varphi$即可得到相关结果。}:
		\begin{gather*}
			\forall\;A\in\mathscr{F},\;-\infty<\varphi(A)\leqslant+\infty \\
			\forall\;A\in\mathscr{F},\;-\infty\leqslant\varphi(A)<+\infty
		\end{gather*}
		\item 若$A,B\in\mathscr{F},\;B\subseteq A$且$|\varphi(A)|<+\infty$,则$|\varphi(B)|<+\infty$;
		\item 若$\{A_n\}$两两不交且满足:
		\begin{equation*}
			\left|\varphi\left(\underset{n=1}{\overset{+\infty}{\cup}}A_n\right)\right|<+\infty
		\end{equation*}
		则有:
		\begin{equation*}
			\sum_{n=1}^{+\infty}|\varphi(A_n)|<+\infty
		\end{equation*}
		\item 记:
		\begin{equation*}
			\forall\;A\in\mathscr{F},\;\varphi^*(A)=\sup\{\varphi(B):B\subseteq A,\;B\in\mathscr{F}\}
		\end{equation*}
		则$\varphi^*$是$\mathscr{F}$上单调且满足$\varphi^*(\varnothing)=0$的非负集函数;
	\end{enumerate}
\end{property}
\begin{proof}
	(1)由符号测度的定义显然可得。\par
	(2)设$A,B\in\mathscr{F}$且$\varphi(A)=+\infty,\varphi(B)=-\infty$,则由(1)和\cref{prop:SigmaField}(4)可得:
	\begin{equation*}
		\varphi(A\cup B)=\varphi(A)+\varphi(B\backslash A)=\varphi(B)+\varphi(A\backslash B)
	\end{equation*}
	要使得上式有意义,$\varphi(A\cup B)$必须既等于$+\infty$又等于$-\infty$,矛盾。\par
	(3)由(1)可得$\varphi(A)=\varphi(B)+\varphi(A\backslash B)$,当$|\varphi(A)|<+\infty$时,上式有意义必须满足$|\varphi(B)|<+\infty$。\par
	(4)记:
	\begin{equation*}
		A_n^+=
		\begin{cases}
			\varnothing,&\varphi(A_n)\leqslant0 \\
			A_n,&\varphi(A_n)>0
		\end{cases}
		\quad
		A_n^-=
		\begin{cases}
			A_n,&\varphi(A_n)\leqslant0 \\
			\varnothing,&\varphi(A_n)>0
		\end{cases}
	\end{equation*}
	则:
	\begin{equation*}
		\underset{n=1}{\overset{+\infty}{\cup}}A_n=\left(\underset{n=1}{\overset{+\infty}{\cup}}A_n^+\right)\cup\left(\underset{n=1}{\overset{+\infty}{\cup}}A_n^-\right)
	\end{equation*}
	由(3)可得:
	\begin{equation*}
		\left|\varphi\left(\underset{n=1}{\overset{+\infty}{\cup}}A_n^+\right)\right|<+\infty,\quad
		\left|\varphi\left(\underset{n=1}{\overset{+\infty}{\cup}}A_n^-\right)\right|<+\infty
	\end{equation*}
	因为$\{A_n\}$互不相交,由$\{A_n^+\},\{A_n^-\}$的构造方式显然二者内部互不相交且二者之间也互不相交,于是有:
	\begin{align*}
		\sum_{n=1}^{+\infty}|\varphi(A_n)|&=\sum_{n=1}^{+\infty}[|\varphi(A_n^+)|+|\varphi(A_n^-)|] =\sum_{n=1}^{+\infty}\varphi(A_n^+)+\sum_{n=1}^{+\infty}|\varphi(A_n^-)| \\
		&=\sum_{n=1}^{+\infty}\varphi(A_n^+)+\left|\sum_{n=1}^{+\infty}\varphi(A_n^-)\right|
		=\varphi\left(\underset{n=1}{\overset{+\infty}{\cup}}A_n^+\right)+\left|\varphi\left(\underset{n=1}{\overset{+\infty}{\cup}}A_n^-\right)\right|<+\infty
	\end{align*}\par
	(5)由$\varphi(\varnothing)=0$显然可得。
\end{proof}
\begin{definition}
	设$(X,\mathscr{F})$是一个可测空间,$\mu,\nu$为其上的两个测度。只要$\mu,\nu$中有一个是有限的,则可定义符号测度:
	\begin{equation*}
		\forall\;A\in\mathscr{F},\;(\mu-\nu)(A)\coloneq\mu(A)-\nu(A)
	\end{equation*}
\end{definition}
\begin{lemma}
	设$(X,\mathscr{F})$是一个可测空间,$\varphi$是其上的符号测度。
	\begin{enumerate}
		\item 若$A\in\mathscr{F}$且$\varphi(A)<+\infty$,则对任意的$\varepsilon>0$,存在$B\subseteq A$满足$\varphi(B)\geqslant0$且$\varphi^*(A\backslash B)\leqslant\varepsilon$;
		\item 若$A\in\mathscr{F}$且$\varphi(A)<0$,则存在$B\in\mathscr{F}$满足$B\subseteq A,\;\varphi(B)<0$且$\varphi^*(B)=0$。
	\end{enumerate}
\end{lemma}
\begin{proof}
	(1)
\end{proof}