\section{不定积分}

\begin{definition}
	设$(X,\mathscr{F},\mu)$是一个测度空间,$f$是其上的可测函数,$\varphi$是$\mathscr{F}$上的集函数。若:
	\begin{equation*}
		\forall\;A\in\mathscr{F},\;\varphi(A)=\int_{A}f(x)\dif\mu
	\end{equation*}
	则称$\varphi$为$f$的\gls{IndefiniteIntegral},$f$为$\varphi$关于$\mu$的\textbf{导数}。
\end{definition}

\subsection{符号测度}
\begin{definition}
	设$(X,\mathscr{F})$是一个可测空间,若从$\mathscr{F}$到$\overline{\mathbb{R}}$的集函数$\varphi$满足:
	\begin{enumerate}
		\item $\varphi(\varnothing)=0$;
		\item $\varphi$具有可列可加性。
	\end{enumerate}
	则称$\varphi$	为\gls{SignedMeasure}。若对任意的$A\in\mathscr{F}$有$|\varphi(A)|<+\infty$,则称$\varphi$是有限的;若存在$X$的可列可测分割$\{A_n\}\subseteq\mathscr{F}$满足对任意的$n\in\mathbb{N}^+$有$|\varphi(A_n)|<+\infty$,则称$\varphi$是$\sigma$有限的。
\end{definition}
\begin{property}\label{prop:SignedMeasure}
	设$(X,\mathscr{F})$是一个可测空间,其上的符号测度$\varphi$具有如下性质:
	\begin{enumerate}
		\item $\varphi$具有有限可加性;
		\item $\varphi$只可能出现以下两种情况中的一种\footnote{下面所涉及的结论都是关于第一种情况的,对于第二种情况只需取$-\varphi$即可得到相关结果。}:
		\begin{gather*}
			\forall\;A\in\mathscr{F},\;-\infty<\varphi(A)\leqslant+\infty \\
			\forall\;A\in\mathscr{F},\;-\infty\leqslant\varphi(A)<+\infty
		\end{gather*}
		\item 若$A,B\in\mathscr{F},\;B\subseteq A$且$|\varphi(A)|<+\infty$,则$|\varphi(B)|<+\infty$;
		\item 若$\{A_n\}\subseteq\mathscr{F}$互不相交且满足:
		\begin{equation*}
			\left|\varphi\left(\underset{n=1}{\overset{+\infty}{\cup}}A_n\right)\right|<+\infty
		\end{equation*}
		则有:
		\begin{equation*}
			\sum_{n=1}^{+\infty}|\varphi(A_n)|<+\infty
		\end{equation*}
		\item 对$a\in\mathbb{R}^{}$,定义$(a\varphi)(A)\coloneq a\varphi(A),\;\forall\;A\in\mathscr{F}$,则$a\varphi$也是$(X,\mathscr{F})$上的一个符号测度;
		\item 设$\mu,\nu$为$(X,\mathscr{F})$上的两个测度。只要$\mu,\nu$中有一个是有限的,则可定义$(X,\mathscr{F})$上的符号测度:
		\begin{equation*}
			\forall\;A\in\mathscr{F},\;(\mu-\nu)(A)\coloneq\mu(A)-\nu(A)
		\end{equation*}
		\item $\varphi$在$\mathscr{F}$的子$\sigma$域$\mathscr{A}$上也是一个符号测度。
	\end{enumerate}
\end{property}
\begin{proof}
	(1)由符号测度的定义显然可得。\par
	(2)设$A,B\in\mathscr{F}$且$\varphi(A)=+\infty,\varphi(B)=-\infty$,则由(1)和\cref{prop:SigmaField}(4)可得:
	\begin{equation*}
		\varphi(A\cup B)=\varphi(A)+\varphi(B\setminus A)=\varphi(B)+\varphi(A\setminus B)
	\end{equation*}
	要使得上式有意义,$\varphi(A\cup B)$必须既等于$+\infty$又等于$-\infty$,矛盾。\par
	(3)由(1)和\cref{prop:SigmaField}(4)可得$\varphi(A)=\varphi(B)+\varphi(A\setminus B)$,当$|\varphi(A)|<+\infty$时,上式有意义必须满足$|\varphi(B)|<+\infty$。\par
	(4)记:
	\begin{equation*}
		A_n^+=
		\begin{cases}
			\varnothing,&\varphi(A_n)\leqslant0 \\
			A_n,&\varphi(A_n)>0
		\end{cases}
		\quad
		A_n^-=
		\begin{cases}
			A_n,&\varphi(A_n)\leqslant0 \\
			\varnothing,&\varphi(A_n)>0
		\end{cases}
	\end{equation*}
	则:
	\begin{equation*}
		\underset{n=1}{\overset{+\infty}{\cup}}A_n=\left(\underset{n=1}{\overset{+\infty}{\cup}}A_n^+\right)\bigcup\left(\underset{n=1}{\overset{+\infty}{\cup}}A_n^-\right)
	\end{equation*}
	由(3)可得:
	\begin{equation*}
		\left|\varphi\left(\underset{n=1}{\overset{+\infty}{\cup}}A_n^+\right)\right|<+\infty,\quad
		\left|\varphi\left(\underset{n=1}{\overset{+\infty}{\cup}}A_n^-\right)\right|<+\infty
	\end{equation*}
	因为$\{A_n\}$互不相交,由$\{A_n^+\},\{A_n^-\}$的构造方式显然二者内部互不相交且二者之间也互不相交,于是根据\cref{prop:Series}(4)可得:
	\begin{align*}
		\sum_{n=1}^{+\infty}|\varphi(A_n)|&=\sum_{n=1}^{+\infty}[|\varphi(A_n^+)|+|\varphi(A_n^-)|] =\sum_{n=1}^{+\infty}\varphi(A_n^+)+\sum_{n=1}^{+\infty}|\varphi(A_n^-)| \\
		&=\sum_{n=1}^{+\infty}\varphi(A_n^+)+\left|\sum_{n=1}^{+\infty}\varphi(A_n^-)\right|
		=\varphi\left(\underset{n=1}{\overset{+\infty}{\cup}}A_n^+\right)+\left|\varphi\left(\underset{n=1}{\overset{+\infty}{\cup}}A_n^-\right)\right|<+\infty
	\end{align*}\par
	(5)(6)由定义和\cref{prop:Series}(4)即可得到。\par
	(7)根据定义即可验证得到。
\end{proof}
\subsection{Hahn, Jordan分解}
\begin{definition}
	 设$(X,\mathscr{F})$是一个可测空间,$\varphi$是其上的符号测度。定义$\varphi^{\star}$:
	\begin{equation*}
		\forall\;A\in\mathscr{F},\;\varphi^{\star}(A)=\sup\{\varphi(B):B\subseteq A,\;B\in\mathscr{F}\}
	\end{equation*}
\end{definition}
\begin{property}\label{prop:varphiStar}
	设$(X,\mathscr{F})$是一个可测空间,$\varphi$是其上的符号测度。$\varphi^{\star}$具有如下性质:
	\begin{enumerate}
		\item $\varphi^{\star}(\varnothing)=0$;
		\item $\varphi^{\star}$具有单调性;
		\item $\varphi^{\star}$是非负集函数;
		\item $\mathscr{A}=\{A\in\mathscr{F}:\varphi^{\star}(A)=0\}$是一个$\sigma$环。
	\end{enumerate}
\end{property}
\begin{proof}
	(1)由$\varphi^{\star}$的定义和$\varphi(\varnothing)=0$即可得出。\par
	(2)由$\varphi^{\star} $的定义即可得出。\par
	(3)由(1)(2)立即可得。\par
	(4)由(1)可知$\mathscr{A}\ne\varnothing$。任取$A,B\in\mathscr{A}$,由\cref{prop:SigmaField}(4)和(3)(2)可得:
	\begin{equation*}
		0\leqslant\varphi^{\star}(A\setminus B)\leqslant\varphi^{\star}(A)=0
	\end{equation*}
	所以$\varphi^{\star}(A\setminus B)=0,\;A\setminus B\in\mathscr{A}$。\par
	任取互不相交的$\{A_n\}\subseteq\mathscr{A}$,由(3)、\cref{prop:SetOperation}(4)、\cref{prop:SigmaField}(2)、上确界的性质可得:
	\begin{align*}
		&0\leqslant\varphi^{\star}\left(\underset{n=1}{\overset{+\infty}{\cup}}A_n\right)=\sup\left\{\varphi(B):B\subseteq\underset{n=1}{\overset{+\infty}{\cup}}A_n,\;B\in\mathscr{F}\right\} \\
		=&\sup\left\{\varphi\left[B\cap\left(\underset{n=1}{\overset{+\infty}{\cup}}A_n\right)\right]:B\subseteq\underset{n=1}{\overset{+\infty}{\cup}}A_n,\;B\in\mathscr{F}\right\} \\
		=&\sup\left\{\varphi\left[\underset{n=1}{\overset{+\infty}{\cup}}(A_n\cap B)\right]:B\subseteq\underset{n=1}{\overset{+\infty}{\cup}}A_n,\;B\in\mathscr{F}\right\} \\
		=&\sup\left\{\sum_{n=1}^{+\infty}\varphi(A_n\cap B):B\subseteq\underset{n=1}{\overset{+\infty}{\cup}}A_n,\;B\in\mathscr{F}\right\} \\
		\leqslant&\sup\left\{\sum_{n=1}^{+\infty}\varphi(B_n):B_n\subseteq A_n,\;B_n\in\mathscr{F}\right\} \\
		\leqslant&\sum_{n=1}^{+\infty}\sup\{\varphi(B_n):B_n\subseteq A_n,\;B_n\in\mathscr{F}\}=\sum_{n=1}^{+\infty}\varphi^{\star}(A_n)=0
	\end{align*}
	所以$\underset{n=1}{\overset{+\infty}{\cup}}A_n\in\mathscr{A}$。\par
	综上,$\mathscr{A}$是一个$\sigma$环。
\end{proof}
\begin{lemma}\label{lem:HahnDecomposition}
	设$(X,\mathscr{F})$是一个可测空间,$\varphi$是其上的符号测度。
	\begin{enumerate}
		\item 若$A\in\mathscr{F}$且$\varphi(A)<+\infty$,则对任意的$\varepsilon>0$,存在$B\in\mathscr{F}$满足$B\subseteq A,\varphi(B)\geqslant0$且$\varphi^{\star}(A\setminus B)\leqslant\varepsilon$;
		\item 若$A\in\mathscr{F}$且$\varphi(A)<0$,则存在$B\in\mathscr{F}$满足$B\subseteq A,\;\varphi(B)<0$且$\varphi^{\star}(B)=0$。
	\end{enumerate}
\end{lemma}
\begin{proof}
	(1)若此时不满足结论,即存在$\varepsilon>0$,对于任意的$B\subseteq A$都有$\varphi(B)<0$或$\varphi(B)\geqslant$且$\varphi^{\star}(A\setminus B)>\varepsilon$。取$B=\varnothing$即可排除$\varphi(B)<0$,即此时必须有$\varphi(B)\geqslant$且$\varphi^{\star}(A\setminus B)>\varepsilon$。取$B_0=\varnothing$,根据归纳假设可得$\varphi^{\star}(A\setminus B_0)=\varphi^{\star}(A)>\varepsilon$,由$\varphi^{\star}$的定义可知存在$B_1\in\mathscr{F}$满足$B_1\subseteq A\setminus B_0=A$且$\varphi(B_1)>\varepsilon$。根据归纳假设,因为$B_1\subseteq A$,所以有$\varphi^{\star}(A\setminus B_1)>\varepsilon$,即存在$B_2\in\mathscr{F}$满足$B_2\subseteq A\setminus B_1\subseteq A$且$\varphi(B_2)>\varepsilon$。根据归纳假设,因为$B_1\cup B_2\subseteq A$,所以有$\varphi^{\star}[A\setminus(B_1\cup B_2)]>\varepsilon$,即存在$B_3\in\mathscr{F}$满足$B_3\subseteq A\setminus(B_1\cup B_2)\subseteq A$且$\varphi(B_3)>\varepsilon$。继续操作下去,可以得到互不相交的集合序列$\{B_n\}$满足:
	\begin{equation*}
		B_n\subseteq A,\;B_n\in\mathscr{F},\;\varphi(B_n)>\varepsilon
	\end{equation*}
	令$B=\underset{n=1}{\overset{+\infty}{\cup}}B_n$,则有:
	\begin{equation*}
		B\subseteq A,\;B\in\mathscr{F},\;\varphi(B)=\sum_{n=1}^{+\infty}\varphi(B_n)=+\infty
	\end{equation*}
	由\cref{prop:SignedMeasure}(3)可得$|\varphi(A)|=+\infty$,于是只能有$\varphi(A)=-\infty$,矛盾,所以结论成立。\par
	(2)由(1)可知存在$B_1\subseteq A$满足$\varphi(B_1)\geqslant0$且$\varphi^{\star}(A\setminus B_1)\leqslant1$。根据$\varphi^{\star}$的定义可得:
	\begin{equation*}
		\varphi(A\setminus B_1)\leqslant\varphi^{\star}(A\setminus B_1)\leqslant1
	\end{equation*}
	所以存在$B_2\subseteq A\setminus B_1$满足$\varphi(B_2)\geqslant0$且$\varphi^{\star}[A\setminus (B_1\cup B_2)]\leqslant\dfrac{1}{2}$。继续操作下去,可以得到互不相交的集合序列$\{B_n\}\subseteq\mathscr{F}$满足:
	\begin{equation*}
		B_n\subseteq A,\;\varphi(B_n)\geqslant0,\;\varphi^{\star}\left[A\Big\backslash\left(\underset{i=1}{\overset{n}{\cup}}B_n\right)\right]\leqslant\frac{1}{n}
	\end{equation*}
	令$B=\underset{n=1}{\overset{+\infty}{\cup}}B_n$,由符号测度的可列可加性以及\cref{prop:varphiStar}(2)(3)可得:
	\begin{equation*}
		\varphi(B)=\sum_{n=1}^{+\infty}\varphi(B_n)\geqslant0,\quad0\leqslant\varphi^{\star}(A\setminus B)\leqslant\varphi^{\star}\left[A\Big\backslash\left(\underset{i=1}{\overset{n}{\cup}}B_n\right)\right]\leqslant\frac{1}{n}
	\end{equation*}
	所以$\varphi^{\star}(A\setminus B)=0$。取$A_0=A\setminus B$,根据\cref{prop:SigmaField}(4)可得:
	\begin{equation*}
		A_0\in\mathscr{F},\;A_0\subseteq A,\;\varphi^{\star}(A_0)=0
	\end{equation*}
	而由\cref{prop:SignedMeasure}(1)可得:
	\begin{equation*}
		\varphi(A)=\varphi(A_0)+\varphi(B)
	\end{equation*}
	因为$\varphi(A)<0,\;\varphi(B)\geqslant0$,所以$\varphi(A_0)<0$,$A_0$即满足条件。
\end{proof}
\begin{theorem}[Hahn Decomposition]\label{theo:HahnDecomposition}
	设$\varphi$是可测空间$(X,\mathscr{F})$上的符号测度,则存在$X^{\pm}\in\mathscr{F}$满足:
	\begin{equation*}
		X^+\cup X^-=X,\quad X^+\cap X^-=\varnothing
	\end{equation*}
	并且有(\cref{prop:SigmaField}(2)):
	\begin{equation*}
		\forall\;A\in\mathscr{F},\;\varphi(A\cap X^+)\geqslant0\geqslant\varphi(A\cap X^-)
	\end{equation*}
	且$X^{\pm}$在下列意义下是唯一的:若$\{X^+_1,X^-_1\},\{X^+_2,X^-_2\}$都满足上述条件,则:
	\begin{gather*}
		\forall\;A\in\mathscr{F},A\subseteq X^+_1\Delta X^+_2\Rightarrow\varphi(A)=0 \\
		\forall\;A\in\mathscr{F},A\subseteq X^-_1\Delta X^-_2\Rightarrow\varphi(A)=0
	\end{gather*}
	称$\{X^+,X^-\}$为$\varphi$在$X$上的\textbf{Hahn分解}。
\end{theorem}
\begin{proof}
	令$\mathscr{A}=\{A\in\mathscr{F}:\varphi^{\star}(A)=0\}$,记$\alpha=\inf\{\varphi(A):A\in\mathscr{A}\}$。\par
	取$\{A_n\}\subseteq\mathscr{A}$满足$\lim_{n\to+\infty}\limits\varphi(A_n)=\alpha$,令$X^-=\underset{n=1}{\overset{+\infty}{\cup}}A_n$。由\cref{prop:varphiStar}(4)可知$\mathscr{A}$是一个$\sigma$环,所以$X^-\in\mathscr{A}$,$X^-\in\mathscr{F}$。对任意的$A\in\mathscr{F}$,根据$\varphi^{\star}$的定义和\cref{prop:varphiStar}(2)可得:
	\begin{equation*}
		\varphi(A\cap X^-)\leqslant\varphi^{\star}(A\cap X^-)\leqslant\varphi^{\star}(X^-)=0
	\end{equation*}
	所以$X^-$满足要求。\par
	令$X^+=X\setminus X^-$,则$X^+\cup X^-=X$且$X^+\cap X^-=\varnothing$。由\cref{prop:SigmaField}(4)可知$X^+\in\mathscr{F}$。若存在$A\in\mathscr{F}$使得$\varphi(A\cap X^+)<0$,即存在$X^+$的子集$A$使得$\varphi(A)<0$,由\cref{lem:HahnDecomposition}(2)可知存在$A$的子集$B\in\mathscr{F}$满足$\varphi(B)<0$且$\varphi^{\star}(B)=0$,即$B\in\mathscr{A}$。由\cref{prop:SignedMeasure}(1)可得:
	\begin{equation*}
		\varphi(B\cup X^-)=\varphi(B)+\varphi(X^-)<\varphi(X^-)
	\end{equation*}
	对任意的$n\in\mathbb{N}^+$,根据\cref{prop:SigmaField}(4)、\cref{prop:SignedMeasure}(1)和\cref{prop:varphiStar}(4)可得:
	\begin{equation*}
		\varphi(X^-)=\varphi(A_n)+\varphi(X^-\setminus A_n)\leqslant\varphi(A_n)+\varphi^{\star}(X^-\setminus A_n)=\varphi(A_n)
	\end{equation*}
	由\cref{prop:RSeq}(6)可得:
	\begin{equation*}
		\varphi(X^-)\leqslant\lim_{n\to+\infty}\varphi(A_n)=\alpha
	\end{equation*}
	于是$\varphi(B\cup X^-)<\varphi(X^-)\leqslant\alpha$。由\cref{prop:varphiStar}(4)可知$B\cup X^-\in\mathscr{A}$,所以应有$\varphi(B\cup X^-)\geqslant\alpha$,矛盾,于是对任意的$A\in\mathscr{F}$有$\varphi(A\cap X^+)\geqslant0$。\par
	综上,$X^-$与$X^+$的存在性得证。下证唯一性。\par
	若存在不同的$X_1^+,X_1^-$和$X_2^+,X_2^-$满足要求,任取$A\subseteq X_2^+\setminus X_1^+$,其中$A\in\mathscr{F}$,则$A\subseteq X_2^+$且$\varphi(A)\geqslant0$,而$A\subseteq (X_1^+)^c=X_1^-$,所以$\varphi(A)\leqslant0$,即$\varphi(A)=0$。同理可得$X_1^+\setminus X_2^+$的子集都满足符号测度为$0$,于是对任意满足$A\subseteq X_1^+\Delta X_2^+$的$A\in\mathscr{F}$有$\varphi(A)=0$。同理又可得出$A\subseteq X_1^-\Delta X_2^-$时的情况。
\end{proof}
\begin{theorem}[Jordan Decomposition]
	\label{theo:JordanDecomposition}
	设$\varphi$是可测空间$(X,\mathscr{F})$上的符号测度,$\{X^+,X^-\}$为$\varphi$在$X$上的Hahn分解,则存在$(X,\mathscr{F})$上的测度$\varphi^+(A)=\varphi(A\cap X^+)$和有限测度$\varphi^-(A)=-\varphi(A\cap X^-)$使得$\varphi(A)=\varphi^+(A)-\varphi^-(A)$并且有:
	\begin{equation*}
		\forall\;A\in\mathscr{F},\;\varphi^+(A)=\varphi^{\star}(A),\;\varphi^-(A)=(-\varphi)^{\star}
	\end{equation*}
	称分解式$\varphi=\varphi^+-\varphi^-$为$\varphi$的\textbf{Jordan分解},分别称$\varphi^+,\varphi^-,|\varphi|\coloneq\varphi^++\varphi^-$为$\varphi$的\gls{PositiveVariation}、\gls{NegativeVariation}和\gls{TotalVariation}。
\end{theorem}
\begin{proof}
	由\cref{prop:SetOperation}(4)和符号测度的定义逐步验证定义可知$\varphi^+$和$\varphi^-$都是$(X,\mathscr{F})$上的测度,由\cref{prop:SignedMeasure}(2)处的约定可得$\varphi^-$是有限的,根据\cref{theo:HahnDecomposition}、\cref{prop:SetOperation}(4)和\cref{prop:Measure}(1)可知$\varphi=\varphi^+-\varphi^-$。对任意的$A\in\mathscr{F}$和任意$A$的子集$B\in\mathscr{F}$,由\cref{prop:SetOperation}(4)和\cref{prop:Measure}(1)(3)(单调性)有:
	\begin{equation*}
		\varphi(B)=\varphi^+(B)-\varphi^-(B)\leqslant\varphi^+(B)\leqslant\varphi^+(A)
	\end{equation*}
	根据上确界的不等式性可知$\varphi^+(A)\geqslant\varphi^{\star}(A)$,而:
	\begin{equation*}
		\varphi^+(A)=\varphi(A\cap X^+)\leqslant\sup\{\varphi(B):B\subseteq A,\;B\in\mathscr{F}\}=\varphi^*(A)
	\end{equation*}
	于是$\varphi^+(A)=\varphi^*(A)$,$\varphi^-(A)$的结论类似可得。
\end{proof}

\subsection{Randon-Nikodym导数}
\begin{definition}
	设$\varphi$和$\mu$分别是可测空间$(X,\mathscr{F})$上的符号测度与测度。若对任何的$\mu$零测集$A$都有$\varphi(A)=0$,则称$\varphi$对$\mu$\gls{AbsolutelyContinuous},记作$\varphi\ll\mu$。
\end{definition}
在接下来的讨论中,我们记:
\begin{equation*}
	\mathscr{C}=\left\{g\in L_1(X,\mathscr{F},\mu):g\geqslant0,\;\int_{A}g(x)\dif\mu\leqslant\varphi(A),\;\forall\;A\in\mathscr{F}\right\}
\end{equation*}
\begin{lemma}\label{lem:RandonNikodym1}
	若$\varphi$和$\mu$都是可测空间$(X,\mathscr{F})$上的有限测度,则存在$f\in\mathscr{C}$使得:
	\begin{equation*}
		\int_{X}f(x)\dif\mu=\sup\left\{\int_{X}g(x)\dif\mu:g\in\mathscr{C}\right\}
	\end{equation*}
\end{lemma}
\begin{proof}
	记:
	\begin{equation*}
		\sup\left\{\int_{X}g(x)\dif\mu:g\in\mathscr{C}\right\}=\alpha
	\end{equation*}
	取$\{g_n\}\subseteq\mathscr{C}$使得:
	\begin{equation*}
		\lim_{n\to+\infty}\int_{X}g_n(x)\dif\mu=\alpha
	\end{equation*}
	令:
	\begin{equation*}
		f_n=\max_{1\leqslant k\leqslant n}g_k,\quad f=\sup\{g_n\}
	\end{equation*}
	由\cref{prop:MeasurableFunction}(6)可知$f_n$和$f$为非负可测函数且$f_n\uparrow f$。因为$g_n\leqslant f$,由\cref{prop:NonnegativeMeasurableIntegral}(6)和\cref{prop:RSeq}(6)可得:
	\begin{equation*}
		\alpha=\lim_{n\to+\infty}\int_{X}g_n(x)\dif\mu\leqslant\int_{X}f(x)\dif\mu
	\end{equation*}
	记:
	\begin{equation*}
		A_{nk}=\{f_n=g_k\},\quad B_{nk}=A_{nk}\Big\backslash\left(\underset{i=1}{\overset{k-1}{\cup}}A_{ni}\right)
	\end{equation*}
	于是$\{B_{nk}:k=1,2,\dots,n\}$互不相交且:
	\begin{equation*}
		\underset{k=1}{\overset{n}{\cup}}B_{nk}=\underset{k=1}{\overset{n}{\cup}}A_{nk}=X
	\end{equation*}
	由\cref{prop:SetOperation}(4)和\cref{prop:Measure}(1)可知:
	\begin{align*}
		&\int_{A}f_n(x)\dif\mu=\int_{A\cap X}f_n(x)\dif\mu=\int_{A\cap\left(\underset{k=1}{\overset{n}{\cup}}B_{nk}\right)}f_n(x)\dif\mu=\int_{\underset{k=1}{\overset{n}{\cup}}(A\cap B_{nk})}f_n(x)\dif\mu \\
		=&\sum_{k=1}^{n}\int_{A\cap B_{nk}}f_n(x)\dif\mu=\sum_{k=1}^{n}\int_{A\cap B_{nk}}g_k(x)\dif\mu\leqslant\sum_{k=1}^{n}\varphi(A\cap B_{nk})=\varphi(A)
	\end{align*}
	根据\cref{theo:LeviTheorem}和\cref{prop:RSeq}(6)可得:
	\begin{equation*}
		\int_{A}f(x)\dif\mu=\lim_{n\to+\infty}\int_{A}f_n(x)\dif\mu\leqslant\varphi(A)
	\end{equation*}
	因为$\varphi$是$(X,\mathscr{F})$上的有限测度,所以$f$在$X$上可积,于是$f\in\mathscr{C}$,即:
	\begin{equation*}
		\int_{X}f(x)\dif\mu=\alpha\qedhere
	\end{equation*}
\end{proof}
\begin{lemma}\label{lem:RandonNikodym2}
	设$\varphi$和$\mu$都是可测空间$(X,\mathscr{F})$上的有限测度。若$\varphi\ll\mu$,则存在非负的$f\in L_1(X,\mathscr{F},\mu)$使得对任意的$A\in\mathscr{F}$有:
	\begin{equation*}
		\varphi(A)=\int_{A}f(x)\dif\mu
	\end{equation*}
	且$f$在a.e.于$(X,\mathscr{F},\mu)$的意义下唯一。
\end{lemma}
\begin{proof}
	取满足\cref{lem:RandonNikodym1}中条件的$f$,对任意的$A\in\mathscr{F}$,令:
	\begin{equation*}
		\nu(A)=\varphi(A)-\int_{A}f(x)\dif\mu
	\end{equation*}
	由\cref{lem:RandonNikodym1}、\cref{prop:MeasurableIntegral}(1)和\cref{theo:MeasurableCountableIntegral}可知$\nu$是$(X,\mathscr{F})$上的测度。对任意的$n\in\mathbb{N}^+$和任意的$A\in\mathscr{F}$,定义:
	\begin{equation*}
		\nu_n(A)=\nu(A)-\frac{1}{n}\mu(A)
	\end{equation*}
	因为$\nu$和$\mu$是$(X,\mathscr{F})$上的测度,所以$\nu_n$为$(X,\mathscr{F})$上的符号测度。记$\{X_n^+,X_n^-\}$为$\nu_n$对应的Hahn分解,令:
	\begin{equation*}
		X^+=\underset{n=1}{\overset{+\infty}{\cup}}X_n^+,\quad X^-=\underset{n=1}{\overset{+\infty}{\cap}}X_n^-
	\end{equation*}
	由\cref{prop:SetOperation}(7)可得:
	\begin{equation*}
		X^-=\underset{n=1}{\overset{+\infty}{\cap}}(X\setminus X_n^+)=X\Big\backslash\left(\underset{n=1}{\overset{+\infty}{\cup}}X_n^+\right)
	\end{equation*}
	所以$X^+\cap X^-=\varnothing$且$X^+\cup X^-=X$。\par
	因为$X^-\subseteq X_n^-$,所以$\nu_n(X^-)\leqslant0$,于是由测度的非负性和$\mu$的有限性可得:
	\begin{equation*}
		0\leqslant\nu(X^-)=\nu_n(X^-)+\frac{1}{n}\mu(X^-)\leqslant\frac{1}{n}\mu(X^-)\to0
	\end{equation*}
	即$\nu(X^-)=0$。对任意的$n\in\mathbb{N}^+$和任意的$A\in\mathscr{F}$,由\cref{prop:SimpleFunction}(3)(1)、\cref{prop:NonnegativeMeasurableIntegral}(10)、\cref{prop:Measure}(3)(单调性)、\cref{prop:NonnegativeSimpleIntegral}(5)和$\varphi$的有限性可得:
	\begin{align*}
		&\int_{A}\left[f(x)+\frac{1}{n}I(x\in X_n^+)\right]\dif\mu=\int_{A}f(x)\dif\mu+\int_{A}\frac{1}{n}I(x\in X_n^+)\dif\mu \\
		=&\varphi(A)-\nu(A)+\int_{A}\frac{1}{n}I(x\in X_n^+)\dif\mu\leqslant\varphi(A)-\nu(A\cap X_n^+)+\int_{A}\frac{1}{n}I(x\in X_n^+)\dif\mu \\
		=&\varphi(A)-\nu(A\cap X_n^+)+\frac{1}{n}\int_{A}I(x\in X_n^+)\dif\mu=\varphi(A)-\nu(A\cap X_n^+)+\frac{1}{n}\mu(A\cap X_n^+) \\
		=&\varphi(A)-\nu_n(A\cap X_n^+)\leqslant\varphi(A)<+\infty
	\end{align*}
	所以$f(x)+\dfrac{1}{n}I(x\in X_n^+)\in\mathscr{F}$。取$A=X$,根据\cref{lem:RandonNikodym1}可得:
	\begin{align*}
		&\int_{X}\left[f(x)+\frac{1}{n}I(x\in X_n^+)\right]\dif\mu=\int_{X}f(x)\dif\mu+\frac{1}{n}\mu(X_n^+) \\
		\leqslant&\sup\left\{\int_{X}g(x)\dif\mu:g\in\mathscr{C}\right\}=\int_{X}f(x)\dif\mu
	\end{align*}
	所以$\mu(X_n^+)=0$。由测度的非负性和\cref{prop:Measure}(3)(次可列可加性)可得:
	\begin{equation*}
		0\leqslant\mu(X^+)\leqslant\sum_{n=1}^{+\infty}\mu(X_n^+)=0
	\end{equation*}
	所以$\mu(X^+)=0$。因为$\varphi\ll\mu$,所以$\varphi(X^+)=0$,于是由\cref{prop:NonnegativeSimpleIntegral}(3)可得:
	\begin{equation*}
		\nu(X^+)=\varphi(X^+)-\int_{X^+}f(x)\dif\mu=0
	\end{equation*}
	根据\cref{prop:Measure}(1)可得:
	\begin{equation*}
		\nu(X)=\nu(X^+\cup X^-)=\nu(X^+)+\nu(X^-)=0
	\end{equation*}
	由\cref{prop:Measure}(3)(单调性)可得对任意的$A\in\mathscr{F}$有:
	\begin{equation*}
		\nu(A)=\varphi(A)-\int_{A}f(x)\dif\mu=0
	\end{equation*}
	$f$的唯一性由\cref{prop:MeasurableIntegral}(11)立即可得。
\end{proof}
\begin{lemma}\label{lem:RandonNikodym3}
	设$\varphi$和$\mu$分别是可测空间$(X,\mathscr{F})$上的$\sigma$有限符号测度和有限测度。若$\varphi\ll\mu$,则存在有限的可测函数$f$使得对任意的$A\in\mathscr{F}$有:
	\begin{equation*}
		\int_{A}f^-(x)\dif\mu<+\infty,\quad\varphi(A)=\int_{A}f(x)\dif\mu
	\end{equation*}
	同时$f$在a.e.于$(X,\mathscr{F},\mu)$的意义下唯一。
\end{lemma}
\begin{proof}
	先讨论$\varphi$是有限符号测度的情况,此时$\varphi$的Jordan分解$\varphi^+$和$\varphi^-$都是$(X,\mathscr{F})$上的有限测度。对任意满足$\varphi(A)=0$的$A\in\mathscr{F}$,由\cref{prop:Measure}(3)可知$\varphi^+(A)=\varphi^-(A)=0$,所以$\varphi^+,\varphi^-\ll\mu$。根据\cref{lem:RandonNikodym2}可知存在非负的$f^+,f^-\in L_1(X,\mathscr{F},\mu)$使得:
	\begin{equation*}
		\forall\;A\in\mathscr{F},\;\varphi^+(A)=\int_{A}f^-(x)\dif\mu,\;\varphi^-(A)=\int_{A}f^-(x)\dif\mu
	\end{equation*}
	且$f^+,f^-$在a.e.于$(X,\mathscr{F},\mu)$的意义下唯一,由\cref{prop:NonnegativeMeasurableIntegral}(8)(7)和\cref{prop:MeasurableFunction}(9)可取$f^+,f^-$为有限的可测函数。取$f=f^+-f^-$,则$f$是有限的可测函数,由\cref{prop:MeasurableFunction}(5.a)、\cref{prop:MeasurableIntegral}(6)就有:
	\begin{equation*}
		\forall\;A\in\mathscr{F},\;\int_{A}f(x)\dif\mu=\int_{A}f^+(x)\dif\mu-\int_{A}f^-(x)\dif\mu=\varphi^+(A)-\varphi^-(A)
	\end{equation*}
	$f$的唯一性由\cref{prop:Measure}(3)(次有限可加性)保证。\par
	当$\varphi$是$\sigma$有限符号测度时,存在互不相交$\{A_n\}\subseteq\mathscr{F}$使得:
	\begin{equation*}
		\underset{n=1}{\overset{+\infty}{\cup}}A_n=X,\quad\forall\;A\in\mathscr{F},\;|\varphi(A_n)|<+\infty
	\end{equation*}
	由\cref{prop:SignedMeasure}(3)可知$\varphi$限制在$(A_n,A_n\cap\mathscr{F})$上是有限符号测度,而$\mu$限制在$(A_n,A_n\cap\mathscr{F})$上是有限测度,且此时仍有$\varphi\ll\mu$,于是根据之前$\varphi$是有限符号测度情况的讨论可知在每个$(A_n,A_n\cap\mathscr{F})$存在有限的可测函数$f_n$满足$\int_{A}f_n^-(x)\dif\mu<+\infty$且:
	\begin{equation*}
		\forall\;A\in A_n\cap\mathscr{F},\;\varphi(A)=\int_{A}f_n(x)\dif\mu
	\end{equation*}
	同时$f_n$在a.e.于$(A_n,A_n\cap\mathscr{F},\mu)$的意义下唯一。将$f_n$延拓到$X$上:
	\begin{equation*}
		f_n'(x)=
		\begin{cases}
			f_n(x),&x\in A_n \\
			0,&x\in A_n^c
		\end{cases}
	\end{equation*}
	由\cref{prop:MeasurableFunction}(1)可验证得到$f_n'$是$(X,\mathscr{F})$上有限的可测函数,取:
	\begin{equation*}
		g_n=\sum_{i=1}^{n}f_i',\quad f=\sum_{n=1}^{+\infty}f_n'
	\end{equation*}
	因为$f_n'$在$(A_n,A_n\cap\mathscr{F})$上有限,所以$f$在$X$上有限,而$f=\lim\limits_{n\to+\infty}g_n$,根据\cref{prop:MeasurableFunction}(5.a)(6)可知$f$是可测函数,由\cref{prop:Measure}(3)(次可列可加性)可得$f$在a.e.于$(X,\mathscr{F},\mu)$的意义下唯一。对任意的$A\in\mathscr{F}$有:
	\begin{equation*}
		\varphi(A_n\cap A)=\int_{A_n\cap A}f_n(x)\dif\mu=\int_{A_n\cap A}f(x)\dif\mu
	\end{equation*}
	于是由\cref{theo:MeasurableCountableIntegral}、\cref{prop:SigmaField}(4)、\cref{prop:NonnegativeMeasurableIntegral}(5)、\cref{prop:MeasurableIntegral}(6)、\cref{prop:Series}(4)和\cref{prop:SignedMeasure}(2)处的约定可得:
	\begin{align*}
		&\int_{X}f^-(x)\dif\mu=\int_{\underset{n=1}{\overset{+\infty}{\cup}}A_n}f^-(x)\dif\mu=\sum_{n=1}^{+\infty}\int_{A_n}f^-(x)\dif\mu=\sum_{n=1}^{+\infty}\int_{A_n\cap\{f<0\}}-f(x)\dif\mu \\
		=&-\sum_{n=1}^{+\infty}\int_{A_n\cap\{f<0\}}f(x)\dif\mu=-\sum_{n=1}^{+\infty}\varphi(A_n\cap\{f<0\})=-\varphi(\{f<0\})<+\infty
	\end{align*}
	所以根据\cref{prop:MeasurableIntegral}(3)可知对任意的$A\in\mathscr{F}$有:
	\begin{equation*}
		\int_{A}f^-(x)\dif\mu<+\infty
	\end{equation*}
	由\cref{prop:SetOperation}(4)和\cref{theo:MeasurableCountableIntegral}可知对任意的$A\in\mathscr{F}$有:
	\begin{align*}
		&\varphi(A)=\varphi(A\cap X)=\varphi\left[A\cap\left(\underset{n=1}{\overset{+\infty}{\cup}}A_n\right)\right]=\varphi\left[\underset{n=1}{\overset{+\infty}{\cup}}(A_n\cap A)\right] \\
		=&\sum_{n=1}^{+\infty}\varphi(A_n\cap A)=\sum_{n=1}^{+\infty}\int_{A_n\cap A}f(x)\dif\mu=\int_{A}f(x)\dif\mu\qedhere
	\end{align*}
\end{proof}
\begin{lemma}\label{lem:RandonNikodym4}
	设$\varphi$和$\mu$分别是可测空间$(X,\mathscr{F})$上的符号测度和有限测度。若$\varphi\ll\mu$,则存在可测函数$f$使得对任意的$A\in\mathscr{F}$有:
	\begin{equation*}
		\int_{A}f^-(x)\dif\mu<+\infty,\quad\varphi(A)=\int_{A}f(x)\dif\mu
	\end{equation*}
	同时$f$在a.e.于$(X,\mathscr{F},\mu)$的意义下唯一。若$\varphi$是$\sigma$有限的,则$f$也是有限的。
\end{lemma}
\begin{proof}
	记:
	\begin{equation*}
		\mathscr{D}=\left\{A\in\mathscr{F}:\text{存在互不相交的}\{A_n\}\subseteq\mathscr{F}\text{使得}\underset{n=1}{\overset{+\infty}{\cup}}A_n=A\text{且对任意的}n\in\mathbb{N}^+\text{有}|\varphi(A_n)|<+\infty\right\}
	\end{equation*}
	由\info{可列个可列是可列}可知$\mathscr{D}$对可列不交并封闭。任取$A,B\in\mathscr{D}$,则存在互不相交的$\{A_n\}\subseteq\mathscr{F}$和互不相交的$\{B_n\}\subseteq\mathscr{F}$满足:
	\begin{equation*}
		A=\underset{n=1}{\overset{+\infty}{\cup}}A_n,\; B=\underset{n=1}{\overset{+\infty}{\cup}}B_n,\quad|\varphi(A_n)|,|\varphi(B_n)|<+\infty,\;\forall\;n\in\mathbb{N}^+
	\end{equation*}
	于是由\cref{prop:SetOperation}(4)(6)(7)可得:
	\begin{align*}
		&A\setminus B=\left(\underset{n=1}{\overset{+\infty}{\cup}}A_n\right)\Big\backslash\left(\underset{n=1}{\overset{+\infty}{\cup}}B_n\right)=\underset{n=1}{\overset{+\infty}{\bigcup}}\left[A_n\Big\backslash\left(\underset{n=1}{\overset{+\infty}{\cup}}B_n\right)\right] \\
		=&\underset{n=1}{\overset{+\infty}{\bigcup}}\left[A_n\bigcap\left(\underset{n=1}{\overset{+\infty}{\cup}}B_n\right)^c\right]=\underset{n=1}{\overset{+\infty}{\bigcup}}\left[A_n\bigcap\left(\underset{n=1}{\overset{+\infty}{\cap}}B_n^c\right)\right]
	\end{align*}
	根据\cref{prop:SigmaField}(2)和\cref{prop:SignedMeasure}(3)可得:
	\begin{equation*}
		A_n\bigcap\left(\underset{n=1}{\overset{+\infty}{\cap}}B_n^c\right)\in\mathscr{F},\;\left\lvert\varphi\left[A_n\bigcap\left(\underset{n=1}{\overset{+\infty}{\cap}}B_n^c\right)\right]\right\rvert<|\varphi(A_n)|<+\infty
	\end{equation*}
	因为$\{A_n\}$互不相交,所以$\left\{A_n\bigcap\left(\underset{n=1}{\overset{+\infty}{\cap}}B_n^c\right)\right\}$也互不相交,于是$A\setminus B\in\mathscr{D}$,$\mathscr{D}$对差封闭。任取$\{A_n\}\subseteq\mathscr{D}$,则:
	\begin{equation*}
		\underset{n=1}{\overset{+\infty}{\cup}}\left[A_n\Big\backslash\left(\underset{i=1}{\overset{n-1}{\cup}}A_n\right)\right]
	\end{equation*}
	因为$\mathscr{D}$对差和可列不交并封闭且$\varnothing\in\mathscr{D}$,所以$\underset{n=1}{\overset{+\infty}{\cup}}A_n\in\mathscr{D}$,即$\mathscr{D}$对可列并封闭。\par
	综上,$\mathscr{D}$是一个$\sigma$环。\par
	记:
	\begin{equation*}
		\alpha=\sup\{\mu(A):A\in\mathscr{D}\}
	\end{equation*}
	取$\{B_n\}\subseteq\mathscr{D}$满足:
	\begin{equation*}
		\lim_{n\to+\infty}\mu(B_n)=\alpha
	\end{equation*}
	令$B=\underset{n=1}{\overset{+\infty}{\cup}}B_n$,因为$\mathscr{D}$是一个$\sigma$环,所以$B\in\mathscr{D}$,由\cref{prop:Measure}(3)(单调性)可得:
	\begin{equation*}
		\alpha\geqslant\mu(B)\geqslant\mu(B_n)\to\alpha
	\end{equation*}
	根据\cref{prop:RSeq}(4)可知$\mu(B)=\alpha$。\par
	因为$\varphi$和$\mu$分别是$(B,B\cap\mathscr{F})$上的$\sigma$有限符号测度和有限测度,$\varphi\ll\mu$,由\cref{lem:RandonNikodym3}可知存在有限的可测函数$g$使得对任意的$A\in B\cap\mathscr{F}$有:
	\begin{equation*}
		\int_{A}g^-(x)\dif\mu<+\infty,\quad\varphi(A)=\int_{A}g(x)\dif\mu
	\end{equation*}
	且$g$在a.e.于$(B,B\cap\mathscr{F},\mu)$的意义下唯一。\par
	对任意的$A\in B^c\cap\mathscr{F}$,若$\mu(A)=0$,因为$\varphi\ll\mu$,所以$\varphi(A)=0$;若$\mu(A)>0$,则$\varphi(A)=+\infty$:若$\varphi(A)\ne+\infty$,则$A\cup B\in\mathscr{D}$,由\cref{prop:Measure}(1)可得:
	\begin{equation*}
		\mu(A\cup B)=\mu(A)+\mu(B)>\mu(B)
	\end{equation*}
	与$B$的取法矛盾。所以根据\cref{prop:MeasurableIntegral}(1)和\cref{prop:NonnegativeMeasurableIntegral}(8):
	\begin{equation*}
		\varphi(A)=\int_{A}+\infty\dif\mu=
		\begin{cases}
			0,&\mu(A)=0 \\
			+\infty,&\mu(A)>0
		\end{cases}
	\end{equation*}\par
	取$f(x)=g(x)I(x\in B)+\infty I(x\in B^c)$,由\cref{prop:MeasurableFunction}(1)可验证得到$f$是可测函数。对任意的$A\in\mathscr{F}$,根据\cref{prop:SigmaField}(2)(4)和\cref{prop:NonnegativeMeasurableIntegral}(5)可知:
	\begin{equation*}
		\int_{A}f^-(x)\dif\mu=\int_{A\cap B}g^-(x)\dif\mu<+\infty
	\end{equation*}
	由\cref{prop:SetOperation}(4)、\cref{prop:Measure}(1)和\cref{theo:MeasurableCountableIntegral}可得:
	\begin{align*}
		&\varphi(A)=\varphi(A\cap B)+\varphi(A\cap B^c)=\int_{A\cap B}g(x)\dif\mu+\int_{A\cap B^c}+\infty\dif\mu \\
		=&\int_{A\cap B}f(x)\dif\mu+\int_{A\cap B^c}f(x)\dif\mu=\int_{A}f(x)\dif\mu
	\end{align*}
	$f$的唯一性可由$g$的唯一性直接得到。若$\varphi$是$\sigma$有限的,则$\mathscr{D}=\mathscr{F}$,由\cref{prop:Measure}(3)可知$B=X$,所以$f=g$,$f$有限。
\end{proof}
\begin{theorem}[Randon-Nikodym Theorem]\label{theo:RandonNikodym}
	设$\varphi$和$\mu$分别是可测空间$(X,\mathscr{F})$上的符号测度和$\sigma$有限测度。若$\varphi\ll\mu$,则存在$(X,\mathscr{F})$上的可测函数$f$满足:
	\begin{equation*}
		\forall\;A\in\mathscr{F},\;\varphi(A)=\int_{A}f(x)\dif\mu,\;\int_{A }f^-(x)\dif\mu<+\infty
	\end{equation*}
	且$f$在a.e.于$(X,\mathscr{F},\mu)$的意义下唯一。若$\varphi$是$\sigma$有限的,则$f$也是有限的。称$f$为$\varphi$关于$\mu$的Randon-Nikodym导数,记作$\dfrac{\dif\varphi}{\dif\mu}$。
\end{theorem}
\begin{proof}
	因为$\mu$是$\sigma$有限测度,所以可取$(X,\mathscr{F})$的一个可列可测分割$\{A_n\}$,满足对任意的$n\in\mathbb{N}^+$有$\mu(A_n)<+\infty$。由\cref{prop:Measure}(3)(单调性)和\cref{lem:RandonNikodym4}可知存在$(A_n,A_n\cap\mathscr{F})$上的可测函数$f_n$使得对任意的$A\in A_n\cap\mathscr{F}$有:
	\begin{equation*}
		\int_{A}f_n^-(x)\dif\mu<+\infty,\quad\varphi(A)=\int_{A}f_n(x)\dif\mu
	\end{equation*}
	同时$f_n$在a.e.于$(A_n,A_n\cap\mathscr{F},\mu)$的意义下唯一,若$\varphi$在$(A_n,A_n\cap\mathscr{F})$上是$\sigma$有限的,则$f_n$也是有限的。取$f=\sum\limits_{n=1}^{+\infty}f_n$,与\cref{lem:RandonNikodym3}完全一样的过程可以证明得到$f$是一个可测函数,在a.e.于$(X,\mathscr{F},\mu)$的意义下唯一且满足:
	\begin{equation*}
		\forall\;A\in\mathscr{F},\;\varphi(A)=\int_{A}f(x)\dif\mu,\;\int_{A}f^-(x)\dif\mu<+\infty
	\end{equation*}
	若$\varphi$是$\sigma$有限的,则由\cref{prop:SignedMeasure}(1)(4)可得$\varphi$在$(A_n,A_n\cap\mathscr{F})$上是$\sigma$有限的,所以$f_n$在$A_n$上是有限的,于是$f$在$X$上是有限的。
\end{proof}
\begin{lemma}\label{lem:IntChangeOfMeasure}
	设$\varphi$和$\mu$是可测空间$(X,\mathscr{F})$上的$\sigma$有限测度,$\varphi\ll\mu$。对任意$(X,\mathscr{F})$上的可测函数$f$和任意的$A\in\mathscr{F}$,只要:
	\begin{equation*}
		\int_{A}f(x)\dif\varphi,\quad\int_{A}f(x)\frac{\dif\varphi}{\dif\mu}\dif\mu
	\end{equation*}
	之一有意义,二者一定相等。
\end{lemma}
\begin{proof}
	使用典型方法进行证明。仅对上左侧式子有意义的情况进行证明,右侧有意义的情况可由等号的传递性得出。\par
	由\cref{prop:MeasurableFunction}(1)可知:
	\begin{equation*}
		\left\{\frac{\dif\varphi}{\dif\mu}<0\right\}\in\mathscr{F}
	\end{equation*}
	若上述集合关于$\mu$的测度大于$0$,由\cref{prop:NonnegativeMeasurableIntegral}(2)(9)和\cref{prop:MeasurableIntegral}(6)可得:
	\begin{equation*}
		\varphi\left(\left\{\frac{\dif\varphi}{\dif\mu}<0\right\}\right)=\int_{\left\{\frac{\dif\varphi}{\dif\mu}<0\right\}}\frac{\dif\varphi}{\dif\mu}(x)\dif\mu<0
	\end{equation*}
	与$\varphi$是测度相矛盾,所以有:
	\begin{equation*}
		\mu\left(\left\{\frac{\dif\varphi}{\dif\mu}<0\right\}\right)=0
	\end{equation*}\par
	\textbf{(1)非负简单函数:}设$f$是$(X,\mathscr{F})$上的非负简单函数,由\cref{prop:MeasurableIntegral}(6)和\cref{theo:MeasurableCountableIntegral}可知:
	\begin{align*}
		\int_{A}f(x)\dif\varphi&=\sum_{i=1}^{n}c_i\varphi(A\cap E_i)=\sum_{i=1}^{n}\left[c_i\int_{A\cap E_i}\frac{\dif\varphi}{\dif\mu}(x)\dif\mu\right] \\
		&=\sum_{i=1}^{n}\left[\int_{A\cap E_i}f(x)\frac{\dif\varphi}{\dif\mu}(x)\dif\mu\right]=\int_{A}f(x)\frac{\dif\varphi}{\dif\mu}(x)\dif\mu
	\end{align*}\par
	\textbf{(2)非负可测函数:}设$f$是$(X,\mathscr{F})$上的非负可测函数,根据\cref{prop:MeasurableFunction}(8),取$(X,\mathscr{F})$上的非负简单函数列$\{f_n\}$满足$f_n\uparrow f$。由\cref{prop:NonnegativeMeasurableIntegral}(4)和非负简单函数时的结论可得:
	\begin{equation*}
		\int_{A}f(x)\dif\varphi=\lim_{n\to+\infty}\left[\int_{A}f_n(x)\dif\varphi\right]=\lim_{n\to+\infty}\left[\int_{A}f_n(x)\frac{\dif\varphi}{\dif\mu}(x)\dif\mu\right]
	\end{equation*}
	因为$\mu\left(\left\{\dfrac{\dif\varphi}{\dif\mu}<0\right\}\right)=0$,所以$f_n\dfrac{\dif\varphi}{\dif\mu}$非负a.e.于$(X,\mathscr{F},\mu)$且有$f_n\dfrac{\dif\varphi}{\dif\mu}\Big\uparrow f\dfrac{\dif\varphi}{\dif\mu}$,由\cref{theo:LeviTheorem}即可得到:
	\begin{equation*}
		\int_{A}f(x)\dif\varphi=\lim_{n\to+\infty}\left[\int_{A}f_n(x)\frac{\dif\varphi}{\dif\mu}\dif\mu\right]=\int_{A}f(x)\frac{\dif\varphi}{\dif\mu}\dif\mu
	\end{equation*}\par
	\textbf{(3)一般可测函数:}设$f$是$(X,\mathscr{F})$上的可测函数,由非负可测函数时的情形和\cref{prop:MeasurableIntegral}(8)可得:
	\begin{align*}
		\int_{A}f(x)\dif\varphi&=\int_{A}f^+(x)\dif\varphi-\int_{A}f^-(x)\dif\varphi=\int_{A}f^+(x)\frac{\dif\varphi}{\dif\mu}(x)\dif\mu-\int_{A}f^-(x)\frac{\dif\varphi}{\dif\mu}(x)\dif\mu \\
		&=\int_{A}\left[f(x)\frac{\dif\varphi}{\dif\mu}(x)\right]^+\dif\mu-\int_{A}\left[f(x)\frac{\dif\varphi}{\dif\mu}(x)\right]^-\dif\mu=\int_{A}f(x)\frac{\dif\varphi}{\dif\mu}(x)\dif\mu\qedhere
	\end{align*}
\end{proof}
\begin{property}
	设$(X,\mathscr{F})$是一个可测空间。Randon-Nikodym导数的计算具有如下性质:
	\begin{enumerate}
		\item 设$\varphi$是$(X,\mathscr{F})$上的符号测度,$\nu$和$\mu$是$(X,\mathscr{F})$上的$\sigma$有限测度且$\varphi\ll\nu\ll\mu$,则有:
		\begin{equation*}
			\frac{\dif\varphi}{\dif\mu}=\frac{\dif\varphi}{\dif\nu}\cdot\frac{\dif\nu}{\dif\mu}\;\text{a.e.于}(X,\mathscr{F},\nu)
		\end{equation*}
		\item 设$\nu$和$\mu$是$(X,\mathscr{F})$上的$\sigma$有限测度且$\nu\ll\mu$,则$\mu\ll\nu$当且仅当$\dfrac{\dif\nu}{\dif\mu}>0\;$a.e.于$(X,\mathscr{F},\mu)$,此时有:
		\begin{equation*}
			\frac{\dif\nu}{\dif\mu}\cdot\frac{\dif\mu}{\dif\nu}=1\;\text{a.e.于}(X,\mathscr{F},\mu)\text{或}(X,\mathscr{F},\nu)
		\end{equation*}
		\item 若$\varphi$是$(X,\mathscr{F})$上的符号测度,$\mu$是$(X,\mathscr{F})$上的$\sigma$有限测度,$\varphi\ll\mu$,则对任意的$a\in\mathbb{R}^{}$有:
		\begin{equation*}
			\frac{\dif(a\varphi)}{\dif\mu}=a\frac{\dif\varphi}{\dif\mu}\;\text{a.e.于}(X,\mathscr{F},\mu)
		\end{equation*}
		\item 若$\varphi,\psi$是$(X,\mathscr{F})$上的符号测度,$\mu$是$(X,\mathscr{F})$上的$\sigma$有限测度且$\varphi,\psi\ll\mu$。定义$(\varphi+\psi)(A)\coloneq\varphi(A)+\psi(A),\;\forall\;A\in\mathscr{F}$,如果$\varphi+\psi$也是$(X,\mathscr{F})$上的符号测度,则有:
		\begin{equation*}
			\frac{\dif(\varphi+\psi)}{\dif\mu}=\frac{\dif\varphi}{\dif\mu}+\frac{\dif\psi}{\dif\mu}\;\text{a.e.于}(X,\mathscr{F},\mu)
		\end{equation*}
	\end{enumerate}
\end{property}
\begin{proof}
	(1)因为$\varphi$是符号测度,$\nu$是$\sigma$有限测度且$\varphi\ll\nu$,根据\cref{theo:RandonNikodym}可知存在$(X,\mathscr{F},\nu)$上在a.e.意义下唯一的可测函数$\dfrac{\dif\varphi}{\dif\nu}$使得:
	\begin{equation*}
		\forall\;A\in\mathscr{F},\;\varphi(A)=\int_{A}\frac{\dif\varphi}{\dif\nu}(x)\dif\nu
	\end{equation*}
	由\cref{lem:IntChangeOfMeasure}可得:
	\begin{equation*}
		\varphi(A)=\int_{A}\frac{\dif\varphi}{\dif\nu}(x)\frac{\dif\nu}{\dif\mu}(x)\dif\mu
	\end{equation*}
	因为$\varphi$是符号测度,$\mu$是$\sigma$有限测度且$\varphi\ll\mu$,根据\cref{theo:RandonNikodym}可知存在$(X,\mathscr{F},\mu)$上在a.e.意义下唯一的可测函数$\dfrac{\dif\varphi}{\dif\mu}$使得:
	\begin{equation*}
		\forall\;A\in\mathscr{F},\;\varphi(A)=\int_{A}\frac{\dif\varphi}{\dif\mu}(x)\dif\mu
	\end{equation*}
	因为$\nu\ll\mu$,所以在$(X,\mathscr{F},\mu)$上a.e.即在$(X,\mathscr{F},\nu)$上a.e.,于是有:
	\begin{equation*}
		\frac{\dif\varphi}{\dif\nu}\frac{\dif\nu}{\dif\mu}=\frac{\dif\varphi}{\dif\mu}\;\text{a.e.于}(X,\mathscr{F},\nu)
	\end{equation*}\par
	(2)因为$\nu\ll\mu$且$\nu$和$\mu$是$\sigma$有限测度,根据\cref{theo:RandonNikodym}可知存在$(X,\mathscr{F},\mu)$上在a.e.意义下唯一的可测函数$\dfrac{\dif\nu}{\dif\mu}$使得:
	\begin{equation*}
		\forall\;A\in\mathscr{F},\;\nu(A)=\int_{A}\frac{\dif\nu}{\dif\mu}(x)\dif\mu
	\end{equation*}
	由\cref{lem:IntChangeOfMeasure}中的论证可得:
	\begin{equation*}
		\mu\left(\left\{\frac{\dif\nu}{\dif\mu}<0\right\}\right)=0
	\end{equation*}\par
	\textbf{必要性:}取集合:
	\begin{equation*}
		N=\left\{\frac{\dif\nu}{\dif\mu}=0\right\}
	\end{equation*}
	因为$\dfrac{\dif\nu}{\dif\mu}$是可测函数,由\cref{prop:MeasurableFunction}(2)可知$N\in\mathscr{F}$。由简单函数的定义可知$\dfrac{\dif\nu}{\dif\mu}$限制在$N$上时为非负简单函数,根据非负简单函数积分的定义可得到:
	\begin{equation*}
		\nu(N)=\int_{N}\frac{\dif\nu}{\dif\mu}\dif\mu=0
	\end{equation*}
	因为$\mu\ll\nu$,所以$\mu(N)=0$。由\cref{prop:Measure}(1)可得:
	\begin{equation*}
		\mu\left(\left\{\dfrac{\dif\nu}{\dif\mu}\leqslant0\right\}\right)=\mu\left(\left\{\dfrac{\dif\nu}{\dif\mu}<0\right\}\right)+\mu\left(\left\{\dfrac{\dif\nu}{\dif\mu}=0\right\}\right)=0
	\end{equation*}
	所以$\dfrac{\dif\nu}{\dif\mu}>0\;$a.e.于$(X,\mathscr{F},\mu)$。\par
	\textbf{充分性:}若此时不满足$\mu\ll\nu$,即存在满足$\nu(A)=0$的$A\in\mathscr{F}$有$\mu(A)>0$。因为$\dfrac{\dif\nu}{\dif\mu}>0\;$a.e.于$(X,\mathscr{F},\mu)$,由\cref{prop:MeasurableIntegral}(8)和\cref{prop:NonnegativeMeasurableIntegral}(2)(9)可得:
	\begin{equation*}
		\nu(A)=\int_{A}\frac{\dif\nu}{\dif\mu}(x)\dif\mu>0
	\end{equation*}
	与$\nu(A)=0$矛盾。\par
	由\cref{lem:IntChangeOfMeasure}可知:
	\begin{equation*}
		\forall\;A\in\mathscr{F},\;\nu(A)=\int_{A}\frac{\dif\nu}{\dif\mu}\dif\mu=\int_{A}\frac{\dif\nu}{\dif\mu}\cdot\frac{\dif\mu}{\dif\nu}\dif\nu
	\end{equation*}
	而:
	\begin{equation*}
		\forall\;A\in\mathscr{F},\;\int_{A}1\dif\nu=\nu(A)
	\end{equation*}
	于是由\cref{prop:MeasurableIntegral}(11)可得:
	\begin{equation*}
		\frac{\dif\nu}{\dif\mu}\cdot\frac{\dif\mu}{\dif\nu}=1\;\text{a.e.于}(X,\mathscr{F},\nu)
	\end{equation*}
	因为$\mu\ll\nu$,所以上式成立也a.e.于$(X,\mathscr{F},\mu)$。\par
	(3)根据\cref{prop:SignedMeasure}(5)可知$a\varphi$也是一个符号测度。因为$\varphi\ll\mu$,由$a\varphi$的定义可得$a\varphi\ll\mu$,根据\cref{theo:RandonNikodym}可知存在$(X,\mathscr{F},\mu)$上在a.e.意义下唯一的可测函数$\dfrac{\dif(a\varphi)}{\dif\mu}$使得:
	\begin{equation*}
		\forall\;A\in\mathscr{F},\;(a\varphi)(A)=a\varphi(A)=\int_{A}\frac{\dif(a\varphi)}{\dif\mu}(x)\dif\mu
	\end{equation*}
	由\cref{prop:MeasurableIntegral}(6)可知:
	\begin{equation*}
		\varphi(A)=\int_{A}\frac{1}{a}\frac{\dif(a\varphi)}{\dif\mu}(x)\dif\mu
	\end{equation*}
	根据\cref{theo:RandonNikodym}可知:
	\begin{equation*}
		\dfrac{1}{a}\frac{\dif(a\varphi)}{\dif\mu}=\dfrac{\dif\varphi}{\dif\mu}\;\text{a.e.于}(X,\mathscr{F},\mu)
	\end{equation*}
	即:
	\begin{equation*}
		\frac{\dif(a\varphi)}{\dif\mu}=a\dfrac{\dif\varphi}{\dif\mu}\;\text{a.e.于}(X,\mathscr{F},\mu)
	\end{equation*}\par
	(4)因为$\varphi,\psi\ll\mu$,根据$\varphi+\psi$的定义可得$\varphi+\psi\ll\mu$。由\cref{theo:RandonNikodym}可知存在$(X,\mathscr{F},\mu)$上在a.e.意义下唯一的可测函数$\dfrac{\dif(\varphi+\psi)}{\dif\mu}$使得:
	\begin{equation*}
		\forall\;A\in\mathscr{F},\;(\varphi+\psi)(A)=\varphi(A)+\psi(A)=\int_{A}\frac{\dif(\varphi+\psi)}{\dif\mu}(x)\dif\mu
	\end{equation*}
	同理可得:
	\begin{equation*}
		\varphi(A)=\int_{A}\frac{\dif\varphi}{\dif\mu}(x)\dif\mu,\quad\psi(A)=\int_{A}\frac{\dif\psi}{\dif\mu}(x)\dif\mu
	\end{equation*}
	由\cref{theo:RandonNikodym}中Randon-Nikodym导数负部的可积性和\cref{prop:MeasurableIntegral}(6)可得:
	\begin{equation*}
		\varphi(A)+\psi(A)=\int_{A}\frac{\dif\varphi}{\dif\mu}(x)\dif\mu+\int_{A}\frac{\dif\psi}{\dif\mu}(x)\dif\mu=\int_{A}\left[\frac{\dif\varphi}{\dif\mu}(x)+\frac{\dif\psi}{\dif\mu}(x)\right]\dif\mu
	\end{equation*}
	根据\cref{theo:RandonNikodym}可知:
	\begin{equation*}
		\frac{\dif(\varphi+\psi)}{\dif\mu}=\frac{\dif\varphi}{\dif\mu}+\frac{\dif\psi}{\dif\mu}\;\text{a.e.于}(X,\mathscr{F},\mu)\qedhere
	\end{equation*}
\end{proof}

\subsection{Lebesgue分解}
\begin{definition}
	设$\varphi$和$\psi$是可测空间$(X,\mathscr{F})$上的符号测度。若存在$N\in\mathscr{F}$使得$|\varphi|(N^c)=|\psi|(N)=0$,则称$\varphi$和$\psi$是\gls{MutuallySingular}的,记作$\varphi\perp\psi$。
\end{definition}
\begin{definition}
	设$\varphi$和$\psi$是可测空间$(X,\mathscr{F})$上的符号测度。若$|\varphi|\ll|\psi|$,则称$\varphi$对$\psi$绝对连续,记作$\varphi\ll\psi$。
\end{definition}
\begin{property}\label{prop:SingularSignedMeasure}
	设$\varphi$和$\psi$是可测空间$(X,\mathscr{F})$上的符号测度,则:
	\begin{enumerate}
		\item $\varphi\perp\psi$当且仅当存在$N\in\mathscr{F}$对任意的$A\in\mathscr{F}$有:
		\begin{equation*}
			\varphi(A\cap N^c)=\psi(A\cap N)=0
		\end{equation*}
		且上述$N$即为使得$|\varphi|(N^c)=|\psi|(N)=0$的集合;
		\item 若$\varphi\ll\psi$且$\varphi\perp\psi$,则$\varphi=0$。
	\end{enumerate}
\end{property}
\begin{proof}
	(1)\textbf{必要性:}因为$\varphi\perp\psi$,所以存在$N\in\mathscr{F}$使得$|\varphi|(N^c)=|\psi|(N)=0$,即$\varphi^{\pm}(N^c)=\psi^{\pm}(N)=0$。对任意的$A\in\mathscr{F}$,由测度的非负性和\cref{prop:Measure}(3)(单调性)可得:
	\begin{gather*}
		\varphi(A\cap N^c)=\varphi^+(A\cap N^c)-\varphi^-(A\cap N^c)=0 \\
		\psi(A\cap N^c)=\psi^+(A\cap N^c)-\psi^-(A\cap N^c)=0
	\end{gather*}\par
	\textbf{充分性:}取$N\in\mathscr{F}$使得对任意的$A\in\mathscr{F}$有$\varphi(A\cap N^c)=\psi(A\cap N)=0$。设$\{X_{\varphi}^+,X_{\varphi}^-\}$是$\varphi$的Hahn分解,由\cref{theo:HahnDecomposition}可知:
	\begin{equation*}
		|\varphi|(N^c)=\varphi^+(N^c)+\varphi^-(N^c)=\varphi(N^c\cap X_{\varphi}^+)+\varphi(N^c\cap X_{\varphi}^-)=0
	\end{equation*}
	同理可得$|\psi|(N)=0$。\par
	(2)取$N\in\mathscr{F}$使得$|\varphi|(N^c)=|\psi|(N)=0$。对任意的$A\in\mathscr{F}$,由测度的非负性、\cref{prop:SigmaField}(2)和\cref{prop:Measure}(3)(单调性)可得:
	\begin{equation*}
		0\leqslant|\varphi|(A\cap N^c)\leqslant|\varphi|(N^c)=0
	\end{equation*}
	所以$|\varphi|(A\cap N^c)=0$,同理可得$|\psi|(A\cap N)=0$。因为$\varphi\ll\psi$,所以$|\varphi|(A\cap N)=0$。由\cref{prop:Measure}(1)可得:
	\begin{equation*}
		|\varphi|(A)=|\varphi|(A\cap N^c)+|\varphi|(A\cap N)=0
	\end{equation*}
	所以$\varphi^{\pm}(A)=0$,$\varphi(A)=0$。
\end{proof}
\begin{lemma}\label{lem:LebesgueDecomposition1}
	设$\varphi,\mu$是可测空间$(X,\mathscr{F})$上的有限测度,则存在两个$(X,\mathscr{F})$上的有限测度$\varphi_c,\varphi_s$使得:
	\begin{equation*}
		\varphi=\varphi_c+\varphi_s,\;\varphi_c\ll\mu,\;\varphi_s\perp\mu
	\end{equation*}
\end{lemma}
\begin{proof}
	由定义可验证得到$\varphi+\mu$是一个测度,根据测度的非负性可知$\varphi\ll\varphi+\mu$,由\cref{theo:RandonNikodym}可知对任意的$A\in\mathscr{F}$有:
	\begin{equation*}
		0\leqslant\varphi(A)=\int_{A}\frac{\dif\varphi}{\dif(\varphi+\mu)}(x)\dif(\varphi+\mu)\leqslant\varphi(A)+\mu(A)=\int_{A}1\dif(\varphi+\mu)
	\end{equation*}
	根据\cref{prop:MeasurableIntegral}(10)可知:
	\begin{equation*}
		\frac{\dif\varphi}{\dif(\varphi+\mu)}\leqslant1\;\text{a.e.于}(X,\mathscr{F},\varphi+\mu)
	\end{equation*}
	记:
	\begin{equation*}
		N=\left\{\frac{\dif\varphi}{\dif(\varphi+\mu)}=1\right\}
	\end{equation*}
	由\cref{prop:MeasurableFunction}(2)可知$N\in\mathscr{F}$。根据\cref{prop:SigmaField}(2),对任意的$A\in\mathscr{F}$定义:
	\begin{equation*}
		\varphi_c(A)=\varphi(A\cap N^c),\quad\varphi_s(A)=\varphi(A\cap N)
	\end{equation*}
	根据定义和\cref{prop:Measure}(3)(单调性)(1)可验证得到$\varphi_c,\varphi_s$都是有限测度,且有$\varphi=\varphi_c+\varphi_s$。\par
	若$A\in\mathscr{F}$使得$\mu(A)=0$,则由\cref{prop:MeasurableIntegral}(6)和\cref{prop:Measure}(3)(单调性)可得:
	\begin{equation*}
		\int_{A\cap N^c}\left[1-\frac{\dif\varphi}{\dif(\varphi+\mu)}\right]\dif(\varphi+\mu)=(\varphi+\mu)(A\cap N^c)-\varphi(A\cap N^c)=\mu(A\cap N^c)=0
	\end{equation*}
	注意到$1-\dfrac{\dif\varphi}{\dif(\varphi+\mu)}>0\;$a.e.于$(A\cap N^c,A\cap N^c\cap\mathscr{F},\varphi+\mu)$,所以由\cref{prop:MeasurableIntegral}(8)和\cref{prop:NonnegativeMeasurableIntegral}(9)可知$(\varphi+\mu)(A\cap N^c)=0$。因为$\varphi\ll\varphi+\mu$,所以$\varphi_c(A)=\varphi(A\cap N^c)=0$,由$A$的任意性可得$\varphi_c\ll\mu$。\par
	因为:
	\begin{equation*}
		\varphi(N)=\int_{N}\frac{\dif\varphi}{\dif(\varphi+\mu)}(x)\dif(\varphi+\mu)=\int_{N}1\dif(\varphi+\mu)=\varphi(N)+\mu(N)
	\end{equation*}
	所以$\mu(N)=0$。而:
	\begin{equation*}
		\varphi_s(N^c)=\varphi(N^c\cap N)=\varphi(\varnothing)=0
	\end{equation*}
	所以$\varphi_s\perp\mu$。
\end{proof}
\begin{lemma}\label{lem:LebesgueDecomposition2}
	设$\varphi,\mu$是可测空间$(X,\mathscr{F})$上的$\sigma$有限测度,则存在两个$(X,\mathscr{F})$上的$\sigma$有限测度$\varphi_c,\varphi_s$使得:
	\begin{equation*}
		\varphi=\varphi_c+\varphi_s,\;\varphi_c\ll\mu,\;\varphi_s\perp\mu
	\end{equation*}
\end{lemma}
\begin{proof}
	取$(X,\mathscr{F})$的一个可列可测分割$\{A_n\}$满足:
	\begin{equation*}
		\forall\;n\in\mathbb{N}^+,\;\varphi(A_n)<+\infty,\;\mu(A_n)<+\infty
	\end{equation*}
	$\{A_n\}$的存在性由\cref{prop:Measure}(3)(单调性)和\cref{prop:SigmaField}(2)保证。根据\cref{prop:Measure}(3)(单调性)可知$\varphi,\mu$限制在$(A_n,A_n\cap\mathscr{F})$上是有限测度,由\cref{lem:LebesgueDecomposition1}可知存在$(A_n,A_n\cap\mathscr{F})$上的有限测度$\varphi_{cn}',\varphi_{sn}'$满足:
	\begin{equation*}
		\varphi=\varphi_{cn}'+\varphi_{sn}',\;\varphi_{cn}'\ll\mu,\;\varphi_{sn}'\perp\mu
	\end{equation*}
	将$\varphi_{cn},\varphi_{sn}$延拓到$(X,\mathscr{F})$上:
	\begin{equation*}
		\forall\;A\in\mathscr{F},\;
		\varphi_{cn}(A)=
		\begin{cases}
			\varphi_{cn}'(A),&A\in A_n\cap\mathscr{F} \\
			0,&A\notin A_n\cap \mathscr{F}
		\end{cases},\;
		\varphi_{sn}(A)=
		\begin{cases}
			\varphi_{sn}'(A),&A\in A_n\cap\mathscr{F} \\
			0,&A\notin A_n\cap \mathscr{F}
		\end{cases}
	\end{equation*}
	此时有$\varphi_{cn}\ll\mu,\;\varphi_{sn}\perp\mu$。对任意的$A\in\mathscr{F}$,定义:
	\begin{equation*}
		\varphi_c=\sum_{n=1}^{+\infty}\varphi_{cn},\;\varphi_s=\sum_{n=1}^{+\infty}\varphi_{sn}
	\end{equation*}
	则$\varphi=\varphi_c+\varphi_s$,由定义可以验证得到$\varphi_c,\varphi_s$是$(X,\mathscr{F})$上的$\sigma$有限测度。\par
	若$A\in\mathscr{F}$使得$\mu(A)=0$,则:
	\begin{equation*}
		\varphi_c(A)=\sum_{n=1}^{+\infty}\varphi_{cn}(A)=0
	\end{equation*}
	由$A$的任意性可得$\varphi_c\ll\mu$。\par
	因为$\varphi_{sn}\perp\mu$,取$N_n\in A_n\cap\mathscr{F}$满足:
	\begin{equation*}
		\mu(N_n)=\varphi_{sn}(A_n\setminus N_n)=0
	\end{equation*}
	令$N=\underset{n=1}{\overset{+\infty}{\cup}}N_n\in\mathscr{F}$,由测度的非负性、\cref{prop:Measure}(3)(次可列可加性)、\cref{prop:SigmaField}(2)、\cref{prop:Measure}(1)、\cref{prop:Series}(4)、\cref{prop:SetOperation}(6)和\cref{prop:Measure}(3)(单调性)可得:
	\begin{gather*}
		0\leqslant\mu(N)\leqslant\sum_{n=1}^{+\infty}\mu(N_n)=0 \\
		\begin{aligned}
			&0\leqslant\varphi_s(N^c)=\sum_{n=1}^{+\infty}\varphi_{sn}(N^c)=\sum_{n=1}^{+\infty}\varphi_{sn}(N^c\cap A_n) \\
			=&\sum_{n=1}^{+\infty}\varphi_{sn}(A_n\setminus N)\leqslant\sum_{n=1}^{+\infty}\varphi_{sn}(A_n\setminus N_n)=0
		\end{aligned}
	\end{gather*}
	所以$\mu(N)=\varphi_s(N^c)=0$,即$\varphi_s\perp\mu$。
\end{proof}
\begin{lemma}\label{lem:LebesgueDecomposition3}
	设$\varphi,\mu$分别是可测空间$(X,\mathscr{F})$上的$\sigma$有限符号测度和$\sigma$有限测度,则存在两个$(X,\mathscr{F})$上的$\sigma$有限符号测度$\varphi_c,\varphi_s$使得:
	\begin{equation*}
		\varphi=\varphi_c+\varphi_s,\;\varphi_c\ll\mu,\;\varphi_s\perp\mu
	\end{equation*}
	并且这样的分解是唯一的。
\end{lemma}
\begin{proof}
	\textbf{存在性:}取$\varphi$的Jordan分解$\varphi^{\pm}$。因为$\varphi$是$(X,\mathscr{F})$上的$\sigma$有限符号测度,根据\cref{theo:JordanDecomposition}可知$\varphi^-$是有限测度,所以$\varphi^+$是$\sigma$有限测度。由\cref{lem:LebesgueDecomposition2}可知存在$(X,\mathscr{F})$上的$\sigma$有限测度$\varphi_c^{\pm},\varphi_s^{\pm}$使得:
	\begin{equation*}
		\varphi^{\pm}=\varphi_c^{\pm}+\varphi_s^{\pm},\;\varphi_c^{\pm}\ll\mu,\;\varphi_s^{\pm}\perp\mu
	\end{equation*}
	记:
	\begin{equation*}
		\varphi_c=\varphi_c^+-\varphi_c^-,\quad\varphi_s=\varphi_s^+-\varphi_s^-
	\end{equation*}
	则$\varphi=\varphi_c+\varphi_s$。由\cref{prop:Measure}(3)(单调性)和\cref{prop:SigmaField}(2)可知$\varphi_c$和$\varphi_s$是$\sigma$有限符号测度。\par
	任取$A\in\mathscr{F}$使得$\mu(A)=0$,因为$\varphi_c^{\pm}\ll\mu$,所以:
	\begin{equation*}
		\varphi_c(A)=\varphi_c^+(A)-\varphi_c^-(A)=0
	\end{equation*}
	由$A$的任意性可得$\varphi_c\ll\mu$。\par
	取$N^{\pm}\in\mathscr{F}$满足$\mu(N^{\pm})=\varphi_s^{\pm}[(N^{\pm})^c]=0$,由\cref{prop:SigmaField}(3)可知$N=N^+\cup N^-\in\mathscr{F}$,根据测度的非负性、\cref{prop:Measure}(3)(次有限可加性)、\cref{prop:SetOperation}(7)和\cref{prop:Measure}(3)(单调性)可得:
	\begin{gather*}
		0\leqslant\mu(N)\leqslant\mu(N^+)+\mu(N^-)=0 \\
		0\leqslant\varphi_s^{\pm}(N^c)=\varphi_s^{\pm}[(N^+\cup N^-)^c]=\varphi_s^{\pm}[(N^+)^c\cap (N^-)^c]\leqslant\varphi_s^{\pm}[(N^{\pm})^c]=0 \\
		0\leqslant\varphi_s(N^c)=\varphi_s^+(N^c)-\varphi_s^-(N^c)=0
	\end{gather*}
	所以$\mu(N)=\varphi_s(N^c)=0$,即$\varphi_s\perp\mu$。\par
	\textbf{唯一性:}对$i=1,2$,设$\sigma$有限符号测度$\varphi_{ci},\varphi_{si}$满足:
	\begin{equation*}
		\varphi=\varphi_{ci}+\varphi_{si},\;\varphi_{ci}\ll\mu,\;\varphi_{si}\perp\mu
	\end{equation*}
	根据\cref{prop:SingularSignedMeasure}(1),取$N_i$使得:
	\begin{equation*}
		\mu(N_i)=0,\quad\forall\;A\in\mathscr{F},\;\varphi_{si}(A\cap N_i^c)=0
	\end{equation*}
	令$N=N_1\cup N2$,则由测度的非负性和\cref{prop:Measure}(3)(次有限可加性,单调性)可得$\mu(N)=0$且对任意的$A\in\mathscr{F}$有$\mu(A\cap N)=0$。因为$\varphi_{ci}\ll\mu$,所以$\varphi_{ci}(A\cap N)=0$。由\cref{prop:SetOperation}(7)和\cref{prop:SigmaField}(2)可知:
	\begin{gather*}
		\varphi_{s1}(A\cap N^c)=\varphi_{s1}[A\cap(N_1\cup N_2)^c]=\varphi_{s1}(A\cap N_2^c\cap N_1^c)=0 \\
		\varphi_{s2}(A\cap N^c)=\varphi_{s2}[A\cap(N_1^\cup N_2)^c]=\varphi_{s2}(A\cap N_1^c\cap N_2^c)=0 \\
	\end{gather*}
	所以由\cref{prop:Measure}(1):
	\begin{align*}
		&\varphi_{c1}(A)=\varphi_{c1}(A\cap N)+\varphi_{c1}(A\cap N^c)=\varphi_{c1}(A\cap N^c)=\varphi_{c1}(A\cap N^c)+\varphi_{s1}(A\cap N^c) \\
		=&\varphi(A\cap N^c)=\varphi_{c2}(A\cap N^c)+\varphi_{s2}(A\cap N^c)=\varphi_{c2}(A\cap N^c)=\varphi_{c2}(A\cap N^c)+\varphi_{c2}(A\cap N) \\
		=&\varphi_{c2}(A)
	\end{align*}
	同理可得$\varphi_{s1}(A)=\varphi_{s2}(A)$。由$A$的任意性可知$\varphi_{c1}=\varphi_{c2},\;\varphi_{s1}=\varphi_{s2}$,唯一性得证。
\end{proof}
\begin{theorem}[Lebesgue Decomposition]
	\label{theo:LebesgueDecomposition}
	设$\varphi$和$\psi$是可测空间$(X,\mathscr{F})$上的$\sigma$有限符号测度,则存在两个$(X,\mathscr{F})$上的$\sigma$有限符号测度$\varphi_c,\varphi_s$使得:
	\begin{equation*}
		\varphi=\varphi_c+\varphi_s,\;\varphi_c\ll\psi,\;\varphi_s\perp\psi
	\end{equation*}
	并且这样的分解是唯一的。
\end{theorem}
\begin{proof}
	因为$\psi$是$\sigma$有限符号测度,由$|\psi|=\psi^++\psi^-$和测度的定义可以验证得到$|\psi|$是$\sigma$有限测度,根据\cref{lem:LebesgueDecomposition3}可知存在$(X,\mathscr{F})$上的$\sigma$有限符号测度$\varphi_c,\varphi_s$使得:
	\begin{equation*}
		\varphi=\varphi_c+\varphi_s,\;\varphi_c\ll|\psi|,\;\varphi_s\perp|\psi|
	\end{equation*}
	并且这样的分解是唯一的。显然有$\varphi_c\ll\psi,\;\varphi_s\perp\psi$。
\end{proof}