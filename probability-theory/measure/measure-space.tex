\section{测度空间}
\subsection{集函数与测度}
\begin{definition}
	设$\mathscr{A}$是$X$上的一个集族。称定义在$\mathscr{A}$上并且取值非负的函数为非负集函数。
\end{definition}
\begin{definition}
	设$\mathscr{A}$是$X$上的集族,$\mu$是定义在其上的非负集函数。
	\begin{enumerate}
		\item 如果对$\mathscr{A}$中任意互不相交的$A_1,A_2,\dots,A_n$,有:
		\begin{equation*}
			\mu\left(\underset{i=1}{\overset{n}{\cup}}A_i\right)=\sum_{i=1}^{n}\mu(A_i)
		\end{equation*}
		则称$\mu$具有\gls{FiniteAdditivity};
		\item 如果对$\mathscr{A}$中任意的$A_1,A_2,\dots,A_n$,有:
		\begin{equation*}
			\mu\left(\underset{i=1}{\overset{n}{\cup}}A_i\right)\leqslant\sum_{i=1}^{n}\mu(A_i)
		\end{equation*}
		则称$\mu$具有\gls{FiniteSubadditivity};
		\item 如果对任意的$A,B\in\mathscr{A},\;A\subset B,\;B\backslash A\in\mathscr{A}$,只要$\mu(A)<+\infty$,就有:
		\begin{equation*}
			\mu(B\backslash A)=\mu(B)-\mu(A)
		\end{equation*}
		则称$\mu$具有\gls{Reducibility};
		\item 若对任意的$A,B\in\mathscr{A},\;A\subset B$,有$\mu(A)\leqslant\mu(B)$,则称$\mu$具有\gls{Monotonicity};
		\item 若对$\mathscr{A}$中任意互不相交的集合序列$\{A_n\}$,只要$\underset{n=1}{\overset{+\infty}{\cup}}A_n\in\mathscr{A}$,就有:
		\begin{equation*}
			\mu\left(\underset{n=1}{\overset{+\infty}{\cup}}A_n\right)=\sum_{n=1}^{+\infty}\mu(A_n)
		\end{equation*}
		则称$\mu$具有\gls{CountablyAdditivity};
		\item 若对$\mathscr{A}$中任意的集合序列$\{A_n\}$,只要$\underset{n=1}{\overset{+\infty}{\cup}}A_n\in\mathscr{A}$,就有:
		\begin{equation*}
			\mu\left(\underset{n=1}{\overset{+\infty}{\cup}}A_n\right)\leqslant\sum_{n=1}^{+\infty}\mu(A_n)
		\end{equation*}
		则称$\mu$具有\gls{CountablySubadditivity};
		\item 若对任意的$\{A_n\}\subset\mathscr{A}$且$A_n\uparrow A\in\mathscr{A}$,有:
		\begin{equation*}
			\mu\left(A\right)=\lim_{n\to+\infty}\mu(A_n)
		\end{equation*}
		则称$\mu$具有\gls{ContinuityFromBelow};
		\item 若对任意的$\{A_n\}\subset\mathscr{A}$且$A_n\downarrow A,\;\mu(A_1)<+\infty$,有:
		\begin{equation*}
			\mu\left(A\right)=\lim_{n\to+\infty}\mu(A_n)
		\end{equation*}
		则称$\mu$具有\gls{ContinuityFromAbove}。
	\end{enumerate}
\end{definition}
\begin{definition}
	设$\mathscr{A}$是$X$上的集族,$\varnothing\in\mathscr{A}$。如果$\mathscr{A}$上的非负集函数$\mu$满足:
	\begin{enumerate}
		\item $\mu(\varnothing)=0$;
		\item $\mu$具有可列可加性。
	\end{enumerate}
	则称$\mu$为$\mathscr{A}$上的\gls{Measure}。如果对任意的$A\in\mathscr{A}$有$\mu(A)<+\infty$,则称测度$\mu$是有限的;如果对任意的$A\in\mathscr{A}$,存在$\mathscr{A}$中的集合序列$\{A_n\}$,满足$\mu(A_n)<+\infty,\;\forall\;n\in\mathbb{N}^+$,使得$A\subset\underset{n=1}{\overset{+\infty}{\cup}}A_n$,则称测度$\mu$是$\sigma$有限的。
\end{definition}
\begin{theorem}\label{theo:MeasureFiniteAdditivityReducibility}
	测度具有有限可加性与可减性。
\end{theorem}
\begin{proof}
	设$\mu$是$\mathscr{A}$上的测度。\par
	(1)任取互不相交的$A_1,A_2,\dots,A_n\in\mathscr{A}$,由$\mu$的可列可加性可得:
	\begin{equation*}
		\mu\left(\underset{i=1}{\overset{n}{\cup}}A_i\right)=\mu(A_1\cup A_2\cdots\cup A_n\cup\varnothing\cdots)=\sum_{i=1}^{n}\mu(A_i)+0=\sum_{i=1}^{n}\mu(A_i)
	\end{equation*}
	由$A_1,A_2,\dots,A_n$的任意性,$\mu$具有有限可加性。\par
	(2)任取$A,B\in\mathscr{A},\;A\subset B,\;B\backslash A\in\mathscr{A},\;\mu(A)<+\infty$,显然$(B\backslash A)\cap A=\varnothing$,由测度的有限可加性:
	\begin{equation*}
		\mu(B)=\mu[(B\backslash A)\cup A]=\mu(B\backslash A)+\mu(A)
	\end{equation*}
	即:
	\begin{equation*}
		\mu(B\backslash A)=\mu(B)-\mu(A)
	\end{equation*}
	于是$\mu$具有可减性。
\end{proof}
\begin{definition}
	设$X$是一个集合,$\mathscr{F}$是$X$的一些子集生成的$\sigma$域,$\mu$是$\mathscr{F}$上的测度。称$(X,\mathscr{F},\mu)$为\gls{MeasureSpace}。若$A\in \mathscr{F}$且$\mu(A)=0$,则称$A$为\gls{NullSet}。若$\mathscr{F}$中零测集的子集还属于$\mathscr{A}$,则称测度空间$(X,\mathscr{F},\mu)$是\gls{CompleteMeasureSpace}。若测度空间$(X,\mathscr{F},P)$满足$P(X)=1$,则称其为\gls{ProbabilitySpace},对应的$P$叫做\textbf{概率测度},$\mathscr{F}$中的元素叫做\gls{Event},$P(A)$叫做事件$A$发生的概率。
\end{definition}
\subsubsection{半环上的测度}
\begin{theorem}\label{theo:SemiringFiniteAdditivitySetfunction}
	半环$\mathscr{A}$上有有限可加性的非负集函数$\mu$必有单调性和可减性。
\end{theorem}
\begin{proof}
	(1)设$A,B\in\mathscr{A},\;A\subset B$。因为$\mathscr{A}$是一个半环,所以存在互不相交的$\seq{C}{n}\in \mathscr{A}$使得:
	\begin{equation*}
		B\backslash A=\underset{i=1}{\overset{n}{\cup}}C_i
	\end{equation*}
	由$\mu$的有限可加性可得:
	\begin{equation*}
		\mu(B)=\mu\left[A\cup\left(\underset{i=1}{\overset{n}{\cup}}C_i\right)\right]=\mu(A)+\sum_{i=1}^{n}\mu(C_i)\geqslant\mu(A)
	\end{equation*}
	所以$\mu$有单调性。\par
	(2)由(1)可得:
	\begin{equation*}
		\mu(B\backslash A)=\mu\left(\underset{i=1}{\overset{n}{\cup}}C_i\right)=\sum_{i=1}^{n}\mu(C_i)=\mu(B)-\mu(A)
	\end{equation*}
	于是$\mu$有可减性。
\end{proof}
\begin{theorem}\label{theo:SemiringCountableAdditivitySetFunction}
	半环$\mathscr{A}$上有可列可加性的非负集函数$\mu$具有次可列可加性、下连续性和上连续性。
\end{theorem}
\begin{proof}
	因为$\mu$具有可列可加性,所以:
	\begin{equation*}
		\mu(\varnothing)=\sum_{n=1}^{+\infty}\mu(\varnothing)
	\end{equation*}
	于是$\mu(\varnothing)=0$或$\mu(\varnothing)=+\infty$。\par
	\textbf{(1)$\;\mu(\varnothing)=0$:}此时$\mu$是$\mathscr{A}$上的测度。\par
	\textbf{下连续性:}设$\{A_n\}$是$\mathscr{A}$中一个单调递增的集合序列且有$A_n\uparrow A$,$A_0=\varnothing$。由$\mu$的可列可加性和有限可加性(\cref{theo:MeasureFiniteAdditivityReducibility})可得:
	\begin{align*}
		\mu(A)&=\mu\left(\underset{n=1}{\overset{+\infty}{\cup}}A_n\right)
		=\mu\left[\underset{n=1}{\overset{+\infty}{\cup}}(A_{n}\backslash A_{n-1})\right] =\sum_{n=1}^{+\infty}\mu(A_n\backslash A_{n-1}) \\
		&=\lim_{n\to+\infty}\left[\sum_{i=1}^{n}\mu(A_i\backslash A_{i-1})\right]
		=\lim_{n\to+\infty}\mu(A_n)
	\end{align*}
	下连续性得证。\par
	\textbf{上连续性:}设$\{A_n\}$是$\mathscr{A}$中一个单调递减的集合序列且有$A_n\downarrow A$,$\mu(A_1)<+\infty$。由$\mu$的可列可加性可得:
	\begin{align*}
		\mu(A_n)&=\mu\left\{A\cup\left[\underset{i=n}{\overset{+\infty}{\cup}}(A_i\backslash A_{i+1})\right]\right\}
		=\mu(A)+\sum_{i=n}^{+\infty}\mu(A_i\backslash A_{i+1})
	\end{align*}
	因为$\mathscr{A}$是一个半环且$\{A_n\}$是$\mathscr{A}$中的集合序列,所以存在互不相交的$\seq{C}{k_n}\in \mathscr{A}$使得:
	\begin{equation*}
		A_n\backslash A_{n+1}=\underset{i=1}{\overset{k_n}{\cup}}C_i
	\end{equation*}
	由$\mu$的有限可加性可得:
	\begin{equation*}
		\mu(A_n)=\mu(A)+\sum_{i=n}^{+\infty}\mu(A_i\backslash A_{i+1})=\mu(A)+\sum_{i=n}^{+\infty}\mu\left(\underset{j=1}{\overset{k_i}{\cup}}C_j\right)=\mu(A)+\sum_{i=n}^{+\infty}\sum_{j=1}^{k_i}\mu(C_j)
	\end{equation*}
	注意到:
	\begin{equation*}
		\mu(A_1)=\mu(A)+\sum_{i=1}^{+\infty}\sum_{j=1}^{k_i}\mu(C_j)<+\infty
	\end{equation*}
	所以级数:
	\begin{equation*}
		\sum_{i=1}^{+\infty}\sum_{j=1}^{k_i}\mu(C_j)
	\end{equation*}
	收敛。由级数收敛的必要性条件\info{级数收敛的必要条件}可得:
	\begin{equation*}
		\lim_{n\to+\infty}\left[\sum_{i=n}^{+\infty}\sum_{j=1}^{k_i}\mu(C_j)\right]=0
	\end{equation*}
	于是有:
	\begin{equation*}
		\lim_{n\to+\infty}\mu(A_n)=\mu(A)+\lim_{n\to+\infty}\left[\sum_{i=n}^{+\infty}\sum_{j=1}^{k_i}\mu(C_j)\right]=\mu(A)
	\end{equation*}
	上连续性得证。\par
	\textbf{次可列可加性:}设$\{A_n\}$是$\mathscr{A}$中的一个集合序列,由生成的定义可知$\{A_n\}$也是$r(\mathscr{A})$中的一个集合序列,令$A_0=\varnothing$。由环的定义可得:
	\begin{equation*}
		\underset{i=1}{\overset{n-1}{\cup}}A_i\in r(\mathscr{A}),\;A_n\backslash\underset{i=1}{\overset{n-1}{\cup}}A_i\in\mathscr{A}
	\end{equation*}
	因为环也是半环(\cref{theo:SetNecessarilySet1}),再根据\cref{theo:RingGeneratedBySemiring}可得存在互不相交的$C_{n1},C_{n2},\dots,C_{nk_n}\in\mathscr{A}$使得:
	\begin{equation*}
		A_n\backslash\underset{i=1}{\overset{n-1}{\cup}}A_i=\underset{i=1}{\overset{k_n}{\cup}}C_{ni}
	\end{equation*}
	同理,存在互不相交的$D_{n1},D_{n2},\dots,D_{nl_n}\in\mathscr{A}$使得:
	\begin{equation*}
		A_n\backslash\underset{i=1}{\overset{k_n}{\cup}}C_{ni}=\underset{i=1}{\overset{l_n}{\cup}}D_{ni}
	\end{equation*}
	显然$C_{n1},C_{n2},\dots,C_{nk_n},D_{n1},D_{n2},\dots,D_{nl_n}$互不相交,同时有:
	\begin{equation*}
		A_n=\left(\underset{i=1}{\overset{k_n}{\cup}}C_{ni}\right)\bigcup\left(\underset{i=1}{\overset{l_n}{\cup}}D_{ni}\right)
	\end{equation*}
	由$\mu$的可列可加性与有限可加性(\cref{theo:MeasureFiniteAdditivityReducibility}):
	\begin{align*}
		\mu\left(\underset{n=1}{\overset{+\infty}{\cup}}A_n\right)
		&=\mu\left(\underset{n=1}{\overset{+\infty}{\cup}}\underset{i=1}{\overset{k_n}{\cup}}C_{ni}\right) =\sum_{n=1}^{+\infty}\sum_{i=1}^{k_n}\mu(C_{ni}) \\
		&\leqslant\sum_{n=1}^{+\infty}\sum_{i=1}^{k_n}\mu(C_{ni})+\sum_{n=1}^{+\infty}\sum_{i=1}^{l_n}\mu(D_{ni}) \\
		&=\sum_{n=1}^{+\infty}\left[\mu\left(\underset{i=1}{\overset{k_n}{\cup}}C_{ni}\right)+\mu\left(\underset{i=1}{\overset{l_n}{\cup}}D_{ni}\right)\right] \\
		&=\sum_{n=1}^{+\infty}\mu\left[\left(\underset{i=1}{\overset{k_n}{\cup}}C_{ni}\right)\bigcup\left(\underset{i=1}{\overset{l_n}{\cup}}D_{ni}\right)\right]=\sum_{n=1}^{+\infty}\mu(A_n)
	\end{align*}
	次可列可加性得证。\par
	\textbf{(2)$\;\mu(\varnothing)=+\infty$:}此时显然满足所有条件。
\end{proof}
\begin{theorem}\label{theo:MeasureOfSemiring}
	半环上的测度具有单调性、可减性、次可列可加性、下连续性和上连续性。
\end{theorem}
\begin{proof}
	测度具有非负性和可列可加性,由\cref{theo:SemiringCountableAdditivitySetFunction}可知半环上的测度具有次可列可加性、下连续性和上连续性。由\cref{theo:MeasureFiniteAdditivityReducibility}可知可列可加性蕴含有限可加性,所以根据\cref{theo:SemiringFiniteAdditivitySetfunction}可知半环上的测度具有单调性和可减性。
\end{proof}
\begin{theorem}
	设$\mu$是环$\mathscr{A}$上的非负集函数,则:
	\begin{align*}
		&(1)\mu\text{可列可加} \\
		\Leftrightarrow&(2)\mu\text{次可列可加且有限可加} \\
		\Leftrightarrow&(3)\mu\text{下连续且有限可加} \\
		\Rightarrow&(4)\mu\text{上连续}
	\end{align*}
\end{theorem}
\begin{proof}
	由\cref{theo:SetNecessarilySet1}可知环是半环,所以根据\cref{theo:MeasureOfSemiring}可得三个必要性成立,下分别证明两个充要性。\par
	(1)设$\{A_n\}$是$\mathscr{A}$中的一个互不相交的集合序列。由$\mu$的次可列可加性可得:
	\begin{equation*}
		\mu\left(\underset{n=1}{\overset{+\infty}{\cup}}A_n\right)\leqslant
		\sum_{n=1}^{+\infty}\mu(A_n)
	\end{equation*}
	由\cref{theo:SemiringFiniteAdditivitySetfunction}可知$\mu$具有单调性,因为$\mu$具有有限可加性,所以对任意的$m\in\mathbb{N}^+$有:
	\begin{equation*}
		\mu\left(\underset{n=1}{\overset{+\infty}{\cup}}A_n\right)\geqslant\mu\left(\underset{n=1}{\overset{m}{\cup}}A_n\right)=\sum_{n=1}^{m}\mu(A_n)
	\end{equation*}
	于是由极限的不等式性可得:
	\begin{equation*}
		\mu\left(\underset{n=1}{\overset{+\infty}{\cup}}A_n\right)\geqslant
		\sum_{n=1}^{+\infty}\mu(A_n)
	\end{equation*}
	所以有:
	\begin{equation*}
		\mu\left(\underset{n=1}{\overset{+\infty}{\cup}}A_n\right)=\sum_{n=1}^{+\infty}\mu(A_n)
	\end{equation*}
	即$\mu$可列可加。\par
	(2)设$\{A_n\}$是$\mathscr{A}$中的一个互不相交的集合序列,显然有:
	\begin{equation*}
		\underset{i=1}{\overset{n}{\cup}}A_i\uparrow\underset{n=1}{\overset{+\infty}{\cup}}A_n
	\end{equation*}
	由$\mu$的有限可加性与下连续性可得:
	\begin{align*}
		\mu\left(\underset{n=1}{\overset{+\infty}{\cup}}A_n\right)
		&=\lim_{n\to+\infty}\left[\mu\left(\underset{i=1}{\overset{n}{\cup}}A_i\right)\right] =\lim_{n\to+\infty}\left[\sum_{i=1}^{n}\mu(A_i)\right]=\sum_{n=1}^{+\infty}\mu(A_n)
	\end{align*}
	即$\mu$可列可加。
\end{proof}

\subsection{外测度}
\begin{definition}
	设$X$是一个集合,$\tau$是$X$的所有子集构成的集族$\mathscr{A}$到$\overline{\mathbb{R}}$上的函数,如果:
	\begin{enumerate}
		\item $\tau(\varnothing)=0$;
		\item 若$A\subset B$且$A,B\in\mathscr{A}$,则有$\tau(A)\leqslant\tau(B)$;
		\item $\tau$具有次可列可加性。
	\end{enumerate}
	则称$\tau$为$X$上的\gls{ExteriorMeasure}。
\end{definition}
\begin{theorem}\label{theo:SubadditivityExteriorMeasure}
	外测度具有次有限可加性。
\end{theorem}
\begin{proof}
	设$X$是一个集合,$\mathscr{A}$是$X$的所有子集构成的集族,$\tau$是$X$上的外测度。任取$A_1,A_2,\dots,A_n\in\mathscr{A}$,由$\tau$的次可列可加性可得:
	\begin{equation*}
		\tau\left(\underset{i=1}{\overset{n}{\cup}}A_i\right)=\tau(A_1\cup A_2\cdots\cup A_n\cup\varnothing\cdots)\leqslant\sum_{i=1}^{n}\tau(A_i)+0=\sum_{i=1}^{n}\tau(A_i)
	\end{equation*}
	由$A_1,A_2,\dots,A_n$的任意性,$\tau$具有次有限可加性。
\end{proof}
\begin{theorem}
	设$X$是一个集合,$\mathscr{A}$是$X$的所有子集构成的集族,则$\mathscr{A}$上的测度$\tau$一定是$X$上的外测度。
\end{theorem}
\begin{proof}
	显然$\mathscr{A}$是一个$\sigma$域。由\cref{theo:SigmaFieldIsField}可知$\mathscr{A}$是一个域,根据\cref{theo:SetNecessarilySet1}可知$\mathscr{A}$是一个半环,所以$\tau$是半环上的测度。由\cref{theo:MeasureOfSemiring}可知$\tau$具有单调性和次可列可加性。因为$\tau$是一个测度,所以$\tau(\varnothing)=0$。综上,$\tau$是$X$上的外测度。
\end{proof}
\begin{theorem}
	设$\mathscr{A}$是一个包含$\varnothing$的集族,$\mu$是$\mathscr{A}$上的一个非负集函数且满足$\mu(\varnothing)=0$,若对于任意的$A\in\mathscr{A}$有:
	\begin{equation*}
		\tau(A)=\inf\left\{\sum_{n=1}^{+\infty}\mu(B_n):B_n\in\mathscr{A},\;A\subset\underset{n=1}{\overset{+\infty}{\cup}}B_n\right\}
	\end{equation*}
	则$\tau$是一个外测度,称$\tau$为\textbf{由$\mu$生成的外测度}。
\end{theorem}
\begin{proof}
	(1)因为$\varnothing=\underset{n=1}{\overset{+\infty}{\cup}}\varnothing$,所以:
	\begin{equation*}
		0\leqslant\tau(\varnothing)\leqslant\sum_{n=1}^{+\infty}\mu(\varnothing)=0
	\end{equation*}
	所以$\tau(\varnothing)=0$。\par
	(2)设$A\subset B$且$A,B\in\mathscr{A}$,对于满足条件:
	\begin{equation*}
		B\subset\underset{n=1}{\overset{+\infty}{\cup}}B_n
	\end{equation*}
	的$\{B_n\}$,自然有:
	\begin{equation*}
		A\subset\underset{n=1}{\overset{+\infty}{\cup}}B_n
	\end{equation*}
	所以:
	\begin{equation*}
		\tau(A)\leqslant\sum_{n=1}^{+\infty}\mu(B_n)
	\end{equation*}
	对右边取下确界即有$\tau(A)\leqslant\tau(B)$。\par
	(3)设$\{A_n\}$是$\mathscr{A}$中的一个集合序列。若存在$n_0\in\mathbb{N}^+$使得$\tau(A_{n_0})=+\infty$,则由$\tau$的定义和(2)可得:
	\begin{equation*}
		\tau\left(\underset{n=1}{\overset{+\infty}{\cup}}A_n\right)\leqslant+\infty=\tau(A_{n_0})\leqslant\sum_{n=1}^{+\infty}\tau(A_n)
	\end{equation*}
	即$\tau$具有次可列可加性。\par
	若$\tau(A_n)<+\infty$对$n\in\mathbb{N}^+$都成立,任取$\varepsilon>0$,对任意的$n\in\mathbb{N}^+$,存在$\mathscr{A}$中的一个集合序列$\{B_{ni}\}$使得:
	\begin{equation*}
		A_n\subset\underset{i=1}{\overset{+\infty}{\cup}}B_{ni},\;
		\sum_{i=1}^{+\infty}\mu(B_{ni})<\tau(A_n)+\frac{\varepsilon}{2^n}
	\end{equation*}
	于是:
	\begin{equation*}
		\sum_{n=1}^{+\infty}\sum_{i=1}^{+\infty}\mu(B_{ni})<\sum_{n=1}^{+\infty}\tau(A_n)+\varepsilon
	\end{equation*}
	由$\{B_{n_i}\}$的取法,显然:
	\begin{equation*}
		\tau\left(\underset{n=1}{\overset{+\infty}{\cup}}A_n\right)\leqslant\sum_{n=1}^{+\infty}\sum_{i=1}^{+\infty}\mu(B_{ni})
	\end{equation*}
	由$\varepsilon$的任意性可得:
	\begin{equation*}
		\tau\left(\underset{n=1}{\overset{+\infty}{\cup}}A_n\right)\leqslant\sum_{n=1}^{+\infty}\tau(A_n)
	\end{equation*}
	即$\tau$具有次可列可加性。\par
	综上,$\tau$是一个外测度。
\end{proof}

\begin{definition}[Caratheodory condition]
	设$X$是一个集合,$\mathscr{A}$是$X$的所有子集构成的集族,$\tau$是$X$上的外测度。称满足条件:
	\begin{equation*}
		\tau(T)=\tau(T\cap A)+\tau(T\cap A^c),\;\forall\;T\in\mathscr{A}
	\end{equation*}
	的集合$A\in\mathscr{A}$为$\tau$\gls{MeasurableSet}。将由所有$\tau$可测集构成的集族记作$\mathscr{A}_{\tau}$。
\end{definition}
\begin{lemma}\label{lem:EmeasureAB}
	设$X$是一个集合,$\mathscr{A}$是$X$的所有子集构成的集族,$\tau$是$X$上的外测度。集合$E\in \mathscr{A}_{\tau}$的充要条件是对与$\forall\;A\subset E,\;\forall\;B\subset E^c$,总有:
	\begin{equation*}
		\tau(A\cup B)=\tau(A)+\tau(B)
	\end{equation*}
\end{lemma}
\begin{proof}
	必要性:对任意的$A\subset E,\;\forall\;B\subset E^c$,取$T=A\cup B$,因为$E\in \mathscr{A}_{\tau}$,那么对于这个$T$,应有:
	\begin{equation}
		\tau(A\cup B)=\tau(T)=\tau(T\cap E)+\tau(T\cap E^c)=\tau(A)+\tau(B)\notag
	\end{equation}
	充分性:对任意的$T\in \mathscr{A}$,$\exists\;A\subset E,\;\exists\; B\subset E^c$,使得$T=A\cup B$,那么就有:
	\begin{equation}
		\tau(T)=\tau(A\cup B)=\tau(A)+\tau(B)=\tau(T\cap E)+\tau(T\cap E^c)\notag
	\end{equation}
	由$T$的任意性,$E\in \mathscr{A}_{\tau}$。
\end{proof}
\begin{property}\label{prop:tauMeasurableSetCollection}
	设$X$是一个集合,$\mathscr{A}$是$X$的所有子集构成的集族,$\tau$是$X$上的外测度。$\mathscr{A}_{\tau}$具有如下性质:
	\begin{enumerate}
		\item $\varnothing,X\in \mathscr{A}_{\tau}$; 
		\item $S\in \mathscr{A}_{\tau}$的充要条件是$S^c\in \mathscr{A}_{\tau}$;
		\item 设$\seq{S}{n}$是$\tau$可测集,则有$\underset{i=1}{\overset{n}{\cup}}S_i\in \mathscr{A}_{\tau}$,并且当$\seq{S}{n}$互不相交时可得:
		\begin{equation*}
			\tau\left(\underset{i=1}{\overset{n}{\cup}}S_i\right)=\sum_{i=1}^{n}\tau(S_i)
		\end{equation*}
		\item 设$\seq{S}{n}$是$\tau$可测集,则$\underset{i=1}{\overset{n}{\cap}}S_i\in \mathscr{A}_{\tau}$;
		\item 设$S_1,S_2\in \mathscr{A}_{\tau}$,则$S_1\backslash S_2\in \mathscr{A}_{\tau}$;
		\item 设$\{S_n\}$是一列$\tau$可测集,则$\underset{n=1}{\overset{+\infty}{\cup}}S_n\in \mathscr{A}_{\tau}$,并且当$\{S_n\}$互不相交时可得:
		\begin{equation*}
			\tau\left(\underset{n=1}{\overset{+\infty}{\cup}}S_i\right)=\sum_{n=1}^{+\infty}\tau(S_i)
		\end{equation*}
		\item 设$\{S_n\}$是一列$\tau$可测集,则$\underset{n=1}{\overset{+\infty}{\cap}}S_n\in\mathscr{A}_{\tau}$。
	\end{enumerate}
\end{property}
\begin{proof}
	(1)代入定义直接可得。\par
	(2)若$S\in \mathscr{A}_{\tau}$,对任意的$T\in \mathscr{A}$,则有:
	\begin{align*}
		\tau(T)=\tau(T\cap S)+\tau(T\cap S^c)
		&=\tau[T\cap(S^c)^c]+\tau(T\cap S^c) \\
		&=\tau(T\cap S^c)+\tau[T\cap(S^c)^c]
	\end{align*}\par
	(3)因为$S_1\in \mathscr{A}_{\tau}$,对任意的$T$都有:
	\begin{equation}\label{eq:S_1measure}
		\tau(T)=\tau(T\cap S_1)+\tau(T\cap S_1^c)
	\end{equation}
	因为$S_2\in \mathscr{A}_{\tau}$,对于$\tau(T\cap S_1^c)$有:
	\begin{equation}\label{eq:TcapS_1measure}
		\tau(T\cap S_1^c)=\tau[(T\cap S_1^c)\cap S_2]+\tau[(T\cap S_1^c)\cap S_2^c]
	\end{equation}
	将\eqref{eq:TcapS_1measure}式代入\eqref{eq:S_1measure}式,再由\cref{theo:DeMorganLaw},得到:
	\begin{align*}
		\tau(T)&=\tau(T\cap S_1)+\tau[(T\cap S_1^c)\cap S_2]+\tau[(T\cap S_1^c)\cap S_2^c] \\
		&=\tau(T\cap S_1)+\tau[(T\cap S_1^c)\cap S_2]+\tau[T\cap(S_1\cup S_2)^c]
	\end{align*}
	由于$T\cap S_1\subset S_1$,$(T\cap S_1^c)\cap S_2\subset S_1^c$,满足\cref{lem:EmeasureAB}条件,因此上式的前两项可以合并:
	\begin{align*}
		\tau(T\cap S_1)+\tau[(T\cap S_1^c)\cap S_2]&=\tau[(T\cap S_1)\cup(T\cap S_1^c\cap S_2)] \\
		&=\tau\{T\cap[S_1\cup(S_1^c\cap S_2)]\} \\
		&=\tau\{T\cap[(S_1\cup S_1^c)\cap(S_1\cup S_2)]\} \\
		&=\tau[T\cap(S_1\cup S_2)]
	\end{align*}
	那么就有:
	\begin{equation*}
		\tau(T)=\tau[T\cap(S_1\cup S_2)]+\tau[T\cap(S_1\cup S_2)^c]
	\end{equation*}
	由$T$的任意性,$S_1\cup S_2\in \mathscr{A}_{\tau}$。\par
	当$S_1\cap S_2=\varnothing$时,显然$S_2\subset S_1^c$,那么就有$T\cap S_2\subset S_1^c$,由\cref{lem:EmeasureAB}:
	\begin{align*}
		\tau[T\cap(S_1\cup S_2)]&=\tau[(T\cap S_1)\cup(T\cap S_2)]\\
		&=\tau(T\cap S_1)+\tau(T\cap S_2)
	\end{align*}
	由数学归纳法,$\underset{i=1}{\overset{n}{\cup}}S_i\in \mathscr{A}_{\tau}$且当$\seq{S}{n}$互不相交时可得:
	\begin{equation*}
		\tau\left(\underset{i=1}{\overset{n}{\cup}}S_i\right)=\sum_{i=1}^{n}\tau(S_i)
	\end{equation*}\par
	(4)由\cref{theo:DeMorganLaw}以及(2)(3)直接得到:
	\begin{equation*}
		\underset{i=1}{\overset{n}{\cap}}S_i=\left(\underset{i=1}{\overset{n}{\cup}}S_i^c\right)^c\in \mathscr{A}_{\tau}
	\end{equation*}\par
	(5)$\;S_1\backslash S_2=S_1\cap S_2^c$,由(2)(4)可知$S_1\backslash S_2\in \mathscr{A}_{\tau}$。\par
	(6)设$\{S_n\}$互不相交。由(3)可得对任意的$n$,$\underset{i=1}{\overset{n}{\cup}}S_i\in \mathscr{A}_{\tau}$,那么对任意的$T\in \mathscr{A}$,就有(第一行到第二行利用外测度的单调性,第二行到第三行利用(3)):
	\begin{align*}
		\tau(T)&=\tau\left[T\cap\left(\underset{i=1}{\overset{n}{\cup}}S_i\right)\right]+\tau\left[T\cap\left(\underset{i=1}{\overset{n}{\cup}}S_i\right)^c\right] \\
		&\geqslant \tau\left[T\cap\left(\underset{i=1}{\overset{n}{\cup}}S_i\right)\right]+\tau\left[T\cap\left(\underset{n=1}{\overset{+\infty}{\cup}}S_n\right)^c\right] \\
		&=\sum_{i=1}^n\tau(T\cap S_i)+\tau\left[T\cap\left(\underset{n=1}{\overset{+\infty}{\cup}}S_n\right)^c\right]
	\end{align*}
	令$n\to+\infty$,有(第一行到第二行利用极限的不等式性,第二行到第三行利用外测度的次可列可加性):
	\begin{align}
		\tau(T)&\geqslant\sum_{i=1}^n\tau(T\cap S_i)+\tau\left[T\cap\left(\underset{n=1}{\overset{+\infty}{\cup}}S_n\right)^c\right]\notag \\
		&\geqslant\sum_{i=1}^{+\infty} \tau(T\cap S_i)+\tau\left[T\cap\left(\underset{n=1}{\overset{+\infty}{\cup}}S_n\right)^c\right] \label{eq:TcupSi}\\
		&\geqslant \tau\left[\underset{i=1}{\overset{+\infty}{\cup}}(T\cap S_i)\right]+\tau\left[T\cap\left(\underset{n=1}{\overset{+\infty}{\cup}}S_n\right)^c\right] \notag \\ &=\tau\left[T\cap\left(\underset{n=1}{\overset{+\infty}{\cup}}S_n\right)\right]+\tau\left[T\cap\left(\underset{n=1}{\overset{+\infty}{\cup}}S_n\right)^c\right]\notag
	\end{align}
	又因:
	\begin{equation*}
		T=\left[T\cap\left(\underset{n=1}{\overset{+\infty}{\cup}}S_n\right)\right]\cup\left[T\cap\left(\underset{n=1}{\overset{+\infty}{\cup}}S_n\right)^c\right]
	\end{equation*}
	由外测度的次有限可加性(\cref{theo:SubadditivityExteriorMeasure})可得:
	\begin{equation*}
		\tau(T)\leqslant \tau\left[T\cap\left(\underset{n=1}{\overset{+\infty}{\cup}}S_n\right)\right]+\tau\left[T\cap\left(\underset{n=1}{\overset{+\infty}{\cup}}S_n\right)^c\right]
	\end{equation*}
	因此:
	\begin{equation*}
		\tau(T)= \tau\left[T\cap\left(\underset{n=1}{\overset{+\infty}{\cup}}S_n\right)\right]+\tau\left[T\cap\left(\underset{n=1}{\overset{+\infty}{\cup}}S_n\right)^c\right]
	\end{equation*}
	由$T$的任意性,$\underset{n=1}{\overset{+\infty}{\cup}}S_n$可测。\par
	令$T=\underset{n=1}{\overset{+\infty}{\cup}}S_n$,代入\cref{eq:TcupSi}式,则:
	\begin{align*}
		\tau\left(\underset{n=1}{\overset{+\infty}{\cup}}S_n\right)
		&\geqslant\sum_{i=1}^{+\infty} \tau\left[\left(\underset{n=1}{\overset{+\infty}{\cup}}S_n\right)\cap S_i\right]+\tau\left[\left(\underset{n=1}{\overset{+\infty}{\cup}}S_n\right)\cap\left(\underset{n=1}{\overset{+\infty}{\cup}}S_n\right)^c\right] \\
		&=\sum_{i=1}^{+\infty} \tau(S_i)
	\end{align*}
	但是由外测度的次可列可加性有:
	\begin{equation*}
		\tau\left(\underset{n=1}{\overset{+\infty}{\cup}}S_n\right)\leqslant\sum_{n=1}^{+\infty} \tau(S_n)
	\end{equation*}
	因此:
	\begin{equation*}
		\tau\left(\underset{n=1}{\overset{+\infty}{\cup}}S_n\right)=\sum_{n=1}^{+\infty} \tau(S_n)
	\end{equation*}\par
	若$\{S_n\}$不满足互不相交,则$\underset{n=1}{\overset{+\infty}{\cup}}S_n$可被表示为互不相交的可数个集合的并:
	\begin{equation*}
		\underset{n=1}{\overset{+\infty}{\cup}}S_n=S_1\cup (S_2\backslash S_1)\cup[S_3\backslash(S_1\cup S_2)]\cdots
	\end{equation*}
	由(5)和之前$\{S_n\}$互不相交时的论述即可得$\underset{n=1}{\overset{+\infty}{\cup}}S_n\in \mathscr{A}_{\tau}$。\par
	(7)由\cref{theo:DeMorganLaw}和(2)可知:
	\begin{equation*}
		\underset{n=1}{\overset{+\infty}{\cap}}S_n=\left(\underset{n=1}{\overset{+\infty}{\cup}}S_n^c\right)^c\in \mathscr{A}_{\tau}\qedhere
	\end{equation*}
\end{proof}
\begin{theorem}
	设$X$是一个集合,$\mathscr{A}$是$X$的所有子集构成的集族,$\tau$是$X$上的外测度。$(X,\mathscr{A}_{\tau},\tau)$是一个完全测度空间。
\end{theorem}
\begin{proof}
	由\cref{prop:tauMeasurableSetCollection}(6)和外测度的定义可知$\tau$是$\mathscr{A}_{\tau}$上的测度。任取$A\in \mathscr{A}$满足$\tau(A)=0$,对任何的$E\in \mathscr{A}$,由外测度的单调性和非负性可得$\tau(E\cap A)=0$,于是由外测度的单调性可得:
	\begin{equation*}
		\tau(E)\geqslant\tau(E\cap A^c)=\tau(E\cap A)+\tau(E\cap A^c)
	\end{equation*}
	因为$E=(E\cap A)\cup(E\cap A^c)$,由外测度的次可列可加性可得:
	\begin{equation*}
		\tau(E)\leqslant\tau(E\cap A)+\tau(E\cap A^c)
	\end{equation*}
	于是有:
	\begin{equation*}
		\tau(E)=\tau(E\cap A)+\tau(E\cap A^c)
	\end{equation*}
	由$E$的任意性,$A\in \mathscr{A}_{\tau}$,即$\tau$的零测集都属于$\mathscr{A}_{\tau}$。由外测度的单调性,零测集的子集都是零测集,所以$\tau$的零测集的子集都属于$\mathscr{A}_{\tau}$,$(X,\mathscr{A}_{\tau},\tau)$是一个完全测度空间。
\end{proof}

\subsection{测度的扩张}
\begin{definition}
	设$\mu,\tau$分别为集族$\mathscr{A},\mathscr{B}$上的测度,并且有$\mathscr{A}\subset\mathscr{B}$。如果对任意的$A\in \mathscr{A}$都有$\mu(A)=\tau(A)$,则称$\tau$是$\mu$在$\mathscr{B}$上的\gls{Extension}。
\end{definition}
\begin{lemma}
	设$X$是一个集合,$\mathscr{A}$是$X$上的一个$\pi$系,如果$\sigma(\mathscr{A})$上的测度$\mu,\tau$满足:
	\begin{enumerate}
		\item 对任意的$A\in \mathscr{A}$,有$\mu(A)=\tau(B)$;
		\item 存在$\mathscr{A}$中互不相交的集合序列$\{A_n\}$使得:
		\begin{equation*}
			\underset{n=1}{\overset{+\infty}{\cup}}A_n=X,\;\mu(A_n)<+\infty,\;\forall\;n\in\mathbb{N}^+
		\end{equation*}
	\end{enumerate}
	则对任何的$A\in\sigma(\mathscr{A})$,有$\mu(A)=\tau(A)$。
\end{lemma}
\begin{proof}
	对任意的$B\in \mathscr{A}$且$\mu(B)<+\infty$,令:
	\begin{equation*}
		\mathscr{C}=\{A\in\sigma(\mathscr{A}):\mu(A\cap B)=\tau(A\cap B)\}
	\end{equation*}
	下证明$\mathscr{C}$是一个$\lambda$系。\par
	(1)因为$\sigma(\mathscr{A})$是一个$\sigma$域,所以$X\in\sigma(\mathscr{A})$,于是有$\mu(X\cap B)=\mu(B),\;\tau(X\cap B)=\tau(B)$。由条件(1),因为$B\in \mathscr{A}$,所以$\mu(X\cap B)=\tau(X\cap B)=\mu(B)=\tau(B)$,于是$X\in \mathscr{C}$。\par
	(2)任取$C,D\in \mathscr{C}$且有$C\subset D$,于是有$\mu(C\cap B)=\tau(C\cap B),\;\mu(D\cap B)=\tau(D\cap B)$。考虑$D\backslash C$,则:
	\begin{equation*}
		\mu[(D\backslash C)\cap B]=\mu[(D\cap B)\backslash(C\cap B)]
	\end{equation*}
	因为$C\subset D$,所以$C\cap B\subset D\cap B$,由测度的有限可加性即可得:
	\begin{equation*}
		\mu[(D\backslash C)\cap B]=\mu[(D\cap B)\backslash(C\cap B)]=\mu(D\cap B)-\mu(C\cap B)
	\end{equation*}
	因为$B\in \mathscr{A}$,所以$D\cap B,C\cap B\in \mathscr{A}$,由测度的有限可加性即可得到:
	\begin{equation*}
		\mu[(D\backslash C)\cap B]=\mu(D\cap B)-\mu(C\cap B)=\tau(D\cap B)-\tau(C\cap B)=\tau[(D\backslash C)\cap B]
	\end{equation*}
	所以$D\backslash C\in \mathscr{C}$。\par
	(3)任取$\mathscr{C}$中一个单调不减的集合序列$\{C_n\}$,则有$\mu(C_n\cap B)=\tau(C_n\cap B)$对$n\in\mathbb{N}^+$成立。令$C_0=\varnothing\in \mathscr{C}$,由测度的可列可加性、有限可加性即可得到:
	\begin{align*}
		\mu\left[\left(\underset{n=1}{\overset{+\infty}{\cup}}C_n\right)\cap B\right]
		&=\mu\left\{\left[\underset{n=1}{\overset{+\infty}{\cup}}(C_n\backslash C_{n-1})\right]\cap B\right\}
		=\mu\left\{\underset{n=1}{\overset{+\infty}{\cup}}[(C_n\backslash C_{n-1})\cap B]\right\} \\
		&=\sum_{n=1}^{+\infty}\mu[(C_n\backslash C_{n-1})\cap B]
		=\sum_{n=1}^{+\infty}\mu[(C_n\cap B)\backslash (C_{n-1}\cap B)] \\
		&=\sum_{n=1}^{+\infty}[\mu(C_n\cap B)-\mu(C_{n-1}\cap B)]
		=\sum_{n=1}^{+\infty}[\tau(C_n\cap B)-\tau(C_{n-1}\cap B)] \\
		&=\sum_{n=1}^{+\infty}\tau[(C_n\cap B)\backslash (C_{n-1}\cap B)]
		=\sum_{n=1}^{+\infty}\tau[(C_n\backslash C_{n-1})\cap B] \\
		&=\tau\left\{\underset{n=1}{\overset{+\infty}{\cup}}[(C_n\backslash C_{n-1})\cap B]\right\}
		=\tau\left\{\left[\underset{n=1}{\overset{+\infty}{\cup}}(C_n\backslash C_{n-1})\right]\cap B\right\} \\
		&=\tau\left[\left(\underset{n=1}{\overset{+\infty}{\cup}}C_n\right)\cap B\right]
	\end{align*}
	于是:
	\begin{equation*}
		\underset{n=1}{\overset{+\infty}{\cup}}C_n\in \mathscr{C}
	\end{equation*}\par
	综上,$\mathscr{C}$是一个$\lambda$系。\par
	因为$\mathscr{A}$是一个$\pi$系,对交封闭,所以$\mathscr{A}\subset \mathscr{C}$。由\cref{cor:SigmaPi=LambdaPi}可知$\sigma(\mathscr{A})\subset \mathscr{C}$,于是任意的$A\in\sigma(\mathscr{A})$对任意的$B\in \mathscr{A}$且$\mu(B)<+\infty$,有:
	\begin{equation*}
		\mu(A\cap B)=\tau(A\cap B)
	\end{equation*}
	取条件(2)中的集合序列$\{A_n\}$,由测度的可列可加性可得:
	\begin{align*}
		\mu(A)&=\mu\left[A\cap\left(\underset{n=1}{\overset{+\infty}{\cup}}A_n\right)\right]
		=\mu\left[\underset{n=1}{\overset{+\infty}{\cup}}(A_n\cap A)\right]
		=\sum_{n=1}^{+\infty}\mu(A_n\cap A)=\sum_{n=1}^{+\infty}\tau(A_n\cap A) \\
		&=\tau\left[\underset{n=1}{\overset{+\infty}{\cup}}(A_n\cap A)\right]
		=\tau\left[A\cap\left(\underset{n=1}{\overset{+\infty}{\cup}}A_n\right)\right]
		=\tau(A)\qedhere
	\end{align*}
\end{proof}
\begin{theorem}
	对于半环$\mathscr{A}$上的测度$\mu$,存在$\sigma(\mathscr{A})$上$\mu$的扩张$\tau$。若存在$\mathscr{A}$中互不相交的集合序列$\{A_n\}$使得:
	\begin{equation*}
		\underset{n=1}{\overset{+\infty}{\cup}}A_n=X,\;\mu(A_n)<+\infty,\;\forall\;n\in\mathbb{N}^+
	\end{equation*}
	则$\tau$是唯一的且是由$\mu$生成的外测度。
\end{theorem}
\begin{proof}
	取$\tau$为由$\mu$生成的外测度。证明分成以下三步:
	\begin{enumerate}
		\item 证明对于任意的$A\in \mathscr{A}$,有$\mu(A)=\tau(A)$;
		\item 证明$\tau$是$\sigma(\mathscr{A})$上的测度:
		\begin{enumerate}
			\item 对任何的$A,B\in \mathscr{A}$,有:
			\begin{equation*}
				\tau(B)\geqslant\tau(B\cap A)+\tau(B\cap A^c)
			\end{equation*}
			\item 设$\mathscr{F}$是$X$的所有子集构成的集族,$\mathscr{F}_{\tau}$是$\tau$可测集,则$\mathscr{A}\subset \mathscr{F}_{\tau}$。
		\end{enumerate}
		\item 满足定理条件时$\tau$是唯一的。
	\end{enumerate}\par
	(1)取任意的$A\in \mathscr{A}$,对任意满足$A\subset\underset{n=1}{\overset{+\infty}{\cup}}A_n$的$\mathscr{A}$中的集合序列$\{A_n\}$,由半环上测度的次可列可加性和单调性(\cref{theo:MeasureOfSemiring})可得:
	\begin{equation*}
		\mu(A)=\mu\left[A\cap\left(\underset{n=1}{\overset{+\infty}{\cup}}A_n\right)\right]=\mu\left[\underset{n=1}{\overset{+\infty}{\cup}}(A\cap A_n)\right]\leqslant\sum_{n=1}^{+\infty}\mu(A\cap A_n)\leqslant\sum_{n=1}^{+\infty}\mu(A_n)
	\end{equation*}
	由下确界的不等式性可得:
	\begin{equation*}
		\mu(A)\leqslant\inf\left\{\sum_{n=1}^{+\infty}\mu(A_n):A_n\in\mathscr{A},\;A\subset\underset{n=1}{\overset{+\infty}{\cup}}A_n\right\}=\tau(A)
	\end{equation*}
	再取$B_1=A,\;B_n=\varnothing,\;\forall\;n\geqslant2$,有:
	\begin{equation*}
		\mu(A)=\sum_{n=1}^{+\infty}\mu(B_n)\geqslant\inf\left\{\sum_{n=1}^{+\infty}\mu(A_n):A_n\in\mathscr{A},\;A\subset\underset{n=1}{\overset{+\infty}{\cup}}A_n\right\}=\tau(A)
	\end{equation*}
	于是有$\mu(A)=\tau(A)$。\par
	(2.a)因为$\mathscr{A}$是半环,所以存在互不相交的$\{C_n\}\subset\mathscr{A}$使得:
	\begin{equation*}
		B\cap A^c=B\backslash A=\underset{i=1}{\overset{n}{\cup}}C_i
	\end{equation*}
	由(1)可得$\mu(B)=\tau(B)$,根据半环对交的封闭性与测度的有限可加性可得:
	\begin{align*}
		\tau(B)&=\mu(B)=\mu[(B\cap A)\cup (B\cap A^c)]=\mu(B\cap A)+\mu(B\cap A^c) \\
		&=\mu(B\cap A)+\mu\left(\underset{i=1}{\overset{n}{\cup}}C_i\right)=\mu(B\cap A)+\sum_{i=1}^{n}\mu(C_i)
	\end{align*}
	因为:
	\begin{equation*}
		B\cap A^c\subset\underset{i=1}{\overset{n}{\cup}}C_i
	\end{equation*}
	而半环对包含的差封闭\info{没有证完,涉及到半环的现代式定义}\par
	(2.b)即要证对任意的$A\in \mathscr{A}$和任意的$B\in \mathscr{F}$,有:
	\begin{equation*}
		\tau(B)=\tau(B\cap A)+\tau(B\cap A^c)
	\end{equation*}
\end{proof}