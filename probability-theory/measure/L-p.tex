\section{$L_p$与$L_{\infty}$}

\subsection{$L_p$}
\begin{definition}
	设$(X,\mathscr{F},\mu)$是一个测度空间,$p>1$,$E\in\mathscr{F}$,$f,g$是$E$上的可测函数。定义等价关系$\sim$:$f\sim g\Leftrightarrow f=g\;$a.e.于$E$,将商空间:
	\begin{equation*}
		\left\{f:\int_{E}^{}|f(x)|^p\dif x<+\infty\right\}/\sim
	\end{equation*}
	称之为$E$上的$L_p(X,\mathscr{F},\mu)$空间,简记为$L_p(E)$。
\end{definition}
\begin{property}
	设$(X,\mathscr{F},\mu)$是一个测度空间,$p>1$,$E\in\mathscr{F}$,则$L_p(E)$是一个线性空间。
\end{property}
\begin{proof}
	由\cref{ineq:else-1}和\cref{prop:NonnegativeMeasurablegIntegral}(6)直接得到。
\end{proof}
\subsubsection{$L_p(X,\mathscr{F},\mu)$上的距离}
\begin{definition}
	设$(X,\mathscr{F},\mu)$是一个测度空间,$p>1$,$E\in\mathscr{F}$。在$L_p(E)$中定义元素$x=x(t)$和元素$y=y(t)$之间的距离为:
	\begin{equation*}
		\rho(x,y)=\left[\int_{E}^{}|x(t)-y(t)|^p\dif\mu\right]^\frac{1}{p}
	\end{equation*}
	则$(L_p(E),\rho)$是一个度量空间。
\end{definition}
下证明上式定义的距离满足距离公理:
\begin{proof}
	(1)$\;\rho\in R$:由\cref{ineq:else-1}可得:
	\begin{equation*}
		|x(t)-y(t)|^p\leqslant\Bigl[|x(t)|+|y(t)|\Bigr]^p\leqslant2^{p-1}\Bigl[|x(t)|^p+|y(t)|^p\Bigr]
	\end{equation*}
	于是:
	\begin{equation*}
		\rho(x,y)\leqslant\left[2^{p-1}\int_{E}^{}|x(t)|^p\dif\mu+2^{p-1}\int_{E}^{}|y(t)|^p\dif\mu\right]^{\frac{1}{p}}
	\end{equation*}
	由$L_p(X,\mathscr{F},\mu)$空间定义,$\rho\in\mathbb{R}$。
	(2)非负性由\cref{prop:NonnegativeMeasurablegIntegral}(2)(10)直接可得;(3)对称性直接可得;
	(4)三角不等式:设$x(t),y(t),z(t)\in L_p(E)$。$p=1$时可由绝对值的三角不等式立即得到,$p>1$时,由Minkowski不等式(即\cref{ineq:minkowski-ineq-Lebesgue})可得到:
	\begin{equation*}
		\left[\int_{E}|x(t)-z(t)|^p\dif\mu\right]^{\frac{1}{p}} \leqslant \left[\int_{E}|x(t)-y(t)|^p\dif\mu\right]^{\frac{1}{p}} + \left[\int_{E}|y(t)-z(t)|^p\dif\mu\right]^{\frac{1}{p}}
	\end{equation*}
	即:
	\begin{equation*}
		\rho(x,z)\leqslant\rho(x,y)+\rho(y,z)\qedhere
	\end{equation*}
\end{proof}
\subsubsection{$L_p(X,\mathscr{F},\mu)$上的范数}
\begin{definition}
	设$(X,\mathscr{F},\mu)$是一个测度空间,$p>1$,$E\in\mathscr{F}$。在$L_p(E)$中定义元素$x=x(t)$的范数为:
	\begin{equation*}
		||x||=\left[\int_{E}^{}|x(t)|^p\dif t\right]^\frac{1}{p}
	\end{equation*}
	则$L_p(X,\mathscr{F},\mu)(E)$成为一个赋范线性空间。
\end{definition}
\begin{proof}
	(1)由$L_p(X,\mathscr{F},\mu)$的定义即可得到$||x||\in\mathbb{R}$。(2)非负性由\cref{prop:NonnegativeMeasurablegIntegral}(2)(10)直接得到。(3)数乘由\cref{prop:NonnegativeMeasurablegIntegral}(6)直接得到,(4)三角不等式的证明可由Minkowski不等式(即\cref{ineq:minkowski-ineq-Lebesgue})直接得到。
\end{proof}

\subsection{$L_{\infty}$}
\begin{definition}
	设$(X,\mathscr{F},\mu)$是一个测度空间,$f$是$\mathscr{F}$上的一个可测函数,令:
	\begin{equation*}
		G(f)=\Bigl\{c>0:\mu(\{x:|f(x)|>c\})=0\Bigr\}
	\end{equation*}
	称:
	\begin{equation*}
		||f||_{\infty}=
		\begin{cases}
			\inf G(f),&G(f)\ne\varnothing \\
			+\infty,&G(f)=\varnothing
		\end{cases}
	\end{equation*}
	为$f$的无穷范数,也称其为$f$的\gls{EssenSup}。
\end{definition}
\begin{theorem}
	无穷范数具有如下等价定义:
	\begin{equation*}
		||f||_\infty=\inf_{me=0}\sup_{x\in E\backslash e}|f(x)|
	\end{equation*}
\end{theorem}
\begin{proof}
	等价性由定义可直接证得,略去。
\end{proof}
\begin{definition}
	设$(X,\mathscr{F},\mu)$是一个测度空间,$E\in\mathscr{F}$,$f,g$是$E$上的可测函数。定义等价关系$\sim$:$f\sim g\Leftrightarrow f=g\;$a.e.于$E$,称商空间$\{f:||f||_\infty<+\infty\}/\sim$为$E$上的$L_{\infty}$空间,简记为$L_{\infty}(E)$。
\end{definition}
\begin{property}
	设$(X,\mathscr{F},\mu)$是一个测度空间,$E\in\mathscr{F}$,则$L_{\infty}(E)$是一个线性空间。
\end{property}
\begin{proof}
	任取$x,y\in L_{\infty}(E)$和$\alpha,\beta\in\mathbb{R}^{}$。因为$x,y$各自除了一个零测集外有界,由由半环上测度的次有限可加性(\cref{theo:MeasureOfSemiring}中的次可列可加性推得这两个零测集的并集也是零测集,于是$\alpha x+\beta y$在除了一个零测集以外的空间上作的是$\mathbb{R}^{}$中的加减乘除,于是$\alpha x+\beta y$除了一个零测集外有界,所以$\alpha x+\beta y\in L_{\infty}(E)$,即$L_{\infty}(E)$是一个线性空间。
\end{proof}
\subsubsection{$L_{\infty}$上的距离}
\begin{definition}
	设$(X,\mathscr{F},\mu)$是一个测度空间,$E\in\mathscr{F}$。在$L_{\infty}(E)$中定义元素$x=x(t)$和元素$y=y(t)$之间的距离为:
	\begin{equation*}
		\rho(x,y)=||x-y||_\infty
	\end{equation*}
	则$(L_{\infty}(E),\rho)$是一个度量空间。
\end{definition}
下证明上式定义的距离满足距离公理:
\begin{proof}
	(1)$\;\rho\in R$: 因为$L_{\infty}$是一个线性空间,所以$x-y\in L_{\infty}(E)$,由定义可得$||x-y||_{\infty}\in\mathbb{R}^{}$。\par
	(2)非负性:由无穷范数的定义:
	\begin{equation*}
		\|x-y\|_\infty \geqslant 0.
	\end{equation*}
	且有 \(\|x-y\|_\infty=0\) 当且仅当有 \(|x(t)-y(t)|=0\;\)a.e.于$E$,即 \(x=y\)。\par
	(3)对称性显然。\par
	(4)三角不等式:
	取任意 \(x,y,z\in L_{\infty}(E)\)。由无穷范数等价定义以及下确界定义,对任意的$\varepsilon>0$,存在$E_1,E_2\subset E,\;\mu(E_1)=\mu(E_2)=0$,使得:
	\begin{equation*}
		\sup_{t\in E\setminus E_1}|x(t)-z(t)|\leqslant \rho(x,z)+\frac{\varepsilon}{2},
		\quad
		\sup_{t\in E\setminus E_2}|z(t)-y(t)|\leqslant \rho(z,y)+\frac{\varepsilon}{2}
	\end{equation*}
	由半环上测度的次有限可加性(\cref{theo:MeasureOfSemiring}中的次可列可加性推得),$\mu(E_1\cup E_2)\leqslant\mu(E_1)+\mu(E_2)=0$,所以:
	\begin{align*}
		\rho(x,y)
		&\leqslant\sup_{t\in E\backslash(E_1\cup E_2)}|x(t)-y(t)| \\
		&\leqslant\sup_{t\in E\backslash(E_1\cup E_2)}|x(t)-z(t)|+\sup_{t\in E\backslash(E_1\cup E_2)}|z(t)-y(t)| \\
		&\leqslant\sup_{t\in E\backslash E_1}|x(t)-z(t)|+\sup_{t\in E\backslash E_2}|z(t)-y(t)| \\
		&\leqslant\rho(x,z)+\rho(z,y)+\varepsilon
	\end{align*}
	由$\varepsilon$的任意性可得到:
	\begin{equation*}
		\rho(x,y)\leqslant\rho(x,z)+\rho(z,y)\qedhere
	\end{equation*}
\end{proof}
\subsubsection{$L_{\infty}$上的范数}
\begin{definition}
	设$(X,\mathscr{F},\mu)$是一个测度空间,$E\in\mathscr{F}$。在$L_{\infty}(E)$中定义元素$x=x(t)$的范数为:
	\begin{equation*}
		||x||=||x||_\infty
	\end{equation*}
	则$L_{\infty}(E)$成为一个赋范线性空间。
\end{definition}
下证明上式定义的范数满足范数定义:
\begin{proof}
	(1)$\;||x||\in\mathbb{R}$由$L_{\infty}(E)$空间的定义直接可得。(2)非负性和(3)数乘由无穷范数的定义是显然的。\par
	(4)三角不等式:由本性上确界定义可得:
	\begin{equation*}
		|x(t)|\leqslant||x||_\infty,\;|y(t)|\leqslant||y||_\infty
	\end{equation*}
	a.e.于$E$,所以:
	\begin{equation*}
		|x(t)+y(t)|\leqslant|x(t)|+|y(t)|\leqslant||x||_\infty+||y||_\infty
	\end{equation*}
	a.e.于$E$。于是$||x(t)||_\infty+||y(t)||_\infty\in G(x+y)$,所以:
	\begin{equation*}
		||x+y||_\infty=\inf G(x+y)\leqslant||x||_\infty+||y||_\infty\qedhere
	\end{equation*}
\end{proof}

\subsection{收敛性}
\begin{theorem}
	设$(X,\mathscr{F},\mu)$是一个测度空间,$E\in\mathscr{F}$,$1\leqslant p\leqslant+\infty$。若$\{f_n\}\subset L_p(E)$且:
	\begin{equation*}
		m>n,\;\lim_{n\to+\infty}||f_m-f_n||_p=0
	\end{equation*}
	则存在$f\in L_p(E)$使得$\{f_n\}$依范数收敛于$f$,即在$L_p(E)$中引入范数导出的距离时,$L_p(E)$是一个Banach空间。
\end{theorem}
\begin{proof}
	
\end{proof}
\begin{definition}
	设$(X,\mathscr{F},\mu)$是一个测度空间,$E\in\mathscr{F}$,$1\leqslant p\leqslant+\infty$,$\{f_n\}\subset L_p(E),\;f\in L_p(E)$。若:
	\begin{equation*}
		\lim_{n\to+\infty}||f_n-f||_p=0
	\end{equation*}
	则称$\{f_n\}$\textbf{($p$阶)平均收敛}到$f$,记为$f_n\overset{L_p}{\longrightarrow}f$。
\end{definition}
\begin{theorem}
	设$(X,\mathscr{F},\mu)$是一个测度空间,$E\in\mathscr{F}$,$1\leqslant p\leqslant+\infty$,$\{f_n\}\subset L_p(E),\;f\in L_p(E)$。
	\begin{enumerate}
		\item 若$f_n\overset{L_p}{\longrightarrow}f$,则$f_n\overset{\mu}{\longrightarrow}f$且$||f_n||_p\to||f||_p$;
		\item 若$f_n\overset{\text{a.e.}}{\longrightarrow}f$或$f_n\overset{\mu}{\longrightarrow}f$,则:
		\begin{equation*}
			||f_n||_p\to||f||_p\Leftrightarrow f_n\overset{L_p}{\longrightarrow}f
		\end{equation*}
	\end{enumerate}
\end{theorem}
\begin{proof}
	(1)设$f_n\overset{L_p}{\longrightarrow}f$,则对任意的$\varepsilon>0$,由\cref{prop:NonnegativeMeasurablegIntegral}(7)可得:
	\begin{align*}
		\mu(|f_n-f|\geqslant\varepsilon)
		&=\int_{\{|f_n-f|\geqslant\varepsilon\}}^{}\dif\mu\leqslant\int_{\{|f_n-f|\geqslant\varepsilon\}}^{}\frac{|f_n-f|^p}{\varepsilon^p}\dif\mu \\
		&\leqslant\frac{1}{\varepsilon^p}\int_{-\infty}^{+\infty}|f_n-f|^p\dif\mu=\frac{||f_n-f||^p_p}{\varepsilon^p}
	\end{align*}
	而$||f_n-f||\to0$,所以:
	\begin{equation*}
		\lim_{n\to+\infty}\mu(||f_n-f||\geqslant\varepsilon)=0
	\end{equation*}
	即$f_n\overset{\mu}{\longrightarrow}f$。由\cref{ineq:minkowski-ineq-norm-all}可得:
	\begin{equation*}
		\Bigl|||f_n||_p-||f||_p\Bigr|\leqslant||f_n-f||_p
	\end{equation*}
	于是$||f_n||_p\to||f||_p$。\par
	(2)充分性由(1)直接得到,下证必要性。
\end{proof}