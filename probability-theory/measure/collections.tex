\section{集合与集族}
\begin{definition}
	一个\gls{Set}是指具有某种性质的事物的全体。组成集合的每个事物称为该集合的\gls{Element}。若集合的元素只有有限多个,则称该集合为\gls{FiniteSet},否则称为\gls{InfiniteSet};不含任何元素的集合称为\gls{EmptySet},用符号$\varnothing$表示。语境中所有相关元素构成的集合称之为\gls{UniversalSet},一般用$X$表示,也称为\textbf{空间}或\textbf{全空间}。
\end{definition}
\begin{definition}
	若集合$A$由集合构成,则称$A$是一个\gls{FamilyOfSets}。记$A$的所有子集构成的集合为$\mathscr{P}(A)$。
\end{definition}

\subsection{集合相关}
\subsubsection{集合的关系}
\begin{definition}
	设$A,B$是两个集合:
	\begin{enumerate}
		\item $A$\textbf{包含}$B$是指对任意的$x\in B$有$x\in A$,记作$B\subseteq A$,也称作$B$包含于$A$,$B$是$A$的\gls{Subset};
		\item $A$和$B$\textbf{相等}是指$A\subseteq B$且$B\subseteq A$;
		\item $A$和$B$\gls{Disjoint}是指$A$和$B$没有相同的元素;
	\end{enumerate}
	若$A\ne B$且有$A$包含$B$,则称$B$是$A$的\gls{ProperSubset},记作$B\subsetneq A$。
\end{definition}
\subsubsection{集合的运算}
\begin{definition}
	设$A,B$是全集$X$的两个子集,定义:
	\begin{enumerate}
		\item $A\cup B\coloneq\{x:x\in A\text{或}x\in B\}$,称之为$A$与$B$的\gls{Union};
		\item $A\cap B\coloneq\{x:x\in A\text{且}x\in B\}$,称之为$A$与$B$的\gls{Intersection};
		\item $A\setminus B\coloneq\{x:x\in A\text{且}x\notin B\}$,称之为$A$与$B$的\gls{Difference};
		\item $A\Delta B\coloneq (A\setminus B)\cup(B\setminus A)$,称之为$A$与$B$的\gls{SymmetricDifference};
		\item $A^c\coloneq X\setminus A$,称之为$A$的\gls{Complement}。
	\end{enumerate}
\end{definition}
\begin{property}\label{prop:SetOperation}
	设$A,B,C$是全集$X$的三个子集,$\{A_n:n\in I\}$是$X$的子集构成的一个集族,其中$I$是一个指标集。集合的运算有如下性质:
	\begin{enumerate}
		\item \textbf{幂等性:}$A\cup A=A,\;A\cap A=A$;
		\item \textbf{交换律:}$A\cup B=B\cup A,\;A\cap B=B\cap A,\;A\Delta B=B\Delta A$;
		\item \textbf{结合律:}$(A\cup B)\cup C=A\cup(B\cup C),\;(A\cap B)\cap C=A\cap(B\cap C)$;
		\item \textbf{分配律:}
		\begin{gather*}
			A\bigcap\left(\underset{n\in I}{\overset{}{\bigcup}}A_n\right)=\underset{n\in I}{\overset{}{\bigcup}}(A\cap A_n),\quad A\bigcup\left(\underset{n\in I}{\overset{}{\bigcap}}A_n\right)=\underset{n\in I}{\overset{}{\bigcap}}(A\cup A_n) \\
			\left(\underset{n\in I}{\overset{}{\bigcup}}A_n\right)\Big\backslash A=\underset{n\in I}{\overset{}{\bigcup}}(A_n\setminus A),\quad\left(\underset{n\in I}{\overset{}{\bigcap}}A_n\right)\Big\backslash A=\underset{n\in I}{\overset{}{\bigcap}}(A_n\setminus A) \\
			(A\setminus B)\cap C=(A\cap C)\setminus(B\cap C)
		\end{gather*}
		\item $(A\setminus B)\setminus C=A\setminus(B\cup C)=(A\setminus B)\cap(A\setminus C)$;
		\item $A\setminus B=A\cap B^c=A\setminus (A\cap B)$;
		\item \textbf{De-Morgan Law:}
		\begin{equation*}
			A\Big\backslash\left(\underset{n\in I}{\bigcup}A_n\right)=\underset{n\in I}{\bigcap}(A\backslash A_n),\quad
			A\Big\backslash\left(\underset{n\in I}{\bigcap}A_n\right)=\underset{n\in I}{\bigcup}(A\backslash A_n)
		\end{equation*}
	\end{enumerate}
\end{property}
\begin{proof}
	利用集合相等的定义即可得到。
\end{proof}
\begin{definition}
	设$\{A_n\}$为一个集合序列。
	\begin{enumerate}
		\item 若$A_n\subseteq A_{n+1},\;\forall\;n\in\mathbb{N}^+$,则称$\{A_n\}$为单调递增的集合序列,记为$A_n\uparrow$;
		\item 若$A_n\supseteq A_{n+1},\;\forall\;n\in\mathbb{N}^+$,则称$\{A_n\}$为单调递减的集合序列,记为$A_n\downarrow$;
	\end{enumerate}
	单调递增和单调递减的集合序列统称为单调的集合序列。
\end{definition}
\begin{definition}
	设$\{A_n\}$为一个集合序列,其上下极限定义如下:
	\begin{equation*}
		\varliminf_{n\to+\infty}A_n=\underset{n=1}{\overset{+\infty}{\cup}}\underset{k=n}{\overset{+\infty}{\cap}}A_k,\quad
		\varlimsup_{n\to+\infty}A_n=\underset{n=1}{\overset{+\infty}{\cap}}\underset{k=n}{\overset{+\infty}{\cup}}A_k
	\end{equation*}
	若:
	\begin{equation*}
		\varliminf_{n\to+\infty}A_n=\varlimsup_{n\to+\infty}A_n
	\end{equation*}
	则认为$\{A_n\}$极限存在,记:
	\begin{equation*}
		\lim_{n\to+\infty}A_n=\varliminf_{n\to+\infty}A_n=\varlimsup_{n\to+\infty}A_n
	\end{equation*}
\end{definition}
\begin{property}\label{prop:SetLimit}
	设$\{A_n\}$为一个集合序列,则:
	\begin{enumerate}
		\item $\{A_n\}$的上下极限具有如下等价定义:
		\begin{gather*}
			\varliminf_{n\to+\infty}A_n=\{x:\exists\;N\in\mathbb{N}^+,\;\forall\;n\geqslant N,\;x\in A_n\} \\
			\varlimsup_{n\to+\infty}A_n=\{x:\forall\;N\in\mathbb{N}^+,\;\exists\;n\geqslant N,\;x\in A_n\}=\{x:x\text{属于无穷多个$A_n$}\}
		\end{gather*}
		\item $\{A_n\}$的下极限包含于上极限,即:
		\begin{equation*}
			\varliminf_{n\to+\infty}A_n\subseteq\varlimsup_{n\to+\infty}A_n
		\end{equation*}
		\item 若$\{A_n\}$单调,则$\lim\limits_{n\to+\infty}A_n$存在,且:
		\begin{equation*}
			\lim_{n\to+\infty}A_n=
			\begin{cases}
				\underset{n=1}{\overset{+\infty}{\cup}}A_n,&A_n\uparrow \\
				\underset{n=1}{\overset{+\infty}{\cap}}A_n,&A_n\downarrow 
			\end{cases}
		\end{equation*}
	\end{enumerate}
\end{property}
\begin{proof}
	(1)对于下极限来讲:
	\begin{equation*}
		x\in\varliminf_{n\to+\infty}A_n
		\iff
		\exists\;n\in\mathbb{N}^+,\;x\in\underset{k=n}{\overset{+\infty}{\cap}}A_k
		\iff
		\exists\;N\in\mathbb{N}^+,\;\forall\;n\geqslant N,\;x\in A_n
	\end{equation*}\par
	对于上极限来讲:
	\begin{equation*}
		x\in\varlimsup_{n\to+\infty}A_n
		\iff
		\forall\;n\in\mathbb{N}^+,\;x\in\underset{k=n}{\overset{+\infty}{\cup}}A_k
		\iff
		\forall\;N\in\mathbb{N}^+,\;\exists\;n\geqslant N,\;x\in A_n
	\end{equation*}\par
	下证明上极限的第二个等价定义。\par
	\textbf{必要性:}若此时$x$没有出现在无穷多个$A_n$中,则存在$N\in\mathbb{N}^+$,当$n>N$时有$x\notin A_n$,矛盾。\par
	\textbf{充分性:}当$x$存在于无穷多个$A_n$中时,若存在$N\in\mathbb{N}^+$,当$n>N$时有$x\notin A_n$,则$x$最多存在于$N$个集合中,矛盾。\par
	(2)由(1)可直接得出结论。\par
	(3)当$A_n\uparrow$时,由集合序列上下极限的定义可得:
	\begin{equation*}
		\varliminf_{n\to+\infty}A_n=\underset{n=1}{\overset{+\infty}{\cup}}\underset{k=n}{\overset{+\infty}{\cap}}A_k=\underset{n=1}{\overset{+\infty}{\cup}}A_n,\quad\varlimsup_{n\to+\infty}A_n=\underset{n=1}{\overset{+\infty}{\cap}}\underset{k=n}{\overset{+\infty}{\cup}}A_k=\underset{n=1}{\overset{+\infty}{\cup}}A_n
	\end{equation*}
	所以:
	\begin{equation*}
		\lim_{n\to+\infty}A_n=\varliminf_{n\to+\infty}A_n=\varlimsup_{n\to+\infty}A_n=\underset{n=1}{\overset{+\infty}{\cup}}A_n
	\end{equation*}\par
	当$A_n\downarrow$时,由集合序列上下极限的定义可得:
	\begin{equation*}
		\varliminf_{n\to+\infty}A_n=\underset{n=1}{\overset{+\infty}{\cup}}\underset{k=n}{\overset{+\infty}{\cap}}A_k=\underset{n=1}{\overset{+\infty}{\cap}}A_n,\quad\varlimsup_{n\to+\infty}A_n=\underset{n=1}{\overset{+\infty}{\cap}}\underset{k=n}{\overset{+\infty}{\cup}}A_k=\underset{n=1}{\overset{+\infty}{\cap}}A_n
	\end{equation*}
	所以:
	\begin{equation*}
		\lim_{n\to+\infty}A_n=\varliminf_{n\to+\infty}A_n=\varlimsup_{n\to+\infty}A_n=\underset{n=1}{\overset{+\infty}{\cap}}A_n\qedhere
	\end{equation*}
\end{proof}

\subsection{重要集族}
\subsubsection{$\pi$系、半环、半代数、环、域}
\begin{definition}
	如果$X$上的非空集族$\mathscr{A}$对有限交的运算是封闭的,即:
	\begin{equation*}
		\forall\;A,B\in\mathscr{A},\;A\cap B\in\mathscr{A}
	\end{equation*}
	则称$\mathscr{A}$是一个$\pi$系。
\end{definition}
\begin{definition}
	如果$X$上的非空集族$\mathscr{A}$满足:
	\begin{enumerate}
		\item $\varnothing\in\mathscr{A}$;
		\item 对有限交的运算封闭;
		\item 若$A,B\in\mathscr{A}$,则存在有限个互不相交的$\{C_i\in\mathscr{A}:i=1,2,\dots,n\}$,使得:
		\begin{equation*}
			A\setminus B=\underset{i=1}{\overset{n}{\cup}}C_i
		\end{equation*}
	\end{enumerate}
	则称$\mathscr{A}$为\gls{Semiring}。
\end{definition}
\begin{property}\label{prop:Semiring}
	半环具有如下性质:
	\begin{enumerate}
		\item 若$\mathscr{A}$是一个半环,$\seq{A}{m},\seq{B}{n}\in\mathscr{A}$,则$\underset{i=1}{\overset{m}{\cup}}A_i,\;\left(\underset{i=1}{\overset{m}{\cup}}A_i\right)\Big\backslash\left(\underset{i=1}{\overset{n}{\cup}}B_i\right)$都可以表示为$\mathscr{A}$中的有限不交并;
		\item 半环定义中的第三条可修改为:若$A,B\in\mathscr{A},\;B\subseteq A$,则$A\setminus B$可以表示为$\mathscr{A}$中的有限不交并;
		\item 若$X=\mathbb{R}^{n}$,则集族$\{(a,b]:a,b\in\mathbb{R}^{n}\},\{[a,b):a,b\in\mathbb{R}^{n}\}$都是半环。
	\end{enumerate}
\end{property}
\begin{proof}
	(1)先证明$A\setminus\left(\underset{i=1}{\overset{m}{\cup}}B_i\right)$能表示为$\mathscr{A}$中的有限不交并。\par
	当$m=1$时由半环的定义即可得出结论。假设对$m\in\mathbb{N}^+$成立,下证明对$m+1$成立。\par
	由归纳假设和\cref{prop:SetOperation}(4)可得存在互不相交的$\seq{C}{n}$和$C_{ij},\;i=1,2,\dots,n,\;j=1,2,\dots,n_i$使得:
	\begin{equation*}
		A\Big\backslash\left(\underset{i=1}{\overset{m+1}{\cup}}B_i\right)=\left[A\Big\backslash\left(\underset{i=1}{\overset{m}{\cup}}B_i\right)\right]\setminus B_{m+1}=\left(\underset{i=1}{\overset{n}{\cup}}C_i\right)\setminus B_{m+1}=\underset{i=1}{\overset{n}{\cup}}(C_i\setminus B_{m+1})=\underset{i=1}{\overset{n}{\cup}}\left(\underset{j=1}{\overset{n_i}{\cup}}C_{ij}\right)
	\end{equation*}
	于是结论成立。\par
	注意到:
	\begin{equation*}
		\underset{i=1}{\overset{m}{\cup}}A_i=\underset{i=1}{\overset{m}{\cup}}\left[A_i\Big\backslash\left(\underset{j=1}{\overset{i-1}{\cup}}A_j\right)\right]
	\end{equation*}
	由前面的结论可知$A_i\Big\backslash\left(\underset{j=1}{\overset{i-1}{\cup}}A_j\right)$可以表示为$\mathscr{A}$中的有限不交并,而对于不同的$i$,它们又是不交的,于是结论成立。\par
	(2)只需证明二者互为充要条件。必要性显然,下证充分性。\par
	任取$A,B\in\mathscr{A}$,由\cref{prop:SetOperation}(6)可得$A\setminus B=A\setminus(A\cap B)$,因为$A\cap B\subseteq A$且根据半环的定义可得$A\cap B\in\mathscr{A}$,所以$A\setminus B$可以表示为$\mathscr{A}$中的有限不交并,充分性得证。\par
	(3)逐条验证定义即可,很简单,略去证明。
\end{proof}
\begin{definition}
	如果$X$上的半环$\mathscr{A}$包含$X$,则称$\mathscr{A}$为\gls{SemiAlgebra}。
\end{definition}
\begin{definition}
	如果$X$上的非空集族$\mathscr{A}$对并和差的运算是封闭的,即对任意的$A,B\in\mathscr{A}$:
	\begin{enumerate}
		\item $A\cup B\in\mathscr{A}$;
		\item $A\setminus B\in\mathscr{A}$。
	\end{enumerate}
	则称$\mathscr{A}$为\gls{Ring}。
\end{definition}
\begin{definition}
	如果$X$上的非空集族$\mathscr{A}$对交和补的运算是封闭的,且$X$也在其中,即:
	\begin{enumerate}
		\item $\forall\;A,B\in\mathscr{A},\;A\cap B\in\mathscr{A}$;
		\item $\forall\;A\in\mathscr{A},\;A^c\in\mathscr{A}$;
		\item $X\in\mathscr{A}$。
	\end{enumerate}
	则称$\mathscr{A}$为\gls{FieldOfSets}或\gls{AlgebraOfSets}。
\end{definition}
\begin{property}\label{prop:FieldOfSets}
	域具有如下性质:
	\begin{enumerate}
		\item 域对有限交、并、补、差封闭;
		\item 域定义中的第一条可修改为:$\forall\;A,B\in\mathscr{A},\;A\cup B\in\mathscr{A}$。
	\end{enumerate}
\end{property}
\begin{proof}
	(1)有限交和有限补由定义立即可得。由\cref{prop:SetOperation}(7)可得有限并的封闭性,根据\cref{prop:SetOperation}(6)可得有限差的封闭性。\par
	(2)只需证明二者互为充要条件,由\cref{prop:SetOperation}(7)即可得到。
\end{proof}
\begin{theorem}\label{theo:SetNecessarilySet1}
	半环必是$\pi$系,环必是半环,域必是环和半代数。
\end{theorem}
\begin{proof}
	(1)半环必是$\pi$系可直接由半环的定义得出。\par
	(2)设$\mathscr{A}$是一个环,$A,B\in\mathscr{A}$。根据环的定义,$A\setminus A=\varnothing\in\mathscr{A}$。由集合的运算可得:
	\begin{equation*}
		A\cap B=(A\cup B)\setminus(A\Delta B)=(A\cup B)\setminus[(A\setminus B)\cup(B\setminus A)] 
	\end{equation*}
	因为$\mathscr{A}$是一个环,所以$A\setminus B,B\setminus A\in\mathscr{A}$,$[(A\setminus B)\cup(B\setminus A)]\in\mathscr{A}$,$A\cup B\in\mathscr{A}$,所以$A\cap B=(A\cup B)\setminus[(A\setminus B)\cup(B\setminus A)]\in\mathscr{A}$,即$\mathscr{A}$对交的运算是封闭的。\par
	因为$\mathscr{A}$是一个环,对差的运算封闭,所以取$C=A\setminus B$即有$A\setminus B=C\in\mathscr{A}$。\par
	(3)由\cref{prop:FieldOfSets}(1)可知域是环。\par
	(4)由(3)(2)和域的定义立即可得。
\end{proof}
\subsubsection{单调系、$\lambda$系、$\sigma$环、$\sigma$域}
\begin{definition}
	根据\cref{prop:SetLimit}(3),如果集族$\mathscr{A}$中的所有单调序列$\{A_n\}$都满足$\lim\limits_{n\to+\infty}A_n\in\mathscr{A}$,则称$\mathscr{A}$为\gls{MonotoneClass}。
\end{definition}
\begin{definition}
	如果$X$上的集族$\mathscr{A}$满足:
	\begin{enumerate}
		\item $X\in\mathscr{A}$;
		\item 若$A,B\in\mathscr{A},\;B\subseteq A$,则有$A\setminus B\in\mathscr{A}$;
		\item 单调递增集合序列$\{A_n\}$的极限$\underset{n=1}{\overset{+\infty}{\cup}}A_n\in\mathscr{A}$。
	\end{enumerate}
	则称$\mathscr{A}$为$\lambda$系。
\end{definition}
\begin{property}\label{prop:lambda-System}
	$\lambda$系对补封闭。
\end{property}
\begin{proof}
	设$\mathscr{A}$是一个$\lambda$系,任取$A\in\mathscr{A}$,由$\lambda$系的定义可知$\mathscr{A}$对差封闭,所以有$A^c=X\setminus A\in\mathscr{A}$,即$\lambda$系对补封闭。
\end{proof}
\begin{definition}
	如果$X$上的集族$\mathscr{A}$满足:
	\begin{enumerate}
		\item 若$A_n\in\mathscr{A},\;\forall\;n\in\mathbb{N}^+$,则$\underset{n=1}{\overset{+\infty}{\cup}}A_n\in\mathscr{A}$;
		\item 若$A,B\in\mathscr{A}$,则$A\setminus B\in\mathscr{A}$;
	\end{enumerate}
	则称$\mathscr{A}$为$\sigma$环。
\end{definition}
\begin{definition}
	如果$X$上的集族$\mathscr{A}$满足:
	\begin{enumerate}
		\item $X\in\mathscr{A}$;
		\item 若$A\in\mathscr{A}$,则$A^c\in\mathscr{A}$;
		\item 若$A_n\in\mathscr{A},\;\forall\;n\in\mathbb{N}^+$,则$\underset{n=1}{\overset{+\infty}{\cup}}A_n\in\mathscr{A}$。
	\end{enumerate}
	则称$\mathscr{A}$为$\sigma$域。
\end{definition}
\begin{property}\label{prop:SigmaField}
	$\sigma$域具有如下性质:
	\begin{enumerate}
		\item $\sigma$域是域;
		\item $\sigma$域对有限交和可列交封闭;
		\item $\sigma$域对有限并封闭;
		\item $\sigma$域对有限差和可列差封闭;
		\item 若$\mathscr{A}_i,\;i\in I$都是$\sigma$域,则$\underset{i\in I}{\overset{}{\cap}}\mathscr{A}_i$也是$\sigma$域;
		\item 设$A\subseteq X$且$A\ne\varnothing$,$\mathscr{A}$是$X$上的$\sigma$域,则$A\cap\mathscr{A}\coloneq\{A\cap E:E\in\mathscr{A}\}$是$A$上的$\sigma$域。
	\end{enumerate}
\end{property}
\begin{proof}
	设$\mathscr{A}$是一个$\sigma$域。\par
	(1)设$A,B\in\mathscr{A}$。由$\sigma$域的定义,$\mathscr{A}$对补的运算封闭并且$X\in\mathscr{A}$。根据\cref{prop:SetOperation}(7)可得:
	\begin{equation*}
		A\cap B=A\cap B\cap X\cap\cdots=(A^c\cup B^c\cup\varnothing\cup\cdots)^c\in\mathscr{A}
	\end{equation*}
	所以$\mathscr{A}$是一个域。\par
	(2)有限交由(1)和\cref{prop:FieldOfSets}(1)即可得到。\par
	任取集合序列$\{A_n\}\subseteq\mathscr{A}$,由\cref{prop:SetOperation}(7)和$\sigma$域的定义可得:
	\begin{equation*}
		\underset{n=1}{\overset{+\infty}{\cap}}A_n=\left(\underset{n=1}{\overset{+\infty}{\cup}}A_n^c\right)^c\in\mathscr{A}
	\end{equation*}
	所以$\mathscr{A}$对可列交封闭。\par
	(3)由$\sigma$域的定义可知$\varnothing=X^c\in\mathscr{A}$,所以对任意的$A,B\in\mathscr{A}$有:
	\begin{equation*}
		A\cup B=A\cup B\cup\varnothing\cup\cdots\in\mathscr{A}
	\end{equation*}
	所以$\mathscr{A}$对有限并封闭。\par
	(4)有限差由(1)和\cref{prop:FieldOfSets}(1)即可得到,可列差由\cref{prop:SetOperation}(7)、有限差的结论和(2)即可得到。\par
	(5)由定义逐条验证即可。\par
	(6)因为$X\in\mathscr{A}$,所以$A=A\cap X\in A\cap\mathscr{A}$。\par
	对任意的$B\in A\cap\mathscr{A}$,有$B=A\cap C$,其中$C\in\mathscr{A}$,于是由\cref{prop:SetOperation}(7)(4):
	\begin{equation*}
		A\setminus B=A\cap B^c=A\cap(A\cap C)^c=A\cap(A^c\cup C^c)=(A\cap A^c)\cup(A\cap C^c)=A\cap C^c
	\end{equation*}
	因为$C\in\mathscr{A}$,所以$C^c\in\mathscr{A}$,于是$A\setminus B\in\mathscr{A}$,即$B$在$A$中的补集仍然在$\mathscr{A}$中。\par
	对任意的$\{B_n\}\subseteq A\cap\mathscr{A}$,有$B_n=A\cap C_n$,其中$C_n\in\mathscr{A}$。由\cref{prop:SetOperation}(4)可得:
	\begin{equation*}
		\underset{n=1}{\overset{+\infty}{\cup}}B_n=\underset{n=1}{\overset{+\infty}{\cup}}(A\cap C_n)=A\cap\left(\underset{n=1}{\overset{+\infty}{\cup}}C_n\right)
	\end{equation*}
	因为$\mathscr{A}$是$\sigma$域,所以$\underset{n=1}{\overset{+\infty}{\cup}}C_n\in\mathscr{A}$,于是$\underset{n=1}{\overset{+\infty}{\cup}}B_n\in A\cap\mathscr{A}$。\par
	综上,$A\cap\mathscr{A}$是$A$上的$\sigma$域。
\end{proof}
\begin{theorem}\label{theo:SetNecessarilySet2}
	$\lambda$系是单调系,$\sigma$域是$\lambda$系。
\end{theorem}
\begin{proof}
	(1)设$\mathscr{A}$是一个$\lambda$系。由$\lambda$系的定义,$\mathscr{A}$中单调递增的集合序列必在$\mathscr{A}$中有极限。任取$\mathscr{A}$中的单调递减序列$\{A_n\}$,由\cref{prop:lambda-System}可知$\{A_n^c\}$是$\mathscr{A}$中的一个单调递增序列,于是根据\cref{prop:SetLimit}(3)可得:
	\begin{equation*}
		\lim_{n\to+\infty}A_n^c=\underset{n=1}{\overset{+\infty}{\cup}}A_n^c\in\mathscr{A}
	\end{equation*}
	所以根据\cref{prop:SetOperation}(7)和\cref{prop:lambda-System}可得:
	\begin{equation*}
		\underset{n=1}{\overset{+\infty}{\cap}}A_n=\left(\underset{n=1}{\overset{+\infty}{\cup}}A_n^c\right)^c\in\mathscr{A}
	\end{equation*}
	即$\{A_n\}$在$\mathscr{A}$中有极限。由$\{A_n\}$的任意性,$\mathscr{A}$中单调递减的集合序列也必在$\mathscr{A}$中有极限。综上,$\mathscr{A}$是一个单调系。\par
	(2)由\cref{prop:SigmaField}(4)可知$\sigma$域对差的运算封闭,根据定义即可得出结论。
\end{proof}
\subsubsection{集族的关系总结}
上面提到的集族之间有如下关系:
\begin{equation*}
	\text{单调系}\supset\text{$\lambda$系}\supset\text{$\sigma$域}\subset\text{域}\subset\text{环}\subset\text{半环}\subset\text{$\pi$系}
\end{equation*}
\begin{theorem}\label{theo:RingContainX=Field}
	一个包含$X$的环是域,一个包含$X$的$\sigma$环是$\sigma$域。
\end{theorem}
\begin{proof}
	(1)设$\mathscr{A}$是一个环且$X\in\mathscr{A}$。由\cref{theo:SetNecessarilySet1}可知$\mathscr{A}$对交的运算封闭。\par
	因为$A^c=X\setminus A$,所以$A^c\in\mathscr{A}$。由$A$的任意性,$\mathscr{A}$对补的运算封闭。\par
	综上,$\mathscr{A}$是一个域,即一个包含$X$的环是域。\par
	(2)设$\mathscr{A}$是$\sigma$环,$A\in\mathscr{A}$。因为$X\in\mathscr{A},\;A^c=X\setminus A$,而$\sigma$环对差的运算封闭,所以$\mathscr{A}$对补的运算封闭。由定义可知$\mathscr{A}$是$\sigma$域,即一个包含$X$的$\sigma$环是$\sigma$域。
\end{proof}
\begin{theorem}\label{theo:Monotone+Field=SigmaField}
	一个既是单调系又是域的集族必是$\sigma$域。
\end{theorem}
\begin{proof}
	设$\mathscr{A}$既是单调系又是域。因为$\mathscr{A}$是一个域,所以对补的运算封闭且$X\in\mathscr{A}$。任取$A_n\in\mathscr{A},\;n\in\mathbb{N}^+$,由\cref{theo:SetNecessarilySet1}可知$\mathscr{A}$对有限并封闭,即$\underset{i=1}{\overset{n}{\cup}}A_i\in\mathscr{A},\;\forall\;n\in\mathbb{N}^+$。因为$\mathscr{A}$是单调系,根据\cref{prop:SetLimit}(3),单调递增集合序列:
	\begin{equation*}
		\left\{B_n=\underset{i=1}{\overset{n}{\cup}}A_i\right\}
	\end{equation*}
	的极限:
	\begin{equation*}
		\lim_{n\to+\infty}B_n=\underset{n=1}{\overset{+\infty}{\cup}}B_n=\underset{n=1}{\overset{+\infty}{\cup}}\underset{i=1}{\overset{n}{\cup}}A_i=\underset{n=1}{\overset{+\infty}{\cup}}A_n\in\mathscr{A}
	\end{equation*}
	由$\{A_n\}$的任意性,$\mathscr{A}$对可列并封闭。综上,$\mathscr{A}$是一个$\sigma$域,即一个既是单调系又是域的集族必是$\sigma$域。
\end{proof}
\begin{theorem}\label{theo:Lambda+Pi=Sigma}
	一个既是$\lambda$系又是$\pi$系的集族必是$\sigma$域。
\end{theorem}
\begin{proof}
	设$\mathscr{A}$既是$\lambda$系又是$\pi$系。因为$\mathscr{A}$是$\lambda$系,所以$X\in\mathscr{A}$。任取$A\in\mathscr{A}$,由$\lambda$系的定义可得$A^c=X\setminus A\in\mathscr{A}$。由$A$的任意性,$\mathscr{A}$对补的运算封闭。又因为$\mathscr{A}$是一个$\pi$系,所以$\mathscr{A}$对交的运算封闭。综上可知$\mathscr{A}$是一个域。由\cref{theo:SetNecessarilySet2}可知$\lambda$系是单调系,所以$\mathscr{A}$既是域又是单调系,由\cref{theo:Monotone+Field=SigmaField}可得$\mathscr{A}$是$\sigma$域。
\end{proof}

\subsection{集族的生成}
\begin{definition}
	设$\mathscr{A},\mathscr{B}$是$X$上的集族,$\mathscr{B}$是环(或单调系,或$\lambda$系,或$\sigma$域)。若:
	\begin{enumerate}
		\item $\mathscr{A}\subseteq\mathscr{B}$;
		\item 对$X$上任意的另一环(或单调系,或$\lambda$系,或$\sigma$域)$\mathscr{C}$,若$\mathscr{A}\subseteq\mathscr{C}$,就有$\mathscr{B}\subseteq\mathscr{C}$。
	\end{enumerate}
	则称$\mathscr{B}$是由集族$\mathscr{A}$生成的环(或单调系,或$\lambda$系,或$\sigma$域),即由集族$\mathscr{A}$生成的环(或单调系,或$\lambda$系,或$\sigma$域)是包含$\mathscr{A}$的最小的环(或单调系,或$\lambda$系,或$\sigma$域),将由集族$\mathscr{A}$生成的环、单调系、$\lambda$系和$\sigma$域分别记作$r(\mathscr{A}),\;m(\mathscr{A}),\;l(\mathscr{A}),\;\sigma(\mathscr{A})$。
\end{definition}
\begin{theorem}
	由任何集族$\mathscr{A}$生成的环、单调系、$\lambda$系和$\sigma$域都存在。
\end{theorem}
\begin{proof}
	设$\mathscr{B}$为$X$的所有子集构成的集族,则$\mathscr{B}$是一个$\sigma$域,所以由\cref{prop:SigmaField}(1)和\cref{theo:SetNecessarilySet1}可得$\mathscr{B}$是一个环(根据\cref{theo:SetNecessarilySet2}可知或单调系,或$\lambda$系)并且有$\mathscr{A}\subseteq\mathscr{B}$。把所有包含集族$\mathscr{A}$的环(或单调系,或$\lambda$系,或$\sigma$域)的全体记为$\mathbf{A}$,则$\mathscr{B}\in \mathbf{A}$,于是$\mathbf{A}$非空。记:
	\begin{equation*}
		\mathscr{C}=\underset{\mathscr{D}\in \mathbf{A}}{\cap}\mathscr{D}
	\end{equation*}
	$\mathscr{C}$就是由$\mathscr{A}$生成的环(或单调系,或$\lambda$系,或$\sigma$域),根据定义逐条验证即可。
\end{proof}
\begin{theorem}\label{theo:RingGeneratedBySemiring}
	如果$\mathscr{A}$是半环,则:
	\begin{equation*}
		r(\mathscr{A})=
		\underset{n=1}{\overset{+\infty}{\cup}}
		\left\{\underset{i=1}{\overset{n}{\cup}}A_i:A_i\in\mathscr{A};\;A_i\cap A_j=\varnothing,\forall\;i\ne j\right\}
	\end{equation*}
\end{theorem}
\begin{proof}
	令:
	\begin{equation*}
		\mathscr{B}=\underset{n=1}{\overset{+\infty}{\cup}}
		\left\{\underset{i=1}{\overset{n}{\cup}}A_i:A_i\in\mathscr{A};\;A_i\cap A_j=\varnothing,\forall\;i\ne j\right\}
	\end{equation*}
	由$\mathscr{B}$的定义,$\mathscr{A}\subseteq\mathscr{B}$。因为环对有限并封闭,所以包含$\mathscr{A}$的环必然包含$\mathscr{B}$。若证得$\mathscr{B}$是一个环,则可得到$r(\mathscr{A})=\mathscr{B}$。\par
	任取$A,B\in\mathscr{B}$,则存在$m,n\in\mathbb{N}^+$和互不相交的$\seq{A}{m}\in\mathscr{A}$、互不相交的$\seq{B}{n}\in\mathscr{A}$使得:
	\begin{equation*}
		A=\underset{i=1}{\overset{m}{\cup}}A_i,\;
		B=\underset{i=1}{\overset{n}{\cup}}B_i
	\end{equation*}
	于是由\cref{prop:SetOperation}(6)(4)(7)可得:
	\begin{equation*}
		A\setminus B=A\cap B^c=\underset{i=1}{\overset{m}{\cup}}(A_i\cap B^c)=\underset{i=1}{\overset{m}{\cup}}\left[A_i\cap\left(\underset{j=1}{\overset{n}{\cap}}B_i^c\right)\right]=\underset{i=1}{\overset{m}{\cup}}\underset{j=1}{\overset{n}{\cap}}(A_i\cap B_j^c)=\underset{i=1}{\overset{m}{\cup}}\underset{j=1}{\overset{n}{\cap}}(A_i\setminus B_j)
	\end{equation*}
	因为$\mathscr{A}$是半环,于是存在互不相交的$C_{ij1},C_{ij2},\dots,C_{ijr_{ij}}\in\mathscr{A}$使得:
	\begin{equation*}
		A_i\setminus B_j=\underset{k=1}{\overset{r_{ij}}{\cup}}C_{ijk}
	\end{equation*}
	于是:
	\begin{align*}
		A\setminus B&=\underset{i=1}{\overset{m}{\cup}}\underset{j=1}{\overset{n}{\cap}}(A_i\setminus B_j)=\underset{i=1}{\overset{m}{\cup}}\underset{j=1}{\overset{n}{\cap}}\underset{k=1}{\overset{r_{ij}}{\cup}}C_{ijk} \\
		&=\underset{i=1}{\overset{m}{\cup}}[(C_{111}\cup C_{112}\cdots\cup C_{11r_{11}})\cap(C_{121}\cup C_{122}\cdots\cup C_{12r_{12}})\cdots] \\
		&=\underset{i=1}{\overset{m}{\cup}}\left\{\underset{j=1}{\overset{r_{11}}{\cup}}[C_{11j}\cap(C_{121}\cup C_{122}\cdots\cup C_{12r_{12}})]\right\}\cap(C_{131}\cup C_{132}\cdots\cup C_{13r_{13}})\cdots \\
		&=\underset{i=1}{\overset{m}{\cup}}\left[\underset{l=1}{\overset{r_{11}}{\cup}}\underset{k=1}{\overset{r_{12}}{\cup}}(C_{11l}\cap C_{12k})\right]\cap(C_{131}\cup C_{132}\cdots\cup C_{13r_{13}})\cdots \\
		&=\underset{i=1}{\overset{n}{\cup}}\underset{k_1=1}{\overset{r_{i1}}{\cup}}\underset{k_2=1}{\overset{r_{i2}}{\cup}}\cdots\underset{k_n=1}{\overset{r_{i_n}}{\cup}}\underset{j=1}{\overset{n}{\cap}}C_{ijk_j}
	\end{align*}
	因为$\mathscr{A}$是半环,所以$\underset{j=1}{\overset{n}{\cap}}C_{ijk_j}\in\mathscr{A}$。因为$C_{ij1},C_{ij2},\dots,C_{ijr_{ij}}$互不相交且$\seq{A}{m}$互不相交,所以$\underset{j=1}{\overset{n}{\cap}}C_{ijk_j}$互不相交,于是有$A\setminus B\in\mathscr{B}$,即$\mathscr{B}$对差封闭。\par
	考虑$A\cup B=B\cup(A\setminus B)$,因为$B\cap (A\setminus B)=\varnothing$,所以$A\cup B\in\mathscr{B}$。\par
	综上,$\mathscr{B}$对并和差封闭,所以$\mathscr{B}$是一个环。
\end{proof}
\begin{note}
	下面两个定理的证明思路具有统一的结构。其核心困难在于无法直接验证生成族本身的封闭性,证明中采用“固定一个集合,构造与其运算良好的集合族”的方法:先固定$A$,构造:
	\begin{equation*}
		\mathscr{B_A}=\{B:\text{与 }A\text{ 运算后仍留在生成族中}\}
	\end{equation*}
	再证明$\mathscr{B_A}$本身是单调系(或$\lambda$系)并包含原集族族。由生成的最小性可知生成族必包含于$\mathscr{B_A}$,从而推出“与$A$的运算对生成族中的任意元素都成立”。再重复该过程,得到生成族中任意两元素的封闭性。
\end{note}

\begin{theorem}\label{theo:SigmaField=MonotoneField}
	若$\mathscr{A}$是域,则$\sigma(\mathscr{A})=m(\mathscr{A})$。
\end{theorem}
\begin{proof}
	由\cref{theo:SetNecessarilySet2}可知$\sigma(\mathscr{A})$是包含$\mathscr{A}$的单调系,所以$m(\mathscr{A})\subseteq\sigma(\mathscr{A})$。下证$\sigma(\mathscr{A})\subseteq m(\mathscr{A})$。\par
	若能证得$m(\mathscr{A})$是一个$\sigma$域即可得出结论,由\cref{theo:Monotone+Field=SigmaField}可得一个既是单调系又是域的集族必是$\sigma$域,所以证得$m(\mathscr{A})$是一个域即可。又因为$\mathscr{A}$是域,所以$X\in\mathscr{A}$,同时$X\in m(\mathscr{A})$,由\cref{theo:RingContainX=Field}可得一个包含$X$的环是域,所以只需证明$m(\mathscr{A})$是一个环。\par
	对任意的$A\in\mathscr{A}$,令:
	\begin{equation*}
		\mathscr{B}_A=\{B\in m(\mathscr{A}):A\cup B,A\setminus B\in m(\mathscr{A})\}
	\end{equation*}
	任取$\mathscr{B}_A$中一个单调不减序列$\{B_n\}$,根据\cref{prop:SetLimit}(3)可知:
	\begin{equation*}
		A\cup\left(\lim_{n\to+\infty}B_n\right)=A\cup\left(\underset{n=1}{\overset{+\infty}{\cup}}B_n\right)
		=\underset{n=1}{\overset{+\infty}{\cup}}(A\cup B_n)
	\end{equation*}
	且$\{A\cup B_n\}$也是一个单调不减序列。因为对任意的$n\in\mathbb{N}^+$,有$A\cup B_n\in m(\mathscr{A})$,所以根据\cref{prop:SetLimit}(3)可知:
	\begin{equation*}
		A\cup\left(\lim_{n\to+\infty}B_n\right)=\lim_{n\to+\infty}(A\cup B_n)\in m(\mathscr{A})
	\end{equation*}
	根据\cref{prop:SetLimit}(3)和\cref{prop:SetOperation}(7)可知:
	\begin{equation*}
		A\setminus\left(\lim_{n\to+\infty}B_n\right)=A\setminus\left(\underset{n=1}{\overset{+\infty}{\cup}}B_n\right)
		=\underset{n=1}{\overset{+\infty}{\cap}}(A\setminus B_n)
	\end{equation*}
	则$\{A\setminus B_n\}$是一个单调不增序列。因为对任意的$n\in\mathbb{N}^+$,有$A\setminus B_n\in m(\mathscr{A})$,所以由\cref{prop:SetOperation}(6)(7)和\cref{prop:SetLimit}(3)可得:
	\begin{equation*}
		A\setminus\left(\underset{n=1}{\overset{+\infty}{\cup}}B_n\right)=A\cap\left(\underset{n=1}{\overset{+\infty}{\cup}}B_n\right)^c=A\cap\left(\underset{n=1}{\overset{+\infty}{\cap}}B_n^c\right)=\underset{n=1}{\overset{+\infty}{\cap}}(A\cap B_n^c)=\lim_{n\to+\infty}(A\setminus B_n)\in m(\mathscr{A})
	\end{equation*}
	于是:
	\begin{equation*}
		\lim_{n\to+\infty}B_n=\underset{n=1}{\overset{+\infty}{\cup}}B_n\in\mathscr{B}_A
	\end{equation*}
	由$\{B_n\}$的任意性,$\mathscr{B}_A$对单调不减序列的极限封闭。同理,$\mathscr{B}_A$对单调不增序列的极限封闭。\par
	综上,$\mathscr{B}_A$是一个单调系。\par
	因为$\mathscr{A}$是一个域,由\cref{theo:SetNecessarilySet1}可知$\mathscr{A}$是一个环,所以$\mathscr{A}$对并和差封闭,即$\mathscr{A}\subseteq\mathscr{B}_A$,于是$m(\mathscr{A})\subseteq\mathscr{B}_A$,因此:
	\begin{equation*}
		\forall\;A\in\mathscr{A},\;\forall\;B\in m(\mathscr{A}),\; A\cup B,A\setminus B\in m(\mathscr{A})
	\end{equation*}\par
	对任意的$D\in m(\mathscr{A})$,令:
	\begin{equation*}
		\mathscr{C}_D=\{C\in m(\mathscr{A}):C\cup D,C\setminus D\in m(\mathscr{A})\}
	\end{equation*}
	与之前类似可得$\mathscr{C}_D$是一个单调系。由:
	\begin{equation*}
		\forall\;A\in\mathscr{A},\;\forall\;B\in m(\mathscr{A}),\;A\cup B,A\setminus B\in m(\mathscr{A})
	\end{equation*}
	可得$\mathscr{A}\subseteq\mathscr{C}_D$,由生成的定义,$m(\mathscr{A})\subseteq\mathscr{C}_D$,再根据$\mathscr{C}_D$定义中$D$的任意性可得:
	\begin{equation*}
		\forall\;A,B\in m(\mathscr{A}),\;A\cup B,A\setminus B\in m(\mathscr{A})
	\end{equation*}
	所以$m(\mathscr{A})$是一个环。
\end{proof}
\begin{corollary}\label{cor:SigmaField=MonotoneField}
	如果$\mathscr{A}$是域,$\mathscr{B}$是单调系,则:
	\begin{equation*}
		\mathscr{A}\subseteq\mathscr{B}\Rightarrow\sigma(\mathscr{A})\subseteq\mathscr{B}
	\end{equation*}
	并且该推论与上一定理等价。
\end{corollary}
\begin{proof}
	\textbf{(1)必要性:}因为$\mathscr{A}$是域,所以$\sigma(\mathscr{A})=m(\mathscr{A})$。因为$\mathscr{B}$是包含$\mathscr{A}$的单调系,由生成的定义,$\sigma(\mathscr{A})\subseteq\mathscr{B}$。\par
	\textbf{(2)充分性:}由$\mathscr{B}$的任意性和生成的定义直接可得。
\end{proof}
\begin{theorem}\label{theo:SigmaPi=LambdaPi}
	如果$\mathscr{A}$是$\pi$系,则$\sigma(\mathscr{A})=l(\mathscr{A})$。
\end{theorem}
\begin{proof}
	由\cref{theo:SetNecessarilySet2}可知$\sigma(\mathscr{A})$是一个$\lambda$系,所以$l(\mathscr{A})\subseteq\sigma(\mathscr{A})$。下证$\sigma(\mathscr{A})\subseteq l(\mathscr{A})$。\par
	若证得$l(\mathscr{A})$是一个$\sigma$域即可得出结论。由\cref{theo:Lambda+Pi=Sigma}可知只需证明$l(\mathscr{A})$是一个$\pi$系。\par
	对任意的$A\in\mathscr{A}$,令:
	\begin{equation*}
		\mathscr{B}_A=\{B\in l(\mathscr{A}):A\cap B\in l(\mathscr{A})\}
	\end{equation*}
	因为$l(\mathscr{A})$是$\lambda$系,所以$X\in l(\mathscr{A})$,而$A\cap X=A\in\mathscr{A}$,由生成的定义,$A\cap X\in l(\mathscr{A})$,于是$X\in\mathscr{B}_A$。\par
	任取$C,D\in\mathscr{B}_A$且$C\subseteq D$,则有$C,D\in l(\mathscr{A})$,于是$D\setminus C\in l(\mathscr{A})$。由\cref{prop:SetOperation}(4)可得:
	\begin{equation*}
		A\cap(D\setminus C)=(A\cap D)\setminus(A\cap C)
	\end{equation*}
	因为$C,D\in\mathscr{B}_A$,所以$A\cap C,A\cap D\in l(\mathscr{A})$。因为$C\subseteq D$,所以$A\cap C\subseteq A\cap D$,由$\lambda$系的定义可得$(A\cap D)\setminus(A\cap C)\in l(\mathscr{A})$,即$A\cap(D\setminus C)\in l(\mathscr{A})$,所以$D\setminus C\in\mathscr{B}_A$。\par
	任取$\mathscr{B}_A$中的一个单调不减的集合列$\{B_n\}$,则有$B_n\in l(\mathscr{A}),A\cap B_n\in l(\mathscr{A})$对$n\in\mathbb{N}^+$成立,于是$\{B_n\}$是$l(\mathscr{A})$中单调不减的集合列。由\cref{prop:SetLimit}(3)和$\lambda$系的定义可知:
	\begin{equation*}
		\lim_{n\to+\infty}B_n=\underset{n=1}{\overset{+\infty}{\cup}}B_n\in l(\mathscr{A})
	\end{equation*}
	根据\cref{prop:SetOperation}(4)可得:
	\begin{equation*}
		A\cap\left(\underset{n=1}{\overset{+\infty}{\cup}}B_n\right)
		=\underset{n=1}{\overset{+\infty}{\cup}}(A\cap B_n)
	\end{equation*}
	因为$\{B_n\}$单调不减,所以$\{A\cap B_n\}$是$l(\mathscr{A})$中单调不减的集合列,由\cref{prop:SetLimit}(3)和$\lambda$系的定义可得:
	\begin{equation*}
		A\cap\left(\underset{n=1}{\overset{+\infty}{\cup}}B_n\right)=\lim_{n\to+\infty}(A\cap B_n)\in l(\mathscr{A})
	\end{equation*}
	所以有:
	\begin{equation*}
		\lim_{n\to+\infty}B_n=\underset{n=1}{\overset{+\infty}{\cup}}B_n\in\mathscr{B}_A
	\end{equation*}
	综上,$\mathscr{B}_A$是一个$\lambda$系。\par
	因为$\mathscr{A}$是一个$\pi$系,所以$\mathscr{A}\subseteq\mathscr{B}_A$,即$l(\mathscr{A})\subseteq\mathscr{B}_A$。这说明:
	\begin{equation*}
		\forall\;A\in\mathscr{A},\;\forall\;B\in l(\mathscr{A}),\;A\cap B\in l(\mathscr{A})
	\end{equation*}\par
	对任意的$D\in l(\mathscr{A})$,令:
	\begin{equation*}
		\mathscr{C}_D=\{C\in l(\mathscr{A}):C\cap D\in l(\mathscr{A})\}
	\end{equation*}
	与之前类似可得$\mathscr{C}_D$是一个$\lambda$系。由:
	\begin{equation*}
		\forall\;A\in\mathscr{A},\;\forall\;B\in l(\mathscr{A}),\;A\cap B\in l(\mathscr{A})
	\end{equation*}
	可知$\mathscr{A}\subseteq\mathscr{C}_D$,所以$l(\mathscr{A})\subseteq\mathscr{C}_D$。由$\mathscr{C}_D$的定义可知$l(\mathscr{A})$对交封闭,所以$l(\mathscr{A})$是一个$\pi$系。
\end{proof}
\begin{corollary}\label{cor:SigmaPi=LambdaPi}
	如果$\mathscr{A}$是$\pi$系,$\mathscr{B}$是$\lambda$系,则:
	\begin{equation*}
		\mathscr{A}\subseteq\mathscr{B}\Rightarrow\sigma(\mathscr{A})\subseteq\mathscr{B}
	\end{equation*}
	并且该推论与上一定理等价。
\end{corollary}
\begin{proof}
	\textbf{(1)必要性:}因为$\mathscr{A}$是$\pi$系,所以$\sigma(\mathscr{A})=l(\mathscr{A})$。因为$\mathscr{B}$是包含$\mathscr{A}$的$\lambda$系,由生成的定义,$\sigma(\mathscr{A})\subseteq\mathscr{B}$。\par
	\textbf{(2)充分性:}由$\mathscr{B}$的任意性和生成的定义直接可得。
\end{proof}

\subsection{Borel$\;\sigma$域}
%\subsubsection{$\mathbb{R}$上开集与闭集的构造}
%\begin{definition}
%	设$E$是$\mathbb{R}$上的开集,如果开区间$(\alpha,\beta)\subseteq E$且$\alpha,\beta\notin E$,则称$(\alpha,\beta)$为$E$的\gls{ComponentInterval}。
%\end{definition}
%\begin{theorem}\label{theo:ROpenClosedSetComponentInterval}
%	$\mathbb{R}$上任一非空开集$E$可以表示为至多可列个不相交的构成区间的并集,任一闭集是从$\mathbb{R}$上挖掉至多可列个互不相交的开区间所得到的集合。
%\end{theorem}
%\begin{proof}
%	\textbf{开集:}该定理的证明分为如下三步:
%	\begin{enumerate}
%		\item $E$的任意两个不同的构成区间不相交;
%		\item $E$中的任意一点必含在一个构成区间中;
%		\item $E$的所有构成区间的并集为$E$且构成区间至多可列。
%	\end{enumerate}\par
%	(1)任取$E$的两个不同的构成区间$(\alpha_1,\beta_1),\;(\alpha_2,\beta_2)$,若这两个构成区间相交,则$\alpha_1,\beta_1,\alpha_2,\beta_2$这四个点至少有一个点在另一个构成区间内,从而在$E$中,这与构成区间的定义矛盾,所以$(\alpha_1,\beta_1),\;(\alpha_2,\beta_2)$不相交。由$(\alpha_1,\beta_1),\;(\alpha_2,\beta_2)$的任意性可得$E$的任意两个不同的构成区间必不相交。\par
%	(2)任取$x\in E$,记:
%	\begin{equation*}
%		\mathscr{A}=\{(\alpha,\beta):x\in(\alpha,\beta)\subseteq E\}
%	\end{equation*}
%	因为$E$是开集,所以$\mathscr{A}\ne\varnothing$。取:
%	\begin{equation*}
%		\alpha_0=\inf\{\alpha:(\alpha,\beta)\in\mathscr{A}\},\;
%		\beta_0=\sup\{\beta:(\alpha,\beta)\in\mathscr{A}\}
%	\end{equation*}
%	作开区间$(\alpha_0,\beta_0)$,显然有$x\in(\alpha_0,\beta_0)$。下面证明$(\alpha_0,\beta_0)$是一个构成区间。\par
%	任取$x_1\in(\alpha_0,\beta_0)$,则$\alpha_0<x_1<\beta_0$。由上下确界的定义,存在$(\alpha_1,\beta_1)\in\mathscr{A}$使得$x_1\in(\alpha_1,\beta_1)\in\mathscr{A}$($A,B\in\mathscr{A}\Rightarrow A\cup B\in\mathscr{A}$),于是$x_1\in E$,由$x_1$的任意性可得$(\alpha_0,\beta_0)\subseteq E$。\par
%	若$\alpha_0\in E$,因为$E$是一个开集,所以必然存在一个$\varepsilon>0$使得$(\alpha_0-\varepsilon,\alpha_0+\varepsilon)\subset E$,则$(\alpha_0-\varepsilon,\beta_0)\subseteq E$且$x\in(\alpha_0-\varepsilon,\beta_0)$,那么就有$(\alpha_0-\varepsilon,\beta_0)\in\mathscr{A}$,这与$\alpha_0$是下确界矛盾,于是$\alpha_0\notin E$。同理,$\beta_0\notin E$。\par
%	综上,$(\alpha_0,\beta_0)$是一个构成区间。\par
%	由先前$x$的任意性可得对于任意的$x\in E$,$x$必含在$E$的一个构成区间中,具体的构成区间由上述$\alpha_0,\beta_0$的产生过程给出。\par
%	(3)由(2)可知$E$中任意一点必含在一个构成区间中,对$E$中所有的点取其对应的构成区间的并集即可得到所有构成区间的并集为$E$。由有理数在实数系中的稠密性\info{有理数的稠密性},各构成区间必含有一个有理数。由(1)可得不同的构成区间不相交,于是每个构成区间可由其中包含的一个有理数来表示,根据\cref{cor:QCountable}可知$E$的构成区间至多可列。\par
%	\textbf{闭集:}设闭集$E\subset\mathbb{R}$,由\cref{prop:OpenClosedSet}(5)可知$E^c$是一个开集,则$E^c$可表示为其构成区间的并集,于是$E=\mathbb{R}\setminus E^c$是从$\mathbb{R}$上挖掉至多可列个互不相交的开区间所得到的集合。
%\end{proof}
%\subsubsection{$\mathbb{R}^{}$上的Borel$\;\sigma$域}
%\begin{definition}
%	称$\sigma(\{(a,b):a,b\in\mathbb{R}\})$为\gls{BorelSigmaField},记作$\mathcal{B}$。$\mathcal{B}$中的元素被称为\gls{BorelSet}。
%\end{definition}
%\begin{lemma}\label{lem:OpenSet=CountableFiniteOpenSetUnion}
%	$\mathbb{R}$上任意非空开集可以表示为至多可列个有限开区间的并集。
%\end{lemma}
%\begin{proof}
%	设$E$是$\mathbb{R}$上的一个非空开集,取$E$的一个构成区间$(\alpha,\beta)$。\par
%	\textbf{(1)$\;\alpha,\beta\in\mathbb{R}$:}此时其自身即为有限开区间。\par
%	\textbf{(2)$\;\alpha=-\infty,\beta\in\mathbb{R}$:}此时有:
%	\begin{equation*}
%		(\alpha,\beta)=\underset{n=1}{\overset{+\infty}{\cup}}(\beta-n,\beta)
%	\end{equation*}\par
%	\textbf{(3)$\;\alpha\in\mathbb{R},\beta=+\infty$:}此时有:
%	\begin{equation*}
%		(\alpha,\beta)=\underset{n=1}{\overset{+\infty}{\cup}}(\alpha,\alpha+n)
%	\end{equation*}\par
%	\textbf{(4)$\;\alpha=-\infty,\beta=+\infty$:}此时有:
%	\begin{equation*}
%		(\alpha,\beta)=\underset{n=1}{\overset{+\infty}{\cup}}(2-n,1+n)
%	\end{equation*}\par
%	上述结果表明开集的构成区间可由至多可列个有限开区间的并集表示,由\cref{theo:ROpenClosedSetComponentInterval}可知开集至多由可列个构成区间的并集表示,因为\cref{theo:CountableUnionCountable},所以$\mathbb{R}$上任意开集可以表示为至多可列个有限开区间的并集。
%\end{proof}
%\begin{property}\label
%	Borel$\;\sigma$域有如下性质:	
%	\begin{enumerate}
%		\item $\mathcal{B}$有如下等价定义:
%		\begin{enumerate}
%			\item 设$\mathbb{R}$上所有开集构成的集族为$\mathcal{O}$,则$\sigma(\mathcal{O})=\mathcal{B}$;
%			\item 
%			设$\mathbb{R}$上所有闭集构成的集族为$\mathcal{D}$,则$\sigma(\mathcal{D})=\mathcal{B}$;
%			\item $\mathcal{B}=\sigma(\{(-\infty,a]:a\in\mathbb{R}\})=\sigma(\{(-\infty,a):a\in\mathbb{R}\})=\sigma(\{(a,+\infty):a\in\mathbb{R}\})=\sigma(\{[a,+\infty):a\in\mathbb{R}\})$;
%			\item $\mathcal{B}=\sigma(\{(a,b]:a,b\in\mathbb{R}\})=\sigma(\{[a,b):a,b\in\mathbb{R}\})$;
%			\item $\mathcal{B}=\sigma(\{[a,b]:a,b\in\mathbb{R}\})$;
%		\end{enumerate}
%		且将c、d、e中的$\mathbb{R}$换成$\mathbb{R}^{}$中任一可数稠密子集$D$也成立;
%		\item $\mathbb{R}$中的开集、闭集、区间(有限或无穷)、单点集都是Borel集。
%	\end{enumerate}
%\end{property}
%\begin{proof}
%	(1) \textbf{ a:}由\cref{prop:OpenClosedSet}(1)可知$\{(a,b):a,b\in\mathbb{R}\}\subseteq\mathcal{O}$,于是有$\{(a,b):a,b\in\mathbb{R}\}\subseteq\sigma(\mathcal{O})$,由生成的定义可得$\mathcal{B}=\sigma(\{(a,b):a,b\in\mathbb{R}\})\subseteq\sigma(\mathcal{O})$。\par
%	由\cref{lem:OpenSet=CountableFiniteOpenSetUnion}可知对$\mathbb{R}$中的任意一个开集,它都可以表示为至多可列个有限开区间的并集,所以$\mathcal{O}\subseteq\sigma(\{(a,b):a,b\in\mathbb{R}\})=\mathcal{B}$,由生成的定义可得$\sigma(\mathcal{O})\subseteq\mathcal{B}$。\par
%	综上,$\sigma(\mathcal{O})=\mathcal{B}$。\par
%	\textbf{b:}由\cref{prop:OpenClosedSet}(5)和开集时的情况即可得到。\par
%	\textbf{cde:}仅给出构造公式。
%	\begin{gather*}
%		(a,b)=\underset{n=1}{\overset{+\infty}{\bigcup}}\left\{\left(-\infty,b-\frac{1}{n}\right]\Big\backslash\left(-\infty,a+\frac{1}{n}\right]\right\}\in\sigma(\{(-\infty,a]:a\in\mathbb{R}\}) \\
%		(-\infty,a]=\left[\underset{n=1}{\overset{+\infty}{\cup}}(a-n,a)\right]\bigcup\left[\underset{n=1}{\overset{+\infty}{\bigcap}}\left(a-\frac{1}{n},a+\frac{1}{n}\right)\right]\in\sigma(\{(a,b):a,b\in\mathbb{R}\}) \\
%		(a,b)=(-\infty,b)\Big\backslash\left[\underset{n=1}{\overset{+\infty}{\bigcap}}\left(-\infty,a+\frac{1}{n}\right)\right]\in\sigma(\{(-\infty,a):a\in\mathbb{R}\}) \\
%		(-\infty,a)=\underset{n=1}{\overset{+\infty}{\cup}}(a-n,a)\in\sigma(\{(a,b):a,b\in\mathbb{R}\}) \\
%		(a,b)=\underset{n=1}{\overset{+\infty}{\bigcup}}\left(a,b-\frac{1}{n}\right]\in\sigma(\{(a,b]:a,b\in\mathbb{R}\}) \\
%		(a,b]=(a,b)\bigcup\left[\underset{n=1}{\overset{+\infty}{\bigcap}}\left(b-\frac{1}{n},b+\frac{1}{n}\right)\right]\in\sigma(\{(a,b):a,b\in\mathbb{R}\}) \\
%		(a,b)=\underset{n=1}{\overset{+\infty}{\bigcup}}\left[a+\frac{1}{n},b\right)\in\sigma(\{[a,b):a,b\in\mathbb{R}\}) \\
%		[a,b)=(a,b)\bigcup\left[\underset{n=1}{\overset{+\infty}{\bigcap}}\left(a-\frac{1}{n},a+\frac{1}{n}\right)\right]\in\sigma(\{(a,b):a,b\in\mathbb{R}\}) \\
%		(a,b)=[a,b]\Big\backslash\left\{\underset{n=1}{\overset{+\infty}{\bigcap}}\left[a,a+\frac{1}{n}\right]\right\}\Big\backslash\left\{\underset{n=1}{\overset{+\infty}{\bigcap}}\left[b,b+\frac{1}{n}\right]\right\}\in\sigma(\{[a,b]:a,b\in\mathbb{R}\}) \\
%		[a,b]=\underset{n=1}{\overset{+\infty}{\bigcap}}\left(a-\frac{1}{n},b+\frac{1}{n}\right)\in\sigma(\{(a,b):a,b\in\mathbb{R}\})
%	\end{gather*}\par
%	稠密子集时的情况由\cref{theo:Density}(3)即可得到。\par
%	(2)由(1)立即可得。
%\end{proof}
%\begin{definition}
%	定义$\mathcal{B}_{\overline{\mathbb{R}}}=\sigma(\{\mathcal{B},\{-\infty\},\{+\infty\}\})$。
%\end{definition}
%\begin{theorem}\label{theo:BorelRwqEquivDef}
%	$\mathcal{B}_{\overline{\mathbb{R}}}$有如下等价定义:
%	\begin{align*}
%		\mathcal{B}_{\overline{\mathbb{R}}}
%		&=\sigma(\{[-\infty,a):a\in\mathbb{R}\}) \\
%		&=\sigma(\{[-\infty,a]:a\in\mathbb{R}\}) \\
%		&=\sigma(\{(a,+\infty]:a\in\mathbb{R}\}) \\
%		&=\sigma(\{[a,+\infty]:a\in\mathbb{R}\})
%	\end{align*}
%	且将其中的$\mathbb{R}$换成$\mathbb{R}^{}$中任一稠密子集$D$也成立。
%\end{theorem}
%\begin{proof}
%	下给出构造公式。
%	\begin{gather*}
%		[-\infty,a)=\{-\infty\}\cup(-\infty,a)\in\mathcal{B}_{\overline{\mathbb{R}}} \\
%		\{-\infty\}=\underset{n=1}{\overset{+\infty}{\cap}}[-\infty, -n),\{+\infty\}=\underset{n=1}{\overset{+\infty}{\cap}}[-\infty,n)^c\in\sigma(\{[-\infty,a):a\in\mathbb{R}\}) \\
%		(-\infty, a)=[-\infty,a)\setminus\{-\infty\}\in\sigma(\{[-\infty,a):a\in\mathbb{R}\}) \\
%		[-\infty,a]=\{-\infty\}\cup(-\infty,a]\in\mathcal{B}_{\overline{\mathbb{R}}} \\
%		\{-\infty\}=\underset{n=1}{\overset{+\infty}{\cap}}[-\infty,-n],\{+\infty\}=\underset{n=1}{\overset{+\infty}{\cap}}[-\infty,n]^c\in\sigma(\{[-\infty,a]:a\in\mathbb{R}\}) \\
%		(-\infty,a]=[-\infty,a]\setminus\{-\infty\}\in\sigma(\{[-\infty,a):a\in\mathbb{R}\})
%	\end{gather*}
%	稠密子集时的情况由\cref{theo:Density}(3)即可得到。
%\end{proof}
%\subsubsection{$\mathbb{R}^{n}$上的Borel$\;\sigma$域}
\begin{definition}
	设$\mathcal{O}$为$\mathbb{R}^{n}$上所有开集构成的集族,称$\sigma(\mathcal{O})$为\gls{BorelSigmaField},记作$\mathcal{B}(\mathbb{R}^{n})$。$\mathcal{B}(\mathbb{R}^{n})$中的元素被称为\gls{BorelSet}。定义$\overline{\mathbb R}^n$上的Borel$\;\sigma$域为:
	\begin{equation*}
		\mathcal{B}(\overline{\mathbb R}^n)\coloneq\sigma\Bigl(\mathcal{B}(\mathbb R^n)\cup\bigl\{\{x\}\subseteq\overline{\mathbb R}^n:\ x\in\overline{\mathbb R}^n\setminus\mathbb R^n\bigr\}\Bigr)
	\end{equation*}
\end{definition}
\begin{definition}
	设\(a=(a_1,\dots,a_n),b=(b_1,\dots,b_n)\in\overline{\mathbb{R}}^n\)且\(a_i\leqslant b_i\),定义:
	\begin{align*}
		(a,b)&=\{x\in\mathbb{R}^{n}:a_i<x_i<b_i,\;i=1,2,\dots,n\} \\
		[a,b)&=\{x\in\mathbb{R}^{n}:a_i\leqslant x_i<b_i,\;i=1,2,\dots,n\} \\
		(a,b]&=\{x\in\mathbb{R}^{n}:a_i<x_i\leqslant b_i,\;i=1,2,\dots,n\} \\
		[a,b]&=\{x\in\mathbb{R}^{n}:a_i\leqslant x_i\leqslant b_i,\;i=1,2,\dots,n\}
	\end{align*}
\end{definition}
\begin{lemma}\label{lem:OpenCountableCubes}
	$\mathbb{R}^{n}$上任意非空开集可以表示为至多可列个有限开方块的并集。
\end{lemma}
\begin{proof}
	因为$E$是开集,对任意$x=(\seq{x}{n})\in E$存在$r_x>0$使得开球$U(x,r_x)\subseteq E$,而$U(x,r_x)$中存在以$x$为中心且各端点都是有理数的开方块:
	\begin{equation*}
		\delta=\frac{r_x}{\sqrt{2n}},\quad a_{i,x}\in\mathbb{Q}\cap(x_i-\delta,x_i),\quad b_{i_x}\in\mathbb{Q}\cap(x_i,x_i+\delta),\quad G_x=\prod_{i=1}^{n}(a_{i,x},b_{i,x})
	\end{equation*}
	由\cref{theo:CountableUnionCountable}可知结论成立。
\end{proof}
\begin{property}\label{prop:BorelSigmaField}
	Borel$\;\sigma$域有如下性质:
	\begin{enumerate}
		\item $\mathcal{B}(\mathbb{R}^{n})$有如下等价定义:
		\begin{enumerate}
			\item 设$\mathbb{R}^{n}$上所有闭集构成的集族为$\mathcal{C}$,则$\sigma(\mathcal{C})=\mathcal{B}(\mathbb{R}^{n})$;
			\item $\mathcal{B}(\mathbb{R}^{n})$可由方块生成:
			\begin{align*}
				\mathcal{B}(\mathbb{R}^{n})&=\sigma(\{(-\infty,a]:a\in\mathbb{R}^{n}\})=\sigma(\{(-\infty,a):a\in\mathbb{R}^{n}\}) \\
				&=\sigma(\{(a,+\infty):a\in\mathbb{R}^{n}\})=\sigma(\{[a,+\infty):a\in\mathbb{R}^{n}\}) \\
				&=\sigma(\{(a,b]:a,b\in\mathbb{R}^{n}\})=\sigma(\{[a,b):a,b\in\mathbb{R}^{n}\}) \\
				&=\sigma(\{[a,b]:a,b\in\mathbb{R}^{n}\})
			\end{align*}
		\end{enumerate}
		且可将b中的$\mathbb{R}^{n}$换成$\mathbb{R}^{n}$中一可数稠密子集;
		\item $\mathcal{B}(\overline{\mathbb{R}}^{n})$有如下等价定义:
		\begin{align*}
			\mathcal{B}(\overline{\mathbb{R}}^{n})
			&=\sigma(\{[-\infty,a):a\in\mathbb{R}^{n}\}) \\
			&=\sigma(\{[-\infty,a]:a\in\mathbb{R}^{n}\}) \\
			&=\sigma(\{(a,+\infty]:a\in\mathbb{R}^{n}\}) \\
			&=\sigma(\{[a,+\infty]:a\in\mathbb{R}^{n}\})
		\end{align*}
		且可将$\mathbb{R}^{n}$换成$\mathbb{R}^{n}$中一可数稠密子集。
	\end{enumerate}
\end{property}
\begin{proof}
	由\cref{prop:OpenClosedSet}和\cref{lem:OpenCountableCubes}可构造得到。
\end{proof}