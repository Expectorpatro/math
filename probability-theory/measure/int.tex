\section{积分论}

\subsection{非负简单函数的积分}
\begin{definition}
	设$\varphi$为测度空间$(X,\mathscr{F},\mu)$上的一个非负简单函数,即$X$可表示为有限个互不相交的集合$E_1,E_2,\dots,E_n\in\mathscr{F}$的并,且在$E_i$上$\varphi=a_i\geqslant0$,即:
	\begin{equation*}
		\varphi(x)=\sum_{i=1}^{n}a_iI(x\in E_i)
	\end{equation*}
	其中$I(x\in E_i)$为表示$x$是否在$E_i$中的示性函数。对于任意的$A\in\mathscr{F}$,根据\cref{prop:SigmaField}(2),将$\varphi(x)$在$A$上的积分定义为:
	\begin{equation*}
		\int_{A}\varphi(x)\dif\mu=\sum_{i=1}^{n}a_i\mu(A\cap E_i)
	\end{equation*}
\end{definition}
\begin{note}
	在定义积分的时候,一些书只定义函数在空间$X$上的积分,根据\cref{prop:SigmaField}(6)可知可以将可测集$A$视为空间,此时的$A\cap\mathscr{F}$构成了$\mathscr{F}$的一个子$\sigma$域,而由\cref{prop:Measure}(4)可知$\mu$在$A\cap\mathscr{F}$上也构成测度,所以函数在可测集$A$上的积分可以定义为函数在测度空间$(A,A\cap\mathscr{F},\mu)$的空间$A$上的积分。
\end{note}
\begin{property}\label{prop:NonnegativeSimpleIntegral}
	设$\varphi,\;\psi$为测度空间$(X,\mathscr{F},\mu)$上的非负简单函数,可分别表示为:
	\begin{equation*}
		\varphi(x)=\sum_{i=1}^{m}a_iI(x\in E_i),\quad
		\psi(x)=\sum_{j=1}^{n}b_jI(x\in F_j)
	\end{equation*}
	则:
	\begin{enumerate}
		\item 对任意的$A\in\mathscr{F}$,$\varphi(x)$的所有表达式在$A$上的积分值相同;
		\item 对于任意的$A\in\mathscr{F}$,有:
		\begin{equation*}
			\int_{A}\varphi(x)\dif \mu\geqslant0
		\end{equation*}
		\item 若$A\in\mathscr{F}$且$\mu(A)=0$,则有:
		\begin{equation*}
			\int_{A}\varphi(x)\dif\mu=0
		\end{equation*}
		\item 设$A,B\in\mathscr{F}$且$A\cap B=\varnothing$,则:
		\begin{equation*}
			\int_{A\cup B}\varphi(x)\dif\mu=\int_{A}\varphi(x)\dif\mu+\int_{B}\varphi(x)\dif\mu
		\end{equation*}
		\item 对任意的$A\in\mathscr{F}$和$\alpha,\beta\in \mathbb{R}$且$\alpha,\beta\geqslant0$:
		\begin{equation*}
			\int_{A}\left[\alpha\varphi(x)+\beta\psi(x)\right]\dif\mu
			=\alpha\int_{A}\varphi(x)\dif\mu+\beta\int_{A}\psi(x)\dif\mu
		\end{equation*}
		\item 取$A\in\mathscr{F}$,若对任意的$x\in A$有$\varphi(x)\geqslant \psi(x)$,则有:
		\begin{equation*}
			\int_{A}\varphi(x)\dif\mu\geqslant\int_{A}\psi(x)\dif\mu
		\end{equation*}
		\item 设$\{A_n\}\subseteq\mathscr{F}$,$A_n\uparrow E\in\mathscr{F}$或$\mu(A_1)<+\infty$且$A_n\downarrow E\in\mathscr{F}$,则:
		\begin{equation*}
			\lim_{n\to+\infty}\left[\int_{A_n}\varphi(x)\dif\mu\right]=\int_{E}\varphi(x)\dif\mu
		\end{equation*}
		\item 取$A\in\mathscr{F}$,若非负简单函数列$\varphi_n(x)\uparrow$且对任意的$x\in A$有$\lim\limits_{n\to+\infty}\varphi_n(x)\geqslant \psi(x)$,则有:
		\begin{equation*}
			\lim_{n\to+\infty}\left[\int_{A}\varphi_n(x)\dif\mu\right]\geqslant\int_{A}\psi(x)\dif\mu
		\end{equation*}
	\end{enumerate}
\end{property}
\begin{proof}
	(1)由\cref{prop:MeasurableFunction}(4),将$\varphi$表示为:
	\begin{equation*}
		\varphi(x)=\sum_{k=1}^{p}c_kI(x\in\{f=c_k\})
	\end{equation*}
	其中$\{c_k:k=1,2,\dots,p\}$为$\varphi$的值域,所以$p\leqslant m$。对任意的$i$和$k$有:
	\begin{equation*}
		E_i\subseteq\{f=c_k\}\quad\text{或}\quad E_i\cap\{f=c_k\}=\varnothing
	\end{equation*}
	当$E_i\subseteq\{f=c_k\}$时有$a_i=c_k$。记$A_k=\{f=c_k\}$,由\cref{prop:SetOperation}(4)和\cref{prop:Measure}(1)可得:
	\begin{align*}
		\sum_{i=1}^{m}a_i\mu(A\cap E_i)
		&=\sum_{i=1}^{m}a_i\mu[(A\cap E_i)\cap X]
		=\sum_{i=1}^{m}a_i\mu\left[(A\cap E_i)\cap\left(\underset{k=1}{\overset{p}{\cup}}A_k\right)\right] \\
		&=\sum_{i=1}^{m}a_i\mu\left[\underset{k=1}{\overset{p}{\cup}}(A\cap E_i\cap A_k)\right]
		=\sum_{i=1}^{m}a_i\sum_{k=1}^{p}\mu(A\cap E_i\cap A_k) \\
		&=\sum_{i=1}^{m}\sum_{E_i\subseteq A_k}a_i\mu(A\cap E_i\cap A_k)
		=\sum_{i=1}^{m}\sum_{E_i\subseteq A_k}c_k\mu(A\cap E_i\cap A_k) \\
		&=\sum_{i=1}^{m}\sum_{k=1}^{p}c_k\mu(A\cap E_i\cap A_k)
		=\sum_{k=1}^{p}\sum_{i=1}^{m}c_k\mu(A\cap A_k\cap E_i) \\
		&=\sum_{k=1}^{p}c_k\sum_{i=1}^{m}\mu(A\cap A_k\cap E_i)
		=\sum_{k=1}^{p}c_k\mu\left[\underset{i=1}{\overset{m}{\cup}}(A\cap A_k\cap E_i)\right] \\
		&=\sum_{k=1}^{p}c_k\mu\left[(A\cap A_k)\cap\left(\underset{i=1}{\overset{m}{\cup}}E_i\right)\right]
		=\sum_{k=1}^{p}c_k\mu[(A\cap A_k)\cap X] \\
		&=\sum_{k=1}^{p}c_k\mu(A\cap A_k)
	\end{align*}\par
	(2)由非负简单函数积分的定义和测度的非负性直接可得。\par
	(3)由非负简单函数积分的定义、测度的非负性和\cref{prop:Measure}(3)(单调性)直接可得。\par
	(4)由非负简单函数积分的定义、\cref{prop:SetOperation}(4)和\cref{prop:Measure}(1)可得:
	\begin{align*}
		\int_{A\cup B}\varphi(x)\dif\mu
		&=\sum_{i=1}^{n}a_i\mu[(A\cup B)\cap E_i]
		=\sum_{i=1}^{n}a_i\mu[(A\cap E_i)\cup(B\cap E_i)] \\
		&=\sum_{i=1}^{n}a_i[\mu(A\cap E_i)+\mu(B\cap E_i)]
		=\sum_{i=1}^{n}a_i\mu(A\cap E_i)+\sum_{i=1}^{n}a_i\mu(B\cap E_i) \\
		&=\int_{A}\varphi(x)\dif\mu+\int_{B}\varphi(x)\dif\mu
	\end{align*}\par
	(5)由\cref{prop:SimpleFunction}(2)可得$\alpha\varphi+\beta\psi$也是非负简单函数。由非负简单函数积分的定义、\cref{prop:Measure}(1)和\cref{prop:SetOperation}(4)可得:
	\begin{align*}
		&\int_{A}[\alpha\varphi(x)+\beta\psi(x)]\dif\mu
		=\sum_{i=1}^{m}\sum_{j=1}^{n}(\alpha a_i+\beta b_j)\mu[A\cap (E_i\cap F_j)] \\
		=&\sum_{i=1}^{m}\alpha a_i\left[\sum_{j=1}^{n}\mu(A\cap E_i\cap F_j)\right]+\sum_{j=1}^{n}\beta b_j\left[\sum_{i=1}^{m}\mu(A\cap E_i\cap F_j)\right] \\ =&\sum_{i=1}^{m}\alpha a_i\mu(A\cap E_i)+\sum_{j=1}^{n}\beta b_j\mu(A\cap F_j) =\alpha\sum_{i=1}^{m}a_i\mu(A\cap E_i)+\beta\sum_{j=1}^{n}b_j\mu(A\cap F_j) \\
		=&\alpha\int_{A}\varphi(x)\dif\mu+\beta\int_{A}\psi(x)\dif\mu
	\end{align*}\par
	(6)因为$\varphi,\psi$是非负简单函数,由\cref{prop:SimpleFunction}(2)可知$\varphi(x)-\psi(x)$也是非负简单函数。根据(5)(2)可得:
	\begin{align*}
		\int_{A}\varphi(x)\dif\mu&=\int_{A}[\psi(x)+\varphi(x)-\psi(x)]\dif\mu=\int_{A}\psi(x)\dif\mu+\int_{A}[\varphi(x)-\psi(x)]\dif\mu \\
		&\geqslant\int_{A}\psi(x)\dif\mu
	\end{align*}\par
	(7)由非负简单函数积分的定义、\cref{prop:RSeq}(8.b)(8.c)和\cref{prop:Measure}(3)(上下连续性)可得:
	\begin{equation*}
		\lim_{n\to+\infty}\left[\int_{A_n}\varphi(x)\dif\mu\right]
		=\lim_{n\to+\infty}\left[\sum_{i=1}^{m}a_i\mu(A_n\cap E_i)\right]
		=\sum_{i=1}^{m}a_i\mu(E\cap E_i)
		=\int_{E}\varphi(x)\dif\mu
	\end{equation*}\par
	(8)对任意的$\alpha\in(0,1)$,记$A_n(\alpha)=\{\varphi_n\geqslant\alpha\psi\}\cap A$。由\cref{prop:SimpleFunction}(1)可知$\{\varphi_n\},\psi(x)$是可测函数,根据\cref{prop:MeasurableFunction}(5.a)可得$\alpha\psi$也是可测函数。由\cref{prop:MeasurableFunction}(3)和\cref{prop:SigmaField}(2)可知$A_n(\alpha)\in\mathscr{F}$。设$\varphi_n$可表示为:
	\begin{equation*}
		\varphi_n(x)=\sum_{k=1}^{p_n}a_{nk}I(x\in E_{nk})
	\end{equation*}
	其中$\{E_{nk}\}$是$X$的有限可测分割。由\cref{prop:SigmaField}(2)可得:
	\begin{equation*}
		\varphi_n(x)I[x\in A_n(\alpha)]=\sum_{k=1}^{p_n}a_{nk}I[x\in E_{nk}\cap A_n(\alpha)],\;E_{nk}\cap A_n(\alpha)\in\mathscr{F}
	\end{equation*}
	所以$\varphi_n(x)I[x\in A_n(\alpha)]$也是一个非负简单函数。同理,$\psi I[x\in A_n(\alpha)]$也是一个非负简单函数。因为$\varphi_n\geqslant\varphi_nI[x\in A_n(\alpha)]\geqslant\alpha\psi I[x\in A_n(\alpha)$,由(6)(5)、\cref{prop:SigmaField}(4)和(4)可得:
	\begin{align*}
		\int_{A}\varphi_n(x)\dif\mu
		&\geqslant\int_{A}\varphi_n(x)I[x\in A_n(\alpha)](x)\dif\mu\geqslant\int_{A}\alpha\psi(x)I[x\in A_n(\alpha)](x)\dif\mu \\
		&=\alpha\int_{A}\psi(x)I[x\in A_n(\alpha)](x)\dif\mu=\alpha\int_{A_n(\alpha)}\psi(x)\dif\mu
	\end{align*}
	因为$\varphi_n\uparrow$且$\lim\limits_{n\to+\infty}\varphi_n(x)\geqslant\psi(x)$对任意$x\in A$成立,所以$A_n(\alpha)\uparrow A$。由\cref{prop:RSeq}(6)(8.c)和(7)可得:
	\begin{equation*}
		\lim_{n\to+\infty}\left[\int_{A}\varphi_n(x)\dif\mu\right]\geqslant\alpha\int_{A}\psi(x)\dif\mu
	\end{equation*}
	再取$\alpha\to 1$,由\cref{prop:RMap}(4)(5.c)即可得到结论。
\end{proof}

\subsection{非负可测函数的积分}
\begin{definition}
	设$f$是测度空间$(X,\mathscr{F},\mu)$上的一个非负可测函数,对于任意的$A\in\mathscr{F}$,将$f$在$A$上的积分定义为:
	\begin{equation*}
		\int_{A}f(x)\dif\mu=\sup_{\varphi(x)}\left\{\int_{A}\varphi(x)\dif\mu:\varphi(x)\text{是非负简单函数,且}\;\forall\;x\in A,\;\varphi(x)\leqslant f(x)\right\}
	\end{equation*}
	若$\int_{A}f(x)\dif\mu<+\infty$,则称$f(x)$在$A$上可积。
\end{definition}
\begin{property}\label{prop:NonnegativeMeasurableIntegral}
	设$f$和$g$为测度空间$(X,\mathscr{F},\mu)$上的非负可测函数,则:
	\begin{enumerate}
		\item 若$f(x)$是非负简单函数,则其在非负简单函数下定义的积分值与在非负可测函数下定义的积分值相同;
		\item 对于任意的$A\in\mathscr{F}$,$\int_{A}f(x)\dif\mu\geqslant0$;
		\item 若$\mu(A)=0$且$A\in\mathscr{F}$,则$\int_{A}f(x)\dif\mu=0$;
		\item 对于任意的$A\in\mathscr{F}$,若$\{f_n\}$是非负简单函数列且$f_n\uparrow f$,则:
		\begin{align*}
			&\int_{A}f(x)\dif\mu
			=\lim_{n\to+\infty}\left[\int_{A}f_n(x)\dif\mu\right] \\
			=&\lim_{n\to+\infty}\left\{\sum_{j=1}^{n2^n}\frac{j-1}{2^n}\mu\left[\left\{\frac{j-1}{2^n}\leqslant f<\frac{j}{2^n}\right\}\bigcap A\right]+n\mu[\{f\geqslant n\}\cap A]\right\}
		\end{align*}
		\item 设$A,B\in\mathscr{F}$且$A\cap B=\varnothing$,则:
		\begin{equation*}
			\int_{A\cup B}f(x)\dif\mu=\int_{A}f(x)\dif\mu+\int_{B}f(x)\dif\mu
		\end{equation*}
		\item 若$f\leqslant g\;$a.e.于$(A,A\cap\mathscr{F},\mu)$,$A\in\mathscr{F}$,则$\int_{A}f(x)\dif\mu\leqslant\int_{A}g(x)\dif\mu$;
		\item 若$f=g\;$a.e.于$(A,A\cap\mathscr{F},\mu)$,$A\in\mathscr{F}$,则$\int_{A}f(x)\dif\mu=\int_{A}g(x)\dif\mu$;
		\item 取$A\in\mathscr{F}$,若$\int_{A}f(x)\dif\mu<+\infty$,则$f(x)$有限a.e.于$(A,A\cap\mathscr{F},\mu)$;
		\item 取$A\in\mathscr{F}$,$\int_{A}f(x)\dif\mu=0$的充分必要条件为$f(x)=0\;$a.e.于$(A,A\cap\mathscr{F},\mu)$;
		\item 对任意的$A\in\mathscr{F},\;\alpha\geqslant0$有\footnote{请关注$\alpha$为正无穷时的证明,尤其需要关注定性无穷与定量无穷的区别。}:
		\begin{equation*}
			\int_{A}\alpha f(x)\dif\mu=\alpha\int_{A}f(x)\dif\mu,\quad\int_{A}[f(x)+g(x)]\dif\mu=\int_{A}f(x)\dif\mu+\int_{A}g(x)\dif\mu
		\end{equation*}
	\end{enumerate}
\end{property}
\begin{proof}
	(1)由非负可测函数积分的定义和\cref{prop:NonnegativeSimpleIntegral}(6)直接可得。\par
	(2)由非负可测函数积分的定义和\cref{prop:NonnegativeSimpleIntegral}(2)直接可得。\par
	(3)由非负可测函数积分的定义和\cref{prop:NonnegativeSimpleIntegral}(3)直接可得。\par
	(4)由非负可测函数积分的定义和所给条件可知对任意的$n\in\mathbb{N}^+$有:
	\begin{equation*}
		\int_{A}f_n(x)\dif\mu\leqslant\int_{A}f(x)\dif\mu
	\end{equation*}
	由\cref{prop:RSeq}(6)可得:
	\begin{equation*}
		\lim_{n\to+\infty}\left[\int_{A}f_n(x)\dif\mu\right]\leqslant\int_{A}f(x)\dif\mu
	\end{equation*}
	对任意满足$\varphi\leqslant f$的非负简单函数$\varphi(x)$,有:
	\begin{equation*}
		\lim_{n\to+\infty}f_n=f\geqslant\varphi
	\end{equation*}
	于是由\cref{prop:NonnegativeSimpleIntegral}(8)可得:
	\begin{equation*}
		\lim_{n\to+\infty}\left[\int_{A}f_n(x)\dif\mu\right]\geqslant\int_{A}\varphi(x)\dif\mu
	\end{equation*}
	由上确界的不等式性可得:
	\begin{equation*}
		\lim_{n\to+\infty}\left[\int_{A}f_n(x)\dif\mu\right]\geqslant\int_{A}f(x)\dif\mu
	\end{equation*}
	于是就有:
	\begin{equation*}
		\lim_{n\to+\infty}\left[\int_{A}f_n(x)\dif\mu\right]=\int_{A}f(x)\dif\mu
	\end{equation*}
	由\cref{prop:MeasurableFunction}(8)可得到积分值的具体表示。\par
	(5)设$\varphi(x)$是$A\cup B$上任一满足$\varphi\leqslant f$的非负简单函数,于是由\cref{prop:NonnegativeSimpleIntegral}(4)可得:
	\begin{equation*}
		\int_{A\cup B}\varphi(x)\dif\mu=\int_{A}\varphi(x)\dif\mu+\int_{B}\varphi(x)\dif\mu\leqslant\int_{A}f(x)\dif\mu+\int_{B}f(x)\dif\mu
	\end{equation*}
	由上确界的不等式性可得:
	\begin{equation*}
		\int_{A\cup B}f(x)\dif\mu\leqslant\int_{A}f(x)\dif\mu+\int_{B}f(x)\dif\mu
	\end{equation*}
	另一方面:
	\begin{equation*}
		\int_{A\cup B}f(x)\dif\mu\geqslant\int_{A\cup B}\varphi(x)\dif\mu=\int_{A}\varphi(x)\dif\mu+\int_{B}\varphi(x)\dif\mu
	\end{equation*}
	由上确界的不等式性又可得:
	\begin{equation*}
		\int_{A\cup B}f(x)\dif\mu\geqslant\int_{A}f(x)\dif\mu+\int_{B}f(x)\dif\mu
	\end{equation*}
	所以:
	\begin{equation*}
		\int_{A\cup B}f(x)\dif\mu=\int_{A}f(x)\dif\mu+\int_{B}f(x)\dif\mu
	\end{equation*}\par
	(6)令$A_1=\{f\leqslant g\}\cap A,\;A_2=\{f>g\}\cap A$,由\cref{prop:MeasurableFunction}(3)和\cref{prop:SigmaField}(2)可得$A_1,A_2\in\mathscr{F}$,同时有:
	\begin{equation*}
		A_1\cap A_2=\varnothing,\;A_1\cup A_2=A,\;\mu(A_2)=0
	\end{equation*}
	由(5)(3)可得:
	\begin{gather*}
		\int_{A}f(x)\dif\mu=\int_{A_1\cup A_2}f(x)\dif\mu=\int_{A_1}f(x)\dif\mu+\int_{A_2}f(x)\dif\mu=\int_{A_1}f(x)\dif\mu \\
		\int_{A}g(x)\dif\mu=\int_{A_1\cup A_2}g(x)\dif\mu=\int_{A_1}g(x)\dif\mu+\int_{A_2}g(x)\dif\mu=\int_{A_1}g(x)\dif\mu
	\end{gather*}
	对于满足$\varphi\leqslant f$的非负简单函数$\varphi(x)$,必然也有$\varphi\leqslant g$,于是由非负可测函数积分的定义可得:
	\begin{equation*}
		\int_{A_1}f(x)\dif\mu\leqslant\int_{A_1}g(x)\dif\mu
	\end{equation*}
	也即:
	\begin{equation*}
		\int_{A}f(x)\dif\mu\leqslant\int_{A}g(x)\dif\mu
	\end{equation*}\par
	(7)由(6)立即可得。\par
	(8)令$A_\infty=\{f=+\infty\}\cap A$。对任意的$n\in\mathbb{N}^+$,令:
	\begin{equation*}
		\varphi_n(x)=
		\begin{cases}
			n,&x\in A_\infty \\
			0,&x\in A_\infty^c
		\end{cases}
	\end{equation*}
	因为$f$是可测函数,由\cref{prop:MeasurableFunction}(2)和\cref{prop:SigmaField}(2)可得$A_\infty\in\mathscr{F}$,因此$\varphi_n(x)$是非负简单函数。由非负可测函数积分的定义可得:
	\begin{equation*}
		\int_{A}f(x)\dif\mu\geqslant\int_{A}\varphi_n(x)\dif\mu=n\mu(A_\infty)\geqslant0
	\end{equation*}
	所以:
	\begin{equation*}
		\forall\;n\in\mathbb{N}^+,\;0\leqslant \mu(A_\infty)\leqslant\frac{1}{n}\int_{A}f(x)\dif\mu
	\end{equation*}
	因为$\int_{A}f(x)\dif\mu<+\infty$,所以$\mu(A_\infty)=0$,即$f(x)$有限a.e.于$(A,A\cap\mathscr{F},\mu)$。\par
	(9)\textbf{必要性:}对任意的$n\in\mathbb{N}^+$,令:
	\begin{equation*}
		A_n=\left\{f\geqslant\frac{1}{n}\right\}\bigcap A,\quad
		\varphi_n(x)=
		\begin{cases}
			\dfrac{1}{n},&x\in A_n \\
			0,&x\in A_n^c
		\end{cases}
	\end{equation*}
	因为$f$是可测函数,由\cref{prop:MeasurableFunction}(1)和\cref{prop:SigmaField}(2)可得$A_n\in\mathscr{F}$,所以$\varphi_n(x)$是非负简单函数,于是:
	\begin{equation*}
		0=\int_{A}f(x)\dif\mu\geqslant\int_{A}\varphi_n(x)\dif\mu=\frac{1}{n}\mu(A_n)\geqslant0
	\end{equation*}
	所以对任意的$n\in\mathbb{N}^+,\;\mu(A_n)=0$。因为:
	\begin{equation*}
		\{f>0\}\cap A=\underset{n=1}{\overset{+\infty}{\cup}}A_n
	\end{equation*}
	由\cref{prop:Measure}(3)(次可列可加性)以及测度的非负性可得$\mu(\{f>0\}\cap A)=0$,即$f(x)=0\;$a.e.于$(A,A\cap\mathscr{F},\mu)$。\par
	\textbf{充分性:}函数$g(x)=0,\;\forall\;x\in A$在$A$上的积分为$0$,由(7)立即可证得充分性。\par
	(10)\textbf{数乘:}对任意的$\alpha\geqslant0$且$\alpha\in\mathbb{R}^{}$,根据\cref{prop:MeasurableFunction}(8)取非负简单函数列$\{f_n\}$满足$f_n\uparrow f$,于是有$\alpha f_n\uparrow \alpha f$,由\cref{prop:SimpleFunction}(2.a)可知$\alpha f_n$为非负简单函数。由(4)、\cref{prop:NonnegativeSimpleIntegral}(5)和\cref{prop:RSeq}(8.c)可得:
	\begin{align*}
		\int_{A}\alpha f(x)\dif\mu&=\lim_{n\to+\infty}\left[\int_{A}\alpha f_n(x)\dif\mu\right]=\lim_{n\to+\infty}\left[\alpha\int_{A}f_n(x)\dif\mu\right] \\
		&=\alpha\lim_{n\to+\infty}\left[\int_{A}f_n(x)\dif\mu\right]=\alpha\int_{A}f(x)\dif\mu
	\end{align*}\par
	当$\alpha=+\infty$时,分两种情况进行讨论。\par
	若$\int_{A}f(x)\dif\mu=0$,由(9)可得$f=0\;$a.e.于$(A,A\cap\mathscr{F},\mu)$,于是$\alpha f=0\;$a.e.于$(A,A\cap\mathscr{F},\mu)$。根据(9)可得:
	\begin{equation*}
		\int_{A}\alpha f(x)\dif\mu=0=\alpha\int_{A}f(x)\dif\mu
	\end{equation*}\par
	若$0<\int_{A}f(x)\dif\mu$,由(9)可得$f=0$不a.e.于$(A,A\cap\mathscr{F},\mu)$,于是$\mu(\{\alpha f=+\infty\})>0$,根据(8)可知$\int_{A}\alpha f(x)\dif\mu=+\infty$,所以有:
	\begin{equation*}
		\int_{A}\alpha f(x)\dif\mu=\alpha\int_{A}f(x)\dif\mu=+\infty
	\end{equation*}
	\textbf{请注意这里并非数值意义上的相等,而是定性的无穷。}\par
	\textbf{加法:}根据\cref{prop:MeasurableFunction}(8)取非负简单函数列$\{f_n\},\{g_n\}$满足$f_n\uparrow f,g_n\uparrow g$,于是有$f_n+g_n\uparrow$。由\cref{prop:RSeq}(8.b)可得:
	\begin{equation*}
		\lim_{n\to+\infty}(f_n+g_n)=\lim_{n\to+\infty}f_n+\lim_{n\to+\infty}g_n=f+g
	\end{equation*}
	所以$f_n+g_n\uparrow f+g$。由(4)、\cref{prop:NonnegativeSimpleIntegral}(5)和\cref{prop:RSeq}(8.b)可得:
	\begin{align*}
		\int_{A}[f(x)+g(x)]\dif\mu
		&=\lim_{n\to+\infty}\left\{\int_{A}[f_n(x)+g_n(x)]\dif\mu\right\} \\
		&=\lim_{n\to+\infty}\left[\int_{A}f_n(x)\dif\mu+\int_{A}g_n(x)\dif\mu\right] \\
		&=\lim_{n\to+\infty}\left[\int_{A}f_n(x)\dif\mu\right]+\lim_{n\to+\infty}\left[\int_{A}g_n(x)\dif\mu\right] \\
		&=\int_{A}f(x)\dif\mu+\int_{A}g(x)\dif\mu\qedhere
	\end{align*}
\end{proof}

\subsection{一般可测函数的积分}
\begin{definition}
	设$f$是测度空间$(X,\mathscr{F},\mu)$上的可测函数,$A\in\mathscr{F}$。若$\int_{A}f^+(x)\dif\mu$和$\int_{A}f^-(x)\dif\mu$中至少一个有限,则称$f$在$A$上\textbf{积分存在},将$f$在$A$上的积分定义为:
	\begin{equation*}
		\int_{A}f(x)\dif\mu=\int_{A}f^+(x)\dif\mu-\int_{A}f^-(x)\dif\mu
	\end{equation*}
	若$\int_{A}f^+(x)\dif\mu$和$\int_{A}f^-(x)\dif\mu$都有限,则称$f(x)$在$A$上\textbf{可积}。
\end{definition}
\begin{property}\label{prop:MeasurableIntegral}
	设$f$和$g$为测度空间$(X,\mathscr{F},\mu)$上的可测函数,则:
	\begin{enumerate}
		\item 若$A\in\mathscr{F},\;\mu(A)=0$,则任何可测函数$f$都在$A$上可积,并且有$\int_{A}f(x)\dif\mu=0$;
		\item 若$f$在$A\in\mathscr{F}$上积分存在,则$|\int_{A}f(x)\dif\mu|\leqslant\int_{A}|f(x)|\dif\mu$;
		\item 若$f$在$A\in\mathscr{F}$上积分存在(可积),则$f$在$A$的满足$B\in\mathscr{F}$的子集$B$上也积分存在(可积);
		\item $f$在$A\in\mathscr{F}$上可积的充分必要条件为$|f|$在$A$上可积;
		\item 若$f$在$A\in\mathscr{F}$上可积,则$|f|<+\infty\;$a.e.于$(A,A\cap\mathscr{F},\mu)$;
		\item 若$f,g$在$A\in\mathscr{F}$上积分存在,则对$\forall\;\alpha\in\mathbb{R}$, $af$的积分存在且:
		\begin{equation*}
			\int_{A}\alpha f(x)\dif\mu=\alpha\int_{A}f(x)\dif\mu
		\end{equation*}
		若$\int_{A}f(x)\dif\mu+\int_{A}g(x)\dif\mu$有意义,则$f+g\;$a.e.有定义,其积分存在且:
		\begin{equation*}
			\int_{A}[f(x)+g(x)]\dif\mu=\int_{A}f(x)\dif\mu+\int_{A}g(x)\dif\mu
		\end{equation*}
		\item 若$f,g$在$A\in\mathscr{F}$上积分存在且$f\leqslant g\;$a.e.于$(A,A\cap\mathscr{F},\mu)$,$A\in\mathscr{F}$,则:
		\begin{equation*}
			\int_{A}f(x)\dif\mu\leqslant\int_{A}g(x)\dif\mu
		\end{equation*}
		\item 若$f=g\;$a.e.于$(A,A\cap\mathscr{F},\mu)$,$A\in\mathscr{F}$,则只要其中任意一个的积分存在,另一个的积分也存在并且有:
		\begin{equation*}
			\int_{A}f(x)\dif\mu=\int_{A}g(x)\dif\mu
		\end{equation*}
		\item $A\in\mathscr{F}$,若$f=0\;$a.e.于$(A,A\cap\mathscr{F},\mu)$,则$\int_{A}f(x)\dif\mu=0$;若$\int_{A}f(x)\dif\mu=0$且$f\geqslant0\;$a.e.于$(A,A\cap\mathscr{F},\mu)$或$f\leqslant0\;$a.e.于$(A,A\cap\mathscr{F},\mu)$,则$f=0\;$a.e.于$(A,A\cap\mathscr{F},\mu)$;
		\item 若$f,g$都在$A\in\mathscr{F}$上可积且对任意的$E\in A\cap\mathscr{F}$有$\int_{E}f(x)\dif\mu\leqslant\int_{E}g(x)\dif\mu$,则$f\leqslant g\;$a.e.于$(A,A\cap\mathscr{F},\mu)$;
		\item 若$f,g$都在$A\in\mathscr{F}$上可积且对任意的$E\in A\cap\mathscr{F}$有$\int_{E}f(x)\dif\mu=\int_{E}g(x)\dif\mu$,则$f=g\;$a.e.于$(A,A\cap\mathscr{F},\mu)$;
		\item 若$\mu(A)>0$且$f>0\;$a.e.于$A$,则$\int_{A}f(x)\dif\mu>0$;若$\mu(A)>0$且$f<0\;$a.e.于$A$,则$\int_{A}f(x)\dif\mu<0$。\info{还得推广}
	\end{enumerate}
\end{property}
\begin{proof}
	(1)任选$A$上的一个可测函数$f(x)$。因为$\mu(A)=0$,由\cref{prop:NonnegativeMeasurableIntegral}(3)可知:
	\begin{equation*}
		\int_{A}f^+(x)\dif\mu=\int_{A}f^-(x)\dif\mu=0
	\end{equation*}
	于是:
	\begin{equation*}
		\int_{A}f(x)\dif\mu=\int_{A}f^+(x)\dif\mu-\int_{A}f^-(x)\dif\mu=0
	\end{equation*}\par
	(2)由\cref{prop:NonnegativeMeasurableIntegral}(2)(10)可得:
	\begin{align*}
		&\left|\int_{A}f(x)\dif\mu\right|=\left|\int_{A}f^+(x)\dif\mu-\int_{A}f^-(x)\dif\mu\right| \\
		\leqslant&\int_{A}f^+(x)\dif\mu+\int_{A}f^-(x)\dif\mu=\int_{A}[f^+(x)+f^-(x)]\dif\mu \\
		=&\int_{A}|f(x)|\dif\mu
	\end{align*}\par
	(3)任取$B\subseteq A$且$B\in\mathscr{F}$。因为$f$在$A$上积分存在,所以$\int_{A}f^+(x)\dif\mu$和$\int_{A}f^-(x)\dif\mu$至少有一个有限。设$\int_{A}f^+(x)\dif\mu$有限,另一种情况可对称讨论。由\cref{prop:NonnegativeMeasurableIntegral}(5)、\cref{prop:SigmaField}(4)和\cref{prop:NonnegativeMeasurableIntegral}(2)可得:
	\begin{equation*}
		+\infty>\int_{A}f^+(x)\dif\mu=\int_{B}f^+(x)\dif\mu+\int_{A\setminus B}f^+(x)\dif\mu\geqslant\int_{B}f^+(x)\dif\mu
	\end{equation*}
	故$f(x)$在$B$上积分存在。由$B$的任意性,命题成立。\par
	(4)\textbf{必要性:}由\cref{prop:NonnegativeMeasurableIntegral}(10)可得:
	\begin{align*}
		f\text{可积}&\Rightarrow\int_{A}f^+(x)\dif\mu,\int_{A}f^-(x)\dif\mu\in\mathbb{R} \\
		&\Rightarrow\int_{A}f^+(x)\dif\mu+\int_{A}f^-(x)\dif\mu\in\mathbb{R} \\
		&\Rightarrow\int_{A}[f^+(x)+f^-(x)]\dif\mu=\int_{A}|f(x)|\dif\mu\in\mathbb{R}
	\end{align*}\par
	\textbf{充分性:}由\cref{prop:NonnegativeMeasurableIntegral}(6)可得:
	\begin{gather*}
				\int_{A}f^+(x)\dif\mu\leqslant\int_{A}[f^+(x)+f^-(x)]\dif\mu=\int_{A}|f(x)|\dif\mu \\
				\int_{A}f^-(x)\dif\mu\leqslant\int_{A}[f^+(x)+f^-(x)]\dif\mu=\int_{A}|f(x)|\dif\mu
	\end{gather*}\par
	(5)由(4)和\cref{prop:NonnegativeMeasurableIntegral}(8)即可得到。\par
	(6)\textbf{数乘:}对实数$\alpha\geqslant0$,因为$f$在$A$上积分存在,由\cref{prop:NonnegativeMeasurableIntegral}(10)可得:
	\begin{equation*}
		\int_{A}\alpha f^+(x)\dif\mu=\alpha\int_{A}f^+(x)\dif\mu,\quad\int_{A}\alpha f^-(x)\dif\mu=\alpha\int_{A}f^-(x)\dif\mu
	\end{equation*}
	二式中至少有一个有限。根据$(\alpha f)^+=\alpha f^+,(\alpha f)^-=\alpha f^-$可知$\alpha f$积分存在且:
	\begin{align*}
		&\int_{A}\alpha f(x)\dif\mu=\int_{A}\alpha f^+(x)\dif\mu-\int_{A}\alpha f^-(x)\dif\mu \\
		=&\alpha\left[\int_{A}f^+(x)\dif\mu-\int_{A}f^-(x)\dif\mu\right]=\alpha\int_{A}f(x)\dif\mu
	\end{align*}
	当实数$\alpha<0$时,注意到$\alpha f=(-\alpha)f^--(-\alpha)f^+$,与$\alpha\geqslant0$时的情况同理可得:
	\begin{align*}
		&\int_{A}\alpha f(x)\dif\mu=\int_{A}(-\alpha)f^-(x)\dif\mu-\int_{A}(-\alpha)f^+(x)\dif\mu \\
		=&-\alpha\left[\int_{A}f^-(x)\dif\mu-\int_{A}f^+(x)\dif\mu\right]=\alpha\int_{A}f(x)\dif\mu
	\end{align*}\par
	\textbf{加法:}\info{未完成}\par
	(7)因为$f\leqslant g\;$a.e.于$(A,A\cap\mathscr{F},\mu)$,由\cref{prop:Measure}(3)可得$f^+\leqslant g^+\;$a.e.于$(A,A\cap\mathscr{F},\mu)$,$f^-\geqslant g^-\;$a.e.于$(A,A\cap\mathscr{F},\mu)$。由\cref{prop:NonnegativeMeasurableIntegral}(6)可得:
	\begin{equation*}
		\int_{A}f(x)\dif\mu=\int_{A}f^+(x)\dif\mu-\int_{A}f^-(x)\dif\mu\leqslant\int_{A}g^+(x)\dif\mu-\int_{A}g^-(x)\dif\mu=\int_{A}g(x)\dif\mu
	\end{equation*}\par
	(8)仅对$f$积分存在的情形进行讨论,$g$积分存在时可对称得到结论。又可只对$\int_{A}f^+(x)\dif\mu<+\infty$时的情况进行讨论,$\int_{A}f^-(x)\dif\mu$时的情况可类似得到。因为$f=g\;$a.e.于$(A,A\cap\mathscr{F},\mu)$,由\cref{prop:Measure}(3)(单调性)可得$f^+=g^+\;$a.e.于$(A,A\cap\mathscr{F},\mu)$且$f^-=g^-\;$a.e.于$(A,A\cap\mathscr{F},\mu)$。根据\cref{prop:NonnegativeMeasurableIntegral}(7)可得:
	\begin{equation*}
		\int_{A}f^+(x)\dif\mu=\int_{A}g^+(x)\dif\mu<+\infty,\quad\int_{A}f^-(x)\dif\mu=\int_{A}g^-(x)\dif\mu
	\end{equation*}
	所以$g$在$A$上的积分存在,并且有:
	\begin{equation*}
		\int_{A}f(x)\dif\mu=\int_{A}f^+(x)\dif\mu-\int_{A}f^-(x)\dif\mu=\int_{A}g^+(x)\dif\mu-\int_{A}g^-(x)\dif\mu=\int_{A}g(x)\dif\mu
	\end{equation*}\par
	(9)第一个结论由\cref{prop:Measure}(3)(单调性)和\cref{prop:NonnegativeMeasurableIntegral}(9)可知成立,下证第二个结论。只需证明$f\geqslant0\;$a.e.于$A$时的情况,对于$f\leqslant0\;$a.e.于$A$,只需取$-f$并由(6)即可得出结论。\par 
	由\cref{prop:MeasurableFunction}(9)和(8)可知,可以改变$f$在零测集$\{f<0\}\cap A$上的值使得得到的函数$f'$为非负可测函数且满足$\int_{A}f'(x)\dif\mu=0$,由\cref{prop:NonnegativeMeasurableIntegral}(9)可知$f'=0\;$a.e.于$(A,A\cap\mathscr{F},\mu)$,于是根据\cref{prop:SetOperation}(4)和\cref{prop:Measure}(1)可得:
	\begin{equation*}
		\mu(\{f\ne0\}\cap A)=\mu[(\{f'\ne0\}\cup\{f<0\})\cap A]=\mu(\{f'\ne0\}\cap A)+\mu(\{f<0\}\cap A)=0
	\end{equation*}
	于是$f=0\;$a.e.于$(A,A\cap\mathscr{F},\mu)$。\par
	(10)对任意的$E\in A\cap\mathscr{F}$取$\{f>g\}\cap E$,因为$f,g$都是可测函数,由\cref{prop:MeasurableFunction}(3)和\cref{prop:SigmaField}(2)可知$\{f>g\}\cap E\in\mathscr{F}$。根据(3)可知:
	\begin{equation*}
		\int_{\{f>g\}\cap E}f(x)\dif\mu,\int_{\{f>g\}\cap E}g(x)\dif\mu\in\mathbb{R}
	\end{equation*}
	于是由(6)和条件有:
	\begin{equation*}
		\int_{\{f>g\}\cap E}f(x)\dif\mu-\int_{\{f>g\}\cap E}g(x)\dif\mu=\int_{\{f>g\}\cap E}[f(x)-g(x)]\dif\mu\leqslant0
	\end{equation*}
	根据(7)可得:
	\begin{equation*}
		\int_{\{f>g\}\cap E}[f(x)-g(x)]\dif\mu\geqslant0
	\end{equation*}
	所以有:
	\begin{equation*}
		\int_{\{f>g\}\cap E}[f(x)-g(x)]\dif\mu=0
	\end{equation*}
	取$E=A$,由(9)可得$\mu(\{f>g\}\cap A)=0$,即$f\leqslant g\;$a.e.于$(A,A\cap \mathscr{F},\mu)$。\par
	(11)由(10)立即可得。\par
	(12)因为$f$是一个可测函数,由\cref{prop:MeasurableFunction}(1)可知$\{f\leqslant0\}\in\mathscr{F}$,所以$\{f\leqslant0\}\in A\cap\mathscr{F}$。因为$f>0\;$a.e.于$A$。所以$\mu(\{f\leqslant0\})=0$。由\cref{prop:MeasurableFunction}(9),更改$f$在零测集$\{f\leqslant0\}$上的值从而得到的非负可测函数$f'$。根据\cref{prop:MeasurableFunction}(1)可知对任意的$n\in\mathbb{N}^+$有:
	\begin{equation*}
		\left\{f'>\frac{1}{n}\right\}\in\mathscr{F}\Rightarrow\left\{f'>\frac{1}{n}\right\}\in A\cap\mathscr{F}
	\end{equation*}
	由\cref{prop:Measure}(3)(次可列可加性)可得:
	\begin{equation*}
		\mu(A)=\mu(\{f'>0\})=\mu\left[\underset{n=1}{\overset{+\infty}{\cup}}\left(\left\{f'>\frac{1}{n}\right\}\right)\right]\leqslant\sum_{n=1}^{+\infty}\mu\left(\left\{f'>\frac{1}{n}\right\}\right)
	\end{equation*}
	因为$\mu(A)>0$,则必然存在一个$n$使得:
	\begin{equation*}
		\mu\left(\left\{f'>\frac{1}{n}\right\}\right)>0
	\end{equation*}
	由(8)(7)、\cref{prop:SimpleFunction}(3)(1)、\cref{prop:MeasurableFunction}(5.b)、(6)和非负简单函数积分的定义可知:
	\begin{align*}
		\int_{A}f(x)\dif\mu&=\int_{A}f'(x)\dif\mu\geqslant\int_{A}f'(x)I_{\left\{f'>\frac{1}{n}\right\}}(x)\dif\mu \\
		&=\int_{A}\frac{1}{n}I_{\left\{f'>\frac{1}{n}\right\}}(x)\dif\mu=\frac{1}{n}\mu\left(\left\{f'>\frac{1}{n}\right\}\right)>0
	\end{align*}
	同理可得$f<0\;$a.e.于$A$时的情况。
\end{proof}
\begin{theorem}[Levi theorem]\label{theo:LeviTheorem}
	设$f(x),\{f_n\}$是测度空间$(X,\mathscr{F},\mu)$上的可测函数,$A\in\mathscr{F}$。若$f,\{f_n\}\;$非负a.e.于$(A,A\cap\mathscr{F},\mu)$,且$f_n\uparrow f\;$a.e.于$(A,A\cap\mathscr{F},\mu)$,则:
	\begin{equation*}
		\int_{A}f_n(x)\dif\mu\Big\uparrow\int_{A}f(x)\dif\mu
	\end{equation*}
\end{theorem}
\begin{proof}
	由\cref{prop:MeasurableFunction}(9)、\cref{prop:MeasurableIntegral}(8)和\cref{prop:Measure}(3)(次有限可加性),可仅对$f,\{f_n\}$是非负可测函数且$f_n\uparrow f$讨论。对每个$n\in\mathbb{N}^+$作非负简单函数列$\{f_{nm}\}$使得$f_{nm}\uparrow f_n$,令$g_k(x)=\max\limits_{1\leqslant n\leqslant k}f_{nk}(x)$。\par
	由\cref{prop:SimpleFunction}(2.e)可知$g_k$是非负简单函数。因为$\{f_{nm}\}\uparrow$,所以:
	\begin{equation*}
		g_k=\max_{1\leqslant n\leqslant k}f_{nk}\leqslant\max_{1\leqslant n\leqslant k}f_{n(k+1)}\leqslant\max_{1\leqslant n\leqslant k+1}f_{n(k+1)}=g_{k+1}
	\end{equation*}
	所以$g_k\uparrow$。因为
	\begin{equation*}
		f_{nk}\leqslant g_k=\max_{1\leqslant n\leqslant k}f_{nk}\leqslant\max_{1\leqslant n\leqslant k}f_n=f_k
	\end{equation*}
	由\cref{prop:RSeq}(6)可得:
	\begin{equation*}
		\lim_{k\to+\infty}f_{nk}=f_n\leqslant\lim_{k\to+\infty}g_k\leqslant\lim_{k\to+\infty}f_k=f,\quad\lim_{n\to+\infty}f_n=f\leqslant\lim_{k\to+\infty}g_k
	\end{equation*}
	所以:
	\begin{equation*}
		\lim_{k\to+\infty}g_k=f
	\end{equation*}
	由\cref{prop:NonnegativeMeasurableIntegral}(6)(4)可得:
	\begin{equation*}
		\int_{A}g_k(x)\dif\mu\Big\uparrow\int_{A}f(x)\dif\mu
	\end{equation*}
	又因为$g_k\leqslant f_k\leqslant f$且$f_k\uparrow$,由\cref{prop:NonnegativeMeasurableIntegral}(6)可得:
	\begin{equation*}
		\int_{A}g_k(x)\dif\mu\leqslant\int_{A}f_k(x)\dif\mu\leqslant\int_{A}f_{k+1}(x)\dif\mu\leqslant\int_{A}f(x)\dif\mu
	\end{equation*}
	根据\cref{prop:RSeq}(4)可知:
	\begin{equation*}
		\int_{A}f_n(x)\dif\mu\Big\uparrow\int_{A}f(x)\dif\mu\qedhere
	\end{equation*}
\end{proof}
\begin{theorem}[Fatou Lemma]\label{theo:FatouLemma}
	设$\{f_n\}$是测度空间$(X,\mathscr{F},\mu)$上的可测函数列,$A\in\mathscr{F}$,$\{f_n\}$非负a.e.于$(A,A\cap\mathscr{F},\mu)$,则:
	\begin{equation*}
		\int_{A}\left[\varliminf_{n\to+\infty}f_n(x)\right]\dif\mu\leqslant\varliminf_{n\to+\infty}\left[\int_{A}f_n(x)\dif\mu\right]
	\end{equation*}
\end{theorem}
\begin{proof}
	令$g_k(x)=\inf\limits_{n\geqslant k}f_n(x)$,则$g_k\uparrow\varliminf\limits_{n\to+\infty}f_n$。因为$\{f_n\}$是可测函数列,由\cref{prop:MeasurableFunction}(6)可知$\{g_k\}$是可测函数列,$\varliminf\limits_{n\to+\infty}f_n$是可测函数。因为$\{f_n\}$非负a.e.于$(A,A\cap\mathscr{F},\mu)$,根据\cref{prop:Measure}(3)(次可列可加性)和测度的非负性可得:
	\begin{equation*}
		0\leqslant\mu\left[\underset{n=1}{\overset{+\infty}{\cup}}(\{f_n<0\}\cap A)\right]\leqslant\sum_{n=1}^{+\infty}\mu(\{f_n<0\}\cap A)=0
	\end{equation*}
	即:
	\begin{equation*}
		\mu\left[\underset{n=1}{\overset{+\infty}{\cup}}(\{f_n<0\}\cap A)\right]=0
	\end{equation*}
	根据\cref{prop:Measure}(3)(单调性)可知$\{g_k\}$非负a.e.于$(A,A\cap\mathscr{F},\mu)$,于是由\cref{prop:RSeq}(6)可得$\varliminf\limits_{n\to+\infty}f_n\;$非负a.e.于$(A,A\cap\mathscr{F},\mu)$,所以由\cref{theo:LeviTheorem}可得:
	\begin{align*}
		\left[\int_{A}g_k(x)\dif\mu\right]\Big\uparrow\int_{A}\left[\varliminf_{n\to+\infty}f_n(x)\right]\dif\mu
	\end{align*}
	因为:
	\begin{equation*}
		g_k(x)\leqslant f_n(x),\;\forall\;n\geqslant k
	\end{equation*}
	由\cref{prop:MeasurableIntegral}(7)可得:
	\begin{equation*}
		\int_{A}g_k(x)\dif\mu\leqslant\int_{A}f_n(x)\dif\mu,\;\forall\;n\geqslant k
	\end{equation*}
	所以:
	\begin{equation*}
		\int_{A}g_k(x)\dif\mu\leqslant\inf_{n\geqslant k}\left[\int_{A}f_n(x)\dif\mu\right]
	\end{equation*}
	由\cref{prop:RSeq}(6)可得:
	\begin{equation*}
		\lim_{k\to+\infty}\left[\int_{A}g_k(x)\dif\mu\right]\leqslant\lim_{k\to+\infty}\left\{\inf_{n\geqslant k}\left[\int_{A}f_n(x)\dif\mu\right]\right\}=\varliminf_{n\to+\infty}\left[\int_{A}f_n(x)\dif\mu\right]
	\end{equation*}
	即:
	\begin{equation*}
		\int_{A}\left[\varliminf_{n\to+\infty}f_n(x)\right]\dif\mu\leqslant\varliminf_{n\to+\infty}\left[\int_{A}f_n(x)\dif\mu\right]\qedhere
	\end{equation*}
\end{proof}
\begin{corollary}\label{cor:FatouLemma}
	设$\{f_n\}$是测度空间$(X,\mathscr{F},\mu)$上的可测函数列,$A\in\mathscr{F}$。
	\begin{enumerate}
		\item 若存在上述测度空间上的在$A$上可积的函数$g$使得$f_n\geqslant g\;$a.e.于$(A,A\cap\mathscr{F},\mu)$对任意的$n\in\mathbb{N}^+$成立,则$\varliminf\limits_{n\to+\infty}f_n(x)$的积分存在且:
		\begin{equation*}
			\int_{A}\left[\varliminf_{n\to+\infty}f_n(x)\right]\dif\mu\leqslant\varliminf_{n\to+\infty}\left[\int_{A}f_n(x)\dif\mu\right]
		\end{equation*}
		\item 若存在上述测度空间上的在$A$上可积的函数$g$使得$f_n\leqslant g\;$a.e.于$(A,A\cap\mathscr{F},\mu)$对任意的$n\in\mathbb{N}^+$成立,则$\varlimsup\limits_{n\to+\infty}f_n(x)$的积分存在且:
		\begin{equation*}
			\int_{A}\left[\varlimsup_{n\to+\infty}f_n(x)\right]\dif\mu\geqslant\varlimsup_{n\to+\infty}\left[\int_{A}f_n(x)\dif\mu\right]
		\end{equation*}
	\end{enumerate}
\end{corollary}
\begin{proof}
	(1)构造a.e.非负的可测函数列$\{h_n=f_n-g\}$(\cref{prop:MeasurableFunction}(5.a)(9)和\cref{prop:MeasurableIntegral}(5)),由\cref{prop:MeasurableFunction}(9)和\cref{prop:MeasurableIntegral}(8)可将$\{h_n\}$就看做非负可测函数。根据\cref{theo:FatouLemma}可得:
	\begin{equation*}
		\int_{A}\left[\varliminf_{n\to+\infty}h_n(x)\right]\dif\mu\leqslant\varliminf_{n\to+\infty}\left[\int_{A}h_n(x)\dif\mu\right]
	\end{equation*}
	由下极限的性质可得:
	\begin{equation*}
		\varliminf_{n\to+\infty}f_n(x)=\varliminf_{n\to+\infty}[h_n(x)+g(x)]=\varliminf_{n\to+\infty}h_n(x)+g(x)
	\end{equation*}
	同时由\cref{prop:MeasurableFunction}(6)可得$\varliminf\limits_{n\to+\infty}h_n(x)$是一个非负可测函数。因为$g$可积,由\cref{prop:MeasurableIntegral}(6)可得:
	\begin{equation*}
		\int_{A}\left[\varliminf_{n\to+\infty}f_n(x)\right]\dif\mu=\int_{A}\left[\varliminf_{n\to+\infty}h_n(x)+g(x)\right]\dif\mu=\int_{A}\left[\varliminf_{n\to+\infty}h_n(x)\right]\dif\mu+\int_{A}g(x)\dif\mu
	\end{equation*}
	根据\cref{prop:MeasurableIntegral}(6)和\info{下极限的线性性质}可得:
	\begin{align*}
		\varliminf_{n\to+\infty}\left[\int_{A}f(x)\dif\mu\right]&=\varliminf_{n\to+\infty}\left\{\int_{A}[h_n(x)+g(x)]\dif\mu\right\} \\
		&=\varliminf_{n\to+\infty}\left[\int_{A}h_n(x)\dif\mu+\int_{A}g(x)\dif\mu\right] \\
		&=\varliminf_{n\to+\infty}\left[\int_{A}h_n(x)\dif\mu\right]+\int_{A}g(x)\dif\mu
	\end{align*}
	于是:
	\begin{gather*}
		\int_{A}\left[\varliminf_{n\to+\infty}h_n(x)\right]\dif\mu+\int_{A}g(x)\dif\mu\leqslant\varliminf_{n\to+\infty}\left[\int_{A}h_n(x)\dif\mu\right]+\int_{A}g(x)\dif\mu \\
		\int_{A}\left[\varliminf_{n\to+\infty}f_n(x)\right]\dif\mu\leqslant\varliminf_{n\to+\infty}\left[\int_{A}f_n(x)\dif\mu\right]
	\end{gather*}\par
	(2)构造a.e.非负的可测函数列$\{h_n=g-f_n\}$,与(1)的证明类似。
\end{proof}
\begin{theorem}[Lebesgue控制收敛定理]\label{theo:DominatedConvergenceTheorem}
	设$\{f_n\}$是测度空间$(X,\mathscr{F},\mu)$上的可测函数列。若存在$A\in\mathscr{F}$上的非负可积函数$g$使得对任意的$n\in\mathbb{N}^+$有$|f_n|\leqslant g\;$a.e.于$(A,A\cap\mathscr{F},\mu)$,则$f_n\overset{\text{a.e.}}{\longrightarrow}f$或$f_n\overset{\mu}{\longrightarrow}f$蕴含:
	\begin{equation*}
		\lim_{n\to+\infty}\left[\int_{A}f_n(x)\dif\mu\right]=\int_{A}f(x)\dif\mu
	\end{equation*}
\end{theorem}
\begin{proof}
	由\cref{prop:MeasurableIntegral}(7)(4)可知$f_n$在$A$上可积。\par
	\textbf{(1)$\;\text{a.e.}$}由\cref{prop:MeasurableIntegral}(8)、极限与上下极限的关系、\cref{cor:FatouLemma}(1)、上下极限的大小关系和\cref{cor:FatouLemma}(2)可得:
	\begin{align*}
		\int_{A}f(x)\dif\mu&=\int_{A}\left[\lim_{n\to+\infty}f_n(x)\right]\dif\mu=\int_{A}\left[\varliminf_{n\to+\infty}f_n(x)\right]\dif\mu \\
		&\leqslant\varliminf_{n\to+\infty}\left[\int_{A}f_n(x)\dif\mu\right]\leqslant\varlimsup_{n\to+\infty}\left[\int_{A}f_n(x)\dif\mu\right] \\
		&\leqslant\int_{A}\left[\varlimsup_{n\to+\infty}f_n(x)\right]\dif\mu=\int_{A}\left[\lim_{n\to+\infty}f_n(x)\right]\dif\mu \\
		&=\int_{A}f(x)\dif\mu
	\end{align*}
	于是有:
	\begin{equation*}
		\varliminf_{n\to+\infty}\left[\int_{A}f_n(x)\dif\mu\right]=\varlimsup_{n\to+\infty}\left[\int_{A}f_n(x)\dif\mu\right]=\int_{A}f(x)\dif\mu
	\end{equation*}
	由极限与上下极限的关系可得:
	\begin{equation*}
		\lim_{n\to+\infty}\left[\int_{A}f_n(x)\dif\mu\right]=\int_{A}f(x)\dif\mu
	\end{equation*}\par
	\textbf{(2)$\;\mu$}\info{未完成}
\end{proof}
\begin{corollary}[Lebesgue有界收敛定理]
	设$\{f_n\}$和$f$是测度空间$(X,\mathscr{F},\mu)$上的可测函数,$A\in\mathscr{F}$且$\mu(A)<+\infty$。若存在$M>0$使得对任意的$n\in\mathbb{N}^+$有$|f_n|\leqslant M\;$a.e.于$A$,则$f_n\overset{\text{a.e.}}{\longrightarrow}f$或$f_n\overset{\mu}{\longrightarrow}f$蕴含:
	\begin{equation*}
		\lim_{n\to+\infty}\left[\int_{A}f(x)\dif\mu\right]=\int_{A}f(x)\dif\mu
	\end{equation*}
\end{corollary}
\begin{proof}
	令$g\equiv M$,于是$g$在$A$上可积。由\cref{theo:DominatedConvergenceTheorem}直接可得结论。
\end{proof}
\begin{theorem}[积分的绝对连续性]
	设$f$是测度空间$(X,\mathscr{F},\mu)$上的可积函数,对于任意的$\varepsilon>0$,$\exists\;\delta>0$,使得对于任意的$A\in\mathscr{F}$,只要$\mu(A)<\delta$,就有:
	\begin{equation*}
		\int_{A}|f(x)|\dif\mu<\varepsilon
	\end{equation*}
\end{theorem}
\begin{proof}
	由$f$可积和\cref{prop:MeasurableIntegral}(4)可知$|f|$可积。由\cref{prop:MeasurableFunction}(8)和\cref{prop:NonnegativeMeasurableIntegral}(4)可知存在非负简单函数列$\{f_n\}$满足$f_n\uparrow |f|$且:
	\begin{equation*}
		\int_{A}|f(x)|\dif\mu=\lim_{n\to+\infty}\left[\int_{A}f_n(x)\dif\mu\right]
	\end{equation*}
	所以对于任意的$\varepsilon>0$,存在$N\in\mathbb{N}^+$使得:
	\begin{equation*}
		\int_{A}|f(x)|\dif\mu<\frac{\varepsilon}{2}+\int_{A}f_N(x)\dif\mu
	\end{equation*}
	取$f_N$在$A$上的最大值$M$,由非负简单函数积分的定义可得:
	\begin{equation*}
		\int_{A}|f(x)|\dif\mu<\frac{\varepsilon}{2}+M\mu(A)
	\end{equation*}
	所以对于这个$\varepsilon$而言,只要取$\delta<\dfrac{\varepsilon}{2M}$即可。
\end{proof}
\begin{theorem}\label{theo:MeasurableCountableIntegral}
	设$f$是测度空间$(X,\mathscr{F},\mu)$上的可测函数。若$f$在$A\in\mathscr{F}$上积分存在,则对任一$A$的可列可测分割或有限可测分割$\{A_n\}$有:
	\begin{equation*}
		\int_{A}f(x)\dif\mu=\sum_{n=1}^{+\infty}\int_{A_n}f(x)\dif\mu
	\end{equation*}
\end{theorem}
\begin{proof}
	构造函数列:
	\begin{equation*}
		f_n(x)=f(x)I\left(x\in\underset{i=1}{\overset{n}{\cup}}A_i\right)
	\end{equation*}
	则有:
	\begin{equation*}
		f_n^+\uparrow f^+,\;f_n^-\uparrow f^-
	\end{equation*}
	由\cref{prop:SigmaField}(3)、\cref{prop:SimpleFunction}(3)(1)和\cref{prop:MeasurableFunction}(5.b)可知$f_n$是可测函数,于是由\cref{prop:MeasurableFunction}(7)可得$f_n^+,f_n^-$是非负可测函数。根据\cref{theo:LeviTheorem}可知:
	\begin{equation*}
		\int_{A}f_n^+(x)\dif\mu\Big\uparrow\int_{A}f^+(x)\dif\mu,\quad\int_{A}f_n^-(x)\dif\mu\Big\uparrow\int_{A}f^-(x)\dif\mu
	\end{equation*}
	而由\cref{prop:SigmaField}(4)和\cref{prop:NonnegativeMeasurableIntegral}(5)(9)可得:
	\begin{gather*}
		\begin{aligned}
			\int_{A}f_n^+(x)\dif\mu
			&=\int_{\underset{i=1}{\overset{n}{\cup}}A_n}f_n^+(x)\dif\mu+\int_{A\setminus\underset{i=1}{\overset{n}{\cup}}A_n}f_n^+(x)\dif\mu \\
			&=\int_{\underset{i=1}{\overset{n}{\cup}}A_n}f^+(x)\dif\mu
			=\sum_{i=1}^{n}\int_{A_i}f^+(x)\dif\mu
		\end{aligned} \\
		\begin{aligned}
			\int_{A}f_n^-(x)\dif\mu
			&=\int_{\underset{i=1}{\overset{n}{\cup}}A_n}f_n^-(x)\dif\mu+\int_{A\setminus\underset{i=1}{\overset{n}{\cup}}A_n}f_n^-(x)\dif\mu \\
			&=\int_{\underset{i=1}{\overset{n}{\cup}}A_n}f^-(x)\dif\mu
			=\sum_{i=1}^{n}\int_{A_i}f^-(x)\dif\mu
		\end{aligned} 
	\end{gather*}
	因为$f$在$A$上积分存在,所以:
	\begin{equation*}
		\lim_{n\to+\infty}\left[\int_{A}f_n^+(x)\dif\mu\right],\quad\lim_{n\to+\infty}\left[\int_{A}f_n^-(x)\dif\mu\right]
	\end{equation*}
	中至少一个为有限值,根据\cref{prop:MeasurableIntegral}(3)可知$f$在$A_i$上的积分存在,于是由\cref{prop:RSeq}(8.b)可得:
	\begin{align*}
		\int_{A}f(x)\dif\mu&=\int_{A}f^+(x)\dif\mu-\int_{A}f^-(x)\dif\mu \\
		&=\lim_{n\to+\infty}\left[\int_{A}f_n^+(x)\dif\mu\right]-\lim_{n\to+\infty}\left[\int_{A}f_n^-(x)\dif\mu\right] \\
		&=\lim_{n\to+\infty}\left[\int_{A}f_n^+(x)\dif\mu-\int_{A}f_n^-(x)\dif\mu\right] \\
		&=\lim_{n\to+\infty}\left\{\sum_{i=1}^{n}\left[\int_{A_i}f^+(x)\dif\mu-\int_{A_i}f^-(x)\dif\mu\right]\right\} \\
		&=\lim_{n\to+\infty}\left\{\sum_{i=1}^{n}\left[\int_{A_i}f(x)\dif\mu\right]\right\}=\sum_{n=1}^{+\infty}\left[\int_{A_n}f(x)\dif\mu\right]
	\end{align*}
	有限可测分割的情况由$\mu(\varnothing)=0$、\cref{prop:MeasurableIntegral}(1)和可列可测分割的情况即可得到。
\end{proof}
\begin{theorem}\label{theo:IntBySubstitution}
	设$f$是由测度空间$(X,\mathscr{F},\mu)$到可测空间$(Y,\mathscr{C})$上的可测映射,对于任意的$A\in\mathscr{C}$,令$\nu(A)=\mu[f^{-1}(A)]$,由\cref{prop:MeasurableMapping}(4)可知$(Y,\mathscr{C},\nu)$是一个测度空间。对$(Y,\mathscr{C},\nu)$上的任何可测函数$g$和任意$A\in\mathscr{C}$,若$\int_Ag(y)\dif\nu$存在,则:
	\begin{equation*}
		\int_Ag(y)\dif\nu=\int_{f^{-1}(A)}g\circ f(x)\dif\mu
	\end{equation*}
\end{theorem}
\begin{proof}
	使用典型方法进行证明。\par
	\textbf{(1)非负简单函数:}取$(Y,\mathscr{C},\nu)$上的非负简单函数$g$:
	\begin{equation*}
		g(y)=\sum_{i=1}^{n}a_iI(y\in A_i),\quad a_i\geqslant0,\;A_i\in\mathscr{C},\;i=1,2,\dots,n,\;\underset{i=1}{\overset{n}{\cup}}A_i=Y
	\end{equation*}
	于是由\cref{theo:PropertyOfPreimage}(4)和\cref{prop:NonnegativeSimpleIntegral}(5)可得:
	\begin{align*}
		\int_{A}g(y)\dif\nu&=\sum_{i=1}^{n}a_i\nu(A_i\cap A)=\sum_{i=1}^{n}a_i\mu[f^{-1}(A_i\cap A)]=\sum_{i=1}^{n}a_i\mu[f^{-1}(A_i)\cap f^{-1}(A)] \\
		&=\sum_{i=1}^{n}a_i\int_{f^{-1}(A)}I[x\in f^{-1}(A_i)]\dif\mu=\sum_{i=1}^{n}a_i\int_{f^{-1}(A)}I[f(x)\in A_i]\dif\mu \\
		&=\int_{f^{-1}(A)}\sum_{i=1}^{n}a_iI[f(x)\in A_i]\dif\mu=\int_{f^{-1}(A)}g[f(x)]\dif\mu=\int_{f^{-1}(A)}g\circ f(x)\dif\mu
	\end{align*}\par
	\textbf{(2)非负可测函数:}取$(Y,\mathscr{C},\nu)$上的非负可测函数$g$,由\cref{prop:MeasurableFunction}(8)可知存在非负简单函数列$\{g_n\}$使得$g_n\uparrow g$。由非负简单函数时的情形可得:
	\begin{equation*}
		g_n\circ f(x)=g_n[f(x)]=\sum_{i=1}^{j_n}a_{ni}I[f(x)\in A_{ni}]=\sum_{i=1}^{j_n}a_{ni}I[x\in f^{-1}(A_{ni})],\quad\underset{i=1}{\overset{j_n}{\cup}}A_{ni}=Y
	\end{equation*}
	其中$a_{ni}>0,\;A_{ni}\in\mathscr{C}$。由可测函数的定义,$f^{-1}(A_{ni})\in\mathscr{F}$,所以$g_n\circ f$也是一个非负简单函数。因为$g_n\uparrow g$,所以有$g_n\circ f\uparrow g\circ f$,根据\cref{prop:MeasurableMapping}(2)可知$g\circ f$是非负可测函数,由\cref{prop:NonnegativeMeasurableIntegral}(4)和非负简单函数时的结论可得:
	\begin{align*}
		\int_{A}g(y)\dif\nu&=\int_{A}\left[\lim_{n\to+\infty}g_n(y)\right]\dif\nu
		=\lim_{n\to+\infty}\left[\int_{A}g_n(y)\dif\nu\right] \\
		&=\lim_{n\to+\infty}\left[\int_{f^{-1}(A)}g_n\circ f(x)\dif\mu\right]
		=\int_{f^{-1}(A)}g\circ f(x)\dif\mu
	\end{align*}\par
	\textbf{(3)一般可测函数:}取$(Y,\mathscr{C},\nu)$上的一般可测函数$g$,由非负可测函数时的情形可得:
	\begin{align*}
		\int_{A}g(y)\dif\nu&=\int_{A}g^+(y)\dif\nu-\int_{A}g^-(y)\dif\nu \\
		&=\int_{f^{-1}(A)}g^+\circ f(x)\dif\mu-\int_{f^{-1}(A)}g^-\circ f(x)\dif\mu \\
		&=\int_{f^{-1}(A)}(g\circ f)^+(x)\dif\mu-\int_{f^{-1}(A)}(g\circ f)^-(x)\dif\mu \\
		&=\int_{f^{-1}(A)}g\circ f(x)\dif\mu\qedhere
	\end{align*}
\end{proof}