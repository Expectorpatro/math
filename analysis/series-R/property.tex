\section{收敛级数的性质}

\subsection{收敛级数的可结合性}
\begin{theorem}
	设有收敛级数:
	\begin{equation*}
		a_1+a_2+\cdots+a_n+\cdots
	\end{equation*}
	如果把这个级数的若干个相继的项归并为一项,即将该级数变为如下形式:
	\begin{equation*}
		(a_1+a_2+\cdots+a_{n_1})+(a_{n_1+1}+a_{n_1+2}+\cdots+a_{n_2})+\cdots+(a_{n_k+1}+a_{n_k+2}+\cdots+a_{n_{k+1}})+\cdots
	\end{equation*}
	则结合后的级数仍然收敛,且与原级数有相等的和。
\end{theorem}
\begin{proof}
	结合后级数的部分和序列是原级数部分和序列的子列。
\end{proof}
如果原级数为定号级数(即正项级数或负项级数),逆命题也成立。若不定号,考虑级数:
\begin{equation*}
	(1-1)+(1-1)+\cdots+(1-1)+\cdots
\end{equation*}
这个级数当然是收敛的,但级数:
\begin{equation*}
	1-1+1-1+\cdots+(-1)^{n-1}+\cdots
\end{equation*}
显然还是发散的。

\subsection{绝对收敛级数的性质}
\subsubsection{可交换性}
设$\sum\limits_{n=1}^{+\infty}a_n$是一个级数,我们对其进行重排,即把该序列中的所有项无重复、无遗漏地改变一个顺序重新排出来。用符号表示即为:
\begin{equation*}
	\alpha^{'}_n=\alpha_{\varphi(n)}
\end{equation*}
其中$\varphi$是一个从$\mathbb{N}^+$到$\mathbb{N}^+$的双射。
\begin{theorem}
	若级数$\sum\limits_{n=1}^{+\infty}a_n$绝对收敛,则重排后的级数$\sum\limits_{n=1}^{+\infty}a^{'}_n$也绝对收敛,并且二者值相等。
\end{theorem}
\begin{proof}
	我们先来讨论正项级数,再来讨论任意项级数:\par
	(1)正项级数:\par
	设级数$\sum\limits_{n=1}^{+\infty}a_n$为绝对收敛的正项级数,则级数$\sum\limits_{n=1}^{+\infty}a^{'}_n$也是一个正项级数。由题目条件,显然可得:
	\begin{equation*}
		\sum_{n=1}^Na^{'}_n\leqslant\sum_{n=1}^{+\infty}a_n,\;\forall\;N\in\mathbb{N}^+
	\end{equation*}
	由正项级数的收敛原理,级数$\sum\limits_{n=1}^{+\infty}a^{'}_n$收敛。由极限的不等式性也有:
	\begin{equation*}
		\sum_{n=1}^{+\infty}a^{'}_n\leqslant\sum_{n=1}^{+\infty}a_n
	\end{equation*}
	反之,可认为级数$\sum\limits_{n=1}^{+\infty}a_n$是由级数$\sum\limits_{n=1}^{+\infty}a^{'}_n$重排后的结果,因此也可得到:
	\begin{equation*}
		\sum_{n=1}^{+\infty}a_n\leqslant\sum_{n=1}^{+\infty}a^{'}_n
	\end{equation*}
	于是:
	\begin{equation*}
		\sum_{n=1}^{+\infty}a_n=\sum_{n=1}^{+\infty}a^{'}_n
	\end{equation*}
	(2)任意项级数:
	设级数$\sum\limits_{n=1}^{+\infty}a_n$为绝对收敛的任意项级数。令:
	\begin{equation*}
		p_n=\frac{|a_n|+a_n}{2},\;q_n=\frac{|a_n|-a_n}{2},\;n\in\mathbb{N}^+
	\end{equation*}
	显然有:
	\begin{equation*}
		0\leqslant p_n\leqslant|a_n|,\;0\leqslant q_n\leqslant|a_n|,\;n\in\mathbb{N}^+
	\end{equation*}
	取比较级数$\sum\limits_{n=1}^{+\infty}|a_n|$,由正项级数的比较判别法,$\sum\limits_{n=1}^{+\infty}p_n$与$\sum\limits_{n=1}^{+\infty}q_n$都是正项收敛级数。由(1),任意重排后的级数$\sum\limits_{n=1}^{+\infty}p^{'}_n$和$\sum\limits_{n=1}^{+\infty}q^{'}_n$也都收敛,并且有:
	\begin{equation*}
		\sum_{n=1}^{+\infty}p^{'}_n=\sum_{n=1}^{+\infty}p_n,\quad
		\sum_{n=1}^{+\infty}q^{'}_n=\sum_{n=1}^{+\infty}q_n
	\end{equation*}
	因此级数:
	\begin{equation*}
		\sum_{n=1}^{+\infty}|a^{'}_n|=\sum_{n=1}^{+\infty}(p^{'}_n-q^{'}_n)
	\end{equation*}
	由级数的线性运算也收敛,即级数$\sum\limits_{n=1}^{+\infty}a^{'}_n$绝对收敛,并且有:
	\begin{align*}
		\sum_{n=1}^{+\infty}a^{'}_n
		&=\sum_{n=1}^{+\infty}(p^{'}_n-q^{'}_n) \\
		&=\sum_{n=1}^{+\infty}p^{'}_n-\sum_{n=1}^{+\infty}q^{'}_n \\
		&=\sum_{n=1}^{+\infty}p_n-\sum_{n=1}^{+\infty}q_n \\
		&=\sum_{n=1}^{+\infty}(p_n-q_n) \\
		&=\sum_{n=1}^{+\infty}a_n
	\end{align*}
	第一行到第二行利用级数的线性运算,第二行到第三行利用(1)的结果,第三行到第四行再次利用级数的线性运算。
\end{proof}


\subsubsection{条件收敛级数并不满足可交换性}
\begin{theorem}
	设$\sum\limits_{n=1}^{+\infty}a_n$是一个条件收敛级数,则对任意的$\xi\in\overline{\mathbb{R}}$,都存在$\sum\limits_{n=1}^{+\infty}a_n$的一个重排级数$\sum\limits_{n=1}^{+\infty}a^{'}_n$,使:
	\begin{equation*}
		\sum_{n=1}^{+\infty}a^{'}_n=\xi
	\end{equation*}
\end{theorem}
\begin{proof}
	(1)对于$\xi\in\mathbb{R}$:\par
	令:
	\begin{equation*}
		p_n=\frac{|a_n|+a_n}{2},\;q_n=\frac{|a_n|-a_n}{2},\;n\in\mathbb{N}^+
	\end{equation*}
	显然$\sum\limits_{n=1}^{+\infty}p_n$与$\sum\limits_{n=1}^{+\infty}q_n$都是正项级数,并且有:
	\begin{gather*}
		\lim_{n\to+\infty}p_n=\lim_{n\to+\infty}\frac{|a_n|+a_n}{2}=0 \\
		\lim_{n\to+\infty}q_n=\lim_{n\to+\infty}\frac{|a_n|-a_n}{2}=0 \\
		\sum_{n=1}^{+\infty}p_n=\sum_{n=1}^{+\infty}\frac{|a_n|+a_n}{2}=\frac{1}{2}\sum_{n=1}^{+\infty}|a_n|+\frac{1}{2}\sum_{n=1}^{+\infty}a_n=+\infty \\
		\sum_{n=1}^{+\infty}q_n=\sum_{n=1}^{+\infty}\frac{|a_n|-a_n}{2}=\frac{1}{2}\sum_{n=1}^{+\infty}|a_n|-\frac{1}{2}\sum_{n=1}^{+\infty}a_n=+\infty
	\end{gather*}
	前两个公式可由Cauchy收敛准则的必要性推得。接下来来考察序列:
	\begin{equation*}
		a_1,a_2,\cdots,a_n,\cdots
	\end{equation*}
	令$P_n$表示这个序列中第$n$个非负项,以$Q_n$表示其中第$n$个负项的绝对值。则$\{P_n\}$是$\{p_n\}$去除一部分值为$0$的项后剩下的子序列(去除的是$a_n\leqslant0$导致$p_n=0$的项),$\{Q_n\}$是$\{q_n\}$去除所有值为$0$的项后剩下的子序列(即$a_n\geqslant0$导致$q_n=0$的项)。由$\{p_n\}$和$\{q_n\}$作为实数序列与作为级数项的收敛性,可得到:
	\begin{gather*}
		\lim_{n\to+\infty}P_n=\lim_{n\to+\infty}Q_n=0 \\
		\sum_{n=1}^{+\infty}P_n=\sum_{n=1}^{+\infty}Q_n=+\infty
	\end{gather*}
	同时注意到,$\{P_n\}$与$\{-Q_n\}$的各项其实都是原本$\{a_n\}$中的某项。我们依次考察$P_1,P_2,\dots$中的各项,设$P_{m_1}$是第一个满足以下条件的项(存在性由$\sum\limits_{n=1}^{+\infty}P_n=+\infty$保证):
	\begin{equation*}
		P_1+P_2+\cdots+P_{m_1}>\xi
	\end{equation*}
	再依次考察$Q_1,Q_2,\dots$中的各项,设$Q_{n_1}$是第一个满足以下条件的项(存在性由$\sum\limits_{n=1}^{+\infty}Q_n=+\infty$保证):
	\begin{equation*}
		P_1+P_2+\cdots+P_{m_1}-Q_1-Q_2-\cdots-Q_{n_1}<\xi
	\end{equation*}
	再考虑考察$P_{m_1+1},P_{m_1+2},\dots$中的各项,设$P_{m_2}$是第一个满足以下条件的项:
	\begin{gather*}
		P_1+P_2+\cdots+P_{m_1}-Q_1-Q_2-\cdots-Q_{n_1} \\
		+P_{m_1+1}+P_{m_1+2}+\cdots+P_{m_2}>\xi
	\end{gather*}
	不断重复下去,我们可以得到$\sum\limits_{n=1}^{+\infty}a_n$的一个重排级数$\sum\limits_{n=1}^{+\infty}a^{'}_n$:
	\begin{gather*}
		P_1+P_2+\cdots+P_{m_1}-Q_1-Q_2-\cdots-Q_{n_1} \\
		+P_{m_1+1}+P_{m_1+2}+\cdots+P_{m_2}-Q_{n_1+1}-Q_{n_1+2}-\cdots-Q_{n_2} \\
		\vdots
	\end{gather*}
	令$R_k$和$L_k$分别表示级数$\sum\limits_{n=1}^{+\infty}a^{'}_n$末项为$P_{m_k}$的部分和与末项为$Q_{n_k}$的部分和,则由:
	\begin{gather*}
		|R_k-\xi|\leqslant P_{m_k},\;k=2,3,\dots, \\
		|L_k-\xi|\leqslant Q_{n_k},\;k=1,2,3,\dots,
	\end{gather*}
	而:
	\begin{equation*}
		\lim_{n\to+\infty}P_{m_k}=\lim_{n\to+\infty}Q_{n_k}=0
	\end{equation*}
	所以:
	\begin{equation*}
		\lim_{n\to+\infty}R_k=\lim_{n\to+\infty}L_k=\xi
	\end{equation*}
	因为级数$\sum\limits_{n=1}^{+\infty}a^{'}_n$的任意一个部分和$S^{'}_n$必定介于一对$R_k$和$L_k$之间(因为$P_{m_k}$和$Q_{n_k}$都是取的第一个满足条件的),且随着$n$的增大,可以让$k$也随之增大直至无穷。因此由夹逼定理:
	\begin{equation*}
		\sum_{n=1}^{+\infty}a^{'}_n=\xi
	\end{equation*}
	(2)对于$\xi$为正无穷或负无穷的情况,仅对正无穷情况进行讨论,负无穷类似:\par
	任取一个单调上升且趋于无穷的实数数列$\{\xi_n\}$。沿用(1)中的记号。
	设$P_{m_1}$是第一个满足以下条件的项(存在性由$\sum\limits_{n=1}^{+\infty}P_n=+\infty$保证):
	\begin{equation*}
		P_1+P_2+\cdots+P_{m_1}>\xi_1
	\end{equation*}
	再依次考察$Q_1,Q_2,\dots$中的各项,设$Q_{n_1}$是第一个满足以下条件的项(存在性由$\sum\limits_{n=1}^{+\infty}Q_n=+\infty$保证):
	\begin{equation*}
		P_1+P_2+\cdots+P_{m_1}-Q_1-Q_2-\cdots-Q_{n_1}<\xi_1
	\end{equation*}
	再考虑考察$P_{m_1+1},P_{m_1+2},\dots$中的各项,设$P_{m_2}$是第一个满足以下条件的项:
	\begin{gather*}
		P_1+P_2+\cdots+P_{m_1}-Q_1-Q_2-\cdots-Q_{n_1} \\
		+P_{m_1+1}+P_{m_1+2}+\cdots+P_{m_2}>\xi_2
	\end{gather*}
	不断重复下去,我们可以得到$\sum\limits_{n=1}^{+\infty}a_n$的一个重排级数$\sum\limits_{n=1}^{+\infty}a^{'}_n$:
	\begin{gather*}
		P_1+P_2+\cdots+P_{m_1}-Q_1-Q_2-\cdots-Q_{n_1} \\
		+P_{m_1+1}+P_{m_1+2}+\cdots+P_{m_2}-Q_{n_1+1}-Q_{n_1+2}-\cdots-Q_{n_2} \\
		\vdots
	\end{gather*}
	而这个重排级数显然满足:
	\begin{equation*}
		\sum_{n=1}^{+\infty}a^{'}_n=+\infty
	\end{equation*}
\end{proof}












