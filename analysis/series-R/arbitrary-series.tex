\section{任意项级数}

本节讨论\gls{ArbitrarySeries},即不对各项的正负性做出要求的级数。

\subsection{条件收敛的判别}
\subsubsection{Abel引理}
\begin{lemma}
	设$\alpha_i,\beta_i,\;i=1,2,\dots,p$是实数,且:
	\begin{equation*}
		B_k=\sum_{i=1}^k\beta_i,\;k=1,2,\dots,p
	\end{equation*}
	则有:
	\begin{enumerate}
		\item $\sum\limits_{i=1}^p\alpha_i\beta_i=\sum\limits_{i=1}^{p-1}(\alpha_i-\alpha_{i+1})B_i+\alpha_pB_p$
		\item 如果$\{\alpha_i\}$单调,并且有:
		\begin{equation*}
			|B_k|\leqslant L,\;k=1,2,\dots,p
		\end{equation*}
		那么就有:
		\begin{equation*}
			\left|\sum_{i=1}^p\alpha_i\beta_i\right|\leqslant L\left(|\alpha_1|+2|\alpha_p|\right)
		\end{equation*}
	\end{enumerate}
\end{lemma}
(1)这个公式又称为\gls{AbelTrans}、\gls{SumByParts}。之所以被称之为分部求和公式,是因为它可以写为如下形式:
\begin{equation*}
	\sum_{i=1}^p\alpha_i\Delta B_i = \alpha_jB_j\Big|_{j=0}^p-\sum_{i=1}^{p-1}B_i\Delta\alpha_i
\end{equation*}
其中:
\begin{gather*}
	a_0=0,\quad B_0=0 \\
	\Delta B_k=B_k-B_{k-1}=\beta_k,\quad k=1,2,\dots,p, \\
	\Delta\alpha_i=\alpha_{i+1}-\alpha_i,\quad i=1,2,\dots,p-1
\end{gather*}
\begin{proof}
	记$B_0=0$,于是有:\par
	(1)
	\begin{align*}
		\sum_{i=1}^p\alpha_i\beta_i
		&=\sum_{i=1}^p\alpha_i(B_i-B_{i-1}) \\
		&=\sum_{i=1}^p\alpha_iB_i-\sum_{i=1}^p\alpha_iB_{i-1} \\
		&=\sum_{i=1}^p\alpha_iB_i-\sum_{i=0}^{p-1}\alpha_{i+1}B_{i} \\
		&=\sum_{i=1}^p\alpha_iB_i-\sum_{i=1}^{p-1}\alpha_{i+1}B_{i} \\
		&=\sum_{i=1}^{p-1}(\alpha_i-\alpha_{i+1})B_i+\alpha_pB_p
	\end{align*}\par
	(2)下第三行到第四行利用了$\{\alpha_i\}$的单调性。
	\begin{align*}
		\left|\sum_{i=1}^p\alpha_i\beta_i\right|
		&=\left|\sum_{i=1}^{p-1}(\alpha_i-\alpha_{i+1})B_i+\alpha_pB_p\right| \\
		&\leqslant\sum_{i=1}^{p-1}|\alpha_i-\alpha_{i+1}|\;|B_i|+|\alpha_p|\;|B_p| \\
		&\leqslant L\left(\sum_{i=1}^{p-1}|\alpha_i-\alpha_{i+1}|+|\alpha_p|\right) \\
		&=L\left(|\alpha_1-\alpha_p|+|\alpha_p|\right) \\
		&\leqslant L\left(|\alpha_1|+2|\alpha_p|\right)\qedhere
	\end{align*}
\end{proof}

\subsubsection{Dirichlet判别法}
\begin{theorem}
	对于级数$\sum\limits_{n=1}^{+\infty}a_nb_n$,如果:
	\begin{enumerate}
		\item 序列$\{a_n\}$单调收敛于$0$;
		\item 序列$\{\sum\limits_{i=1}^nb_i\}$有界;
	\end{enumerate}
	那么级数收敛。
\end{theorem}
\begin{proof}
	我们来估计:
	\begin{equation*}
		\left|\sum_{i=n+1}^{n+p}a_ib_i\right|
	\end{equation*}
	令:
	\begin{equation*}
		B_k=\sum_{i=1}^{n+k}b_i
	\end{equation*}
	因为序列$\{\sum\limits_{i=1}^nb_i\}$有界,因此有:
	\begin{equation*}
		\left|\sum\limits_{i=1}^nb_i\right|\leqslant L,\;n\in\mathbb{N}^+
	\end{equation*}
	也就有:
	\begin{equation*}
		|B_k|=\left|\sum_{i=1}^{n+k}b_i-\sum_{i=1}^nb_i\right|\leqslant 2L
	\end{equation*}
	因为$\{\alpha_i\}$单调,于是:
	\begin{equation*}
		\left|\sum_{i=n+1}^{n+p}a_ib_i\right|\leqslant 2L(|\alpha_{n+1}|+2|\alpha_{n+p}|)
	\end{equation*}
	又因为$\{\alpha_i\}\rightarrow0$,显然对任意的$\varepsilon>0$,当$n$足够大时可以有:
	\begin{equation*}
		\left|\sum_{i=n+1}^{n+p}a_ib_i\right|<\varepsilon
	\end{equation*}
	由Cauchy收敛准则,级数$\sum\limits_{n=1}^{+\infty}a_nb_n$收敛。
\end{proof}

\subsubsection{Abel判别法}
\begin{theorem}
	对于级数$\sum\limits_{n=1}^{+\infty}a_nb_n$,如果:
	\begin{enumerate}
		\item 序列$\{a_n\}$单调有界;
		\item 级数$\sum\limits_{n=1}^{+\infty}b_n$收敛;
	\end{enumerate}
	那么级数收敛。
\end{theorem}
\begin{proof}
	因为$\{a_n\}$单调有界,由实数序列的单调收敛原理,可以设:
	\begin{equation*}
		\lim_{n\to+\infty}a_n=a
	\end{equation*}
	则序列$\{a_n-a\}$单调趋于$0$。又因级数$\sum\limits_{n=1}^{+\infty}b_n$收敛,所以序列$\{\sum\limits_{i=1}^nb_i\}$有界。由Dirichlet判别法,级数$\sum\limits_{n=1}^{+\infty}(a_n-a)b_n$收敛。又因为级数$\sum\limits_{n=1}^{+\infty}ab_n$收敛,而:
	\begin{equation*}
		\sum_{n=1}^{+\infty}a_nb_n=\sum_{n=1}^{+\infty}\left[(a_n-a)b_n+ab_n\right]=\sum_{n=1}^{+\infty}\left[(a_n-a)b_n\right]+\sum_{n=1}^{+\infty}ab_n
	\end{equation*}
	因此级数$\sum\limits_{n=1}^{+\infty}a_nb_n$收敛。
\end{proof}

\subsubsection{Leibniz判别法}
\begin{theorem}
	设序列$\{a_n\}$单调且收敛于$0$,则以下级数收敛:
	\begin{equation*}
		\sum_{n=1}^{+\infty}(-1)^{n-1}a_n
	\end{equation*}
\end{theorem}
\begin{proof}
	使用Dirichlet判别法可直接证得。
\end{proof}


























