\section{正项级数}

\begin{definition}
	如果级数$\sum\limits_{n=1}^{+\infty}a_n$的每一项都是非负数,则称该级数为\gls{PosSeries}。
\end{definition}

\subsection{收敛原理}
显然正项级数的部分和序列是单调递增序列。\info{记得把单调序列的收敛原理放在这里}由单调序列的收敛原理,有如下定理:
\begin{theorem}
	正项级数$\sum\limits_{n=1}^{+\infty}a_n$收敛的充分必要条件为它的部分和序列$\{S_n\}$有上界。
\end{theorem}

\subsection{比较判别法}
以下方法我们称之为\gls{ComparisonTest}。
\subsubsection{一般形式}
\begin{theorem}
	设$\sum\limits_{n=1}^{+\infty}a_n$和$\sum\limits_{n=1}^{+\infty}b_n$都是正项级数。
	\begin{enumerate}
		\item 若$\sum\limits_{n=1}^{+\infty}b_n$收敛,且$\exists\;N\in\mathbb{N}^+,\;\forall\;n>N,\;a_n\leqslant cb_n,\;c\in[0,+\infty)$,那么$\sum\limits_{n=1}^{+\infty}a_n$也收敛。
		\item 若$\sum\limits_{n=1}^{+\infty}b_n$发散,且$\exists\;N\in\mathbb{N}^+,\;\forall\;n>N,\;a_n\geqslant cb_n,\;c\in(0,+\infty]$,那么$\sum\limits_{n=1}^{+\infty}a_n$也发散。
	\end{enumerate}
\end{theorem}
\begin{proof}
	(1)若条件得到满足,则:
	\begin{align*}
		\sum_{n=1}^{+\infty}a_n&=\sum_{n=1}^Na_n+\sum_{n=N+1}^{+\infty}a_n \\
		&\leqslant\sum_{n=1}^Na_n+\sum_{n=N+1}^{+\infty}cb_n \\
		&\leqslant\sum_{n=1}^Na_n+\sum_{n=1}^{+\infty}cb_n \\
		&=\sum_{n=1}^Na_n+c\sum_{n=1}^{+\infty}b_n
	\end{align*}
	显然$\sum\limits_{n=1}^{+\infty}a_n$有上界,因此收敛。\par
	(2)若此时$a_n$收敛,则:
	\begin{equation*}
		\exists\;N\in\mathbb{N}^+,\;\forall\;n>N,\;b_n\leqslant\frac{1}{c}a_n,\;\frac{1}{c}\in[0,+\infty)
	\end{equation*}
	$\sum\limits_{n=1}^{+\infty}b_n$就应该收敛,矛盾。
\end{proof}
\subsubsection{极限形式}
\begin{theorem}
	设$\sum\limits_{n=1}^{+\infty}a_n$和$\sum\limits_{n=1}^{+\infty}b_n$都是正项级数。且有:
	\begin{equation*}
		\lim\frac{a_n}{b_n}=\gamma,\;0\leqslant\gamma\leqslant+\infty
	\end{equation*}
	\begin{enumerate}
		\item 若$\sum\limits_{n=1}^{+\infty}b_n$收敛且$\gamma<+\infty$,则$\sum\limits_{n=1}^{+\infty}a_n$收敛。
		\item 若$\sum\limits_{n=1}^{+\infty}b_n$发散且$\gamma>0$,则$\sum\limits_{n=1}^{+\infty}a_n$发散。
	\end{enumerate}
\end{theorem}
\begin{proof}
	(1)若条件得到满足,取$\varepsilon<+\infty$,则$\exists\;N\in\mathbb{N}^+,\;\forall\;n>N,a_n<(\gamma+\varepsilon)b_n$。\par
	(2)若条件得到满足,取$\varepsilon<\frac{\gamma}{2}$,则$\exists\;N\in\mathbb{N}^+,\;\forall\;n>N,a_n>(\gamma-\varepsilon)b_n$。
\end{proof}

\subsection{Cauchy根式判别法}
以下定理我们称之为Cauchy's\gls{RootTest}。该判别法的实质其实就是将$\sum\limits_{n=1}^{+\infty}a_n$与$\sum\limits_{n=1}^{+\infty}r^n$作比较。当$r<1$时,由收敛准则易证$\sum\limits_{n=1}^{+\infty}r^n$收敛。
\subsubsection{一般形式}
\begin{theorem}
	设$\sum\limits_{n=1}^{+\infty}a_n$是正项级数。
	\begin{enumerate}
		\item 若$\exists\;r<1,\;\exists\;N\in\mathbb{N}^+,\;\forall\;n>N,\;\sqrt[n]{a_n}<r$,则$\sum\limits_{n=1}^{+\infty}a_n$收敛\footnote{思考能否去掉$r$,直接写$\sqrt[n]{a_n}<1$。}。
		\item 若对于无穷个$n$有$\sqrt[n]{a_n}\geqslant1$,则$\sum\limits_{n=1}^{+\infty}a_n$发散。
	\end{enumerate}
\end{theorem}
\begin{proof}
	(1)若条件满足,则$\forall\;n>N,\;a_n<r^n$。由比较判别法,$\sum\limits_{n=1}^{+\infty}a_n$收敛。\par
	(2)由条件,$S_n-S_{n-1}$不趋向于$0$,那么$\{S_n\}$也就不收敛。
\end{proof}
\subsubsection{上下极限形式}
\begin{theorem}
	设$\sum\limits_{n=1}^{+\infty}a_n$是正项级数。且:
	\begin{equation*}
		\varlimsup \sqrt[n]{a_n}=q
	\end{equation*}
	\begin{enumerate}
		\item 若$q<1$,则$\sum\limits_{n=1}^{+\infty}a_n$收敛。
		\item 若$q>1$,则$\sum\limits_{n=1}^{+\infty}a_n$发散。
	\end{enumerate}
\end{theorem}
\subsubsection{极限形式}
\begin{theorem}
	设$\sum\limits_{n=1}^{+\infty}a_n$是正项级数。且:
	\begin{equation*}
		\lim \sqrt[n]{a_n}=q
	\end{equation*}
	\begin{enumerate}
		\item 若$q<1$,则$\sum\limits_{n=1}^{+\infty}a_n$收敛。
		\item 若$q>1$,则$\sum\limits_{n=1}^{+\infty}a_n$发散。
	\end{enumerate}
\end{theorem}

\subsection{比值判别法}
\begin{definition}
	对于级数$\sum\limits_{n=1}^{+\infty}a_n$,如果$\exists\;N\in\mathbb{N}^+,\;\forall\;n>N,\;a_n>0$,则称该级数为\gls{SPosSeries}。
\end{definition}
针对严格正项级数的敛散性问题,有如下的\gls{RatioTest}。
\begin{theorem}
	设$\sum\limits_{n=1}^{+\infty}a_n$和$\sum\limits_{n=1}^{+\infty}b_n$是严格正项级数。
	\begin{enumerate}
		\item 若$\sum\limits_{n=1}^{+\infty}b_n$收敛,且$\exists\;N\in\mathbb{N}^+$,当$n>N$时满足:
		\begin{equation*}
			\frac{a_{n+1}}{a_n}\leqslant\frac{b_{n+1}}{b_n}
		\end{equation*}
		则$\sum\limits_{n=1}^{+\infty}a_n$也收敛。
		\item 若$\sum\limits_{n=1}^{+\infty}b_n$发散,且$\exists\;N\in\mathbb{N}^+$,当$n>N$时满足:
		\begin{equation*}
			\frac{a_{n+1}}{a_n}\geqslant\frac{b_{n+1}}{b_n}
		\end{equation*}
		则$\sum\limits_{n=1}^{+\infty}a_n$也发散。
	\end{enumerate}
\end{theorem}
\begin{proof}
	(1)由题目条件可得:
	\begin{equation*}
		\frac{a_{N+2}}{a_{N+1}}\leqslant\frac{b_{N+2}}{b_{N+1}},\;
		\frac{a_{N+3}}{a_{N+2}}\leqslant\frac{b_{N+3}}{b_{N+2}},\;
		\cdots,\;
		\frac{a_{n+1}}{a_n}\leqslant\frac{b_{n+1}}{b_n}
	\end{equation*}
	作乘法即可得:
	\begin{equation*}
		\frac{a_{n+1}}{a_{N+1}}\leqslant\frac{b_{n+1}}{b_{N+1}}
	\end{equation*}
	即:
	\begin{equation*}
		a_{n+1}\leqslant\frac{a_{N+1}}{b_{N+1}}b_{n+1},\;\forall\;n>N
	\end{equation*}
	由比较判别法即可得出$\sum\limits_{n=1}^{+\infty}a_n$收敛。\par
	(2)类似(1)。
\end{proof}

\subsection{D'Alembert判别法}
以下定理我们称之为D'Alembert判别法。该判别法的实质其实就是将$\sum\limits_{n=1}^{+\infty}a_n$与$\sum\limits_{n=1}^{+\infty}r^n$作比较。当$r<1$时,由收敛准则易证$\sum\limits_{n=1}^{+\infty}r^n$收敛。
\subsubsection{一般形式}
\begin{theorem}
	设$\sum\limits_{n=1}^{+\infty}a_n$是严格正项级数。
	\begin{enumerate}
		\item 若$\exists\;r<1,\;\exists\;N\in\mathbb{N}^+$,当$n>N$时满足:
		\begin{equation*}
			\frac{a_{n+1}}{a_n}<r
		\end{equation*}
		则$\sum\limits_{n=1}^{+\infty}a_n$收敛。
		\item 若$\exists\;N\in\mathbb{N}^+$,当$n>N$满足\footnote{思考能否像柯西根式判别法一样写成对无穷个$n$都成立。}:
		\begin{equation*}
			\frac{a_{n+1}}{a_n}\geqslant1
		\end{equation*}
		则$\sum\limits_{n=1}^{+\infty}a_n$发散。
	\end{enumerate}
\end{theorem}
\subsubsection{上下极限形式}
\begin{theorem}
	设$\sum\limits_{n=1}^{+\infty}a_n$是严格正项级数。
	\begin{enumerate}
		\item 如果:
		\begin{equation*}
			\varlimsup\frac{a_{n+1}}{a_n}<1
		\end{equation*}
		则$\sum\limits_{n=1}^{+\infty}a_n$收敛。
		\item 如果:
		\begin{equation*}
			\varliminf\frac{a_{n+1}}{a_n}>1
		\end{equation*}
		则$\sum\limits_{n=1}^{+\infty}a_n$发散。
	\end{enumerate}
\end{theorem}
\subsubsection{极限形式}
\begin{theorem}
	设$\sum\limits_{n=1}^{+\infty}a_n$是严格正项级数。且:
	\begin{equation*}
		\lim \frac{a_{n+1}}{a_n}=q
	\end{equation*}
	\begin{enumerate}
		\item 若$q<1$,则$\sum\limits_{n=1}^{+\infty}a_n$收敛。
		\item 若$q>1$,则$\sum\limits_{n=1}^{+\infty}a_n$发散。
	\end{enumerate}
\end{theorem}
\subsubsection{D'Alembert判别法与Cauchy根式判别法的比较}
\begin{theorem}
	(1)\;$\varlimsup\frac{a_{n+1}}{a_n}<1\Rightarrow\varlimsup\sqrt[n]{a_n}<1$;(2)\;$\varliminf\frac{a_{n+1}}{a_n}>1\Rightarrow\varlimsup\sqrt[n]{a_n}>1$。即能用D'Alembert判别法判别的,一定也能用Cauchy根式判别法判别。
\end{theorem}
\begin{proof}
	只需利用下列关系:
	\begin{align*}
		\varliminf\frac{a_{n+1}}{a_n}
		&\leqslant\varliminf\sqrt[n]{a_n} \\
		&\leqslant\varlimsup\sqrt[n]{a_n} \\
		&\leqslant\varlimsup\frac{a_{n+1}}{a_n}
	\end{align*}
	下给出证明。\par
	(1)设:
	\begin{equation*}
		\varlimsup\frac{a_{n+1}}{a_n}=\xi
	\end{equation*}
	取$\lambda>\xi$,则$\exists\;N\in\mathbb{N}^+,\;\forall\;n>N$,使得:
	\begin{equation*}
		\frac{a_{n+1}}{a_n}<\lambda
	\end{equation*}	
	即有:
	\begin{align*}
		\sqrt[n]{a_n}&=\sqrt[n]{\frac{a_n}{a_{n-1}}\frac{a_{n-1}}{a_{n-2}}\cdots\frac{a_{N+2}}{a_{N+1}}a_{N+1}} \\
		&<\sqrt[n]{\lambda^{n-N-1}a_{N+1}} \\
		&=\lambda^{1-\frac{N-1}{n}}\sqrt[n]{a_{N+1}}
	\end{align*}
	于是有:
	\begin{equation*}
		\varlimsup\sqrt[n]{a_n}\leqslant\lim_{n\to+\infty}\lambda^{1-\frac{N-1}{n}}\sqrt[n]{a_{N+1}}=\lambda
	\end{equation*}
	可以取$\lambda\rightarrow\xi$,就有:
	\begin{equation*}
		\varlimsup\sqrt[n]{a_n}\leqslant\xi=\varlimsup\frac{a_{n+1}}{a_n}
	\end{equation*}\par
	(2)设:
	\begin{equation*}
		\varliminf\frac{a_{n+1}}{a_n}=\xi
	\end{equation*}
	取$\lambda<\xi$,则$\exists\;N\in\mathbb{N}^+,\;\forall\;n>N$,使得:
	\begin{equation*}
		\frac{a_{n+1}}{a_n}>\lambda
	\end{equation*}	
	即有:
	\begin{align*}
		\sqrt[n]{a_n}&=\sqrt[n]{\frac{a_n}{a_{n-1}}\frac{a_{n-1}}{a_{n-2}}\cdots\frac{a_{N+2}}{a_{N+1}}a_{N+1}} \\
		&>\sqrt[n]{\lambda^{n-N-1}a_{N+1}} \\
		&=\lambda^{1-\frac{N-1}{n}}\sqrt[n]{a_{N+1}}
	\end{align*}
	于是有:
	\begin{equation*}
		\varliminf\sqrt[n]{a_n}\geqslant\lim_{n\to+\infty}\lambda^{1-\frac{N-1}{n}}\sqrt[n]{a_{N+1}}=\lambda
	\end{equation*}
	可以取$\lambda\rightarrow\xi$,就有:
	\begin{equation*}
		\varliminf\sqrt[n]{a_n}\geqslant\xi=\varliminf\frac{a_{n+1}}{a_n}\qedhere
	\end{equation*}
\end{proof}

\subsection{Cauchy积分判别法}
本节介绍Cauchy's\gls{IntTest}。
\begin{lemma}
	设$f(x)$在$[1,+\infty)$上非负。记
	\begin{equation*}
		F(x)=\int_1^xf(t)\dif t
	\end{equation*}
	则级数$\sum\limits_{n=1}^{+\infty}[F(n+1)-F(n)]$与广义积分$\int_1^{+\infty}f(x)\dif x$同收敛。
\end{lemma}
\begin{proof}
	(1)如果广义积分收敛,则:
	\begin{align*}
		\lim_{n\to+\infty}\sum_{i=1}^n[F(i+1)-F(i)]&=	\lim_{n\to+\infty}F(n+1) \\
		&=\lim_{n\to+\infty}\int_1^{n+1}f(x)\dif x \\
		&=\int_1^{+\infty}f(x)\dif x
	\end{align*}
	因此级数收敛。\par
	(2)如果级数收敛,对任意的$H>0$,令$N=[H]$(取整函数),则:
	\begin{align*}
		\int_1^{+\infty}f(x)\dif x
		&=\lim_{H\to+\infty}\int_1^{H}f(x)\dif x \\
		&\leqslant\lim_{H\to+\infty}\int_1^{N+1}f(x)\dif x \\
		&=\lim_{N\to+\infty}\int_1^{N+1}f(x)\dif x \\
		&=\lim_{N\to+\infty}\sum_{n=1}^{N}[F(n+1)-F(n)] \\
		&=\sum_{n=1}^{+\infty}[F(n+1)-F(n)]
	\end{align*}
	因此广义积分收敛。
\end{proof}
\begin{theorem}
	设$f(x)$在$[1,+\infty)$单调下降且非负,则级数:
	\begin{equation*}
		\sum_{n=1}^{+\infty}f(n)
	\end{equation*}
	与广义积分:
	\begin{equation*}
		\int_{1}^{+\infty}f(x)\dif x
	\end{equation*}
	同敛态。
\end{theorem}
\begin{proof}
	(1)如果广义积分收敛,则级数:
	\begin{equation*}
		\sum_{n=2}^{+\infty}[F(n)-F(n-1)]
	\end{equation*}
	收敛,由$f(x)$单调下降且非负,有:
	\begin{equation*}
		f(n)\leqslant\int_{n-1}^{n}f(x)\dif x=F(n)-F(n-1)
	\end{equation*}
	由比较判别法,级数收敛。\par
	(2)如果广义积分发散,则级数:
	\begin{equation*}
		\sum_{n=1}^{+\infty}[F(n+1)-F(n)]
	\end{equation*}
	发散,由$f(x)$单调下降且非负,有:
	\begin{equation*}
		f(n)\geqslant\int_{n}^{n+1}f(x)\dif x=F(n+1)-F(n)
	\end{equation*}
	由比较判别法,级数发散。
\end{proof}

\subsection{比较尺度问题}
由柯西积分判别法,我们可以很容易地判断以下级数是否收敛:
\begin{enumerate}
	\item $\sum\limits_{n=1}^{+\infty}\frac{1}{n^p}$。
	\item $\sum\limits_{n=2}^{+\infty}\frac{1}{n(\ln n)^p}$。
	\item $\sum\limits_{n=3}^{+\infty}\frac{1}{n\ln n(\ln\ln n)^p}$。
	\item $\cdots$
\end{enumerate}
上述级数在$p>1$时都收敛,反之发散。\par
我们会谈论比较尺度的问题,简单来讲就是说对于一个级数的敛散性问题,如果a判别法无法判别但b判别法可以,那么b的比较尺度应该是更加精细的。这个问题局限于判别法,但也不局限于判别法,如何理解呢?D'Alembert判别法和Raabe判别法的背后其实都是比较判别法,方法是一样的,但选取的比较级数不一样,也带来了它们比较尺度的不一样。上面这句话不是很直观,我们来看上面提到的由柯西积分法带来的三个级数(其实不止三个,按照换元积分法的规则,可以利用的级数能够无穷无尽地写下去)。可以看出,越往下写,通项在$n$相同时就会变得越大。也就是说它的尺度会变得更精细,之前无法判断为收敛的现在可以了。

\subsection{Raabe判别法}
以下定理我们称之为Raabe判别法。该判别法的实质其实就是将$\sum\limits_{n=1}^{+\infty}a_n$与$\sum\limits_{n=1}^{+\infty}\dfrac{1}{n^p}$作比较。当$p>1$时,由Cauchy积分判别法易证$\sum\limits_{n=1}^{+\infty}\dfrac{1}{n^p}$收敛。
\subsubsection{一般形式}
\begin{theorem}
	设$\sum\limits_{n=1}^{+\infty}a_n$是严格正项级数。
	\begin{enumerate}
		\item  如果$\exists\;q>1,\;\exists\;N\in\mathbb{N}^+$,使得对任意的$n>N$有:
		\begin{equation*}
			n\left(\frac{a_n}{a_{n+1}}-1\right)\geqslant q
		\end{equation*}
		则$\sum\limits_{n=1}^{+\infty}a_n$收敛。
		\item  如果$\exists\;N\in\mathbb{N}^+$,使得对任意的$n>N$有:
		\begin{equation*}
			n\left(\frac{a_n}{a_{n+1}}-1\right)\leqslant1
		\end{equation*}
		则$\sum\limits_{n=1}^{+\infty}a_n$发散。
	\end{enumerate}
\end{theorem}
\begin{proof}
	(1)所给的条件等价于对任意的$n>N$有:
	\begin{equation*}
		\frac{a_n}{a_{n+1}}\geqslant1+\frac{q}{n}
	\end{equation*}
	取$p\in\mathbb{R}$满足$1<p<q$,取级数$\sum\limits_{n=1}^{+\infty}\dfrac{1}{n^p}$,令$b_n=\frac{1}{n^p}$。当$n$足够大的时候,有:
	\begin{align*}
		\frac{b_n}{b_{n+1}}&=\frac{(n+1)^p}{n^p} \\
		&=(1+\frac{1}{n})^p \\
		&=1+\frac{p}{n}+O(\frac{1}{n^2}) \\
		&<1+\frac{q}{n}\leqslant\frac{a_n}{a_{n+1}}
	\end{align*}
	由比值判别法可得$\sum\limits_{n=1}^{+\infty}a_n$收敛。\par
	(2)所给的条件等价于对任意的$n>N$有:
	\begin{equation*}
		\frac{a_n}{a_{n+1}}\leqslant1+\frac{1}{n}=\frac{\frac{1}{n}}{\frac{1}{n+1}}
	\end{equation*}
	由比值判别法可得$\sum\limits_{n=1}^{+\infty}a_n$发散。
\end{proof}
\subsubsection{上下极限形式}
\begin{theorem}
	设$\sum\limits_{n=1}^{+\infty}a_n$是严格正项级数。
	\begin{enumerate}
		\item 如果:
		\begin{equation*}
			\varliminf n\left(\frac{a_{n}}{a_{n+1}}-1\right)>1
		\end{equation*}
		则$\sum\limits_{n=1}^{+\infty}a_n$收敛。
		\item 如果:
		\begin{equation*}
			\varlimsup n\left(\frac{a_{n}}{a_{n+1}}-1\right)<1
		\end{equation*}
		则$\sum\limits_{n=1}^{+\infty}a_n$发散。
	\end{enumerate}
\end{theorem}
\subsubsection{极限形式}
\begin{theorem}
	设$\sum\limits_{n=1}^{+\infty}a_n$是严格正项级数。且:
	\begin{equation*}
		\lim_{n\to+\infty} n\left(\frac{a_{n}}{a_{n+1}}-1\right)=q
	\end{equation*}
	\begin{enumerate}
		\item 若$q>1$,则$\sum\limits_{n=1}^{+\infty}a_n$收敛。
		\item 若$q<1$,则$\sum\limits_{n=1}^{+\infty}a_n$发散。
	\end{enumerate}
\end{theorem}

\subsection{Gauss判别法}
下面介绍Gauss判别法,它可以概括D'Alembert判别法与Raabe判别法,在比较尺度上能够达到$\sum\limits_{n=2}^{+\infty}\frac{1}{n(\ln n)^p}$的精度。
\begin{theorem}
	设$\sum\limits_{n=1}^{+\infty}a_n$是严格正项级数。若:
	\begin{equation*}
		\frac{a_n}{a_{n+1}}=\lambda+\frac{\mu}{n}+\frac{\nu}{n\ln n}+o(\frac{1}{n\ln n})
	\end{equation*}
	则:
	\begin{enumerate}
		\item 若$\lambda>1$,则级数$\sum\limits_{n=1}^{+\infty}a_n$收敛;若$\lambda<1$,则级数$\sum\limits_{n=1}^{+\infty}a_n$发散;
		\item 若$\lambda=1,\;\mu>1$,则级数$\sum\limits_{n=1}^{+\infty}a_n$收敛;若$\lambda=1,\;\mu<1$,则级数$\sum\limits_{n=1}^{+\infty}a_n$发散;
		\item 若$\lambda=1,\;\mu=1,\;\nu>1$,则级数$\sum\limits_{n=1}^{+\infty}a_n$收敛;若$\lambda=1,\;\mu=1,\;\nu<1$,则级数$\sum\limits_{n=1}^{+\infty}a_n$发散。
	\end{enumerate}
\end{theorem}
\begin{proof}
	(1)可以归结为D'Alembert判别法;(2)可以归结为Raabe判别法;\par
	(3)我们以级数:
	\begin{equation*}
		\sum\limits_{n=2}^{+\infty}b_n=\sum\limits_{n=2}^{+\infty}\frac{1}{n(\ln n)^p}
	\end{equation*}
	作为比较的尺度。计算可得:
	\begin{align*}
		\frac{b_n}{b_{n+1}}
		&=\frac{(n+1)(\ln(n+1))^p}{n(\ln n)^p} \\
		&=\left(1+\frac{1}{n}\right)\left(1+\frac{\ln(1+\frac{1}{n})}{\ln n}\right)^p \\
		&=\left(1+\frac{1}{n}\right)\left(1+\frac{\ln(1+\frac{1}{n})}{\ln n}\right)^p \\
		&=\left(1+\frac{1}{n}\right)\left(1+\frac{1}{n\ln n}+o(\frac{1}{n\ln n})\right)^p \\
		&=1+\frac{1}{n}+\frac{p}{n\ln n}+o\left(\frac{1}{n\ln n}\right)
	\end{align*}
	其中第三行到第四行利用了泰勒展开。\par
	回归原级数,如果$\lambda=\mu=1,\;\nu>1$,则可以选取$p\in\mathbb{R}$,使得:
	\begin{equation*}
		1<p<\nu
	\end{equation*}
	当$n$足够大的时候,就有:
	\begin{equation*}
		\frac{a_n}{a_{n+1}}\geqslant\frac{b_n}{b_{n+1}}
	\end{equation*}
	此时$\sum\limits_{n=1}^{+\infty}b_n$收敛,由比值判别法,级数$\sum\limits_{n=1}^{+\infty}a_n$收敛。\par
	如果$\lambda=\mu=1,\;\nu<1$,则可以选取$p\in\mathbb{R}$,使得:
	\begin{equation*}
		\nu<p<1
	\end{equation*}
	当$n$足够大的时候,就有:
	\begin{equation*}
		\frac{a_n}{a_{n+1}}\leqslant\frac{b_n}{b_{n+1}}
	\end{equation*}
	此时$\sum\limits_{n=1}^{+\infty}b_n$发散,由比值判别法,级数$\sum\limits_{n=1}^{+\infty}a_n$发散。
\end{proof}







