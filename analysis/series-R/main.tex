\chapter{数项级数}

\begin{definition}
	设$a_n\in\mathbb{R},\;\forall\;n\in\mathbb{N}^+$。我们把记号:
	\begin{equation*}
		\sum_{n=1}^{+\infty}a_n=a_1+a_2+\cdots+a_n+\cdots
	\end{equation*}
	称为以$\seq{a}{n},\dots$为项的\gls{series}。把$\{S_n\}$:
	\begin{equation*}
		S_n=\sum_{i=1}^na_i
	\end{equation*}
	称为级数$\sum\limits_{n=1}^{+\infty}a_n$的\gls{SeqOfPartialSum}。如果$\{S_n\}$收敛,则称级数$\sum\limits_{n=1}^{+\infty}a_n$收敛,同时将级数的值记为$\{S_n\}$的极限。
\end{definition}
本章的目的即为讨论确定级数$\sum\limits_{n=1}^{+\infty}a_n$敛散性的方法。\par
\subsubsection{级数与序列的关系}
级数和序列其实是相通的。我们通过部分和序列来定义级数的敛散性与级数的和,反过来,关于序列极限的问题也可以转化为级数的相应问题。具体来讲:
\begin{enumerate}
	\item 序列$\{a_n\}$的敛散性与级数$a_1+\sum\limits_{i=1}^{+\infty}(a_{i+1}-a_i)$的敛散性相同。
	\item 序列$\{a_n\}\to a$可以转化为级数$a_1+\sum\limits_{i=1}^{+\infty}(a_{i+1}-a_i)=a$。
\end{enumerate}
\subsubsection{级数的柯西收敛原理}
可根据序列的Cauchy收敛原理给出如下级数的Cauchy收敛原理:
\begin{theorem}
	级数$\sum\limits_{n=1}^{+\infty}a_n$收敛的充分必要条件为:
	\begin{equation*}
		\forall\;\varepsilon>0,\;\exists\;N\in\mathbb{N}^+,\;\forall\;n,m>N,\;m>n,\;\left|\sum_{i=n}^ma_i\right|<\varepsilon
	\end{equation*}
\end{theorem}
\subsubsection{绝对收敛与条件收敛}
我们注意到,当级数$\sum\limits_{n=1}^{+\infty}|a_n|$收敛时,级数$\sum\limits_{n=1}^{+\infty}a_n$必定也收敛,这点可以由级数的Cauchy收敛原理及绝对值的三角不等式来证明。因此:
\begin{definition}
	对级数的收敛做如下划分:\par
	(1)如果级数$\sum\limits_{n=1}^{+\infty}|a_n|$收敛,则称级数$\sum\limits_{n=1}^{+\infty}a_n$\gls{AbsoluteConvergence}。\par
	(2)如果级数$\sum\limits_{n=1}^{+\infty}|a_n|$发散,但级数$\sum\limits_{n=1}^{+\infty}a_n$收敛,则称级数$\sum\limits_{n=1}^{+\infty}a_n$\gls{ConditionalConvergence}。
\end{definition}
如果一个级数的各项都是非负的,那么显然它的敛散性会更好讨论一点。故我们首先来讨论各项都非负的级数的敛散性问题,再去讨论任意项级数的敛散性问题。\par
在此之前先给出一个很容易证得的定理:
\begin{theorem}
	级数的运算满足线性性。
\end{theorem}


\section{正项级数}

\begin{definition}
	如果级数$\sum\limits_{n=1}^{+\infty}a_n$的每一项都是非负数,则称该级数为\gls{PosSeries}。
\end{definition}

\subsection{收敛原理}
显然正项级数的部分和序列是单调递增序列。\info{记得把单调序列的收敛原理放在这里}由单调序列的收敛原理,有如下定理:
\begin{theorem}
	正项级数$\sum\limits_{n=1}^{+\infty}a_n$收敛的充分必要条件为它的部分和序列$\{S_n\}$有上界。
\end{theorem}

\subsection{比较判别法}
以下方法我们称之为\gls{ComparisonTest}。
\subsubsection{一般形式}
\begin{theorem}
	设$\sum\limits_{n=1}^{+\infty}a_n$和$\sum\limits_{n=1}^{+\infty}b_n$都是正项级数。
	\begin{enumerate}
		\item 若$\sum\limits_{n=1}^{+\infty}b_n$收敛,且$\exists\;N\in\mathbb{N}^+,\;\forall\;n>N,\;a_n\leqslant cb_n,\;c\in[0,+\infty)$,那么$\sum\limits_{n=1}^{+\infty}a_n$也收敛。
		\item 若$\sum\limits_{n=1}^{+\infty}b_n$发散,且$\exists\;N\in\mathbb{N}^+,\;\forall\;n>N,\;a_n\geqslant cb_n,\;c\in(0,+\infty]$,那么$\sum\limits_{n=1}^{+\infty}a_n$也发散。
	\end{enumerate}
\end{theorem}
\begin{proof}
	(1)若条件得到满足,则:
	\begin{align*}
		\sum_{n=1}^{+\infty}a_n&=\sum_{n=1}^Na_n+\sum_{n=N+1}^{+\infty}a_n \\
		&\leqslant\sum_{n=1}^Na_n+\sum_{n=N+1}^{+\infty}cb_n \\
		&\leqslant\sum_{n=1}^Na_n+\sum_{n=1}^{+\infty}cb_n \\
		&=\sum_{n=1}^Na_n+c\sum_{n=1}^{+\infty}b_n
	\end{align*}
	显然$\sum\limits_{n=1}^{+\infty}a_n$有上界,因此收敛。\par
	(2)若此时$a_n$收敛,则:
	\begin{equation*}
		\exists\;N\in\mathbb{N}^+,\;\forall\;n>N,\;b_n\leqslant\frac{1}{c}a_n,\;\frac{1}{c}\in[0,+\infty)
	\end{equation*}
	$\sum\limits_{n=1}^{+\infty}b_n$就应该收敛,矛盾。
\end{proof}
\subsubsection{极限形式}
\begin{theorem}
	设$\sum\limits_{n=1}^{+\infty}a_n$和$\sum\limits_{n=1}^{+\infty}b_n$都是正项级数。且有:
	\begin{equation*}
		\lim\frac{a_n}{b_n}=\gamma,\;0\leqslant\gamma\leqslant+\infty
	\end{equation*}
	\begin{enumerate}
		\item 若$\sum\limits_{n=1}^{+\infty}b_n$收敛且$\gamma<+\infty$,则$\sum\limits_{n=1}^{+\infty}a_n$收敛。
		\item 若$\sum\limits_{n=1}^{+\infty}b_n$发散且$\gamma>0$,则$\sum\limits_{n=1}^{+\infty}a_n$发散。
	\end{enumerate}
\end{theorem}
\begin{proof}
	(1)若条件得到满足,取$\varepsilon<+\infty$,则$\exists\;N\in\mathbb{N}^+,\;\forall\;n>N,a_n<(\gamma+\varepsilon)b_n$。\par
	(2)若条件得到满足,取$\varepsilon<\frac{\gamma}{2}$,则$\exists\;N\in\mathbb{N}^+,\;\forall\;n>N,a_n>(\gamma-\varepsilon)b_n$。
\end{proof}

\subsection{Cauchy根式判别法}
以下定理我们称之为Cauchy's\gls{RootTest}。该判别法的实质其实就是将$\sum\limits_{n=1}^{+\infty}a_n$与$\sum\limits_{n=1}^{+\infty}r^n$作比较。当$r<1$时,由收敛准则易证$\sum\limits_{n=1}^{+\infty}r^n$收敛。
\subsubsection{一般形式}
\begin{theorem}
	设$\sum\limits_{n=1}^{+\infty}a_n$是正项级数。
	\begin{enumerate}
		\item 若$\exists\;r<1,\;\exists\;N\in\mathbb{N}^+,\;\forall\;n>N,\;\sqrt[n]{a_n}<r$,则$\sum\limits_{n=1}^{+\infty}a_n$收敛\footnote{思考能否去掉$r$,直接写$\sqrt[n]{a_n}<1$。}。
		\item 若对于无穷个$n$有$\sqrt[n]{a_n}\geqslant1$,则$\sum\limits_{n=1}^{+\infty}a_n$发散。
	\end{enumerate}
\end{theorem}
\begin{proof}
	(1)若条件满足,则$\forall\;n>N,\;a_n<r^n$。由比较判别法,$\sum\limits_{n=1}^{+\infty}a_n$收敛。\par
	(2)由条件,$S_n-S_{n-1}$不趋向于$0$,那么$\{S_n\}$也就不收敛。
\end{proof}
\subsubsection{上下极限形式}
\begin{theorem}
	设$\sum\limits_{n=1}^{+\infty}a_n$是正项级数。且:
	\begin{equation*}
		\varlimsup \sqrt[n]{a_n}=q
	\end{equation*}
	\begin{enumerate}
		\item 若$q<1$,则$\sum\limits_{n=1}^{+\infty}a_n$收敛。
		\item 若$q>1$,则$\sum\limits_{n=1}^{+\infty}a_n$发散。
	\end{enumerate}
\end{theorem}
\subsubsection{极限形式}
\begin{theorem}
	设$\sum\limits_{n=1}^{+\infty}a_n$是正项级数。且:
	\begin{equation*}
		\lim \sqrt[n]{a_n}=q
	\end{equation*}
	\begin{enumerate}
		\item 若$q<1$,则$\sum\limits_{n=1}^{+\infty}a_n$收敛。
		\item 若$q>1$,则$\sum\limits_{n=1}^{+\infty}a_n$发散。
	\end{enumerate}
\end{theorem}

\subsection{比值判别法}
\begin{definition}
	对于级数$\sum\limits_{n=1}^{+\infty}a_n$,如果$\exists\;N\in\mathbb{N}^+,\;\forall\;n>N,\;a_n>0$,则称该级数为\gls{SPosSeries}。
\end{definition}
针对严格正项级数的敛散性问题,有如下的\gls{RatioTest}。
\begin{theorem}
	设$\sum\limits_{n=1}^{+\infty}a_n$和$\sum\limits_{n=1}^{+\infty}b_n$是严格正项级数。
	\begin{enumerate}
		\item 若$\sum\limits_{n=1}^{+\infty}b_n$收敛,且$\exists\;N\in\mathbb{N}^+$,当$n>N$时满足:
		\begin{equation*}
			\frac{a_{n+1}}{a_n}\leqslant\frac{b_{n+1}}{b_n}
		\end{equation*}
		则$\sum\limits_{n=1}^{+\infty}a_n$也收敛。
		\item 若$\sum\limits_{n=1}^{+\infty}b_n$发散,且$\exists\;N\in\mathbb{N}^+$,当$n>N$时满足:
		\begin{equation*}
			\frac{a_{n+1}}{a_n}\geqslant\frac{b_{n+1}}{b_n}
		\end{equation*}
		则$\sum\limits_{n=1}^{+\infty}a_n$也发散。
	\end{enumerate}
\end{theorem}
\begin{proof}
	(1)由题目条件可得:
	\begin{equation*}
		\frac{a_{N+2}}{a_{N+1}}\leqslant\frac{b_{N+2}}{b_{N+1}},\;
		\frac{a_{N+3}}{a_{N+2}}\leqslant\frac{b_{N+3}}{b_{N+2}},\;
		\cdots,\;
		\frac{a_{n+1}}{a_n}\leqslant\frac{b_{n+1}}{b_n}
	\end{equation*}
	作乘法即可得:
	\begin{equation*}
		\frac{a_{n+1}}{a_{N+1}}\leqslant\frac{b_{n+1}}{b_{N+1}}
	\end{equation*}
	即:
	\begin{equation*}
		a_{n+1}\leqslant\frac{a_{N+1}}{b_{N+1}}b_{n+1},\;\forall\;n>N
	\end{equation*}
	由比较判别法即可得出$\sum\limits_{n=1}^{+\infty}a_n$收敛。\par
	(2)类似(1)。
\end{proof}

\subsection{D'Alembert判别法}
以下定理我们称之为D'Alembert判别法。该判别法的实质其实就是将$\sum\limits_{n=1}^{+\infty}a_n$与$\sum\limits_{n=1}^{+\infty}r^n$作比较。当$r<1$时,由收敛准则易证$\sum\limits_{n=1}^{+\infty}r^n$收敛。
\subsubsection{一般形式}
\begin{theorem}
	设$\sum\limits_{n=1}^{+\infty}a_n$是严格正项级数。
	\begin{enumerate}
		\item 若$\exists\;r<1,\;\exists\;N\in\mathbb{N}^+$,当$n>N$时满足:
		\begin{equation*}
			\frac{a_{n+1}}{a_n}<r
		\end{equation*}
		则$\sum\limits_{n=1}^{+\infty}a_n$收敛。
		\item 若$\exists\;N\in\mathbb{N}^+$,当$n>N$满足\footnote{思考能否像柯西根式判别法一样写成对无穷个$n$都成立。}:
		\begin{equation*}
			\frac{a_{n+1}}{a_n}\geqslant1
		\end{equation*}
		则$\sum\limits_{n=1}^{+\infty}a_n$发散。
	\end{enumerate}
\end{theorem}
\subsubsection{上下极限形式}
\begin{theorem}
	设$\sum\limits_{n=1}^{+\infty}a_n$是严格正项级数。
	\begin{enumerate}
		\item 如果:
		\begin{equation*}
			\varlimsup\frac{a_{n+1}}{a_n}<1
		\end{equation*}
		则$\sum\limits_{n=1}^{+\infty}a_n$收敛。
		\item 如果:
		\begin{equation*}
			\varliminf\frac{a_{n+1}}{a_n}>1
		\end{equation*}
		则$\sum\limits_{n=1}^{+\infty}a_n$发散。
	\end{enumerate}
\end{theorem}
\subsubsection{极限形式}
\begin{theorem}
	设$\sum\limits_{n=1}^{+\infty}a_n$是严格正项级数。且:
	\begin{equation*}
		\lim \frac{a_{n+1}}{a_n}=q
	\end{equation*}
	\begin{enumerate}
		\item 若$q<1$,则$\sum\limits_{n=1}^{+\infty}a_n$收敛。
		\item 若$q>1$,则$\sum\limits_{n=1}^{+\infty}a_n$发散。
	\end{enumerate}
\end{theorem}
\subsubsection{D'Alembert判别法与Cauchy根式判别法的比较}
\begin{theorem}
	(1)\;$\varlimsup\frac{a_{n+1}}{a_n}<1\Rightarrow\varlimsup\sqrt[n]{a_n}<1$;(2)\;$\varliminf\frac{a_{n+1}}{a_n}>1\Rightarrow\varlimsup\sqrt[n]{a_n}>1$。即能用D'Alembert判别法判别的,一定也能用Cauchy根式判别法判别。
\end{theorem}
\begin{proof}
	只需利用下列关系:
	\begin{align*}
		\varliminf\frac{a_{n+1}}{a_n}
		&\leqslant\varliminf\sqrt[n]{a_n} \\
		&\leqslant\varlimsup\sqrt[n]{a_n} \\
		&\leqslant\varlimsup\frac{a_{n+1}}{a_n}
	\end{align*}
	下给出证明。\par
	(1)设:
	\begin{equation*}
		\varlimsup\frac{a_{n+1}}{a_n}=\xi
	\end{equation*}
	取$\lambda>\xi$,则$\exists\;N\in\mathbb{N}^+,\;\forall\;n>N$,使得:
	\begin{equation*}
		\frac{a_{n+1}}{a_n}<\lambda
	\end{equation*}	
	即有:
	\begin{align*}
		\sqrt[n]{a_n}&=\sqrt[n]{\frac{a_n}{a_{n-1}}\frac{a_{n-1}}{a_{n-2}}\cdots\frac{a_{N+2}}{a_{N+1}}a_{N+1}} \\
		&<\sqrt[n]{\lambda^{n-N-1}a_{N+1}} \\
		&=\lambda^{1-\frac{N-1}{n}}\sqrt[n]{a_{N+1}}
	\end{align*}
	于是有:
	\begin{equation*}
		\varlimsup\sqrt[n]{a_n}\leqslant\lim_{n\to+\infty}\lambda^{1-\frac{N-1}{n}}\sqrt[n]{a_{N+1}}=\lambda
	\end{equation*}
	可以取$\lambda\rightarrow\xi$,就有:
	\begin{equation*}
		\varlimsup\sqrt[n]{a_n}\leqslant\xi=\varlimsup\frac{a_{n+1}}{a_n}
	\end{equation*}\par
	(2)设:
	\begin{equation*}
		\varliminf\frac{a_{n+1}}{a_n}=\xi
	\end{equation*}
	取$\lambda<\xi$,则$\exists\;N\in\mathbb{N}^+,\;\forall\;n>N$,使得:
	\begin{equation*}
		\frac{a_{n+1}}{a_n}>\lambda
	\end{equation*}	
	即有:
	\begin{align*}
		\sqrt[n]{a_n}&=\sqrt[n]{\frac{a_n}{a_{n-1}}\frac{a_{n-1}}{a_{n-2}}\cdots\frac{a_{N+2}}{a_{N+1}}a_{N+1}} \\
		&>\sqrt[n]{\lambda^{n-N-1}a_{N+1}} \\
		&=\lambda^{1-\frac{N-1}{n}}\sqrt[n]{a_{N+1}}
	\end{align*}
	于是有:
	\begin{equation*}
		\varliminf\sqrt[n]{a_n}\geqslant\lim_{n\to+\infty}\lambda^{1-\frac{N-1}{n}}\sqrt[n]{a_{N+1}}=\lambda
	\end{equation*}
	可以取$\lambda\rightarrow\xi$,就有:
	\begin{equation*}
		\varliminf\sqrt[n]{a_n}\geqslant\xi=\varliminf\frac{a_{n+1}}{a_n}\qedhere
	\end{equation*}
\end{proof}

\subsection{Cauchy积分判别法}
本节介绍Cauchy's\gls{IntTest}。
\begin{lemma}
	设$f(x)$在$[1,+\infty)$上非负。记
	\begin{equation*}
		F(x)=\int_1^xf(t)\dif t
	\end{equation*}
	则级数$\sum\limits_{n=1}^{+\infty}[F(n+1)-F(n)]$与广义积分$\int_1^{+\infty}f(x)\dif x$同收敛。
\end{lemma}
\begin{proof}
	(1)如果广义积分收敛,则:
	\begin{align*}
		\lim_{n\to+\infty}\sum_{i=1}^n[F(i+1)-F(i)]&=	\lim_{n\to+\infty}F(n+1) \\
		&=\lim_{n\to+\infty}\int_1^{n+1}f(x)\dif x \\
		&=\int_1^{+\infty}f(x)\dif x
	\end{align*}
	因此级数收敛。\par
	(2)如果级数收敛,对任意的$H>0$,令$N=[H]$(取整函数),则:
	\begin{align*}
		\int_1^{+\infty}f(x)\dif x
		&=\lim_{H\to+\infty}\int_1^{H}f(x)\dif x \\
		&\leqslant\lim_{H\to+\infty}\int_1^{N+1}f(x)\dif x \\
		&=\lim_{N\to+\infty}\int_1^{N+1}f(x)\dif x \\
		&=\lim_{N\to+\infty}\sum_{n=1}^{N}[F(n+1)-F(n)] \\
		&=\sum_{n=1}^{+\infty}[F(n+1)-F(n)]
	\end{align*}
	因此广义积分收敛。
\end{proof}
\begin{theorem}
	设$f(x)$在$[1,+\infty)$单调下降且非负,则级数:
	\begin{equation*}
		\sum_{n=1}^{+\infty}f(n)
	\end{equation*}
	与广义积分:
	\begin{equation*}
		\int_{1}^{+\infty}f(x)\dif x
	\end{equation*}
	同敛态。
\end{theorem}
\begin{proof}
	(1)如果广义积分收敛,则级数:
	\begin{equation*}
		\sum_{n=2}^{+\infty}[F(n)-F(n-1)]
	\end{equation*}
	收敛,由$f(x)$单调下降且非负,有:
	\begin{equation*}
		f(n)\leqslant\int_{n-1}^{n}f(x)\dif x=F(n)-F(n-1)
	\end{equation*}
	由比较判别法,级数收敛。\par
	(2)如果广义积分发散,则级数:
	\begin{equation*}
		\sum_{n=1}^{+\infty}[F(n+1)-F(n)]
	\end{equation*}
	发散,由$f(x)$单调下降且非负,有:
	\begin{equation*}
		f(n)\geqslant\int_{n}^{n+1}f(x)\dif x=F(n+1)-F(n)
	\end{equation*}
	由比较判别法,级数发散。
\end{proof}

\subsection{比较尺度问题}
由柯西积分判别法,我们可以很容易地判断以下级数是否收敛:
\begin{enumerate}
	\item $\sum\limits_{n=1}^{+\infty}\frac{1}{n^p}$。
	\item $\sum\limits_{n=2}^{+\infty}\frac{1}{n(\ln n)^p}$。
	\item $\sum\limits_{n=3}^{+\infty}\frac{1}{n\ln n(\ln\ln n)^p}$。
	\item $\cdots$
\end{enumerate}
上述级数在$p>1$时都收敛,反之发散。\par
我们会谈论比较尺度的问题,简单来讲就是说对于一个级数的敛散性问题,如果a判别法无法判别但b判别法可以,那么b的比较尺度应该是更加精细的。这个问题局限于判别法,但也不局限于判别法,如何理解呢?D'Alembert判别法和Raabe判别法的背后其实都是比较判别法,方法是一样的,但选取的比较级数不一样,也带来了它们比较尺度的不一样。上面这句话不是很直观,我们来看上面提到的由柯西积分法带来的三个级数(其实不止三个,按照换元积分法的规则,可以利用的级数能够无穷无尽地写下去)。可以看出,越往下写,通项在$n$相同时就会变得越大。也就是说它的尺度会变得更精细,之前无法判断为收敛的现在可以了。

\subsection{Raabe判别法}
以下定理我们称之为Raabe判别法。该判别法的实质其实就是将$\sum\limits_{n=1}^{+\infty}a_n$与$\sum\limits_{n=1}^{+\infty}\dfrac{1}{n^p}$作比较。当$p>1$时,由Cauchy积分判别法易证$\sum\limits_{n=1}^{+\infty}\dfrac{1}{n^p}$收敛。
\subsubsection{一般形式}
\begin{theorem}
	设$\sum\limits_{n=1}^{+\infty}a_n$是严格正项级数。
	\begin{enumerate}
		\item  如果$\exists\;q>1,\;\exists\;N\in\mathbb{N}^+$,使得对任意的$n>N$有:
		\begin{equation*}
			n\left(\frac{a_n}{a_{n+1}}-1\right)\geqslant q
		\end{equation*}
		则$\sum\limits_{n=1}^{+\infty}a_n$收敛。
		\item  如果$\exists\;N\in\mathbb{N}^+$,使得对任意的$n>N$有:
		\begin{equation*}
			n\left(\frac{a_n}{a_{n+1}}-1\right)\leqslant1
		\end{equation*}
		则$\sum\limits_{n=1}^{+\infty}a_n$发散。
	\end{enumerate}
\end{theorem}
\begin{proof}
	(1)所给的条件等价于对任意的$n>N$有:
	\begin{equation*}
		\frac{a_n}{a_{n+1}}\geqslant1+\frac{q}{n}
	\end{equation*}
	取$p\in\mathbb{R}$满足$1<p<q$,取级数$\sum\limits_{n=1}^{+\infty}\dfrac{1}{n^p}$,令$b_n=\frac{1}{n^p}$。当$n$足够大的时候,有:
	\begin{align*}
		\frac{b_n}{b_{n+1}}&=\frac{(n+1)^p}{n^p} \\
		&=(1+\frac{1}{n})^p \\
		&=1+\frac{p}{n}+O(\frac{1}{n^2}) \\
		&<1+\frac{q}{n}\leqslant\frac{a_n}{a_{n+1}}
	\end{align*}
	由比值判别法可得$\sum\limits_{n=1}^{+\infty}a_n$收敛。\par
	(2)所给的条件等价于对任意的$n>N$有:
	\begin{equation*}
		\frac{a_n}{a_{n+1}}\leqslant1+\frac{1}{n}=\frac{\frac{1}{n}}{\frac{1}{n+1}}
	\end{equation*}
	由比值判别法可得$\sum\limits_{n=1}^{+\infty}a_n$发散。
\end{proof}
\subsubsection{上下极限形式}
\begin{theorem}
	设$\sum\limits_{n=1}^{+\infty}a_n$是严格正项级数。
	\begin{enumerate}
		\item 如果:
		\begin{equation*}
			\varliminf n\left(\frac{a_{n}}{a_{n+1}}-1\right)>1
		\end{equation*}
		则$\sum\limits_{n=1}^{+\infty}a_n$收敛。
		\item 如果:
		\begin{equation*}
			\varlimsup n\left(\frac{a_{n}}{a_{n+1}}-1\right)<1
		\end{equation*}
		则$\sum\limits_{n=1}^{+\infty}a_n$发散。
	\end{enumerate}
\end{theorem}
\subsubsection{极限形式}
\begin{theorem}
	设$\sum\limits_{n=1}^{+\infty}a_n$是严格正项级数。且:
	\begin{equation*}
		\lim_{n\to+\infty} n\left(\frac{a_{n}}{a_{n+1}}-1\right)=q
	\end{equation*}
	\begin{enumerate}
		\item 若$q>1$,则$\sum\limits_{n=1}^{+\infty}a_n$收敛。
		\item 若$q<1$,则$\sum\limits_{n=1}^{+\infty}a_n$发散。
	\end{enumerate}
\end{theorem}

\subsection{Gauss判别法}
下面介绍Gauss判别法,它可以概括D'Alembert判别法与Raabe判别法,在比较尺度上能够达到$\sum\limits_{n=2}^{+\infty}\frac{1}{n(\ln n)^p}$的精度。
\begin{theorem}
	设$\sum\limits_{n=1}^{+\infty}a_n$是严格正项级数。若:
	\begin{equation*}
		\frac{a_n}{a_{n+1}}=\lambda+\frac{\mu}{n}+\frac{\nu}{n\ln n}+o(\frac{1}{n\ln n})
	\end{equation*}
	则:
	\begin{enumerate}
		\item 若$\lambda>1$,则级数$\sum\limits_{n=1}^{+\infty}a_n$收敛;若$\lambda<1$,则级数$\sum\limits_{n=1}^{+\infty}a_n$发散;
		\item 若$\lambda=1,\;\mu>1$,则级数$\sum\limits_{n=1}^{+\infty}a_n$收敛;若$\lambda=1,\;\mu<1$,则级数$\sum\limits_{n=1}^{+\infty}a_n$发散;
		\item 若$\lambda=1,\;\mu=1,\;\nu>1$,则级数$\sum\limits_{n=1}^{+\infty}a_n$收敛;若$\lambda=1,\;\mu=1,\;\nu<1$,则级数$\sum\limits_{n=1}^{+\infty}a_n$发散。
	\end{enumerate}
\end{theorem}
\begin{proof}
	(1)可以归结为D'Alembert判别法;(2)可以归结为Raabe判别法;\par
	(3)我们以级数:
	\begin{equation*}
		\sum\limits_{n=2}^{+\infty}b_n=\sum\limits_{n=2}^{+\infty}\frac{1}{n(\ln n)^p}
	\end{equation*}
	作为比较的尺度。计算可得:
	\begin{align*}
		\frac{b_n}{b_{n+1}}
		&=\frac{(n+1)(\ln(n+1))^p}{n(\ln n)^p} \\
		&=\left(1+\frac{1}{n}\right)\left(1+\frac{\ln(1+\frac{1}{n})}{\ln n}\right)^p \\
		&=\left(1+\frac{1}{n}\right)\left(1+\frac{\ln(1+\frac{1}{n})}{\ln n}\right)^p \\
		&=\left(1+\frac{1}{n}\right)\left(1+\frac{1}{n\ln n}+o(\frac{1}{n\ln n})\right)^p \\
		&=1+\frac{1}{n}+\frac{p}{n\ln n}+o\left(\frac{1}{n\ln n}\right)
	\end{align*}
	其中第三行到第四行利用了泰勒展开。\par
	回归原级数,如果$\lambda=\mu=1,\;\nu>1$,则可以选取$p\in\mathbb{R}$,使得:
	\begin{equation*}
		1<p<\nu
	\end{equation*}
	当$n$足够大的时候,就有:
	\begin{equation*}
		\frac{a_n}{a_{n+1}}\geqslant\frac{b_n}{b_{n+1}}
	\end{equation*}
	此时$\sum\limits_{n=1}^{+\infty}b_n$收敛,由比值判别法,级数$\sum\limits_{n=1}^{+\infty}a_n$收敛。\par
	如果$\lambda=\mu=1,\;\nu<1$,则可以选取$p\in\mathbb{R}$,使得:
	\begin{equation*}
		\nu<p<1
	\end{equation*}
	当$n$足够大的时候,就有:
	\begin{equation*}
		\frac{a_n}{a_{n+1}}\leqslant\frac{b_n}{b_{n+1}}
	\end{equation*}
	此时$\sum\limits_{n=1}^{+\infty}b_n$发散,由比值判别法,级数$\sum\limits_{n=1}^{+\infty}a_n$发散。
\end{proof}








\section{任意项级数}

本节讨论\gls{ArbitrarySeries},即不对各项的正负性做出要求的级数。

\subsection{条件收敛的判别}
\subsubsection{Abel引理}
\begin{lemma}
	设$\alpha_i,\beta_i,\;i=1,2,\dots,p$是实数,且:
	\begin{equation*}
		B_k=\sum_{i=1}^k\beta_i,\;k=1,2,\dots,p
	\end{equation*}
	则有:
	\begin{enumerate}
		\item $\sum\limits_{i=1}^p\alpha_i\beta_i=\sum\limits_{i=1}^{p-1}(\alpha_i-\alpha_{i+1})B_i+\alpha_pB_p$
		\item 如果$\{\alpha_i\}$单调,并且有:
		\begin{equation*}
			|B_k|\leqslant L,\;k=1,2,\dots,p
		\end{equation*}
		那么就有:
		\begin{equation*}
			\left|\sum_{i=1}^p\alpha_i\beta_i\right|\leqslant L\left(|\alpha_1|+2|\alpha_p|\right)
		\end{equation*}
	\end{enumerate}
\end{lemma}
(1)这个公式又称为\gls{AbelTrans}、\gls{SumByParts}。之所以被称之为分部求和公式,是因为它可以写为如下形式:
\begin{equation*}
	\sum_{i=1}^p\alpha_i\Delta B_i = \alpha_jB_j\Big|_{j=0}^p-\sum_{i=1}^{p-1}B_i\Delta\alpha_i
\end{equation*}
其中:
\begin{gather*}
	a_0=0,\quad B_0=0 \\
	\Delta B_k=B_k-B_{k-1}=\beta_k,\quad k=1,2,\dots,p, \\
	\Delta\alpha_i=\alpha_{i+1}-\alpha_i,\quad i=1,2,\dots,p-1
\end{gather*}
\begin{proof}
	记$B_0=0$,于是有:\par
	(1)
	\begin{align*}
		\sum_{i=1}^p\alpha_i\beta_i
		&=\sum_{i=1}^p\alpha_i(B_i-B_{i-1}) \\
		&=\sum_{i=1}^p\alpha_iB_i-\sum_{i=1}^p\alpha_iB_{i-1} \\
		&=\sum_{i=1}^p\alpha_iB_i-\sum_{i=0}^{p-1}\alpha_{i+1}B_{i} \\
		&=\sum_{i=1}^p\alpha_iB_i-\sum_{i=1}^{p-1}\alpha_{i+1}B_{i} \\
		&=\sum_{i=1}^{p-1}(\alpha_i-\alpha_{i+1})B_i+\alpha_pB_p
	\end{align*}\par
	(2)下第三行到第四行利用了$\{\alpha_i\}$的单调性。
	\begin{align*}
		\left|\sum_{i=1}^p\alpha_i\beta_i\right|
		&=\left|\sum_{i=1}^{p-1}(\alpha_i-\alpha_{i+1})B_i+\alpha_pB_p\right| \\
		&\leqslant\sum_{i=1}^{p-1}|\alpha_i-\alpha_{i+1}|\;|B_i|+|\alpha_p|\;|B_p| \\
		&\leqslant L\left(\sum_{i=1}^{p-1}|\alpha_i-\alpha_{i+1}|+|\alpha_p|\right) \\
		&=L\left(|\alpha_1-\alpha_p|+|\alpha_p|\right) \\
		&\leqslant L\left(|\alpha_1|+2|\alpha_p|\right)\qedhere
	\end{align*}
\end{proof}

\subsubsection{Dirichlet判别法}
\begin{theorem}
	对于级数$\sum\limits_{n=1}^{+\infty}a_nb_n$,如果:
	\begin{enumerate}
		\item 序列$\{a_n\}$单调收敛于$0$;
		\item 序列$\{\sum\limits_{i=1}^nb_i\}$有界;
	\end{enumerate}
	那么级数收敛。
\end{theorem}
\begin{proof}
	我们来估计:
	\begin{equation*}
		\left|\sum_{i=n+1}^{n+p}a_ib_i\right|
	\end{equation*}
	令:
	\begin{equation*}
		B_k=\sum_{i=1}^{n+k}b_i
	\end{equation*}
	因为序列$\{\sum\limits_{i=1}^nb_i\}$有界,因此有:
	\begin{equation*}
		\left|\sum\limits_{i=1}^nb_i\right|\leqslant L,\;n\in\mathbb{N}^+
	\end{equation*}
	也就有:
	\begin{equation*}
		|B_k|=\left|\sum_{i=1}^{n+k}b_i-\sum_{i=1}^nb_i\right|\leqslant 2L
	\end{equation*}
	因为$\{\alpha_i\}$单调,于是:
	\begin{equation*}
		\left|\sum_{i=n+1}^{n+p}a_ib_i\right|\leqslant 2L(|\alpha_{n+1}|+2|\alpha_{n+p}|)
	\end{equation*}
	又因为$\{\alpha_i\}\rightarrow0$,显然对任意的$\varepsilon>0$,当$n$足够大时可以有:
	\begin{equation*}
		\left|\sum_{i=n+1}^{n+p}a_ib_i\right|<\varepsilon
	\end{equation*}
	由Cauchy收敛准则,级数$\sum\limits_{n=1}^{+\infty}a_nb_n$收敛。
\end{proof}

\subsubsection{Abel判别法}
\begin{theorem}
	对于级数$\sum\limits_{n=1}^{+\infty}a_nb_n$,如果:
	\begin{enumerate}
		\item 序列$\{a_n\}$单调有界;
		\item 级数$\sum\limits_{n=1}^{+\infty}b_n$收敛;
	\end{enumerate}
	那么级数收敛。
\end{theorem}
\begin{proof}
	因为$\{a_n\}$单调有界,由实数序列的单调收敛原理,可以设:
	\begin{equation*}
		\lim_{n\to+\infty}a_n=a
	\end{equation*}
	则序列$\{a_n-a\}$单调趋于$0$。又因级数$\sum\limits_{n=1}^{+\infty}b_n$收敛,所以序列$\{\sum\limits_{i=1}^nb_i\}$有界。由Dirichlet判别法,级数$\sum\limits_{n=1}^{+\infty}(a_n-a)b_n$收敛。又因为级数$\sum\limits_{n=1}^{+\infty}ab_n$收敛,而:
	\begin{equation*}
		\sum_{n=1}^{+\infty}a_nb_n=\sum_{n=1}^{+\infty}\left[(a_n-a)b_n+ab_n\right]=\sum_{n=1}^{+\infty}\left[(a_n-a)b_n\right]+\sum_{n=1}^{+\infty}ab_n
	\end{equation*}
	因此级数$\sum\limits_{n=1}^{+\infty}a_nb_n$收敛。
\end{proof}

\subsubsection{Leibniz判别法}
\begin{theorem}
	设序列$\{a_n\}$单调且收敛于$0$,则以下级数收敛:
	\begin{equation*}
		\sum_{n=1}^{+\infty}(-1)^{n-1}a_n
	\end{equation*}
\end{theorem}
\begin{proof}
	使用Dirichlet判别法可直接证得。
\end{proof}



























\section{收敛级数的性质}

\subsection{收敛级数的可结合性}
\begin{theorem}
	设有收敛级数:
	\begin{equation*}
		a_1+a_2+\cdots+a_n+\cdots
	\end{equation*}
	如果把这个级数的若干个相继的项归并为一项,即将该级数变为如下形式:
	\begin{equation*}
		(a_1+a_2+\cdots+a_{n_1})+(a_{n_1+1}+a_{n_1+2}+\cdots+a_{n_2})+\cdots+(a_{n_k+1}+a_{n_k+2}+\cdots+a_{n_{k+1}})+\cdots
	\end{equation*}
	则结合后的级数仍然收敛,且与原级数有相等的和。
\end{theorem}
\begin{proof}
	结合后级数的部分和序列是原级数部分和序列的子列。
\end{proof}
如果原级数为定号级数(即正项级数或负项级数),逆命题也成立。若不定号,考虑级数:
\begin{equation*}
	(1-1)+(1-1)+\cdots+(1-1)+\cdots
\end{equation*}
这个级数当然是收敛的,但级数:
\begin{equation*}
	1-1+1-1+\cdots+(-1)^{n-1}+\cdots
\end{equation*}
显然还是发散的。

\subsection{绝对收敛级数的性质}
\subsubsection{可交换性}
设$\sum\limits_{n=1}^{+\infty}a_n$是一个级数,我们对其进行重排,即把该序列中的所有项无重复、无遗漏地改变一个顺序重新排出来。用符号表示即为:
\begin{equation*}
	\alpha^{'}_n=\alpha_{\varphi(n)}
\end{equation*}
其中$\varphi$是一个从$\mathbb{N}^+$到$\mathbb{N}^+$的双射。
\begin{theorem}
	若级数$\sum\limits_{n=1}^{+\infty}a_n$绝对收敛,则重排后的级数$\sum\limits_{n=1}^{+\infty}a^{'}_n$也绝对收敛,并且二者值相等。
\end{theorem}
\begin{proof}
	我们先来讨论正项级数,再来讨论任意项级数:\par
	(1)正项级数:\par
	设级数$\sum\limits_{n=1}^{+\infty}a_n$为绝对收敛的正项级数,则级数$\sum\limits_{n=1}^{+\infty}a^{'}_n$也是一个正项级数。由题目条件,显然可得:
	\begin{equation*}
		\sum_{n=1}^Na^{'}_n\leqslant\sum_{n=1}^{+\infty}a_n,\;\forall\;N\in\mathbb{N}^+
	\end{equation*}
	由正项级数的收敛原理,级数$\sum\limits_{n=1}^{+\infty}a^{'}_n$收敛。由\cref{prop:RSeq}(6)也有:
	\begin{equation*}
		\sum_{n=1}^{+\infty}a^{'}_n\leqslant\sum_{n=1}^{+\infty}a_n
	\end{equation*}
	反之,可认为级数$\sum\limits_{n=1}^{+\infty}a_n$是由级数$\sum\limits_{n=1}^{+\infty}a^{'}_n$重排后的结果,因此也可得到:
	\begin{equation*}
		\sum_{n=1}^{+\infty}a_n\leqslant\sum_{n=1}^{+\infty}a^{'}_n
	\end{equation*}
	于是:
	\begin{equation*}
		\sum_{n=1}^{+\infty}a_n=\sum_{n=1}^{+\infty}a^{'}_n
	\end{equation*}
	(2)任意项级数:
	设级数$\sum\limits_{n=1}^{+\infty}a_n$为绝对收敛的任意项级数。令:
	\begin{equation*}
		p_n=\frac{|a_n|+a_n}{2},\;q_n=\frac{|a_n|-a_n}{2},\;n\in\mathbb{N}^+
	\end{equation*}
	显然有:
	\begin{equation*}
		0\leqslant p_n\leqslant|a_n|,\;0\leqslant q_n\leqslant|a_n|,\;n\in\mathbb{N}^+
	\end{equation*}
	取比较级数$\sum\limits_{n=1}^{+\infty}|a_n|$,由正项级数的比较判别法,$\sum\limits_{n=1}^{+\infty}p_n$与$\sum\limits_{n=1}^{+\infty}q_n$都是正项收敛级数。由(1),任意重排后的级数$\sum\limits_{n=1}^{+\infty}p^{'}_n$和$\sum\limits_{n=1}^{+\infty}q^{'}_n$也都收敛,并且有:
	\begin{equation*}
		\sum_{n=1}^{+\infty}p^{'}_n=\sum_{n=1}^{+\infty}p_n,\quad
		\sum_{n=1}^{+\infty}q^{'}_n=\sum_{n=1}^{+\infty}q_n
	\end{equation*}
	因此级数:
	\begin{equation*}
		\sum_{n=1}^{+\infty}|a^{'}_n|=\sum_{n=1}^{+\infty}(p^{'}_n-q^{'}_n)
	\end{equation*}
	由级数的线性运算也收敛,即级数$\sum\limits_{n=1}^{+\infty}a^{'}_n$绝对收敛,并且有:
	\begin{align*}
		\sum_{n=1}^{+\infty}a^{'}_n
		&=\sum_{n=1}^{+\infty}(p^{'}_n-q^{'}_n) \\
		&=\sum_{n=1}^{+\infty}p^{'}_n-\sum_{n=1}^{+\infty}q^{'}_n \\
		&=\sum_{n=1}^{+\infty}p_n-\sum_{n=1}^{+\infty}q_n \\
		&=\sum_{n=1}^{+\infty}(p_n-q_n) \\
		&=\sum_{n=1}^{+\infty}a_n
	\end{align*}
	第一行到第二行利用级数的线性运算,第二行到第三行利用(1)的结果,第三行到第四行再次利用级数的线性运算。
\end{proof}


\subsubsection{条件收敛级数并不满足可交换性}
\begin{theorem}
	设$\sum\limits_{n=1}^{+\infty}a_n$是一个条件收敛级数,则对任意的$\xi\in\overline{\mathbb{R}}$,都存在$\sum\limits_{n=1}^{+\infty}a_n$的一个重排级数$\sum\limits_{n=1}^{+\infty}a^{'}_n$,使:
	\begin{equation*}
		\sum_{n=1}^{+\infty}a^{'}_n=\xi
	\end{equation*}
\end{theorem}
\begin{proof}
	(1)对于$\xi\in\mathbb{R}$:\par
	令:
	\begin{equation*}
		p_n=\frac{|a_n|+a_n}{2},\;q_n=\frac{|a_n|-a_n}{2},\;n\in\mathbb{N}^+
	\end{equation*}
	显然$\sum\limits_{n=1}^{+\infty}p_n$与$\sum\limits_{n=1}^{+\infty}q_n$都是正项级数,并且有:
	\begin{gather*}
		\lim_{n\to+\infty}p_n=\lim_{n\to+\infty}\frac{|a_n|+a_n}{2}=0 \\
		\lim_{n\to+\infty}q_n=\lim_{n\to+\infty}\frac{|a_n|-a_n}{2}=0 \\
		\sum_{n=1}^{+\infty}p_n=\sum_{n=1}^{+\infty}\frac{|a_n|+a_n}{2}=\frac{1}{2}\sum_{n=1}^{+\infty}|a_n|+\frac{1}{2}\sum_{n=1}^{+\infty}a_n=+\infty \\
		\sum_{n=1}^{+\infty}q_n=\sum_{n=1}^{+\infty}\frac{|a_n|-a_n}{2}=\frac{1}{2}\sum_{n=1}^{+\infty}|a_n|-\frac{1}{2}\sum_{n=1}^{+\infty}a_n=+\infty
	\end{gather*}
	前两个公式可由Cauchy收敛准则的必要性推得。接下来来考察序列:
	\begin{equation*}
		a_1,a_2,\cdots,a_n,\cdots
	\end{equation*}
	令$P_n$表示这个序列中第$n$个非负项,以$Q_n$表示其中第$n$个负项的绝对值。则$\{P_n\}$是$\{p_n\}$去除一部分值为$0$的项后剩下的子序列(去除的是$a_n\leqslant0$导致$p_n=0$的项),$\{Q_n\}$是$\{q_n\}$去除所有值为$0$的项后剩下的子序列(即$a_n\geqslant0$导致$q_n=0$的项)。由$\{p_n\}$和$\{q_n\}$作为实数序列与作为级数项的收敛性,可得到:
	\begin{gather*}
		\lim_{n\to+\infty}P_n=\lim_{n\to+\infty}Q_n=0 \\
		\sum_{n=1}^{+\infty}P_n=\sum_{n=1}^{+\infty}Q_n=+\infty
	\end{gather*}
	同时注意到,$\{P_n\}$与$\{-Q_n\}$的各项其实都是原本$\{a_n\}$中的某项。我们依次考察$P_1,P_2,\dots$中的各项,设$P_{m_1}$是第一个满足以下条件的项(存在性由$\sum\limits_{n=1}^{+\infty}P_n=+\infty$保证):
	\begin{equation*}
		P_1+P_2+\cdots+P_{m_1}>\xi
	\end{equation*}
	再依次考察$Q_1,Q_2,\dots$中的各项,设$Q_{n_1}$是第一个满足以下条件的项(存在性由$\sum\limits_{n=1}^{+\infty}Q_n=+\infty$保证):
	\begin{equation*}
		P_1+P_2+\cdots+P_{m_1}-Q_1-Q_2-\cdots-Q_{n_1}<\xi
	\end{equation*}
	再考虑考察$P_{m_1+1},P_{m_1+2},\dots$中的各项,设$P_{m_2}$是第一个满足以下条件的项:
	\begin{gather*}
		P_1+P_2+\cdots+P_{m_1}-Q_1-Q_2-\cdots-Q_{n_1} \\
		+P_{m_1+1}+P_{m_1+2}+\cdots+P_{m_2}>\xi
	\end{gather*}
	不断重复下去,我们可以得到$\sum\limits_{n=1}^{+\infty}a_n$的一个重排级数$\sum\limits_{n=1}^{+\infty}a^{'}_n$:
	\begin{gather*}
		P_1+P_2+\cdots+P_{m_1}-Q_1-Q_2-\cdots-Q_{n_1} \\
		+P_{m_1+1}+P_{m_1+2}+\cdots+P_{m_2}-Q_{n_1+1}-Q_{n_1+2}-\cdots-Q_{n_2} \\
		\vdots
	\end{gather*}
	令$R_k$和$L_k$分别表示级数$\sum\limits_{n=1}^{+\infty}a^{'}_n$末项为$P_{m_k}$的部分和与末项为$Q_{n_k}$的部分和,则由:
	\begin{gather*}
		|R_k-\xi|\leqslant P_{m_k},\;k=2,3,\dots, \\
		|L_k-\xi|\leqslant Q_{n_k},\;k=1,2,3,\dots,
	\end{gather*}
	而:
	\begin{equation*}
		\lim_{n\to+\infty}P_{m_k}=\lim_{n\to+\infty}Q_{n_k}=0
	\end{equation*}
	所以:
	\begin{equation*}
		\lim_{n\to+\infty}R_k=\lim_{n\to+\infty}L_k=\xi
	\end{equation*}
	因为级数$\sum\limits_{n=1}^{+\infty}a^{'}_n$的任意一个部分和$S^{'}_n$必定介于一对$R_k$和$L_k$之间(因为$P_{m_k}$和$Q_{n_k}$都是取的第一个满足条件的),且随着$n$的增大,可以让$k$也随之增大直至无穷。因此由\cref{prop:RSeq}(4):
	\begin{equation*}
		\sum_{n=1}^{+\infty}a^{'}_n=\xi
	\end{equation*}
	(2)对于$\xi$为正无穷或负无穷的情况,仅对正无穷情况进行讨论,负无穷类似:\par
	任取一个单调上升且趋于无穷的实数数列$\{\xi_n\}$。沿用(1)中的记号。
	设$P_{m_1}$是第一个满足以下条件的项(存在性由$\sum\limits_{n=1}^{+\infty}P_n=+\infty$保证):
	\begin{equation*}
		P_1+P_2+\cdots+P_{m_1}>\xi_1
	\end{equation*}
	再依次考察$Q_1,Q_2,\dots$中的各项,设$Q_{n_1}$是第一个满足以下条件的项(存在性由$\sum\limits_{n=1}^{+\infty}Q_n=+\infty$保证):
	\begin{equation*}
		P_1+P_2+\cdots+P_{m_1}-Q_1-Q_2-\cdots-Q_{n_1}<\xi_1
	\end{equation*}
	再考虑考察$P_{m_1+1},P_{m_1+2},\dots$中的各项,设$P_{m_2}$是第一个满足以下条件的项:
	\begin{gather*}
		P_1+P_2+\cdots+P_{m_1}-Q_1-Q_2-\cdots-Q_{n_1} \\
		+P_{m_1+1}+P_{m_1+2}+\cdots+P_{m_2}>\xi_2
	\end{gather*}
	不断重复下去,我们可以得到$\sum\limits_{n=1}^{+\infty}a_n$的一个重排级数$\sum\limits_{n=1}^{+\infty}a^{'}_n$:
	\begin{gather*}
		P_1+P_2+\cdots+P_{m_1}-Q_1-Q_2-\cdots-Q_{n_1} \\
		+P_{m_1+1}+P_{m_1+2}+\cdots+P_{m_2}-Q_{n_1+1}-Q_{n_1+2}-\cdots-Q_{n_2} \\
		\vdots
	\end{gather*}
	而这个重排级数显然满足:
	\begin{equation*}
		\sum_{n=1}^{+\infty}a^{'}_n=+\infty
	\end{equation*}
\end{proof}













\section{级数乘法}

本节介绍\gls{SeriesMulti}。\par
对于两个无穷级数:
\begin{equation*}
	\sum_{n=1}^{+\infty}a_n,\quad\sum_{n=1}^{+\infty}b_n
\end{equation*}
它们的乘积可以写作如下形式:
\begin{equation*}
	\begin{aligned}
		&a_1b_1	 &&a_1b_2	&&a_1b_3	&&\cdots\\
		&a_2b_1	 &&a_2b_2	&&a_2b_3	&&\cdots\\
		&\vdots	 &&\vdots	&&\vdots\\
		&a_ib_1	 &&a_ib_2	&&a_ib_3	&&\cdots\\
		&\vdots	 &&\vdots	&&\vdots
	\end{aligned}
\end{equation*}
这些数可以用很多种方式排成数列。\par
我们称以下排列方式为三角形排列:
\begin{equation*}
	a_1b_1,a_1b_2,a_2b_1,a_1b_3,a_2b_2,a_3b_1,a_1b_4\dots
\end{equation*}
称以下排列方式为正方形排列:
\begin{equation*}
	a_1b_1,a_1b_2,a_2b_2,a_2b_1,a_1b_3,a_2b_3,a_3b_3\dots
\end{equation*}
\begin{theorem}
	如果级数$\sum\limits_{n=1}^{+\infty}a_n$和级数$\sum\limits_{n=1}^{+\infty}b_n$绝对收敛,并且:
	\begin{equation*}
		\sum_{n=1}^{+\infty}a_n=A,\quad\sum_{n=1}^{+\infty}b_n=B
	\end{equation*}
	则这两个级数的无穷乘积中的元素$a_ib_j$按任意方式排列成的级数都是绝对收敛的,并且其和为$AB$。
\end{theorem}
\begin{proof}
	设$a_{i_k}b_{j_k},\;k\in\mathbb{N}^+$是$a_ib_j,\;i,j\in\mathbb{N}^+$的任意一种排列。把$i_1,i_2,\dots,i_n,\;j_1,j_2,\dots,j_n$中最大的记为$N$,则:
	\begin{align*}
		\sum_{k=1}^n|a_{i_k}b_{j_k}|
		&\leqslant\sum_{i=1}^N|a_i|\sum_{j=1}^N|b_j| \\
		&\leqslant\sum_{i=1}^{+\infty}|a_i|\sum_{j=1}^{+\infty}|b_j|
	\end{align*}
	由\cref{prop:RSeq}(6):
	\begin{equation*}
		\sum_{n=1}^{+\infty}a_{i_k}b_{j_k}
	\end{equation*}
	绝对收敛。按正方形形式重排该级数可得到:
	\begin{align*}
		\sum_{n=1}^{+\infty}a_{i_k}b_{j_k}
		&=\lim_{n\to+\infty}\left(\sum_{i=1}^na_i\right)\left(\sum_{i=1}^nb_i\right) \\
		&=\left(\sum_{i=1}^{+\infty}a_i\right)\left(\sum_{i=1}^{+\infty}b_i\right) \\
		&=AB\qedhere
	\end{align*}
\end{proof}
