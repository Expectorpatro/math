\section{级数乘法}

本节介绍\gls{SeriesMulti}。\par
对于两个无穷级数:
\begin{equation*}
	\sum_{n=1}^{+\infty}a_n,\quad\sum_{n=1}^{+\infty}b_n
\end{equation*}
它们的乘积可以写作如下形式:
\begin{equation*}
	\begin{aligned}
		&a_1b_1	 &&a_1b_2	&&a_1b_3	&&\cdots\\
		&a_2b_1	 &&a_2b_2	&&a_2b_3	&&\cdots\\
		&\vdots	 &&\vdots	&&\vdots\\
		&a_ib_1	 &&a_ib_2	&&a_ib_3	&&\cdots\\
		&\vdots	 &&\vdots	&&\vdots
	\end{aligned}
\end{equation*}
这些数可以用很多种方式排成数列。\par
我们称以下排列方式为三角形排列:
\begin{equation*}
	a_1b_1,a_1b_2,a_2b_1,a_1b_3,a_2b_2,a_3b_1,a_1b_4\dots
\end{equation*}
称以下排列方式为正方形排列:
\begin{equation*}
	a_1b_1,a_1b_2,a_2b_2,a_2b_1,a_1b_3,a_2b_3,a_3b_3\dots
\end{equation*}
\begin{theorem}
	如果级数$\sum\limits_{n=1}^{+\infty}a_n$和级数$\sum\limits_{n=1}^{+\infty}b_n$绝对收敛,并且:
	\begin{equation*}
		\sum_{n=1}^{+\infty}a_n=A,\quad\sum_{n=1}^{+\infty}b_n=B
	\end{equation*}
	则这两个级数的无穷乘积中的元素$a_ib_j$按任意方式排列成的级数都是绝对收敛的,并且其和为$AB$。
\end{theorem}
\begin{proof}
	设$a_{i_k}b_{j_k},\;k\in\mathbb{N}^+$是$a_ib_j,\;i,j\in\mathbb{N}^+$的任意一种排列。把$i_1,i_2,\dots,i_n,\;j_1,j_2,\dots,j_n$中最大的记为$N$,则:
	\begin{align*}
		\sum_{k=1}^n|a_{i_k}b_{j_k}|
		&\leqslant\sum_{i=1}^N|a_i|\sum_{j=1}^N|b_j| \\
		&\leqslant\sum_{i=1}^{+\infty}|a_i|\sum_{j=1}^{+\infty}|b_j|
	\end{align*}
	由极限的不等式性:
	\begin{equation*}
		\sum_{n=1}^{+\infty}a_{i_k}b_{j_k}
	\end{equation*}
	绝对收敛。按正方形形式重排该级数可得到:
	\begin{align*}
		\sum_{n=1}^{+\infty}a_{i_k}b_{j_k}
		&=\lim_{n\to+\infty}\left(\sum_{i=1}^na_i\right)\left(\sum_{i=1}^nb_i\right) \\
		&=\left(\sum_{i=1}^{+\infty}a_i\right)\left(\sum_{i=1}^{+\infty}b_i\right) \\
		&=AB\qedhere
	\end{align*}
\end{proof}
