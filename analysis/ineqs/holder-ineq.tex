\section{Holder不等式}

\subsection{离散形式的Holder不等式}
\subsubsection{有穷级数形式}
\begin{theorem}
	设$\xi_i,\eta_i,i=1,2,\dots,n$同为实数或复数,$1<p,q<+\infty$,满足$\frac{1}{p}+\frac{1}{q}=1$,则有如下不等式:
	\begin{inequality*}\label{ineq:holder-ineq-finite-series}
		\sum_{i=1}^n|\xi_i\eta_i|\leqslant\left(\sum_{i=1}^n|\xi_i|^p\right)^\frac{1}{p}\cdot\left(\sum_{i=1}^n|\eta_i|^q\right)^\frac{1}{q}
	\end{inequality*}
\end{theorem}
\begin{proof}
	在Young's不等式(即\cref{ineq:young-ineq-Simple})中取:
	\begin{equation*}
		a=\frac{|\xi_i|}{\left(\sum\limits_{i=1}^n|\xi_i|^p\right)^\frac{1}{p}},\quad
		b=\frac{|\eta_i|}{\left(\sum\limits_{i=1}^n|\eta_i|^q\right)^\frac{1}{q}}
	\end{equation*}
	即有:
	\begin{gather*}
		\frac{|\xi_i||\eta_i|}{\left(\sum\limits_{i=1}^n|\xi_i|^p\right)^\frac{1}{p}\cdot\left(\sum\limits_{i=1}^n|\eta_i|^q\right)^\frac{1}{q}}
		\leqslant\frac{|\xi_i|^p}{p\left(\sum\limits_{i=1}^n|\xi_i|^p\right)}+\frac{|\eta_i|^q}{q\left(\sum\limits_{i=1}^n|\eta_i|^q\right)} \\
		\frac{\sum\limits_{i=1}^n|\xi_i\eta_i|}{\left(\sum\limits_{i=1}^n|\xi_i|^p\right)^\frac{1}{p}\cdot\left(\sum\limits_{i=1}^n|\eta_i|^q\right)^\frac{1}{q}}
		\leqslant\frac{\sum\limits_{i=1}^n|\xi_i|^p}{p\left(\sum\limits_{i=1}^n|\xi_i|^p\right)}+\frac{\sum\limits_{i=1}^n|\eta_i|^q}{q\left(\sum\limits_{i=1}^n|\eta_i|^q\right)}\leqslant\frac{1}{p}+\frac{1}{q}=1\qedhere
	\end{gather*}
\end{proof}
\subsubsection{无穷级数形式}
\begin{theorem}
	设$\xi_i,\eta_i,i\in\mathbb{N}^+$同为实数或复数,$1<p,q<+\infty$,满足$\frac{1}{p}+\frac{1}{q}=1$,则有如下不等式:
	\begin{inequality*}\label{ineq:holder-ineq-infty-series}
		\sum_{i=1}^{+\infty}|\xi_i\eta_i|\leqslant\left(\sum_{i=1}^{+\infty}|\xi_i|^p\right)^\frac{1}{p}\cdot\left(\sum_{i=1}^{+\infty}|\eta_i|^q\right)^\frac{1}{q}
	\end{inequality*}
\end{theorem}
\begin{proof}
	证明过程与有穷级数几乎一样,只是把求和中的$n$改为$+\infty$的区别。
\end{proof}

\subsection{积分形式的Holder不等式}
\begin{theorem}
	设	$(X,\mathscr{F},\mu)$是一个测度空间,$E\in\mathscr{F}$,$1<p,q<+\infty$,满足$\frac{1}{p}+\frac{1}{q}=1$,$f\in L_p(E),\;g\in L_q(E)$,则有如下不等式:
	\begin{inequality*}\label{ineq:holder-ineq-Lebesgue}
		\int_{E}^{}|f(x)g(x)|\dif \mu\leqslant\left[\int_{E}^{}|f(x)|^p\dif\mu\right]^{\frac{1}{p}}\left[\int_{E}^{}|g(x)|^q\dif\mu\right]^{\frac{1}{q}}
	\end{inequality*}
	等号成立当且仅当存在不全为$0$的$\alpha,\beta\geqslant0$使得:
	\begin{equation*}
		\alpha|f|^p=\beta|g|^q\;\text{a.e.于$E$}
	\end{equation*}
\end{theorem}
\begin{proof}
	\textbf{(1)$\;f(x)=0$和$g(x)=0\;$都不a.e.于$E$:}在Young's不等式(即\cref{ineq:young-ineq-Simple})中取:
	\begin{equation*}
		a=\frac{|f(x)|}{\left[\int_{E}^{}|f(x)|^p\dif\mu\right]^\frac{1}{p}},\quad
		b=\frac{|g(x)|}{\left[\int_{E}^{}|g(x)|^q\dif\mu\right]^\frac{1}{q}}
	\end{equation*}
	即有:
	\begin{equation*}
		\frac{|f(x)||g(x)|}{\left[\int_{E}^{}|f(x)|^p\dif\mu\right]^\frac{1}{p}\cdot\left[\int_{E}^{}|g(x)|^q\dif\mu\right]^\frac{1}{q}}
		\leqslant\frac{|f(x)|^p}{p\left[\int_{E}^{}|f(x)|^p\dif\mu\right]}+\frac{|g(x)|^q}{q\left[\int_{E}^{}|g(x)|^q\dif\mu\right]}
	\end{equation*}
	两边进行积分可得:
	\begin{gather*}
		\int_{E}\frac{|f(x)g(x)|}{\left[\int_{E}^{}|f(x)|^p\dif\mu\right]^\frac{1}{p}\cdot\left[\int_{E}^{}|g(x)|^q\dif\mu\right]^\frac{1}{q}}\dif\mu\leqslant\frac{1}{p}+\frac{1}{q} \\
		\int_{E}|f(x)g(x)|\dif\mu\leqslant\left(\frac{1}{p}+\frac{1}{q}\right)\left[\int_{E}^{}|f(x)|^p\dif\mu\right]^\frac{1}{p}\cdot\left[\int_{E}^{}|g(x)|^q\dif\mu\right]^\frac{1}{q} \\
		\int_{E}|f(x)g(x)|\dif\mu\leqslant\left[\int_{E}^{}|f(x)|^p\dif\mu\right]^\frac{1}{p}\cdot\left[\int_{E}^{}|g(x)|^q\dif\mu\right]^\frac{1}{q}
	\end{gather*}
	由Young's不等式可知等号成立当且仅当:
	\begin{gather*}
		\frac{|f(x)|^p}{\int_{E}^{}|f(x)|^p\dif\mu}=\frac{|g(x)|^q}{\int_{E}^{}|g(x)|^q\dif\mu} \\
		|f(x)|^p\int_{E}^{}|g(x)|^q\dif\mu=|g(x)|^q\int_{E}^{}|f(x)|^p\dif\mu
	\end{gather*}
	但使用Young's不等式接下来的是积分的操作,我们只需要积分保持不等号即可,所以由\cref{prop:NonnegativeMeasurableIntegral}(7)可知Young's不等式得到的那个不等式只要a.e.成立即可,即上式只要a.e.成立即可。\par
	\textbf{(2)$\;f(x)=0$或$g(x)=0\;$a.e.于$E$:}$|f(x)g(x)|=0\;$a.e.于$E$,由\cref{prop:NonnegativeMeasurableIntegral}(10)可得$\int_{E}|f(x)g(x)|\dif\mu=0$,再根据\cref{prop:NonnegativeMeasurableIntegral}(2)可得$\int_{A}|f(x)|^p\dif\mu,\int_{E}|g(x)|^q\dif\mu\geqslant0$,于是不等式成立。\par
	以上两种情况都可以推导出存在不全为$0$的$\alpha,\beta\geqslant0$使得:
	\begin{equation*}
		\alpha|f|^p=\beta|g|^q\;\text{a.e.于$E$}
	\end{equation*}
	即等号成立时上式也成立,下面证明反过来也对。当上不等式成立时,进行分类讨论。\par
	\textbf{(1)$\;f(x)=0$或$g(x)=0\;$a.e.于$E$:}仅对$|f(x)|=0\;$a.e.于$E$的情况进行证明,$g(x)=0\;$a.e.于$E$的情况可对称得到。此时$|f(x)|^p=0$和$|f(x)g(x)|$都a.e.于$E$,于是由\cref{prop:NonnegativeMeasurableIntegral}(10)可得:
	\begin{equation*}
		\int_{E}|f(x)g(x)|\dif\mu=0,\;\int_{E}|f(x)|^p\dif\mu=0
	\end{equation*}
	所以:
	\begin{equation*}
		\int_{E}^{}|f(x)g(x)|\dif \mu=\left[\int_{E}^{}|f(x)|^p\dif\mu\right]^{\frac{1}{p}}\left[\int_{E}^{}|g(x)|^q\dif\mu\right]^{\frac{1}{q}}=0
	\end{equation*}\par
	\textbf{(2)$\;f(x)=0$和$g(x)=0\;$都不a.e.于$E$:}设$\alpha\ne0$,对$\beta\ne0$的情况可对称得到。此时有:
	\begin{equation*}
		|f|^p=\frac{\beta}{\alpha}|g|^q\;\text{a.e.于$E$}
	\end{equation*}
	所以:
	\begin{equation*}
		|f(x)g(x)|=|f(x)||g(x)|=\left(\frac{\beta}{\alpha}\right)^{-p}|g(x)|^{\frac{q}{p}+1}
	\end{equation*}
	因为$\dfrac{1}{p}+\dfrac{1}{q}=1$,所以$\dfrac{q}{p}+1=q$,于是由\cref{prop:NonnegativeMeasurableIntegral}(6)可得:
	\begin{equation*}
		\int_{E}|f(x)g(x)|\dif\mu=\int_{E}\left(\frac{\beta}{\alpha}\right)^{-p}|g(x)|^{q}\dif\mu=\left(\frac{\beta}{\alpha}\right)^{-p}\int_{E}|g(x)|^q\dif\mu
	\end{equation*}
	对$\alpha|f|^p=\beta|g|^q$两边积分,由\cref{prop:NonnegativeMeasurableIntegral}(8)(6)可得:
	\begin{gather*}
		\int_{E}\alpha|f(x)|^p\dif\mu=\int_{E}\beta |g(x)|^q\dif\mu \\
		\alpha\int_{E}|f(x)|^p\dif\mu=\beta\int_{E}|g(x)|^q\dif\mu \\
		\frac{\int_{E}|f(x)|^p\dif\mu}{\int_{E}|g(x)|^q\dif\mu}=\frac{\beta}{\alpha} \\
		\left[\frac{\int_{E}|f(x)|^p\dif\mu}{\int_{E}|g(x)|^q\dif\mu}\right]^{-p}=\left(\frac{\beta}{\alpha}\right)^{-p}
	\end{gather*}
	将其代入到之前的式子可得:
	\begin{align*}
		\int_{E}|f(x)g(x)|\dif\mu&=\left(\frac{\beta}{\alpha}\right)^{-p}\int_{E}|g(x)|^q\dif\mu \\
		&=\left[\frac{\int_{E}|f(x)|^p\dif\mu}{\int_{E}|g(x)|^q\dif\mu}\right]^{-p}\int_{E}|g(x)|^q\dif\mu \\
		&=\left[\int_{E}|f(x)|^p\dif\mu\right]^{\frac{1}{p}}\left[\int_{E}|g(x)|^q\dif\mu\right]^{\frac{1}{q}}\qedhere
	\end{align*}
\end{proof}
\begin{theorem}
	设	$(X,\mathscr{F},\mu)$是一个测度空间,$E\in\mathscr{F}$,$1<p,q<+\infty$,满足$\frac{1}{p}+\frac{1}{q}=1$。对任意的$f\in L_p(E),\;g\in L_q(E),\;h\in L_1(E)$,在$L_1(E),L_p(E)$和$L_q(E)$中引入范数:
	\begin{equation*}
		||f||_p=\left[\int_{E}^{}|f(x)|^p\dif\mu\right]^\frac{1}{p},\;||g||_q=\left[\int_{E}|g(x)|^q\dif\mu\right]^{\frac{1}{q}},\;||h||_1=\int_{E}|h(x)|\dif\mu
	\end{equation*}
	则积分形式的Holder不等式可写为:
	\begin{inequality*}\label{ineq:holder-ineq-norm}
		||fg||_1\leqslant||f||_p\;||g||_q
	\end{inequality*}
	其中$fg\in L_1(E)$由$f\in L_p(E),g\in L_q(E)$以及积分形式的Holder不等式保证。
\end{theorem}
\begin{note}
	范数是对于空间内的元素定义的,所以写范数的时候首先要保证元素在这个空间内,这里做的其实是一个假设并验证的过程,假设在空间内,然后计算,由结果发现确实是在空间内的。
\end{note}
\begin{theorem}
	设	$(X,\mathscr{F},\mu)$是一个测度空间,$E\in\mathscr{F}$。对任意的$f\in L_1(E),\;g\in L_{\infty}(E)$,在$L_{\infty}(E)$和$L_1(E)$中分别定义范数为元素的无穷范数和:
	\begin{equation*}
		||f||_1=\int_{E}|f(x)|\dif\mu
	\end{equation*}
	则有:
	\begin{inequality*}\label{ineq:holder-ineq-norm+infty}
		||fg||_1\leqslant||f||_1\;||g||_{\infty}
	\end{inequality*}
\end{theorem}
\begin{proof}
	由\cref{prop:NonnegativeMeasurableIntegral}(8)(6)和无穷范数的定义可得:
	\begin{equation*}
		||fg||_1=\int_{E}|f(x)g(x)|\dif\mu\leqslant\int_{E}|f(x)|\;||g||_{\infty}\dif\mu=||g||_{\infty}\int_{E}|f(x)|\dif\mu=||f||_1\;||g||_{\infty}\qedhere
	\end{equation*}
	其中$fg\in L_1(E)$由$f\in L_1(E),g\in L_{\infty}(E)$以及上式保证。
\end{proof}






