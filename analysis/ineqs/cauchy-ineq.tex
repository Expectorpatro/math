\section{Cauchy不等式}

\subsection{实数域上的Cauchy不等式}
\begin{theorem}
	若$a_i,b_i\in\mathbb{R},i=1,2,\dots,n$,则有如下不等式:
	\begin{inequality*}\label{ineq:cauchy-ineq-R}
		\left(\sum_{i=1}^na_ib_i\right)^2\leqslant\sum_{i=1}^na_i^2\cdot\sum_{i=1}^nb_i^2
	\end{inequality*}
\end{theorem}
\begin{proof}
	取$\lambda\in\mathbb{R}$:
	\begin{equation*}
		0\leqslant\sum_{i=1}^n\left(a_i+\lambda b_i\right)^2=
		\sum_{i=1}^na_i^2+2\lambda\sum_{i=1}^na_ib_i+\lambda^2\sum_{i=1}^nb_i^2
	\end{equation*}
	将$\lambda$看作自变量,$a_i,b_i$为常数。由判别式可得:
	\begin{equation*}
		4\left(\sum_{i=1}^na_ib_i\right)^2\leqslant4\sum_{i=1}^na_i^2\cdot\sum_{i=1}^nb_i^2\qedhere
	\end{equation*}
\end{proof}

\subsection{复数域上的Cauchy不等式}
\begin{theorem}
	若$a_i,b_i\in\mathbb{C},i=1,2,\dots,n$,则有如下不等式:
	\begin{inequality*}\label{ineq:cauchy-ineq-C}
		\left(\sum_{i=1}^n|a_ib_i|\right)^2\leqslant\sum_{i=1}^n|a_i|^2\cdot\sum_{i=1}^n|b_i|^2
	\end{inequality*}
\end{theorem}
\begin{proof}
	由复数模的性质可得到
	\begin{equation*}
		|a_ib_i|=|a_i||b_i|
	\end{equation*}
	然后使用实数域上的Cauchy不等式即可立即得到结果。
\end{proof}

\subsection{内积导出的Cauchy-Schiwarz不等式}
\begin{theorem}
	设$X$是一个实或复内积空间,$x,y\in X$,则有如下不等式:
	\begin{inequality*}\label{ineq:cauchy-schiwarz-inner-product}
		|(x,y)|^2\leqslant(x,x)(y,y)
	\end{inequality*}
\end{theorem}
下给出复数域内的证明,实数域的证明显然类似。
\begin{proof}
	设$x,y\in X$。对任意的$\lambda\in\mathbb{C}$,有:
	\begin{equation*}
		(x+\lambda y,x+\lambda y)\geqslant 0
	\end{equation*}
	即:
	\begin{equation*}
		(x,x)+\overline{\lambda}(x,y)+\lambda(y,x)+|\lambda|^2(y,y)\geqslant 0
	\end{equation*}
	令$\lambda=-\frac{(x,y)}{(y,y)}$,得到:
	\begin{gather*}
		(x,x)-2\frac{|(x,y)|^2}{(y,y)}+\frac{|(x,y)|^2}{(y,y)^2}(y,y)\geqslant0 \\
		|(x,y)|^2\leqslant(x,x)(y,y)\qedhere
	\end{gather*}
\end{proof}

\subsection{期望形式的Cauchy-Schiwarz不等式}
\begin{theorem}
	设$\operatorname{E}(X^2)<+\infty,\;\operatorname{E}(Y^2)<+\infty$,则有:
	\begin{inequality*}\label{ineq:cauchy-schiwarz-expectations}
		|\operatorname{E}(XY)|\leqslant\sqrt{\operatorname{E}(X^2)\operatorname{E}(Y^2)}
	\end{inequality*}
	等号成立的充要条件为存在不全为$0$的常数$a,b$使得$aX+bY=0\;$a.s.成立。
\end{theorem}
\begin{proof}
	对任意的常数$a,b$,由\cref{prop:NonnegativeMeasurableIntegral}(2)(6)可知二次型:
	\begin{equation*}
		\operatorname{E}[(aX+bY)^2]=a^2\operatorname{E}(X^2)+2ab\operatorname{E}(XY)+b^2\operatorname{E}(Y^2)=(a,b)\Sigma(a,b)^T\geqslant0
	\end{equation*}
	其中:
	\begin{equation*}
		\Sigma=
		\begin{pmatrix}
			\operatorname{E}(X^2) & \operatorname{E}(XY) \\
			\operatorname{E}(XY) & \operatorname{E}(Y^2)
		\end{pmatrix}
	\end{equation*}
	所以$\Sigma$是一个半正定矩阵,由\cref{theo:PositiveSemidefinite}第三条的(6)可知$|\Sigma|\geqslant0$,即$|\operatorname{E}(XY)|\leqslant\sqrt{\operatorname{E}(X^2)\operatorname{E}(Y^2)}$。等号成立当且仅当$\Sigma$退化,当且仅当有不全为$0$的常数$a,b$使得$\operatorname{E}[(aX+bY)^2]=0$($\Sigma$退化时列向量线性相关,存在不全为$0$的$a,b$使得$\Sigma(a,b)^T=\mathbf{0}$,即$(a,b)\Sigma(a,b)^T=0$;当存在不全为$0$的$a,b$使得$\operatorname{E}[(aX+bY)^2]=0$时,$(a,b)\Sigma(a,b)^T=0$,\info{需要证明此时一定退化}),由\cref{prop:NonnegativeMeasurableIntegral}(10)可知此时当且仅当有不全为$0$的常数$a,b$使得$aX+bY=0\;$a.s.。
\end{proof}

\subsection{Rayleigh商的Cauchy–Schwarz不等式}
\begin{theorem}
	设$A$为$n\times n$实正定矩阵,$a\in\mathbb{R}^n$,则有:
	\begin{inequality*}\label{ineq:cauchy-schiwarz-rayleigh}
		\sup_{b\ne\mathbf{0}}\frac{(a^Tb)^2}{b^TAb}=a^TA^{-1}a
	\end{inequality*}
\end{theorem}
\begin{proof}
	由于$A$是正定矩阵,所以存在可逆的$A^{1/2}$。令$x = A^{1/2}b$,即$b = A^{-1/2}x$,则:
	\begin{align*}
		\frac{(a^Tb)^2}{b^TAb} &= \frac{(a^T A^{-1/2} x)^2}{x^T x}=\frac{[(A^{-1/2} a)^T x]^2}{\|x\|^2}
	\end{align*}
	记$u = A^{-1/2}a$,由\cref{ineq:cauchy-schiwarz-inner-product}有:
	\begin{equation*}
		|(u,x)|^2 \leqslant \|u\|^2 \cdot \|x\|^2
	\end{equation*}
	于是:
	\begin{equation*}
		\frac{[(A^{-1/2} a)^T x]^2}{\|x\|^2} \leqslant \|A^{-1/2}a\|^2 = (A^{-1/2}a)^T (A^{-1/2}a) = a^T A^{-1} a
	\end{equation*}
	即:
	\begin{equation*}
		\sup_{b\ne\mathbf{0}}\frac{(a^Tb)^2}{b^TAb}=a^TA^{-1}a \qedhere
	\end{equation*}
\end{proof}
