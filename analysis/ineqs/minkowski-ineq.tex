\section{Minkowski不等式}

\subsection{离散形式的Minkowski不等式}
\subsubsection{有穷级数形式}
设$\xi_i,\eta_i,i=1,2,\dots,n$同为实数或复数,$1\geqslant p<+\infty$,则有如下不等式:
\begin{inequality*}\label{ineq:minkowski-ineq-finite-series}
	\left(\sum_{i=1}^n|\xi_i+\eta_i|^p\right)^\frac{1}{p}\leqslant\left(\sum_{i=1}^n|\xi_i|^p\right)^\frac{1}{p}+\left(\sum_{i=1}^n|\eta_i|^p\right)^\frac{1}{p}
\end{inequality*}
\begin{proof}
	当$p=1$时结论显然成立。当$p>1$时,取$p,q\in\mathbb{R}$且$p,q>1$,满足$\frac{1}{p}+\frac{1}{q}=1$,由Holder不等式(即\cref{ineq:holder-ineq-finite-series}):
	\begin{gather*}
		\sum_{i=1}^n|\xi_i||\xi_i+\eta_i|^\frac{p}{q}\leqslant\left(\sum_{i=1}^n|\xi_i|^p\right)^\frac{1}{p}\cdot\left(\sum_{i=1}^n|\xi_i+\eta_i|^p\right)^\frac{1}{q} \\
		\sum_{i=1}^n|\eta_i||\xi_i+\eta_i|^\frac{p}{q}\leqslant\left(\sum_{i=1}^n|\eta_i|^p\right)^\frac{1}{p}\cdot\left(\sum_{i=1}^n|\xi_i+\eta_i|^p\right)^\frac{1}{q} 
	\end{gather*}
	因此:
	\begin{align*}
		\sum_{i=1}^n|\xi_i+\eta_i|^p&=\sum_{i=1}^n\left[|\xi_i+\eta_i|\;|\xi_i+\eta_i|^{p-1}\right] \\
		&\leqslant\sum_{i=1}^n\left[(|\xi_i|+|\eta_i|)|\xi_i+\eta_i|^\frac{p}{q}\right] \\
		&=\sum_{i=1}^n|\xi_i||\xi_i+\eta_i|^\frac{p}{q}+\sum_{i=1}^n|\eta_i||\xi_i+\eta_i|^\frac{p}{q} \\
		&\leqslant\left[\left(\sum_{i=1}^n|\xi_i|^p\right)^\frac{1}{p}+\left(\sum_{i=1}^n|\eta_i|^p\right)^\frac{1}{p}\right]\cdot\left(\sum_{i=1}^n|\xi_i+\eta_i|^p\right)^\frac{1}{q} 
	\end{align*}
	上式第一行到第二行需要注意到$p-1=\frac{p}{q}$。两边同除$\left(\sum\limits_{i=1}^n|\xi_i+\eta_i|^p\right)^\frac{1}{q}$(若其为$0$则结论显然也成立)即有:
	\begin{equation*}
		\left(\sum_{i=1}^n|\xi_i+\eta_i|^p\right)^\frac{1}{p}\leqslant\left(\sum_{i=1}^n|\xi_i|^p\right)^\frac{1}{p}+\left(\sum_{i=1}^n|\eta_i|^p\right)^\frac{1}{p}\qedhere
	\end{equation*}
\end{proof}
\subsubsection{无穷级数形式}
设$\xi_i,\eta_i,i=1,2,\dots,n$同为实数或复数,$1\geqslant p<+\infty$,则有如下不等式:
\begin{inequality*}\label{ineq:minkowski-ineq-infty-series}
	\left(\sum_{i=1}^{+\infty}|\xi_i+\eta_i|^p\right)^\frac{1}{p}\leqslant\left(\sum_{i=1}^{+\infty}|\xi_i|^p\right)^\frac{1}{p}+\left(\sum_{i=1}^{+\infty}|\eta_i|^p\right)^\frac{1}{p}
\end{inequality*}
\begin{proof}
	证明过程与有穷级数几乎一样,只是把求和中的$n$改为$+\infty$的区别。
\end{proof}

\subsection{积分形式的Minkowski不等式}
\begin{theorem}
	设	$(X,\mathscr{F},\mu)$是一个测度空间,$1\leqslant p<+\infty$,$E\in\mathscr{F}$,$f,g\in L_p(E)$,则有如下不等式:
	\begin{inequality*}\label{ineq:minkowski-ineq-Lebesgue}
		\left[\int_{E}^{}|f(x)+g(x)|^p\dif\mu\right]^\frac{1}{p}\leqslant\left[\int_{E}^{}|f(x)|^p\dif\mu\right]^\frac{1}{p}+\left[\int_{E}^{}|g(x)|^p\dif\mu\right]^\frac{1}{p}
	\end{inequality*}
	且:
	\begin{enumerate}
		\item $p=1$时等号成立当且仅当$fg\geqslant0\;$a.e.于$E$;
		\item $p>1$时等号成立当且仅当存在不全为$0$的$\alpha,\beta\geqslant0$使得$\alpha f=\beta g$。
	\end{enumerate}
\end{theorem}
\begin{proof}
	\textbf{(1)$\;p=1$:}由绝对值的三角不等式以及\cref{prop:NonnegativeMeasurableIntegral}(6)直接可得。等号成立当且仅当$f(x)g(x)\geqslant0$,考虑到\cref{prop:NonnegativeMeasurableIntegral}(8),只要$f(x)g(x)\geqslant0\;$a.e.于$E$即可。\par
	\textbf{(2)$\;p>1$:}当$|f(x)+g(x)|=0\;$a.e.于$E$时,由\cref{prop:NonnegativeMeasurableIntegral}(10)可知$\left[\int_{E}^{}|f(x)+g(x)|^p\dif\mu\right]^\frac{1}{p}=0$,根据\cref{prop:NonnegativeMeasurableIntegral}(2)可得结论成立。此时由\cref{prop:NonnegativeMeasurableIntegral}(2)(10)可知等号成立当且仅当$f=0\;$a.e.于$E$且$g=0\;$a.e.于$E$,该条件可推出存在不全为$0$的$\alpha,\beta\geqslant0$使得$\alpha f=\beta g\;$a.e.于$E$。\par
	当$|f(x)+g(x)|=0\;$不a.e.于$E$时,取$q\in\mathbb{R}$且$q>1$,满足$\frac{1}{p}+\frac{1}{q}=1$,由Holder不等式(即\cref{ineq:holder-ineq-Lebesgue}):
	\begin{gather*}
		\int_{E}^{}|f(x)|\;|f(x)+g(x)|^\frac{p}{q}\dif\mu\leqslant\left[\int_{E}^{}|f(x)|^p\dif\mu\right]^\frac{1}{p}\left[\int_{E}^{}|f(x)+g(x)|^p\dif\mu\right]^\frac{1}{q} \\
		\int_{E}^{}|g(x)|\;|f(x)+g(x)|^\frac{p}{q}\dif\mu\leqslant\left[\int_{E}^{}|g(x)|^p\dif\mu\right]^\frac{1}{p}\left[\int_{E}^{}|f(x)+g(x)|^p\dif\mu\right]^\frac{1}{q}
	\end{gather*}
	注意到$p+q=pq$,所以$q(p-1)=qp-q=p$,因此由绝对值的三角不等式以及\cref{prop:NonnegativeMeasurableIntegral}(6)可得:
	\begin{align*}
		\int_{E}^{}|f(x)+g(x)|^p\dif\mu
		&=\int_{E}^{}|f(x)+g(x)|\;|f(x)+g(x)|^{p-1}\dif\mu \\
		&\leqslant\int_{E}^{}\Bigl[|f(x)|+|g(x)|\Bigr]\;|f(x)+g(x)|^{p-1}\dif\mu \\
		&=\int_{E}^{}|f(x)|\;|f(x)+g(x)|^{p-1}\dif\mu+\int_{E}^{}|g(x)|\;|f(x)+g(x)|^{p-1}\dif\mu \\
		&\leqslant\left\{\left[\int_{E}^{}|f(x)|^p\dif\mu\right]^\frac{1}{p}+\left[\int_{E}^{}|g(x)|^p\dif\mu\right]^\frac{1}{p}\right\}\left[\int_{E}^{}|f(x)+g(x)|^p\dif\mu\right]^\frac{1}{q}
	\end{align*}
	两边同除$\left[\int_{E}^{}|f(x)+g(x)|^p\dif\mu\right]^\frac{1}{q}$即有:
	\begin{equation*}
			\left[\int_{E}^{}|f(x)+g(x)|^p\dif\mu\right]^\frac{1}{p}\leqslant\left[\int_{E}^{}|f(x)|^p\dif\mu\right]^\frac{1}{p}+\left[\int_{E}^{}|g(x)|^p\dif\mu\right]^\frac{1}{p}
	\end{equation*}
	由放缩的过程注意到等号成立当且仅当Holder不等式取等且$f(x)g(x)\geqslant0\;$a.e.于$E$,而Holder不等式取等当且仅当存在不全为$0$的$\alpha,\beta\geqslant0$和不全为$0$的$\gamma,\delta\geqslant0$使得:
	\begin{equation*}
		\alpha|f|^p=\beta|f+g|^p,\;\gamma|g|^p=\delta|f+g|^p\;\text{a.e.于$E$}
	\end{equation*}
	因为此时$f(x),g(x)$同号a.e.于$E$,所以也即存在不全为$0$的$\alpha,\beta\geqslant0$和不全为$0$的$\gamma,\delta\geqslant0$使得:
	\begin{equation*}
		\alpha f=\beta (f+g),\;\gamma g=\delta(f+g)\;\text{a.e.于$E$}
	\end{equation*}
	该条件可推出存在不全为$0$的$\alpha,\beta\geqslant0$使得$\alpha f=\beta g\;$a.e.于$E$。\par
	接下来证明存在不全为$0$的$\alpha,\beta\geqslant0$使得$\alpha f=\beta g\;$a.e.于$E$也可推出等式成立。仅对$\alpha\ne0$的情况进行证明,$\beta\ne0$的情况可对称得到。\par
	\textbf{(1)$\;|f(x)+g(x)|=0\;$a.e.于$E$:}此时可得到$g=0\;$a.e.于$E$,进而得到$f=0\;$a.e.成立。由\cref{prop:NonnegativeMeasurableIntegral}(10)可得:
	\begin{equation*}
		\left[\int_{E}^{}|f(x)+g(x)|^p\dif\mu\right]^\frac{1}{p}=\left[\int_{E}^{}|f(x)|^p\dif\mu\right]^\frac{1}{p}+\left[\int_{E}^{}|g(x)|^p\dif\mu\right]^\frac{1}{p}=0
	\end{equation*}\par
	\textbf{(2)$\;|f(x)+g(x)|=0$不a.e.于$E$:}此时可得到:
	\begin{equation*}
		f=\frac{\beta}{\alpha}g\;\text{a.e.于$E$}
	\end{equation*}
	于是由\cref{prop:NonnegativeMeasurableIntegral}(6)(8)可得:
	\begin{align*}
		\left[\int_{E}^{}|f(x)+g(x)|^p\dif\mu\right]^{\frac{1}{p}}&=\left[\int_{E}^{}\left|\frac{\beta+\alpha}{\alpha}g(x)\right|^p\dif\mu\right]^\frac{1}{p} \\
		&=\frac{\beta+\alpha}{\alpha}\left[\int_{E}^{}|g(x)|^p\dif\mu\right]^\frac{1}{p} \\
		&=\frac{\beta}{\alpha}\left[\int_{E}^{}|g(x)|^p\dif\mu\right]^\frac{1}{p}+\left[\int_{E}^{}|g(x)|^p\dif\mu\right]^\frac{1}{p} \\
		&=\left[\int_{E}\left|\frac{\beta}{\alpha}g(x)\right|^p\dif\mu\right]^{\frac{1}{p}}+\left[\int_{E}^{}|g(x)|^p\dif\mu\right]^\frac{1}{p} \\
		&=\left[\int_{E}^{}|f(x)|^p\dif\mu\right]^\frac{1}{p}+\left[\int_{E}^{}|g(x)|^p\dif\mu\right]^\frac{1}{p}\qedhere
	\end{align*}
\end{proof}
\begin{theorem}
	设	$(X,\mathscr{F},\mu)$是一个测度空间,$1\leqslant p<+\infty$,$E\in\mathscr{F}$。对任意的$f,g\in L_p(E)$,在$L_p(E)$空间中引入范数:
	\begin{equation*}
		||f||_p=\left[\int_{E}^{}|f(x)|^p\dif\mu\right]^\frac{1}{p}
	\end{equation*}
	则积分形式的Minkowski不等式可写为:
	\begin{inequality*}\label{ineq:minkowski-ineq-norm}
		||f+g||_p\leqslant||f||_p+||g||_p
	\end{inequality*}
\end{theorem}
\begin{theorem}
	设	$(X,\mathscr{F},\mu)$是一个测度空间,$E\in\mathscr{F}$。在$L_{\infty}(E)$中定义范数为元素的无穷范数,则有:
	\begin{inequality*}\label{ineq:minkowski-ineq-norm+infty}
		||f+g||_{\infty}\leqslant||f||_{\infty}+||g||_{\infty}
	\end{inequality*}
\end{theorem}
\begin{proof}
	由绝对值的三角不等式以及上确界的性质:
	\begin{equation*}
		\sup_{x\in E\backslash e}|f(x)+g(x)|\leqslant\sup_{x\in E\backslash e}[|f(x)|+|g(x)|]\leqslant\sup_{x\in E\backslash e}|f(x)|+\sup_{x\in E\backslash e}|g(x)|
	\end{equation*}
	由下确界的性质即可得:
	\begin{equation*}
		||f+g||_{\infty}\leqslant||f||_{\infty}+||g||_{\infty}\qedhere
	\end{equation*}
\end{proof}
\begin{theorem}
	设$(X,\mathscr{F},\mu)$是一个测度空间,$1\leqslant p\leqslant+\infty$,$E\in\mathscr{F}$。对任意的$f,g\in L_p(E)$,有:
	\begin{inequality*}\label{ineq:minkowski-ineq-norm-all}
		||f+g||_p\leqslant||f||_p+||g||_p,\quad
		\Big|||f||_p-||g||_p\Big|\leqslant||f-g||_p
	\end{inequality*}
\end{theorem}
\begin{proof}
	第一式由\cref{ineq:minkowski-ineq-norm}和\cref{ineq:holder-ineq-norm+infty}直接推出。第二式只需注意到:
	\begin{gather*}
		||f||_p=||f-g+g||_p\leqslant||f-g||_p+||g||_p \\
		||g||_p=||g-f+f||_p\leqslant||g-f||_p+||f||_p=||f-g||_p+||f||_p\qedhere
	\end{gather*}
\end{proof}