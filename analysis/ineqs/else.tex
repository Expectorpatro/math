\section{Something else}

\begin{theorem}
	对于所有 $a,b\in\mathbb{R}$ 及 $p\ge 1$,有不等式
	\begin{inequality*}\label{ineq:else-1}
		|a+b|^p\leqslant\Bigl(|a|+|b|\Bigr)^p \leqslant 2^{p-1}\Bigl(|a|^p+|b|^p\Bigr).
	\end{inequality*}
\end{theorem}
\begin{proof}
	第一式显然成立。令 $a,b\in\mathbb{R}$,并设 $x=|a|$ 和 $y=|b|$(显然 $x,y\geqslant0$)。考虑函数
	\begin{equation*}
		f(t)=t^p,\quad t\ge 0
	\end{equation*}
	由于 $p\geqslant1$,函数 $f(t)$ 是凸函数。根据凸函数的定义,对于任意 $x,y\ge 0$ 和 $\lambda\in[0,1]$ 有
	\begin{equation*}
		f\bigl(\lambda x+(1-\lambda)y\bigr)\leqslant\lambda f(x)+(1-\lambda)f(y)
	\end{equation*}
	取 $\lambda=\frac{1}{2}$,则上式变为:
	\begin{equation*}
		\left(\frac{x+y}{2}\right)^p\leqslant\frac{x^p+y^p}{2}
	\end{equation*}
	将上式两边同时乘以 $2^p$,得到
	\begin{equation*}
		(x+y)^p\leqslant2^{p-1}\Bigl(x^p+y^p\Bigr)
	\end{equation*}
	将 $x=|a|$ 和 $y=|b|$ 代回,即得所需的不等式。
\end{proof}

\subsection{Jensen不等式}
\begin{theorem}
	设$(X,\mathscr{F},P)$是一个概率空间,$\mathscr{A}\subseteq\mathscr{F}$,$f$是$(X,\mathscr{F})$上可积的Borel函数,$\varphi$为$(\mathbb{R}^{},\mathcal{B})$上的Borel函数且是$\mathbb{R}^{}$上的凸函数,$\varphi\circ f$在$X$上可积,则有:
	\begin{inequality*}\label{ineq:Jensen}
		\varphi[\operatorname{E}(f|\mathscr{A})]\leqslant\operatorname{E}(\varphi\circ f|\mathscr{A})
	\end{inequality*}
	等号成立a.s.于$(X,\mathscr{A},P)$的一个充分条件为$f=\operatorname{E}(f|\mathscr{A})\;$a.s.于$X$。若$\varphi$是严格凸的,则等号成立a.s.于$(X,\mathscr{A},P)$的充要条件为$f=\operatorname{E}(f|\mathscr{A})\;$a.s.于$X$。
\end{theorem}
\begin{proof}
	令:
	\begin{equation*}
		x_0=\operatorname{E}(f|\mathscr{A})
	\end{equation*}
	根据\cref{prop:ConvexFunction}(3)可知存在$a,b\in\mathbb{R}^{}$使得:
	\begin{equation*}
		\forall\;x\in\mathbb{R}^{},\;ax+b\leqslant\varphi(x)
	\end{equation*}
	并满足$ax_0+b=\varphi(x_0)$。根据\cref{prop:MeasurableIntegral}(6),令$x=f(\omega)$对两侧求关于$\mathscr{A}$的条件期望,由\cref{prop:ConditionalExpectation}(5)(1)可得:
	\begin{gather*}
		\operatorname{E}(af+b|\mathscr{A})=a\operatorname{E}(f|\mathscr{A})+b\operatorname{E}(1|\mathscr{A})\;\text{a.s.于}(X,\mathscr{A},P) \\ a\operatorname{E}(f|\mathscr{A})+b\operatorname{E}(1|\mathscr{A})=a\operatorname{E}(f|\mathscr{A})+b\;\text{a.s.于}(X,\mathscr{A},P)
	\end{gather*}
	根据\cref{prop:Measure}(3)(次有限可加性)和测度的非负性可知$\operatorname{E}(af+b|\mathscr{A})=a\operatorname{E}(f|\mathscr{A})+b\;$a.s.于$(X,\mathscr{A},P)$。显然$af(\omega)+b$在$X$上可积,由\cref{prop:ConditionalExpectation}(4)、\cref{prop:Measure}(3)(次有限可加性)和测度的非负性即可得:
	\begin{equation*}
		a\operatorname{E}(f|\mathscr{A})+b=\varphi[\operatorname{E}(f|\mathscr{A})]\leqslant\operatorname{E}(\varphi\circ f|\mathscr{A})\;\text{a.s.于}(X,\mathscr{A},P)
	\end{equation*}
	当$f=\operatorname{E}(f|\mathscr{A})\;$a.s.于$(X,\mathscr{A},P)$时,$\varphi(f)=\varphi[\operatorname{E}(f|\mathscr{A})]\;$a.s.于$(X,\mathscr{A},P)$,因为$\varphi\circ f\in L_1(X)$,根据\cref{prop:MeasurableIntegral}(7)可知$\varphi[\operatorname{E}(f|\mathscr{A})]\in L_1(X)$。由\cref{prop:MeasurableMapping}(2)和\cref{prop:ConditionalExpectation}(1)可得$\operatorname{E}\{\varphi[\operatorname{E}(f|\mathscr{A})]|\mathscr{A}\}=\varphi[\operatorname{E}(f|\mathscr{A})]\;$a.s.于$(X,\mathscr{A},P)$。由\cref{prop:ConditionalExpectation}(4)、\cref{prop:Measure}(3)(次有限可加性)和测度的非负性可得:
	\begin{equation*}
		\operatorname{E}(\varphi\circ f|\mathscr{A})=\operatorname{E}\{\varphi[\operatorname{E}(f|\mathscr{A})]|\mathscr{A}\}=\varphi[\operatorname{E}(f|\mathscr{A})]\;\text{a.s.于}(X,\mathscr{A},P)
	\end{equation*}\par
	当$\varphi$严格凸时,对任意的$x\in\mathbb{R}^{}\setminus\{x_0\}$有$ax+b<\varphi(x)$。若$\varphi[\operatorname{E}(f|\mathscr{A})]=\operatorname{E}(\varphi\circ f|\mathscr{A})\;$a.s.于$(X,\mathscr{A},P)$,则:
	\begin{equation*}
		\operatorname{E}\{\varphi[\operatorname{E}(f|\mathscr{A})]\}=\operatorname{E}[\operatorname{E}(\varphi\circ f|\mathscr{A})]=\operatorname{E}(\varphi\circ f)
	\end{equation*}\info{未完成}
\end{proof}

\subsection{Chebyshev不等式}
\begin{theorem}
	设$X$是一个随机变量,其数学期望和方差都存在,则对任意的$\varepsilon>0$,有:
	\begin{inequality*}\label{ineq:Chebyshev}
		P(|X-\operatorname{E}(X)|\geqslant\varepsilon)\leqslant\frac{\operatorname{Var}(X)}{\varepsilon^2}
	\end{inequality*}
\end{theorem}
\begin{proof}
	设$X$的分布函数为$F(x)$,则:
	\begin{align*}
		P(|X-\operatorname{E}(X)|\geqslant\varepsilon)
		&=\int_{\{x:|x-\operatorname{E}(X)|\geqslant\varepsilon\}}^{}\dif F(x) \\
		&\leqslant\int_{\{x:|x-\operatorname{E}(X)|\geqslant\varepsilon\}}^{}\frac{|x-\operatorname{E}(X)|^2}{\varepsilon^2}\dif F(x) \\
		&\leqslant\frac{1}{\varepsilon^2}\int_{-\infty}^{+\infty}|x-\operatorname{E}(X)|^2\dif F(x) \\
		&=\frac{\operatorname{Var}(X)}{\varepsilon^2}\qedhere
	\end{align*}
\end{proof}
