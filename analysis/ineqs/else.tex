\section{Something else}

\begin{theorem}
	对于所有 $a,b\in\mathbb{R}$ 及 $p\ge 1$,有不等式
	\begin{inequality*}\label{ineq:else-1}
		|a+b|^p\leqslant\Bigl(|a|+|b|\Bigr)^p \leqslant 2^{p-1}\Bigl(|a|^p+|b|^p\Bigr).
	\end{inequality*}
\end{theorem}
\begin{proof}
	第一式显然成立。令 $a,b\in\mathbb{R}$,并设 $x=|a|$ 和 $y=|b|$(显然 $x,y\geqslant0$)。考虑函数
	\begin{equation*}
		f(t)=t^p,\quad t\ge 0
	\end{equation*}
	由于 $p\geqslant1$,函数 $f(t)$ 是凸函数。根据凸函数的定义,对于任意 $x,y\ge 0$ 和 $\lambda\in[0,1]$ 有
	\begin{equation*}
		f\bigl(\lambda x+(1-\lambda)y\bigr)\leqslant\lambda f(x)+(1-\lambda)f(y)
	\end{equation*}
	取 $\lambda=\frac{1}{2}$,则上式变为:
	\begin{equation*}
		\left(\frac{x+y}{2}\right)^p\leqslant\frac{x^p+y^p}{2}
	\end{equation*}
	将上式两边同时乘以 $2^p$,得到
	\begin{equation*}
		(x+y)^p\leqslant2^{p-1}\Bigl(x^p+y^p\Bigr)
	\end{equation*}
	将 $x=|a|$ 和 $y=|b|$ 代回,即得所需的不等式。
\end{proof}

\subsection{Chebyshev不等式}
\begin{theorem}
	设$X$是一个随机变量,其数学期望和方差都存在,则对任意的$\varepsilon>0$,有:
	\begin{inequality*}\label{ineq:Chebyshev}
		P(|X-\operatorname{E}(X)|\geqslant\varepsilon)\leqslant\frac{\operatorname{Var}(X)}{\varepsilon^2}
	\end{inequality*}
\end{theorem}
\begin{proof}
	设$X$的分布函数为$F(x)$,则:
	\begin{align*}
		P(|X-\operatorname{E}(X)|\geqslant\varepsilon)
		&=\int_{\{x:|x-\operatorname{E}(X)|\geqslant\varepsilon\}}^{}\dif F(x) \\
		&\leqslant\int_{\{x:|x-\operatorname{E}(X)|\geqslant\varepsilon\}}^{}\frac{|x-\operatorname{E}(X)|^2}{\varepsilon^2}\dif F(x) \\
		&\leqslant\frac{1}{\varepsilon^2}\int_{-\infty}^{+\infty}|x-\operatorname{E}(X)|^2\dif F(x) \\
		&=\frac{\operatorname{Var}(X)}{\varepsilon^2}\qedhere
	\end{align*}
\end{proof}