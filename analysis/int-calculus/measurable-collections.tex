\section{可测集类}
在这一节里,我们来讨论$\mathbb{R}^n$上什么样的集合是可测的。先给出以下总结:
\begin{enumerate}
	\item 外测度为$0$的集合都可测,称为零测集。
	\item 零测集的任何子集都可测,并且仍然是零测集。
	\item 有限个或可数个零测集的并可测,并且仍然是零测集。
	\item 区间都可测,并且测度为其体积。
	\item 开集和闭集都可测。
	\item Borel集都可测。
\end{enumerate}
\subsubsection{零测集}
\begin{definition}
	外测度为$0$的集合称之为\gls{NullSet}。
\end{definition}
\begin{theorem}
	零测集都可测。
\end{theorem}
\begin{proof}
	任取一个零测集$E$,对任意的$T\in \mathbb{R}^n$,有:
	\begin{align*}
		m^*(T)&\geqslant m^*(T\cap E^c) \\
		&=0+m^*(T\cap E^c) \\
		&=m^*(T\cap E)+m^*(T\cap E^c)
	\end{align*}
	又因:
	\begin{equation*}
		T=(T\cap E)\cup(T\cap E^c)
	\end{equation*}
	由外测度性质(3):
	\begin{equation*}
		m^*(T)\leqslant m^*(T\cap E)+m^*(T\cap E^c)
	\end{equation*}
	因此:
	\begin{equation*}
		m^*(T)=m^*(T\cap E)+m^*(T\cap E^c)
	\end{equation*}
	即$E$可测。由$E$的任意性,零测集都可测。
\end{proof}
\begin{theorem}
	零测集的任何子集都可测,并且仍然是零测集。
\end{theorem}
\begin{proof}
	由外测度性质(2),零测集的任何子集的外测度都为$0$,即它们都是零测集,而零测集可测,所以零测集的任何子集都可测。
\end{proof}
\begin{theorem}
	有限个或可数个零测集的并可测,并且仍然是零测集。
\end{theorem}
\begin{proof}
	由外测度的次可列可加性可得有限个或可数个零测集的外测度为$0$,即它们都是零测集,而零测集可测,所以有限个或可数个零测集的并可测。
\end{proof}
\subsubsection{区间}
\begin{theorem}
	区间都可测,并且$m(I)=|I|$。
\end{theorem}
\begin{proof}
	
\end{proof}
\subsubsection{开闭集}
\begin{theorem}
	开集闭集都可测。
\end{theorem}
\begin{proof}
	
\end{proof}
\subsubsection{Borel代数}
\begin{definition}
	设$\Sigma$是$\mathbb{R}^n$的一个子集族,则称所有包含$\Sigma$的$\sigma$代数的交集为$\Sigma$产生的$\sigma$代数。
\end{definition}
\begin{definition}
	由$\mathbb{R}^n$中全体开集组成的集类生成的$\sigma$代数称为\gls{BorelAlgebra},记为$\mathscr{B}$,其中的元素被称为\gls{BorelSet}。
\end{definition}
\begin{theorem}
	博雷尔集都是可测的。
\end{theorem}
\begin{proof}
	开集都是可测的。
\end{proof}
\subsubsection{可测集类的通性}
\begin{definition}
	设集合$G$可表示成一列开集$\{G_i\}$的交集:
	\begin{equation*}
		G=\underset{i=1}{\overset{+\infty}{\cap}}G_i
	\end{equation*}
	则称$G$为$G_\delta$型集。
\end{definition}
\begin{definition}
	设集合$F$可表示成一列闭集$\{F_i\}$的并集:
	\begin{equation*}
		F=\underset{i=1}{\overset{+\infty}{\cup}}F_i
	\end{equation*}
	则称$F$为$F_\sigma$型集。
\end{definition}
\begin{theorem}
	设$E$是任意可测集,则一定存在$G_\delta$型集$G$使$E\subset G$,且$m(G\;\backslash\; E)=0$。
\end{theorem}
\begin{proof}
	(1)先证对任意的$\varepsilon>0$,存在开集$G$,使$E\subset G$,且$m(G\;\backslash\; E)<\varepsilon$。\par
	对于测度有限的集合$E$,由测度定义,存在一列开区间$\{I_i\}$,使得$E\subset\underset{i=1}{\overset{+\infty}{\cup}}I_i$,并且有:
	\begin{equation*}
		\sum_{i=1}^\infty|I_i|<m(E)+\varepsilon
	\end{equation*}
	令$G=\underset{i=1}{\overset{+\infty}{\cup}}I_i$,则$G$是开集,并且$E\subset G$。同时由外测度性质(2)和(3):
	\begin{equation*}
		m(E)\leqslant m(G)=\sum_{i=1}^\infty m(I_i)=\sum_{i=1}^\infty|I_i|<m(E)+\varepsilon
	\end{equation*}
	因此:
	\begin{equation*}
		m(G\;\backslash\; E)=m(G)-m(E)<\varepsilon
	\end{equation*}
	\par 若$m(E)=\infty$,则$E$一定可表示为可列个互不相交并且测度有限的可测集的并集,即$E=\underset{i=1}{\overset{+\infty}{\cup}}E_i$。对每个$E_i$应用上面的结果,可找到开集$G_i$使$E_i\subset G_i$,并且有$m(G_i)<m(E_i)+\frac{\varepsilon}{2^i}$。令$G=\underset{i=1}{\overset{+\infty}{\cup}}G_i$,则$G$是开集,$E\subset G$,并且有:
	\begin{gather*}
		G\;\backslash\; E=\underset{i=1}{\overset{+\infty}{\cup}}G_i\;\backslash\;\underset{i=1}{\overset{+\infty}{\cup}}E_i\subset \underset{i=1}{\overset{+\infty}{\cup}}(G_i\;\backslash\; E_i) \\
		m(G\;\backslash\; E)\leqslant m\left[\underset{i=1}{\overset{+\infty}{\cup}}(G_i\;\backslash\; E_i)\right]\leqslant\sum_{i=1}^\infty m(G_i\;\backslash\; E_i)<\varepsilon
	\end{gather*}
	(2)依次取$\varepsilon=\frac{1}{n},n\in\mathbb{N}^+$,由(1)可知存在开集$G_n$使得$E\subset G_n$,且$m(G_n\;\backslash\; E)<\frac{1}{n}$。令$G=\underset{i=1}{\overset{+\infty}{\cap}}G_i$,则$G$为$G_\delta$型集,$E\subset G$,且有:
	\begin{equation*}
		m(G\;\backslash\; E)\leqslant m(G_n\;\backslash\; E)<\frac{1}{n}
	\end{equation*}
	对任意的$n\in\mathbb{N}^+$成立,即$m(G\;\backslash\; E)=0$。
\end{proof}
\begin{theorem}
	设$E$是任意可测集,则一定存在$F_\sigma$型集$F$使$F\subset E$,且$m(E\;\backslash\; F)=0$。
\end{theorem}
\begin{proof}
	因为$E$可测,所以$E^c$也可测。那么存在$G_\delta$型集$G$,使得$E^c\subset G$,且$m(G\;\backslash\;E^c)=0$。令$F=G^c$,则显然$F$是一个$F_\sigma$型集,且有$F\subset E$,同时有:
	\begin{equation*}
		m(E\;\backslash\;F)=m(E\;\backslash\;G^c)=m(E\cap G)=m(G\;\backslash\;E^c)=0\qedhere
	\end{equation*}
\end{proof}

