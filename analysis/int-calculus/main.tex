\chapter{不定积分与定积分}

\section{定积分}
\begin{definition}
	定义闭区间$[a,b]\subseteq\mathbb{R}^{}$的一个\textbf{分割}为插入在$a$和$b$之间的有限个分点:
	\begin{equation*}
		P:\;a=x_0<x_1<\cdots<x_n=b
	\end{equation*}
	这些分点把$[a,b]$分成$n$个闭子区间$[x_{i-1},x_{i}],\;i=1,2,\dots,n$,其中第$i$个闭子区间的长度为$\Delta x_i=x_i-x_{i-1}$,称$|P|=\max\{\Delta x_i:i=1,2,\dots,n\}$为分割$P$的\textbf{模}。在分割$P$的每一个闭子区间上任取一点:
	\begin{equation*}
		\xi_i\in[x_{i-1},x_i],\quad i=1,2,\dots,n
	\end{equation*}
	称这$n$个点$\seq{\xi}{n}$为对应于分割$P$的一组\textbf{标志点},用$\xi$来表示它们。
\end{definition}
\begin{definition}
	如果闭区间$[a,b]\subseteq\mathbb{R}^{}$上的分割序列$\{P_n\}$满足:
	\begin{equation*}
		\lim_{n\to+\infty}|P_n|=0
	\end{equation*}
	则称$\{P_n\}$是一个\textbf{无穷细分割序列}。
\end{definition}
\begin{definition}
	设函数$f$在闭区间$[a,b]\subseteq\mathbb{R}^{}$上有定义,对于$[a,b]$的任意一个分割:
	\begin{equation*}
		P:\;a=x_0<x_1<\cdots<x_n=b
	\end{equation*}
	和对应于这一分割的任意一组标志点$\xi$,可以作和:
	\begin{equation*}
		\sigma(f,P,\xi)=\sum_{i=1}^{n}f(\xi_i)\Delta x_i
	\end{equation*}
	称和$\sigma(f,P,\xi)$为函数$f$在闭区间$[a,b]$上的\textbf{Riemann和}。
\end{definition}
\begin{definition}
	设函数$f$在闭区间$[a,b]\subseteq\mathbb{R}^{}$上有定义,$I\in\mathbb{R}^{}$。若对任意的$\varepsilon>0$,存在$\delta>0$使得只要$|P|<\delta$,不论分割$P$对应的标志点组$\xi$如何选择,都有:
	\begin{equation*}
		|\sigma(f,P,\xi)-I|<\varepsilon
	\end{equation*}
	则称$f$在区间$[a,b]$上\textbf{Riemann可积},并且把$I$称为函数$f$在区间$[a,b]$上的\textbf{积分},记为:
	\begin{equation*}
		\int_{a}^{b}f(x)\dif x=\lim_{|P|\to0}\sigma(f,P,\xi)=I
	\end{equation*}
\end{definition}

\subsection{定积分存在的一般条件}
\begin{theorem}\label{theo:RiemannIntegrableBounded}
	设函数$f$在闭区间$[a,b]\subseteq\mathbb{R}^{}$上可积,则$f$在$[a,b]$上有界。
\end{theorem}
\begin{proof}
	设$f$在$[a,b]$上无界。因为$f$在闭区间$[a,b]$上可积,所以对于$\varepsilon=1$,存在$\delta>0$使得只要$|P|<\delta$,不论分割$P$对应的标志点组$\xi$如何选择,都有$|\sigma(f,P,\xi)-I|<1$,即
	\begin{equation*}
		|\sigma(f,P,\xi)|=|\sigma(f,P,\xi)-I+I|\leqslant|\sigma(f,P,\xi)-I|+|I|<1+|I|
	\end{equation*}
	取上述满足$|P|<\delta$的分割$P$。因为$f$在$[a,b]$上无界,所以$f$至少在分割$P$的一个闭子区间$[x_{j-1},x_j]$上无界。任意选定$\xi_i\in[x_{i-1},x_i],\;\forall\;i\ne j$,则可以选择$\xi_j\in[x_{j-1},x_j]$使得:
	\begin{equation*}
		|f(\xi_j)\Delta x_j|>\sum_{i\ne j}|f(\xi_i)\Delta x_i|+1+|I|
	\end{equation*}
	注意到$|a|-|b|=|a+b-b|-|b|\leqslant|a+b|+|-b|-|b|=|a+b|$,于是就有:
	\begin{equation*}
		1+|I|>|\sigma(f,P,\xi)|=\left|\sum_{i=1}^{n}f(\xi_i)\Delta x_i\right|\geqslant|f(\xi_j)\Delta x_j|-\left|\sum_{i\ne j}^{}f(\xi_i)\Delta x_i\right|>1+|I|
	\end{equation*}
	矛盾,所以$f$在$[a,b]$上有界。
\end{proof}
\begin{definition}
	设函数$f$在闭区间$[a,b]\subseteq\mathbb{R}^{}$上有定义且有界,对于$[a,b]$的一个分割:
	\begin{equation*}
		P:\;a=x_0<x_1<\cdots<x_n=b
	\end{equation*}
	$f$在任一子区间$[x_{i-1},x_i]$上都有有限的下确界和上确界,记为:
	\begin{equation*}
		m_i=\inf_{x\in[x_{i-1},x_i]}f(x),\quad M_i=\sup_{x\in[x_{i-1},x_i]}f(x),\quad\omega_i=M_i-m_i
	\end{equation*}
	记1:
	\begin{equation*}
		m=\inf_{x\in[a,b]}f(x),\quad M=\sup_{x\in[a,b]}f(x),\quad\omega=M-m
	\end{equation*}
\end{definition}
\begin{definition}
	设函数$f$在闭区间$[a,b]\subseteq\mathbb{R}^{}$上有定义且有界,$P$是$[a,b]$上的一个分割,分别称:
	\begin{equation*}
		L(f,P)=\sum_{i=1}^{n}m_i\Delta x_i,\quad U(f,P)=\sum_{i=1}^{n}M_i\Delta x_i,\quad\Omega(f,P)=U(f,P)-L(f,P)
	\end{equation*}
	为$f$关于$P$的\textbf{下和}与\textbf{上和}。
\end{definition}
\begin{definition}
	记:
	\begin{equation*}
		\underline{I}=\sup_PL(f,P),\quad\overline{I}=\inf_PU(f,P)
	\end{equation*}
\end{definition}
\begin{property}\label{prop:DarbouxSum}
	设函数$f$在闭区间$[a,b]\subseteq\mathbb{R}^{}$上有定义且有界,$P$是$[a,b]$上的一个分割,$f$关于$P$的下和$L(f,P)$和上和$U(f,P)$具有如下性质:
	\begin{enumerate}
		\item $-L(-f,P)=U(f,P)$;
		\item $L(f,P)\leqslant\sigma(f,P,\xi)\leqslant U(f,P),\;\inf\limits_{\xi}\sigma(f,P,\xi)=L(f,P),\;\sup\limits_{\xi}\sigma(f,P,\xi)=U(f,P)$;
		\item 设分割$P'$是由分割$P$添加$l$个分点构成的分割,则:
		\begin{align*}
			L(f,P)\leqslant L(f,P')\leqslant L(f,P)+l\omega|P| \\
			U(f,P)\geqslant U(f,P')\geqslant U(f,P)-l\omega|P|
		\end{align*}
		\item 设$P_1,P_2$是$[a,b]$的任意两个分割,则有$L(f,P_1)\leqslant U(f,P_2)$;
		\item $\lim\limits_{|P|\to0}L(f,P)=\underline{I},\;\lim\limits_{|P|\to0}U(f,P)=\overline{I}$。
	\end{enumerate}
\end{property}
\begin{proof}
	(1)注意到:
	\begin{equation*}
		-L(-f,P)=-\sum_{i=1}^{n}-M_i\Delta x_i=\sum_{i=1}^{n}M_i\Delta x_i=U(f,P)
	\end{equation*}\par
	(2)由定义即可得出结论。\par
	(3)只证明在子区间$[x_{i-1},x_{i}]$上添加一个分点$x'$的情况,多个分点的结论可直接由此推出。此时$L(f,P)$和$L(f,P')$的区别仅在于前者的$m_i(x_i-x_{i-1})$被$m_i'(x'-x_{i-1})+m_i''(x_i-x')$替代,于是:
	\begin{align*}
		L(f,P')-L(f,P)&=m_i'(x'-x_{i-1})+m_i''(x_i-x')-m_i(x_i-x_{i-1}) \\
		&=m_i'(x'-x_{i-1})+m_i''(x_i-x')-m_i(x_i-x'+x'-x_{i-1}) \\
		&=(m_i''-m_i)(x_i-x')+(m_i'-m_i)(x'-x_{i-1})
	\end{align*}
	注意到$m_i\leqslant m_i',m_i''\leqslant M_i$,所以有:
	\begin{equation*}
		0\leqslant L(f,P')-L(f,P)\leqslant(M_i-m_i)(x_i-x'+x'-x_{i-1})=\omega_i\Delta x_i\leqslant\omega|P|
	\end{equation*}
	即:
	\begin{equation*}
		L(f,P)\leqslant L(f,P')\leqslant L(f,P)+\omega|P|
	\end{equation*}\par
	由(1)可得:
	\begin{equation*}
		0\leqslant U(f,P)-U(f,P')=-L(-f,P)+L(-f,P')\leqslant-\omega|P|
	\end{equation*}
	即:
	\begin{equation*}
		U(f,P)\geqslant U(f,P')\geqslant U(f,P)-\omega|P|
	\end{equation*}\par
	(4)设$P'$为$P_1$和$P_2$的分点合并在一起构成的分割,则由(3)(2)可得:
	\begin{equation*}
		L(f,P_1)\leqslant L(f,P')\leqslant U(f,P')\leqslant U(f,P_2)
	\end{equation*}\par
	(5)由上确界的定义可得对任意的$\varepsilon>0$,存在$[a,b]$上的分割:
	\begin{equation*}
		P_0:\;a=x_0<x_1<\cdots<x_n=b
	\end{equation*}
	使得:
	\begin{equation*}
		\underline{I}-\frac{\varepsilon}{2}<L(f,P_0)<\underline{I}
	\end{equation*}
	取$[a,b]$上的分割$P$满足:
	\begin{equation*}
		|P|<\delta=\frac{\varepsilon}{2l\omega+1}
	\end{equation*}
	将$P_0,P$得分点合在一起得到分割$P'$,则由(3)可得:
	\begin{equation*}
		\underline{I}\geqslant L(f,P)\geqslant L(f,P')-l\omega|P|\geqslant L(f,P_0)-l\omega|P|>\underline{I}-\frac{\varepsilon}{2}-\frac{\varepsilon}{2}=\underline{I}-\varepsilon
	\end{equation*}
	即对任意的$\varepsilon>0$,只要$|P|<\delta=\dfrac{\varepsilon}{2l\omega+1}$,就有$|L(f,P)-\underline{I}|<\varepsilon$,所以:
	\begin{equation*}
		\lim_{|P|\to0}L(f,P)=\underline{I}
	\end{equation*}\par
	由(1)、\info{极限的线性性}和\info{上下确界的数乘}可得:
	\begin{align*}
		\lim_{|P|\to0}U(f,P)&=\lim_{|P|\to0}-L(-f,P)=-\lim_{|P|\to0}L(-f,P)=-\sup_PL(-f,P) \\
		&=-\sup_P[-U(f,P)]=\inf_PU(f,P)=\overline{I}\qedhere
	\end{align*}
\end{proof}
\begin{theorem}\label{theo:RiemannIntegrableIff}
	设函数$f$在闭区间$[a,b]\subseteq\mathbb{R}^{}$上有定义并且有界,则下述三条结论等价:
	\begin{enumerate}
		\item 对任意的$\varepsilon>0$,存在$[a,b]$上的分割$P$使得$\Omega(f,P)=U(f,P)-L(f,P)<\varepsilon$;
		\item $f$在$[a,b]$上满足$\underline{I}=\overline{I}$,也即$\lim\limits_{|P|\to0}\Omega(f,P)=0$;
		\item $f$在$[a,b]$上可积且$\int_{a}^{b}f(x)\dif x=\overline{I}=\underline{I}$。
	\end{enumerate}
\end{theorem}
\begin{proof}
	\textbf{(1)$\Rightarrow$(2):}由\info{极限的线性性}可得:
	\begin{equation*}
		\overline{I}-\underline{I}=\lim_{|P|\to0}U(f,P)-\lim_{|P|\to0}L(f,P)=\lim_{|P|\to0}[U(f,P)-L(f,P)]=\lim_{|P|\to0}\Omega(f,P)
	\end{equation*}
	因为对$[a,b]$上的任意分割$P$都有$U(f,P)-L(f,P)\geqslant0$和$U(f,P)-L(f,P)\geqslant\overline{I}-\underline{I}$,所以由极限的不等式性可得:
	\begin{equation*}
		0\leqslant\overline{I}-\underline{I}\leqslant U(f,P)-L(f,P)
	\end{equation*}
	对任意的$\varepsilon>0$,令$P$为使得$U(f,P)-L(f,P)<\varepsilon$成立的分割,就有$\overline{I}-\underline{I}<\varepsilon$,即$\overline{I}=\underline{I}$。\par
	\textbf{(2)$\Rightarrow$(3):}记$\overline{I}=\underline{I}=I$,所以:
	\begin{equation*}
		\lim_{|P|\to0}L(f,P)=\lim_{|P|\to0}U(f,P)=I
	\end{equation*}
	由\cref{prop:DarbouxSum}(2)和\info{夹逼定理}可得:
	\begin{equation*}
		\lim_{|P|\to0}\sigma(f,P,\xi)=I
	\end{equation*}
	即$f$在$[a,b]$上可积且积分值为$I$。\par
	\textbf{(3)$\Rightarrow$(1):}由可积的定义,对任意的$\varepsilon>0$,存在$\delta>0$使得只要$|P|<\delta$,就有:
	\begin{equation*}
		I-\frac{\varepsilon}{3}<\sigma(f,P,\xi)<I+\frac{\varepsilon}{3}
	\end{equation*}
	对于这个$P$,由\cref{prop:DarbouxSum}(2)和上下确界的不等式性可得:
	\begin{equation*}
		I-\frac{\varepsilon}{3}\leqslant L(f,P)\leqslant U(f,P)\leqslant I+\frac{\varepsilon}{3}
	\end{equation*}
	于是有:
	\begin{equation*}
		U(f,P)-L(f,P)\leqslant\frac{2\varepsilon}{3}<\varepsilon\qedhere
	\end{equation*}
\end{proof}

\subsection{可积函数类}
\begin{lemma}\label{lem:omega=SupVari}
	设函数$f$在区间$J$上有定义,记:
	\begin{equation*}
		M=\sup_{x\in J}f(x),\quad m=\inf_{x\in J}f(x),\quad\omega=M-m
	\end{equation*}
	则有:
	\begin{equation*}
		\omega=\sup_{x_1,x_2\in J}|f(x_1)-f(x_2)|
	\end{equation*}
\end{lemma}
\begin{proof}
	若$\omega=0$,则结论显然成立,下对$\omega>0$的情况进行证明。\par
	对任意的$x_1,x_2\in J$,显然有:
	\begin{equation*}
		|f(x_1)-f(x_2)|\leqslant\omega
	\end{equation*}
	由上确界的不等式性即可得:
	\begin{equation*}
		\sup_{x_1,x_2\in J}|f(x_1)-f(x_2)|\leqslant\omega
	\end{equation*}
	对于任意的$0<\varepsilon<\omega$,由上确界和下确界的定义可得存在$x_1,x_2\in J$使得:
	\begin{equation*}
		f(x_1)>M-\frac{\varepsilon}{2}>m+\frac{\varepsilon}{2}>f(x_2)
	\end{equation*}
	于是:
	\begin{equation*}
		f(x_1)-f(x_2)>M-\frac{\varepsilon}{2}-m-\frac{\varepsilon}{2}=\omega-\varepsilon
	\end{equation*}
	由上确界的不等式性可得:
	\begin{equation*}
		\sup_{x_1,x_2\in J}|f(x_1)-f(x_2)|\geqslant\omega
	\end{equation*}
	所以:
	\begin{equation*}
		\omega=\sup_{x_1,x_2\in J}|f(x_1)-f(x_2)|\qedhere
	\end{equation*}
\end{proof}
\begin{theorem}\label{theo:RiemannIntegrable4ArithmeticOperation}
	设函数$f,g$在闭区间$[a,b]\subseteq\mathbb{R}^{}$上可积,$\alpha,\beta\in\mathbb{R}^{}$,则:
	\begin{enumerate}
		\item $\alpha f+\beta g$在$[a,b]$上可积且:
		\begin{equation*}
			\int_{a}^b[\alpha f(x)+\beta g(x)]\dif x=\alpha\int_{a}^bf(x)\dif x+\beta\int_{a}^bg(x)\dif x
		\end{equation*}
		\item $fg$在$[a,b]$上可积;
		\item 若存在常数$c>0$使得$|f(x)|\geqslant c,\;\forall\;x\in[a,b]$,则$\dfrac{1}{f}$在$[a,b]$上可积;
		\item $|f|$在$[a,b]$上可积。
	\end{enumerate}
\end{theorem}
\begin{proof}
	(1)因为$f,g$可积,所以可设:
	\begin{equation*}
		\lim_{|P|\to0}\sigma(f,P,\xi)=I_1\in\mathbb{R}^{},\quad\lim_{|P|\to0}\sigma(g,P,\xi)=I_2\in\mathbb{R}^{}
	\end{equation*}
	由极限的线性性质可得:
	\begin{align*}
		\lim_{|P|\to0}\sigma(\alpha f+\beta g,P,\xi)&=\lim_{|P|\to0}[\alpha\sigma(f,P,\xi)+\beta\sigma(g,P,\xi)] \\
		&=\alpha\lim_{|P|\to0}\sigma(f,P,\xi)+\beta\lim_{|P|\to0}\sigma(g,P,\xi) \\
		&=\alpha I_1+\beta I_2\in\mathbb{R}^{}
	\end{align*}
	由定义可得$\alpha f+\beta g$可积,所以:
	\begin{equation*}
		\int_{a}^b[\alpha f(x)+\beta g(x)]\dif x=\alpha\int_{a}^bf(x)\dif x+\beta\int_{a}^bg(x)\dif x
	\end{equation*}\par
	(2)由\cref{theo:RiemannIntegrableBounded}可知$f,g$在$[a,b]$上有界,设:
	\begin{equation*}
		|f(x)|\leqslant K,\;|g(x)|\leqslant L,\quad\forall\;x\in[a,b]
	\end{equation*}
	对于任意的$x,y\in[x_{i-1},x_i]$有:
	\begin{align*}
		|f(x)g(x)-f(y)g(y)|&=|f(x)g(x)-f(y)g(x)+f(y)g(x)-f(y)g(y)| \\
		&=|[f(x)-f(y)]g(x)+[g(x)-g(y)]f(y)| \\
		&\leqslant|f(x)-f(y)||g(x)|+|g(x)-g(y)||f(y)| \\
		&\leqslant|f(x)-f(y)|L+|g(x)-g(y)|K\leqslant\omega_i(f)L+\omega_i(g)K
	\end{align*}
	于是由\cref{lem:omega=SupVari}和上确界的不等式性有:
	\begin{equation*}
		\omega_i(fg)=\sup_{x,y\in[x_{i-1},x_i]}|f(x)g(x)-f(y)g(y)|\leqslant L\omega_i(f)+K\omega_i(g)
	\end{equation*}
	因为$\Omega_i(fg)=\omega_i(fg)\Delta x_i$,所以对$P$有:
	\begin{equation*}
		\Omega(fg,P)=\sum_{i=1}^{n}\omega_i(fg)\Delta x_i\leqslant L\sum_{i=1}^{n}\omega_i(f)\Delta x_i+K\sum_{i=1}^{n}\omega_i(g)\Delta x_i=L\Omega(f,P)+K\Omega(g,P)
	\end{equation*}
	因为$f,g$可积,由\cref{theo:RiemannIntegrableIff}(2)可得:
	\begin{equation*}
		\lim_{|P|\to0}\Omega(f,P)=\lim_{|P|\to0}\Omega(g,P)=0
	\end{equation*}
	由\info{极限的不等式性和线性性质}可得:
	\begin{equation*}
		0\leqslant\lim_{|P|\to0}\Omega(fg,P)\leqslant L\lim_{|P|\to0}\Omega(f,P)+K\lim_{|P|\to0}\Omega(g,P)=0
	\end{equation*}
	即:
	\begin{equation*}
		\lim_{|P|\to0}\Omega(fg,P)=0
	\end{equation*}
	由\cref{theo:RiemannIntegrableIff}(2)可得$fg$在$[a,b]$上可积。\par
	(3)对任意的$x,y\in[x_{i-1},x_i]$,有:
	\begin{equation*}
		\left|\frac{1}{f(x)}-\frac{1}{f(y)}\right|=\left|\frac{f(y)-f(x)}{f(x)f(y)}\right|\leqslant\frac{|f(y)-f(x)|}{c^2}
	\end{equation*}
	由\cref{lem:omega=SupVari}和上确界的不等式性可得:
	\begin{equation*}
		\omega_i\left(\frac{1}{f}\right)=\sup_{x,y\in[x_{i-1},x_i]}\left|\frac{1}{f(x)}-\frac{1}{f(y)}\right|\leqslant\sup_{x,y\in[x_{i-1},x_i]}\frac{|f(y)-f(x)|}{c^2}=\frac{\omega_i(f)}{c^2}
	\end{equation*}
	因为$\Omega_i\left(\dfrac{1}{f}\right)=\omega_i\left(\dfrac{1}{f}\right)\Delta x_i$,所以对$P$有:
	\begin{equation*}
		\Omega\left(\dfrac{1}{f},P\right)=\sum_{i=1}^{n}\omega_i\left(\dfrac{1}{f}\right)\Delta x_i\leqslant\sum_{i=1}^{n}\frac{\omega_i(f)\Delta x_i}{c^2}=\frac{\Omega(f,P)}{c^2}
	\end{equation*}
	因为$f$可积,由\cref{theo:RiemannIntegrableIff}(2)可得:
	\begin{equation*}
		\lim_{|P|\to0}\Omega(f,P)=0
	\end{equation*}
	由\info{极限的不等式性和线性性质}可得:
	\begin{equation*}
		0\leqslant\lim_{|P|\to0}\Omega\left(\frac{1}{f},P\right)\leqslant\lim_{|P|\to0}\frac{\Omega(f,P)}{c^2}=0
	\end{equation*}
	即:
	\begin{equation*}
		\lim_{|P|\to0}\Omega\left(\frac{1}{f},P\right)=0
	\end{equation*}
	由\cref{theo:RiemannIntegrableIff}(2)可得$\dfrac{1}{f}$在$[a,b]$上可积。\par
	(4)对任意的$x,y\in[x_{i-1},x_i]$有:
	\begin{equation*}
		\Big||f(x)|-|f(y)|\Big|\leqslant|f(x)-f(y)|
	\end{equation*}
	所以由\cref{lem:omega=SupVari}和上确界的不等式性可得:
	\begin{equation*}
		\omega_i(|f|)=\sup_{x,y\in[x_{i-1},x_i]}\Big||f(x)-f(y)|\Big|\leqslant\sup_{x,y\in[x_{i-1},x_i]}|f(x)-f(y)|=\omega_i(f)
	\end{equation*}
	因为$\Omega_i(|f|)=\omega_i(|f|)\Delta x_i$,所以对$P$有:
	\begin{equation*}
		\Omega(|f|,P)=\sum_{i=1}^{n}\omega_i(|f|)\Delta x_i\leqslant\sum_{i=1}^{m}\omega_i(f)\Delta x_i=\Omega(f,P)
	\end{equation*}
	因为$f$可积,由\cref{theo:RiemannIntegrableIff}(2)可得:
	\begin{equation*}
		\lim_{|P|\to0}\Omega(f,P)=0
	\end{equation*}
	由\info{极限的不等式性}可得:
	\begin{equation*}
		0\leqslant\lim_{|P|\to0}\Omega(|f|,P)\leqslant\lim_{|P|\to0}\Omega(f,P)=0
	\end{equation*}
	即:
	\begin{equation*}
		\lim_{|P|\to0}\Omega\left(|f|,P\right)=0
	\end{equation*}
	由\cref{theo:RiemannIntegrableIff}(2)可得$|f|$在$[a,b]$上可积。
\end{proof}
\begin{theorem}\label{theo:RiemannIntegrable}
	关于函数的可积性,有如下结论:
	\begin{enumerate}
		\item 设$[a_1,b_1]\subseteq[a,b]\subseteq\mathbb{R}^{}$,若函数$f$在$[a,b]$上可积,则$f$在$[a_1,b_1]$上可积;
		\item 若函数$f$在闭区间$[a,b]\subseteq\mathbb{R}^{}$上有定义且单调有界,则$f$在$[a,b]$上可积;
		\item 若函数$f$在闭区间$[a,b]\subseteq\mathbb{R}^{}$上连续,则$f$在$[a,b]$上可积;
		\item 若函数$f$在闭区间$[a,b]\subseteq\mathbb{R}^{}$上有界且除了有限个间断点外在其它各点连续,则$f$在$[a,b]$上可积;
		\item 若函数$f,g$在闭区间$[a,b]\subseteq\mathbb{R}^{}$上有定义且有界,除了有限个点$\seq{c}{l}$以外有$f=g$,此时如果$f$可积,则$g$也可积,且有:
		\begin{equation*}
			\int_{a}^bf(x)\dif x=\int_{a}^bg(x)\dif x
		\end{equation*}
	\end{enumerate}
\end{theorem}
\begin{proof}
	(1)因为$f$在$[a,b]$上可积,所以对任意的$\varepsilon>0$,存在$\delta>0$使得只要$[a,b]$上的分割$P_1$满足$|P_1|<\delta$就有$\Omega(f,P_1)<\varepsilon$。为$P_1$再增加两个分点$a_1,b_1$得到分割$P_2$(若原本就是分点则不作改变),由\cref{prop:DarbouxSum}(3)可知$\Omega(f,P_2)<\Omega(f,P_1)$,对于$P_2$中$[a_1,b_1]$上的子分割$P$有:
	\begin{equation*}
		0<\Omega(f,P)<\Omega(f,P_2)<\Omega(f,P_1)<\varepsilon
	\end{equation*}
	而$|P|<|P_1|<\delta$,所以由\info{夹逼定理和极限的不等式性}可得:
	\begin{equation*}
		0\leqslant\lim_{|P|\to0}\Omega(f,P)=\lim_{|P_1|\to0}\Omega(f,P)\leqslant\lim_{|P_1|\to0}\Omega(f,P_1)=0
	\end{equation*}
	于是:
	\begin{equation*}
		\lim_{|P|\to0}\Omega(f,P)=0
	\end{equation*}
	由\cref{theo:RiemannIntegrableIff}(2)可得$f$在$[a_1,b_1]$上可积。\par
	(2)设$f$单调增。$P$是$[a,b]$上的分割,对任意的$\varepsilon>0$,取:
	\begin{equation*}
		\delta=\frac{\varepsilon}{f(b)-f(a)+1}
	\end{equation*}
	则当$|P|<\delta$时有:
	\begin{equation*}
		\Omega(f,P)=\sum_{i=1}^{n}\omega_i\Delta x_i<\delta\sum_{i=1}^{n}\omega_i=\delta[f(b)-f(a)]=\frac{\varepsilon[f(b)-f(a)]}{f(b)-f(a)+1}<\varepsilon
	\end{equation*}
	所以:
	\begin{equation*}
		\lim_{|P|\to0}\Omega(f,P)=0
	\end{equation*}
	由\cref{theo:RiemannIntegrableIff}(2)可得$f$在$[a,b]$上可积。\par
	对$f$单调减的情况只需在上述过程中取:
	\begin{equation*}
		\delta=\frac{\varepsilon}{f(a)-f(b)+1}
	\end{equation*}\par
	(3)因为$f$在$[a,b]$上连续,由\info{连续则一致连续}可得$f$在$[a,b]$上一致连续,所以对任意的$\varepsilon>0$,存在$\delta>0$,当$|x-y|<\delta$时就有$|f(x)-f(y)|<\varepsilon$。取$|P|<\delta$,于是:
	\begin{equation*}
		\Omega(f,P)=\sum_{i=1}^{n}\omega_i\Delta x_i<\varepsilon\sum_{i=1}^{n}\Delta x_i=(b-a)\varepsilon
	\end{equation*}
	所以:
	\begin{equation*}
		\lim_{|P|\to0}\Omega(f,P)=0
	\end{equation*}
	由\cref{theo:RiemannIntegrableIff}(2)可得$f$在$[a,b]$上可积。\par
	(4)设$f$的间断点为$\seq{c}{l}$,对任意给定的充分小的$\eta$,取$l$个开区间:
	\begin{equation*}
		J_j=(c_j-\eta,c_j+\eta),\quad j=1,2,\dots,l
	\end{equation*}
	在$[a,b]$减去$\seq{J}{l}$后所余下的有限个闭子区间上,$f$是连续的,由\info{连续则一致连续}可得$f$在其上一致连续,所以存在$\delta>0$,使得只要:
	\begin{equation*}
		x_,y\in[a,b]\Big\backslash\left(\underset{j=1}{\overset{l}{\cup}}J_j\right),\;|x-y|<\delta
	\end{equation*}
	就有$|f(x)-f(y)|<\eta$。取$[a,b]$上的分割$P$使得$|P|<\min\{\eta,\delta\}$。因为$|P|<\eta$,注意到一个$J_j$最多与四个$P$的闭子区间相交,所以在$P$的各闭子区间中,最多只有总长度不超过$4l\eta$的子区间与某个$J_j$相交。可以把$\Omega(f,P)$分成两部分:
	\begin{equation*}
		\Omega(f,P)=\sum_s'\omega_s\Delta x_s+\sum_t''\omega_t\Delta x_t
	\end{equation*}
	第一部分与所有的$J_j$都不相交,第二部分与某个$J_j$相交,于是有:
	\begin{align*}
		\Omega(f,P)&=\sum_s'\omega_s\Delta x_s+\sum_t''\omega_t\Delta x_t<\eta\sum_{s}'\omega_s+\omega\sum_{t}''\Delta x_t \\
		&\leqslant\eta(b-a)+\omega4l\eta=(b-a+4l\omega)\eta
	\end{align*}
	对任意的$\varepsilon>0$,只要取:
	\begin{equation*}
		\eta<\frac{\varepsilon}{b-a+4l\omega}
	\end{equation*}
	就可以得到$\Omega(f,P)<\varepsilon$,即:
	\begin{equation*}
		\lim_{|P|\to0}\Omega(f,P)=0
	\end{equation*}
	由\cref{theo:RiemannIntegrableIff}(2)可得$f$在$[a,b]$上可积。\par
	(5)记:
	\begin{equation*}
		K=\max\{|g(c_1)-f(c_1)|,|g(c_2)-f(c_2)|,\dots,|g(c_l)-f(c_l)|\}
	\end{equation*}
	设$P$是$[a,b]$上的任意分割,则在$P$的各个闭子区间中,最多只有$2l$个能够包含某个$c_i$(考虑$|P|$足够小的情况,这时一个闭子区间只能含有一个$c_i$),同时当且仅当$f$在$P$的包含$c_i$的闭子区间上的最大值或最小值是$f(c_i)$或$g$在该区间上的极值为$g(c_i)$时$\omega_i(f)$和$\omega_i(g)$才会不一样。分类讨论可得此时$|\omega_i(g)-\omega_i(f)|\leqslant K$,所以:
	\begin{align*}
		|\sigma(g,P,\xi)-\sigma(f,P,\xi)|&\leqslant\sum_{i=1}^{n}|[\omega_i(g)-\omega_i(f)]\Delta x_i| \\
		&\leqslant\sum_{i=1}^{n}|[\omega_i(g)-\omega_i(f)]||P|\leqslant2lK|P|
	\end{align*}
	于是:
	\begin{equation*}
		\int_{a}^bf(x)\dif x=\lim_{|P|\to0}\sigma(f,P,\xi)=\lim_{|P|\to0}\sigma(g,P,\xi)=\int_{a}^bg(x)\dif x\qedhere
	\end{equation*}
\end{proof}

\subsection{定积分的性质}
\begin{property}
	定积分具有如下性质:
	\begin{enumerate}
		\item 设$a<b<c$,若$f$在$[a,b]$和$[b,c]$上都可积,则$f$在$[a,c]$上也可积,且:
		\begin{equation*}
			\int_{a}^cf(x)\dif x=\int_{a}^bf(x)\dif x+\int_{b}^cf(x)\dif x
		\end{equation*}
		\item 若函数$f,g$在闭区间$[a,b]\in\mathbb{R}^{}$上可积且$f\leqslant g$在$[a,b]$上恒成立,则:
		\begin{equation*}
			\int_{a}^bf(x)\dif x\leqslant\int_{a}^bg(x)\dif x
		\end{equation*}
	\end{enumerate}
\end{property}
\begin{proof}
	(1)因为,由\cref{theo:RiemannIntegrableBounded}可知$f$在$[a,c]$上有界,即存在$M>0$使得$|f(x)|\leqslant M$对任意的$x\in[a,c]$成立。\par
	取$[a,c]$上的一个分割$P$,若$b$在$P$的标志点集$\xi$中,则令$P'=P,\xi'=\xi$,其中$\xi'$是$P'$的标志点集。若$b$不在$\xi$中,则令:
	\begin{equation*}
		P':\;a=x_0<x_1<x_2<\cdots<x_j<b<x_{j+1}<\cdots<x_n=c,\quad \xi'=\xi\cup{b}
	\end{equation*}
	其中$x_i\in\xi,\;i=1,2,\dots,n$。注意到$\sigma(f,P,\xi)$和$\sigma(f,P',\xi')$的差别至多为:
	\begin{align*}
		|\sigma(f,P',\xi')-\sigma(f,P,\xi)|&=|f(\xi'_{j+1})(b-x_j)+f(\xi'_{j+2})(x_{j+1}-b)-f(\xi_{j+1})(x_{j+1}-x_j)| \\
		&<|f(\xi'_{j+1})+f(\xi'_{j+2})-f(\xi_{j+1})||P|\leqslant3M|P|
	\end{align*}
	即:
	\begin{equation*}
		\lim_{|P|\to0}\sigma(f,P,\xi)=\sigma(f,P',\xi')
	\end{equation*}
	将$P'$分别限制在$[a,b]$和$[b,c]$上可以得到分割$P_1,P_2$及其对应的标志点组$\varphi,\psi$。因为$f$在$[a,b]$和$[b,c]$上都可积,所以极限:
	\begin{equation*}
		\lim_{|P_1|\to0}\sigma(f,P_1,\varphi),\quad\lim_{|P_2|\to0}\sigma(f,P_2,\psi)
	\end{equation*}
	存在,因为:
	\begin{equation*}
		\sigma(f,P',\xi')=\sigma(f,P_1,\varphi)+\sigma(f,P_2,\psi),\quad|P'|\to0\longrightarrow|P_1|,|P_2|\to0
	\end{equation*}
	由\info{极限的运算法则}可得:
	\begin{equation*}
		\lim_{|P'|\to0}\sigma(f,P',\xi)=\lim_{|P_1|\to0}\sigma(f,P_1,\varphi)+\lim_{|P_2|\to0}\sigma(f,P_2,\psi)=\int_{a}^bf(x)\dif x+\int_{b}^cf(x)\dif x
	\end{equation*}
	而$|P|\to0$包含了$|P'|\to0$,所以:
	\begin{equation*}
		\lim_{|P|\to0}\sigma(f,P,\xi)=\int_{a}^bf(x)\dif x+\int_{b}^cf(x)\dif x\in\mathbb{R}^{}
	\end{equation*}
	由定积分的定义即可得到:
	\begin{equation*}
		\int_{a}^cf(x)\dif x=\int_{a}^bf(x)\dif x+\int_{b}^cf(x)\dif x
	\end{equation*}\par
	(2)因为$f\leqslant g$在$[a,b]$上恒成立,所以$\varphi=g-f\geqslant0$在$[a,b]$上恒成立。因为$f,g$在$[a,b]$上可积,由\cref{theo:RiemannIntegrable4ArithmeticOperation}(1)可知$\varphi$在$[a,b]$上可积且:
	\begin{equation*}
		\int_{a}^b\varphi(x)\dif x=\int_{a}^bg(x)\dif x-\int_{a}^bf(x)\dif x
	\end{equation*}
	因为$\varphi$在$[a,b]$上非负,所以对$[a,b]$上任意的分割$P$及其标志点组$\xi$都有$\sigma(\varphi,P,\xi)\geqslant0$,由\info{极限的不等式性}可得:
	\begin{equation*}
		\int_{a}^b\varphi(x)\dif x=\lim_{|P|\to0}\sigma(\varphi,P,\xi)\geqslant0
	\end{equation*}
	所以:
	\begin{equation*}
		\int_{a}^bg(x)\dif x-\int_{a}^bf(x)\dif x\geqslant0
	\end{equation*}\par
\end{proof}

\chapter{Lebesgue积分}
本章讨论Lebesgue积分。\par
在前半部分作好定义Lebesgue积分的所有准备工作,即讨论$\mathbb{R}^n$上的测度。\par
\gls{measure}是一种映射,它把集合映射到某一个实数,是一种将几何空间的度量(长度、面积、体积)和其他常见概念(如大小、质量和事件的概率)广义化后产生的概念。\par
我们的目的是建立一种定义\footnote{这里的定义并不是函数的定义域那种含义的定义,可以证明实数直线上存在不可测集合,这里类似于概率空间$(\Omega,\mathcal{F},P)$,实数直线是类似样本空间的概念。}在$\mathbb{R}^n$上的测度,使它满足以下的\gls{lmeasureaxiom}:
\begin{axiom}\label{axi:Lebesguem}
	对于$\mathbb{R}^n$上的某一集合族$\mathcal{F}$(待定义\label{sec:lebesgue可测集合族}),测度\gls{lmeasure}$m$有如下性质:
	\begin{enumerate}
		\item 非负性:对任意的$ E\in\mathcal{F}$,$m(E)\geqslant0$。
		\item 可列可加性:若$E_n\in\mathcal{F},\;\forall\;n\in\mathbb{N}^+$,且对任意的$i,j\in\mathbb{N}^+,\;i\ne j,\;E_i\cap E_j=\varnothing$,则有$m(\underset{i=1}{\overset{+\infty}{\cup}} E_i)=\sum\limits_{i=1}^{+\infty}m(E_i)$。
		\item 正则性:单位立方体$[0,1]^n\subset\mathcal{F}$且它的测度$m([0,1]^n)=1$。
	\end{enumerate}
\end{axiom}
\section{外测度}
\subsection{外测度的定义}
\begin{definition}
	设$E$为$\mathbb{R}^n$中的任一点集,$E$的Lebesgue\gls{EMeasure}定义为所有能够覆盖$E$的开区间列的体积总和的下确界。即:
	\begin{equation}
		m^*(E)=\inf_{E\subset\underset{i=1}{\overset{+\infty}{\cup}}I_i}\sum_{i=1}^{+\infty}|I_i|\notag
	\end{equation}
\end{definition}
需要注意如下事项:
\begin{enumerate}
	\item 必须是无穷多个开区间的体积总和的下确界。如果将定义改为有限个:取$E$为$[0,1]$内的有理数集,若$E$被有限个开区间覆盖,即$E\subset\underset{i=1}{\overset{n}{\cup}}I_i$,那么$\underset{i=1}{\overset{n}{\cup}}I_i$一定覆盖$[0,1]$(反证法,有理数集的稠密性),即这种定义下$E$的外测度等于$1$。同理,$[0,1]$内无理数集的外测度也为$1$,由测度公理$(2)$,$[0,1]$的测度为$2$,而由测度公理$(3)$,$[0,1]$的测度应为$1$,矛盾。
	\item 定义虽然是无穷多个开区间,但实际仍可以是有限的,因为可以取空集(就比如概率论可列可加性推有限可加性)。 
	\item 体积总和可以是$+\infty$。
\end{enumerate}
\subsection{外测度的性质}
\begin{property}
	外测度有以下三条基本性质:
	\begin{enumerate}
		\item $m^*(E)\geqslant0$,等号成立当且仅当$E=\varnothing$。
		\item 若$A\subset B$,则有$m^*(A)\leqslant m^*(B)$。
		\item 	$m^*(\underset{i=1}{\overset{+\infty}{\cup}}A_i)\leqslant\sum\limits_{i=1}^{+\infty}m^*(A_i)$。
	\end{enumerate}
\end{property}
\begin{proof}
	(1)显然成立。\par
	(2)对于满足条件$B\subset\underset{i=1}{\overset{+\infty}{\cup}}I_i$的任意开区间列$\{I_i,\;i\in\mathbb{N}^+\}$,必然有$A\subset\underset{i=1}{\overset{+\infty}{\cup}}I_i$,又因为$m^*(A)$是长度总和的下确界,那么就有:
	\begin{equation}
		m^*(A)\leqslant\sum_{i=1}^{+\infty}|I_i|\notag
	\end{equation}
	由下确界的保号性即可推得:
	\begin{equation}
		m^*(A)\leqslant\inf_{B\subset\underset{i=1}{\overset{+\infty}{\cup}}I_i}\sum_{i=1}^{+\infty}|I_i|=m^*(B)\notag
	\end{equation}\par
	(3)从单个$A_i$开始着手考虑。因为外测度是下确界,对任意$\varepsilon>0$和每个$A_n,n\in\mathbb{N}^+$,都存在一个开区间列$\{I_{ni},\;i\in\mathbb{N}^+\}$使\footnote{$\frac{\varepsilon}{2^n}$是一种为了在求和后得到$\varepsilon$的常用规范化取法,这是因为$\sum\limits_{n=1}^{+\infty}\frac{1}{2^n}=1$}:
	\begin{equation}
		\sum_{i=1}^{+\infty}|I_{ni}|<m^*(A_n)+\frac{\varepsilon}{2^n},\;A_n\subset\underset{i=1}{\overset{+\infty}{\cup}}I_{ni}\notag
	\end{equation}
	那么就有(第一个不等式是因为外测度是下确界,第二个不等式是上式的求和形式):
	\begin{equation}
		m^*(\underset{i=1}{\overset{+\infty}{\cup}}A_i)\leqslant\sum_{n=1}^{+\infty}\sum_{i=1}^{+\infty}|I_{ni}|<\sum_{n=1}^{+\infty} m^*(A_n)+\varepsilon\notag
	\end{equation}
	由$\varepsilon$的任意性,有:
	\begin{equation*}
		m^*(\underset{i=1}{\overset{+\infty}{\cup}}A_i)\leqslant\sum_{n=1}^{+\infty}\sum_{i=1}^{+\infty}|I_{ni}|\leqslant\sum_{n=1}^{+\infty} m^*(A_n)\qedhere 
	\end{equation*} 
\end{proof}
\subsection{区间的外测度}
\begin{theorem}
	区间的外测度就是区间的体积,即$m^*(I)=|I|$。
\end{theorem}
从直观上这一点很好理解,如果$m^*(I)>|I|$,则必然能找到一个体积总和更小的开区间列覆盖$I$;如果$m^*(I)<|I|$,则$I$必然没有被对应的开区间列全覆盖。
\begin{proof}
	(1)对任意区间$I$,必存在一个开区间$I'$使$I\subset I'$且$|I'|<|I|+\varepsilon$。那么就有(第一个不等式是因为外测度是下确界):
	\begin{equation}
		m^*(I)\leqslant|I'|<|I|+\varepsilon\notag
	\end{equation}
	由$\varepsilon$的任意性,即有:
	\begin{equation}
		m^*(I)\leqslant|I|\notag
	\end{equation}
	(2)如果$m^*(I)<|I|$,由外测度定义,必存在一个开区间列$\{A_i\}$,使$I\subset\underset{i=1}{\overset{+\infty}{\cup}}A_i$且$\sum\limits_{i=1}^{+\infty}|A_i|<|I|$,而这是不可能的。
\end{proof}













\section{可测集的定义与性质}
事情是这样的:我们就非要把外测度变成Lebesgue测度。目前为止外测度是不是Lebesgue测度呢?由外测度的性质$(1)$和区间的外测度,外测度显然满足\cref{axi:Lebesguem}的第一条和第三条,但是在$\mathbb{R}^n$上,人们确实能够证明外测度不具有可列可加性。事实上,$\mathbb{R}^n$上的确存在互不相交的一列集合$\{E_i\}$,使得:
\begin{equation}
	m(\underset{i=1}{\overset{+\infty}{\cup}}E_i)<\sum\limits_{i=1}^{+\infty} m(E_i)\notag
\end{equation}
因此外测度还不是Lebesgue测度。于是我们选择修改外测度的定义域,找到某个定义在$\mathbb{R}^n$上的集合族$\mathcal{F}$,使得外测度在$\mathcal{F}$上成为Lebesgue测度,不在$\mathcal{F}$中的$\mathbb{R}^n$的子集便成为不可测集。\par
下面给出这个集合族的定义。
\begin{definition}[Caratheodory condition]
	设$E$为$\mathbb{R}^n$中的点集,如果对$\mathbb{R}^n$中的任一点集$T$,都有:
	\begin{equation}
		m^*(T)=m^*(T\cap E)+m^*(T\cap E^c)\notag
	\end{equation}
	则$E\in\mathcal{F}$。此时称$E$是Lebesgue可测的,$E$的Lebesgue测度即为$E$的外测度\footnote{此时外测度的性质便成为Lebesgue测度的性质了。},记为$m(E)$。
\end{definition}
下面我们来探索这个定义带来的可测集的性质。
\begin{lemma}\label{lem:EmeasureAB}
	集合$E$可测的充要条件是对与$\forall\;A\subset E,\;\forall\;B\subset E^c$,总有:
	\begin{equation}
		m^*(A\cup B)=m^*(A)+m^*(B)\notag
	\end{equation}
\end{lemma}
\begin{proof}
	必要性:对任意的$A\subset E,\;\forall\;B\subset E^c$,取$T=A\cup B$,因为$E$可测,那么对于这个$T$,应有:
	\begin{equation}
		m^*(A\cup B)=m^*(T)=m^*(T\cap E)+m^*(T\cap E^c)=m^*(A)+m^*(B)\notag
	\end{equation}
	充分性:对任意的$T$,$\exists\;A\subset E,\;\exists\; B\subset E^c$,使得$T=A\cup B$,那么就有:
	\begin{equation}
		m^*(T)=m^*(A\cup B)=m^*(A)+m^*(B)=m^*(T\cap E)+m^*(T\cap E^c)\notag
	\end{equation}
	由$T$的任意性,$E$可测。
\end{proof}
\begin{theorem}
	$\varnothing$和$\mathbb{R}^n$可测。
\end{theorem}
\begin{proof}
	代入定义直接可得。
\end{proof}
\begin{theorem}
	$S$可测的充要条件是$S^c$可测。
\end{theorem}
\begin{proof}
	若$S$可测,对任意的$T\in\mathbb{R}^n$,则有:
	\begin{align*}
		m^*(T)=m^*(T\cap S)+m^*(T\cap S^c)
		&=m^*[T\cap(S^c)^c]+m^*(T\cap S^c) \\
		&=m^*(T\cap S^c)+m^*[T\cap(S^c)^c]\qedhere
	\end{align*} 
\end{proof}
\begin{theorem}
	若$S_1,S_2$都可测,则$S_1\cup S_2$也可测。当$S_1\cap S_2=\varnothing$时,对任意的$T$都有:
	\begin{equation}
		m^*[T\cap(S_1\cup S_2)]=m^*(T\cap S_1)+m^*(T\cap S_2)\notag
	\end{equation}
\end{theorem}
\begin{proof}
	因为$S_1$可测,对任意的$T$都有:
	\begin{equation}\label{eq:S_1measure}
		m^*(T)=m^*(T\cap S_1)+m^*(T\cap S_1^c)
	\end{equation}
	因为$S_2$可测,对于$m^*(T\cap S_1^c)$有:
	\begin{equation}\label{eq:TcapS_1measure}
		m^*(T\cap S_1^c)=m^*[(T\cap S_1^c)\cap S_2]+m^*[(T\cap S_1^c)\cap S_2^c]
	\end{equation}
	将\eqref{eq:TcapS_1measure}式代入\eqref{eq:S_1measure}式,再由德摩根公式,得到:
	\begin{align*}
		m^*(T)&=m^*(T\cap S_1)+m^*[(T\cap S_1^c)\cap S_2]+m^*[(T\cap S_1^c)\cap S_2^c] \\
		&=m^*(T\cap S_1)+m^*[(T\cap S_1^c)\cap S_2]+m^*[T\cap(S_1\cup S_2)^c]
	\end{align*}
	由于$T\cap S_1\subset S_1$,$(T\cap S_1^c)\cap S_2\subset S_1^c$,满足\cref{lem:EmeasureAB}条件,因此上式的前两项可以合并:
	\begin{align*}
		m^*(T\cap S_1)+m^*[(T\cap S_1^c)\cap S_2]&=m^*[(T\cap S_1)\cup(T\cap S_1^c\cap S_2)] \\
		&=m^*\{T\cap[S_1\cup(S_1^c\cap S_2)]\} \\
		&=m^*\{T\cap[(S_1\cup S_1^c)\cap(S_1\cup S_2)]\} \\
		&=m^*[T\cap(S_1\cup S_2)]
	\end{align*}
	那么就有:
	\begin{equation*}
		m^*(T)=m^*[T\cap(S_1\cup S_2)]+m^*[T\cap(S_1\cup S_2)^c]
	\end{equation*}
	由$T$的任意性,$S_1\cup S_2$可测。\par
	当$S_1\cap S_2=\varnothing$时,显然$S_2\subset S_1^c$,那么就有$T\cap S_2\subset S_1^c$,由\cref{lem:EmeasureAB}:
	\begin{align*}
		m^*[T\cap(S_1\cup S_2)]&=m^*[(T\cap S_1)\cup(T\cap S_2)]\\
		&=m^*(T\cap S_1)+m^*(T\cap S_2)\qedhere
	\end{align*}
\end{proof}
\begin{corollary}\label{cor:ncup}
	设$S_i(i=1,2,\dots,n)$都可测,则$\underset{i=1}{\overset{n}{\cup}}S_i$也可测。当$S_i\cap S_j=\varnothing$时,有:
	\begin{equation*}
		m^*[T\cap(\underset{i=1}{\overset{n}{\cup}}S_i)]=\sum_{i=1}^nm^*(T\cap S_i)
	\end{equation*}
\end{corollary}
\begin{theorem}
	若$S_1,S_2$都可测,则$S_1\cap S_2$也可测。
\end{theorem}
\begin{proof}
	$S_1\cap S_2=[(S_1\cap S_2)^c]^c=[S_1^c\cup S_2^c]^c$。
\end{proof}
\begin{corollary}
	设$S_i(i=1,2,\dots,n)$都可测,则$\underset{i=1}{\overset{n}{\cap}}S_i$也可测。
\end{corollary}
\begin{theorem}
	若$S_1,S_2$都可测,则$S_1\backslash S_2$也可测。
\end{theorem}
\begin{proof}
	$S_1\backslash S_2=S_1\cap S_2^c$。
\end{proof}
\begin{theorem}
	设$\{S_i\}$是一列互不相交的可测集,则$\underset{i=1}{\overset{+\infty}{\cup}}S_i$也可测,并且有:
	\begin{equation*}
		m(\underset{i=1}{\overset{+\infty}{\cup}}S_i)=\sum_{i=1}^{+\infty} m(S_i)
	\end{equation*}
\end{theorem}
\begin{proof}
	由\cref{cor:ncup},对任意的$n$,$\underset{i=1}{\overset{n}{\cup}}S_i$都可测,那么对任意的$T$,就有(第一行到第二行利用外测度性质(2),第二行到第三行利用\cref{cor:ncup}):
	\begin{align*}
		m^*(T)&=m^*[T\cap(\underset{i=1}{\overset{n}{\cup}}S_i)]+m^*[T\cap(\underset{i=1}{\overset{n}{\cup}}S_i)^c] \\
		&\geqslant m^*[T\cap(\underset{i=1}{\overset{n}{\cup}}S_i)]+m^*[T\cap(\underset{i=1}{\overset{+\infty}{\cup}}S_i)^c] \\
		&=\sum_{i=1}^nm^*(T\cap S_i)+m^*[T\cap(\underset{i=1}{\overset{+\infty}{\cup}}S_i)^c]
	\end{align*}
	令$n\to+\infty$,有(第一行到第二行利用极限的不等式性,第二行到第三行利用外测度的性质(3)):
	\begin{align}
		m^*(T)&\geqslant\sum_{i=1}^nm^*(T\cap S_i)+m^*[T\cap(\underset{i=1}{\overset{+\infty}{\cup}}S_i)^c]\notag \\
		&\geqslant\sum_{i=1}^{+\infty} m^*(T\cap S_i)+m^*[T\cap(\underset{i=1}{\overset{+\infty}{\cup}}S_i)^c] \label{eq:TcupSi}\\
		&\geqslant m^*[\underset{i=1}{\overset{+\infty}{\cup}}(T\cap S_i)]+m^*[T\cap(\underset{i=1}{\overset{+\infty}{\cup}}S_i)^c] \notag \\ &=m^*[T\cap(\underset{i=1}{\overset{+\infty}{\cup}}S_i)]+m^*[T\cap(\underset{i=1}{\overset{+\infty}{\cup}}S_i)^c]\notag
	\end{align}
	又因:
	\begin{equation*}
		T=[T\cap(\underset{i=1}{\overset{+\infty}{\cup}}S_i)]\cup[T\cap(\underset{i=1}{\overset{+\infty}{\cup}}S_i)^c]
	\end{equation*}
	由外测度的性质(3),有:
	\begin{equation*}
		m^*(T)\leqslant m^*[T\cap(\underset{i=1}{\overset{+\infty}{\cup}}S_i)]+m^*[T\cap(\underset{i=1}{\overset{+\infty}{\cup}}S_i)^c]
	\end{equation*}
	因此:
	\begin{equation*}
		m^*(T)= m^*[T\cap(\underset{i=1}{\overset{+\infty}{\cup}}S_i)]+m^*[T\cap(\underset{i=1}{\overset{+\infty}{\cup}}S_i)^c]
	\end{equation*}
	由$T$的任意性,$\underset{i=1}{\overset{+\infty}{\cup}}S_i$可测。\par
	令$T=\underset{i=1}{\overset{+\infty}{\cup}}S_i$,代入\cref{eq:TcupSi}式,则:
	\begin{align*}
		m^*(\underset{i=1}{\overset{+\infty}{\cup}}S_i)
		&\geqslant\sum_{i=1}^{+\infty} m^*[(\underset{i=1}{\overset{+\infty}{\cup}}S_i)\cap S_i]+m^*[(\underset{i=1}{\overset{+\infty}{\cup}}S_i)\cap(\underset{i=1}{\overset{+\infty}{\cup}}S_i)^c] \\
		&=\sum_{i=1}^{+\infty} m^*(S_i)
	\end{align*}
	但是由外测度的性质(3)有:
	\begin{equation*}
		m^*(\underset{i=1}{\overset{+\infty}{\cup}}S_i)\leqslant\sum_{i=1}^{+\infty} m^*(S_i)
	\end{equation*}
	因此:
	\begin{equation*}
		m^*(\underset{i=1}{\overset{+\infty}{\cup}}S_i)=\sum_{i=1}^{+\infty} m^*(S_i)\qedhere
	\end{equation*}
\end{proof}
\begin{corollary}
	设$\{S_i\}$是一列可测集合,则$\underset{i=1}{\overset{+\infty}{\cup}}S_i$也可测。
\end{corollary}
\begin{proof}
	$\underset{i=1}{\overset{+\infty}{\cup}}S_i$可被表示为互不相交的可数个集合的并:
	\begin{equation*}
		\underset{i=1}{\overset{+\infty}{\cup}}S_i=S_1\cup (S_2\backslash S_1)\cup[S_3\backslash(S_1\cup S_2)]\cup\cdots\qedhere
	\end{equation*}
\end{proof}
\begin{theorem}
	设$\{S_i\}$是一列可测集合,则$\underset{i=1}{\overset{+\infty}{\cap}}S_i$也可测。
\end{theorem}
\begin{proof}
	由德摩根公式:
	\begin{equation*}
		(\underset{i=1}{\overset{+\infty}{\cap}}S_i)^c=\underset{i=1}{\overset{+\infty}{\cup}}S_i^c\qedhere
	\end{equation*}
\end{proof}
做一个总结:
\begin{enumerate}
	\item $\mathcal{F}$中元素的可测性对可列并、可列交、补、差封闭。
	\item 定义在$\mathcal{F}$中的外测度满足可列可加性。
	\item $\mathcal{F}$是$\mathbb{R}^n$上的一个$\sigma$代数。
	 \footnote{定义:一个集合\(X\)的子集族\(\mathcal{M}\) 被称为一个\emph{σ-代数},当且仅当满足以下三个条件:(1)\(X\in\mathcal{F}\);(2)若\(A\in\mathcal{F}\),则\(A^c\in\mathcal{F}\);(3)若\(A_1,A_2,A_3,\dots\in\mathcal{F} \),则 \(\bigcup_{i=1}^{+\infty}A_i\in\mathcal{F}\)。}
\end{enumerate}

\section{可测集类}

\subsubsection{可测集类的通性}
\begin{definition}
	设集合$G$可表示成一列开集$\{G_i\}$的交集:
	\begin{equation*}
		G=\underset{i=1}{\overset{+\infty}{\cap}}G_i
	\end{equation*}
	则称$G$为$G_\delta$型集。
\end{definition}
\begin{definition}
	设集合$F$可表示成一列闭集$\{F_i\}$的并集:
	\begin{equation*}
		F=\underset{i=1}{\overset{+\infty}{\cup}}F_i
	\end{equation*}
	则称$F$为$F_\sigma$型集。
\end{definition}
\begin{theorem}
	设$E$是任意可测集,则一定存在$G_\delta$型集$G$使$E\subset G$,且$m(G\;\backslash\; E)=0$。
\end{theorem}
\begin{proof}
	(1)先证对任意的$\varepsilon>0$,存在开集$G$,使$E\subset G$,且$m(G\;\backslash\; E)<\varepsilon$。\par
	对于测度有限的集合$E$,由测度定义,存在一列开区间$\{I_i\}$,使得$E\subset\underset{i=1}{\overset{+\infty}{\cup}}I_i$,并且有:
	\begin{equation*}
		\sum_{i=1}^\infty|I_i|<m(E)+\varepsilon
	\end{equation*}
	令$G=\underset{i=1}{\overset{+\infty}{\cup}}I_i$,则$G$是开集,并且$E\subset G$。同时由外测度性质(2)和(3):
	\begin{equation*}
		m(E)\leqslant m(G)=\sum_{i=1}^\infty m(I_i)=\sum_{i=1}^\infty|I_i|<m(E)+\varepsilon
	\end{equation*}
	因此:
	\begin{equation*}
		m(G\;\backslash\; E)=m(G)-m(E)<\varepsilon
	\end{equation*}
	\par 若$m(E)=\infty$,则$E$一定可表示为可列个互不相交并且测度有限的可测集的并集,即$E=\underset{i=1}{\overset{+\infty}{\cup}}E_i$。对每个$E_i$应用上面的结果,可找到开集$G_i$使$E_i\subset G_i$,并且有$m(G_i)<m(E_i)+\frac{\varepsilon}{2^i}$。令$G=\underset{i=1}{\overset{+\infty}{\cup}}G_i$,则$G$是开集,$E\subset G$,并且有:
	\begin{gather*}
		G\;\backslash\; E=\underset{i=1}{\overset{+\infty}{\cup}}G_i\;\backslash\;\underset{i=1}{\overset{+\infty}{\cup}}E_i\subset \underset{i=1}{\overset{+\infty}{\cup}}(G_i\;\backslash\; E_i) \\
		m(G\;\backslash\; E)\leqslant m\left[\underset{i=1}{\overset{+\infty}{\cup}}(G_i\;\backslash\; E_i)\right]\leqslant\sum_{i=1}^\infty m(G_i\;\backslash\; E_i)<\varepsilon
	\end{gather*}
	(2)依次取$\varepsilon=\frac{1}{n},n\in\mathbb{N}^+$,由(1)可知存在开集$G_n$使得$E\subset G_n$,且$m(G_n\;\backslash\; E)<\frac{1}{n}$。令$G=\underset{i=1}{\overset{+\infty}{\cap}}G_i$,则$G$为$G_\delta$型集,$E\subset G$,且有:
	\begin{equation*}
		m(G\;\backslash\; E)\leqslant m(G_n\;\backslash\; E)<\frac{1}{n}
	\end{equation*}
	对任意的$n\in\mathbb{N}^+$成立,即$m(G\;\backslash\; E)=0$。
\end{proof}
\begin{theorem}
	设$E$是任意可测集,则一定存在$F_\sigma$型集$F$使$F\subset E$,且$m(E\;\backslash\; F)=0$。
\end{theorem}
\begin{proof}
	因为$E$可测,所以$E^c$也可测。那么存在$G_\delta$型集$G$,使得$E^c\subset G$,且$m(G\;\backslash\;E^c)=0$。令$F=G^c$,则显然$F$是一个$F_\sigma$型集,且有$F\subset E$,同时有:
	\begin{equation*}
		m(E\;\backslash\;F)=m(E\;\backslash\;G^c)=m(E\cap G)=m(G\;\backslash\;E^c)=0\qedhere
	\end{equation*}
\end{proof}


\section{可测函数}

\subsection{可测函数的性质}
\subsubsection{限制与延拓}
\begin{theorem}
	关于可测函数的限制与延拓有如下结论:
	\begin{enumerate}
		\item 设$f(x)$是可测集$E\subset\mathbb{R}^n$上的可测函数,$E_1$是$E$的可测子集,则$f(x)$限制在$E_1$上时也是可测函数。
		\item 设$f(x)$在可测集$E_i,i=1,2,\dots,n,\;n\in\mathbb{N}^+$上都是可测函数,则$f(x)$延拓在$E=\underset{i=1}{\overset{n}{\cup}}E_i$上时也是可测函数。
	\end{enumerate}
\end{theorem}
\begin{proof}
	(1)只需注意到:
	\begin{equation*}
		E_1(f>a)=E_1\cap E(f>a)
	\end{equation*}
	(2)只需注意到:
	\begin{equation*}
		E(f>a)=\underset{i=1}{\overset{n}{\cup}}E_i(f>a)\qedhere
	\end{equation*}
\end{proof}

\begin{corollary}
	设$\{f_n(x)\}$是$E$上一列可测函数,若$F(x)=\lim\limits_nf_n(x)$几乎处处\footnote{设$\pi$是一个与集合$E$中的点$x$有关的命题,如果$\exists\;M\subset E,\;mM=0$,使得$\pi$在$E\;\backslash\;M$上成立,则称$\pi$在$E$上几乎处处成立,记为$\pi$a.e.(almost everywhere)于$E$。}存在,则$F(x)$是$E$上的可测函数。
\end{corollary}



\subsection{可测函数列与一致收敛}
\begin{theorem}[叶戈罗夫定理]
	设$m(E)<+\infty$,$\{f_n\}$是$E$上一列a.e.收敛于一个a.e.有限的函数$f$的可测函数。对任意的$\delta>0$,$\exists\;E_\delta\subset E$,使得$\{f_n\}$在$E_\delta$上一致收敛,且$m(E\;\backslash\;E_\delta)<\delta$。
\end{theorem}
即$m(E)<+\infty$时,a.e.收敛在收敛对象a.e.有限的时候基本上一致收敛。

\subsection{可测函数与连续函数的关系}
\subsubsection{连续函数都是可测函数}
\begin{theorem}
	可测集$E\subset\mathbb{R}^n$上的连续函数是可测函数。
\end{theorem}
\begin{proof}
	任取一个定义在可测集$E\subset\mathbb{R}^{n}$上的连续函数$f$,$a$是任意实数。由$f$的连续性:
	\begin{equation*}
		\forall\;x\in E(f>a),\;\exists\;U(x),\;U(x)\cap E\subset E(f>a)
	\end{equation*}
	令$G=\bigcup\limits_{x\in E(f>a)}U(x)$,则:
	\begin{equation*}
		G\cap E=\left[\bigcup_{x\in E(f>a)}U(x)\right]\cap E=\bigcup_{x\in E(f>a)}\left[U(x)\cap E\right]\subset E(f>a)
	\end{equation*}
	反之显然有:
	\begin{equation*}
		E(f>a)\subset G\cap E
	\end{equation*}
	因此$E(f>a)=G\cap E$。由于$G$是开集,$E$是可测集,所以$E(f>a)$是可测集。由$a$的任意性,$f$是$E$上的可测函数。由$f$的任意性,命题成立。
\end{proof}
\subsubsection{a.e.有限的可测函数基本上连续}
\begin{theorem}[卢津定理]
	设$f(x)$是$E$上a.e.有限的可测函数。对任意的$\delta>0$,存在闭子集$F_\delta\subset E$,使得$f(x)$在$F_\delta$上连续,并且$m(E\;\backslash\;F_\delta)<\delta$。
\end{theorem}
%\begin{proof}
%	从三个角度逐步考虑。\par
%	(1)简单函数\par
%	设简单函数$f(x)$定义在$E=\underset{i=1}{\overset{n}{\cup}}E_i$上,$f(x)=c_i,\;x\in E_i\;i=1,2,\dots,n$。对任意的$\delta>0$,由于$E_i$是可测集,从而存在闭子集$F_i\subset E_i$,且$m(E_i\;\backslash\;F_i)<\frac{\delta}{n}$。
%\end{proof}
因该定理中函数限制在闭集上连续这一条件有时应用起来不太方便,下给出卢津定理的另一种形式:
\begin{theorem}
	设$f(x)$是$E\subset R$上a.e.有限的可测函数,则对任意的$\delta>0$,存在闭集$F\subset E$及定义在整个$R$上的连续函数$g(x)$($F$和$g(x)$依赖于$\delta$),使得在$F$上$f(x)=g(x)$,并且$m(E\;\backslash\;F)<\delta$。
\end{theorem}



\section{积分论}
\begin{theorem}[Levi theorem]
	设$E\subset\mathbb{R}^{n}$为可测集,$\{f_n\}$是$E$上一列非负可测函数,对任意的$x\in E,\;f_n(x)\leqslant f_{n+1}(x)$,令$f(x)=\lim\limits_{n\to+\infty}f_n(x),\;\forall\;x\in E$,则:
	\begin{equation*}
		\lim_{n\to+\infty}\left[\int_{E}f_n(x)\dif x\right]=\int_{E}\left[\lim_{n\to+\infty}f_n(x)\right]\dif x=\int_{E}f(x)\dif x
	\end{equation*}
\end{theorem}
\begin{proof}
	显然$f(x)$在$E$上非负可测且对任意的$n\in\mathbb{N}^+,\;f_n(x)\leqslant f(x)$,所以:
	\begin{equation*}
		\forall\;n\in\mathbb{N}^+,\;\int_{E}f_n(x)\dif x\leqslant\int_{E}f(x)\dif x
	\end{equation*}
	由极限的不等式性可得:
	\begin{equation*}
		\lim_{n\to+\infty}\left[\int_{E}f_n(x)\dif x\right]\leqslant\int_{E}f(x)\dif x
	\end{equation*}
	任取$E$上一非负简单函数$\varphi(x)$满足条件:对任意的$x\in E.\;\varphi(x)\leqslant f(x)$。任取$0<c<1$,令$E_n=E(f_n\geqslant c\varphi)$,则$E_n$是$E$的可测子集,$E_n\subset E_{n+1},\;\underset{n=1}{\overset{+\infty}{\cup}}E_n=E$且:
	\begin{equation*}
		\int_{E}f_n(x)\dif x\geqslant\int_{E_n}f_n(x)\dif x\geqslant\int_{E_n}c\varphi(x)\dif x\geqslant c\int_{E_n}\varphi(x)\dif x
	\end{equation*}
	所以:
	\begin{equation*}
		\lim_{n\to+\infty}\left[\int_{E}f_n(x)\dif x\right]\geqslant c\left[\lim_{n\to+\infty}\int_{E_n}\varphi(x)\dif x\right]=c\int_{E}\varphi(x)\dif x
	\end{equation*}
	由$c$的任意性和上确界的不等式性可得:
	\begin{equation*}
		\lim_{n\to+\infty}\left[\int_{E}f_n(x)\dif x\right]\geqslant\int_{E}\varphi(x)\dif x
	\end{equation*}
	由$\varphi(x)$的任意性和上确界的不等式性可得:
	\begin{equation*}
		\lim_{n\to+\infty}\left[\int_{E}f_n(x)\dif x\right]\geqslant\int_{E}f(x)\dif x
	\end{equation*}
	综上:
	\begin{equation*}
		\lim_{n\to+\infty}\left[\int_{E}f_n(x)\dif x\right]=\int_{E}f(x)\dif x\qedhere
	\end{equation*}
\end{proof}

\subsection{一般可测函数的Lebesgue积分}

\begin{theorem}[Lebesgue积分的线性性]
	若$f(x)$和$g(x)$都是$E$上的Lebesgue可积函数,则对任意的$\alpha,\beta\in\mathbb{R}$,$\alpha f(x)+\beta g(x)$也在$E$上Lebesgue可积,且:
	\begin{equation*}
		\int_{E}\left[\alpha f(x)+\beta g(x)\right]\dif x=\alpha\int_{E}f(x)\dif x+\beta\int_{E}g(x)\dif x
	\end{equation*}
\end{theorem}
\begin{proof}
	因为$f(x),g(x)$在$E$上Lebesgue可积,所以$f^+,f^-,g^+,g^-$在$E$上Lebesgue可积。由非负可测函数Lebesgue积分的线性性质,$\alpha f^++\beta g^+,\;\alpha f^-+\beta g^-$也在$E$上Lebesgue可积。于是:
	\begin{gather*}
		0\leqslant(\alpha f+\beta g)^+=\max\{\alpha f+\beta g,0\}\leqslant\max\{\alpha f,0\}+\max\{\beta g,0\}=(\alpha f)^++(\beta g)^+ \\
		0\leqslant(\alpha f+\beta g)^-=\max\{-\alpha f-\beta g,0\}\leqslant\max\{-\alpha f,0\}+\max\{-\beta g,0\}=(\alpha f)^-+(\beta g)^-
	\end{gather*}
	所以$(\alpha f+\beta g)^+,(\alpha f+\beta g)^-$在$E$上Lebesgue可积。由:
	\begin{equation*}
		\int_{E}\left[\alpha f(x)+\beta g(x)\right]\dif x=\int_{E}\left[\alpha f(x)+\beta g(x)\right]^+\dif x+\int_{E}\left[\alpha f(x)+\beta g(x)\right]^-\dif x
	\end{equation*}
	可得$\alpha f+\beta g$在$E$上Lebesgue可积。因为:
	\begin{gather*}
		\alpha f=(\alpha f)^+-(\alpha f)^-,\;\beta g=(\beta g)^+-(\beta g)^- \\
		\alpha f+\beta g=(\alpha f+\beta g)^+-(\alpha f+\beta g)^-
	\end{gather*}
	所以:
	\begin{equation*}
		(\alpha f+\beta g)^++(\alpha f)^-+(\beta g)^-=(\alpha f+\beta g)^-+(\alpha f)^++(\beta g)^+
	\end{equation*}
	由非负可测函数Lebesgue积分的线性性质可得:
	\begin{align*}
		&\int_{E}(\alpha f+\beta g)^+(x)\dif x+\int_{E}(\alpha f)^-(x)\dif x+\int_{E}(\beta g)^-(x)\dif x \\
		=&\int_{E}(\alpha f+\beta g)^-(x)\dif x+\int_{E}(\alpha f)^+(x)\dif x+\int_{E}(\beta g)^+(x)\dif x
	\end{align*}
	移项可得:
	\begin{align*}
		&\int_{E}(\alpha f+\beta g)^+(x)\dif x-\int_{E}(\alpha f+\beta g)^-(x) \\
		=&\int_{E}(\alpha f)^+(x)\dif x-\int_{E}(\alpha f)^-(x)+\int_{E}(\beta g)^+(x)\dif x-\int_{E}(\beta g)^-(x)
	\end{align*}
	由非负可测函数Lebesgue积分的线性性质即可得:
	\begin{equation*}
		\int_{E}\left[\alpha f(x)+\beta g(x)\right]\dif x=\alpha\int_{E}f(x)\dif x+\beta\int_{E}g(x)\dif x\qedhere
	\end{equation*}
\end{proof}

\subsection{Riemann积分与Lebesgue积分}
本节就一元函数的情形讨论Riemann积分与Lebesgue积分的关系。将一元函数$f(x)$在$[a,b]$上的Riemann积分和Lebesgue积分分别记为:
\begin{equation*}
	(R)\int_{a}^{b}f(x)\dif x,\;(L)\int_{a}^{b}f(x)\dif x
\end{equation*}\par
Lebesgue积分是Riemann积分的推广,但不是Riemann反常积分的推广。\par
先对Riemann积分做一个简单回顾。\par
设$f(x)$是$[a,b]$上的一个有界函数,当$x\in[a,b]$时有$|f(x)|\leqslant M$。对于任意的$n\in\mathbb{N}^+$,作$[a,b]$的分割:
\begin{equation*}
	P^{(n)}:a=x_0<x_1<\cdots<x_n=b
\end{equation*}
$|P|$表示分割$P$的最大区间长度。函数$f(x)$在每一个子区间$[x_{k-1},x_k]$上有有穷的上确界与下确界,分别记为:
\begin{equation*}
	M_k=\sup_{[x_{k-1},x_k]}\{f(x)\},\;m_k=\inf_{[x_{k-1},x_k]}\{f(x)\},\;\omega_k=M_k-m_k
\end{equation*}
将Darboux上和和下和分别记为:
\begin{equation*}
	L(f,P)=\sum_{i=1}^{n}m_i\Delta x_i,\;	U(f,P)=\sum_{i=1}^{n}M_i\Delta x_i
\end{equation*}
将Darboux上积分和下积分分别记为:
\begin{equation*}
	\overline{I}=\sup_P\{U(f,P)\},\;
	\underline{I}=\inf_P\{L(f,P)\}
\end{equation*}
由Riemann积分的结论,记:
\begin{equation*}
	\omega(x)=\lim_{\delta\to0^+}\sup\{|f(y)-f(z)|:y,z\in(x-\delta,x+\delta)\cap[a,b]\}
\end{equation*}
\begin{theorem}
	$\omega(x)=0$的充要条件为$f(x)$在$x$处连续。
\end{theorem}
\begin{theorem}
	令$E$为所有的划分$P^{(n)},n\in\mathbb{N}^+$的全体分点构成的集合,则$E$是可测集且$m(E)=0$。
\end{theorem}
\begin{theorem}
	设$f(x)$为$[a,b]$上的有界函数,则:
	\begin{equation*}
		(L)\int_{[a,b]}^{}\omega(x)\dif x=\overline{I}-\underline{I}
	\end{equation*}
\end{theorem}
\begin{proof}
	令:
	\begin{equation*}
		h_n(x)=
		\begin{cases}
			M_k^{(n)}-m_k^{(n)},&x_{k-1}^{(n)}<x<x_{k}^{(n)} \\
			0, &x\text{为$P^{(n)}$的分点}
		\end{cases}
	\end{equation*}
	显然$h_n(x)$是一个非负简单函数,并且当$x\in[a,b]$时有$0\leqslant h_n(x)\leqslant 2M$,同时对任意的$x\in[a,b]\backslash E$,有$h_n(x)\to\omega(x)$,即$h_n(x)\to\omega(x)\;$a.e.于$[a,b]$。由有界收敛定理可得:
	\begin{equation*}
		\lim_{n\to+\infty}\left[(L)\int_{[a,b]}^{}h_n(x)\dif x\right]=(L)\int_{[a,b]}^{}\omega(x)\dif x
	\end{equation*}
	由非负简单函数Lebesgue积分的定义和Riemann积分结论:
	\begin{align*}
		\lim_{n\to+\infty}\left[(L)\int_{[a,b]}^{}h_n(x)\dif x\right]
		&=\lim_{n\to+\infty}\left[\sum_{i=1}^{n}(M_i^{(n)}-m_i^{(n)})(x_i^{(n)}-x_{i-1}^{(n)})\right] \\
		&=\lim_{n\to+\infty}\left[\sum_{i=1}^{n}M_i^{(n)}(x_i^{(n)}-x_{i-1}^{(n)})\right]-\lim_{n\to+\infty}\left[\sum_{i=1}^{n}m_i^{(n)}(x_i^{(n)}-x_{i-1}^{(n)})\right] \\
		&=\overline{I}-\underline{I}
	\end{align*}
	所以:
	\begin{equation*}
		(L)\int_{[a,b]}^{}\omega(x)\dif x=\overline{I}-\underline{I}\qedhere
	\end{equation*}
\end{proof}
\begin{theorem}
	设$f(x)$是$[a,b]$上的一个有界函数,则$f(x)$在$[a,b]$上Riemann可积的充要条件为$f(x)$连续a.e.于$[a,b]$,即$f(x)$的不连续点构成一个零测集。
\end{theorem}
\begin{proof}
	由Riemann积分结论、非负可测函数的性质(7)以及$\omega(x)=0$与函数$f(x)$连续性的关系可得:
	\begin{align*}
		f(x)\text{在}[a,b]\text{上Riemann可积}
		&\Leftrightarrow\overline{I}=\underline{I} \\
		&\Leftrightarrow(L)\int_{[a,b]}^{}\omega(x)\dif x=0 \\
		&\Leftrightarrow\omega(x)=0\;\text{a.e.于}[a,b] \\
		&\Leftrightarrow f(x)\text{连续a.e.于}[a,b]\qedhere
	\end{align*}
\end{proof}
\begin{theorem}
	设$f(x)$是$[a,b]$上的一个有界函数。若$f(x)$在$[a,b]$上Riemann可积,则$f(x)$在$[a,b]$上Lebesgue可积,且:
	\begin{equation*}
		(L)\int_{[a,b]}^{}f(x)\dif x=(R)\int_{a}^{b}f(x)\dif x
	\end{equation*}
\end{theorem}
\begin{proof}
	因为$f(x)$在$[a,b]$上Riemann可积,则$f(x)$在$[a,b]$上的不连续点构成一个零测集。
\end{proof}




%\begin{theorem}
%	设$E$为实线性空间$X$的子空间,$f$是定义在$E$上的实线性泛函,$g$是定义在$X$上的次可加正齐次泛函,$f$与$g$之间满足:
%	\begin{equation*}
%		\forall\;x\in E,\;f(x)\leqslant g(x)
%	\end{equation*}
%	则必然存在定义在$X$上的实线性泛函$F$,它是$f$在$X$上的延拓,并且当$x\in X$时,$F(x)\leqslant p(x)$。
%\end{theorem}
%\begin{lemma}
%	设$f$是复赋范线性空间$X$上的有界线性泛函,令:
%	\begin{equation*}
%		\forall\;x\in E,\;\varphi(x)=\Re f(x)
%	\end{equation*}
%	则$\varphi(x)$是$X$上的有界实线性泛函,且:
%	\begin{equation*}
%		f(x)=\varphi(x)-i\varphi(ix)
%	\end{equation*}
%\end{lemma}
%\begin{proof}
%	设$f(x)=\varphi(x)+i\psi(x)$。显然$\varphi(x),\psi(x)$都是$X$上的实线性泛函。由:
%	\begin{equation*}
%		i[\varphi(x)+i\psi(x)]=if(x)=f(ix)=\varphi(ix)+i\psi(ix)
%	\end{equation*}
%\end{proof}
%\begin{theorem}[Hahn-Banach theorem]
%	设$E$是赋范线性空间$X$的子空间,$f$是定义在$E$上的有界线性泛函,则$f$可以延拓到整个$X$上且保持范数不变。
%\end{theorem}
\section{微分与不定积分}

\subsection{Vitali定理}
\begin{definition}
	设$E\subset\mathbb{R}$,$\mathcal{V}$是一个长度为正的区间族。若对于任意的$x\in E$和任意的$\varepsilon>0$,都存在区间$I_x\in\mathcal{V}$使得$x\in I_x$且$mI_x<\varepsilon$,则称$\mathcal{V}$依Vitali意义覆盖$E$,简称$\mathcal{V}$为$E$的V-覆盖。
\end{definition}
\begin{theorem}
	设$E\subset\mathbb{R}$且$m^*(E)<+\infty$,$\mathcal6{V}$为$E$的V-覆盖,则可选出区间列$\{I_n\}\subset\mathcal{V}$,使得$I_n$之间互不相交,同时有:
	\begin{equation*}
		m\left(E\;\backslash\;\underset{n\in\mathbb{N}^+}{\cup}I_n\right)=0
	\end{equation*}
\end{theorem}

\subsection{单调函数的可微性}
\begin{definition}
	设$f(x)$为$[a,b]$上的有界函数,$x_0\in[a,b]$。如果存在数列$h_n\to0(h_n\ne0)$使得极限:
	\begin{equation*}
		\lim_{n\to+\infty}\frac{f(x_0+h_n)-f(x_0)}{h_n}=\lambda
	\end{equation*}
	存在($\lambda$可为$\pm\infty$),则称$\lambda$为$f(x)$在点$x_0$处的一个列导数,记为$Df(x_0)=\lambda$。
\end{definition}
\begin{theorem}
	函数$f(x)$在点$x_0$处存在导数$f'(x_0)$的充要条件为$f(x)$在点$x_0$处的一切列导数都相等。
\end{theorem}
\begin{lemma}
	设$f(x)$为$[a,b]$上的严格增函数,
	\begin{enumerate}
		\item 如果对于$E\subset[a,b]$中的每一个点$x$,都至少有一个列导数$Df(x)\leqslant p(p\geqslant0)$,则$m^*[f(E)]\leqslant pm^*(E)$;
		\item 如果对于$E\subset[a,b]$中的每一个点$x$,都至少有一个列导数$Df(x)\geqslant q(q\geqslant0)$,则$m^*[f(E)]\geqslant qm^*(E)$。
	\end{enumerate}
\end{lemma}
\begin{theorem}[Lebesgue theorem]\label{theo:Lebesgue theorem}
	设$f(x)$为$[a,b]$上的单调函数,则:
	\begin{enumerate}
		\item $f(x)$存在有限导数$f'(x)$a.e.于$[a,b]$;
		\item $f'(x)$在$[a,b]$上可积;
		\item 如果$f'(x)$为增函数,则有:
		\begin{equation*}
			\int_{a}^{b}f'(x)\dif x\leqslant f(b)-f(a)
		\end{equation*}
	\end{enumerate}
\end{theorem}
\begin{proof}
	设$f(x)$为增函数,减函数同理。\par
	令:
	\begin{equation*}
		E=\{x:f'(x)\text{不存在}\}
	\end{equation*}
	于是对任意的$x_0\in E$,总有两个列导数\info{导数不存在的点一定至少存在两个列导数吗?}
\end{proof}

\subsection{有界变差函数}
\begin{definition}
	设$f(x)$为$[a,b]$上的有界函数。如果对于$[a,b]$上的一切划分$P$,都有:
	\begin{equation*}
		\left\{\sum_{i=1}^{n}|f(x_i)-f(x_{i-1})|\right\}
	\end{equation*}
	为一个有界数集(其中$x_i,i=1,2,\dots,n$为划分的分点),则称$f(x)$为$[a,b]$上的有界变差函数,并称:
	\begin{equation*}
		\sup_P\left\{\sum_{i=1}^{n}|f(x_i)-f(x_{i-1})|\right\}
	\end{equation*}
	为$f(x)$在$[a,b]$上的全变差,记为$\bigvee_a^b(f)$。用一个划分作成的和数:
	\begin{equation*}
		\bigvee=\sum_{i=1}^{n}|f(x_i)-f(x_{i-1})|
	\end{equation*}
	称为$f(x)$在此划分下对应的变差。
\end{definition}
\subsubsection{有界变差关于区间的可加性}
\begin{theorem}
	设$f(x)$在$[a,b]$上有界变差,则也在其任意子区间$[a_1,b_1]$上有界变差。
\end{theorem}
\begin{proof}
	对$[a_1,b_1]$取任意一个划分:
	\begin{equation*}
		P_1:a_1=x_0<x_1<x_2<\cdots<x_n=b_1
	\end{equation*}
	其对应的变差为$\bigvee_{}^{}$。此时取$[a,b]$的一个划分;
	\begin{equation*}
		P_2:a=y_0<y_1=x_0<\cdots<y_{n+1}=x_n<y_{n+2}=b
	\end{equation*}
	其对应的变差为$\bigvee_1$,则显然有:
	\begin{equation*}
		\bigvee_{}^{}\leqslant\bigvee_1\leqslant\bigvee_{a}^{b}(f)
	\end{equation*}
	由上确界的不等式性即可得:
	\begin{equation*}
		\bigvee_{a_1}^{b_1}(f)\leqslant\bigvee_{a}^{b}(f)
	\end{equation*}
	即$f(x)$在$[a_1,b_1]$上有界变差。
\end{proof}
\begin{lemma}
	设$f(x)$为$[a,b]$上的函数。对于$[a,b]$上的任一划分,若增加分点,则变差不减;若减少分点,则变差不增。
\end{lemma}
\begin{proof}
	对$[a,b]$取任意一个划分:
	\begin{equation*}
		P_1:a=x_0<x_1<x_2<\cdots<x_n=b
	\end{equation*}
	其对应的变差为:
	\begin{equation*}
		\bigvee_1=\sum_{i=1}^{n}|f(x_i)-f(x_{i-1})|
	\end{equation*}
	若此时增加一个分点$x_m=c$,则划分变为:
	\begin{equation*}
		P_2:a=x_0<x_1<x_2<\cdots<x_m<c<x_{m+1}\cdots<x_n=b
	\end{equation*}
	其对应的变差为:
	\begin{equation*}
		\bigvee_2=\sum_{i=1}^{m}|f(x_i)-f(x_{i-1})|+\sum_{i=m+1}^{n}|f(x_i)-f(x_{i-1})|+|f(c)-f(x_m)|+|f(x_{m+1})-f(c)|
	\end{equation*}
	两个变差的差为:
	\begin{equation*}
		\bigvee_2-\bigvee_1=|f(x_{m+1})-f(c)|+|f(c)-f(x_m)|-|f(x_{m+1})-f(x_m)|
	\end{equation*}
	由绝对值的三角不等式,显然有$\bigvee_2>\bigvee_1$。\par
	增加多个分点的情况可直接由增加单个分点的结论推得,减少分点的情况可直接由增加分点的结论推得。
\end{proof}
\begin{theorem}\label{theo:有界变差关于区间的可加性}
	设$a<c<b$,$f(x)$分别在$[a,c]$和$[c,b]$上有界变差,则$f(x)$在$[a,b]$上也有界变差,同时有:
	\begin{equation*}
		\bigvee_{a}^{b}(f)=\bigvee_{a}^{c}(f)+\bigvee_{c}^{b}(f)
	\end{equation*}
\end{theorem}
\begin{proof}
	对$[a,b]$取任意一个划分:
	\begin{equation*}
		P:a=x_0<x_1<x_2<\cdots<x_n=b
	\end{equation*}
	其对应的变差记为$\bigvee$。对其再插入一个分点$c$,记$[a,c]$上的变差为$\bigvee_1$,$[c,b]$上的变差为$\bigvee_2$,则:
	\begin{equation*}
		\bigvee\leqslant\bigvee_1+\bigvee_2
	\end{equation*}
	由上确界的不等式性,依次对$\bigvee_1,\bigvee_2,\bigvee$取关于划分的上确界可得:
	\begin{equation*}
		\bigvee_{a}^{b}(f)\leqslant\bigvee_{a}^{c}(f)+\bigvee_{c}^{b}(f)
	\end{equation*}
	因为$f(x)$分别在$[a,c]$和$[c,b]$上有界变差,所以$\bigvee_{a}^{c}(f)+\bigvee_{c}^{b}(f)<+\infty$,即$f(x)$在$[a,b]$上也有界变差。\par
	对$[a,c]$和$[c,b]$分别任取两个划分:
	\begin{equation*}
		P_1:a=y_0<y_1<y_2<\cdots<y_m=c,\;
		P_2:c=z_0<z_1<z_2<\cdots<x_n=b
	\end{equation*}
	相应的变差分别为:
	\begin{equation*}
		\bigvee_1=\sum_{i=1}^{m}|f(y_i)-f(y_{i-1})|,\;\bigvee_2=\sum_{i=1}^{n}|f(z_i)-f(z_{i-1})|
	\end{equation*}
	将上述两部分合并起来,则得到了一个$[a,b]$上的划分,所以:
	\begin{equation*}
		\bigvee_1+\bigvee_2\leqslant\bigvee_{a}^{b}(f)
	\end{equation*}
	由上确界的不等式性,依次对$\bigvee_1,\bigvee_2$取关于划分的上确界可得:
	\begin{equation*}
		\bigvee_{a}^{c}(f)+\bigvee_{c}^{b}(f)\leqslant\bigvee_{a}^{b}(f)
	\end{equation*}
	所以:
	\begin{equation*}
		\bigvee_{a}^{c}(f)+\bigvee_{c}^{b}(f)=\bigvee_{a}^{b}(f)\qedhere
	\end{equation*}
\end{proof}
\subsubsection{有界变差与有界的关系}
\begin{theorem}
	设$f(x)$在$[a,b]$上有界变差,则$f(x)$在$[a,b]$上有界。
\end{theorem}
\begin{proof}
	对于任意的$x$满足$a\leqslant x\leqslant b$,有:
	\begin{equation*}
		\bigvee=|f(x)-f(a)|+|f(b)-f(x)|\leqslant\bigvee_{a}^{b}(f)
	\end{equation*}
	于是:
	\begin{equation*}
		|f(x)|-|f(a)|\leqslant|f(x)-f(a)|\leqslant\bigvee_{a}^{b}(f)
	\end{equation*}
	即:
	\begin{equation*}
		|f(x)|\leqslant|f(a)|+\bigvee_{a}^{b}(f)
	\end{equation*}
	因为$f(x)$在$[a,b]$上有界变差,所以$\bigvee_{a}^{b}(f)<+\infty$。若$f(x)$在点$a$处无界,则$f(x)$在$[a,b]$上的所有变差中的第一项$|f(x_1)-f(a)|$都无界,$f(x)$不可能在$[a,b]$上有界变差,所以$|f(a)|<+\infty$。综上,$|f(x)|<+\infty$。由$x$的任意性,$f(x)$在$[a,b]$上有界。
\end{proof}
\subsubsection{有界变差函数的运算}
\begin{theorem}
	设$f(x),g(x)$在$[a,b]$上都有界变差,$\alpha,\beta\in\mathbb{R}$,则$\alpha f(x)+\beta g(x),\;f(x)g(x),\;|f(x)|$也在$[a,b]$上有界变差。
\end{theorem}
\begin{proof}
	(1)$\alpha f(x)+\beta g(x)$:\par
	令$s(x)=\alpha f(x)+\beta g(x)$,由绝对值的三角不等式:
	\begin{equation*}
		|s(x_{m+1})-s(x_m)|\leqslant|\alpha|\;|f(x_{m+1})-f(x_m)|+|\beta|\;|g(x_{m+1})-g(x_m)|
	\end{equation*}
	所以:
	\begin{equation*}
		\bigvee_{a}^{b}(s)\leqslant|\alpha|\bigvee_{a}^{b}(f)+|\beta|\bigvee_{a}^{b}(g)
	\end{equation*}
	因为$f(x),g(x)$在$[a,b]$上都有界变差,所以$\bigvee_{a}^{b}(s)<+\infty$,即$\alpha f(x)+\beta g(x)$在$[a,b]$上有界变差。\par
	(2)令$p(x)=f(x)g(x)$,设$A=\sup|f(x)|,\;B=\sup|g(x)|$。因为有界变差函数都有界,所以$A,B<+\infty$,于是:
	\begin{align*}
		|p(x_m+1)-p(x_m)|
		&=|f(x_{m+1})g(x_{m+1})-f(x_m)g(x_m)| \\
		&=|f(x_{m+1})g(x_{m+1})-f(x_m)g(x_{m+1})+f(x_m)g(x_{m+1})-f(x_m)g(x_m)| \\
		&\leqslant|f(x_{m+1})g(x_{m+1})-f(x_m)g(x_{m+1})|+|f(x_m)g(x_{m+1})-f(x_m)g(x_m)| \\
		&\leqslant B|f(x_{m+1})-f(x_m)|+A|g(x_{m+1})-g(x_m)|
	\end{align*}
	所以:
	\begin{equation*}
		\bigvee_{a}^{b}(p)\leqslant B\bigvee_{a}^{b}(f)+A\bigvee_{a}^{b}(g)<+\infty
	\end{equation*}
	即$f(x)g(x)$在$[a,b]$上有界变差。\par
	(3)因为:
	\begin{equation*}
		\Big||f(x_{m+1})|-|f(x_m)|\Big|\leqslant|f(x_{m+1}-f(x_m))|
	\end{equation*}
	所以:
	\begin{equation*}
		\bigvee_{a}^{b}(|f|)\leqslant\bigvee_{a}^{b}(f)<+\infty
	\end{equation*}
	于是$|f(x)|$在$[a,b]$上有界变差。
\end{proof}
\begin{theorem}[Jordan分解定理]
	$[a,b]$上的任一有界变差函数$f(x)$都可表示为两个增函数的差。
\end{theorem}
\begin{proof}
	由\cref{theo:有界变差关于区间的可加性},函数:
	\begin{equation*}
		g(x)=\bigvee_{a}^{x},\;x\in[a,b]
	\end{equation*}
	是$[a,b]$上的增函数。令:
	\begin{equation*}
		h(x)=g(x)-f(x)
	\end{equation*}
	对于任意的$x_1,x_2$满足$a\leqslant x_1<x_2\leqslant b$,有:
	\begin{align*}
		h(x_2)-h(x_1)
		&=g(x_2)-g(x_1)-[f(x_2)-f(x_1)] \\
		&=\bigvee_{x_1}^{x_2}(f)-[f(x_2)-f(x_1)] \\
		&\geqslant|f(x_2)-f(x_1)|-[f(x_2)-f(x_1)]\geqslant0
	\end{align*}
	所以$h(x)$为单调增函数。综上,$f(x)$可表示为$g(x)-h(x)$,其中$g(x),h(x)$都是增函数。
\end{proof}
\begin{corollary}
	有界变差函数至多有可数个不连续点。
\end{corollary}
\begin{proof}
	单调函数至多有可数个不连续点。\info{补充证明}
\end{proof}
\begin{corollary}
	设$f(x)$为$[a,b]$上的有界变差函数,则:
	\begin{enumerate}
		\item $f(x)$存在导数$f'(x)$a.e.于$E$。
		\item $f'(x)$在$[a,b]$上可积。
	\end{enumerate}
\end{corollary}
\begin{proof}
	由\cref{theo:Lebesgue theorem}、极限的可加运算和积分的可加运算可立即得到。
\end{proof}

\subsection{不定积分}
\begin{definition}
	设$f(x)$在$[a,b]$上Lebesgue可积,则$[a,b]$上的函数:
	\begin{equation*}
		F(x)=\int_{a}^{x}f(t)\dif t+C
	\end{equation*}
	称为$f(x)$的一个不定积分,其中$C$为任意常数。
\end{definition}
\begin{definition}
	设$F(x)$是$[a,b]$上的有界函数。若对于任意的$\varepsilon>0$,存在$\delta>0$,使得对$[a,b]$中互不相交的任意有限个开区间$(a_i,b_i),\;i=1,2,\dots,n$,只要$\sum\limits_{i=1}^{n}(b_i-a_i)<\delta$,就有:
	\begin{equation*}
		\sum_{i=1}^{n}|F(b_i)-F(a_i)|<\varepsilon
	\end{equation*}
	则称$F(x)$为$[a,b]$上的绝对连续函数。
\end{definition}
\subsubsection{绝对连续函数的性质}
\begin{property}
	设$f(x),g(x)$是$[a,b]$上的绝对连续函数,$\alpha,\beta\in\mathbb{R}$,则:
	\begin{enumerate}
		\item $f(x)$是一致连续的;
		\item $f(x)$是有界变差的。
		\item $\alpha f(x)+\beta g(x),f(x)g(x),\dfrac{1}{f(x)}$(除法中$f(x)\ne0$)也是绝对连续函数;
	\end{enumerate}
\end{property}
\begin{proof}
	(1)由绝对连续函数的定义,对任意的$\varepsilon>0$,当$x,y\in[a,b]$且$|x-y|<\delta$时,就有$|f(x)-f(y)|<\varepsilon$,即$f(x)$是一致连续的。\par
	(2)\info{需要再思考如何证明}\par
	(3)$(a_i,b_i),\;i=1,2,\dots,n,\;n\in\mathbb{N}^+$是$[a,b]$上互不相交的有限个开区间。\par
	$\alpha f(x)+\beta g(x)$:\par
	由绝对值的三角不等式可得:
	\begin{equation*}
		\sum_{i=1}^{n}|\alpha f(b_i)+\beta g(b_i)-\alpha f(a_i)-\beta g(a_i)|
		\leqslant|\alpha|\sum_{i=1}^{n}|f(b_i)-f(a_i)|+|\beta|\sum_{i=1}^{n}|g(b_i)-g(a_i)|
	\end{equation*}\par
	$f(x)g(x)$:\par
	因为$f(x),g(x)$是绝对连续函数,由(1)可得它们都是一致连续的,从而在$[a,b]$上连续。因为连续函数在闭区间上有界,设$|f(x)|\leqslant M_1,|g(x)|\leqslant M_2,\;x\in[a,b]$。所以:
	\begin{align*}
		\sum_{i=1}^{n}|f(b_i)g(b_i)-f(a_i)g(a_i)|
		&=\sum_{i=1}^{n}|f(b_i)g(b_i)-f(a_i)g(b_i)+f(a_i)g(b_i)-f(a_i)g(a_i)| \\
		&\leqslant\sum_{i=1}^{n}|f(b_i)g(b_i)-f(a_i)g(b_i)|+\sum_{i=1}^{n}|f(a_i)g(b_i)-f(a_i)g(a_i)| \\
		&\leqslant M_2\sum_{i=1}^{n}|f(b_i)-f(a_i)|+M_1\sum_{i=1}^{n}|g(b_i)-g(a_i)|
	\end{align*}\par
	$\dfrac{1}{f(x)}$:\par
	因为存在$M>0$使得$|f(x)|\leqslant M,\;x\in[a,b]$,所以
	\begin{align*}
		\sum_{i=1}^{n}\left|\frac{1}{f(b_i)}-\frac{1}{f(a_i)}\right|
		&=\sum_{i=1}^{n}\left|\frac{f(a_i)-f(b_i)}{f(a_i)f(b_i)}\right|
	\end{align*}
	由因为$f(x)$是绝对连续函数,所以对任意的$\varepsilon>0$,存在$\delta_i$,当$b_i-a_i<\delta_i$时,有:
	\begin{equation*}
		|f(a_i)-f(b_i)|<\frac{\varepsilon |f(a_i)f(b_i)|}{n}
	\end{equation*}
	于是对任意的$\varepsilon>0$,只要取$\delta<\min\{\delta_1,\delta_2,\dots,\delta_n\}$,就有:
	\begin{equation*}
		\sum_{i=1}^{n}\left|\frac{1}{f(b_i)}-\frac{1}{f(a_i)}\right|<\sum_{i=1}^{n}\frac{\varepsilon}{n}=\varepsilon\qedhere
	\end{equation*}
\end{proof}
\begin{theorem}
	设$f(x)$在$[a,b]$上Lebesgue可积,则其不定积分$F(x)$为绝对连续函数,$F'(x)$存在a.e.于$[a,b]$且$F'(x)=f(x)\;$a.e.于$[a,b]$。\info{这部分还没证明}
\end{theorem}
\begin{proof}
	任取$[a,b]$上互不相交的有限个开区间$(a_i,b_i),\;i=1,2,\dots,n,\;n\in\mathbb{N}^+$。因为开区间都是可测集,所以$(a_i,b_i)$可测,$\underset{i=1}{\overset{n}{\cup}}(a_i,b_i)$可测。
	由Lebesgue积分的线性性质,
	\begin{align*}
		\sum_{i=1}^{n}|F(b_i)-F(a_i)|
		&=\sum_{i=1}^{n}\left|\int_{(a_i,b_i)}f(x)\dif x\right| \\
		&\leqslant\sum_{i=1}^{n}\int_{(a_i,b_i)}|f(x)|\dif x \\
		&=\int_{\underset{i=1}{\overset{n}{\cup}}(a_i,b_i)}^{}|f(x)|\dif x
	\end{align*}
	由Lebesgue积分的绝对连续性,对任意的$\varepsilon>0$,存在$\delta>0$,当$m\left[\underset{i=1}{\overset{n}{\cup}}(a_i,b_i)\right]=\sum\limits_{i=1}^{n}(b_i-a_i)<\delta$时,就有:
	\begin{equation*}
		\int_{\underset{i=1}{\overset{n}{\cup}}(a_i,b_i)}^{}|f(x)|\dif x<\delta
	\end{equation*}
	即:
	\begin{equation*}
		\sum_{i=1}^{n}|F(b_i)-F(a_i)|<\delta
	\end{equation*}
	所以$F(x)$是绝对连续函数。
\end{proof}
\begin{theorem}
	设$F(x)$为$[a,b]$上的绝对连续函数,且$F'(x)=0\;$a.e.于$[a,b]$,则$F(x)$为一常数。
\end{theorem}
\begin{theorem}
	设$f(x)$在$[a,b]$上Lebesgue可积,则存在绝对连续函数$F(x)$使得$F'(x)=f(x)\;$a.e.于$[a,b]$。
\end{theorem}
\begin{theorem}
	设$F(x)$时$[a,b]$上的绝对连续函数,则$F'(x)$存在a.e.于$[a,b]$且$F'(x)$在$[a,b]$上可积,同时有:
	\begin{equation*}
		F(x)=F(a)+\int_{[a,x]}^{}F'(t)\dif t
	\end{equation*}
\end{theorem}
\begin{corollary}
	$F(x)$是$[a,b]$上的绝对连续函数的充要条件为$F(x)$是一个Lebesgue可积函数的不定积分。
\end{corollary}