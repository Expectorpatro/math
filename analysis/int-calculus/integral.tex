\section{积分论}
\begin{theorem}[Levi theorem]
	设$E\subset\mathbb{R}^{n}$为可测集,$\{f_n\}$是$E$上一列非负可测函数,对任意的$x\in E,\;f_n(x)\leqslant f_{n+1}(x)$,令$f(x)=\lim\limits_{n\to+\infty}f_n(x),\;\forall\;x\in E$,则:
	\begin{equation*}
		\lim_{n\to+\infty}\left[\int_{E}f_n(x)\dif x\right]=\int_{E}\left[\lim_{n\to+\infty}f_n(x)\right]\dif x=\int_{E}f(x)\dif x
	\end{equation*}
\end{theorem}
\begin{proof}
	显然$f(x)$在$E$上非负可测且对任意的$n\in\mathbb{N}^+,\;f_n(x)\leqslant f(x)$,所以:
	\begin{equation*}
		\forall\;n\in\mathbb{N}^+,\;\int_{E}f_n(x)\dif x\leqslant\int_{E}f(x)\dif x
	\end{equation*}
	由极限的不等式性可得:
	\begin{equation*}
		\lim_{n\to+\infty}\left[\int_{E}f_n(x)\dif x\right]\leqslant\int_{E}f(x)\dif x
	\end{equation*}
	任取$E$上一非负简单函数$\varphi(x)$满足条件:对任意的$x\in E.\;\varphi(x)\leqslant f(x)$。任取$0<c<1$,令$E_n=E(f_n\geqslant c\varphi)$,则$E_n$是$E$的可测子集,$E_n\subset E_{n+1},\;\underset{n=1}{\overset{+\infty}{\cup}}E_n=E$且:
	\begin{equation*}
		\int_{E}f_n(x)\dif x\geqslant\int_{E_n}f_n(x)\dif x\geqslant\int_{E_n}c\varphi(x)\dif x\geqslant c\int_{E_n}\varphi(x)\dif x
	\end{equation*}
	所以:
	\begin{equation*}
		\lim_{n\to+\infty}\left[\int_{E}f_n(x)\dif x\right]\geqslant c\left[\lim_{n\to+\infty}\int_{E_n}\varphi(x)\dif x\right]=c\int_{E}\varphi(x)\dif x
	\end{equation*}
	由$c$的任意性和上确界的不等式性可得:
	\begin{equation*}
		\lim_{n\to+\infty}\left[\int_{E}f_n(x)\dif x\right]\geqslant\int_{E}\varphi(x)\dif x
	\end{equation*}
	由$\varphi(x)$的任意性和上确界的不等式性可得:
	\begin{equation*}
		\lim_{n\to+\infty}\left[\int_{E}f_n(x)\dif x\right]\geqslant\int_{E}f(x)\dif x
	\end{equation*}
	综上:
	\begin{equation*}
		\lim_{n\to+\infty}\left[\int_{E}f_n(x)\dif x\right]=\int_{E}f(x)\dif x\qedhere
	\end{equation*}
\end{proof}

\subsection{一般可测函数的Lebesgue积分}

\begin{theorem}[Lebesgue积分的线性性]
	若$f(x)$和$g(x)$都是$E$上的Lebesgue可积函数,则对任意的$\alpha,\beta\in\mathbb{R}$,$\alpha f(x)+\beta g(x)$也在$E$上Lebesgue可积,且:
	\begin{equation*}
		\int_{E}\left[\alpha f(x)+\beta g(x)\right]\dif x=\alpha\int_{E}f(x)\dif x+\beta\int_{E}g(x)\dif x
	\end{equation*}
\end{theorem}
\begin{proof}
	因为$f(x),g(x)$在$E$上Lebesgue可积,所以$f^+,f^-,g^+,g^-$在$E$上Lebesgue可积。由非负可测函数Lebesgue积分的线性性质,$\alpha f^++\beta g^+,\;\alpha f^-+\beta g^-$也在$E$上Lebesgue可积。于是:
	\begin{gather*}
		0\leqslant(\alpha f+\beta g)^+=\max\{\alpha f+\beta g,0\}\leqslant\max\{\alpha f,0\}+\max\{\beta g,0\}=(\alpha f)^++(\beta g)^+ \\
		0\leqslant(\alpha f+\beta g)^-=\max\{-\alpha f-\beta g,0\}\leqslant\max\{-\alpha f,0\}+\max\{-\beta g,0\}=(\alpha f)^-+(\beta g)^-
	\end{gather*}
	所以$(\alpha f+\beta g)^+,(\alpha f+\beta g)^-$在$E$上Lebesgue可积。由:
	\begin{equation*}
		\int_{E}\left[\alpha f(x)+\beta g(x)\right]\dif x=\int_{E}\left[\alpha f(x)+\beta g(x)\right]^+\dif x+\int_{E}\left[\alpha f(x)+\beta g(x)\right]^-\dif x
	\end{equation*}
	可得$\alpha f+\beta g$在$E$上Lebesgue可积。因为:
	\begin{gather*}
		\alpha f=(\alpha f)^+-(\alpha f)^-,\;\beta g=(\beta g)^+-(\beta g)^- \\
		\alpha f+\beta g=(\alpha f+\beta g)^+-(\alpha f+\beta g)^-
	\end{gather*}
	所以:
	\begin{equation*}
		(\alpha f+\beta g)^++(\alpha f)^-+(\beta g)^-=(\alpha f+\beta g)^-+(\alpha f)^++(\beta g)^+
	\end{equation*}
	由非负可测函数Lebesgue积分的线性性质可得:
	\begin{align*}
		&\int_{E}(\alpha f+\beta g)^+(x)\dif x+\int_{E}(\alpha f)^-(x)\dif x+\int_{E}(\beta g)^-(x)\dif x \\
		=&\int_{E}(\alpha f+\beta g)^-(x)\dif x+\int_{E}(\alpha f)^+(x)\dif x+\int_{E}(\beta g)^+(x)\dif x
	\end{align*}
	移项可得:
	\begin{align*}
		&\int_{E}(\alpha f+\beta g)^+(x)\dif x-\int_{E}(\alpha f+\beta g)^-(x) \\
		=&\int_{E}(\alpha f)^+(x)\dif x-\int_{E}(\alpha f)^-(x)+\int_{E}(\beta g)^+(x)\dif x-\int_{E}(\beta g)^-(x)
	\end{align*}
	由非负可测函数Lebesgue积分的线性性质即可得:
	\begin{equation*}
		\int_{E}\left[\alpha f(x)+\beta g(x)\right]\dif x=\alpha\int_{E}f(x)\dif x+\beta\int_{E}g(x)\dif x\qedhere
	\end{equation*}
\end{proof}

\subsection{Riemann积分与Lebesgue积分}
本节就一元函数的情形讨论Riemann积分与Lebesgue积分的关系。将一元函数$f(x)$在$[a,b]$上的Riemann积分和Lebesgue积分分别记为:
\begin{equation*}
	(R)\int_{a}^{b}f(x)\dif x,\;(L)\int_{a}^{b}f(x)\dif x
\end{equation*}\par
Lebesgue积分是Riemann积分的推广,但不是Riemann反常积分的推广。\par
先对Riemann积分做一个简单回顾。\par
设$f(x)$是$[a,b]$上的一个有界函数,当$x\in[a,b]$时有$|f(x)|\leqslant M$。对于任意的$n\in\mathbb{N}^+$,作$[a,b]$的分割:
\begin{equation*}
	P^{(n)}:a=x_0<x_1<\cdots<x_n=b
\end{equation*}
$|P|$表示分割$P$的最大区间长度。函数$f(x)$在每一个子区间$[x_{k-1},x_k]$上有有穷的上确界与下确界,分别记为:
\begin{equation*}
	M_k=\sup_{[x_{k-1},x_k]}\{f(x)\},\;m_k=\inf_{[x_{k-1},x_k]}\{f(x)\},\;\omega_k=M_k-m_k
\end{equation*}
将Darboux上和和下和分别记为:
\begin{equation*}
	L(f,P)=\sum_{i=1}^{n}m_i\Delta x_i,\;	U(f,P)=\sum_{i=1}^{n}M_i\Delta x_i
\end{equation*}
将Darboux上积分和下积分分别记为:
\begin{equation*}
	\overline{I}=\sup_P\{U(f,P)\},\;
	\underline{I}=\inf_P\{L(f,P)\}
\end{equation*}
由Riemann积分的结论,记:
\begin{equation*}
	\omega(x)=\lim_{\delta\to0^+}\sup\{|f(y)-f(z)|:y,z\in(x-\delta,x+\delta)\cap[a,b]\}
\end{equation*}
\begin{theorem}
	$\omega(x)=0$的充要条件为$f(x)$在$x$处连续。
\end{theorem}
\begin{theorem}
	令$E$为所有的划分$P^{(n)},n\in\mathbb{N}^+$的全体分点构成的集合,则$E$是可测集且$m(E)=0$。
\end{theorem}
\begin{theorem}
	设$f(x)$为$[a,b]$上的有界函数,则:
	\begin{equation*}
		(L)\int_{[a,b]}^{}\omega(x)\dif x=\overline{I}-\underline{I}
	\end{equation*}
\end{theorem}
\begin{proof}
	令:
	\begin{equation*}
		h_n(x)=
		\begin{cases}
			M_k^{(n)}-m_k^{(n)},&x_{k-1}^{(n)}<x<x_{k}^{(n)} \\
			0, &x\text{为$P^{(n)}$的分点}
		\end{cases}
	\end{equation*}
	显然$h_n(x)$是一个非负简单函数,并且当$x\in[a,b]$时有$0\leqslant h_n(x)\leqslant 2M$,同时对任意的$x\in[a,b]\backslash E$,有$h_n(x)\to\omega(x)$,即$h_n(x)\to\omega(x)\;$a.e.于$[a,b]$。由有界收敛定理可得:
	\begin{equation*}
		\lim_{n\to+\infty}\left[(L)\int_{[a,b]}^{}h_n(x)\dif x\right]=(L)\int_{[a,b]}^{}\omega(x)\dif x
	\end{equation*}
	由非负简单函数Lebesgue积分的定义和Riemann积分结论:
	\begin{align*}
		\lim_{n\to+\infty}\left[(L)\int_{[a,b]}^{}h_n(x)\dif x\right]
		&=\lim_{n\to+\infty}\left[\sum_{i=1}^{n}(M_i^{(n)}-m_i^{(n)})(x_i^{(n)}-x_{i-1}^{(n)})\right] \\
		&=\lim_{n\to+\infty}\left[\sum_{i=1}^{n}M_i^{(n)}(x_i^{(n)}-x_{i-1}^{(n)})\right]-\lim_{n\to+\infty}\left[\sum_{i=1}^{n}m_i^{(n)}(x_i^{(n)}-x_{i-1}^{(n)})\right] \\
		&=\overline{I}-\underline{I}
	\end{align*}
	所以:
	\begin{equation*}
		(L)\int_{[a,b]}^{}\omega(x)\dif x=\overline{I}-\underline{I}\qedhere
	\end{equation*}
\end{proof}
\begin{theorem}
	设$f(x)$是$[a,b]$上的一个有界函数,则$f(x)$在$[a,b]$上Riemann可积的充要条件为$f(x)$连续a.e.于$[a,b]$,即$f(x)$的不连续点构成一个零测集。
\end{theorem}
\begin{proof}
	由Riemann积分结论、非负可测函数的性质(7)以及$\omega(x)=0$与函数$f(x)$连续性的关系可得:
	\begin{align*}
		f(x)\text{在}[a,b]\text{上Riemann可积}
		&\Leftrightarrow\overline{I}=\underline{I} \\
		&\Leftrightarrow(L)\int_{[a,b]}^{}\omega(x)\dif x=0 \\
		&\Leftrightarrow\omega(x)=0\;\text{a.e.于}[a,b] \\
		&\Leftrightarrow f(x)\text{连续a.e.于}[a,b]\qedhere
	\end{align*}
\end{proof}
\begin{theorem}
	设$f(x)$是$[a,b]$上的一个有界函数。若$f(x)$在$[a,b]$上Riemann可积,则$f(x)$在$[a,b]$上Lebesgue可积,且:
	\begin{equation*}
		(L)\int_{[a,b]}^{}f(x)\dif x=(R)\int_{a}^{b}f(x)\dif x
	\end{equation*}
\end{theorem}
\begin{proof}
	因为$f(x)$在$[a,b]$上Riemann可积,则$f(x)$在$[a,b]$上的不连续点构成一个零测集。
\end{proof}




%\begin{theorem}
%	设$E$为实线性空间$X$的子空间,$f$是定义在$E$上的实线性泛函,$g$是定义在$X$上的次可加正齐次泛函,$f$与$g$之间满足:
%	\begin{equation*}
%		\forall\;x\in E,\;f(x)\leqslant g(x)
%	\end{equation*}
%	则必然存在定义在$X$上的实线性泛函$F$,它是$f$在$X$上的延拓,并且当$x\in X$时,$F(x)\leqslant p(x)$。
%\end{theorem}
%\begin{lemma}
%	设$f$是复赋范线性空间$X$上的有界线性泛函,令:
%	\begin{equation*}
%		\forall\;x\in E,\;\varphi(x)=\Re f(x)
%	\end{equation*}
%	则$\varphi(x)$是$X$上的有界实线性泛函,且:
%	\begin{equation*}
%		f(x)=\varphi(x)-i\varphi(ix)
%	\end{equation*}
%\end{lemma}
%\begin{proof}
%	设$f(x)=\varphi(x)+i\psi(x)$。显然$\varphi(x),\psi(x)$都是$X$上的实线性泛函。由:
%	\begin{equation*}
%		i[\varphi(x)+i\psi(x)]=if(x)=f(ix)=\varphi(ix)+i\psi(ix)
%	\end{equation*}
%\end{proof}
%\begin{theorem}[Hahn-Banach theorem]
%	设$E$是赋范线性空间$X$的子空间,$f$是定义在$E$上的有界线性泛函,则$f$可以延拓到整个$X$上且保持范数不变。
%\end{theorem}