\section{可测集的定义与性质}
事情是这样的:我们就非要把外测度变成Lebesgue测度。目前为止外测度是不是Lebesgue测度呢?由外测度的性质$(1)$和区间的外测度,外测度显然满足\cref{axi:Lebesguem}的第一条和第三条,但是在$\mathbb{R}^n$上,人们确实能够证明外测度不具有可列可加性。事实上,$\mathbb{R}^n$上的确存在互不相交的一列集合$\{E_i\}$,使得:
\begin{equation}
	m(\underset{i=1}{\overset{+\infty}{\cup}}E_i)<\sum\limits_{i=1}^{+\infty} m(E_i)\notag
\end{equation}
因此外测度还不是Lebesgue测度。于是我们选择修改外测度的定义域,找到某个定义在$\mathbb{R}^n$上的集合族$\mathcal{F}$,使得外测度在$\mathcal{F}$上成为Lebesgue测度,不在$\mathcal{F}$中的$\mathbb{R}^n$的子集便成为不可测集。\par
下面给出这个集合族的定义。
\begin{definition}[Caratheodory condition]
	设$E$为$\mathbb{R}^n$中的点集,如果对$\mathbb{R}^n$中的任一点集$T$,都有:
	\begin{equation}
		m^*(T)=m^*(T\cap E)+m^*(T\cap E^c)\notag
	\end{equation}
	则$E\in\mathcal{F}$。此时称$E$是Lebesgue可测的,$E$的Lebesgue测度即为$E$的外测度\footnote{此时外测度的性质便成为Lebesgue测度的性质了。},记为$m(E)$。
\end{definition}
下面我们来探索这个定义带来的可测集的性质。
\begin{lemma}\label{lem:EmeasureAB}
	集合$E$可测的充要条件是对与$\forall\;A\subset E,\;\forall\;B\subset E^c$,总有:
	\begin{equation}
		m^*(A\cup B)=m^*(A)+m^*(B)\notag
	\end{equation}
\end{lemma}
\begin{proof}
	必要性:对任意的$A\subset E,\;\forall\;B\subset E^c$,取$T=A\cup B$,因为$E$可测,那么对于这个$T$,应有:
	\begin{equation}
		m^*(A\cup B)=m^*(T)=m^*(T\cap E)+m^*(T\cap E^c)=m^*(A)+m^*(B)\notag
	\end{equation}
	充分性:对任意的$T$,$\exists\;A\subset E,\;\exists\; B\subset E^c$,使得$T=A\cup B$,那么就有:
	\begin{equation}
		m^*(T)=m^*(A\cup B)=m^*(A)+m^*(B)=m^*(T\cap E)+m^*(T\cap E^c)\notag
	\end{equation}
	由$T$的任意性,$E$可测。
\end{proof}
\begin{theorem}
	$\varnothing$和$\mathbb{R}^n$可测。
\end{theorem}
\begin{proof}
	代入定义直接可得。
\end{proof}
\begin{theorem}
	$S$可测的充要条件是$S^c$可测。
\end{theorem}
\begin{proof}
	若$S$可测,对任意的$T\in\mathbb{R}^n$,则有:
	\begin{align*}
		m^*(T)=m^*(T\cap S)+m^*(T\cap S^c)
		&=m^*[T\cap(S^c)^c]+m^*(T\cap S^c) \\
		&=m^*(T\cap S^c)+m^*[T\cap(S^c)^c]\qedhere
	\end{align*} 
\end{proof}
\begin{theorem}
	若$S_1,S_2$都可测,则$S_1\cup S_2$也可测。当$S_1\cap S_2=\varnothing$时,对任意的$T$都有:
	\begin{equation}
		m^*[T\cap(S_1\cup S_2)]=m^*(T\cap S_1)+m^*(T\cap S_2)\notag
	\end{equation}
\end{theorem}
\begin{proof}
	因为$S_1$可测,对任意的$T$都有:
	\begin{equation}\label{eq:S_1measure}
		m^*(T)=m^*(T\cap S_1)+m^*(T\cap S_1^c)
	\end{equation}
	因为$S_2$可测,对于$m^*(T\cap S_1^c)$有:
	\begin{equation}\label{eq:TcapS_1measure}
		m^*(T\cap S_1^c)=m^*[(T\cap S_1^c)\cap S_2]+m^*[(T\cap S_1^c)\cap S_2^c]
	\end{equation}
	将\eqref{eq:TcapS_1measure}式代入\eqref{eq:S_1measure}式,再由德摩根公式,得到:
	\begin{align*}
		m^*(T)&=m^*(T\cap S_1)+m^*[(T\cap S_1^c)\cap S_2]+m^*[(T\cap S_1^c)\cap S_2^c] \\
		&=m^*(T\cap S_1)+m^*[(T\cap S_1^c)\cap S_2]+m^*[T\cap(S_1\cup S_2)^c]
	\end{align*}
	由于$T\cap S_1\subset S_1$,$(T\cap S_1^c)\cap S_2\subset S_1^c$,满足\cref{lem:EmeasureAB}条件,因此上式的前两项可以合并:
	\begin{align*}
		m^*(T\cap S_1)+m^*[(T\cap S_1^c)\cap S_2]&=m^*[(T\cap S_1)\cup(T\cap S_1^c\cap S_2)] \\
		&=m^*\{T\cap[S_1\cup(S_1^c\cap S_2)]\} \\
		&=m^*\{T\cap[(S_1\cup S_1^c)\cap(S_1\cup S_2)]\} \\
		&=m^*[T\cap(S_1\cup S_2)]
	\end{align*}
	那么就有:
	\begin{equation*}
		m^*(T)=m^*[T\cap(S_1\cup S_2)]+m^*[T\cap(S_1\cup S_2)^c]
	\end{equation*}
	由$T$的任意性,$S_1\cup S_2$可测。\par
	当$S_1\cap S_2=\varnothing$时,显然$S_2\subset S_1^c$,那么就有$T\cap S_2\subset S_1^c$,由\cref{lem:EmeasureAB}:
	\begin{align*}
		m^*[T\cap(S_1\cup S_2)]&=m^*[(T\cap S_1)\cup(T\cap S_2)]\\
		&=m^*(T\cap S_1)+m^*(T\cap S_2)\qedhere
	\end{align*}
\end{proof}
\begin{corollary}\label{cor:ncup}
	设$S_i(i=1,2,\dots,n)$都可测,则$\underset{i=1}{\overset{n}{\cup}}S_i$也可测。当$S_i\cap S_j=\varnothing$时,有:
	\begin{equation*}
		m^*[T\cap(\underset{i=1}{\overset{n}{\cup}}S_i)]=\sum_{i=1}^nm^*(T\cap S_i)
	\end{equation*}
\end{corollary}
\begin{theorem}
	若$S_1,S_2$都可测,则$S_1\cap S_2$也可测。
\end{theorem}
\begin{proof}
	$S_1\cap S_2=[(S_1\cap S_2)^c]^c=[S_1^c\cup S_2^c]^c$。
\end{proof}
\begin{corollary}
	设$S_i(i=1,2,\dots,n)$都可测,则$\underset{i=1}{\overset{n}{\cap}}S_i$也可测。
\end{corollary}
\begin{theorem}
	若$S_1,S_2$都可测,则$S_1\backslash S_2$也可测。
\end{theorem}
\begin{proof}
	$S_1\backslash S_2=S_1\cap S_2^c$。
\end{proof}
\begin{theorem}
	设$\{S_i\}$是一列互不相交的可测集,则$\underset{i=1}{\overset{+\infty}{\cup}}S_i$也可测,并且有:
	\begin{equation*}
		m(\underset{i=1}{\overset{+\infty}{\cup}}S_i)=\sum_{i=1}^{+\infty} m(S_i)
	\end{equation*}
\end{theorem}
\begin{proof}
	由\cref{cor:ncup},对任意的$n$,$\underset{i=1}{\overset{n}{\cup}}S_i$都可测,那么对任意的$T$,就有(第一行到第二行利用外测度性质(2),第二行到第三行利用\cref{cor:ncup}):
	\begin{align*}
		m^*(T)&=m^*[T\cap(\underset{i=1}{\overset{n}{\cup}}S_i)]+m^*[T\cap(\underset{i=1}{\overset{n}{\cup}}S_i)^c] \\
		&\geqslant m^*[T\cap(\underset{i=1}{\overset{n}{\cup}}S_i)]+m^*[T\cap(\underset{i=1}{\overset{+\infty}{\cup}}S_i)^c] \\
		&=\sum_{i=1}^nm^*(T\cap S_i)+m^*[T\cap(\underset{i=1}{\overset{+\infty}{\cup}}S_i)^c]
	\end{align*}
	令$n\to+\infty$,有(第一行到第二行利用极限的不等式性,第二行到第三行利用外测度的性质(3)):
	\begin{align}
		m^*(T)&\geqslant\sum_{i=1}^nm^*(T\cap S_i)+m^*[T\cap(\underset{i=1}{\overset{+\infty}{\cup}}S_i)^c]\notag \\
		&\geqslant\sum_{i=1}^{+\infty} m^*(T\cap S_i)+m^*[T\cap(\underset{i=1}{\overset{+\infty}{\cup}}S_i)^c] \label{eq:TcupSi}\\
		&\geqslant m^*[\underset{i=1}{\overset{+\infty}{\cup}}(T\cap S_i)]+m^*[T\cap(\underset{i=1}{\overset{+\infty}{\cup}}S_i)^c] \notag \\ &=m^*[T\cap(\underset{i=1}{\overset{+\infty}{\cup}}S_i)]+m^*[T\cap(\underset{i=1}{\overset{+\infty}{\cup}}S_i)^c]\notag
	\end{align}
	又因:
	\begin{equation*}
		T=[T\cap(\underset{i=1}{\overset{+\infty}{\cup}}S_i)]\cup[T\cap(\underset{i=1}{\overset{+\infty}{\cup}}S_i)^c]
	\end{equation*}
	由外测度的性质(3),有:
	\begin{equation*}
		m^*(T)\leqslant m^*[T\cap(\underset{i=1}{\overset{+\infty}{\cup}}S_i)]+m^*[T\cap(\underset{i=1}{\overset{+\infty}{\cup}}S_i)^c]
	\end{equation*}
	因此:
	\begin{equation*}
		m^*(T)= m^*[T\cap(\underset{i=1}{\overset{+\infty}{\cup}}S_i)]+m^*[T\cap(\underset{i=1}{\overset{+\infty}{\cup}}S_i)^c]
	\end{equation*}
	由$T$的任意性,$\underset{i=1}{\overset{+\infty}{\cup}}S_i$可测。\par
	令$T=\underset{i=1}{\overset{+\infty}{\cup}}S_i$,代入\cref{eq:TcupSi}式,则:
	\begin{align*}
		m^*(\underset{i=1}{\overset{+\infty}{\cup}}S_i)
		&\geqslant\sum_{i=1}^{+\infty} m^*[(\underset{i=1}{\overset{+\infty}{\cup}}S_i)\cap S_i]+m^*[(\underset{i=1}{\overset{+\infty}{\cup}}S_i)\cap(\underset{i=1}{\overset{+\infty}{\cup}}S_i)^c] \\
		&=\sum_{i=1}^{+\infty} m^*(S_i)
	\end{align*}
	但是由外测度的性质(3)有:
	\begin{equation*}
		m^*(\underset{i=1}{\overset{+\infty}{\cup}}S_i)\leqslant\sum_{i=1}^{+\infty} m^*(S_i)
	\end{equation*}
	因此:
	\begin{equation*}
		m^*(\underset{i=1}{\overset{+\infty}{\cup}}S_i)=\sum_{i=1}^{+\infty} m^*(S_i)\qedhere
	\end{equation*}
\end{proof}
\begin{corollary}
	设$\{S_i\}$是一列可测集合,则$\underset{i=1}{\overset{+\infty}{\cup}}S_i$也可测。
\end{corollary}
\begin{proof}
	$\underset{i=1}{\overset{+\infty}{\cup}}S_i$可被表示为互不相交的可数个集合的并:
	\begin{equation*}
		\underset{i=1}{\overset{+\infty}{\cup}}S_i=S_1\cup (S_2\backslash S_1)\cup[S_3\backslash(S_1\cup S_2)]\cup\cdots\qedhere
	\end{equation*}
\end{proof}
\begin{theorem}
	设$\{S_i\}$是一列可测集合,则$\underset{i=1}{\overset{+\infty}{\cap}}S_i$也可测。
\end{theorem}
\begin{proof}
	由德摩根公式:
	\begin{equation*}
		(\underset{i=1}{\overset{+\infty}{\cap}}S_i)^c=\underset{i=1}{\overset{+\infty}{\cup}}S_i^c\qedhere
	\end{equation*}
\end{proof}
做一个总结:
\begin{enumerate}
	\item $\mathcal{F}$中元素的可测性对可列并、可列交、补、差封闭。
	\item 定义在$\mathcal{F}$中的外测度满足可列可加性。
	\item $\mathcal{F}$是$\mathbb{R}^n$上的一个$\sigma$代数。
	 \footnote{定义:一个集合\(X\)的子集族\(\mathcal{M}\) 被称为一个\emph{σ-代数},当且仅当满足以下三个条件:(1)\(X\in\mathcal{F}\);(2)若\(A\in\mathcal{F}\),则\(A^c\in\mathcal{F}\);(3)若\(A_1,A_2,A_3,\dots\in\mathcal{F} \),则 \(\bigcup_{i=1}^{+\infty}A_i\in\mathcal{F}\)。}
\end{enumerate}
