\section{微分与不定积分}

\subsection{Vitali定理}
\begin{definition}
	设$E\subset\mathbb{R}$,$\mathcal{V}$是一个长度为正的区间族。若对于任意的$x\in E$和任意的$\varepsilon>0$,都存在区间$I_x\in\mathcal{V}$使得$x\in I_x$且$mI_x<\varepsilon$,则称$\mathcal{V}$依Vitali意义覆盖$E$,简称$\mathcal{V}$为$E$的V-覆盖。
\end{definition}
\begin{theorem}
	设$E\subset\mathbb{R}$且$m^*(E)<+\infty$,$\mathcal6{V}$为$E$的V-覆盖,则可选出区间列$\{I_n\}\subset\mathcal{V}$,使得$I_n$之间互不相交,同时有:
	\begin{equation*}
		m\left(E\;\backslash\;\underset{n\in\mathbb{N}^+}{\cup}I_n\right)=0
	\end{equation*}
\end{theorem}

\subsection{单调函数的可微性}
\begin{definition}
	设$f(x)$为$[a,b]$上的有界函数,$x_0\in[a,b]$。如果存在数列$h_n\to0(h_n\ne0)$使得极限:
	\begin{equation*}
		\lim_{n\to+\infty}\frac{f(x_0+h_n)-f(x_0)}{h_n}=\lambda
	\end{equation*}
	存在($\lambda$可为$\pm\infty$),则称$\lambda$为$f(x)$在点$x_0$处的一个列导数,记为$Df(x_0)=\lambda$。
\end{definition}
\begin{theorem}
	函数$f(x)$在点$x_0$处存在导数$f'(x_0)$的充要条件为$f(x)$在点$x_0$处的一切列导数都相等。
\end{theorem}
\begin{lemma}
	设$f(x)$为$[a,b]$上的严格增函数,
	\begin{enumerate}
		\item 如果对于$E\subset[a,b]$中的每一个点$x$,都至少有一个列导数$Df(x)\leqslant p(p\geqslant0)$,则$m^*[f(E)]\leqslant pm^*(E)$;
		\item 如果对于$E\subset[a,b]$中的每一个点$x$,都至少有一个列导数$Df(x)\geqslant q(q\geqslant0)$,则$m^*[f(E)]\geqslant qm^*(E)$。
	\end{enumerate}
\end{lemma}
\begin{theorem}[Lebesgue theorem]\label{theo:Lebesgue theorem}
	设$f(x)$为$[a,b]$上的单调函数,则:
	\begin{enumerate}
		\item $f(x)$存在有限导数$f'(x)$a.e.于$[a,b]$;
		\item $f'(x)$在$[a,b]$上可积;
		\item 如果$f'(x)$为增函数,则有:
		\begin{equation*}
			\int_{a}^{b}f'(x)\dif x\leqslant f(b)-f(a)
		\end{equation*}
	\end{enumerate}
\end{theorem}
\begin{proof}
	设$f(x)$为增函数,减函数同理。\par
	令:
	\begin{equation*}
		E=\{x:f'(x)\text{不存在}\}
	\end{equation*}
	于是对任意的$x_0\in E$,总有两个列导数\info{导数不存在的点一定至少存在两个列导数吗?}
\end{proof}

\subsection{有界变差函数}
\begin{definition}
	设$f(x)$为$[a,b]$上的有界函数。如果对于$[a,b]$上的一切划分$P$,都有:
	\begin{equation*}
		\left\{\sum_{i=1}^{n}|f(x_i)-f(x_{i-1})|\right\}
	\end{equation*}
	为一个有界数集(其中$x_i,i=1,2,\dots,n$为划分的分点),则称$f(x)$为$[a,b]$上的有界变差函数,并称:
	\begin{equation*}
		\sup_P\left\{\sum_{i=1}^{n}|f(x_i)-f(x_{i-1})|\right\}
	\end{equation*}
	为$f(x)$在$[a,b]$上的全变差,记为$\bigvee_a^b(f)$。用一个划分作成的和数:
	\begin{equation*}
		\bigvee=\sum_{i=1}^{n}|f(x_i)-f(x_{i-1})|
	\end{equation*}
	称为$f(x)$在此划分下对应的变差。
\end{definition}
\subsubsection{有界变差关于区间的可加性}
\begin{theorem}
	设$f(x)$在$[a,b]$上有界变差,则也在其任意子区间$[a_1,b_1]$上有界变差。
\end{theorem}
\begin{proof}
	对$[a_1,b_1]$取任意一个划分:
	\begin{equation*}
		P_1:a_1=x_0<x_1<x_2<\cdots<x_n=b_1
	\end{equation*}
	其对应的变差为$\bigvee_{}^{}$。此时取$[a,b]$的一个划分;
	\begin{equation*}
		P_2:a=y_0<y_1=x_0<\cdots<y_{n+1}=x_n<y_{n+2}=b
	\end{equation*}
	其对应的变差为$\bigvee_1$,则显然有:
	\begin{equation*}
		\bigvee_{}^{}\leqslant\bigvee_1\leqslant\bigvee_{a}^{b}(f)
	\end{equation*}
	由上确界的不等式性即可得:
	\begin{equation*}
		\bigvee_{a_1}^{b_1}(f)\leqslant\bigvee_{a}^{b}(f)
	\end{equation*}
	即$f(x)$在$[a_1,b_1]$上有界变差。
\end{proof}
\begin{lemma}
	设$f(x)$为$[a,b]$上的函数。对于$[a,b]$上的任一划分,若增加分点,则变差不减;若减少分点,则变差不增。
\end{lemma}
\begin{proof}
	对$[a,b]$取任意一个划分:
	\begin{equation*}
		P_1:a=x_0<x_1<x_2<\cdots<x_n=b
	\end{equation*}
	其对应的变差为:
	\begin{equation*}
		\bigvee_1=\sum_{i=1}^{n}|f(x_i)-f(x_{i-1})|
	\end{equation*}
	若此时增加一个分点$x_m=c$,则划分变为:
	\begin{equation*}
		P_2:a=x_0<x_1<x_2<\cdots<x_m<c<x_{m+1}\cdots<x_n=b
	\end{equation*}
	其对应的变差为:
	\begin{equation*}
		\bigvee_2=\sum_{i=1}^{m}|f(x_i)-f(x_{i-1})|+\sum_{i=m+1}^{n}|f(x_i)-f(x_{i-1})|+|f(c)-f(x_m)|+|f(x_{m+1})-f(c)|
	\end{equation*}
	两个变差的差为:
	\begin{equation*}
		\bigvee_2-\bigvee_1=|f(x_{m+1})-f(c)|+|f(c)-f(x_m)|-|f(x_{m+1})-f(x_m)|
	\end{equation*}
	由绝对值的三角不等式,显然有$\bigvee_2>\bigvee_1$。\par
	增加多个分点的情况可直接由增加单个分点的结论推得,减少分点的情况可直接由增加分点的结论推得。
\end{proof}
\begin{theorem}\label{theo:有界变差关于区间的可加性}
	设$a<c<b$,$f(x)$分别在$[a,c]$和$[c,b]$上有界变差,则$f(x)$在$[a,b]$上也有界变差,同时有:
	\begin{equation*}
		\bigvee_{a}^{b}(f)=\bigvee_{a}^{c}(f)+\bigvee_{c}^{b}(f)
	\end{equation*}
\end{theorem}
\begin{proof}
	对$[a,b]$取任意一个划分:
	\begin{equation*}
		P:a=x_0<x_1<x_2<\cdots<x_n=b
	\end{equation*}
	其对应的变差记为$\bigvee$。对其再插入一个分点$c$,记$[a,c]$上的变差为$\bigvee_1$,$[c,b]$上的变差为$\bigvee_2$,则:
	\begin{equation*}
		\bigvee\leqslant\bigvee_1+\bigvee_2
	\end{equation*}
	由上确界的不等式性,依次对$\bigvee_1,\bigvee_2,\bigvee$取关于划分的上确界可得:
	\begin{equation*}
		\bigvee_{a}^{b}(f)\leqslant\bigvee_{a}^{c}(f)+\bigvee_{c}^{b}(f)
	\end{equation*}
	因为$f(x)$分别在$[a,c]$和$[c,b]$上有界变差,所以$\bigvee_{a}^{c}(f)+\bigvee_{c}^{b}(f)<+\infty$,即$f(x)$在$[a,b]$上也有界变差。\par
	对$[a,c]$和$[c,b]$分别任取两个划分:
	\begin{equation*}
		P_1:a=y_0<y_1<y_2<\cdots<y_m=c,\;
		P_2:c=z_0<z_1<z_2<\cdots<x_n=b
	\end{equation*}
	相应的变差分别为:
	\begin{equation*}
		\bigvee_1=\sum_{i=1}^{m}|f(y_i)-f(y_{i-1})|,\;\bigvee_2=\sum_{i=1}^{n}|f(z_i)-f(z_{i-1})|
	\end{equation*}
	将上述两部分合并起来,则得到了一个$[a,b]$上的划分,所以:
	\begin{equation*}
		\bigvee_1+\bigvee_2\leqslant\bigvee_{a}^{b}(f)
	\end{equation*}
	由上确界的不等式性,依次对$\bigvee_1,\bigvee_2$取关于划分的上确界可得:
	\begin{equation*}
		\bigvee_{a}^{c}(f)+\bigvee_{c}^{b}(f)\leqslant\bigvee_{a}^{b}(f)
	\end{equation*}
	所以:
	\begin{equation*}
		\bigvee_{a}^{c}(f)+\bigvee_{c}^{b}(f)=\bigvee_{a}^{b}(f)\qedhere
	\end{equation*}
\end{proof}
\subsubsection{有界变差与有界的关系}
\begin{theorem}
	设$f(x)$在$[a,b]$上有界变差,则$f(x)$在$[a,b]$上有界。
\end{theorem}
\begin{proof}
	对于任意的$x$满足$a\leqslant x\leqslant b$,有:
	\begin{equation*}
		\bigvee=|f(x)-f(a)|+|f(b)-f(x)|\leqslant\bigvee_{a}^{b}(f)
	\end{equation*}
	于是:
	\begin{equation*}
		|f(x)|-|f(a)|\leqslant|f(x)-f(a)|\leqslant\bigvee_{a}^{b}(f)
	\end{equation*}
	即:
	\begin{equation*}
		|f(x)|\leqslant|f(a)|+\bigvee_{a}^{b}(f)
	\end{equation*}
	因为$f(x)$在$[a,b]$上有界变差,所以$\bigvee_{a}^{b}(f)<+\infty$。若$f(x)$在点$a$处无界,则$f(x)$在$[a,b]$上的所有变差中的第一项$|f(x_1)-f(a)|$都无界,$f(x)$不可能在$[a,b]$上有界变差,所以$|f(a)|<+\infty$。综上,$|f(x)|<+\infty$。由$x$的任意性,$f(x)$在$[a,b]$上有界。
\end{proof}
\subsubsection{有界变差函数的运算}
\begin{theorem}
	设$f(x),g(x)$在$[a,b]$上都有界变差,$\alpha,\beta\in\mathbb{R}$,则$\alpha f(x)+\beta g(x),\;f(x)g(x),\;|f(x)|$也在$[a,b]$上有界变差。
\end{theorem}
\begin{proof}
	(1)$\alpha f(x)+\beta g(x)$:\par
	令$s(x)=\alpha f(x)+\beta g(x)$,由绝对值的三角不等式:
	\begin{equation*}
		|s(x_{m+1})-s(x_m)|\leqslant|\alpha|\;|f(x_{m+1})-f(x_m)|+|\beta|\;|g(x_{m+1})-g(x_m)|
	\end{equation*}
	所以:
	\begin{equation*}
		\bigvee_{a}^{b}(s)\leqslant|\alpha|\bigvee_{a}^{b}(f)+|\beta|\bigvee_{a}^{b}(g)
	\end{equation*}
	因为$f(x),g(x)$在$[a,b]$上都有界变差,所以$\bigvee_{a}^{b}(s)<+\infty$,即$\alpha f(x)+\beta g(x)$在$[a,b]$上有界变差。\par
	(2)令$p(x)=f(x)g(x)$,设$A=\sup|f(x)|,\;B=\sup|g(x)|$。因为有界变差函数都有界,所以$A,B<+\infty$,于是:
	\begin{align*}
		|p(x_m+1)-p(x_m)|
		&=|f(x_{m+1})g(x_{m+1})-f(x_m)g(x_m)| \\
		&=|f(x_{m+1})g(x_{m+1})-f(x_m)g(x_{m+1})+f(x_m)g(x_{m+1})-f(x_m)g(x_m)| \\
		&\leqslant|f(x_{m+1})g(x_{m+1})-f(x_m)g(x_{m+1})|+|f(x_m)g(x_{m+1})-f(x_m)g(x_m)| \\
		&\leqslant B|f(x_{m+1})-f(x_m)|+A|g(x_{m+1})-g(x_m)|
	\end{align*}
	所以:
	\begin{equation*}
		\bigvee_{a}^{b}(p)\leqslant B\bigvee_{a}^{b}(f)+A\bigvee_{a}^{b}(g)<+\infty
	\end{equation*}
	即$f(x)g(x)$在$[a,b]$上有界变差。\par
	(3)因为:
	\begin{equation*}
		\Big||f(x_{m+1})|-|f(x_m)|\Big|\leqslant|f(x_{m+1}-f(x_m))|
	\end{equation*}
	所以:
	\begin{equation*}
		\bigvee_{a}^{b}(|f|)\leqslant\bigvee_{a}^{b}(f)<+\infty
	\end{equation*}
	于是$|f(x)|$在$[a,b]$上有界变差。
\end{proof}
\begin{theorem}[Jordan分解定理]
	$[a,b]$上的任一有界变差函数$f(x)$都可表示为两个增函数的差。
\end{theorem}
\begin{proof}
	由\cref{theo:有界变差关于区间的可加性},函数:
	\begin{equation*}
		g(x)=\bigvee_{a}^{x},\;x\in[a,b]
	\end{equation*}
	是$[a,b]$上的增函数。令:
	\begin{equation*}
		h(x)=g(x)-f(x)
	\end{equation*}
	对于任意的$x_1,x_2$满足$a\leqslant x_1<x_2\leqslant b$,有:
	\begin{align*}
		h(x_2)-h(x_1)
		&=g(x_2)-g(x_1)-[f(x_2)-f(x_1)] \\
		&=\bigvee_{x_1}^{x_2}(f)-[f(x_2)-f(x_1)] \\
		&\geqslant|f(x_2)-f(x_1)|-[f(x_2)-f(x_1)]\geqslant0
	\end{align*}
	所以$h(x)$为单调增函数。综上,$f(x)$可表示为$g(x)-h(x)$,其中$g(x),h(x)$都是增函数。
\end{proof}
\begin{corollary}
	有界变差函数至多有可数个不连续点。
\end{corollary}
\begin{proof}
	单调函数至多有可数个不连续点。\info{补充证明}
\end{proof}
\begin{corollary}
	设$f(x)$为$[a,b]$上的有界变差函数,则:
	\begin{enumerate}
		\item $f(x)$存在导数$f'(x)$a.e.于$E$。
		\item $f'(x)$在$[a,b]$上可积。
	\end{enumerate}
\end{corollary}
\begin{proof}
	由\cref{theo:Lebesgue theorem}、极限的可加运算和积分的可加运算可立即得到。
\end{proof}

\subsection{不定积分}
\begin{definition}
	设$f(x)$在$[a,b]$上Lebesgue可积,则$[a,b]$上的函数:
	\begin{equation*}
		F(x)=\int_{a}^{x}f(t)\dif t+C
	\end{equation*}
	称为$f(x)$的一个不定积分,其中$C$为任意常数。
\end{definition}
\begin{definition}
	设$F(x)$是$[a,b]$上的有界函数。若对于任意的$\varepsilon>0$,存在$\delta>0$,使得对$[a,b]$中互不相交的任意有限个开区间$(a_i,b_i),\;i=1,2,\dots,n$,只要$\sum\limits_{i=1}^{n}(b_i-a_i)<\delta$,就有:
	\begin{equation*}
		\sum_{i=1}^{n}|F(b_i)-F(a_i)|<\varepsilon
	\end{equation*}
	则称$F(x)$为$[a,b]$上的绝对连续函数。
\end{definition}
\subsubsection{绝对连续函数的性质}
\begin{property}
	设$f(x),g(x)$是$[a,b]$上的绝对连续函数,$\alpha,\beta\in\mathbb{R}$,则:
	\begin{enumerate}
		\item $f(x)$是一致连续的;
		\item $f(x)$是有界变差的。
		\item $\alpha f(x)+\beta g(x),f(x)g(x),\dfrac{1}{f(x)}$(除法中$f(x)\ne0$)也是绝对连续函数;
	\end{enumerate}
\end{property}
\begin{proof}
	(1)由绝对连续函数的定义,对任意的$\varepsilon>0$,当$x,y\in[a,b]$且$|x-y|<\delta$时,就有$|f(x)-f(y)|<\varepsilon$,即$f(x)$是一致连续的。\par
	(2)\info{需要再思考如何证明}\par
	(3)$(a_i,b_i),\;i=1,2,\dots,n,\;n\in\mathbb{N}^+$是$[a,b]$上互不相交的有限个开区间。\par
	$\alpha f(x)+\beta g(x)$:\par
	由绝对值的三角不等式可得:
	\begin{equation*}
		\sum_{i=1}^{n}|\alpha f(b_i)+\beta g(b_i)-\alpha f(a_i)-\beta g(a_i)|
		\leqslant|\alpha|\sum_{i=1}^{n}|f(b_i)-f(a_i)|+|\beta|\sum_{i=1}^{n}|g(b_i)-g(a_i)|
	\end{equation*}\par
	$f(x)g(x)$:\par
	因为$f(x),g(x)$是绝对连续函数,由(1)可得它们都是一致连续的,从而在$[a,b]$上连续。因为连续函数在闭区间上有界,设$|f(x)|\leqslant M_1,|g(x)|\leqslant M_2,\;x\in[a,b]$。所以:
	\begin{align*}
		\sum_{i=1}^{n}|f(b_i)g(b_i)-f(a_i)g(a_i)|
		&=\sum_{i=1}^{n}|f(b_i)g(b_i)-f(a_i)g(b_i)+f(a_i)g(b_i)-f(a_i)g(a_i)| \\
		&\leqslant\sum_{i=1}^{n}|f(b_i)g(b_i)-f(a_i)g(b_i)|+\sum_{i=1}^{n}|f(a_i)g(b_i)-f(a_i)g(a_i)| \\
		&\leqslant M_2\sum_{i=1}^{n}|f(b_i)-f(a_i)|+M_1\sum_{i=1}^{n}|g(b_i)-g(a_i)|
	\end{align*}\par
	$\dfrac{1}{f(x)}$:\par
	因为存在$M>0$使得$|f(x)|\leqslant M,\;x\in[a,b]$,所以
	\begin{align*}
		\sum_{i=1}^{n}\left|\frac{1}{f(b_i)}-\frac{1}{f(a_i)}\right|
		&=\sum_{i=1}^{n}\left|\frac{f(a_i)-f(b_i)}{f(a_i)f(b_i)}\right|
	\end{align*}
	由因为$f(x)$是绝对连续函数,所以对任意的$\varepsilon>0$,存在$\delta_i$,当$b_i-a_i<\delta_i$时,有:
	\begin{equation*}
		|f(a_i)-f(b_i)|<\frac{\varepsilon |f(a_i)f(b_i)|}{n}
	\end{equation*}
	于是对任意的$\varepsilon>0$,只要取$\delta<\min\{\delta_1,\delta_2,\dots,\delta_n\}$,就有:
	\begin{equation*}
		\sum_{i=1}^{n}\left|\frac{1}{f(b_i)}-\frac{1}{f(a_i)}\right|<\sum_{i=1}^{n}\frac{\varepsilon}{n}=\varepsilon\qedhere
	\end{equation*}
\end{proof}
\begin{theorem}
	设$f(x)$在$[a,b]$上Lebesgue可积,则其不定积分$F(x)$为绝对连续函数,$F'(x)$存在a.e.于$[a,b]$且$F'(x)=f(x)\;$a.e.于$[a,b]$。\info{这部分还没证明}
\end{theorem}
\begin{proof}
	任取$[a,b]$上互不相交的有限个开区间$(a_i,b_i),\;i=1,2,\dots,n,\;n\in\mathbb{N}^+$。因为开区间都是可测集,所以$(a_i,b_i)$可测,$\underset{i=1}{\overset{n}{\cup}}(a_i,b_i)$可测。
	由Lebesgue积分的线性性质,
	\begin{align*}
		\sum_{i=1}^{n}|F(b_i)-F(a_i)|
		&=\sum_{i=1}^{n}\left|\int_{(a_i,b_i)}f(x)\dif x\right| \\
		&\leqslant\sum_{i=1}^{n}\int_{(a_i,b_i)}|f(x)|\dif x \\
		&=\int_{\underset{i=1}{\overset{n}{\cup}}(a_i,b_i)}^{}|f(x)|\dif x
	\end{align*}
	由Lebesgue积分的绝对连续性,对任意的$\varepsilon>0$,存在$\delta>0$,当$m\left[\underset{i=1}{\overset{n}{\cup}}(a_i,b_i)\right]=\sum\limits_{i=1}^{n}(b_i-a_i)<\delta$时,就有:
	\begin{equation*}
		\int_{\underset{i=1}{\overset{n}{\cup}}(a_i,b_i)}^{}|f(x)|\dif x<\delta
	\end{equation*}
	即:
	\begin{equation*}
		\sum_{i=1}^{n}|F(b_i)-F(a_i)|<\delta
	\end{equation*}
	所以$F(x)$是绝对连续函数。
\end{proof}
\begin{theorem}
	设$F(x)$为$[a,b]$上的绝对连续函数,且$F'(x)=0\;$a.e.于$[a,b]$,则$F(x)$为一常数。
\end{theorem}
\begin{theorem}
	设$f(x)$在$[a,b]$上Lebesgue可积,则存在绝对连续函数$F(x)$使得$F'(x)=f(x)\;$a.e.于$[a,b]$。
\end{theorem}
\begin{theorem}
	设$F(x)$时$[a,b]$上的绝对连续函数,则$F'(x)$存在a.e.于$[a,b]$且$F'(x)$在$[a,b]$上可积,同时有:
	\begin{equation*}
		F(x)=F(a)+\int_{[a,x]}^{}F'(t)\dif t
	\end{equation*}
\end{theorem}
\begin{corollary}
	$F(x)$是$[a,b]$上的绝对连续函数的充要条件为$F(x)$是一个Lebesgue可积函数的不定积分。
\end{corollary}