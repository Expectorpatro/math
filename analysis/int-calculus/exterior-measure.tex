\section{外测度}
\subsection{外测度的定义}
\begin{definition}
	设$E$为$\mathbb{R}^n$中的任一点集,$E$的Lebesgue\gls{EMeasure}定义为所有能够覆盖$E$的开区间列的体积总和的下确界。即:
	\begin{equation}
		m^*(E)=\inf_{E\subset\underset{i=1}{\overset{+\infty}{\cup}}I_i}\sum_{i=1}^{+\infty}|I_i|\notag
	\end{equation}
\end{definition}
需要注意如下事项:
\begin{enumerate}
	\item 必须是无穷多个开区间的体积总和的下确界。如果将定义改为有限个:取$E$为$[0,1]$内的有理数集,若$E$被有限个开区间覆盖,即$E\subset\underset{i=1}{\overset{n}{\cup}}I_i$,那么$\underset{i=1}{\overset{n}{\cup}}I_i$一定覆盖$[0,1]$(反证法,有理数集的稠密性),即这种定义下$E$的外测度等于$1$。同理,$[0,1]$内无理数集的外测度也为$1$,由测度公理$(2)$,$[0,1]$的测度为$2$,而由测度公理$(3)$,$[0,1]$的测度应为$1$,矛盾。
	\item 定义虽然是无穷多个开区间,但实际仍可以是有限的,因为可以取空集(就比如概率论可列可加性推有限可加性)。 
	\item 体积总和可以是$+\infty$。
\end{enumerate}
\subsection{外测度的性质}
\begin{property}
	外测度有以下三条基本性质:
	\begin{enumerate}
		\item $m^*(E)\geqslant0$,等号成立当且仅当$E=\varnothing$。
		\item 若$A\subset B$,则有$m^*(A)\leqslant m^*(B)$。
		\item 	$m^*(\underset{i=1}{\overset{+\infty}{\cup}}A_i)\leqslant\sum\limits_{i=1}^{+\infty}m^*(A_i)$。
	\end{enumerate}
\end{property}
\begin{proof}
	(1)显然成立。\par
	(2)对于满足条件$B\subset\underset{i=1}{\overset{+\infty}{\cup}}I_i$的任意开区间列$\{I_i,\;i\in\mathbb{N}^+\}$,必然有$A\subset\underset{i=1}{\overset{+\infty}{\cup}}I_i$,又因为$m^*(A)$是长度总和的下确界,那么就有:
	\begin{equation}
		m^*(A)\leqslant\sum_{i=1}^{+\infty}|I_i|\notag
	\end{equation}
	由下确界的保号性即可推得:
	\begin{equation}
		m^*(A)\leqslant\inf_{B\subset\underset{i=1}{\overset{+\infty}{\cup}}I_i}\sum_{i=1}^{+\infty}|I_i|=m^*(B)\notag
	\end{equation}\par
	(3)从单个$A_i$开始着手考虑。因为外测度是下确界,对任意$\varepsilon>0$和每个$A_n,n\in\mathbb{N}^+$,都存在一个开区间列$\{I_{ni},\;i\in\mathbb{N}^+\}$使\footnote{$\frac{\varepsilon}{2^n}$是一种为了在求和后得到$\varepsilon$的常用规范化取法,这是因为$\sum\limits_{n=1}^{+\infty}\frac{1}{2^n}=1$}:
	\begin{equation}
		\sum_{i=1}^{+\infty}|I_{ni}|<m^*(A_n)+\frac{\varepsilon}{2^n},\;A_n\subset\underset{i=1}{\overset{+\infty}{\cup}}I_{ni}\notag
	\end{equation}
	那么就有(第一个不等式是因为外测度是下确界,第二个不等式是上式的求和形式):
	\begin{equation}
		m^*(\underset{i=1}{\overset{+\infty}{\cup}}A_i)\leqslant\sum_{n=1}^{+\infty}\sum_{i=1}^{+\infty}|I_{ni}|<\sum_{n=1}^{+\infty} m^*(A_n)+\varepsilon\notag
	\end{equation}
	由$\varepsilon$的任意性,有:
	\begin{equation*}
		m^*(\underset{i=1}{\overset{+\infty}{\cup}}A_i)\leqslant\sum_{n=1}^{+\infty}\sum_{i=1}^{+\infty}|I_{ni}|\leqslant\sum_{n=1}^{+\infty} m^*(A_n)\qedhere 
	\end{equation*} 
\end{proof}
\subsection{区间的外测度}
\begin{theorem}
	区间的外测度就是区间的体积,即$m^*(I)=|I|$。
\end{theorem}
从直观上这一点很好理解,如果$m^*(I)>|I|$,则必然能找到一个体积总和更小的开区间列覆盖$I$;如果$m^*(I)<|I|$,则$I$必然没有被对应的开区间列全覆盖。
\begin{proof}
	(1)对任意区间$I$,必存在一个开区间$I'$使$I\subset I'$且$|I'|<|I|+\varepsilon$。那么就有(第一个不等式是因为外测度是下确界):
	\begin{equation}
		m^*(I)\leqslant|I'|<|I|+\varepsilon\notag
	\end{equation}
	由$\varepsilon$的任意性,即有:
	\begin{equation}
		m^*(I)\leqslant|I|\notag
	\end{equation}
	(2)如果$m^*(I)<|I|$,由外测度定义,必存在一个开区间列$\{A_i\}$,使$I\subset\underset{i=1}{\overset{+\infty}{\cup}}A_i$且$\sum\limits_{i=1}^{+\infty}|A_i|<|I|$,而这是不可能的。
\end{proof}












