\section{可测函数}

\subsection{可测函数的性质}
\subsubsection{限制与延拓}
\begin{theorem}
	关于可测函数的限制与延拓有如下结论:
	\begin{enumerate}
		\item 设$f(x)$是可测集$E\subset\mathbb{R}^n$上的可测函数,$E_1$是$E$的可测子集,则$f(x)$限制在$E_1$上时也是可测函数。
		\item 设$f(x)$在可测集$E_i,i=1,2,\dots,n,\;n\in\mathbb{N}^+$上都是可测函数,则$f(x)$延拓在$E=\underset{i=1}{\overset{n}{\cup}}E_i$上时也是可测函数。
	\end{enumerate}
\end{theorem}
\begin{proof}
	(1)只需注意到:
	\begin{equation*}
		E_1(f>a)=E_1\cap E(f>a)
	\end{equation*}
	(2)只需注意到:
	\begin{equation*}
		E(f>a)=\underset{i=1}{\overset{n}{\cup}}E_i(f>a)\qedhere
	\end{equation*}
\end{proof}

\begin{corollary}
	设$\{f_n(x)\}$是$E$上一列可测函数,若$F(x)=\lim\limits_nf_n(x)$几乎处处\footnote{设$\pi$是一个与集合$E$中的点$x$有关的命题,如果$\exists\;M\subset E,\;mM=0$,使得$\pi$在$E\;\backslash\;M$上成立,则称$\pi$在$E$上几乎处处成立,记为$\pi$a.e.(almost everywhere)于$E$。}存在,则$F(x)$是$E$上的可测函数。
\end{corollary}



\subsection{可测函数列与一致收敛}
\begin{theorem}[叶戈罗夫定理]
	设$m(E)<+\infty$,$\{f_n\}$是$E$上一列a.e.收敛于一个a.e.有限的函数$f$的可测函数。对任意的$\delta>0$,$\exists\;E_\delta\subset E$,使得$\{f_n\}$在$E_\delta$上一致收敛,且$m(E\;\backslash\;E_\delta)<\delta$。
\end{theorem}
即$m(E)<+\infty$时,a.e.收敛在收敛对象a.e.有限的时候基本上一致收敛。

\subsection{可测函数与连续函数的关系}
\subsubsection{连续函数都是可测函数}
\begin{theorem}
	可测集$E\subset\mathbb{R}^n$上的连续函数是可测函数。
\end{theorem}
\begin{proof}
	任取一个定义在可测集$E\subset\mathbb{R}^{n}$上的连续函数$f$,$a$是任意实数。由$f$的连续性:
	\begin{equation*}
		\forall\;x\in E(f>a),\;\exists\;U(x),\;U(x)\cap E\subset E(f>a)
	\end{equation*}
	令$G=\bigcup\limits_{x\in E(f>a)}U(x)$,则:
	\begin{equation*}
		G\cap E=\left[\bigcup_{x\in E(f>a)}U(x)\right]\cap E=\bigcup_{x\in E(f>a)}\left[U(x)\cap E\right]\subset E(f>a)
	\end{equation*}
	反之显然有:
	\begin{equation*}
		E(f>a)\subset G\cap E
	\end{equation*}
	因此$E(f>a)=G\cap E$。由于$G$是开集,$E$是可测集,所以$E(f>a)$是可测集。由$a$的任意性,$f$是$E$上的可测函数。由$f$的任意性,命题成立。
\end{proof}
\subsubsection{a.e.有限的可测函数基本上连续}
\begin{theorem}[卢津定理]
	设$f(x)$是$E$上a.e.有限的可测函数。对任意的$\delta>0$,存在闭子集$F_\delta\subset E$,使得$f(x)$在$F_\delta$上连续,并且$m(E\;\backslash\;F_\delta)<\delta$。
\end{theorem}
%\begin{proof}
%	从三个角度逐步考虑。\par
%	(1)简单函数\par
%	设简单函数$f(x)$定义在$E=\underset{i=1}{\overset{n}{\cup}}E_i$上,$f(x)=c_i,\;x\in E_i\;i=1,2,\dots,n$。对任意的$\delta>0$,由于$E_i$是可测集,从而存在闭子集$F_i\subset E_i$,且$m(E_i\;\backslash\;F_i)<\frac{\delta}{n}$。
%\end{proof}
因该定理中函数限制在闭集上连续这一条件有时应用起来不太方便,下给出卢津定理的另一种形式:
\begin{theorem}
	设$f(x)$是$E\subset R$上a.e.有限的可测函数,则对任意的$\delta>0$,存在闭集$F\subset E$及定义在整个$R$上的连续函数$g(x)$($F$和$g(x)$依赖于$\delta$),使得在$F$上$f(x)=g(x)$,并且$m(E\;\backslash\;F)<\delta$。
\end{theorem}


