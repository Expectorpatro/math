\section{函数序列与函数项级数}

本章讨论各项都是$x$的函数的序列:
\begin{equation*}
	f_1(x),f_2(x),\cdots,f_n(x),\cdots
\end{equation*}
同时也讨论各项都是$x$的函数的级数:
\begin{equation*}
\sum_{n=1}^{+\infty}f_n(x)
\end{equation*}
\begin{definition}
	使得函数序列或函数项级数收敛的$x$的集合,被称为序列或级数的\gls{DomainOfConvergence}。
\end{definition}

\subsection{函数序列的收敛}

\subsection{逐点收敛}
\subsubsection{极限函数的定义}
\begin{definition}
	设$D\subset\mathbb{R}$。如果函数序列$\{f_n(x)\}$的收敛域包含了$D$,那么:
	\begin{equation*}
		\forall\;x\in D,\;\lim_{n\to+\infty}f_n(x)\in\mathbb{R}
	\end{equation*}
	可以定义一个函数:
	\begin{equation*}
		f(x)=\lim_{n\to+\infty}f_n(x),\;f:D\rightarrow\mathbb{R}
	\end{equation*}
	我们把函数$f$称之为函数序列$\{f_n(x)\}$在集合$D$上的\gls{LimitF}。
\end{definition}
\subsubsection{函数序列逐点收敛的定义}
\begin{definition}
	函数序列$\{f_n(x)\}$在集合$E$上\gls{PointwiseConvergence}于函数$f(x)$是指:
	\begin{equation*}
		\forall\;x\in E,\;\forall\;\varepsilon>0,\;\exists\; N(x,\varepsilon)\in\mathbb{N}^+,\;\forall\;n>N,\;|f_n(x)-f(x)|<\varepsilon
	\end{equation*}
\end{definition}

\subsection{一致收敛}
\subsubsection{函数序列一致收敛的定义}
\begin{definition}
	设函数序列$\{f_n(x)\}$在集合$E$上逐点收敛于函数$f(x)$。如果:
	\begin{equation*}
		\forall\;\varepsilon>0,\;\exists\; N(\varepsilon)\in\mathbb{N}^+,\;\forall\;n>N,\;\forall\;x\in E,\;|f_n(x)-f(x)|<\varepsilon
	\end{equation*}
	则称函数序列$\{f_n(x)\}$在集合$E$上\gls{UniformConvergence}于函数$f(x)$,记为:
	\begin{equation*}
		f_n(x)\underset{E}{\rightrightarrows}f(x)\quad(n\to+\infty)
	\end{equation*}
\end{definition}
\subsubsection{函数序列一致收敛的等价叙述}
\begin{theorem}
	设函数序列$\{f_n(x)\}$在集合$E$上逐点收敛于函数$f(x)$。记:
	\begin{equation*}
		\rho(f_n(x),f(x))=\sup_{x\in E}|f_n(x)-f(x)|
	\end{equation*}
	则以下三条陈述等价:
	\begin{enumerate}
		\item 函数序列$\{f_n(x)\}$在集合$E$上一致收敛于函数$f(x)$。
		\item $\lim\limits_{n\to+\infty}\rho(f_n(x),f(x))=0$。
		\item 对任何序列$\{x_n\}\subset E$,都有:
		\begin{equation*}
			\lim_{n\to+\infty}[f_n(x_n)-f(x_n)]=0
		\end{equation*}
	\end{enumerate}
\end{theorem}
\begin{proof}
	$(1)\rightarrow(2)$和$(2)\rightarrow(3)$是显然的。下证$(3)\rightarrow(1)$:\par
	假设(1)不成立,则
	\begin{equation*}
		\exists\;\varepsilon>0,\;\forall\;N(\varepsilon)\in\mathbb{N}^+,\;\exists\;n>N,\;\exists\;x\in E,\;|f_n(x)-f(x)|\geqslant\varepsilon
	\end{equation*}
	取一个固定地$\varepsilon>0$,令$n_0=0$,则$\exists\;n_k\in\mathbb{N}^+,\;n_k>n_{k-1}+1$,且$x_{n_k}\in E$,使得:
	\begin{equation*}
		|f_{n_k}(x_{n_k})-f(x_{n_k})|\geqslant\varepsilon
	\end{equation*}
	显然$\{x_{n_k}\}$这个序列不满足(3)的条件,矛盾。
\end{proof}
\subsubsection{函数序列一致收敛的柯西原理}
以上判别一个函数列是否一致收敛到一个极限函数的方法都需要提前求出极限函数,而以下柯西原理则不需要提前知道极限函数的形式:
\begin{theorem}
	设函数序列$\{f_n(x)\}$的各项在集合$E$上都有定义,则这个函数序列在$E$上一致收敛于某个极限函数的充要条件是:
	\begin{equation*}
		\forall\;\varepsilon>0,\;\exists\;N\in\mathbb{N}^+,\;\forall\;n,m>N,\;m>n,\;\forall\;x\in E,\;|f_m(x)-f_n(x)|<\varepsilon
	\end{equation*}
\end{theorem}
\begin{proof}
	必要性:由三角不等式是显然的。\par
	充分性:由条件,对任意的$x\in E$,$f_n(x)$构成一个$\mathbb{R}$上的柯西序列,因此可定义一个极限函数:
	\begin{equation*}
		f(x)=\lim_{n\to+\infty}f_n(x),\;\forall\;x\in E
	\end{equation*}
	同时,由题目条件可得:
	\begin{equation*}
		\forall\;\varepsilon>0,\;\exists\;N\in\mathbb{N}^+,\;\forall\;n>N,\;\forall\;p\in\mathbb{N}^+,\;\forall\;x\in E,\;|f_{n+p}(x)-f_n(x)|<\varepsilon
	\end{equation*}
	在上式中取$p\to+\infty$即可得到:
	\begin{equation*}
		\forall\;\varepsilon>0,\;\exists\;N\in\mathbb{N}^+,\;\forall\;n>N,\;\forall\;x\in E,\;|f(x)-f_n(x)|\leqslant\varepsilon
	\end{equation*}
	即$\{f_n(x)\}$一致收敛于$f(x)$。
\end{proof}

\subsection{函数项级数的收敛}

\subsection{逐点收敛}
\subsubsection{函数项级数逐点收敛的定义}
\begin{definition}
	函数项级数$\sum\limits_{n=1}^{+\infty}f_n(x)$在集合$E$上逐点收敛于函数$f(x)$是指:
	\begin{equation*}
		\forall\;x\in E,\;\forall\;\varepsilon>0,\;\exists\; N(x,\varepsilon)\in\mathbb{N}^+,\;\forall\;n>N,\;\left|\sum_{i=1}^{n}f_i(x)-f(x)\right|<\varepsilon
	\end{equation*}
\end{definition}

\subsection{一致收敛}
\subsubsection{函数项级数一致收敛的定义}
\begin{definition}
	设函数项级数$\sum\limits_{n=1}^{+\infty}f_n(x)$在集合$E$上逐点收敛于函数$f(x)$。如果:
	\begin{equation*}
		\forall\;\varepsilon>0,\;\exists\; N(\varepsilon)\in\mathbb{N}^+,\;\forall\;n>N,\;\forall\;x\in E,\;\left|\sum_{i=1}^{n}f_i(x)-f(x)\right|<\varepsilon
	\end{equation*}
	则称函数项级数$\sum\limits_{n=1}^{+\infty}f_n(x)$在集合$E$上一致收敛于函数$f(x)$。
\end{definition}
\subsubsection{函数项级数一致收敛的柯西原理}
\begin{theorem}
	设函数项级数$\sum\limits_{n=1}^{+\infty}f_n(x)$在集合$E$上有定义,则这个函数序列在$E$上一致收敛于某个极限函数的充要条件是:
	\begin{equation*}
		\forall\;\varepsilon>0,\;\exists\;N\in\mathbb{N}^+,\;\forall\;n,m>N,\;m>n,\;\forall\;x\in E,\;\left|\sum_{i=n+1}^mf_i(x)\right|<\varepsilon
	\end{equation*}
\end{theorem}
\begin{proof}
	必要性:由三角不等式是显然的。\par
	充分性:由条件,对任意的$x\in E$,$\sum\limits_{i=1}^nf_i(x)$构成一个$\mathbb{R}$上的柯西序列,因此可定义一个极限函数$f(x)$使得:
	\begin{equation*}
		\forall\;x\in E,\;f(x)=\lim_{n\to+\infty}\sum_{i=1}^nf_i(x)
	\end{equation*}
	同时,由题目条件可得:
	\begin{equation*}
		\forall\;\varepsilon>0,\;\exists\;N\in\mathbb{N}^+,\;\forall\;n>N,\;\forall\;p\in\mathbb{N}^+,\;\forall\;x\in E,\;\left|\sum_{i=1}^{n+p}f_i(x)-\sum_{i=1}^{n}f_i(x)\right|<\varepsilon
	\end{equation*}
	在上式中取$p\to+\infty$即可得到:
	\begin{equation*}
		\forall\;\varepsilon>0,\;\exists\;N\in\mathbb{N}^+,\;\forall\;n>N,\;\forall\;x\in E,\;|f(x)-\sum_{i=1}^nf_i(x)|\leqslant\varepsilon
	\end{equation*}
	即函数项级数$\sum\limits_{n=1}^{+\infty}f_n(x)$一致收敛于$f(x)$。
\end{proof}
\begin{corollary}
	若函数项级数$\sum\limits_{n=1}^{+\infty}|f_n(x)|$在集合$E$上一致收敛,则$\sum\limits_{n=1}^{+\infty}f_n(x)$在集合$E$上也一致收敛。
\end{corollary}
\begin{proof}
	由三角不等式即可证得。
\end{proof}
请注意这里没有说二者一致收敛的对象是一样的!!!
\subsubsection{函数项级数绝对一致收敛的判别法}
\begin{theorem}[Weierstrass判别法]
	设函数项级数$\sum\limits_{n=1}^{+\infty}f_n(x)$在集合$E$上有定义。如果存在收敛的正项级数$\sum\limits_{n=1}^{+\infty}x_n$,满足:
	\begin{equation*}
		\forall\;n\in\mathbb{N}^+,\;\forall\;x\in E,\;|f_n(x)|\leqslant x_n
	\end{equation*}
	则函数项级数$\sum\limits_{n=1}^{+\infty}f_n(x)$在集合$E$上绝对一致收敛。称正项级数$\sum\limits_{n=1}^{+\infty}x_n$为优级数。
\end{theorem}
\begin{proof}
	因为数项级数$\sum\limits_{n=1}^{+\infty}x_n$收敛,所以:
	\begin{equation*}
		\forall\;\varepsilon>0,\;\exists\;N\in\mathbb{N}^+,\;\forall\;n,m>N,\;m>n,\;\left|\sum_{i=n+1}^mx_i\right|=\sum_{i=n+1}^mx_i<\varepsilon
	\end{equation*}
	也就有:
	\begin{equation*}
		\forall\;\varepsilon>0,\;\exists\;N\in\mathbb{N}^+,\;\forall\;n,m>N,\;m>n,\;\forall\;x\in E,\;\sum_{i=n+1}^m|f_n(x)|\leqslant\sum_{i=n+1}^mx_i<\varepsilon
	\end{equation*}
	由函数项级数收敛的柯西原理,$\sum\limits_{n=1}^{+\infty}|f_n(x)|$一致收敛,即$\sum\limits_{n=1}^{+\infty}f_n(x)$绝对一致收敛。
\end{proof}
\subsubsection{函数项级数条件一致收敛的判别法}
\begin{theorem}[Dirichlet判别法]
	对函数项级数:
	\begin{equation*}
		\sum_{n=1}^{+\infty}f_n(x)g_n(x),\quad x\in E
	\end{equation*}
	如果:
	\begin{enumerate}
		\item 函数序列$\{f_n(x)\}$对每个取定的$x\in E$都是单调的,并且该函数序列在$E$上一致地趋于$0$;
		\item 函数项级数$\sum\limits_{n=1}^{+\infty}g_n(x)$的部分和序列在$E$上一致有界:
		\begin{equation*}
			\forall\;n\in\mathbb{N}^+,\;\forall\;x\in E,\;\left|\sum_{i=1}^ng_i(x)\right|\leqslant L
		\end{equation*}
	\end{enumerate}
	则函数项级数$\sum\limits_{n=1}^{+\infty}f_n(x)g_n(x)$在$E$上条件一致收敛。
\end{theorem}
\begin{proof}
	利用Abel引理来估计:
	\begin{equation*}
		\left|\sum_{i=n+1}^mf_i(x)g_i(x)\right|
	\end{equation*}
	因为函数项级数$\sum\limits_{n=1}^{+\infty}g_n(x)$的部分和序列在$E$上一致有界,所以:
	\begin{equation*}
		\forall\;n,m\in\mathbb{N}^+,\;m>n,\;\forall\;x\in E,\;\left|\sum_{i=n+1}^mg_i(x)\right|\leqslant\left|\sum_{i=1}^ng_i(x)\right|+\left|\sum_{i=1}^mg_i(x)\right|\leqslant2L
	\end{equation*}
	又因函数序列$\{f_n(x)\}$对每个取定的$x\in E$都是单调的,由Abel引理:
	\begin{equation*}
		\forall\;n,m\in\mathbb{N}^+,\;m>n,\;\forall\;x\in E,\;\left|\sum_{i=n+1}^mf_i(x)g_i(x)\right|\leqslant2L(|f_{n+1}(x)|+2|f_m(x)|)
	\end{equation*}
	因为函数序列$\{f_n(x)\}$在$E$上一致地趋于$0$,所以:
	\begin{equation*}
		\forall\;\varepsilon>0,\;\exists\;N\in\mathbb{N}^+,\;\forall\;n>N,\;\forall\;x\in E,\;|f_n(x)|<\frac{\varepsilon}{6L}
	\end{equation*}
	于是有:
	\begin{equation*}
		\forall\;\varepsilon>0,\;\exists\;N\in\mathbb{N}^+,\;\forall\;n,m>N,\;m>n,\;\forall\;x\in E,\;\left|\sum_{i=n+1}^mf_i(x)g_i(x)\right|<\varepsilon
	\end{equation*}
	由函数项级数一致收敛的柯西原理,级数$\sum\limits_{n=1}^{+\infty}f_n(x)g_n(x)$在$E$上条件一致收敛。
\end{proof}
\begin{theorem}[Abel判别法]
	对函数项级数:
	\begin{equation*}
		\sum_{n=1}^{+\infty}f_n(x)g_n(x),\quad x\in E
	\end{equation*}
	如果:
	\begin{enumerate}
		\item 函数序列$\{f_n(x)\}$对每个取定的$x\in E$都是单调的,并且该函数序列在$E$上一致有界;
		\begin{equation*}
			\forall\;n\in\mathbb{N}^+,\;\forall\;x\in E,\;\left|f_n(x)\right|\leqslant M
		\end{equation*}
		\item 函数项级数$\sum\limits_{n=1}^{+\infty}g_n(x)$在$E$上一致收敛:
	\end{enumerate}
	则函数项级数$\sum\limits_{n=1}^{+\infty}f_n(x)g_n(x)$在$E$上条件一致收敛。
\end{theorem}
\begin{proof}
	因为函数项级数$\sum\limits_{n=1}^{+\infty}g_n(x)$在$E$上一致收敛,则:
	\begin{equation*}
		\forall\;\varepsilon'>0,\;\exists\;N\in\mathbb{N}^+,\;\forall\;n,m>N,\;m>n,\;\left|\sum_{i=n+1}^mg_i(x)\right|<\varepsilon'
	\end{equation*}
	又因函数序列$\{f_n(x)\}$对每个取定的$x\in E$都是单调的,由Abel引理:
	\begin{equation*}
		\forall\;x\in E,\;\left|\sum_{i=n+1}^mf_i(x)g_i(x)\right|\leqslant\varepsilon'(|f_{n+1}(x)|+2|f_m(x)|)\leqslant3M\varepsilon'
	\end{equation*}
	对任意的$\varepsilon>0$,可以选取$\varepsilon'>0$使得:
	\begin{equation*}
		3M\varepsilon'<\varepsilon\qedhere
	\end{equation*}
\end{proof}

\subsection{极限函数的分析性质}

本节来讨论,一致收敛的函数序列或函数项级数需要满足怎样的条件,才能使得它们的极限函数拥有与它们一样的分析性质。在这里讨论的分析性质为:连续性、定积分、微分。

\subsection{连续性}
\subsubsection{极限函数的连续性}
\begin{theorem}
	设函数序列$\{f_n(x)\}$在区间$I$上一致收敛于函数$f(x)$。当函数序列$\{f_n(x)\}$的每一项在$I$上都连续时,$f(x)$在$I$上也连续。
\end{theorem}
\begin{proof}
	任取$x_0\in I,\;x\in I$,可得:
	\begin{equation*}
		|f(x)-f(x_0)|
		\leqslant|f(x)-f_n(x)|+|f_n(x)-f_n(x_0)|+|f_n(x_0)-f(x_0)|
	\end{equation*}
	因为$\{f_n(x)\}$在$I$上一致收敛于函数$f(x)$,所以:
	\begin{equation*}
		\forall\;\varepsilon>0,\;\exists\;N\in\mathbb{N}^+,\;\forall\;n>N,\;\forall\;x\in I,\;|f(x)-f_n(x)|<\varepsilon
	\end{equation*}
	于是:
	\begin{equation*}
		\exists\;N\in\mathbb{N}^+,\;\forall\;n>N,\;|f(x)-f_n(x)|<\varepsilon,\;|f_n(x_0)-f(x_0)|<\varepsilon
	\end{equation*}
	又因为$\{f_n(x)\}$的每一项都在$I$上连续,所以:
	\begin{equation*}
		\forall\;n\in\mathbb{N}^+,\;\forall\;\varepsilon>0,\;\exists\;\delta>0,\;\forall\;x\in\{x\in I:|x-x_0|<\delta\},\;|f_n(x)-f_n(x_0)|<\varepsilon
	\end{equation*}
	取$n\to+\infty$即可得到:
	\begin{equation*}
		\forall\;\varepsilon>0,\;\exists\;\delta>0,\;\forall\;x\in\{x:|x-x_0|<\delta\},\;|f(x)-f(x_0)|<\varepsilon
	\end{equation*}
	即$f(x)$在$x_0$处连续。由$x_0$的任意性,$f(x)$在$I$上连续。
\end{proof}
\begin{theorem}
	设函数项级数$\sum\limits_{n=1}^{+\infty}f_n(x)$在区间$I$上一致收敛于函数$f(x)$。当函数项级数$\sum\limits_{n=1}^{+\infty}f_n(x)$的每一项在$I$上连续时,$f(x)$在$I$上也连续。
\end{theorem}

\subsection{定积分}
\subsubsection{逐项积分定理}
\begin{theorem}
	设函数序列$\{f_n(x)\}$在区间$[a,b]$上一致收敛于函数$f(x)$。当函数序列$\{f_n(x)\}$的每一项在$[a,b]$上都连续时,有:
	\begin{equation*}
		\lim_{n\to+\infty}\left[\int_{a}^{b}f_n(x)\dif x\right]=\int_{a}^{b}f(x)\dif x
	\end{equation*}
\end{theorem}
\begin{proof}
	因为$\{f_n(x)\}$的每一项都在$[a,b]$上连续,所以$f(x)$也连续,于是$f(x)$可积。由积分中值定理:
	\begin{equation*}
		\exists\;\xi\in[a,b],\;\left|\int_{a}^{b}f_n(x)\dif x-\int_{a}^{b}f(x)\dif x\right|=\left|\int_{a}^{b}[f_n(x)-f(x)]\dif x\right|\leqslant(b-a)|f_n(\xi)-f(\xi)|
	\end{equation*}
	因为$\{f_n(x)\}$一致收敛于$f(x)$,所以对任意的$\varepsilon>0,\;\exists\;N\in\mathbb{N}^+,\;\forall\;n>N,\;|f_n(\xi)-f(\xi)|<\varepsilon$,此时即有:
	\begin{equation*}
		\left|\int_{a}^{b}f_n(x)\dif x-\int_{a}^{b}f(x)\dif x\right|<\varepsilon
	\end{equation*}
	也即:
	\begin{equation*}
		\lim_{n\to+\infty}\left[\int_{a}^{b}f_n(x)\dif x\right]=\int_{a}^{b}f(x)\dif x
	\end{equation*}
\end{proof}
\begin{theorem}
	设函数项级数$\sum\limits_{n=1}^{+\infty}f_n(x)$在区间$I$上一致收敛于函数$f(x)$。当函数项级数$\sum\limits_{n=1}^{+\infty}f_n(x)$的每一项在$I$上连续时,有:
	\begin{equation*}
		\sum_{n=1}^{+\infty}\left[\int_{a}^{b}f_n(x)\dif x\right]=\int_{a}^{b}f(x)\dif x
	\end{equation*}
\end{theorem}

\subsection{微分}
\subsubsection{逐项微分定理}
\begin{theorem}
	若函数序列$\{f_n(x)\}$满足:
	\begin{enumerate}
		\item $\{f_n(x)\}$的每一项在$[a,b]$上都连续可微;
		\item 导函数序列$\{f_n'(x)\}$在$[a,b]$上一致收敛于函数$\varphi(x)$;
		\item $\{f_n(x)\}$至少在某一点$x_0\in[a,b]$收敛:
		\begin{equation*}
			\lim_{n\to+\infty}f_n(x_0)=y_0\in\mathbb{R}
		\end{equation*}
	\end{enumerate}
	那么函数序列$\{f_n(x)\}$在$[a,b]$上一致收敛于某个在$[a,b]$连续可微的函数$f(x)$,并且有:
	\begin{equation*}
		f'(x)=\lim_{n\to+\infty}f_n'(x)=\varphi(x)
	\end{equation*}
\end{theorem}
\begin{proof}
	因为$\{f_n(x)\}$的每一项在$[a,b]$上都连续可微,所以导函数序列$\{f_n'(x)\}$的每一项在$[a,b]$上都连续,于是$\{f_n'(x)\}$的每一项在$[a,b]$上都可积,就有:
	\begin{equation}\label{eq:6.3.1}
		\forall\;x\in[a,b],\;\forall\;n\in\mathbb{N}^+,\;f_n(x)=f_n(x_0)+\int_{x_0}^{x}f_n'(\xi)\dif\xi
	\end{equation}
	由此可得:
	\begin{align*}
		|f_m(x)-f_n(x)|
		&\leqslant|f_m(x_0)-f_n(x_0)|+\left|\int_{x_0}^{x}[f_m'(\xi)-f_n'(\xi)]\dif\xi\right| \\
		&\leqslant|f_m(x_0)-f_n(x_0)|+(b-a)\sup_{\xi\in[a,b]}|f_m'(\xi)-f_n'(\xi)|
	\end{align*}
	因为$\{f_n(x)\}$在$x_0$处收敛、导函数序列$\{f_n'(x)\}$在$[a,b]$上一致收敛,所以:
	\begin{gather*}
		\forall\;\varepsilon>0,\;\exists\;N\in\mathbb{N}^+,\;\forall\;m,n>N,\;m>n \\
		|f_m(x_0)-f_n(x_0)|<\frac{\varepsilon}{2},\quad\sup_{\xi\in[a,b]}|f_m'(\xi)-f_n'(\xi)|<\frac{\varepsilon}{2(b-a)}
	\end{gather*}
	于是:
	\begin{equation*}
		\forall\;\varepsilon>0,\;\exists\;N\in\mathbb{N}^+,\;\forall\;m,n>N,\;m>n,\;\forall\;x\in[a,b],\;|f_m(x)-f_n(x)|<\varepsilon
	\end{equation*}
	由柯西收敛准则,$\{f_n(x)\}$一致收敛。设$\{f_n(x)\}$一致收敛的对象为$f(x)$,在\eqref{eq:6.3.1}中取$n\to+\infty$可得:
	\begin{equation*}
		f(x)=f(x_0)+\int_{x_0}^{x}\varphi(\xi)\dif\xi
	\end{equation*}
	对其求导即可得:
	\begin{equation*}
		f'(x)=\varphi(x)\qedhere
	\end{equation*}
\end{proof}
\begin{theorem}
	若函数项级数$\sum\limits_{n=1}^{+\infty}f_n(x)$满足:
	\begin{enumerate}
		\item $\sum\limits_{n=1}^{+\infty}f_n(x)$的每一项在$[a,b]$上都连续可微;
		\item 部分和序列的导函数序列$\left\{\sum\limits_{i=1}^{n}f'_n(x)\right\}$在$[a,b]$上一致收敛于函数$\varphi(x)$;
		\item $\sum\limits_{n=1}^{+\infty}f_n(x)$至少在某一点$x_0\in[a,b]$收敛:
		\begin{equation*}
			\lim_{n\to+\infty}\left[\sum_{i=1}^{n}f_i(x_0)\right]=y_0\in\mathbb{R}
		\end{equation*}
	\end{enumerate}
	那么函数项级数$\sum\limits_{n=1}^{+\infty}f_n(x)$在$[a,b]$上一致收敛于某个在$[a,b]$连续可微的函数$f(x)$,并且有:
	\begin{equation*}
		\left[\sum_{n=1}^{+\infty}f_n(x)\right]'=\sum_{n=1}^{+\infty}f_n'(x)=f'(x)=\varphi(x)
	\end{equation*}
\end{theorem}

\subsection{幂级数}
本节考察如下形式的函数项级数(称之为\gls{PowerSeries}):
\begin{equation*}
	\sum_{n=1}^{+\infty}a_n(x-x_0)^n
\end{equation*}

\subsection{幂级数的收敛半径}
\begin{theorem}[Cauchy-Hadamard identity]
	对于幂级数$\sum\limits_{n=1}^{+\infty}a_n(x-x_0)^n$,记:
	\begin{equation*}
		\rho=\frac{1}{\varlimsup\sqrt[n]{|a_n|}}
	\end{equation*}
	则该幂级数对任意的$x\in\{x:|x-x_0|<\rho\}$绝对收敛,对任意的$x\in\{x:|x-x_0|>\rho\}$发散,称$\rho$为\gls{RadiusOfConvergence}。
\end{theorem}
\begin{proof}
	对于:
	\begin{equation*}
		\varlimsup\sqrt[n]{|a_n(x-x_0)^n|}=|x-x_0|\varlimsup\sqrt[n]{|a_n|}
	\end{equation*}
	由柯西根式判别法,当:
	\begin{equation*}
		|x-x_0|\varlimsup\sqrt[n]{|a_n|}<1
	\end{equation*}
	时,该幂级数绝对收敛。当:
	\begin{equation*}
		\varlimsup\sqrt[n]{|a_n(x-x_0)^n|}>1
	\end{equation*}
	时,有:
	\begin{equation*}
		\exists\;N\in\mathbb{N}^+,\;\forall\;n>N,\;|a_n(x-x_0)^n|>1
	\end{equation*}
	即该幂级数的项不收敛于$0$,不满足级数收敛的必要条件,所以该幂级数发散。
\end{proof}
\begin{theorem}
	幂级数的收敛半径也可由下式计算:
	\begin{equation*}
		\rho=\lim_{n\to+\infty}\left|\frac{a_n}{a_{n+1}}\right|
	\end{equation*}
\end{theorem}
\begin{proof}
	由比值判别法,当:
	\begin{equation*}
		\lim_{n\to+\infty}\left|\frac{a_{n+1}(x-x_0)^{n+1}}{a_n(x-x_0)^n}\right|=\lim_{n\to+\infty}\left|\frac{a_{n+1}(x-x_0)}{a_n}\right|<1
	\end{equation*}
	时,即:
	\begin{equation*}
		|x-x_0|<\lim_{n\to+\infty}\left|\frac{a_n}{a_{n+1}}\right|
	\end{equation*}
	时,该幂级数绝对收敛。当:
	\begin{equation*}
		\lim_{n\to+\infty}\left|\frac{a_{n+1}(x-x_0)^{n+1}}{a_n(x-x_0)^n}\right|=\lim_{n\to+\infty}\left|\frac{a_{n+1}(x-x_0)}{a_n}\right|>1
	\end{equation*}
	时,有:
	\begin{equation*}
		\exists\;N\in\mathbb{N}^+,\;\forall\;n>N,\;|a_{n+1}(x-x_0)^{n+1}|>|a_n(x-x_0)^n|
	\end{equation*}
	即该幂级数的项不收敛于$0$,不满足级数收敛的必要条件,所以该幂级数发散。
\end{proof}
\begin{theorem}
	设幂级数$\sum\limits_{n=1}^{+\infty}a_n(x-x_0)^n$的收敛半径为$\rho$,$[x_0-r,x_0+r]$是包含于$(x_0-\rho,x_0+\rho)$中的任何闭区间,则该幂级数在$[x_0-r,x_0+r]$上绝对一致收敛。
\end{theorem}
\begin{proof}
	在闭区间$[x_0-r,x_0+r]$上,取级数:
	\begin{equation*}
		\sum_{n=1}^{+\infty}|a_n|r^n
	\end{equation*}
	因为$r<\rho$,所以上正项级数收敛。因为:
	\begin{equation*}
		\forall\;n\in\mathbb{N}^+,\;\forall\;x\in[x_0-r,x_0+r],\;|a_n(x-x_0)^n|\leqslant|a_n|r^n
	\end{equation*}
	由Weierstrass判别法,该幂级数在$[x_0-r,x_0+r]$上绝对一致收敛。
\end{proof}

\subsection{幂级数和函数的分析性质}
\subsubsection{连续性}
\begin{theorem}
	设幂级数$\sum\limits_{n=1}^{+\infty}a_n(x-x_0)^n$的收敛半径为$\rho$,则和函数:
	\begin{equation*}
		f(x)=\sum_{n=1}^{+\infty}a_n(x-x_0)^n
	\end{equation*}
	在开区间$(x_0-\rho,x_0+\rho)$内处处连续。若它还在$x=x_0-\rho$处(或在$x=x_0+\rho$处)收敛,则该幂级数在闭区间$[x_0-\rho,x_0]$上(或在闭区间$[x_0,x_0+\rho]$上)一致收敛,所以和函数$f(x)$在$x=x_0-\rho$处右连续(或在$x=x_0+\rho$处左连续)。即:幂级数的和函数在幂级数收敛域的每一点都连续。
\end{theorem}
\begin{proof}
	任取$c\in(x_0-\rho,x_0+\rho)$,则必然$\exists\;r$使得$c\in[x_0-r,x_0+r]$,并且幂级数在$[x_0-r,x_0+r]$上绝对一致收敛。因为幂级数的每一项都是幂函数,所以每一项都在$[x_0-r,x_0+r]$上连续,于是和函数$f(x)$在$[x_0-r,x_0+r]$上连续,由此$f(x)$在$c$点连续。由$c$的任意性,$f(x)$在开区间$(x_0-\rho,x_0+\rho)$内处处连续。\par
	将幂级数写成如下形式:
	\begin{equation*}
		\sum_{n=1}^{+\infty}a_n\rho^n\left(\frac{x-x_0}{\rho}\right)^n
	\end{equation*}
	注意到:
	\begin{enumerate}
		\item 对于$x\in[x_0,x_0+\rho]$,函数序列$\left\{\left(\dfrac{x-x_0}{\rho}\right)^n\right\}$单调下降并且一致有界:
		\begin{equation*}
			1\geqslant\frac{x-x_0}{\rho}\geqslant\left(\frac{x-x_0}{\rho}\right)^2\geqslant\cdots\geqslant\left(\frac{x-x_0}{\rho}\right)^n\geqslant\cdots
		\end{equation*}
		\item 级数$\sum\limits_{n=1}^{+\infty}a_n\rho^n$收敛。
	\end{enumerate}
	由Abel判别法,可以判断幂级数在闭区间$[x_0,x_0+\rho]$上一致收敛。因为幂级数的每一项都是幂函数,所以每一项都在$[x_0,x_0+\rho]$上连续,于是和函数$f(x)$在$[x_0,x_0+\rho]$上连续,由此$f(x)$在$x_0+\rho$左连续。右连续的证明同理可得。综上,幂级数的和函数在幂级数收敛域的每一点都连续。
\end{proof}
\subsubsection{定积分}
\begin{theorem}
	幂级数在任何包含于收敛域的闭区间上都可以逐项积分。
\end{theorem}
\begin{proof}
	由幂级数和函数的连续性可直接得到。
\end{proof}
\subsubsection{微分}
\begin{theorem}
	由幂级数逐项求导所得到的级数:
	\begin{equation*}
		\sum_{n=1}^{+\infty}na_n(x-x_0)^{n-1}
	\end{equation*}
	与原级数有相同的收敛半径。
\end{theorem}
\begin{proof}
	由:
	\begin{equation*}
		\sum_{n=1}^{+\infty}na_n(x-x_0)^{n}=(x-x_0)\left[\sum_{n=1}^{+\infty}na_n(x-x_0)^{n-1}\right]
	\end{equation*}
	可以看出$\sum\limits_{n=1}^{+\infty}na_n(x-x_0)^{n}$与$\sum\limits_{n=1}^{+\infty}na_n(x-x_0)^{n-1}$的收敛域相同,所以它们的收敛半径相同。因为:
	\begin{equation*}
		\varlimsup\sqrt[n]{n|a_n|}=\varlimsup(\sqrt[n]{n}\cdot\sqrt[n]{|a_n|})=\lim\sqrt[n]{n}\cdot\varlimsup\sqrt[n]{|a_n|}=\varlimsup\sqrt[n]{|a_n|}
	\end{equation*}
	所以幂级数逐项求导所得到的级数与原级数有相同的收敛半径。
\end{proof}
\begin{theorem}
	在幂级数$\sum\limits_{n=1}^{+\infty}a_n(x-x_0)^n$收敛域的内部,和函数具有任意阶的导数,并且它的各阶导数可以通过级数逐项求导来计算。
\end{theorem}
\begin{proof}
	设幂级数$\sum\limits_{n=1}^{+\infty}a_n(x-x_0)^n$的收敛半径为$\rho$,那么幂级数$\sum\limits_{n=1}^{+\infty}na_n(x-x_0)^{n-1}$的收敛半径也是$\rho$。对任意的$c\in(x_0-\rho,x_0+\rho)$,可以选取$0<r<\rho$,使得$c\in(x_0-r,x_0+r)$。因为级数$\sum\limits_{n=1}^{+\infty}a_n(x-x_0)^n$和级数$\sum\limits_{n=1}^{+\infty}na_n(x-x_0)^{n-1}$在$[x_0-r,x_0+r]$上一致收敛,并且级数$\sum\limits_{n=1}^{+\infty}a_n(x-x_0)^n$的每一项都连续可微,所以和函数$f(x)=\sum\limits_{n=1}^{+\infty}a_n(x-x_0)^n$在$[x_0-r,x_0+r]$上连续可微,且有:
	\begin{equation*}
		f'(c)=\sum_{n=1}^{+\infty}na_nc^{n-1}
	\end{equation*}
	由$c$的任意性可得:
	\begin{equation*}
		\forall\;x\in(-\rho,\rho),\;f'(x)=\sum_{n=1}^{+\infty}na_n(x-x_0)^{n-1}
	\end{equation*}
	在上述结论下使用数学归纳法即可证明关于任意阶导数的论断。
\end{proof}
