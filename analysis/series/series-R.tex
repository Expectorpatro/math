\section{数项级数}


\subsubsection{Cauchy积分判别法}
本节介绍Cauchy's\gls{IntTest}。
\begin{lemma}
	设$f(x)$在$[1,+\infty)$上非负。记
	\begin{equation*}
		F(x)=\int_1^xf(t)\dif t
	\end{equation*}
	则级数$\sum\limits_{n=1}^{+\infty}[F(n+1)-F(n)]$与广义积分$\int_1^{+\infty}f(x)\dif x$同收敛。
\end{lemma}
\begin{proof}
	(1)如果广义积分收敛,则:
	\begin{align*}
		\lim_{n\to+\infty}\sum_{i=1}^n[F(i+1)-F(i)]&=	\lim_{n\to+\infty}F(n+1) \\
		&=\lim_{n\to+\infty}\int_1^{n+1}f(x)\dif x \\
		&=\int_1^{+\infty}f(x)\dif x
	\end{align*}
	因此级数收敛。\par
	(2)如果级数收敛,对任意的$H>0$,令$N=[H]$(取整函数),则:
	\begin{align*}
		\int_1^{+\infty}f(x)\dif x
		&=\lim_{H\to+\infty}\int_1^{H}f(x)\dif x \\
		&\leqslant\lim_{H\to+\infty}\int_1^{N+1}f(x)\dif x \\
		&=\lim_{N\to+\infty}\int_1^{N+1}f(x)\dif x \\
		&=\lim_{N\to+\infty}\sum_{n=1}^{N}[F(n+1)-F(n)] \\
		&=\sum_{n=1}^{+\infty}[F(n+1)-F(n)]
	\end{align*}
	因此广义积分收敛。
\end{proof}
\begin{theorem}
	设$f(x)$在$[1,+\infty)$单调下降且非负,则级数:
	\begin{equation*}
		\sum_{n=1}^{+\infty}f(n)
	\end{equation*}
	与广义积分:
	\begin{equation*}
		\int_{1}^{+\infty}f(x)\dif x
	\end{equation*}
	同敛态。
\end{theorem}
\begin{proof}
	(1)如果广义积分收敛,则级数:
	\begin{equation*}
		\sum_{n=2}^{+\infty}[F(n)-F(n-1)]
	\end{equation*}
	收敛,由$f(x)$单调下降且非负,有:
	\begin{equation*}
		f(n)\leqslant\int_{n-1}^{n}f(x)\dif x=F(n)-F(n-1)
	\end{equation*}
	由比较判别法,级数收敛。\par
	(2)如果广义积分发散,则级数:
	\begin{equation*}
		\sum_{n=1}^{+\infty}[F(n+1)-F(n)]
	\end{equation*}
	发散,由$f(x)$单调下降且非负,有:
	\begin{equation*}
		f(n)\geqslant\int_{n}^{n+1}f(x)\dif x=F(n+1)-F(n)
	\end{equation*}
	由比较判别法,级数发散。
\end{proof}

\subsection{比较尺度问题}
由柯西积分判别法,我们可以很容易地判断以下级数是否收敛:
\begin{enumerate}
	\item $\sum\limits_{n=1}^{+\infty}\frac{1}{n^p}$。
	\item $\sum\limits_{n=2}^{+\infty}\frac{1}{n(\ln n)^p}$。
	\item $\sum\limits_{n=3}^{+\infty}\frac{1}{n\ln n(\ln\ln n)^p}$。
	\item $\cdots$
\end{enumerate}
上述级数在$p>1$时都收敛,反之发散。\par
我们会谈论比较尺度的问题,简单来讲就是说对于一个级数的敛散性问题,如果a判别法无法判别但b判别法可以,那么b的比较尺度应该是更加精细的。这个问题局限于判别法,但也不局限于判别法,如何理解呢?D'Alembert判别法和Raabe判别法的背后其实都是比较判别法,方法是一样的,但选取的比较级数不一样,也带来了它们比较尺度的不一样。上面这句话不是很直观,我们来看上面提到的由柯西积分法带来的三个级数(其实不止三个,按照换元积分法的规则,可以利用的级数能够无穷无尽地写下去)。可以看出,越往下写,通项在$n$相同时就会变得越大。也就是说它的尺度会变得更精细,之前无法判断为收敛的现在可以了。

\subsection{Raabe判别法}
以下定理我们称之为Raabe判别法。该判别法的实质其实就是将$\sum\limits_{n=1}^{+\infty}a_n$与$\sum\limits_{n=1}^{+\infty}\dfrac{1}{n^p}$作比较。当$p>1$时,由Cauchy积分判别法易证$\sum\limits_{n=1}^{+\infty}\dfrac{1}{n^p}$收敛。
\subsubsection{一般形式}
\begin{theorem}
	设$\sum\limits_{n=1}^{+\infty}a_n$是严格正项级数。
	\begin{enumerate}
		\item  如果$\exists\;q>1,\;\exists\;N\in\mathbb{N}^+$,使得对任意的$n>N$有:
		\begin{equation*}
			n\left(\frac{a_n}{a_{n+1}}-1\right)\geqslant q
		\end{equation*}
		则$\sum\limits_{n=1}^{+\infty}a_n$收敛。
		\item  如果$\exists\;N\in\mathbb{N}^+$,使得对任意的$n>N$有:
		\begin{equation*}
			n\left(\frac{a_n}{a_{n+1}}-1\right)\leqslant1
		\end{equation*}
		则$\sum\limits_{n=1}^{+\infty}a_n$发散。
	\end{enumerate}
\end{theorem}
\begin{proof}
	(1)所给的条件等价于对任意的$n>N$有:
	\begin{equation*}
		\frac{a_n}{a_{n+1}}\geqslant1+\frac{q}{n}
	\end{equation*}
	取$p\in\mathbb{R}$满足$1<p<q$,取级数$\sum\limits_{n=1}^{+\infty}\dfrac{1}{n^p}$,令$b_n=\frac{1}{n^p}$。当$n$足够大的时候,有:
	\begin{align*}
		\frac{b_n}{b_{n+1}}&=\frac{(n+1)^p}{n^p} \\
		&=(1+\frac{1}{n})^p \\
		&=1+\frac{p}{n}+O(\frac{1}{n^2}) \\
		&<1+\frac{q}{n}\leqslant\frac{a_n}{a_{n+1}}
	\end{align*}
	由比值判别法可得$\sum\limits_{n=1}^{+\infty}a_n$收敛。\par
	(2)所给的条件等价于对任意的$n>N$有:
	\begin{equation*}
		\frac{a_n}{a_{n+1}}\leqslant1+\frac{1}{n}=\frac{\frac{1}{n}}{\frac{1}{n+1}}
	\end{equation*}
	由比值判别法可得$\sum\limits_{n=1}^{+\infty}a_n$发散。
\end{proof}
\subsubsection{上下极限形式}
\begin{theorem}
	设$\sum\limits_{n=1}^{+\infty}a_n$是严格正项级数。
	\begin{enumerate}
		\item 如果:
		\begin{equation*}
			\varliminf n\left(\frac{a_{n}}{a_{n+1}}-1\right)>1
		\end{equation*}
		则$\sum\limits_{n=1}^{+\infty}a_n$收敛。
		\item 如果:
		\begin{equation*}
			\varlimsup n\left(\frac{a_{n}}{a_{n+1}}-1\right)<1
		\end{equation*}
		则$\sum\limits_{n=1}^{+\infty}a_n$发散。
	\end{enumerate}
\end{theorem}
\subsubsection{极限形式}
\begin{theorem}
	设$\sum\limits_{n=1}^{+\infty}a_n$是严格正项级数。且:
	\begin{equation*}
		\lim_{n\to+\infty} n\left(\frac{a_{n}}{a_{n+1}}-1\right)=q
	\end{equation*}
	\begin{enumerate}
		\item 若$q>1$,则$\sum\limits_{n=1}^{+\infty}a_n$收敛。
		\item 若$q<1$,则$\sum\limits_{n=1}^{+\infty}a_n$发散。
	\end{enumerate}
\end{theorem}

\subsection{Gauss判别法}
下面介绍Gauss判别法,它可以概括D'Alembert判别法与Raabe判别法,在比较尺度上能够达到$\sum\limits_{n=2}^{+\infty}\frac{1}{n(\ln n)^p}$的精度。
\begin{theorem}
	设$\sum\limits_{n=1}^{+\infty}a_n$是严格正项级数。若:
	\begin{equation*}
		\frac{a_n}{a_{n+1}}=\lambda+\frac{\mu}{n}+\frac{\nu}{n\ln n}+o(\frac{1}{n\ln n})
	\end{equation*}
	则:
	\begin{enumerate}
		\item 若$\lambda>1$,则级数$\sum\limits_{n=1}^{+\infty}a_n$收敛;若$\lambda<1$,则级数$\sum\limits_{n=1}^{+\infty}a_n$发散;
		\item 若$\lambda=1,\;\mu>1$,则级数$\sum\limits_{n=1}^{+\infty}a_n$收敛;若$\lambda=1,\;\mu<1$,则级数$\sum\limits_{n=1}^{+\infty}a_n$发散;
		\item 若$\lambda=1,\;\mu=1,\;\nu>1$,则级数$\sum\limits_{n=1}^{+\infty}a_n$收敛;若$\lambda=1,\;\mu=1,\;\nu<1$,则级数$\sum\limits_{n=1}^{+\infty}a_n$发散。
	\end{enumerate}
\end{theorem}
\begin{proof}
	(1)可以归结为D'Alembert判别法;(2)可以归结为Raabe判别法;\par
	(3)我们以级数:
	\begin{equation*}
		\sum\limits_{n=2}^{+\infty}b_n=\sum\limits_{n=2}^{+\infty}\frac{1}{n(\ln n)^p}
	\end{equation*}
	作为比较的尺度。计算可得:
	\begin{align*}
		\frac{b_n}{b_{n+1}}
		&=\frac{(n+1)(\ln(n+1))^p}{n(\ln n)^p} \\
		&=\left(1+\frac{1}{n}\right)\left(1+\frac{\ln(1+\frac{1}{n})}{\ln n}\right)^p \\
		&=\left(1+\frac{1}{n}\right)\left(1+\frac{\ln(1+\frac{1}{n})}{\ln n}\right)^p \\
		&=\left(1+\frac{1}{n}\right)\left(1+\frac{1}{n\ln n}+o(\frac{1}{n\ln n})\right)^p \\
		&=1+\frac{1}{n}+\frac{p}{n\ln n}+o\left(\frac{1}{n\ln n}\right)
	\end{align*}
	其中第三行到第四行利用了泰勒展开。\par
	回归原级数,如果$\lambda=\mu=1,\;\nu>1$,则可以选取$p\in\mathbb{R}$,使得:
	\begin{equation*}
		1<p<\nu
	\end{equation*}
	当$n$足够大的时候,就有:
	\begin{equation*}
		\frac{a_n}{a_{n+1}}\geqslant\frac{b_n}{b_{n+1}}
	\end{equation*}
	此时$\sum\limits_{n=1}^{+\infty}b_n$收敛,由比值判别法,级数$\sum\limits_{n=1}^{+\infty}a_n$收敛。\par
	如果$\lambda=\mu=1,\;\nu<1$,则可以选取$p\in\mathbb{R}$,使得:
	\begin{equation*}
		\nu<p<1
	\end{equation*}
	当$n$足够大的时候,就有:
	\begin{equation*}
		\frac{a_n}{a_{n+1}}\leqslant\frac{b_n}{b_{n+1}}
	\end{equation*}
	此时$\sum\limits_{n=1}^{+\infty}b_n$发散,由比值判别法,级数$\sum\limits_{n=1}^{+\infty}a_n$发散。
\end{proof}



