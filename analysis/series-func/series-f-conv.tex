\section{函数项级数的收敛}

\subsection{逐点收敛}
\subsubsection{函数项级数逐点收敛的定义}
\begin{definition}
	函数项级数$\sum\limits_{n=1}^{+\infty}f_n(x)$在集合$E$上逐点收敛于函数$f(x)$是指:
	\begin{equation*}
		\forall\;x\in E,\;\forall\;\varepsilon>0,\;\exists\; N(x,\varepsilon)\in\mathbb{N}^+,\;\forall\;n>N,\;\left|\sum_{i=1}^{n}f_i(x)-f(x)\right|<\varepsilon
	\end{equation*}
\end{definition}

\subsection{一致收敛}
\subsubsection{函数项级数一致收敛的定义}
\begin{definition}
	设函数项级数$\sum\limits_{n=1}^{+\infty}f_n(x)$在集合$E$上逐点收敛于函数$f(x)$。如果:
	\begin{equation*}
		\forall\;\varepsilon>0,\;\exists\; N(\varepsilon)\in\mathbb{N}^+,\;\forall\;n>N,\;\forall\;x\in E,\;\left|\sum_{i=1}^{n}f_i(x)-f(x)\right|<\varepsilon
	\end{equation*}
	则称函数项级数$\sum\limits_{n=1}^{+\infty}f_n(x)$在集合$E$上一致收敛于函数$f(x)$。
\end{definition}
\subsubsection{函数项级数一致收敛的柯西原理}
\begin{theorem}
	设函数项级数$\sum\limits_{n=1}^{+\infty}f_n(x)$在集合$E$上有定义,则这个函数序列在$E$上一致收敛于某个极限函数的充要条件是:
	\begin{equation*}
		\forall\;\varepsilon>0,\;\exists\;N\in\mathbb{N}^+,\;\forall\;n,m>N,\;m>n,\;\forall\;x\in E,\;\left|\sum_{i=n+1}^mf_i(x)\right|<\varepsilon
	\end{equation*}
\end{theorem}
\begin{proof}
	必要性:由三角不等式是显然的。\par
	充分性:由条件,对任意的$x\in E$,$\sum\limits_{i=1}^nf_i(x)$构成一个$\mathbb{R}$上的柯西序列,因此可定义一个极限函数$f(x)$使得:
	\begin{equation*}
		\forall\;x\in E,\;f(x)=\lim_{n\to+\infty}\sum_{i=1}^nf_i(x)
	\end{equation*}
	同时,由题目条件可得:
	\begin{equation*}
		\forall\;\varepsilon>0,\;\exists\;N\in\mathbb{N}^+,\;\forall\;n>N,\;\forall\;p\in\mathbb{N}^+,\;\forall\;x\in E,\;\left|\sum_{i=1}^{n+p}f_i(x)-\sum_{i=1}^{n}f_i(x)\right|<\varepsilon
	\end{equation*}
	在上式中取$p\to+\infty$即可得到:
	\begin{equation*}
		\forall\;\varepsilon>0,\;\exists\;N\in\mathbb{N}^+,\;\forall\;n>N,\;\forall\;x\in E,\;|f(x)-\sum_{i=1}^nf_i(x)|\leqslant\varepsilon
	\end{equation*}
	即函数项级数$\sum\limits_{n=1}^{+\infty}f_n(x)$一致收敛于$f(x)$。
\end{proof}
\begin{corollary}
	若函数项级数$\sum\limits_{n=1}^{+\infty}|f_n(x)|$在集合$E$上一致收敛,则$\sum\limits_{n=1}^{+\infty}f_n(x)$在集合$E$上也一致收敛。
\end{corollary}
\begin{proof}
	由三角不等式即可证得。
\end{proof}
请注意这里没有说二者一致收敛的对象是一样的!!!
\subsubsection{函数项级数绝对一致收敛的判别法}
\begin{theorem}[Weierstrass判别法]
	设函数项级数$\sum\limits_{n=1}^{+\infty}f_n(x)$在集合$E$上有定义。如果存在收敛的正项级数$\sum\limits_{n=1}^{+\infty}x_n$,满足:
	\begin{equation*}
		\forall\;n\in\mathbb{N}^+,\;\forall\;x\in E,\;|f_n(x)|\leqslant x_n
	\end{equation*}
	则函数项级数$\sum\limits_{n=1}^{+\infty}f_n(x)$在集合$E$上绝对一致收敛。称正项级数$\sum\limits_{n=1}^{+\infty}x_n$为优级数。
\end{theorem}
\begin{proof}
	因为数项级数$\sum\limits_{n=1}^{+\infty}x_n$收敛,所以:
	\begin{equation*}
		\forall\;\varepsilon>0,\;\exists\;N\in\mathbb{N}^+,\;\forall\;n,m>N,\;m>n,\;\left|\sum_{i=n+1}^mx_i\right|=\sum_{i=n+1}^mx_i<\varepsilon
	\end{equation*}
	也就有:
	\begin{equation*}
		\forall\;\varepsilon>0,\;\exists\;N\in\mathbb{N}^+,\;\forall\;n,m>N,\;m>n,\;\forall\;x\in E,\;\sum_{i=n+1}^m|f_n(x)|\leqslant\sum_{i=n+1}^mx_i<\varepsilon
	\end{equation*}
	由函数项级数收敛的柯西原理,$\sum\limits_{n=1}^{+\infty}|f_n(x)|$一致收敛,即$\sum\limits_{n=1}^{+\infty}f_n(x)$绝对一致收敛。
\end{proof}
\subsubsection{函数项级数条件一致收敛的判别法}
\begin{theorem}[Dirichlet判别法]
	对函数项级数:
	\begin{equation*}
		\sum_{n=1}^{+\infty}f_n(x)g_n(x),\quad x\in E
	\end{equation*}
	如果:
	\begin{enumerate}
		\item 函数序列$\{f_n(x)\}$对每个取定的$x\in E$都是单调的,并且该函数序列在$E$上一致地趋于$0$;
		\item 函数项级数$\sum\limits_{n=1}^{+\infty}g_n(x)$的部分和序列在$E$上一致有界:
		\begin{equation*}
			\forall\;n\in\mathbb{N}^+,\;\forall\;x\in E,\;\left|\sum_{i=1}^ng_i(x)\right|\leqslant L
		\end{equation*}
	\end{enumerate}
	则函数项级数$\sum\limits_{n=1}^{+\infty}f_n(x)g_n(x)$在$E$上条件一致收敛。
\end{theorem}
\begin{proof}
	利用Abel引理来估计:
	\begin{equation*}
		\left|\sum_{i=n+1}^mf_i(x)g_i(x)\right|
	\end{equation*}
	因为函数项级数$\sum\limits_{n=1}^{+\infty}g_n(x)$的部分和序列在$E$上一致有界,所以:
	\begin{equation*}
		\forall\;n,m\in\mathbb{N}^+,\;m>n,\;\forall\;x\in E,\;\left|\sum_{i=n+1}^mg_i(x)\right|\leqslant\left|\sum_{i=1}^ng_i(x)\right|+\left|\sum_{i=1}^mg_i(x)\right|\leqslant2L
	\end{equation*}
	又因函数序列$\{f_n(x)\}$对每个取定的$x\in E$都是单调的,由Abel引理:
	\begin{equation*}
		\forall\;n,m\in\mathbb{N}^+,\;m>n,\;\forall\;x\in E,\;\left|\sum_{i=n+1}^mf_i(x)g_i(x)\right|\leqslant2L(|f_{n+1}(x)|+2|f_m(x)|)
	\end{equation*}
	因为函数序列$\{f_n(x)\}$在$E$上一致地趋于$0$,所以:
	\begin{equation*}
		\forall\;\varepsilon>0,\;\exists\;N\in\mathbb{N}^+,\;\forall\;n>N,\;\forall\;x\in E,\;|f_n(x)|<\frac{\varepsilon}{6L}
	\end{equation*}
	于是有:
	\begin{equation*}
		\forall\;\varepsilon>0,\;\exists\;N\in\mathbb{N}^+,\;\forall\;n,m>N,\;m>n,\;\forall\;x\in E,\;\left|\sum_{i=n+1}^mf_i(x)g_i(x)\right|<\varepsilon
	\end{equation*}
	由函数项级数一致收敛的柯西原理,级数$\sum\limits_{n=1}^{+\infty}f_n(x)g_n(x)$在$E$上条件一致收敛。
\end{proof}
\begin{theorem}[Abel判别法]
	对函数项级数:
	\begin{equation*}
		\sum_{n=1}^{+\infty}f_n(x)g_n(x),\quad x\in E
	\end{equation*}
	如果:
	\begin{enumerate}
		\item 函数序列$\{f_n(x)\}$对每个取定的$x\in E$都是单调的,并且该函数序列在$E$上一致有界;
		\begin{equation*}
			\forall\;n\in\mathbb{N}^+,\;\forall\;x\in E,\;\left|f_n(x)\right|\leqslant M
		\end{equation*}
		\item 函数项级数$\sum\limits_{n=1}^{+\infty}g_n(x)$在$E$上一致收敛:
	\end{enumerate}
	则函数项级数$\sum\limits_{n=1}^{+\infty}f_n(x)g_n(x)$在$E$上条件一致收敛。
\end{theorem}
\begin{proof}
	因为函数项级数$\sum\limits_{n=1}^{+\infty}g_n(x)$在$E$上一致收敛,则:
	\begin{equation*}
		\forall\;\varepsilon'>0,\;\exists\;N\in\mathbb{N}^+,\;\forall\;n,m>N,\;m>n,\;\left|\sum_{i=n+1}^mg_i(x)\right|<\varepsilon'
	\end{equation*}
	又因函数序列$\{f_n(x)\}$对每个取定的$x\in E$都是单调的,由Abel引理:
	\begin{equation*}
		\forall\;x\in E,\;\left|\sum_{i=n+1}^mf_i(x)g_i(x)\right|\leqslant\varepsilon'(|f_{n+1}(x)|+2|f_m(x)|)\leqslant3M\varepsilon'
	\end{equation*}
	对任意的$\varepsilon>0$,可以选取$\varepsilon'>0$使得:
	\begin{equation*}
		3M\varepsilon'<\varepsilon\qedhere
	\end{equation*}
\end{proof}












