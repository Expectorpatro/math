\section{函数序列的收敛}

\subsection{逐点收敛}
\subsubsection{极限函数的定义}
\begin{definition}
	设$D\subset\mathbb{R}$。如果函数序列$\{f_n(x)\}$的收敛域包含了$D$,那么:
	\begin{equation*}
		\forall\;x\in D,\;\lim_{n\to+\infty}f_n(x)\in\mathbb{R}
	\end{equation*}
	可以定义一个函数:
	\begin{equation*}
		f(x)=\lim_{n\to+\infty}f_n(x),\;f:D\rightarrow\mathbb{R}
	\end{equation*}
	我们把函数$f$称之为函数序列$\{f_n(x)\}$在集合$D$上的\gls{LimitF}。
\end{definition}
\subsubsection{函数序列逐点收敛的定义}
\begin{definition}
	函数序列$\{f_n(x)\}$在集合$E$上\gls{PointwiseConvergence}于函数$f(x)$是指:
	\begin{equation*}
		\forall\;x\in E,\;\forall\;\varepsilon>0,\;\exists\; N(x,\varepsilon)\in\mathbb{N}^+,\;\forall\;n>N,\;|f_n(x)-f(x)|<\varepsilon
	\end{equation*}
\end{definition}

\subsection{一致收敛}
\subsubsection{函数序列一致收敛的定义}
\begin{definition}
	设函数序列$\{f_n(x)\}$在集合$E$上逐点收敛于函数$f(x)$。如果:
	\begin{equation*}
		\forall\;\varepsilon>0,\;\exists\; N(\varepsilon)\in\mathbb{N}^+,\;\forall\;n>N,\;\forall\;x\in E,\;|f_n(x)-f(x)|<\varepsilon
	\end{equation*}
	则称函数序列$\{f_n(x)\}$在集合$E$上\gls{UniformConvergence}于函数$f(x)$,记为:
	\begin{equation*}
		f_n(x)\underset{E}{\rightrightarrows}f(x)\quad(n\to+\infty)
	\end{equation*}
\end{definition}
\subsubsection{函数序列一致收敛的等价叙述}
\begin{theorem}
	设函数序列$\{f_n(x)\}$在集合$E$上逐点收敛于函数$f(x)$。记:
	\begin{equation*}
		\rho(f_n(x),f(x))=\sup_{x\in E}|f_n(x)-f(x)|
	\end{equation*}
	则以下三条陈述等价:
	\begin{enumerate}
		\item 函数序列$\{f_n(x)\}$在集合$E$上一致收敛于函数$f(x)$。
		\item $\lim\limits_{n\to+\infty}\rho(f_n(x),f(x))=0$。
		\item 对任何序列$\{x_n\}\subset E$,都有:
		\begin{equation*}
			\lim_{n\to+\infty}[f_n(x_n)-f(x_n)]=0
		\end{equation*}
	\end{enumerate}
\end{theorem}
\begin{proof}
	$(1)\rightarrow(2)$和$(2)\rightarrow(3)$是显然的。下证$(3)\rightarrow(1)$:\par
	假设(1)不成立,则
	\begin{equation*}
		\exists\;\varepsilon>0,\;\forall\;N(\varepsilon)\in\mathbb{N}^+,\;\exists\;n>N,\;\exists\;x\in E,\;|f_n(x)-f(x)|\geqslant\varepsilon
	\end{equation*}
	取一个固定地$\varepsilon>0$,令$n_0=0$,则$\exists\;n_k\in\mathbb{N}^+,\;n_k>n_{k-1}+1$,且$x_{n_k}\in E$,使得:
	\begin{equation*}
		|f_{n_k}(x_{n_k})-f(x_{n_k})|\geqslant\varepsilon
	\end{equation*}
	显然$\{x_{n_k}\}$这个序列不满足(3)的条件,矛盾。
\end{proof}
\subsubsection{函数序列一致收敛的柯西原理}
以上判别一个函数列是否一致收敛到一个极限函数的方法都需要提前求出极限函数,而以下柯西原理则不需要提前知道极限函数的形式:
\begin{theorem}
	设函数序列$\{f_n(x)\}$的各项在集合$E$上都有定义,则这个函数序列在$E$上一致收敛于某个极限函数的充要条件是:
	\begin{equation*}
		\forall\;\varepsilon>0,\;\exists\;N\in\mathbb{N}^+,\;\forall\;n,m>N,\;m>n,\;\forall\;x\in E,\;|f_m(x)-f_n(x)|<\varepsilon
	\end{equation*}
\end{theorem}
\begin{proof}
	必要性:由三角不等式是显然的。\par
	充分性:由条件,对任意的$x\in E$,$f_n(x)$构成一个$\mathbb{R}$上的柯西序列,因此可定义一个极限函数:
	\begin{equation*}
		f(x)=\lim_{n\to+\infty}f_n(x),\;\forall\;x\in E
	\end{equation*}
	同时,由题目条件可得:
	\begin{equation*}
		\forall\;\varepsilon>0,\;\exists\;N\in\mathbb{N}^+,\;\forall\;n>N,\;\forall\;p\in\mathbb{N}^+,\;\forall\;x\in E,\;|f_{n+p}(x)-f_n(x)|<\varepsilon
	\end{equation*}
	在上式中取$p\to+\infty$即可得到:
	\begin{equation*}
		\forall\;\varepsilon>0,\;\exists\;N\in\mathbb{N}^+,\;\forall\;n>N,\;\forall\;x\in E,\;|f(x)-f_n(x)|\leqslant\varepsilon
	\end{equation*}
	即$\{f_n(x)\}$一致收敛于$f(x)$。
\end{proof}