\section{幂级数}
本节考察如下形式的函数项级数(称之为\gls{PowerSeries}):
\begin{equation*}
	\sum_{n=1}^{+\infty}a_n(x-x_0)^n
\end{equation*}

\subsection{幂级数的收敛半径}
\begin{theorem}[Cauchy-Hadamard identity]
	对于幂级数$\sum\limits_{n=1}^{+\infty}a_n(x-x_0)^n$,记:
	\begin{equation*}
		\rho=\frac{1}{\varlimsup\sqrt[n]{|a_n|}}
	\end{equation*}
	则该幂级数对任意的$x\in\{x:|x-x_0|<\rho\}$绝对收敛,对任意的$x\in\{x:|x-x_0|>\rho\}$发散,称$\rho$为\gls{RadiusOfConvergence}。
\end{theorem}
\begin{proof}
	对于:
	\begin{equation*}
		\varlimsup\sqrt[n]{|a_n(x-x_0)^n|}=|x-x_0|\varlimsup\sqrt[n]{|a_n|}
	\end{equation*}
	由柯西根式判别法,当:
	\begin{equation*}
		|x-x_0|\varlimsup\sqrt[n]{|a_n|}<1
	\end{equation*}
	时,该幂级数绝对收敛。当:
	\begin{equation*}
		\varlimsup\sqrt[n]{|a_n(x-x_0)^n|}>1
	\end{equation*}
	时,有:
	\begin{equation*}
		\exists\;N\in\mathbb{N}^+,\;\forall\;n>N,\;|a_n(x-x_0)^n|>1
	\end{equation*}
	即该幂级数的项不收敛于$0$,不满足级数收敛的必要条件,所以该幂级数发散。
\end{proof}
\begin{theorem}
	幂级数的收敛半径也可由下式计算:
	\begin{equation*}
		\rho=\lim_{n\to+\infty}\left|\frac{a_n}{a_{n+1}}\right|
	\end{equation*}
\end{theorem}
\begin{proof}
	由比值判别法,当:
	\begin{equation*}
		\lim_{n\to+\infty}\left|\frac{a_{n+1}(x-x_0)^{n+1}}{a_n(x-x_0)^n}\right|=\lim_{n\to+\infty}\left|\frac{a_{n+1}(x-x_0)}{a_n}\right|<1
	\end{equation*}
	时,即:
	\begin{equation*}
		|x-x_0|<\lim_{n\to+\infty}\left|\frac{a_n}{a_{n+1}}\right|
	\end{equation*}
	时,该幂级数绝对收敛。当:
	\begin{equation*}
		\lim_{n\to+\infty}\left|\frac{a_{n+1}(x-x_0)^{n+1}}{a_n(x-x_0)^n}\right|=\lim_{n\to+\infty}\left|\frac{a_{n+1}(x-x_0)}{a_n}\right|>1
	\end{equation*}
	时,有:
	\begin{equation*}
		\exists\;N\in\mathbb{N}^+,\;\forall\;n>N,\;|a_{n+1}(x-x_0)^{n+1}|>|a_n(x-x_0)^n|
	\end{equation*}
	即该幂级数的项不收敛于$0$,不满足级数收敛的必要条件,所以该幂级数发散。
\end{proof}
\begin{theorem}
	设幂级数$\sum\limits_{n=1}^{+\infty}a_n(x-x_0)^n$的收敛半径为$\rho$,$[x_0-r,x_0+r]$是包含于$(x_0-\rho,x_0+\rho)$中的任何闭区间,则该幂级数在$[x_0-r,x_0+r]$上绝对一致收敛。
\end{theorem}
\begin{proof}
	在闭区间$[x_0-r,x_0+r]$上,取级数:
	\begin{equation*}
		\sum_{n=1}^{+\infty}|a_n|r^n
	\end{equation*}
	因为$r<\rho$,所以上正项级数收敛。因为:
	\begin{equation*}
		\forall\;n\in\mathbb{N}^+,\;\forall\;x\in[x_0-r,x_0+r],\;|a_n(x-x_0)^n|\leqslant|a_n|r^n
	\end{equation*}
	由Weierstrass判别法,该幂级数在$[x_0-r,x_0+r]$上绝对一致收敛。
\end{proof}

\subsection{幂级数和函数的分析性质}
\subsubsection{连续性}
\begin{theorem}
	设幂级数$\sum\limits_{n=1}^{+\infty}a_n(x-x_0)^n$的收敛半径为$\rho$,则和函数:
	\begin{equation*}
		f(x)=\sum_{n=1}^{+\infty}a_n(x-x_0)^n
	\end{equation*}
	在开区间$(x_0-\rho,x_0+\rho)$内处处连续。若它还在$x=x_0-\rho$处(或在$x=x_0+\rho$处)收敛,则该幂级数在闭区间$[x_0-\rho,x_0]$上(或在闭区间$[x_0,x_0+\rho]$上)一致收敛,所以和函数$f(x)$在$x=x_0-\rho$处右连续(或在$x=x_0+\rho$处左连续)。即:幂级数的和函数在幂级数收敛域的每一点都连续。
\end{theorem}
\begin{proof}
	任取$c\in(x_0-\rho,x_0+\rho)$,则必然$\exists\;r$使得$c\in[x_0-r,x_0+r]$,并且幂级数在$[x_0-r,x_0+r]$上绝对一致收敛。因为幂级数的每一项都是幂函数,所以每一项都在$[x_0-r,x_0+r]$上连续,于是和函数$f(x)$在$[x_0-r,x_0+r]$上连续,由此$f(x)$在$c$点连续。由$c$的任意性,$f(x)$在开区间$(x_0-\rho,x_0+\rho)$内处处连续。\par
	将幂级数写成如下形式:
	\begin{equation*}
		\sum_{n=1}^{+\infty}a_n\rho^n\left(\frac{x-x_0}{\rho}\right)^n
	\end{equation*}
	注意到:
	\begin{enumerate}
		\item 对于$x\in[x_0,x_0+\rho]$,函数序列$\left\{\left(\dfrac{x-x_0}{\rho}\right)^n\right\}$单调下降并且一致有界:
		\begin{equation*}
			1\geqslant\frac{x-x_0}{\rho}\geqslant\left(\frac{x-x_0}{\rho}\right)^2\geqslant\cdots\geqslant\left(\frac{x-x_0}{\rho}\right)^n\geqslant\cdots
		\end{equation*}
		\item 级数$\sum\limits_{n=1}^{+\infty}a_n\rho^n$收敛。
	\end{enumerate}
	由Abel判别法,可以判断幂级数在闭区间$[x_0,x_0+\rho]$上一致收敛。因为幂级数的每一项都是幂函数,所以每一项都在$[x_0,x_0+\rho]$上连续,于是和函数$f(x)$在$[x_0,x_0+\rho]$上连续,由此$f(x)$在$x_0+\rho$左连续。右连续的证明同理可得。综上,幂级数的和函数在幂级数收敛域的每一点都连续。
\end{proof}
\subsubsection{定积分}
\begin{theorem}
	幂级数在任何包含于收敛域的闭区间上都可以逐项积分。
\end{theorem}
\begin{proof}
	由幂级数和函数的连续性可直接得到。
\end{proof}
\subsubsection{微分}
\begin{theorem}
	由幂级数逐项求导所得到的级数:
	\begin{equation*}
		\sum_{n=1}^{+\infty}na_n(x-x_0)^{n-1}
	\end{equation*}
	与原级数有相同的收敛半径。
\end{theorem}
\begin{proof}
	由:
	\begin{equation*}
		\sum_{n=1}^{+\infty}na_n(x-x_0)^{n}=(x-x_0)\left[\sum_{n=1}^{+\infty}na_n(x-x_0)^{n-1}\right]
	\end{equation*}
	可以看出$\sum\limits_{n=1}^{+\infty}na_n(x-x_0)^{n}$与$\sum\limits_{n=1}^{+\infty}na_n(x-x_0)^{n-1}$的收敛域相同,所以它们的收敛半径相同。因为:
	\begin{equation*}
		\varlimsup\sqrt[n]{n|a_n|}=\varlimsup(\sqrt[n]{n}\cdot\sqrt[n]{|a_n|})=\lim\sqrt[n]{n}\cdot\varlimsup\sqrt[n]{|a_n|}=\varlimsup\sqrt[n]{|a_n|}
	\end{equation*}
	所以幂级数逐项求导所得到的级数与原级数有相同的收敛半径。
\end{proof}
\begin{theorem}
	在幂级数$\sum\limits_{n=1}^{+\infty}a_n(x-x_0)^n$收敛域的内部,和函数具有任意阶的导数,并且它的各阶导数可以通过级数逐项求导来计算。
\end{theorem}
\begin{proof}
	设幂级数$\sum\limits_{n=1}^{+\infty}a_n(x-x_0)^n$的收敛半径为$\rho$,那么幂级数$\sum\limits_{n=1}^{+\infty}na_n(x-x_0)^{n-1}$的收敛半径也是$\rho$。对任意的$c\in(x_0-\rho,x_0+\rho)$,可以选取$0<r<\rho$,使得$c\in(x_0-r,x_0+r)$。因为级数$\sum\limits_{n=1}^{+\infty}a_n(x-x_0)^n$和级数$\sum\limits_{n=1}^{+\infty}na_n(x-x_0)^{n-1}$在$[x_0-r,x_0+r]$上一致收敛,并且级数$\sum\limits_{n=1}^{+\infty}a_n(x-x_0)^n$的每一项都连续可微,所以和函数$f(x)=\sum\limits_{n=1}^{+\infty}a_n(x-x_0)^n$在$[x_0-r,x_0+r]$上连续可微,且有:
	\begin{equation*}
		f'(c)=\sum_{n=1}^{+\infty}na_nc^{n-1}
	\end{equation*}
	由$c$的任意性可得:
	\begin{equation*}
		\forall\;x\in(-\rho,\rho),\;f'(x)=\sum_{n=1}^{+\infty}na_n(x-x_0)^{n-1}
	\end{equation*}
	在上述结论下使用数学归纳法即可证明关于任意阶导数的论断。
\end{proof}
