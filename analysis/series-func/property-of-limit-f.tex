\section{极限函数的分析性质}

本节来讨论,一致收敛的函数序列或函数项级数需要满足怎样的条件,才能使得它们的极限函数拥有与它们一样的分析性质。在这里讨论的分析性质为:连续性、定积分、微分。

\subsection{连续性}
\subsubsection{极限函数的连续性}
\begin{theorem}
	设函数序列$\{f_n(x)\}$在区间$I$上一致收敛于函数$f(x)$。当函数序列$\{f_n(x)\}$的每一项在$I$上都连续时,$f(x)$在$I$上也连续。
\end{theorem}
\begin{proof}
	任取$x_0\in I,\;x\in I$,可得:
	\begin{equation*}
		|f(x)-f(x_0)|
		\leqslant|f(x)-f_n(x)|+|f_n(x)-f_n(x_0)|+|f_n(x_0)-f(x_0)|
	\end{equation*}
	因为$\{f_n(x)\}$在$I$上一致收敛于函数$f(x)$,所以:
	\begin{equation*}
		\forall\;\varepsilon>0,\;\exists\;N\in\mathbb{N}^+,\;\forall\;n>N,\;\forall\;x\in I,\;|f(x)-f_n(x)|<\varepsilon
	\end{equation*}
	于是:
	\begin{equation*}
		\exists\;N\in\mathbb{N}^+,\;\forall\;n>N,\;|f(x)-f_n(x)|<\varepsilon,\;|f_n(x_0)-f(x_0)|<\varepsilon
	\end{equation*}
	又因为$\{f_n(x)\}$的每一项都在$I$上连续,所以:
	\begin{equation*}
		\forall\;n\in\mathbb{N}^+,\;\forall\;\varepsilon>0,\;\exists\;\delta>0,\;\forall\;x\in\{x\in I:|x-x_0|<\delta\},\;|f_n(x)-f_n(x_0)|<\varepsilon
	\end{equation*}
	取$n\to+\infty$即可得到:
	\begin{equation*}
		\forall\;\varepsilon>0,\;\exists\;\delta>0,\;\forall\;x\in\{x:|x-x_0|<\delta\},\;|f(x)-f(x_0)|<\varepsilon
	\end{equation*}
	即$f(x)$在$x_0$处连续。由$x_0$的任意性,$f(x)$在$I$上连续。
\end{proof}
\begin{theorem}
	设函数项级数$\sum\limits_{n=1}^{+\infty}f_n(x)$在区间$I$上一致收敛于函数$f(x)$。当函数项级数$\sum\limits_{n=1}^{+\infty}f_n(x)$的每一项在$I$上连续时,$f(x)$在$I$上也连续。
\end{theorem}

\subsection{定积分}
\subsubsection{逐项积分定理}
\begin{theorem}
	设函数序列$\{f_n(x)\}$在区间$[a,b]$上一致收敛于函数$f(x)$。当函数序列$\{f_n(x)\}$的每一项在$[a,b]$上都连续时,有:
	\begin{equation*}
		\lim_{n\to+\infty}\left[\int_{a}^{b}f_n(x)\dif x\right]=\int_{a}^{b}f(x)\dif x
	\end{equation*}
\end{theorem}
\begin{proof}
	因为$\{f_n(x)\}$的每一项都在$[a,b]$上连续,所以$f(x)$也连续,于是$f(x)$可积。由积分中值定理:
	\begin{equation*}
		\exists\;\xi\in[a,b],\;\left|\int_{a}^{b}f_n(x)\dif x-\int_{a}^{b}f(x)\dif x\right|=\left|\int_{a}^{b}[f_n(x)-f(x)]\dif x\right|\leqslant(b-a)|f_n(\xi)-f(\xi)|
	\end{equation*}
	因为$\{f_n(x)\}$一致收敛于$f(x)$,所以对任意的$\varepsilon>0,\;\exists\;N\in\mathbb{N}^+,\;\forall\;n>N,\;|f_n(\xi)-f(\xi)|<\varepsilon$,此时即有:
	\begin{equation*}
		\left|\int_{a}^{b}f_n(x)\dif x-\int_{a}^{b}f(x)\dif x\right|<\varepsilon
	\end{equation*}
	也即:
	\begin{equation*}
		\lim_{n\to+\infty}\left[\int_{a}^{b}f_n(x)\dif x\right]=\int_{a}^{b}f(x)\dif x
	\end{equation*}
\end{proof}
\begin{theorem}
	设函数项级数$\sum\limits_{n=1}^{+\infty}f_n(x)$在区间$I$上一致收敛于函数$f(x)$。当函数项级数$\sum\limits_{n=1}^{+\infty}f_n(x)$的每一项在$I$上连续时,有:
	\begin{equation*}
		\sum_{n=1}^{+\infty}\left[\int_{a}^{b}f_n(x)\dif x\right]=\int_{a}^{b}f(x)\dif x
	\end{equation*}
\end{theorem}

\subsection{微分}
\subsubsection{逐项微分定理}
\begin{theorem}
	若函数序列$\{f_n(x)\}$满足:
	\begin{enumerate}
		\item $\{f_n(x)\}$的每一项在$[a,b]$上都连续可微;
		\item 导函数序列$\{f_n'(x)\}$在$[a,b]$上一致收敛于函数$\varphi(x)$;
		\item $\{f_n(x)\}$至少在某一点$x_0\in[a,b]$收敛:
		\begin{equation*}
			\lim_{n\to+\infty}f_n(x_0)=y_0\in\mathbb{R}
		\end{equation*}
	\end{enumerate}
	那么函数序列$\{f_n(x)\}$在$[a,b]$上一致收敛于某个在$[a,b]$连续可微的函数$f(x)$,并且有:
	\begin{equation*}
		f'(x)=\lim_{n\to+\infty}f_n'(x)=\varphi(x)
	\end{equation*}
\end{theorem}
\begin{proof}
	因为$\{f_n(x)\}$的每一项在$[a,b]$上都连续可微,所以导函数序列$\{f_n'(x)\}$的每一项在$[a,b]$上都连续,于是$\{f_n'(x)\}$的每一项在$[a,b]$上都可积,就有:
	\begin{equation}\label{eq:6.3.1}
		\forall\;x\in[a,b],\;\forall\;n\in\mathbb{N}^+,\;f_n(x)=f_n(x_0)+\int_{x_0}^{x}f_n'(\xi)\dif\xi
	\end{equation}
	由此可得:
	\begin{align*}
		|f_m(x)-f_n(x)|
		&\leqslant|f_m(x_0)-f_n(x_0)|+\left|\int_{x_0}^{x}[f_m'(\xi)-f_n'(\xi)]\dif\xi\right| \\
		&\leqslant|f_m(x_0)-f_n(x_0)|+(b-a)\sup_{\xi\in[a,b]}|f_m'(\xi)-f_n'(\xi)|
	\end{align*}
	因为$\{f_n(x)\}$在$x_0$处收敛、导函数序列$\{f_n'(x)\}$在$[a,b]$上一致收敛,所以:
	\begin{gather*}
		\forall\;\varepsilon>0,\;\exists\;N\in\mathbb{N}^+,\;\forall\;m,n>N,\;m>n \\
		|f_m(x_0)-f_n(x_0)|<\frac{\varepsilon}{2},\quad\sup_{\xi\in[a,b]}|f_m'(\xi)-f_n'(\xi)|<\frac{\varepsilon}{2(b-a)}
	\end{gather*}
	于是:
	\begin{equation*}
		\forall\;\varepsilon>0,\;\exists\;N\in\mathbb{N}^+,\;\forall\;m,n>N,\;m>n,\;\forall\;x\in[a,b],\;|f_m(x)-f_n(x)|<\varepsilon
	\end{equation*}
	由柯西收敛准则,$\{f_n(x)\}$一致收敛。设$\{f_n(x)\}$一致收敛的对象为$f(x)$,在\eqref{eq:6.3.1}中取$n\to+\infty$可得:
	\begin{equation*}
		f(x)=f(x_0)+\int_{x_0}^{x}\varphi(\xi)\dif\xi
	\end{equation*}
	对其求导即可得:
	\begin{equation*}
		f'(x)=\varphi(x)\qedhere
	\end{equation*}
\end{proof}
\begin{theorem}
	若函数项级数$\sum\limits_{n=1}^{+\infty}f_n(x)$满足:
	\begin{enumerate}
		\item $\sum\limits_{n=1}^{+\infty}f_n(x)$的每一项在$[a,b]$上都连续可微;
		\item 部分和序列的导函数序列$\left\{\sum\limits_{i=1}^{n}f'_n(x)\right\}$在$[a,b]$上一致收敛于函数$\varphi(x)$;
		\item $\sum\limits_{n=1}^{+\infty}f_n(x)$至少在某一点$x_0\in[a,b]$收敛:
		\begin{equation*}
			\lim_{n\to+\infty}\left[\sum_{i=1}^{n}f_i(x_0)\right]=y_0\in\mathbb{R}
		\end{equation*}
	\end{enumerate}
	那么函数项级数$\sum\limits_{n=1}^{+\infty}f_n(x)$在$[a,b]$上一致收敛于某个在$[a,b]$连续可微的函数$f(x)$,并且有:
	\begin{equation*}
		\left[\sum_{n=1}^{+\infty}f_n(x)\right]'=\sum_{n=1}^{+\infty}f_n'(x)=f'(x)=\varphi(x)
	\end{equation*}
\end{theorem}