\section{$\mathbb{C}^n$}

\begin{definition}
	记$\mathbb{C}^n$为所有$n$维复向量$(\xi_1,\xi_2,\dots,\xi_n)$构成的线性空间。
\end{definition}

\subsection{$\mathbb{C}^n$上的距离}
\begin{definition}
	在$\mathbb{C}^n$中定义向量$x=(\xi_1,\xi_2,\dots,\xi_n)$和向量$y=(\eta_1,\eta_2,\dots,\eta_n)$之间的距离为:
	\begin{equation*}
		\rho(x,y)=\left(\sum_{i=1}^n|\xi_i-\eta_i|^2\right)^{\frac{1}{2}}
	\end{equation*}
	则$(\mathbb{C}^n,\rho)$是一个度量空间。
\end{definition}
下证明上式定义的距离满足距离公理:
\begin{proof}
	(1)显然$\rho\in R$;(2)非负性直接可得;(3)对称性直接可得;\par
	(4)三角不等式:由\cref{ineq:cauchy-ineq-C},可得:
	\begin{align*}
		\sum_{i=1}^n|a_i+b_i|^2
		&=\sum_{i=1}^{n}(a_i+b_i)\overline{(a_i+b_i)} \\
		&=\sum_{i=1}^{n}(a_i+b_i)(\overline{a_i}+\overline{b_i}) \\
		&=\sum_{i=1}^{n}(a_i\overline{a_i}+a_i\overline{b_i}+b_i\overline{a_i}+b_i\overline{b_i}) \\
		&=\sum_{i=1}^{n}|a_i|^2+2\sum_{i=1}^{n}\operatorname{Re}(a_i\overline{b_i})+\sum_{i=1}^{n}|b_i|^2 \\
		&\leqslant\sum_{i=1}^n|a_i|^2+2\sum_{i=1}^n|a_ib_i|+\sum_{i=1}^n|b_i|^2 \\
		&\leqslant\sum_{i=1}^n|a_i|^2+2\left(\sum_{i=1}^n|a_i|^2\cdot\sum_{i=1}^n|b_i|^2\right)^{\frac{1}{2}}+\sum_{i=1}^n|b_i|^2 \\
		&=\left[\left(\sum_{i=1}^k|a_i|^2\right)^{\frac{1}{2}}+\left(\sum_{i=1}^k|b_i|^2\right)^{\frac{1}{2}}\right]^2
	\end{align*}
	设$x=(\xi_1,\xi_2,\dots,\xi_n),y=(\eta_1,\eta_2,\dots,\eta_n),z=(\zeta_1,\zeta_2,\dots,\zeta_n)$是$\mathbb{R}^n$中的任意三点,在上式中令$a_i=(\xi_i-\zeta_i),b_i=(\zeta_i-\eta_i)$,则
	\begin{equation*}
		\left[\sum_{i=1}^n|\xi_i-\eta_i|^2\right]^{\frac{1}{2}}\leqslant	\left[\sum_{i=1}^n|\xi_i-\zeta_i|^2\right]^{\frac{1}{2}}+\left[\sum_{i=1}^n|\zeta_i-\eta_i|^2\right]^{\frac{1}{2}}
	\end{equation*}
	即
	\begin{equation*}
		\rho(x,y)\leqslant\rho(x,z)+\rho(z,y)\qedhere
	\end{equation*}
\end{proof}
\subsubsection{收敛的含义}
\begin{theorem}
	$\mathbb{C}^{n}$在欧氏距离下的收敛等价于按坐标收敛。
\end{theorem}
\begin{proof}
	由下式可立即推出:
	\begin{equation*}
		\max_i|\xi_i-\eta_i|\leqslant	\left[\sum_{i=1}^n|\xi_i-\eta_i|^2\right]^{\frac{1}{2}}\leqslant\sum_{i=1}^{n}|\xi_i-\eta_i|\qedhere
	\end{equation*}
\end{proof}

\subsection{$\mathbb{C}^n$上的范数}
\begin{definition}
	在$\mathbb{C}^n$中定义元素$x=(\xi_1,\xi_2,\dots,\xi_n)$的范数为:
	\begin{equation*}
		||x||=\left(\sum_{i=1}^n|\xi_i|^2\right)^{\frac{1}{2}}
	\end{equation*}
	则$\mathbb{C}^n$成为一个赋范线性空间。
\end{definition}
下证明上式定义的范数满足范数定义:
\begin{proof}
	(1)$||x||\in\mathbb{R}$、(2)非负性和(3)数乘显然,(4)三角不等式的证明可见距离三角不等式的证明。
\end{proof}
\subsubsection{依范数收敛的含义}
\begin{theorem}
	$\mathbb{C}^{n}$中依范数收敛等价于按坐标收敛。
\end{theorem}
\begin{proof}
	依范数收敛等价于按距离收敛。
\end{proof}
