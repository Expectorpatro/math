\section{$l^p$}

\begin{definition}
	记$l^p(1\leqslant p<\infty)$为全体满足下列不等式的实或复数列$\{\xi_n\}$构成的集合:
	\begin{equation*}
		\sum_{n=1}^{\infty}|\xi_n|^p<\infty
	\end{equation*}
\end{definition}

\subsection{$l^p$上的距离}
\begin{definition}
	在$l^p$中定义元素$x=\{x_n\}$和元素$y=\{y_n\}$之间的距离为:
	\begin{equation*}
		\rho(x,y)=\left(\sum_{n=1}^{\infty}|x_n-y_n|^p\right)^\frac{1}{p}
	\end{equation*}
	则$(l^p,\rho)$是一个度量空间。
\end{definition}
下证明上式定义的距离满足距离公理:
\begin{proof}
	(1)$\rho\in R$:由\cref{ineq:else-1}可得:
	\begin{equation*}
		|x_n-y_n|^p\leqslant\Bigl(|x_n|+|y_n|\Bigr)^p\leqslant2^{p-1}\Bigl(|x_n|^p+|y_n|^p\Bigr)
	\end{equation*}
	于是:
	\begin{equation*}
		\rho(x,y)\leqslant\left[\sum_{n=1}^{+\infty}2^{p-1}\Bigl(|x_n|^p+|y_n|^p\Bigr)\right]^{\frac{1}{p}}\leqslant\left[2^{p-1}\sum_{n=1}^{+\infty}|x_n|^p+2^{p-1}\sum_{n=1}^{+\infty}|y_n|^p\right]^{\frac{1}{p}}
	\end{equation*}
	由$l^p$空间定义,$\rho\in\mathbb{R}$。
	(2)非负性直接可得;(3)对称性直接可得;
	(4)三角不等式:设$\{x_n\},\{y_n\},\{z_n\}$是任意的三个实或复有界数列。$p=1$时可由绝对值的三角不等式或模长的三角不等式立即得到,$p>1$时在Minkowski不等式(即\cref{ineq:minkowski-ineq-infty-series})
	\begin{equation*}
		\left(\sum_{i=1}^\infty|\xi_i+\eta_i|^p\right)^\frac{1}{p}\leqslant\left(\sum_{i=1}^\infty|\xi_i|^p\right)^\frac{1}{p}+\left(\sum_{i=1}^\infty|\eta_i|^p\right)^\frac{1}{p}
	\end{equation*}
	中取$\xi_i=x_i-z_i,\eta_i=z_i-y_i$即可立即得到。
\end{proof}

\subsection{$l^p$上的范数}
\begin{definition}
	在$l^p$中定义元素$x=\{x_n\}$的范数为:
	\begin{equation*}
		||x||=\left(\sum_{n=1}^{\infty}|x_n|^p\right)^\frac{1}{p}
	\end{equation*}
	则$l^p$成为一个赋范线性空间。
\end{definition}
下证明上式定义的范数满足范数定义:
\begin{proof}
	(1)$||x||\in\mathbb{R}$、(2)非负性和(3)数乘显然,(4)三角不等式的证明可由Minkowski不等式(即\cref{ineq:minkowski-ineq-infty-series})直接得到。
\end{proof}