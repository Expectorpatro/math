\section{$C[a,b]$}

\begin{definition}
	记$C[a,b]$为闭区间$[a,b]$上全体实值或复值连续函数构成的集合。
\end{definition}

\subsection{$C[a,b]$上的距离}
\begin{definition}
	在$C[a,b]$中定义元素$x=x(t)$和元素$y=y(t)$之间的距离为:
	\begin{equation*}
		\rho(x,y)=\max_{t\in[a,b]}|x(t)-y(t)|
	\end{equation*}
	则$(C[a,b],\rho)$是一个度量空间。
\end{definition}
下证明上式定义的距离满足距离公理:
\begin{proof}
	(1)$\rho\in R$可由三角不等式和连续函数的有界性推出;(2)非负性直接可得;(3)对称性直接可得;\par
	(4)三角不等式:设$x(t),y(t),z(t)$是$[a,b]$上任意的三个连续函数。由绝对值的三角不等式或模长的三角不等式:
	\begin{align*}
		|x(t)-y(t)|&\leqslant|x(t)-z(t)|+|z(t)-y(t)| \\
		&\leqslant\max_{t\in[a,b]}|x(t)-z(t)|+\max_{t\in[a,b]}|z(t)-y(t)| \\
		&=\rho(x,z)+\rho(z,y)
	\end{align*}
	因此:
	\begin{equation*}
		\rho(x,y)=\max_{t\in[a,b]}|x(t)-y(t)|\leqslant\rho(x,z)+\rho(z,y)\qedhere
	\end{equation*}
\end{proof}
\subsubsection{收敛的含义}
\begin{theorem}
	$C[a,b]$在上述距离下的收敛等价于一致收敛。
\end{theorem}
\begin{proof}
	证明太过简单,略去。
\end{proof}

\subsection{$C[a,b]$上的范数}
\begin{definition}
	在$C[a,b]$中定义元素$x=x(t)$的范数为:
	\begin{equation*}
		||x||=\max_{a\leqslant t\leqslant b}|x(t)|
	\end{equation*}
	则$C[a,b]$成为一个赋范线性空间。
\end{definition}
下证明上式定义的范数满足范数定义:
\begin{proof}
	(1)$||x||\in\mathbb{R}$、(2)非负性和(3)数乘显然,(4)三角不等式的证明可由最大值的加法运算得到。
\end{proof}
\subsubsection{依范数收敛的含义}
\begin{theorem}
	$C[a,b]$中依范数收敛等价于一致收敛。
\end{theorem}
\begin{proof}
	依范数收敛等价于按距离收敛。
\end{proof}


\subsection{性质}
\subsubsection{可分性}
\begin{theorem}
	$C[a,b]$是可分的。
\end{theorem}
\begin{proof}
	由伯恩斯坦定理,$[a,b]$上有理系数多项式的全体是$C[a,b]$的一个可列稠密子集。
\end{proof}
\subsubsection{完备性}
\begin{theorem}
	$C[a,b]$是完备的。
\end{theorem}
\begin{proof}
	设$\{x_n\}$是$C[a,b]$中的一个Cauchy点列,于是对任意的$\varepsilon>0$,存在$N>0$,当$m,n>N$时,有$\rho(x_m,x_n)<\varepsilon$,即$|x_m(t)-x_n(t)|$对$t\in[a,b]$一致的成立。由函数序列一致收敛的Cauchy原理与函数序列极限函数的连续性定理,$\{x_n\}$在$[a,b]$上一致收敛于某连续函数$x_0$,而$C[a,b]$中一致收敛与按距离收敛是等价的,所以$\{x_n\}\to x_0\in C[a,b]$,即$C[a,b]$完备。
\end{proof}