\section{无穷限积分}

\subsection{无穷限积分的定义}
\subsubsection{单侧无穷限}
\begin{definition}
	设函数$f$在$[a,+\infty]$上(或在$[-\infty,b]$上)有定义,并且对任意$H>a$(或$H'<b$),$f$在$[a,H]$上(或在$[H',b]$上)可积。如果存在有穷极限:
	\begin{equation*}
		\lim_{H\to+\infty}\int_{a}^{H}f(x)\dif x\quad
		\left(\lim_{H'\to-\infty}\int_{H'}^{b}f(x)\dif x\right)
	\end{equation*}
	则称$f$在$[a,+\infty]$(或$[-\infty,b]$)上广义可积,或者说无穷限积分$\int_{a}^{+\infty}f(x)\dif x$(或$\int_{-\infty}^{b}f(x)\dif x$)收敛,并记:
	\begin{equation*}
		\int_{a}^{+\infty}f(x)\dif x=\lim_{H\to+\infty}\int_{a}^{H}f(x)\dif x\quad
		\left(\int_{-\infty}^{b}f(x)\dif x=\lim_{H'\to-\infty}\int_{H'}^{b}f(x)\dif x\right)
	\end{equation*}
	若不存在有穷极限,则称无穷限积分$\int_{a}^{+\infty}f(x)\dif x$(或$\int_{-\infty}^{b}f(x)\dif x$)发散。
\end{definition}
\subsubsection{双侧无穷限}
\begin{definition}
	设函数$f$在$\mathbb{R}$上有定义。如果$\exists\;c\in\mathbb{R}$,使得下列两个积分:
	\begin{equation*}
		\int_{-\infty}^{c}f(x)\dif x\quad
		\int_{c}^{+\infty}f(x)\dif x
	\end{equation*}
	都收敛,则称积分$\int_{-\infty}^{+\infty}f(x)\dif x$收敛,并定义:
	\begin{equation*}
		\int_{-\infty}^{+\infty}f(x)\dif x=\int_{-\infty}^{c}f(x)\dif x+
		\int_{c}^{+\infty}f(x)\dif x
	\end{equation*}
\end{definition}
实际上,所定义的积分值并不依赖于$c$的选择。
\begin{proof}
	若对于某个$c$,$\int_{c}^{+\infty}f(x)\dif x$收敛,则对任意的$c'\in\mathbb{R}$,有:
	\begin{equation*}
		\int_{c'}^{+\infty}f(x)\dif x=\int_{c'}^{c}f(x)\dif x+\int_{c}^{+\infty}f(x)\dif x
	\end{equation*}
	收敛。同理$\int_{-\infty}^{c'}f(x)\dif x$收敛,并且有:
	\begin{equation*}
		\int_{-\infty}^{c'}f(x)\dif x=\int_{-\infty}^{c}f(x)\dif x+\int_{c}^{c'}f(x)\dif x
	\end{equation*}
	于是:
	\begin{align*}
		\int_{-\infty}^{c'}f(x)\dif x+\int_{c'}^{+\infty}f(x)\dif x
		&=\int_{-\infty}^{c}f(x)\dif x+\int_{c}^{c'}f(x)\dif x+\int_{c'}^{c}f(x)\dif x+\int_{c}^{+\infty}f(x)\dif x \\
		&=\int_{-\infty}^{c}f(x)\dif x+\int_{c}^{+\infty}f(x)\dif x\qedhere
	\end{align*}
\end{proof}
\subsubsection{双侧无穷限的柯西主值}
\begin{definition}
	如果极限:
	\begin{equation*}
		\lim_{H\to+\infty}\int_{-H}^{H}f(x)\dif x
	\end{equation*}
	存在,则称无穷限积分$\int_{-\infty}^{+\infty}f(x)\dif x$在柯西主值意义下收敛,称上述极限为无穷限积分$\int_{-\infty}^{+\infty}f(x)\dif x$的柯西主值,记为:
	\begin{equation*}
		VP\int_{-\infty}^{+\infty}f(x)\dif x=\lim_{H\to+\infty}\int_{-H}^{H}f(x)\dif x
	\end{equation*}
\end{definition}

\subsection{无穷限积分的计算}
\subsubsection{单侧无穷限}
\begin{theorem}
	设函数$f$在$[a,+\infty)$上(或在$(-\infty,b]$上)有定义并且连续,而函数$F$是$f$在$[a,+\infty)$上的(或在$(-\infty,b]$上的)原函数,如果存在(有穷或无穷的)极限:
	\begin{equation*}
		F(+\infty)=\lim_{x\to+\infty}F(x)\quad
		\left(F(-\infty)=\lim_{x\to-\infty}F(x)\right)
	\end{equation*}
	那么就有:
	\begin{gather*}
		\int_{a}^{+\infty}f(x)\dif x=F(+\infty)-F(a)=F(x)\Big|_a^{+\infty} \\
		\left[\int_{-\infty}^{b}f(x)\dif x=F(b)-F(-\infty)=F(x)\Big|_{-\infty}^b\right]
	\end{gather*}
\end{theorem}
\begin{proof}
	对任意$H>a$(或$H'<b$),有:
	\begin{equation*}
		\int_{a}^{H}f(x)\dif x=F(H)-F(a)\quad
		\left(\int_{H'}^{b}f(x)\dif x=F(b)-F(H')\right)
	\end{equation*}
	取$H\to+\infty$(或$H'\to-\infty$)可得:
	\begin{equation*}
		\lim_{H\to+\infty}\int_{a}^{H}f(x)\dif x=\lim_{H\to+\infty}F(H)-F(a)\quad
		\left[\lim_{H'\to-\infty}\int_{H'}^{b}f(x)\dif x=F(b)-\lim_{H'\to-\infty}F(H')\right]
	\end{equation*}
	即:
	\begin{gather*}
		\int_{a}^{+\infty}f(x)\dif x=F(+\infty)-F(a)=F(x)\Big|_a^{+\infty} \\
		\left[\int_{-\infty}^{b}f(x)\dif x=F(b)-F(-\infty)=F(x)\Big|_{-\infty}^b\right]
	\end{gather*}
\end{proof}
\subsubsection{双侧无穷限}
\begin{theorem}
	设函数$f$在$\mathbb{R}$上有定义并且连续,而函数$F$是$f$在$\mathbb{R}$上的原函数,如果存在极限:
	\begin{equation*}
		F(+\infty)=\lim_{x\to+\infty}F(x)\quad
		F(-\infty)=\lim_{x\to-\infty}F(x)
	\end{equation*}
	且$F(+\infty)-F(-\infty)$有意义,那么就有:
	\begin{equation*}
		\int_{-\infty}^{+\infty}f(x)\dif x=F(+\infty)-F(-\infty)=F(x)\Big|_{-\infty}^{+\infty}
	\end{equation*}
\end{theorem}
\begin{proof}
	由双侧无穷限积分的定义,$\exists\;c\in\mathbb{R}$使得:
	\begin{align*}
		\int_{-\infty}^{+\infty}f(x)\dif x
		&=\int_{-\infty}^{c}f(x)\dif x+\int_{c}^{+\infty}f(x)\dif x \\
		&=F(c)-F(-\infty)+F(+\infty)-F(c) \\
		&=F(x)\Big|_{-\infty}^{+\infty}\qedhere
	\end{align*}
\end{proof}

\subsection{无穷限积分的收敛原理}
仅讨论$[a,+\infty)$上的情况。
\begin{theorem}
	无穷限积分$\int_{a}^{+\infty}f(x)\dif x$收敛的充要条件为:
	\begin{equation*}
		\forall\;\varepsilon>0,\;\exists\;\Delta>0,\;\forall\;H'\geqslant H>\Delta,\;\left|\int_{H}^{H'}f(x)\dif x\right|<\varepsilon
	\end{equation*}
\end{theorem}
\begin{proof}
	按照无穷限积分$\int_{a}^{+\infty}f(x)\dif x$的定义,它是函数:
	\begin{equation*}
		\Phi(H)=\int_{a}^{H}f(x)\dif x
	\end{equation*}
	在$H\to+\infty$时的极限,由函数极限的收敛原理即可得到无穷限积分的收敛原理。
\end{proof}
\subsubsection{绝对收敛与条件收敛}
我们注意到,当无穷限积分$\int_{a}^{+\infty}|f(x)|\dif x$收敛时,无穷限积分$\int_{a}^{+\infty}f(x)\dif x$必定也收敛,这点可以由无穷限积分的收敛原理来证明。
\begin{definition}
	如果无穷限积分$\int_{a}^{+\infty}|f(x)|\dif x$收敛,则称无穷限积分$\int_{a}^{+\infty}f(x)\dif x$绝对收敛;如果无穷限积分$\int_{a}^{+\infty}|f(x)|\dif x$发散,但无穷限积分$\int_{a}^{+\infty}f(x)\dif x$收敛,则称无穷限积分$\int_{a}^{+\infty}f(x)\dif x$条件收敛。
\end{definition}

\subsection{无穷限积分敛散性的判别法}
仅讨论$[a,+\infty)$上的情况。
\subsubsection{绝对收敛}
\begin{theorem}
	设函数$f(x)$和$g(x)$在区间$[a,+\infty)$上有定义,在其任何闭子区间$[a,H]$上可积,并且有:
	\begin{equation*}
		\exists\;\Delta>a,\;\forall\;x\in[\Delta,+\infty),\;|f(x)|\leqslant g(x)
	\end{equation*}
	若无穷限积分$\int_{a}^{+\infty}g(x)\dif x$收敛,那么无穷限积分$\int_{a}^{+\infty}f(x)\dif x$绝对收敛。
\end{theorem}
\begin{proof}
	因为非负函数$g(x)$的无穷限积分$\int_{a}^{+\infty}g(x)\dif x$收敛,由无穷限积分的收敛原理可得:
	\begin{equation*}
		\forall\;\varepsilon>0,\;\exists\;\Delta>0,\;\forall\;H'\geqslant H>\Delta,\;\int_{H}^{H'}g(x)\dif x<\varepsilon
	\end{equation*}
	于是:
	\begin{equation*}
		\forall\;\varepsilon>0,\;\exists\;\Delta>0,\;\forall\;H'\geqslant H>\Delta,\;\int_{H}^{H'}|f(x)|\dif x\int_{H}^{H'}g(x)\dif x<\varepsilon
	\end{equation*}
	由收敛原理,无穷限积分$\int_{a}^{+\infty}|f(x)|\dif x$收敛,即$\int_{a}^{+\infty}f(x)\dif x$绝对收敛。
\end{proof}
\subsubsection{条件收敛}
\begin{theorem}[Dirichlet判别法]
	设函数$f(x)$和$g(x)$在区间$[a,+\infty)$上有定义,在其任何闭子区间$[a,H]$上可积,并且有:
	\begin{enumerate}
		\item $\exists\;\Delta>a$使得$f(x)$在$[\Delta,+\infty)$上单调,同时:
		\begin{equation*}
			\lim_{x\to+\infty}f(x)=0
		\end{equation*}
		\item $\exists\;K\geqslant0$使得:
		\begin{equation*}
			\forall\;x\in[a,+\infty),\;\Big|\int_{a}^{H}g(x)\dif x\Big|\leqslant K
		\end{equation*}
	\end{enumerate}
	那么积分:
	\begin{equation*}
		\int_{a}^{+\infty}f(x)g(x)\dif x
	\end{equation*}
	收敛。
\end{theorem}
\begin{proof}
	对充分大的$H$和$H'$,我们来估计:
	\begin{equation*}
		\left|\int_{H}^{H'}f(x)g(x)\dif x\right|
	\end{equation*}
	因为$f(x)$在$[\Delta,+\infty)$上单调,所以当$H$和$H'$充分大时,$f(x)$在$[H,H']$上单调,又因为$g(x)$在$[H,H']$上可积,由第二中值定理可得:
	\begin{equation*}
		\exists\;\xi\in[H,H'],\;\int_{H}^{H'}f(x)g(x)\dif x=f(H)\int_{H}^{\xi}g(x)\dif x+f(H')\int_{\xi}^{H'}g(x)\dif x
	\end{equation*}
	由$g(x)$积分的有界性可得:
	\begin{equation*}
		\left|\int_{H}^{\xi}g(x)\dif x\right|=\left|\int_{a}^{\xi}g(x)\dif x-\int_{a}^{H}g(x)\dif x\right|\leqslant\left|\int_{a}^{\xi}g(x)\dif x\right|+\left|\int_{a}^{H}g(x)\dif x\right|\leqslant2K
	\end{equation*}
	同理可得:
	\begin{equation*}
		\left|\int_{\xi}^{H'}g(x)\dif x\right|\leqslant2K
	\end{equation*}
	于是:
	\begin{align*}
		\left|\int_{H}^{H'}f(x)g(x)\dif x\right|
		&=\left|f(H)\int_{H}^{\xi}g(x)\dif x+f(H')\int_{\xi}^{H'}g(x)\dif x\right| \\
		&=\left|f(H)\int_{H}^{\xi}g(x)\dif x\right|+\left|f(H')\int_{\xi}^{H'}g(x)\dif x\right| \\
		&=|f(H)|\left|\int_{H}^{\xi}g(x)\dif x\right|+|f(H')|\left|\int_{\xi}^{H'}g(x)\dif x\right| \\
		&\leqslant2K[|f(H)|+|f(H')|]
	\end{align*}
	因为:
	\begin{equation*}
		\lim_{x\to+\infty}f(x)=0
	\end{equation*}
	所以:
	\begin{equation*}
		\forall\;\varepsilon>0,\;\exists\;M>0,\;\forall\;H,H'>M,\;|f(H)|<\frac{\varepsilon}{4K},\;|f(H')|<\frac{\varepsilon}{4K}
	\end{equation*}
	于是:
	\begin{equation*}
		\forall\;\varepsilon>0,\;\exists\;M>0,\;\forall\;H,H'>M,\;\left|\int_{H}^{H'}f(x)g(x)\dif x\right|<\varepsilon
	\end{equation*}
	由无穷限积分的收敛原理,$\int_{a}^{+\infty}f(x)g(x)\dif x$收敛。
\end{proof}
\begin{theorem}[Abel判别法]
	设函数$f(x)$和$g(x)$在区间$[a,+\infty)$上有定义,在其任何闭子区间$[a,H]$上可积,并且有:
	\begin{enumerate}
		\item $\exists\;\Delta>a$使得$f(x)$在$[\Delta,+\infty)$上单调并且有界;
		\item $\int_{a}^{+\infty}g(x)\dif x$收敛。
	\end{enumerate}
	那么积分:
	\begin{equation*}
		\int_{a}^{+\infty}f(x)g(x)\dif x
	\end{equation*}
	收敛。
\end{theorem}
\begin{proof}
	因为$f(x)$在$[\Delta,+\infty)$上单调有界,所以$f(x)$有有穷的极限:
	\begin{equation*}
		\lim_{x\to+\infty}f(x)=l
	\end{equation*}
	那么函数$f(x)-l$在$[\Delta,+\infty)$上单调趋于$0$。因为$\int_{a}^{+\infty}g(x)\dif x$收敛,所以$\exists\;K\geqslant0$使得:
	\begin{equation*}
		\forall\;x\in[a,+\infty),\;\Big|\int_{a}^{H}g(x)\dif x\Big|\leqslant K
	\end{equation*}
	由Dirichlet判别法,无穷限积分:
	\begin{equation*}
		\int_{a}^{+\infty}[f(x)-l]g(x)\dif x
	\end{equation*}
	收敛。因为$\int_{a}^{+\infty}g(x)\dif x$收敛,所以$\int_{a}^{+\infty}lg(x)\dif x$收敛,于是:
	\begin{equation*}
		\int_{a}^{+\infty}f(x)g(x)\dif x=\int_{a}^{+\infty}[f(x)-l]g(x)\dif x+\int_{a}^{+\infty}lg(x)\dif x
	\end{equation*}
	收敛。
\end{proof}