\section{瑕积分}
\begin{definition}
	若函数$f$在$a$点邻近无界,则称$a$为$f$的瑕点。
\end{definition}

\subsection{瑕积分的定义}
\subsubsection{瑕点单侧}
\begin{definition}
	设函数$f$在$[a,b)$上(或在$(a,b]$上)有定义,并设对任何的$0<\eta<b-a$,$f$在$[a,b-\eta]$上(或在$[a+\eta,b]$上)可积,如果存在有穷极限:
	\begin{equation*}
		\lim_{\eta\to0+}\int_{a}^{b-\eta}f(x)\dif x\quad
		\left(\lim_{\eta\to0+}\int_{a+\eta}^{b}f(x)\dif x\right)
	\end{equation*}
	则称$f$在$[a,b)$上(或在$(a,b]$上)广义可积,或者说积分$\int_{a}^{b}f(x)\dif x$收敛,并定义:
	\begin{equation*}
		\int_{a}^{b}f(x)\dif x=\lim_{\eta\to0+}\int_{a}^{b-\eta}f(x)\dif x\quad
		\left(\int_{a}^{b}f(x)\dif x=\lim_{\eta\to0+}\int_{a+\eta}^{b}f(x)\dif x\right)
	\end{equation*}
\end{definition}
\subsubsection{瑕点双侧}
\begin{definition}
	设$a<c<b$,函数$f$在$[a,c)$和$(c,b]$上有定义,并且对任何$0<\eta<c-a,\;0<\eta'<b-c$,$f$在$[a,c-\eta]$和$[c+\eta',b]$上都可积。如果下列两个积分:
	\begin{equation*}
		\int_{a}^{c}f(x)\dif x\quad\int_{c}^{b}f(x)\dif x
	\end{equation*}
	都收敛,则称瑕积分$\int_{a}^{b}f(x)\dif x$收敛,并定义:
	\begin{equation*}
		\int_{a}^{b}f(x)\dif x=\int_{a}^{c}f(x)\dif x+\int_{c}^{b}f(x)\dif x
	\end{equation*}
\end{definition}
\subsubsection{瑕点双侧的柯西主值}
\begin{definition}
	如果极限:
	\begin{equation*}
		\lim_{\eta\to0+}\left(\int_{a}^{c-\eta}f(x)\dif x+\int_{c+\eta}^{b}f(x)\dif x\right)
	\end{equation*}
	存在,则称瑕积分$\int_{a}^{b}f(x)\dif x$在柯西主值意义下收敛,称上述极限为瑕积分$\int_{a}^{b}f(x)\dif x$的柯西主值,记为:
	\begin{equation*}
		VP\int_{a}^{b}f(x)\dif x=\lim_{\eta\to0+}\left(\int_{a}^{c-\eta}f(x)\dif x+\int_{c+\eta}^{b}f(x)\dif x\right)
	\end{equation*}
\end{definition}

\subsection{瑕积分的计算}

\subsection{瑕积分的收敛原理}
仅讨论$[a,b)$上的情况。\par
\begin{theorem}
	瑕积分$\int_{a}^{b}f(x)\dif x$收敛的充要条件为:
	\begin{equation*}
		\forall\;\varepsilon>0,\;\exists\;\delta>0,\;\forall\;\eta,\eta',\;0<\eta'<\eta<\delta,\;\left|\int_{b-\eta}^{b-\eta'}f(x)\dif x\right|<\varepsilon
	\end{equation*}
\end{theorem}
\begin{proof}
	按照瑕积分$\int_{a}^{b}f(x)\dif x$的定义,它是函数:
	\begin{equation*}
		\Phi(\eta)=\int_{a}^{b-\eta}f(x)\dif x
	\end{equation*}
	在$\eta\to0+$时的极限,由函数极限的收敛原理即可得到无穷限积分的收敛原理。
\end{proof}
\subsubsection{绝对收敛与条件收敛}
我们注意到,当瑕积分$\int_{a}^{b}|f(x)|\dif x$收敛时,瑕积分$\int_{a}^{b}f(x)\dif x$必定也收敛,这点可以由瑕积分的收敛原理来证明。
\begin{definition}
	如果瑕积分积分$\int_{a}^{b}|f(x)|\dif x$收敛,则称瑕积分$\int_{a}^{b}f(x)\dif x$绝对收敛;如果瑕积分$\int_{a}^{b}|f(x)|\dif x$发散,但瑕积分$\int_{a}^{b}f(x)\dif x$收敛,则称瑕积分$\int_{a}^{b}f(x)\dif x$条件收敛。
\end{definition}

\subsection{瑕积分敛散性的判别法}
仅讨论$[a,b)$上的情况。
\subsubsection{绝对收敛}
\begin{theorem}
	设函数$f(x)$和$g(x)$在区间$[a,b)$上有定义,在其任何闭子区间$[a,b-\eta]$上可积,并且有:
	\begin{equation*}
		\exists\;\delta>a,\;\forall\;x\in[b-\delta,b),\;|f(x)|\leqslant g(x)
	\end{equation*}
	若瑕积分$\int_{a}^{b}g(x)\dif x$收敛,那么瑕积分$\int_{a}^{b}f(x)\dif x$绝对收敛。
\end{theorem}