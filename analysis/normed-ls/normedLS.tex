\section{范数}
\subsubsection{范数的定义}
\begin{definition}
	设$X$是实或者复线性空间,如果对于$X$中的每个元素$x$,都有一个实数与之对应,记为$||x||$,且满足:
	\begin{enumerate}
		\item 非负性:$||x||\geqslant 0$,等号成立当且仅当$x=\mathbf{0}$。
		\item 数乘:$||\alpha x||=|\alpha|\;||x||$,$\alpha\in\mathbb{C}$或$\mathbb{R}$。
		\item 三角不等式:$||x+y||\leqslant||x||+||y||$。
	\end{enumerate}
	则称$X$为实或复的\gls{NormedLS},$||x||$为元素$x$的\gls{norm}。
\end{definition}
\subsubsection{赋范线性空间中的距离}
\begin{definition}
	对于赋范线性空间$X$,我们定义下式来衡量$X$中元素$x$和$y$之间的距离:
	\begin{equation*}
		\rho(x,y)=||x-y||
	\end{equation*}
\end{definition}
下验证它的确符合距离的定义:
\begin{proof}
	(1)非负性可由范数的非负性直接验证。\par
	(2)对称性:由范数定义中的条件(2)可得$\rho(x,y)=||x-y||=|-1|\;||x-y||=||y-x||=\rho(y,x)$。\par
	(3)三角不等式:由范数定义中的条件(3)可得$\rho(x,y)=||x-y||=||x-z+z-y||\leqslant||x-z||+||z-y||=\rho(x,z)+\rho(z,y)$。
\end{proof}
\subsubsection{赋范线性空间中点列的收敛}
\begin{definition}
	赋范线性空间$X$中,若点列$\{x_n\}$收敛于点$x$,则称$\{x_n\}$\gls{convergenceNorm}于$x$,也称$\{x_n\}$\gls{Strongconvergence}于$x$。
\end{definition}
\subsubsection{范数的性质}
\begin{lemma}\label{lem:norm-uneq}
	对于赋范线性空间$X$,$x,y\in X$,有:
	\begin{equation*}
		|\;||x||-||y||\;|\leqslant||x-y||
	\end{equation*}
\end{lemma}
\begin{proof}
	$||x||=||x-y+y||\leqslant||x-y||+||y||$,即$||x||-||y||\leqslant||x-y||$;$||y||=||y-x+x||\leqslant||x-y||+||x||$,即$-||x-y||\leqslant||x||-||y||$。\par
	综上,$-||x-y||\leqslant||x||-||y||\leqslant||x-y||$,即$|\;||x||-||y||\;|\leqslant||x-y||$。
\end{proof}
\begin{property}
	(1)范数是连续泛函。
	(2)依范数收敛满足线性运算法则。 
\end{property}
\begin{proof}
	(1)由\cref{lem:norm-uneq},$|\;||x_n||-||x||\;|\leqslant||x_n-x||$。因此当$\{x_n\}$依范数收敛于$x$时,$||x_n||\to||x||$。\par
	(2)设$\{x_n\}$和$\{y_n\}$都是赋范线性空间$X$中的点列,且$\{x_n\}\rightarrow x$,$\{y_n\}\rightarrow y$,$\{a_n\}$和$\{b_n\}$是$\mathbb{R}^{n}$或$\mathbb{C}^{n}$中的点列,且$\{a_n\}\to a,\;\{b_n\}\to b$,由范数的定义:
	\begin{align*}
		||a_nx_n+b_ny_n-(ax+by)||
		&\leqslant||a_nx_n-ax||+||b_ny_n-by|| \\
		&=||a_nx_n-a_nx+a_nx-ax||+||b_ny_n-b_ny+b_ny-by|| \\
		&\leqslant|a_n|\;||x_n-x||+|a_n-a|\;||x||+|b_n|\;||y_n-y||+|b_n-b|\;||y||
	\end{align*}
	由$\{a_n\}$和$\{b_n\}$的收敛性以及$\{x_n\}$和$\{y_n\}$的收敛性立即可得$\{a_nx_n+b_ny_n\}\to ax+bx$。
\end{proof}
