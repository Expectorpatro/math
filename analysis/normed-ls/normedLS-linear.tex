\section{赋范线性空间作为线性空间的性质}

\subsection{直和}
\begin{theorem}
	设$L_1,L_2,\dots,L_n$都是赋范线性空间,$X$是$L_1,L_2,\dots,L_n$的直和,即:
	\begin{equation*}
		X=L_1\oplus L_2\oplus\cdots\oplus L_n
	\end{equation*}
	可在$X$中定义如下范数:
	\begin{gather*}\label{norm:normedLS-directSum}
		||x||=||x_1||+||x_2||+\cdots+||x_n|| \\
		||x||_1=\max_i||x_i|| \\
		||x||_2=\left(\sum_{i=1}^n||x_i||^2\right)^\frac{1}{2}
	\end{gather*}
	这里$x=\sum\limits_{i=1}^{n}x_i\in X,\;x_i\in L_i(i=1,2,\dots,n)$。$X$按照这些范数都构成赋范线性空间。
\end{theorem}
\begin{proof}
	证明略去,太过简单。
\end{proof}
\begin{theorem}
	如果$L_1,L_2,\dots,L_n$都是Banach空间,$X$是$L_1,L_2,\dots,L_n$的直和,则$X$按照\cref{norm:normedLS-directSum}也成为Banach空间。
\end{theorem}
\begin{proof}
	(1)对于范数$||x||$,在$X$中任取Cauchy点列$\{x_m\}$,记:
	\begin{equation*}
		x_m=\sum_{i=1}^{n}y_m^{i},\quad y_m^{i}\in L_i,\;m\in\mathbb{N}^+,\;i=1,2,\dots,n
	\end{equation*}
	因为$\{x_m\}$是Cauchy点列,所以对于$\forall\;\varepsilon>0,\;\exists\;N\in\mathbb{N}^+,\;\forall\;j,k>N$,有:
	\begin{align*}
		||x_j-x_k||
		&=\left\|\sum_{i=1}^{n}(y_j^i-y_k^i)\right\| \\
		&=\sum_{i=1}^{n}||y_j^i-y_k^i||<\varepsilon
	\end{align*}
	于是$||y_j^i-y_k^i||<\varepsilon$,即$\{y_m^i\}$也构成Cauchy点列,其中$i=1,2,\dots,n$。因为$L_1,L_2,\dots,L_n$都是Banach空间,同时$\{y_m^i\}\subset L_i$,所以存在$y_i\in L_i$,使得$\{y_m^i\}\to y_i$。令$y=\sum\limits_{i=1}^{n}y_i$,显然$y\in X$。则:
	\begin{equation*}
		x_m-y=\sum_{i=1}^{n}(y_m^i-y_i)\to 0\quad(m\to +\infty)
	\end{equation*}
	即$\{x_m\}\to y\in X$。由$\{x_m\}$的任意性,$X$是完备的,所以$X$是一个Banach空间。\par
	(2)与(1)几乎完全一样,只是由$\max\limits_i||y_j^i-y_k^i||<\varepsilon$导出$\{y_m^i\}$是Cauchy点列。\par
	(3)由:
	\begin{equation*}
		\sqrt{\sum_{i=1}^{n}||y_j^i-y_k^i||^2}\leqslant\sum_{i=1}^{n}||y_j^i-y_k^i||
	\end{equation*}
	可导出$\{y_m^i\}$是Cauchy点列。
\end{proof}

\subsection{商空间}
\begin{definition}
	当$L$是$X$的闭子空间时\info{思考为什么一定得是闭子空间},可在商空间$X/L$中引进范数:对任意的$\xi\in X/L$,令:
	\begin{equation*}
		||\xi||=\inf_{x\in\xi}||x||
	\end{equation*}
\end{definition}
\begin{proof}
	(1)非负性与(2)数乘显然,(3)三角不等式由下确界的性质也易得。
\end{proof}
\begin{theorem}
	当$X$是Banach空间,$L$是$X$的闭子空间时,$X/L$也是Banach空间。
\end{theorem}
\begin{proof}
	任取$X/L$中的一个Cauchy点列$\{\xi_n\}$。对于$\xi_n$,存在$x_n\in \xi_n$,使得$\xi_n=x_n+L$。由Cauchy点列的性质,对于$\forall\;\varepsilon>0,\;\exists\;N\in\mathbb{N}^+,\;\forall\;n,m>N$,有:
	\begin{equation*}
		||\xi_m-\xi_n||=||x_m-x_n+L||=\inf_{x\in x_m-x_n+L}||x||<\frac{\varepsilon}{2}
	\end{equation*}
	因为$x_m-x_n+L\subset X$,由下确界的定义,$X$中必然存在元素$y_m,y_n$使得\info{需要证明}
\end{proof}
\subsubsection{Riesz引理}
\begin{lemma}
	设$X_0$是赋范线性空间$X$的真闭子空间,那么对任意的$\varepsilon>0$,存在$x_0\in X$,满足$||x_0||=1$,且对任意的$x\in X_0$,有:
	\begin{equation*}
		||x_0-x||\geqslant 1-\varepsilon
	\end{equation*}
\end{lemma}
\begin{proof}
	当$\varepsilon\geqslant1$时结论是显然的。下证$\varepsilon<1$时结论成立。\par
	因为$X_0$是$X$的真子空间,所以$\exists\;x_1\in X\setminus X_0$。记:
	\begin{equation*}
		d=\inf_{x\in X_0}||x-x_1||
	\end{equation*}
	因为$X_0$是闭的,所以$d>0$。因为$\varepsilon<1$,因此$\frac{d}{1-\varepsilon}>d$。由下确界的定义,存在$x_2\in X_0$,使得:
	\begin{equation*}
		d\leqslant||x_2-x_1||<\frac{d}{1-\varepsilon}
	\end{equation*}
	令$x_0=\frac{x_1-x_2}{||x_1-x_2||}$,则$||x_0||=1$。对任意的$x\in X_0$,注意到$x_2\in X_0$,因此有$||x_1-x_2||x+x_2\in X_0$,于是有:
	\begin{align*}
		||x-x_0||&=\left\|x-\frac{x_1-x_2}{||x_1-x_2||}\right\| \\
		&=\frac{1}{||x_1-x_2||}\left\|\Bigl(||x_1-x_2||x+x_2\Bigr)-x_1\right\| \\
		&\geqslant\frac{d}{||x_1-x_2||} \\
		&>1-\varepsilon\qedhere
	\end{align*}
\end{proof}


