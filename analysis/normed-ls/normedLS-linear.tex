\section{赋范线性空间作为线性空间的性质}

\subsection{基与维数}
\begin{definition}
	若赋范线性空间$X$中存在$n\;(n\geqslant 1)$个元素$e_1,e_2,\dots,e_n$,使得任意的$x\in X$都能唯一地表示成
	\begin{equation*}
		x=\sum_{i=1}^nc_ie_i
	\end{equation*}
	的形式,则称$\{e_1,e_2,\dots,e_n\}$是$X$的一组基,称$(c_1,c_2,\dots,c_n)$是$x$在基$\{e_1,e_2,\dots,e_n\}$下的坐标,称$n$为$X$的维数,称$X$是$n$维赋范线性空间。所有的$n$维赋范线性空间统称为\gls{FDNormedLS},非有限维的赋范线性空间统称为\gls{InfDNormedLS}。
\end{definition}

\subsection{直和}
\begin{theorem}
	设$L_1,L_2,\dots,L_n$都是赋范线性空间,$X$是$L_1,L_2,\dots,L_n$的直和,即:
	\begin{equation*}
		X=L_1\oplus L_2\oplus\cdots\oplus L_n
	\end{equation*}
	可在$X$中定义如下范数:
	\begin{gather*}\label{norm:normedLS-directSum}
		||x||=||x_1||+||x_2||+\cdots+||x_n|| \\
		||x||_1=\max_i||x_i|| \\
		||x||_2=\left(\sum_{i=1}^n||x_i||^2\right)^\frac{1}{2}
	\end{gather*}
	这里$x=\sum\limits_{i=1}^{n}x_i\in X,\;x_i\in L_i(i=1,2,\dots,n)$。$X$按照这些范数都构成赋范线性空间。
\end{theorem}
\begin{proof}
	证明略去,太过简单。
\end{proof}
\begin{theorem}
	如果$L_1,L_2,\dots,L_n$都是Banach空间,$X$是$L_1,L_2,\dots,L_n$的直和,则$X$按照\cref{norm:normedLS-directSum}也成为Banach空间。
\end{theorem}
\begin{proof}
	(1)对于范数$||x||$,在$X$中任取Cauchy点列$\{x_m\}$,记:
	\begin{equation*}
		x_m=\sum_{i=1}^{n}y_m^{i},\quad y_m^{i}\in L_i,\;m\in\mathbb{N}^+,\;i=1,2,\dots,n
	\end{equation*}
	因为$\{x_m\}$是Cauchy点列,所以对于$\forall\;\varepsilon>0,\;\exists\;N\in\mathbb{N}^+,\;\forall\;j,k>N$,有:
	\begin{align*}
		||x_j-x_k||
		&=\left\|\sum_{i=1}^{n}(y_j^i-y_k^i)\right\| \\
		&=\sum_{i=1}^{n}||y_j^i-y_k^i||<\varepsilon
	\end{align*}
	于是$||y_j^i-y_k^i||<\varepsilon$,即$\{y_m^i\}$也构成Cauchy点列,其中$i=1,2,\dots,n$。因为$L_1,L_2,\dots,L_n$都是Banach空间,同时$\{y_m^i\}\subset L_i$,所以存在$y_i\in L_i$,使得$\{y_m^i\}\to y_i$。令$y=\sum\limits_{i=1}^{n}y_i$,显然$y\in X$。则:
	\begin{equation*}
		x_m-y=\sum_{i=1}^{n}(y_m^i-y_i)\to 0\quad(m\to +\infty)
	\end{equation*}
	即$\{x_m\}\to y\in X$。由$\{x_m\}$的任意性,$X$是完备的,所以$X$是一个Banach空间。\par
	(2)与(1)几乎完全一样,只是由$\max\limits_i||y_j^i-y_k^i||<\varepsilon$导出$\{y_m^i\}$是Cauchy点列。\par
	(3)由:
	\begin{equation*}
		\sqrt{\sum_{i=1}^{n}||y_j^i-y_k^i||^2}\leqslant\sum_{i=1}^{n}||y_j^i-y_k^i||
	\end{equation*}
	可导出$\{y_m^i\}$是Cauchy点列。
\end{proof}

\subsection{商空间}
\begin{definition}
	当$L$是$X$的闭子空间时\info{思考为什么一定得是闭子空间},可在商空间$X/L$中引进范数:对任意的$\xi\in X/L$,令:
	\begin{equation*}
		||\xi||=\inf_{x\in\xi}||x||
	\end{equation*}
\end{definition}
\begin{proof}
	(1)非负性与(2)数乘显然,(3)三角不等式由下确界的性质也易得。
\end{proof}
\begin{theorem}
	当$X$是Banach空间,$L$是$X$的闭子空间时,$X/L$也是Banach空间。
\end{theorem}
\begin{proof}
	任取$X/L$中的一个Cauchy点列$\{\xi_n\}$。对于$\xi_n$,存在$x_n\in \xi_n$,使得$\xi_n=x_n+L$。由Cauchy点列的性质,对于$\forall\;\varepsilon>0,\;\exists\;N\in\mathbb{N}^+,\;\forall\;n,m>N$,有:
	\begin{equation*}
		||\xi_m-\xi_n||=||x_m-x_n+L||=\inf_{x\in x_m-x_n+L}||x||<\frac{\varepsilon}{2}
	\end{equation*}
	因为$x_m-x_n+L\subset X$,由下确界的定义,$X$中必然存在元素$y_m,y_n$使得\info{需要证明}
\end{proof}


\subsection{有限维赋范线性空间的性质}
\subsubsection{等距同构与拓扑同构}
\begin{definition}
	设$X$和$Y$都是赋范线性空间,如果满足条件:
	\begin{enumerate}
		\item $X$和$Y$作为线性空间是同构的。
		\item 从$X$到$Y$的同构映射$T$是等距的(同胚的)。
	\end{enumerate}
	则称$X$和$Y$\gls{IsometricIso}(\gls{TopoIso})。
\end{definition}
\begin{lemma}\label{lem:trans of TopoIso}
	若$X$和$Z$拓扑同构,$Y$和$Z$也拓扑同构,则$X$和$Z$拓扑同构。
\end{lemma}
\begin{proof}
	由同构的传递性及两个连续映射的复合也是连续映射可立即推出。
\end{proof}
\begin{theorem}\label{theo:TopoIso of n dimensional normed linear space}
	任意两个同为实或复的$n$维赋范线性空间必拓扑同构。
\end{theorem}
\begin{proof}
	实:由\cref{lem:trans of TopoIso},只需证明任意$n$维赋范线性空间与$\mathbb{R}^n$拓扑同构。\par
	任取一个$n$维赋范线性空间$X$。下证$X$和$\mathbb{R}^n$之间存在一个双射。\par
	设$X$是一个实的$n$维赋范线性空间,$\{e_1,e_2,\dots,e_n\}$是$X$的一个基。定义一个映射$T:\mathbb{R}^n\mapsto X$如下($\xi\in\mathbb{R}^n$):
	\begin{equation*}
		T\xi=\sum_{i=1}^n\xi_ie_i
	\end{equation*}
	由于$\{e_1,e_2,\dots,e_n\}$是$X$的一个基,所以$T\xi$的表出方式唯一,即$T$是一个单射。其次对任意的$y\in X$,它必然能由$\{e_1,e_2,\dots,e_n\}$线性表出,因此有$\eta=(\eta_1,\eta_2,\dots,\eta_n)\in\mathbb{R}^n$,使得$y=\sum\limits_{i=1}^n\eta_ie_i=T\eta$,因此$T$是一个满射。\par
	映射$T$保持线性运算是显然的。\par
	综上,$X$和$\mathbb{R}^n$作为线性空间是同构的。\par
	下证映射$T$是连续映射:
	\begin{align*}
		||T\xi-T\eta||
		&=\left\|\sum_{i=1}^n(\xi_i-\eta_i)e_i\right\| \\
		&\leqslant\sum_{i=1}^n|\xi_i-\eta_i|\;||e_i|| \\
		&\leqslant\left(\sum_{i=1}^n|\xi_i-\eta_i|^2\right)^\frac{1}{2}\cdot\left(\sum_{i=1}^n||e_i||^2\right)^\frac{1}{2} \\
		&=M\left(\sum_{i=1}^n|\xi_i-\eta_i|^2\right)^\frac{1}{2}
	\end{align*}
	由上式,显然$T$是一个连续映射。\par
	下证$T^{-1}$是一个连续映射。\par
	要证$T^{-1}$是一个连续映射,即证$\forall\;x,y\in X$,$\exists\;a\geqslant0$使得:
	\begin{equation*}
		||T^{-1}x-T^{-1}y||\leqslant a||x-y||
	\end{equation*}
	因为已经证得$T$是一个双射且保持线性运算,所以只需证:
	\begin{equation*}
		||T^{-1}x||\leqslant a||x||
	\end{equation*}
	即:
	\begin{equation*}
		\exists\;m>0,\;\frac{||x||}{\left(\sum\limits_{i=1}^n\xi_i^2\right)^\frac{1}{2}}\geqslant m
	\end{equation*}
	由此想到单位球面。对$x=\sum\limits_{i=1}^n\xi_ie_i\in X$,令:
	\begin{equation*}
		f(\xi_1,\xi_2,\dots,\xi_n)=||x||
	\end{equation*}
	当$(\xi_1,\xi_2,\dots,\xi_n)$在$\mathbb{R}^n$的单位球面上时,即$\sum\limits_{i=1}^n\xi_i^2=1$时,有$||x||\ne0$。因为若此时$||x||=0$,则$x$是赋范线性空间$X$的零元,那么也就有$\sum\limits_{i=1}^n\xi_ie_i=\mathbf{0}$。因为$\sum\limits_{i=1}^n\xi_i^2=1$,所以$\xi_i$不全为零,即$\{e_1,e_2,\dots,e_n\}$线性相关。而$\{e_1,e_2,\dots,e_n\}$是$X$的一组基,是线性无关的。由此产生矛盾,也就是说此时$||x||\ne0$。又因为单位球面是一个有界闭集(若某个聚点不在单位球面上,那么单位球面上收敛到这个聚点的点列从某项开始也不会在单位球面上,矛盾),$\mathbb{R}^{n}$上的有界闭集是紧集,所以定义在其上的连续函数可以取到最小值,所以$f(\xi_1,\xi_2,\dots,\xi_n)$在单位球面上有正的下确界$m$,即$||x||\geqslant m$。对任意的$x\in X$进行单位化,即取$x'$满足:
	\begin{equation*}
		x'=\frac{x}{\left(\sum\limits_{i=1}^n\xi_i^2\right)^\frac{1}{2}}
	\end{equation*}
	那么就有$||x'||\geqslant m$,即:
	\begin{equation*}
		||x||\geqslant m\left(\sum\limits_{i=1}^n\xi_i^2\right)^\frac{1}{2}
	\end{equation*}
	于是:
	\begin{equation*}
		||T^{-1}x||=\left(\sum\limits_{i=1}^n\xi_i^2\right)^\frac{1}{2}\leqslant\frac{||x||}{m}
	\end{equation*}
	因此:
	\begin{equation*}
		||T^{-1}x-T^{-1}y||=||T^{-1}(x-y)||\leqslant\frac{||x-y||}{m}
	\end{equation*}
	复:类似,证明与$\mathbb{C}^n$拓扑同构。
\end{proof}
\begin{corollary}
	(1)任一$n\in\mathbb{N}^+$维赋范线性空间都是完备的。(2)任一赋范线性空间$X$的$n\in\mathbb{N}^+$维子空间都是完备的,从而是闭子空间。
\end{corollary}
\begin{proof}
	(1)取$n$维赋范线性空间$X$。因为$X$与$\mathbb{R}^n$是拓扑同构的,那么$X$与$\mathbb{R}^n$之间存在一个双射$T:X\rightarrow\mathbb{R}^n$,且$T$和$T^{-1}$是连续的。任取$X$上的一个Cauchy点列$\{x_n\}$,$Tx_n=y_n\in\mathbb{R}^n$。\par 
	下证$\{y_n\}$也是一个柯西序列。\par
	即证对任意的$\varepsilon>0$,$\exists\;N\in\mathbb{N}^+$,当$n,m>N$时,有$\rho(y_n,y_m)<\varepsilon$。\par
	因为$T$是连续的,所以对上述$\varepsilon$,$\exists\;\delta>0$,当$\rho(x_n,x_m)<\delta$时,有$\rho(y_n,y_m)<\varepsilon$。而$\{x_n\}$是Cauchy点列,因此对上述$\delta$,$\exists\;N'\in\mathbb{N}^+$,当$n,m>N$时,有$\rho(x_n,x_m)<\delta$。所以取$N=N'$即可满足条件。因此$\{y_n\}$是一个柯西序列。\par
	因为$\mathbb{R}^n$是完备的,所以$\{y_n\}\rightarrow y\in\mathbb{R}^n$。那么就有:
	\begin{equation*}
		\lim_{n\to+\infty}T^{-1}y_n=T^{-1}\lim_{n\to+\infty}y_n=T^{-1}y
	\end{equation*}
	因为$T$是双射,那么$\exists\;x\in X$使得$x=T^{-1}y$。因为:
	\begin{equation*}
		\lim_{n\to+\infty}T^{-1}y_n=\lim_{n\to+\infty}x_n
	\end{equation*}
	所以$\{x_n\}\rightarrow x\in X$。由$\{x_n\}$的任意性,$X$是完备的。\par
	(2)完备性由(1)可直接得到。因为完备,所以任一Cauchy点列收敛。因为任一收敛点列都是Cauchy点列,所以聚点都在其内,从而是闭子空间。
\end{proof}
\subsubsection{Riesz引理}
\begin{lemma}
	设$X_0$是赋范线性空间$X$的真闭子空间,那么对任意的$\varepsilon>0$,存在$x_0\in X$,满足$||x_0||=1$,且对任意的$x\in X_0$,有:
	\begin{equation*}
		||x_0-x||\geqslant 1-\varepsilon
	\end{equation*}
\end{lemma}
\begin{proof}
	当$\varepsilon\geqslant1$时结论是显然的。下证$\varepsilon<1$时结论成立。\par
	因为$X_0$是$X$的真子空间,所以$\exists\;x_1\in X\backslash X_0$。记:
	\begin{equation*}
		d=\inf_{x\in X_0}||x-x_1||
	\end{equation*}
	因为$X_0$是闭的,所以$d>0$。因为$\varepsilon<1$,因此$\frac{d}{1-\varepsilon}>d$。由下确界的定义,存在$x_2\in X_0$,使得:
	\begin{equation*}
		d\leqslant||x_2-x_1||<\frac{d}{1-\varepsilon}
	\end{equation*}
	令$x_0=\frac{x_1-x_2}{||x_1-x_2||}$,则$||x_0||=1$。对任意的$x\in X_0$,注意到$x_2\in X_0$,因此有$||x_1-x_2||x+x_2\in X_0$,于是有:
	\begin{align*}
		||x-x_0||&=\left\|x-\frac{x_1-x_2}{||x_1-x_2||}\right\| \\
		&=\frac{1}{||x_1-x_2||}\left\|\Bigl(||x_1-x_2||x+x_2\Bigr)-x_1\right\| \\
		&\geqslant\frac{d}{||x_1-x_2||} \\
		&>1-\varepsilon\qedhere
	\end{align*}
\end{proof}
\subsubsection{有限维赋范线性空间的判别}
\begin{definition}
	$X$是一个赋范线性空间,若$X$的任一有界闭集是紧的,则称$X$是
	\gls{LocallyCompact}。
\end{definition}
\begin{theorem}
	赋范线性空间$X$是有限维的的充分必要条件为$X$是局部紧的。
\end{theorem}
\begin{proof}
	必要性:设$X$是$n$维实赋范线性空间,由\cref{theo:TopoIso of n dimensional normed linear space},$X$与$\mathbb{R}^n$拓扑同构。因此$X$中的有界闭集映成$\mathbb{R}^n$中的有界闭集,反之亦然(闭集由连续映射保证,有界)。而$\mathbb{R}^n$中的有界闭集是紧的,因为$\mathbb{R}^{n}$到$X$上的映射是连续映射,于是$X$中的任一有界闭集也是紧的(连续映射把紧集映射为紧集)。所以$X$是局部紧的。\par
	充分性:若此时$X$是无限维的。取$S=\{x:||x||=1\}$为$X$的单位球面\info{有空证明}
\end{proof}


