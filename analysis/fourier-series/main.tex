\chapter{Fourier级数}

考察在闭区间$[\alpha,\beta]$上所有Riemann可积的函数构成的集合$\mathscr{R}$。\par
对任意的$f,g\in\mathscr{R},\;a\in\mathbb{R}$,定义如下的加法和数量乘法:
\begin{gather*}
	(f+g)(x)=f(x)+g(x),\;(af)(x)=af(x)
\end{gather*}
则$\mathscr{R}$成为一个实线性空间。\par
在$\mathscr{R}$上按如下方式定义内积:
\begin{equation*}
	(f,g)=\rho\int_{\alpha}^{\beta}f(x)g(x)\dif x
\end{equation*}
如果两个函数$f,g\in\mathscr{R}$满足条件$(f,g)=0$,则称这两个函数是正交的。\par
考察函数系$\{\varphi_n(x)\}$,若其中的函数两两正交,则称该函数系是正交的。若函数系$\{\varphi_n(x)\}$满足条件:
\begin{equation*}
	(\varphi_m,\varphi_n)=
	\begin{cases}
		1,&m=n \\
		0,&m\ne n
	\end{cases}
\end{equation*}
则称该函数系是规范正交的。\par
取$\mathscr{R}$中一个规范正交函数系$\{\varphi_n(x)\}$,若函数$f\in\mathscr{R}$能够展开成以下形式的级数:
\begin{equation*}
	f(x)=\sum_{n=0}^{+\infty}c_n\varphi_n(x)
\end{equation*}
以$\varphi_k(x)$乘上式两边可得:
\begin{equation*}
	f(x)\varphi_k(x)=\sum_{n=0}^{+\infty}c_n\varphi_n(x)\varphi_k(x)
\end{equation*}
若上式右端的级数可以逐项积分,就能得到:
\begin{align*}
	(f,\varphi_k)
	=\sum_{n=0}^{+\infty}c_n(\varphi_n,\varphi_k)
	=c_k,\;k=0,1,2,\dots
\end{align*}
即:
\begin{equation*}
	c_k=(f,\varphi_k),\;k=0,1,2,\dots
\end{equation*}
上式被称为函数$f$关于正交函数系$\{\varphi_n(x)\}$的Euler-Fourier公式,按照该公式计算系数,然后做成级数:
\begin{equation*}
	\sum_{n=0}^{+\infty}c_n\varphi_n(x)
\end{equation*}
称该级数为函数$f$关于正交函数系$\{\varphi_n(x)\}$的Fourier级数。

\begin{definition}
	我们把函数系:
	\begin{equation*}
		1,\cos(t),\sin(t),\cos(2t),\sin(2t),\dots,\cos(nt),\sin(nt),\dots
	\end{equation*}
	称之为周期为$2\pi$的基本三角函数系。
\end{definition}
\begin{proof}
	\begin{gather*}
		\int_{-\pi}^{\pi}1\cdot\cos(nt)\dif t=\frac{1}{n}\sin(nt)\Big|_{-\pi}^{\pi}=0,\quad
		\int_{-\pi}^{\pi}1\cdot\sin(nt)\dif t=-\frac{1}{n}\cos(nt)\Big|_{-\pi}^{\pi}=0 \\
		\begin{align*}
			\int_{-\pi}^{\pi}\cos(mt)\cos(nt)\dif t
			&=\int_{-\pi}^{\pi}\frac{1}{2}\{\cos[(m+n)t]+\cos[(m-n)t]\}\dif t \\
			&=
			\begin{cases}
				\frac{1}{2(m+n)}\sin[(m+n)t]\Big|_{-\pi}^{\pi}+\frac{1}{2(m-n)}\sin[(m-n)t]\Big|_{-\pi}^{\pi},&m\ne n \\
				\frac{1}{2(m+n)}\sin[(m+n)t]\Big|_{-\pi}^{\pi}+\frac{1}{2}t\Big|_{-\pi}^{\pi},&m=n
			\end{cases}
			 \\
			&=
			\begin{cases}
				0,&m\ne n \\
				\pi,&m=n
			\end{cases}
		\end{align*} \\
		\begin{align*}
			\int_{-\pi}^{\pi}\sin(mt)\sin(nt)\dif t
			&=\int_{-\pi}^{\pi}-\frac{1}{2}\{\cos[(m+n)t]-\cos[(m-n)t]\}\dif t \\
			&=
			\begin{cases}
				-\frac{1}{2(m+n)}\sin[(m+n)t]\Big|_{-\pi}^{\pi}+\frac{1}{2(m-n)}\sin[(m-n)t]\Big|_{-\pi}^{\pi},&m\ne n \\
				-\frac{1}{2(m+n)}\sin[(m+n)t]\Big|_{-\pi}^{\pi}+\frac{1}{2}t\Big|_{-\pi}^{\pi},&m=n
			\end{cases}
			\\
			&=
			\begin{cases}
				0,&m\ne n \\
				\pi,&m=n
			\end{cases}
		\end{align*} \\
		\begin{align*}
			\int_{-\pi}^{\pi}\cos(mt)\sin(nt)\dif t
			&=\int_{-\pi}^{\pi}\frac{1}{2}\{\sin[(m+n)t]-\sin[(m-n)t]\}\dif t \\
			&=
			\begin{cases}
				-\frac{1}{2(m+n)}\cos[(m+n)t]\Big|_{-\pi}^{\pi}+\frac{1}{2(m-n)}\cos[(m-n)t]\Big|_{-\pi}^{\pi},&m\ne n \\
				-\frac{1}{2(m+n)}\cos[(m+n)t]\Big|_{-\pi}^{\pi},&m=n
			\end{cases} \\
			&=0
		\end{align*}
	\end{gather*}
\end{proof}