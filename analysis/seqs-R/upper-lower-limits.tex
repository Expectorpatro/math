\section{实数序列的上下极限}
\subsubsection{定义}
\begin{definition}
	对于实数序列$\{x_n\}$,我们将
	\begin{gather*}
		\varliminf x_n=\lim_{n\to+\infty}\inf_{k\geqslant n}x_k=\sup_n\inf_{k\geqslant n}x_k \\
		\varlimsup x_n=\lim_{n\to+\infty}\sup_{k\geqslant n}x_k=\inf_n\sup_{k\geqslant n}x_k
	\end{gather*}
	分别称之为$\{x_n\}$的\gls{LLimit}和\gls{ULimit}。
\end{definition}
我们知道不是所有实数序列都有极限,但任意实数序列在$\overline{\mathbb{R}}$中都有上下极限。\par
不难注意到:
\begin{equation*}
	\inf_{k\geqslant n}x_k,\quad
	\sup_{k\geqslant n}x_k
\end{equation*}
分别为单调递增序列与单调递减序列,而单调序列在$\overline{\mathbb{R}}$中都存在极限、上下确界。\info{记得写完以后链接过来,并且在写上下确界与极限的时候就把无穷的情况并进去}

\subsection{上下极限与聚点的关系}
\subsubsection{聚点介于上下极限之间}
\begin{theorem}
	实数序列$\{x_n\}$的任何聚点$\xi$都介于$\eta=\varliminf x_n$和$\zeta=\varlimsup x_n$之间。
\end{theorem}
\begin{proof}
	任取实数序列的一个聚点$\xi$,由聚点定义可知,$\{x_n\}$中存在一个子列$\{x_{n_k}\}$满足:
	\begin{equation*}
		\lim_{k\to+\infty} x_{n_k}=\xi
	\end{equation*}
	令:
	\begin{equation*}
		y_k=\inf_{n\geqslant k}x_n,\quad
		z_k=\sup_{n\geqslant k}x_n
	\end{equation*}
	显然有:
	\begin{equation*}
		y_k\leqslant x_{n_k}\leqslant z_k
	\end{equation*}
	因为$n_k\geqslant k$,所以对上式取极限即为:
	\begin{equation*}
		\eta\leqslant\xi\leqslant\zeta\qedhere
	\end{equation*}
\end{proof}
\subsubsection{上下极限也是聚点}
\begin{theorem}
	实数序列$\{x_n\}$的下极限$\eta=\varliminf x_n$和上极限$\zeta=\varlimsup x_n$是其自身的两个聚点。
\end{theorem}
\begin{proof}
	下给出下极限的证明,上极限可类似得出。\par
	若$\{x_n\}$下方无界,则存在子列$\{x_{n_k}\}$使得:
	\begin{equation*}
		\lim_{k\to+\infty}x_{n_k}=-\infty
	\end{equation*}
	于是$-\infty$成为实数序列$\{x_n\}$的一个聚点。\par
	当$\{x_n\}$下方有界时,记:
	\begin{equation*}
		y_k=\inf_{n\geqslant k}x_n
	\end{equation*}
	由下确界的定义可知,存在$x_{n_k}$满足:
	\begin{equation*}
		y_k\leqslant x_{n_k}\leqslant y_k+\frac{1}{k}
	\end{equation*}
	对上式取极限即为:
	\begin{equation*}
		\lim_{k\to+\infty}x_{n_k}=\lim_{k\to+\infty}y_k=\eta=\varliminf x_n \qedhere
	\end{equation*}	
\end{proof}

\subsection{上下极限与极限的关系}
\begin{theorem}
	设$\{x_n\}$是实数序列,则以下三条陈述互相等价:
	\begin{enumerate}
		\item $\varliminf x_n=\varlimsup x_n=\xi$;
		\item $\lim\limits_{n\to+\infty}x_n=\xi$;
		\item $\{x_n\}$只有一个聚点。
	\end{enumerate}
\end{theorem}
\begin{proof}
	$(2)\rightarrow(3)$和$(3)\rightarrow(1)$是显然的,下证$(1)\rightarrow(2)$。\par
	令:
	\begin{equation*}
		y_n=\inf_{k\geqslant n}x_k,\quad
		z_n=\sup_{k\geqslant n}x_k
	\end{equation*}
	则显然有:
	\begin{equation*}
		y_n\leqslant x_n\leqslant z_n
	\end{equation*}
	又因为:
	\begin{equation*}
		\lim_{n\to+\infty}y_n=\lim_{n\to+\infty}z_n=\xi
	\end{equation*}
	由夹逼定理可得:
	\begin{equation*}
		\lim_{n\to+\infty}x_n=\xi\qedhere
	\end{equation*}
\end{proof}

\subsection{上下极限的大小}
\begin{theorem}
	设$\{x_n\}$是实数序列,
	\begin{enumerate}
		\item 如果$\varliminf x_n>\lambda$,则$\exists\;N\in\mathbb{N}^+,\;\forall\;n>N,\;x_n>\lambda$。
		\item 如果$\varliminf x_n<\rho$,则$\forall\;N\in\mathbb{N}^+,\;\exists\;n>N,\;x_n<\rho$。
		\item 如果$\varlimsup x_n<\rho$,则$\exists\;N\in\mathbb{N}^+,\;\forall\;n>N,\;x_n<\rho$。
		\item 如果$\varlimsup x_n>\lambda$,则$\forall\;N\in\mathbb{N}^+,\;\exists\;n>N,\;x_n>\lambda$。
	\end{enumerate}
\end{theorem}
证明是容易的,利用上下极限定义中的$\sup$和$\inf$即可。同时从$\mathbb{R}$的紧性来讲的话,上面这个定理也是很容易记住的。

\subsection{上下极限的运算}
\begin{theorem}
	设$\{u_n\}$和$\{v_n\}$都是实数序列,则只要以下各式等号或不等号两侧的式子都有意义,则式子成立。
	\begin{enumerate}
		\item 上下极限的加法运算:
		\begin{align*}
			\varliminf u_n+\varliminf v_n &\leqslant\varliminf(u_n+v_n) \\
			&\leqslant
			\begin{cases}
				\varlimsup u_n+\varliminf v_k \\
				\varliminf u_n+\varlimsup v_k
			\end{cases} \\
			&\leqslant\varlimsup(u_n+v_n) \\
			&\leqslant\varlimsup u_n+\varlimsup v_n
		\end{align*}
		\item 上下极限的减法运算:
		\begin{gather*}
			-\varliminf u_n=\varlimsup(-u_n) \\
			-\varlimsup u_n=\varliminf(-u_n)
		\end{gather*}
		\item 上下极限的乘法运算:
		\begin{align*}
			\varliminf u_n\cdot\varliminf v_n &\leqslant\varliminf(u_n\cdot v_n) \\
			&\leqslant
			\begin{cases}
				\varlimsup u_n\cdot\varliminf v_k \\
				\varliminf u_n\cdot\varlimsup v_k
			\end{cases} \\
			&\leqslant\varlimsup(u_n\cdot v_n) \\
			&\leqslant\varlimsup u_n\cdot\varlimsup v_n
		\end{align*}
		\item 上下极限的分式运算($\varliminf u_n>0$):
		\begin{gather*}
			\frac{1}{\varliminf u_n}=\varlimsup\frac{1}{u_n} \\
			\frac{1}{\varlimsup u_n}=\varliminf\frac{1}{u_n}
		\end{gather*}
		\item 上下极限与极限混合的加法运算($\lim\limits_{n\to+\infty}u_n=u$):
		\begin{gather*}
			\varliminf(u_n+v_n)=u+\varliminf v_n \\
			\varlimsup(u_n+v_n)=u+\varlimsup v_n
		\end{gather*}
		\item 上下极限与极限混合的乘法运算($\lim\limits_{n\to+\infty}u_n=u>0,\;v_n\geqslant0,\;\forall\;n\in\mathbb{N}^+$):
		\begin{gather*}
			\varliminf(u_n\cdot v_n)=u\cdot\varliminf v_n \\
			\varlimsup(u_n\cdot v_n)=u\cdot\varlimsup v_n
		\end{gather*}
		\item 上下极限的不等式性($u_n\leqslant v_n$):
		\begin{gather*}
			\varliminf u_n\leqslant\varliminf v_n \\
			\varlimsup u_n\leqslant\varlimsup v_n
		\end{gather*}
	\end{enumerate}
\end{theorem}
\info{有空证明}