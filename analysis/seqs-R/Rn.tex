\section{$\mathbb{R}^n$}
\begin{definition}
	设$\{x_n\}$是$\mathbb{R}^{}$中的点列。
	\begin{enumerate}
		\item 若对任意的$M\in\mathbb{R}^{+}$,存在$N\in\mathbb{N}^+$使得当$n>N$时有$x_n>M$,则称$\{x_n\}$\textbf{发散}于$+\infty$,记为$x_n\to+\infty$或$\lim\limits_{n\to+\infty}x_n=+\infty$;
		\item 若对任意的$M\in\mathbb{R}^{+}$,存在$N\in\mathbb{N}^+$使得当$n>N$时有$x_n<-M$,则称$\{x_n\}$\textbf{发散}于$-\infty$,记为$x_n\to-\infty$或$\lim\limits_{n\to+\infty}x_n=-\infty$;
		\item 若$\{|x_n|\}$发散于$+\infty$,则称$\{x_n\}$为\gls{DivergentSequencetoInfinity}。 
	\end{enumerate}
\end{definition}
\begin{definition}
	规定:
	\begin{enumerate}
		\item 若$x\in\mathbb{R}^{}$,则$-\infty<x<+\infty$;
		\item 若$x\in\mathbb{R}^{}$,则:
		\begin{equation*}
			x+(+\infty)=+\infty,\quad x-(+\infty)=-\infty,\quad x+(-\infty)=0\infty,\quad x-(-\infty)=+\infty
		\end{equation*}
		\item 若$x\in\mathbb{R}^{+}$,则:
		\begin{equation*}
			x\cdot(+\infty)=(+\infty)\cdot x=+\infty,\quad x\cdot(-\infty)=(-\infty)\cdot x=-\infty
		\end{equation*}
		若$x\in\mathbb{R}^{-}$,则:
		\begin{equation*}
			x\cdot(+\infty)=(+\infty)\cdot x=-\infty,\quad x\cdot(-\infty)=(-\infty)\cdot x=+\infty
		\end{equation*}
		\item 若$x\in\mathbb{R}^{}$,则:
		\begin{equation*}
			\frac{x}{+\infty}=\frac{x}{-\infty}=0
		\end{equation*}
		\item 对于无穷的运算:
		\begin{gather*}
			(+\infty)+(+\infty)=+\infty,\quad(+\infty)-(-\infty)=+\infty \\
			(-\infty)+(-\infty)=-\infty,\quad(-\infty)-(+\infty)=-\infty \\
			(+\infty)\cdot(+\infty)=(-\infty)\cdot(-\infty)=+\infty,\quad(+\infty)\cdot(-\infty)=(-\infty)\cdot(+\infty)=-\infty
		\end{gather*}
	\end{enumerate}
	其余涉及无穷的运算皆无意义。
\end{definition}
\begin{definition}
	$\overline{\mathbb{R}^{}}=\mathbb{R}^{}\cup\{+\infty,-\infty\}=\mathbb{R}^{}\cup\{\infty\}$。
\end{definition}
\begin{definition}
	设$\{x_n\}$是$\overline{\mathbb{R}^{}}$中的点列。
	\begin{enumerate}
		\item 若存在$m\in\mathbb{R}^{}$使得对任意的$n\in\mathbb{N}^+$有$x_n\geqslant m$,则称$\{x_n\}$有\gls{LowerBound},$m$是它的一个下界;
		\item 若存在$M\in\mathbb{R}^{}$使得对任意的$n\in\mathbb{N}^+$有$x_n\leqslant M$,则称$\{x_n\}$有\gls{UpperBound},$M$是它的一个上界。
	\end{enumerate}
\end{definition}
\begin{definition}
	设$\{x_n\}$是$\overline{\mathbb{R}^{}}$中的点列。若$x_n\to0$,则称$\{x_n\}$为\gls{NullSequence}。
\end{definition}
\begin{property}\label{prop:RSeq}
	对于$\overline{\mathbb{R}^{}}$中的序列,有:
	\begin{enumerate}
		\item 设$\{x_n\}\subset\overline{\mathbb{R}^{}}$,若$\{x_n\}$有极限,则其极限唯一;
		\item 设$\{x_n\},\{y_n\}\subset \overline{\mathbb{R}^{}}$且为有界序列,则$\{x_n+y_n\},\{x_ny_n\}$都是有界序列;
		\item 设$\{x_n\},\{y_n\}\subset X$且分别为有界序列与无穷小序列,则$\{x_ny_n\}$是无穷小序列;
		\item (Squeeze Theorem)设$\{x_n\},\{y_n\},\{z_n\}\subset\overline{\mathbb{R}^{}}$,对任意的$n\in\mathbb{N}^+$有$x_n\leqslant y_n\leqslant z_n$,若$\lim\limits_{n\to+\infty}x_n=\lim\limits_{n\to+\infty}z_n=a$,则$\lim\limits_{n\to+\infty}y_n=a$;
		\item 设$\{x_n\},\{y_n\}$
	\end{enumerate}
\end{property}
\begin{proof}
	(1)\par
	(2)\textbf{和:}对任意的$n\in\mathbb{N}^+$有:
	\begin{equation*}
		\rho(x_n+y_n,0)=|x_n+y_n|\leqslant|x_n|+|y_n|=\rho(x_n,0)+\rho(y_n,0)<+\infty
	\end{equation*}
	所以$\{x_n+y_n\}$有界。\par
	\textbf{乘积:}因为$\{x_n\}$有界,所以$\rho(x_n,0)=|x_n|<+\infty$,即存在$K\in\mathbb{R}^{+}$使得$|x_n|<K$,同理可得存在$L\in\mathbb{R}^{+}$使得$|y_n|<L$,于是对任意的$n\in\mathbb{N}^+$有:
	\begin{equation*}
		\rho(x_ny_n,0)=|x_ny_n|<KL<+\infty
	\end{equation*}
	所以$\{x_ny_n\}$有界。\par
	(3)因为$\{x_n\}$有界,所以$\rho(x_n,0)=|x_n|<+\infty$,即存在$K\in\mathbb{R}^{+}$使得$|x_n|<K$。因为$\{y_n\}$是无穷小序列,于是对任意的$\varepsilon>0$,有$\dfrac{\varepsilon}{K}>0$,所以存在$N\in\mathbb{N}^+$,当$n>N$时有$\rho(y_n,0)=|y_n|<\dfrac{\varepsilon}{K}$,即$\rho(x_ny_n,0)=|x_ny_n|<\varepsilon$,$x_ny_n\to0$。\par
	(4)因为$x_n\to a,\;z_n\to a$,所以对任意的$\varepsilon>0$,存在$N_1,N_2\in\mathbb{N}^+$,当$n>N_1$时有$a-\varepsilon<x_n<a+\varepsilon$,当$n>N_2$时有$a-\varepsilon<z_n<a+\varepsilon$,于是当$n>\max\{N_1,N_2\}$时有:
	\begin{equation*}
		a-\varepsilon<x_n\leqslant y_n\leqslant z_n<a+\varepsilon
	\end{equation*}
	即$\rho(y_n,a)=|y_n-a|<\varepsilon$,于是有$y_n\to a$。
\end{proof}
在$\mathbb{R}^{n}$中若涉及点列,规定用下标表示点列顺序,上标表示坐标维度。
\begin{definition}
	在$\mathbb{R}^n$中定义向量$x=(\xi_1,\xi_2,\dots,\xi_n)$和向量$y=(\eta_1,\eta_2,\dots,\eta_n)$之间的距离为:
	\begin{equation*}
		\rho(x,y)=\left[\sum_{i=1}^n(\xi_i-\eta_i)^2\right]^{\frac{1}{2}}
	\end{equation*}
	则$(\mathbb{R}^n,\rho)$是一个度量空间,称该距离为\textbf{欧几里得距离},简称为\textbf{欧氏距离}。
\end{definition}
\begin{proof}
	(1)显然$\rho\in R$;(2)非负性直接可得;(3)对称性直接可得;\par
	(4)三角不等式:由\cref{ineq:cauchy-ineq-R},可得:
	\begin{align*}
		\sum_{i=1}^n(a_i+b_i)^2&=\sum_{i=1}^na_i^2+2\sum_{i=1}^na_ib_i+\sum_{i=1}^nb_i^2 \\
		&\leqslant\sum_{i=1}^na_i^2+2\left(\sum_{i=1}^na_i^2\cdot\sum_{i=1}^nb_i^2\right)^{\frac{1}{2}}+\sum_{i=1}^nb_i^2 \\
		&=\left[\left(\sum_{i=1}^na_i^2\right)^{\frac{1}{2}}+\left(\sum_{i=1}^nb_i^2\right)^{\frac{1}{2}}\right]^2
	\end{align*}
	设$x=(\xi_1,\xi_2,\dots,\xi_n),y=(\eta_1,\eta_2,\dots,\eta_n),z=(\zeta_1,\zeta_2,\dots,\zeta_n)$是$\mathbb{R}^n$中的任意三点,在上式中令$a_i=\xi_i-\zeta_i,b_i=\zeta_i-\eta_i$,则
	\begin{equation*}
		\left[\sum_{i=1}^n(\xi_i-\eta_i)^2\right]^{\frac{1}{2}}\leqslant	\left[\sum_{i=1}^n(\xi_i-\zeta_i)^2\right]^{\frac{1}{2}}+\left[\sum_{i=1}^n(\zeta_i-\eta_i)^2\right]^{\frac{1}{2}}
	\end{equation*}
	即
	\begin{equation*}
		\rho(x,y)\leqslant\rho(x,z)+\rho(z,y)\qedhere
	\end{equation*}
\end{proof}
\begin{definition}
	在$\mathbb{R}^n$中定义元素$x=(\xi_1,\xi_2,\dots,\xi_n)$的范数为:
	\begin{equation*}
		||x||=\left(\sum_{i=1}^n\xi_i^2\right)^{\frac{1}{2}}
	\end{equation*}
	则$\mathbb{R}^n$成为一个赋范线性空间。
\end{definition}
\begin{proof}
	(1)$\;||x||\in\mathbb{R}$、(2)非负性和(3)数乘显然,(4)三角不等式的证明可见欧式距离三角不等式的证明。
\end{proof}
\begin{property}\label{prop:RmConvergence}
	$\mathbb{R}^{m}$中的收敛具有如下性质:
	\begin{enumerate}
		\item $\mathbb{R}^{m}$在欧氏距离下的收敛等价于按坐标收敛;
		\item $\mathbb{R}^{m}$中依范数收敛等价于按坐标收敛;
		\item 设$\{x_n\},\{y_n\}\subset\mathbb{R}^{m},\;\{a_n\},\{b_n\}\subset\mathbb{R}^{}$,若$x_n\to x,y_n\to y,a_n\to a,b_n\to b,\;a,b\in\mathbb{R}^{}$,则$a_nx_n+b_ny_n\to ax+by$;
	\end{enumerate}
\end{property}
\begin{proof}
	(1)由下式可立即推出:
	\begin{equation*}
		\max_i|\xi_i-\eta_i|\leqslant	\left[\sum_{i=1}^m(\xi_i-\eta_i)^2\right]^{\frac{1}{2}}\leqslant\left\{\left[\sum_{i=1}^{m}(\xi_i-\eta_i)\right]^2\right\}^{\frac{1}{2}}=\sum_{i=1}^{m}|\xi_i-\eta_i|
	\end{equation*}\par
	(2)依范数收敛等价于按欧氏距离收敛。\par
	(3)由\cref{prop:Norm}(3)立即可得。
\end{proof}
\begin{theorem}
	$\mathbb{R}^{m}$是完备的度量空间。
\end{theorem}
\begin{proof}
	任取$\mathbb{R}^{m}$中的Cauchy点列$\{x_n\}$,则对任意的$\varepsilon>0$,存在$N\in\mathbb{N}^+$,当$n_1,n_2>N$时有$||x_{n_1}-x_{n_2}||<\varepsilon$,于是$|x_{n_1}^i-x_{n_2}^i|\leqslant||x_{n_1}-x_{n_2}||<\varepsilon$,$x_{n}^i$构成$\mathbb{R}^{}$上的Cauchy点列。由$\mathbb{R}^{}$的完备性\info{$R$的完备性}可知:
	\begin{equation*}
		\lim_{n\to+\infty}x_n^i=a_i\in\mathbb{R}^{},\quad i=1,2,\dots,m
	\end{equation*}
	记$a=(\seq{a}{m})$,于是:
	\begin{equation*}
		\lim_{n\to+\infty}||x_n-a||=\lim_{n\to+\infty}\left(\sum_{i=1}^{m}||x_n^i-a^i||\right)=0
	\end{equation*}
	即$\{x_n\}\to  a$。由$\{x_n\}$的任意性即可得出结论。
\end{proof}

\subsection{性质}
\subsubsection{可分性}
\begin{theorem}
	$\mathbb{R}^{n}$是可分的。
\end{theorem}
\begin{proof}
	坐标由有理数构成的点的全体是$\mathbb{R}^{n}$的一个可列稠密子集。
\end{proof}