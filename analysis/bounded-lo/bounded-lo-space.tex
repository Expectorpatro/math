\section{有界线性算子空间}

\subsection{有界线性算子空间的范数}
\begin{theorem}
	设$X$和$Y$都是赋范线性空间,用$\mathscr{B}(X,Y)$表示$X$到$Y$上的所有有界线性算子构成的空间\footnote{若不作特殊说明,有界线性算子的定义域均假定为$X$。}。在$\mathscr{B}(X,Y)$中定义线性运算如下:
	\begin{gather*}
		(T_1+T_2)x=T_1x+T_2x \\
		(\alpha T)x=\alpha Tx
	\end{gather*}
	其中$T,T_1,T_2\in\mathscr{B}(X,Y)$,$\alpha$是一个数。$\mathscr{B}(X,Y)$在该线性运算下构成一个线性空间。定义$\mathscr{B}(X,Y)$中的范数如下:
	\begin{equation*}
		\forall\;T\in\mathscr{B}(X,Y),\;||T||=\sup_{\{x\in X:x\ne\mathbf{0}\}}\frac{||Tx||}{||x||}
	\end{equation*}
	则$\mathscr{B}(X,Y)$构成一个赋范线性空间。
\end{theorem}
\begin{proof}
	\info{有时间证明线性空间。}(1)非负性和(2)数乘是显然的。\par
	(3)三角不等式:
	\begin{align*}
		||T_1+T_2||
		&=\sup_{\{x\in X:x\ne\mathbf{0}\}}\frac{||(T_1+T_2)x||}{||x||} \\
		&\leqslant\sup_{\{x\in X:x\ne\mathbf{0}\}}\frac{||T_1x||+||T_2x||}{||x||} \\
		&\leqslant\sup_{\{x\in X:x\ne\mathbf{0}\}}\frac{||T_1x||}{||x||}+
		\sup_{\{x\in X:x\ne\mathbf{0}\}}\frac{||T_2x||}{||x||} \\
		&=||T_1||+||T_2||\qedhere  
	\end{align*}
\end{proof}
如上定义的范数公式有如下性质:
\begin{theorem}
	(1)\;$\forall\;x\in X,\;||Tx||\leqslant||T||\;||x||$。
	(2)\;$||T||=\underset{\{x\in X:||x||\leqslant1\}}{\sup}||Tx||=\underset{\{x\in X:||x||=1\}}{\sup}||Tx||$。
\end{theorem}
证明是显然的,(1)中只需再验证$X$中的零元,(2)可由范数定义中的数乘与线性算子的数乘直接推出。

\subsection{有界线性算子空间中的依算子范数收敛}
\begin{definition}
	设$\{T_n\}\in\mathscr{B}$。若$\{T_n\}$依$\mathscr{B}$中的范数收敛于$T\in\mathscr{B}$,即:
	\begin{equation*}
		\lim_{n\to+\infty}||T_n-T||=0
	\end{equation*}
	则称$\{T_n\}$依算子范数收敛于$T$或$\{T_n\}$依一致算子拓扑收敛于$T$。
\end{definition}
\subsubsection{依算子范数收敛与一致收敛的等价性}
\begin{theorem}
	设$\{T_n\}\in\mathscr{B}(X,Y)$。$\{T_n\}$依一致算子拓扑收敛于$T\in\mathscr{B}(X,Y)$的充分必要条件为$\{T_n\}$在$X$中的任意有界集上一致收敛于$T$。
\end{theorem}
\begin{proof}
	(1)必要性:任取$A\subset X$为有界集。由$A$的有界性:
	\begin{equation*}
		\exists\;K>0,\;\forall\;x\in A,\;||x||\leqslant K
	\end{equation*}
	因此:
	\begin{equation*}
		||T_nx-Tx||=||(T_n-T)x||\leqslant||T_n-T||\;||x||\leqslant K||T_n-T||
	\end{equation*}
	由$\lim\limits_{n\to+\infty}||T_n-T||=0$可得:
	\begin{equation*}
		\forall\;\varepsilon>0,\;\exists\;N\in\mathbb{N}^+,\;\forall\;n>N,\;||T_n-T||<\frac{\varepsilon}{K}
	\end{equation*}
	于是:
	\begin{equation*}
		\forall\;\varepsilon>0,\;\exists\;N\in\mathbb{N}^+,\;\forall\;n>N,\;\forall\;x\in A,\;||T_nx-Tx||<\varepsilon
	\end{equation*}\par
	(2)充分性:在$X$中取单位球面$S=\{x\in X:||x||=1\}$。显然$S$是一个有界集,由一致收敛可得:
	\begin{equation*}
		\forall\;\varepsilon>0,\;\exists\;N\in\mathbb{N}^+,\;\forall\;n>N,\;\forall\;x\in X,\;||T_nx-Tx||<\varepsilon
	\end{equation*}
	于是:
	\begin{equation*}
		\forall\;\varepsilon>0,\;\exists\;N\in\mathbb{N}^+,\;\forall\;n>N,\;||T_n-T||=\sup_{x\in X,||x||=1}||T_nx-Tx||\leqslant\varepsilon
	\end{equation*}
	即$\{T_n\}$依算子范数收敛到$T$。
\end{proof}
由以上定理,$\{T_n\}$依算子收敛于$T$又称为$\{T_n\}$依一致算子拓扑收敛于$T$。

\subsection{有界线性算子空间中的强收敛}
\begin{definition}
	设$\{T_n\}\in\mathscr{B}(X,Y)$。若:
	\begin{equation*}
		\forall\;x\in X,\;\lim_{n\to+\infty}||T_nx-Tx||=0
	\end{equation*}
	则称$\{T_n\}$\gls{StrongConvergence}到$T$或依强算子拓扑收敛于$T$。
\end{definition}

\subsection{有界线性算子空间的完备性}
\begin{theorem}
	若$Y$是Banach空间,则$\mathscr{B}(X,Y)$也是Banach空间。
\end{theorem}
\begin{proof}
	设$\{T_n\}$是$\mathscr{B}(X,Y)$中的一个Cauchy点列,那么:
	\begin{equation*}
		\forall\;\varepsilon>0,\;\exists\;N\in\mathbb{N}^+,\;\forall\;n,m>N,\;m>n,\;||T_m-T_n||<\varepsilon
	\end{equation*}	
	此时有:
	\begin{equation*}
		\forall\;x\in X,\;||T_mx-T_nx||\leqslant||T_m-T_n||\;||x||<\varepsilon||x||
	\end{equation*}
	因此$\{T_nx\}$($x$为定值)是$Y$中的Cauchy点列。因为$Y$完备,所以$\{T_nx\}$在$Y$中收敛于一个元素,记为$y$,即:
	\begin{equation*}
		\lim_{n\to+\infty}T_nx=y
	\end{equation*}
	定义算子$T:Tx=y$,即:
	\begin{equation*}
		Tx=\lim_{n\to+\infty}T_nx=y
	\end{equation*}
	下证明$T$是定义在$X$上而值域包含在$Y$中的有界线性算子,且是$\{T_n\}$依一致算子拓扑收敛的极限。其中$T$定义域与值域是显然的。\par
	设$x_1,x_2\in X$,可得(第一行到第二行利用了$T_n$的可加性,第二行到第三行利用了范数极限的可加运算):
	\begin{align*}
		T(x_1+x_2)&=\lim_{n\to+\infty}T_n(x_1+x_2) \\
		&=\lim_{n\to+\infty}(T_nx_1+T_nx_2) \\
		&=\lim_{n\to+\infty}T_nx_1+\lim_{n\to+\infty}T_nx_2 \\
		&=Tx_1+Tx_2
	\end{align*}
	可加性得证。设$x_3\in X$,$\alpha$是一个数,可得(第二步到第三步利用了$T_n$的齐次性,第三步到第四步利用了范数极限的数乘运算):
	\begin{equation*}
		T(\alpha x_3)=\lim_{n\to+\infty}T_n(\alpha x_3)=\lim_{n\to+\infty}\alpha T_nx_3=\alpha\lim_{n\to+\infty}T_nx_3=\alpha Tx_3
	\end{equation*}
	齐次性得证。综上,$T$是线性算子。\par
	注意到:
	\begin{equation*}
		|\;||T_n||-||T_m||\;|\leqslant||T_n-T_m||\to0
	\end{equation*}
	所以$\{||T_n||\}$是Cauchy序列。因此$\{||T_n||\}$有界,即:
	\begin{equation*}
		\exists\;M>0,\;\forall\;n\in\mathbb{N}^+,\;||T_n||\leqslant M
	\end{equation*}
	由范数的连续性:
	\begin{equation*}
		||Tx||=||\lim_{n\to+\infty}T_nx||\leqslant\lim_{n\to+\infty}||T_n||\;||x||\leqslant M||x||
	\end{equation*}
	有界性得证。\par
	因为:
	\begin{equation*}
		\forall\;\varepsilon>0,\;\exists\;N\in\mathbb{N}^+,\;\forall\;n,m>N,\;m>n,\;\forall\;x\in X,\;||T_mx-T_nx||<\varepsilon||x||
	\end{equation*}	
	在上式中取$m\to+\infty$,可得(第一步到第二步利用范数的连续性,第三步到第四步利用$\mathscr{B}(X,Y)$中线性运算的定义,注意到此时已经证得$T\in\mathscr{B}(X,Y)$):
	\begin{align*}
		\lim_{m\to+\infty}||T_mx-T_nx||=||\lim_{m\to+\infty}T_mx-T_nx||=||Tx-T_nx||=||(T-T_n)x||\leqslant\varepsilon\;||x||
	\end{align*}
	此时即有:
	\begin{equation*}
		\forall\;x\in X,\;\frac{||(T-T_n)x||}{||x||}\leqslant\varepsilon
	\end{equation*}
	也即$||T-T_n||\to0$。依一致算子拓扑收敛得证。\par
	综上,$\mathscr{B}(X,Y)$是Banach空间。
\end{proof}

\subsection{有界线性算子的乘法}
\begin{definition}
	设$X,Y,Z$都是赋范线性空间。对$T_1\in\mathscr{B}(X,Y)$,$T_2\in\mathscr{B}(Y,Z)$,定义$T_1$与$T_2$的\gls{ProductOfOperators}$T_2T_1$如下:
	\begin{equation*}
		\forall\;x\in X,\;(T_2T_1)x=T_2(T_1x)
	\end{equation*}	
	$T^n$表示$n$个$T$相乘,$T^0$表示单位算子$I$。
\end{definition}
\subsubsection{算子乘法的性质}
\begin{property}
	设$X,Y,Z,K$都是赋范线性空间。
	\begin{enumerate}
		\item 结合律:
		\begin{equation*}
			(T_3T_2)T_1=T_3(T_2T_1),\;(\alpha T_2)T_1=\alpha T_2T_1,\;
			T_2(\alpha T_1)=\alpha(T_2T_1)
		\end{equation*}
		这里$T_1\in\mathscr{B}(X,Y),\;T_2\in\mathscr{B}_(Y,Z),\;T_3\in\mathscr{B}(Z,K)$。
		\item 分配律:
		\begin{equation*}
			T_3(T_1+T_2)=T_3T_1+T_3T_2,\;(T_2+T_3)T_1=T_2T_1+T_3T_1
		\end{equation*}
		第一个式子中$T_1,T_2\in\mathscr{B}(X,Y),\;T_3\in\mathscr{B}(Y,Z)$。\\
		第二个式子中$T_3,T_2\in\mathscr{B}(Y,Z),\;T_1\in\mathscr{B}(X,Y)$。
		\item 当$T_1\in\mathscr{B}(X,Y),\;T_2\in\mathscr{B}(Y,Z)$时,$T_2T_1\in\mathscr{B}(X,Z)$,且有$||T_2T_1||\leqslant||T_2||\;||T_1||$。
	\end{enumerate}
\end{property}
\begin{proof}
	只证(3)。(1)(2)证了没多大意思。对任意的$x\in X$,有:
	\begin{equation*}
		||T_2T_1x||=||T_2(T_1)x||\leqslant||T_2||\;||T_1x||\leqslant||T_2||\;||T_1||\;||x||
	\end{equation*}
	即:
	\begin{equation*}
		\frac{||T_2T_1x||}{||x||}\leqslant||T_2||\;||T_1||
	\end{equation*}
	由上确界的不等式性可得:
	\begin{equation*}
		||T_2T_1||\leqslant||T_2||\;||T_1||\qedhere
	\end{equation*}
\end{proof}






















