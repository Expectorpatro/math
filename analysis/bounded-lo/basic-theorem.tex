\section{Banach空间上有界线性算子的四大定理}

\subsection{开映射定理}
\begin{definition}
	设$X$和$Y$都是赋范线性空间。如果映射$T$把$X$中的任何开集映成$Y$中的开集,则称$T$为\gls{OpenMap}。
\end{definition}
\begin{theorem}[开映射定理]
	设$X$和$Y$都是Banach空间。如果有界线性算子$T$把$X$映成$Y$中的某个第二型集$F$,则:
	\begin{enumerate}
		\item $\exists\;K>0,\;\forall\;y\in Y,\;\exists\;x\in X,\;Tx=y,\;||x||\leqslant K||Tx||$。
		\item $T$是一个开映射。
	\end{enumerate}
\end{theorem}

\subsection{逆算子定理}
\begin{theorem}
	设$X$和$Y$都是Banach空间。如果有界线性算子$T$把$X$映成$Y$中的某个第二型集$F$且$T$是单射,则$T$存在有界逆算子。
\end{theorem}
\begin{proof}
	由开映射定理,$T$是满射,所以$T$是双射,因此$T^{-1}$存在。任取$Y$中一个有界集,记为$E$,由开映射定理:
	\begin{equation*}
		\exists\;K>0,\;\forall\;y\in E,\;||T^{-1}y||\leqslant K||y||
	\end{equation*}
	因为$E$有界,所以$\exists\;M>0,\;||y||\leqslant M$,于是$\forall\;y\in E,\;||T^{-1}y||\leqslant KM$,即$T^{-1}E$是一个有界集。由$E$的任意性,$T^{-1}$将任意有界集映成有界集,故$T^{-1}$有界。
\end{proof}
\subsubsection{等价范数的定义}
\begin{definition}
	$X$是一个线性空间,$||\cdot||_1,\;||\cdot||_2$是定义在$X$上的两个范数,$X$按照这两个范数均为赋范线性空间。若:
	\begin{equation*}
		\exists\;K_1,K_2>0,\;\forall\;x\in X,\;K_1||x||_1\leqslant||x_2||\leqslant K_2||x||_1
	\end{equation*}
	则称范数$||\cdot||_1,\;||\cdot||_2$等价。
\end{definition}
\subsubsection{等价范数与拓扑同构}
\begin{theorem}
	$X$是一个线性空间,$||\cdot||_1,\;||\cdot||_2$是定义在$X$上的两个范数,$X$按照这两个范数均为赋范线性空间。$(X,||\cdot||_1)$与$(X,||\cdot||_2)$拓扑同构的充分必要条件为$||\cdot||_1,\;||\cdot||_2$等价。
\end{theorem}
\begin{proof}
	设$T$为$X$上的单位算子,显然$T$是一个线性算子。\par
	必要性:因为$(X,||\cdot||_1)$与$(X,||\cdot||_2)$拓扑同构,所以$T$连续,进而$T$有界,即$\exists\;K_2>0,\;\forall\;x\in X,\;||Tx||\leqslant K_2||x||$,也即$||x||_2\leqslant K_2||x||_1$。因为$T^{-1}$连续,所以$T^{-1}$有界,即$\exists\;M>0,\;\forall\;x\in X,\;||T^{-1}x||\leqslant M||x||$,即$||x||_1\leqslant M||x||_2$,令$K_1=\frac{1}{M}$即可得到$K_1||x||_1\leqslant||x||_2$。综上,$\exists\;K_1,K_2>0,\;\forall\;x\in X,\;K_1||x||_1\leqslant||x_2||\leqslant||x||_1$,即范数$||\cdot||_1,\;||\cdot||_2$等价。\par
	充分性:任取$(X,||\cdot||_1)$中的一个有界集$E$,有$\exists\;M>0,\;\forall\;x\in E,\;||x||_1<M$。因为范数$||\cdot||_1,\;||\cdot||_2$等价,所以$\exists\;K_2>0,\;\forall\;x\in(X,||\cdot||_1),\;||Tx||=||x||_2\leqslant K_2||x||_1=K_2M$,即$T$是有界线性算子,所以$T$连续。同理,$T^{-1}$也是连续有界线性算子。综上,$(X,||\cdot||_1)$与$(X,||\cdot||_2)$拓扑同构。
\end{proof}
\begin{corollary}
	$(X,||\cdot||_1)$与$(X,||\cdot||_2)$都是Banach空间。若:
	\begin{equation*}
		\exists\;K>0,\;\forall\;x\in X,\;||x||_2\leqslant K||x||_1
	\end{equation*}
	则$||\cdot||_1,\;||\cdot||_2$等价,$(X,||\cdot||_1)$与$(X,||\cdot||_2)$拓扑同构。
\end{corollary}
\begin{proof}
	设$T$为$X$上的单位算子,显然$T$是一个双射并且是一个线性算子。由条件中的不等式可知$T$有界,因此$T$连续。$(X,||\cdot||_2)$是Banach空间,所以$(X,||\cdot||_2)$是第二型集。因为$T$把$(X,||\cdot||_1)$映成第二型集$(X,||\cdot||_2)$且$T$是单射,由逆算子定理,$T^{-1}$存在且有界。易证$T^{-1}$也是一个线性算子,因此$T^{-1}$连续。综上,$(X,||\cdot||_1)$与$(X,||\cdot||_2)$拓扑同构,从而$||\cdot||_1,\;||\cdot||_2$等价。
\end{proof}

\subsection{闭图像定理}
若$X$和$Y$都是赋范线性空间,则在下列讨论中,$X$和$Y$的直和$X\oplus Y$中的范数均定义为:$||(x,y)||=||x||+||y||,\;(x,y)\in X\oplus Y$。
\subsubsection{图像的定义}
\begin{definition}
	$X$和$Y$都是线性空间,$T$是定义在$X$的某个子空间$E$上且值域包含在$Y$中的线性算子。$X\oplus Y$中所有形如$(x,Tx),\;x\in E$的元素构成的集合$\{(x,Tx)\}$称为$T$的\gls{Graph},记为$G_T$。
\end{definition}
\subsubsection{闭算子的定义}
\begin{definition}
	$X$和$Y$都是线性空间,$T$是定义在$X$的某个子空间$E$上且值域包含在$Y$中的线性算子。如果$T$的图像$G_T$是$X\oplus Y$中的闭子空间,则称$T$为\gls{ClosedOperator}。
\end{definition}
\begin{theorem}
	$X$和$Y$都是赋范线性空间,$T$是定义在$X$的某个子空间$E$上且值域包含在$Y$中的线性算子。$T$为闭算子的充分必要条件为:对任意的$\{x_n\}\subset E$,若$\{x_n\}\to x\in X,\;\{Tx_n\}\to y\in Y$,则$(x,y)\in G_T$。
\end{theorem}
\begin{proof}
	充分性:任取$(x,y)\in\overline{G_T}$,则:
	\begin{equation*}
		\exists\;\{x_n\}\in E,\;\{(x_n,Tx_n)\}\to(x,y)
	\end{equation*}
	由$X\oplus Y$中范数的定义可得到:
	\begin{equation*}
		||(x,y)-(x_n,Tx_n)||=||(x-x_n,y-Tx_n)||=||x-x_n||+||y-Tx_n||\to0
	\end{equation*}
	所以$\{x_n\}\to x,\;\{Tx_n\}\to y$,$(x,y)\in G_T$。由$(x,y)$的任意性,$T$是闭算子。\par
	必要性:对任意的$\{x_n\}\subset E$,若$\{x_n\}\to x\in X,\;\{Tx_n\}\to y\in Y$,就有:
	\begin{equation*}
		||(x,y)-(x_n,Tx_n)||=||(x-x_n,y-Tx_n)||=||x-x_n||+||y-Tx_n||\to0
	\end{equation*}
	所以$\{(x_n,Tx_n)\}\to(x,y)$。因为$T$是闭算子,所以$(x,y)\in G_T$。
\end{proof}
\subsubsection{有界线性算子何时是闭算子}
\begin{theorem}
	$X$和$Y$都是赋范线性空间,$T$是定义在$X$的某个子空间$E$上且值域包含在$Y$中的有界线性算子。$T$是闭算子的充分必要条件为$E$是$X$的闭子空间。
\end{theorem}
\begin{proof}
	充分性:任取$(x,y)\in\overline{G_T}$,则$\exists\;\{(x_n,y_n)\}\in G_T$并且$\{(x_n,y_n)\}\to (x,y)$。由$X\oplus Y$中范数的定义可得到$\{x_n\}\to x,\;\{Tx_n\}\to y$。由于$E$是$X$的闭子空间,所以$x\in E$。因为$T$是有界线性算子,所以$T$是连续的,于是$Tx=y$,$(x,y)\in G_T$。由$(x,y)$的任意性,$T$是闭算子。\par
	必要性:任取$x\in\overline{E}$,则$\exists\;\{x_n\}\in E$并且$\{x_n\}\to x$。因为$T$是一个有界线性算子,所以$T$连续,所以$\{Tx_n\}\to Tx$。由$X\oplus Y$中范数的定义,$\{(x_n,Tx_n)\}\to(x,y)$。因为$T$是一个闭算子,所以$(x,y)\in G_T$,即$x\in E$,故$E$是$X$的闭子空间。
\end{proof}
\subsubsection{闭的线性算子何时有界}
\begin{theorem}[closed graph theorem]
	设$X$和$Y$都是Banach空间,$T$是$X$到$Y$的线性算子。$T$有界的充分必要条件为$T$是闭算子。
\end{theorem}
\begin{proof}
	充分性:因为$X$和$Y$都是Banach空间,所以$X\oplus Y$也是Banach空间。因为$T$是闭算子,所以$G_T$是$X\oplus Y$的闭子空间,于是$G_T$也是Banach空间。定义$G_T$到$X$的算子$\tilde{T}$:
	\begin{equation*}
		\tilde{T}(x,Tx)=x
	\end{equation*}
	显然$\tilde{T}$是一个满映射。若$\tilde{T}$不是一个单射,则$\exists\;x\in X$使得$\exists\;y_1,y_2\in Y,\;y_1\ne y_2,\;Tx=y_1,Tx=y_2$,那么$T$就不是一个映射,所以$\tilde{T}$是一个单射。于是$\tilde{T}$是一个双射。\par
	设$a$是一个数,$(y,Ty)\in G_T$。由$T$的线性性可推出:
	\begin{gather*}
		\tilde{T}[(x,Tx)+(y,Ty)]=\tilde{T}(x+y,Tx+Ty)=\tilde{T}[x+y,T(x+y)]=x+y=\tilde{T}(x,Tx)+\tilde{T}(y,Ty) \\
		\tilde{T}[a(x,Tx)]=\tilde{T}(ax,aTx)=\tilde{T}(ax,Tax)=ax=a\tilde{T}(x,Tx)
	\end{gather*}
	即$\tilde{T}$是一个线性算子。\par
	由逆算子定理,$\tilde{T}$存在有界的逆算子$\tilde{T}^{-1}$。于是对任意的$x\in X$,因为$(x,Tx)=\tilde{T}^{-1}x$,所以:
	\begin{equation*}
		||(x,Tx)||\leqslant||\tilde{T}^{-1}||\;||x||
	\end{equation*}
	即:
	\begin{equation*}
		||Tx||\leqslant||\tilde{T}^{-1}||\;||x||
	\end{equation*}
	所以$T$有界。\par
	必要性:任取$(x,y)\in\overline{G_T}$,则$\exists\;\{(x_n,y_n)\}\in G_T$并且$\{(x_n,y_n)\}\to (x,y)$。由$X\oplus Y$中范数的定义可得到$\{x_n\}\to x,\;\{Tx_n\}\to y$。因为$T$有界,所以$T$连续,于是$\{Tx_n\}\to Tx=y$,所以$(x,y)\in G_T$。由$(x,y)$的任意性,$G_T$是闭子空间,$T$是闭算子。
\end{proof}
\subsection{共鸣定理}
\begin{definition}
	设$X$是一个线性空间,$T$是定义在$E$上的泛函。若对任意的$x,y\in X$,有:
	\begin{equation*}
		T(x+y)\leqslant Tx+Ty
	\end{equation*}
	则称$T$是次可加的。若对任意的$\alpha\geqslant0$和$\forall\;x\in X$,有:
	\begin{equation*}
		T(\alpha x)=\alpha Tx
	\end{equation*}
	则称$T$是正齐次的。
\end{definition}
\begin{theorem}[resonance theorem]
	$\{T_i:i\in I\}$是Banach空间$X$到赋范线性空间$Y$上的有界线性算子族。若:
	\begin{equation*}
		\forall\;x\in X,\;\sup_{i\in I}||T_ix||<+\infty
	\end{equation*}
	则$\{||T_i||:i\in I\}$有界,即$\{T_i:i\in I\}$一致有界。
\end{theorem}
\begin{theorem}
	$\{T_n\}$是Banach空间$X$到Banach空间$Y$上的有界线性算子列。$\{T_n\}$按强算子拓扑收敛于某一算子$T\in\mathscr{B}(X,Y)$的充分必要条件为:
	\begin{enumerate}
		\item $\{T_n\}$一致有界。
		\item 存在$X$中的稠密子集$E$,使得对任意的$x\in E$,$\{T_nx\}$在$Y$中收敛。
	\end{enumerate}
	此时还有:
	\begin{equation*}
		||T||\leqslant\varliminf||T_n||
	\end{equation*}
\end{theorem}
\begin{proof}
	必要性:
\end{proof}
\begin{theorem}
	设$X,Y$都是Banach空间,则$\mathscr{B}(X,Y)$关于算子列按强算子拓扑收敛是完备的。
\end{theorem}