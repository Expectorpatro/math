\section{线性算子}

\subsection{线性算子的定义}
\begin{definition}
	设$X$和$Y$都是实(复)线性空间,$T$是由$X$的某个子空间$E$到线性空间$Y$的一个映射,如果对任意的$x,y\in E$,有:
	\begin{equation*}
		T(x+y)=Tx+Ty
	\end{equation*}
	则称$T$是可加的。若对任意的$\alpha\in\mathbb{R}$(或$\forall\;\alpha\in\mathbb{C}$)即$\forall\;x\in E$,有:
	\begin{equation*}
		T(\alpha x)=\alpha Tx
	\end{equation*}
	则称$T$是齐次的。称可加齐次映射为\gls{LinearOperator}。
\end{definition}

\subsection{线性算子的连续性}
\begin{definition}
	设$X$和$Y$都是实(复)的赋范线性空间,$T$是由$X$的某个子空间$E$到线性空间$Y$的线性算子。如果$T$是连续的,则称$T$为\gls{ContinuousLinearOperator}。
\end{definition}
\subsubsection{连续可加等价于连续线性}
\begin{theorem}\label{theo:continuous+additive=homogeneous-R-normed-ls}
	设$X$和$Y$都是实赋范线性空间,$T$是由$X$的某个子空间$E$到线性空间$Y$的连续可加算子,则$T$是连续线性算子。
\end{theorem}
\begin{proof}
	由可加性可以得到:
	\begin{equation*}
		\forall\;n\in\mathbb{N}^+,\;\forall\;x\in X,\;T(nx)=nTx
	\end{equation*}	
	所以$T$对正整数集满足齐次性。\par
	取$x_1=\dfrac{x}{n}$,代入上式可得:
	\begin{equation*}
		Tx=nT\left(\frac{x}{n}\right)
	\end{equation*}
	即:
	\begin{equation*}
		T\left(\frac{x}{n}\right)=\frac{Tx}{n}
	\end{equation*}
	所以$T$对正有理数集满足齐次性。\par
	取$\mathbf{0}_X\in X$,由$T$对正有理数的齐次性,$T\mathbf{0}_X=T(2\times\mathbf{0}_X)=2T\mathbf{0}_X$,所以$T\mathbf{0}_X=\mathbf{0}_Y$。由$T$的可加性可得:
	\begin{equation*}
		T\mathbf{0}_X=T[x+(-x)]=Tx+T(-x)=\mathbf{0}_Y\in Y
	\end{equation*}
	即:
	\begin{equation*}
		T(-x)=-Tx
	\end{equation*}
	所以$T$对有理数集满足齐次性。\par
	因为任意无理数都可用一个有理数序列去逼近,由$T$的连续性,显然$T$对无理数集也满足齐次性。\par
	综上,$T$是一个连续线性算子。
\end{proof}
\begin{theorem}
	设$X$和$Y$都是复赋范线性空间,$T$是由$X$的某个子空间$E$到线性空间$Y$的连续可加算子,且$T(ix)=iTx$,则$T$是连续线性算子。
\end{theorem}
\begin{proof}
	设$a=a_1+ia_2\in\mathbb{C}$,由\cref{theo:continuous+additive=homogeneous-R-normed-ls}:
	\begin{align*}
		T(ax)&=T[(a_1+ia_2)x]=T(a_1x+ia_2x)=T(a_1x)+T(ia_2x) \\
		&=a_1Tx+iT(a_2x)=a_1Tx+ia_2Tx=(a_1+ia_2)Tx=aTx\qedhere
	\end{align*}
\end{proof}

\subsection{线性算子的有界性}
\begin{definition}
	如果$T$将$E$中的任一有界集映射为$Y$中的有界集,则称$T$是\gls{BoundedLinearOperator}。如果存在$E$中的有界集$A$,使得$TA$是$Y$中的无界集,则称$T$是\gls{UnboundedLinearOperator}。
\end{definition}
\subsubsection{线性算子有界的等价定义}
\begin{theorem}
	设$X$和$Y$都是实赋范线性空间,$T$是由$X$的某个子空间$E$到线性空间$Y$的线性算子,则$T$有界的充分必要条件为:
	\begin{equation*}
		\exists\;M>0,\;\forall\;x\in E,\;||Tx||\leqslant M||x||
	\end{equation*}
\end{theorem}
\begin{proof}
	(1)充分性:任取$A\subset E$是一个有界集\info{有时间思考一下有界集范数有界的证明应该放在什么位置},则:
	\begin{equation*}
		\exists\;K>0,\;\forall\;x\in A,\;||x||\leqslant K
	\end{equation*}
	因此:
	\begin{equation*}
		\exists\;M>0,\;\exists\;K>0,\;\forall\;x\in A,\;||Tx||\leqslant M||x||\leqslant MK
	\end{equation*}
	所以$TA$是$Y$中的有界集。由$A$的任意性,$T$有界。\par
	(2)必要性:在$E$中取单位球面$S=\{x\in E:||x||=1\}$,显然$S$有界,因此$TS$有界。于是:
	\begin{equation*}
		\exists\;M>0,\;\forall\;x\in S,\;||Tx||\leqslant M
	\end{equation*}
	设$x$为$E$中任意非$\mathbf{0}$的元素,则:
	\begin{equation*}
		\frac{x}{||x||}\in S
	\end{equation*}
	所以有:
	\begin{equation*}
		\left\|T\left(\frac{x}{||x||}\right)\right\|\leqslant M
	\end{equation*}
	由$T$的齐次性:
	\begin{equation*}
		||Tx||\leqslant M||x||
	\end{equation*}
	因为$X$中的零元必然被线性算子$T$映射为$Y$中的零元\info{单独把这个证明拎出来,然后把连续可加等于连续线性以及Hilbert空间那里的证明放在一起考虑},所以当$x=\mathbf{0}$时,$Tx$必然是$Y$中的零元,上式显然也成立。
\end{proof}

\subsection{线性算子有界与连续的关系}
\begin{theorem}
	设$X$和$Y$都是实赋范线性空间,$T$是由$X$的某个子空间$E$到线性空间$Y$的线性算子,则以下陈述等价:
	\begin{enumerate}
		\item $T$连续。
		\item $T$在$E$中某给定点$x_0$连续。
		\item $T$有界。
	\end{enumerate}
\end{theorem}
\begin{proof}
	$(1)\to(2)$是显然的。下证$(2)\to(3)$。\par
	因为$T$在$x_0$连续,因此对$\varepsilon=1,\;\exists\;\delta>0,\;\forall\;x\in E$,当$||x-x_0||\leqslant\delta$时,有:
	\begin{equation*}
		||Tx-Tx_0||=||T(x-x_0)||\leqslant\varepsilon=1
	\end{equation*}
	任取$x_1\ne x_0$,且$x_1\in E$,因为:
	\begin{equation*}
		\left\|\frac{\delta(x_1-x_0)}{||x_1-x_0||}\right\|=\delta\leqslant\delta
	\end{equation*}
	所以:
	\begin{equation*}
		\left\|T\left[\frac{\delta(x_1-x_0)}{||x_1-x_0||}\right]\right\|\leqslant1
	\end{equation*}
	作变形即有:
	\begin{equation*}
		||T(x_1-x_0)||\leqslant\frac{1}{\delta}||x_1-x_0||
	\end{equation*}
	因为$X$中的零元必然被线性算子$T$映射为$Y$中的零元,所以
	\begin{equation*}
		||T(x_0-x_0)||=0=\frac{1}{\delta}||x_0-x_0||
	\end{equation*}
	所以对任意的$x\in E$,有:
	\begin{equation*}
		||T(x-x_0)||\leqslant\frac{1}{\delta}||x-x_0||
	\end{equation*}
	因此$T$定义在$E-x_0$上的时候是有界的,而$E-x_0$实际上就是$E$,所以$T$是有界的。\par
	$(3)\to(1)$:设$\{x_n\}\in E$,且有$\{x_n\}\to x\in E$,因为$T$有界,所以$\exists\;M$使得:
	\begin{equation*}
		||Tx_n-Tx||=||T(x_n-x)||\leqslant M||x_n-x||
	\end{equation*}
	对任意的$\varepsilon>0$,要使$||Tx_n-Tx||<\varepsilon$,只需$||x_n-x||<\dfrac{\varepsilon}{M}$,所以$T$连续。
\end{proof}




























