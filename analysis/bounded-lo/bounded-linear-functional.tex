\section{有界线性泛函}

\subsection{有界线性泛函的延拓}
\begin{definition}
	设$X$是一个线性空间,$f_1$和$f_2$分别为定义在$X$的子空间$E_1$和$E_2$上的线性泛函。如果满足:
	\begin{enumerate}
		\item $E_1\subset E_2$
		\item $\forall\;x\in E_1,\;f_1(x)=f_2(x)$
	\end{enumerate}
	则称$f_2$是$f_1$在$E_2$上的一个\gls{Continuation}。
\end{definition}

\begin{theorem}
	设$E$是实线性空间$X$的子空间,$f$是定义在$E$上的实线性泛函,$g$是定义在$X$上的次可加正齐次泛函,$f$和$g$之间满足:
	\begin{equation*}
		\forall\;x\in E,\;f(x)\leqslant g(x)
	\end{equation*}
	则必定存在定义在$X$上的实线性泛函$F$,满足:
	\begin{enumerate}
		\item $F$是$f$在$X$上的一个延拓;
		\item 当$x\in X$时,有$F(x)\leqslant g(x)$。
	\end{enumerate}
\end{theorem}
\begin{corollary}
	设$f$是复赋范线性空间$X$上的有界线性泛函,令:
	\begin{equation*}
		\varphi(x)=\operatorname{Re}f(x),\;\forall\;x\in X
	\end{equation*}
	则$\varphi$是$X$上的有界实线性泛函,且:
	\begin{gather*}
		f(x)=\varphi(x)-i\varphi(ix) \\
		||\varphi||\leqslant||f||
	\end{gather*}
\end{corollary}
\begin{theorem}[Hahn-Banach theorem]
	设$E$是赋范线性空间$X$的子空间,$f$是定义在$E$上的有界线性泛函,则$f$可以延拓到整个$X$上且保持范数不变。
\end{theorem}
\begin{corollary}
	设$E$是赋范线性空间$X$的子空间,$x_0\in X\backslash E$。若:
	\begin{equation*}
		d(x_0,E)=\inf_{x\in E}||x_0-x||=\delta>0
	\end{equation*}
	则存在$X$上的有界线性泛函$f$,使得:
	\begin{equation*}
		||f||=1,\;f(x_0)=\delta
	\end{equation*}
	而对任意的$x\in E$,有$f(x)=0$。\par
	于是有:
	\begin{enumerate}
		\item $x_0\in\overline{E}$的充分必要条件为:对$X$上任一满足对任意的$x\in E$,都有$f(x)=0$的有界线性泛函$f$,有$f(x_0)=0$。
		\item 设$x_0\in X$,$E\subset X$,则$x_0$可以用$E$中元素的线性组合以任意精度逼近的充分必要条件是对$X$上任一有界线性泛函$f$,当对任意的$x\in E$,都有$f(x)=0$时,$f(x_0)=0$。
	\end{enumerate}
\end{corollary}
\begin{proof}
	(2)\textbf{充分性:}如果此时$x_0\notin\overline{E}$,则$d(x_0,E)>0$,由前述,存在$X$上的有界线性泛函$f$,对任意的$x\in E$,有$f(x)=0$,且$f(x_0)=d(x_0,E)>0$,矛盾。\par
	\textbf{必要性:}若$x_0\in E$,在这种情况下则显然$f(x_0)=0$;若$x_0\in E'$,则存在$\{x_n\}\subset E$,使得$\{x_n\}\to x_0$。由有界线性算子的连续性可得:
	\begin{equation*}
		f(x_0)=f\left(\lim_{n\to+\infty}x_n\right)=\lim_{n\to+\infty}f(x_n)=0
	\end{equation*}\par
	(3)设$E_1$是由$E$张成的子空间,$x_0$可以用$E$中元素的线性组合以任意精度逼近的充分必要条件是$x_0\in\overline{E_1}$。由(2)可得,$x_0\in\overline{E_1}$的充分必要条件为:对$X$上任一满足对任意的$x\in E_1$,都有$f(x)=0$的有界线性泛函$f$,有$f(x_0)=0$。因为对$E$生成的子空间$E_1$满足该条件,则显然对$E$也满足(只要把$E$中某个元素的系数取为$1$,其它元素系数取为$0$即可)。
\end{proof}
\begin{corollary}
	设$X$是赋范线性空间,且$X\ne\{\mathbf{0}\}$,则对任意的$x\in X,\;x\ne\mathbf{0}$,存在$X$上的有界线性泛函$f$使得:
	\begin{equation*}
		||f||=1,\;f(x_0)=||x_0||
	\end{equation*}
\end{corollary}
\begin{proof}
	$\{\mathbf{0}\}$是$X$的一个子空间,由上一个推论可直接得到对任意的$x\in X,\;x\ne\mathbf{0}$,存在$X$上的有界线性泛函$f$使得:
	\begin{equation*}
		||f||=1,\;f(x_0)=d(x_0,\{\mathbf{0}\})=||x_0-\mathbf{0}||=||x_0||\qedhere
	\end{equation*}
\end{proof}





