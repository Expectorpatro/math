\section{紧集与紧度量空间}
\begin{definition}
	$(X,\rho)$是一个度量空间,$A$是$X$的一个子集。如果$A$的每个点列都有一个收敛子列收敛于$A$中的某一点,则称$A$是\gls{CompactSet}。若$X$是紧集,则称$X$是\gls{CompactMetricSpace}。
\end{definition}

\subsection{紧集的性质}
\begin{theorem}
	度量空间中的紧集是有界且可分的。
\end{theorem}
\begin{proof}
	因为准紧集是有界可分的,且紧集必然是准紧集,所以紧集也是有界可分的。
\end{proof}

\subsection{紧度量空间的性质}
\subsubsection{完备性}
\begin{theorem}
	任一度量空间中的紧集都是完备的。紧度量空间是完备度量空间。
\end{theorem}
\begin{proof}
	设$X$是一个紧集,$\{x_n\}$是$X$中的一个Cauchy点列。由紧集定义可得出$\{x_n\}$存在收敛的子列$\{x_{n_k}\}$,设$\{x_{n_k}\}\to a$,则:
	\begin{equation*}
		\rho(x_n,a)\leqslant\rho(x_n,x_{n_k})+\rho(x_{n_k},a)\to0
	\end{equation*}
	于是$\{x_n\}\to a$。由$\{x_n\}$的任意性,$X$是完备的。
\end{proof}
\begin{theorem}
	$(X,\rho)$是一个度量空间,$\{E_n\}$是$X$中的一列非空紧集,满足:
	\begin{equation*}
		E_1\supset E_2\supset\dots\supset E_n\supset\dots
	\end{equation*}
	则$\underset{n=1}{\overset{+\infty}{\cap}}E_n\ne\varnothing$。
\end{theorem}
\begin{proof}
	在每个$E_n$中选择一点$x_n$,构成序列$\{x_n\}$。因为$\forall\;n\in\mathbb{N}^+,\;x_n\in E_n\subset E_1$,又$E_1$是紧集,所以$\{x_n\}$存在子列$\{x_{n_k}\}\rightarrow x_0\in E_1$。对任意的$n\in\mathbb{N}^+$,当$n_k>n$时,有$x_{n_k}\in E_{n_k}\subset E_n$,又因收敛点列必为Cauchy点列、$E_n$完备,所以$x_0\in E_n$。由$n$的任意性,$x_0\in\underset{n=1}{\overset{+\infty}{\cap}}E_n$,所以$\underset{n=1}{\overset{+\infty}{\cap}}E_n\ne\varnothing$。
\end{proof}

\subsection{紧集的充要条件}
\subsubsection{有限覆盖}
\begin{definition}
	$(X,\rho)$是一个度量空间,$A$是$X$的子集,$\{G_i\}_{i\in I}$(其中$I$是一个指标集)是$X$中某些开集组成的集族。如果:
	\begin{equation*}
		A\subset\underset{i\in I}{\cup}G_i
	\end{equation*}
	则称$\{G_i\}_{i\in I}$为$A$的\gls{OpenCover}。如果$I$是有限集,则称$\{G_i\}_{i\in I}$为$A$的\gls{FiniteOpenCover}。
\end{definition}
\begin{theorem}[有限覆盖定理]
	度量空间$(X,\rho)$的子集$A$是紧集的充分必要条件是从$A$的任一开覆盖$\{G_i\}_{i\in I}$中必可选出一个有限子覆盖。
\end{theorem}
\begin{proof}
	(1)必要性:先来证明$\exists\;\varepsilon>0,\;\forall\;x\in A,\;\exists\;i\in I,\;U(x,\varepsilon)\in G_i$。若不成立,则:
	\begin{equation*}
		\forall\;\varepsilon>0,\;\exists\;x\in A,\;\forall\;i\in I,\;U(x,\varepsilon)\notin G_i
	\end{equation*}
	依次选择$\varepsilon_n=\frac{1}{2^n},\;n\in\mathbb{N}^+$,即可构造出序列$\{x_n\}$满足其中的每个元素都不在任何$G_i$中。因为$A$是紧集,所以$\{x_n\}$存在收敛于$A$中某点$x_0$的子列$\{x_{n_k}\}$。又因为$A\subset\underset{i\in I}{\cup}G_i$,所以$\exists\;i\in I,\;x_0\in G_i$。于是可取充分大的$k$使得$U\left(x_{n_k},\dfrac{1}{2^{n_k}}\right)\subset G_i$,这与$x_{n_k}$的取法矛盾。\par
	记使上述命题成立的$\varepsilon=\varepsilon_0$。因为$A$是紧集,则$A$也是全有界集,故能选择$A$中有限个点,使得对任意的$\varepsilon>0$,这有限个点的开$\varepsilon$邻域能够包含整个$A$。取$\varepsilon=\varepsilon_0$,此时只要对这有限个点的开邻域选择对应的$G_i$,即可选出有限子覆盖。\par
	(2)充分性:设$\{x_n\}$是$A$中的一个点列。如果$\{x_n\}$没有子列在$A$中收敛,则:
	\begin{equation*}
		\forall\;y\in A,\;\exists\;\delta_y>0,\;\exists\;n_y\in\mathbb{N}^+,\;\forall\;n>n_y,\;x_n\notin U(y,\delta_y)
	\end{equation*}
	显然$\{U(y,\delta_y):y\in A\}\supset A$,因此可以选择$A$中有限个点,分别记为$y_1,y_2,\dots,y_{n_0}$,使得$\{U(y_i,\delta_y):i=1,2,\dots,n_0\}\supset A$。则当$n\geqslant
	\max\{n_{y_1},\;n_{y_2},\dots,n_{y_{n_0}}\}$时,$x_n\notin A$,与$\{x_n\}\in A$矛盾。
\end{proof}
\subsubsection{有限交}
\begin{definition}
	$\mathscr{F}$是度量空间$(X,\rho)$中的一个集族。如果从$\mathscr{F}$中选择任意有限个集合,它们都有非空的交集,则称$\mathscr{F}$具有有限交性质。
\end{definition}
\begin{theorem}
	度量空间$(X,\rho)$的闭子集$A$是紧集的充分必要条件是$A$中具有有限交性质的闭子集族$\mathscr{F}$有非空的交。
\end{theorem}
\begin{proof}
	设$\mathscr{F}=\{F_i\}_{i\in I}$是一个具有有限交性质的闭子集族。\par
	(1)必要性:假设该闭子集族的交集为空集。令$G_i=X\backslash F_i$,则$\{G_i\}_{i\in I}$为开集族。由:
	\begin{equation*}
		\underset{i\in I}{\overset{}{\cup}}G_i=\underset{i\in I}{\overset{}{\cup}}(X\backslash F_i)=X\backslash \underset{i\in I}{\overset{}{\cap}}F_i=X
	\end{equation*}
	可知$\{G_i\}_{i\in I}\supset A$。因为$A$是紧集,所以可从$\{G_i\}_{i\in I}$中选择出$A$的有限子覆盖$\{G_{i_j}\}_{j=1}^n$。所以:
	\begin{equation*}
		\underset{j=1}{\overset{n}{\cap}}F_{j}=\underset{j=1}{\overset{n}{\cap}}(X\backslash G_{i_j})=X\backslash\underset{j=1}{\overset{n}{\cup}}G_{i_j}\subset X\backslash A
	\end{equation*}
	又因为$F_{i_j}\subset A$,所以:
	\begin{equation*}
		\underset{j=1}{\overset{n}{\cap}}F_{i_j}\subset A
	\end{equation*}
	综合上两式可得:
	\begin{equation*}
		\underset{j=1}{\overset{n}{\cap}}F_{i_j}\subset(X\backslash A)\cap A=\varnothing
	\end{equation*}
	这与$\mathscr{F}$具有有限交性质矛盾。\par
	(2)充分性:设闭集$A$中任一具有有限交性质的闭子集族具有非空的交。我们用有限覆盖定理来证明$A$是一个紧集。设$\{G_i\}_{i\in I}$为$A$的任一开覆盖,令$F_i=A\backslash G_i$,因为$A$是闭集,所以$F_i$也是闭集。因为:
	\begin{equation*}
		\underset{i\in I}{\overset{}{\cap}}F_i=\underset{i\in I}{\overset{}{\cap}}(A\backslash G_i)=A\backslash\underset{i\in I}{\overset{}{\cup}}G_i=\varnothing
	\end{equation*}
	所以,$A$的闭子集族$\{F_i\}_{i\in I}$不具有有限交性质,否则的话根据假设应有$\underset{i\in I}{\overset{}{\cap}}F_i\ne\varnothing$。于是存在有限子集族$\{F_{i_j}\}_{j=1}^n$使得:
	\begin{equation*}
		\underset{j=1}{\overset{n}{\cap}}F_{i_j}=\varnothing
	\end{equation*}
	它对应的开集族$\{F_{i_j}\}_{j=1}^n$则满足:
	\begin{equation*}
		\underset{j=1}{\overset{n}{\cup}}G_{i_j}=\underset{j=1}{\overset{n}{\cup}}(A\backslash F_{i_j})=A\backslash\underset{j=1}{\overset{n}{\cap}}F_{i_j}=A
	\end{equation*}
	即$\{G_{i_j}\}$是$A$的一个有限开覆盖,于是$A$是紧集。
\end{proof}

\subsection{紧集上的连续映射}
\begin{theorem}
	设$(X,\rho_X)$和$(Y,\rho_Y)$为度量空间,$A$是$X$中的紧集,$T$是$A$到$Y$上的连续映射,则$TA$是$Y$中的紧集。
\end{theorem}
\begin{proof}
	设$\{y_n\}$为$TA$中的一个点列,则有$X$中的点列$\{x_n\}$使得$y_n=Tx_n,\;n\in\mathbb{N}^+$。因为$A$是紧集,所以$\{x_n\}$存在子列$\{x_{n_k}\}\to x_0\in A$。因为$T$连续,所以:
	\begin{equation*}
		\lim_{k\to+\infty}y_{n_k}=\lim_{k\to+\infty}Tx_{n_k}=T\left(\lim_{k\to+\infty}x_{n_k}\right)=Tx_0\in TA
	\end{equation*}
	所以$TA$是紧集。
\end{proof}
\begin{definition}
	设$(X,\rho_X)$和$(Y,\rho_Y)$为度量空间,$T$是$X$到$Y$上的映射。若对于任意的$\varepsilon>0$,存在只与$\varepsilon$有关的$\delta>0$,使得对任意的$x,y\in X$,只要$\rho_X(x,y)<\delta$,就有$\rho_Y(Tx,Ty)<\varepsilon$,则称$T$在$X$上\gls{UnifContinuous}。
\end{definition}
\begin{corollary}
	设$(X,\rho_X)$和$(Y,\rho_Y)$为度量空间,$A$是$X$中的紧集,$T$是$A$到$Y$上的连续泛函,则:
	\begin{enumerate}
		\item $T$在$A$上有界;
		\item $T$在$A$上可达到其上、下确界;
		\item $T$在$A$上一致连续。
	\end{enumerate}
\end{corollary}
\begin{proof}
	(1)由于$TA$是紧集,而紧集是全有界集,全有界集有界,所以$T$在$A$上有界。\par
	(2)因为$TA$是紧集,而紧集是完备的,所以$TA$是闭集,$T$在$A$上可达到其上、下确界。\par
	(3)假设此时$T$不一致连续,则存在$\varepsilon_0>0$以及点列$\{x_n\},\{y_n\}\subset A$,使得:
	\begin{equation*}
		\rho_X(x_n,y_n)\to0,\;\rho_Y(Tx_n,Ty_n)\geqslant\varepsilon_0,\;\forall\;n\in\mathbb{N}^+
	\end{equation*}
	因为$A$是紧集,所以$\{x_n\}$存在子列$\{x_{n_k}\}\to x_0\in A$,即$\rho_X(x_{n_k},x_0)\to 0$,于是:
	\begin{equation*}
		\rho_X(y_{n_k},x_0)\leqslant\rho_X(y_{n_k},x_{n_k})+\rho_X(x_{n_k},x_0)\to0
	\end{equation*}
	因为$T$是连续的,所以:
	\begin{equation*}
		\rho_Y(Tx_{n_k},Tx_0)\to0,\;\rho_Y(Ty_{n_k},Tx_0)\to0
	\end{equation*}
	于是
	\begin{equation*}
		\rho_Y(Tx_{n_k},Ty_{n_k})\leqslant\rho_Y(Tx_{n_k},Tx_0)+\rho_Y(Tx_0,Ty_{n_k})\to 0
	\end{equation*}
	与第一个式子中的第二部分矛盾,所以$T$一致连续。
\end{proof}
