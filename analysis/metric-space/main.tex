\chapter{度量空间}

\begin{definition}
	设$X$为一个非空集合。若对任意的$x,y\in X$,都存在$\rho(x,y)\in\mathbb{R}$与$x,\;y$对应,且满足以下三个条件:
	\begin{enumerate}
		\item 非负性:$\rho(x,y)\geqslant0$,等号成立当且仅当$x=y$。
		\item 对称性:$\rho(x,y)=\rho(y,x)$。
		\item 三角不等式:$\rho(x,y)\leqslant\rho(x,z)+\rho(z,y),\;\forall\;z\in X$。
	\end{enumerate}
	则称$\rho$是$X$上的一个距离,$X$是以$\rho$为距离的\gls{MetricSpace},记为$(X,\rho)$。度量空间中的元素又称之为点。
\end{definition}
\begin{definition}
	度量空间$(X,\rho)$中的点集:
	\begin{equation*}
		\{P\in X:\rho(P,P_0)<\delta\}\qquad (\delta>0)
	\end{equation*}
	被称之为点$P_0$的$\delta\;$\gls{neighbourhood}或开邻域,记为$U(P_0,\delta)$。若在上式中取小于等于号,则称之为闭邻域,用$\bar{U}(P,\delta)$表示\footnote{下面如果不作特殊说明,涉及到邻域时都指开邻域。}。
\end{definition}
\begin{property}\label{prop:Neighbourhood}
	设$(X,\rho)$是一个度量空间,$P\in X$,则有:
	\begin{enumerate}
		\item $\forall\;\delta>0,\;P\in U(P,\delta)$;
		\item $\forall\;\delta_1,\delta_2>0,\;\exists\;\delta_3>0,\;U(P,\delta_3)\subseteq U(P,\delta_1)\cap U(P,\delta_2)$;
		\item $\forall\;P_0\in X,\;P_0\ne P,\;\exists\;U(P),U(P_0),\;U(P)\cap U(P_0)=\varnothing$。
	\end{enumerate}
\end{property}
\begin{proof}
	(1)显然。(2)取$\delta_3\leqslant\min(\delta_1,\delta_2)$即可。(3)取$\delta\in\left(0,\frac{\rho(P,P_0)}{2}\right)$即可。
\end{proof}
\begin{definition}
	设$(X,\rho)$是一个度量空间,$E\subseteq X$。称:
	\begin{equation*}
		\delta_E=\sup_{x,y\in E}\rho(x,y)
	\end{equation*}
	为点集$E$的\gls{diameter}。
\end{definition}
\begin{definition}
	设$(X,\rho)$是一个度量空间,$E\subset X$。若$\delta_E<+\infty$或$E$包含在$X$中某一点的某一邻域内或$E$包含在$X$中任意一点的某一邻域内,则称$E$是$(X,\rho)$中的\gls{BoundedSet}。
\end{definition}
\begin{note}
	三者的等价性很直观:\par
	$(1)\to(3)$:任取$x\in E$,取$\delta=\delta_E$构造闭邻域$\bar{U}(x,\delta)$或取$\delta=\delta_E+\varepsilon(\forall\;\varepsilon>0)$构造开邻域$U(x,\delta)$。\par
	$(1)\to(2)$:$(1)\to(3)\to(2)$。\par
	$(2)\to(1)$:设关于该点满足条件的开邻域或闭邻域的直径为$\delta$,显然$\delta_E\leqslant\delta<+\infty$。\par 
	$(3)\to(1)$:$(3)\to(2)\to(1)$。
\end{note}

\section{度量空间上的收敛点列}
\begin{definition}
	设$(X,\rho)$是一个度量空间,从正整数集$\mathbb{N}^+$到$X$的一个映射,相当于用正整数编号的$X$中的点$x_1,x_2,\cdots$被称为一个\gls{SeqOfPoints},记为$\{x_n\}$。
\end{definition}
\begin{definition}
	设$\{x_n\}$是度量空间$(X,\rho)$中的一个点列,而
	\begin{equation*}
		n_1<n_2<\cdots<n_k<n_{k+1}<\cdots
	\end{equation*}
	是一串严格递增的自然数,则
	\begin{equation*}
		x_{n_1}<x_{n_2}<\cdots<x_{n_k}<x_{n_{k+1}}<\cdots
	\end{equation*}
	也形成一个$(X,\rho)$中的点列,我们把$\{x_{n_k}\}$称之为点列$\{x_n\}$的一个\gls{subsequence}。
\end{definition}
\begin{definition}
	设$\{x_n\}$是度量空间$(X,\rho)$中的点列,$x\in X$。如果对任意的$\varepsilon>0$,$\exists\; N\in\mathbb{N}^+$,当$n>N$时有:
	\begin{equation*}
		\rho(x_n,x)<\varepsilon
	\end{equation*}
	则称$\{x_n\}$是度量空间$(X,\rho)$中的\gls{ConvergentSeqOfPoints},$x$是点列$\{x_n\}$的\gls{limit},记作$\{x_n\}\to x$或$\lim\limits_{n\to+\infty}x_n=x$。
\end{definition}
\begin{property}\label{prop:ConvergentSeqOfPoints}
	设$\{x_n\}$是度量空间$(X,\rho)$中的一个收敛点列,则:
	\begin{enumerate}
		\item $\{x_n\}$的极限是唯一的;
		\item 对任意的$ y\in X$,数列$\{\rho(x_n,y)\}$有界;
		\item $\{x_n\}$是有界点集;
		\item $\{x_n\}$是收敛点列的充要条件为$\{x_n\}$的任一子列是收敛点列且极限相同;
		\item 设$\{x_n\},\{y_n\}\subseteq X$,$\{x_n\}\to x$,若存在$N_1\in\mathbb{N}^+$使得当$n>N_1$时有$\rho(y_n,x)\leqslant\rho(x_n,x)$,则$\{y_n\}\to x$;
	\end{enumerate}
\end{property}
\begin{proof}
	(1)假设极限不唯一,$\{x_n\}$既收敛到$a$又收敛到$b$,则对任意的$\varepsilon>0$,$\exists\; N_1\in\mathbb{N}^+$,当$n>N_1$时有$\rho(x_n,a)<\frac{\varepsilon}{2}$;$\exists\; N_2\in\mathbb{N}^+$,当$n>N_2$时有$\rho(x_n,b)<\frac{\varepsilon}{2}$。取$N=\max\{N1,N2\}$,则当$n>N$时,有:
	\begin{equation*}
		\rho(a,b)\leqslant\rho(a,x_n)+\rho(x_n,b)<\varepsilon
	\end{equation*}
	即$a=b$。\par
	(2)设$\{x_n\}\to x$,由距离的定义:
	\begin{equation*}
		\rho(x_n,y)\leqslant\rho(x_n,x)+\rho(x,y)
	\end{equation*}
	由于$\{x_n\}$收敛,所以对$\varepsilon=1$:
	\begin{equation*}
		\exists\;N\in\mathbb{N}^+,\;\forall\;n>N,\;\rho(x_n,x)<\varepsilon=1
	\end{equation*}
	取$K=\max\{\rho(x_1,x),\rho(x_2,x),\dots,\rho(x_N,x),1\}$,则有:
	\begin{equation*}
		\forall\;n\in\mathbb{N}^+,\;\rho(x_n,y)\leqslant K+\rho(x,y)
	\end{equation*}
	即数列$\{\rho(x_n,y)\}$有界。\par
	(3)任取$y\in X$,由(2)数列$\{\rho(x_n,y)\}$有界,即$\exists\;\delta>0,\;\forall\;n\in\mathbb{N}^+,\;\rho(x_n,y)<\delta$,则$\{x_n\}\subseteq U(y,\delta)$,即$\{x_n\}$有界。\par
	(4)\textbf{必要性:}设$\{x_n\}$的极限为$x$,则对任意的$\varepsilon>0$,$\exists\; N\in\mathbb{N}^+$,当$n>N$时有$\rho(x_n,x)<\varepsilon$,当$k>N$时就有$n_k\geqslant k>N$,也即$\rho(x_{n_k},x)<\varepsilon$。\par
	\textbf{充分性:}$\;\{x_n\}$也是它自己的一个子列。\par
	(5)因为$\{x_n\}\to x$,所以对任意的$\varepsilon>0$,存在$N_2\in\mathbb{N}^+$使得当$n>N_2$时有$\rho(x_n,x)<\varepsilon$,取$N=\max\{N_1,N_2\}$即有$\rho(y_n,x)\leqslant\rho(x_n,x)<\varepsilon$,于是$\{y_n\}\to x$。
\end{proof}

\section{度量空间上的点集}
\subsection{基础知识}
\subsubsection{点的定义}
\begin{definition}
	设$(X,\rho)$是一个度量空间,$E\subseteq X$。若对于点$P\in X$,存在$U(P)\subseteq E$,则称点$P$为$E$的\gls{InteriorP}。
\end{definition}
\begin{definition}
	设$(X,\rho)$是一个度量空间,$E\subseteq X$。若点$P\in X$为$E^c$的内点,则称点$P$为$E$的\gls{ExteriorP}。
\end{definition}
\begin{definition}
	设$(X,\rho)$是一个度量空间,$E\subseteq X$。若对于点$P\in X$,它的任意邻域中都既有$E$中的点,又有$E^c$中的点,则称点$P$为$E$的\gls{BoundaryP}。
\end{definition}
\begin{definition}
	设$(X,\rho)$是一个度量空间,$E\subseteq X$。若点$P\in X$满足以下任一条件:
	\begin{enumerate}
		\item $P$的任何邻域内都存在无穷个$E$中的点。
		\item $P$的任何邻域内都存在一个$E$中异于$P$的点。
		\item $E$中有一个不包含$P$且极限为$P$的点列。
	\end{enumerate}
	则$P$是$E$的一个\gls{LimitP}。
\end{definition}
这三个条件实际上是等价的:
\begin{proof}
	$(1)\to(2)$显然,$(3)\to(1)$显然,下证$(2)\to(3)$。
	取$\delta_n=\dfrac{1}{n}$,在$U(P,\delta_n)$中至少有一点$P_n$,$P_n\in E$且$P_n\ne P$,显然$\{P_n\}\to P$。
\end{proof}
\begin{definition}
	设$(X,\rho)$是一个度量空间,$E\subseteq X$。若点$P\in E$但不是$E$的聚点,即存在$P$的邻域$U(P)$,使得$E\cap U(P)=\{P\}$,则称$P$为$E$的\gls{IsolatedP}。
\end{definition}
\subsubsection{点集的定义}
\begin{definition}
	设$(X,\rho)$是一个度量空间,$E\subseteq X$。$E$的全体内点所组成的集合称为$E$的\gls{interior},记为$\mathring{E}$。
\end{definition}
\begin{definition}
	设$(X,\rho)$是一个度量空间,$E\subseteq X$。$E$的全体聚点所组成的集合称为$E$的\gls{DerivedSet},记为$E'$。
\end{definition}
\begin{definition}
	设$(X,\rho)$是一个度量空间,$E\subseteq X$。$E$的全体界点所组成的集合称为$E$的\gls{boarder},记为$\partial E$。
\end{definition}
\begin{definition}
	设$(X,\rho)$是一个度量空间,$E\subseteq X$。$E\cup E'$称为$E$的\gls{closure},记为$\overline{E}$。
\end{definition}
\begin{definition}
	设$(X,\rho)$是一个度量空间,$E\subseteq X$。若$E=\mathring{E}$,则称E是一个\gls{OpenSet}。
\end{definition}
\begin{definition}
	设$(X,\rho)$是一个度量空间,$E\subseteq X$。若$E'\subseteq E$,则称E是一个\gls{ClosedSet}。
\end{definition}
\begin{property}\label{prop:OpenClosedSet}
	设$(X,\rho)$是一个度量空间,$P\in X,\;E\subseteq X$,则有:
	\begin{enumerate}
		\item $\forall\;P_0\in U(P),\;\exists\;U(P_0)\subset U(P)$,即开邻域是开集。
		\item $\forall\;P_0\in[\bar{U}(P)]',\;P_0\in \bar{U}(P_0)$,即闭邻域是闭集;
		\item $\mathring{E}$是开集,$E'$和$\overline{E}$是闭集;
		\item $X$和$\varnothing$是唯二既开又闭的集合;
		\item 若$E$是开集,则$E^c$是闭集;若$E$是闭集,则$E^c$是开集;
		\item 任意个开集的并是开集,有限个开集的交是开集;任意个闭集的交是闭集,有限个闭集的并是闭集。
	\end{enumerate}
\end{property}
\begin{proof}
	(1)要使得$U(P_0)\subseteq U(P)$,则$\forall\;x\in U(P_0),\;\rho(x,P)<\delta_P$,而$\rho(x,P)<\rho(x,P_0)+\rho(P_0,P)<\delta_{P_0}+\rho(P_0,P)$,故只要使$\delta_{P_0}+\rho(P_0,P)<\delta_P$即可。注意到$\rho(P_0,P)<\delta_P$,由\info{实数的稠密性}可知这样的$\delta_{P_0}$总是存在的,于是结论成立。 \par
	(2)对任意的$P_0\in\left(\bar{U}(P)\right)'$,$\exists\;\bar{U}(P)$中的点列$\{P_n\}\to P_0$。若$P_0\notin \bar{U}(P)$,则$\rho(P_0,P)>\delta_P$。取$\varepsilon=\rho(P_0,P)-\delta_P$,则必然$\exists\;N\in\mathbb{N}^+$,当$n>N$时,有$\rho(P_n,P_0)<\varepsilon$,那么$\rho(P_n,P)>\delta_P$,即$P_n\notin U(P)$,矛盾。因此$P_0\in \bar{U}(P)$。由$P_0$的任意性,闭邻域是闭集。\par
	(3)\textbf{$\;\mathring{E}$是开集:}对任意的$P\in\mathring{E}$,存在$U(P)\subseteq E$,对任意的$Q\in U(P)$,由(1)可得存在$U(Q)\subseteq U(P)\subseteq E$,因此$Q\in\mathring{E}$,即存在$U(P)\subseteq\mathring{E}$,$P$是$\mathring{E}$的内点。由$P$的任意性,$\mathring{E}$是开集。\par
	\textbf{$\;E'$是闭集:}对任意的$P\in (E')'$,由聚点定义,任意$U(P)$中都含有$E'$中的点,任取$Q\in U(P)\cap E'$,任意$U(Q)$都含$E$中的点,而$Q\in U(P)$,由(1)可知存在$U(Q)\subset U(P)$,因此任意$U(P)$中都含有$E$中的点,即$P$是$E$的聚点,$P\in E'$。由$P$的任意性,$E'$是闭集。\par
	\textbf{$\;\overline{E}$是闭集:}$\overline{E}=E\cup E'$,由\cref{theo:cuplimitset eq2 limitsetcup},$(\overline{E})'=E'\cup (E')'$,显然$E'\subseteq \overline{E}$,而$E'$是闭集,$(E')'\subseteq E'\subset\overline{E}$。因此,$(\overline{E})'\subseteq\overline{E}$,即$\overline{E}$是闭集。\par
	(4)\par
	(5)$\;E$是开集:$\forall\; P\in (E^c)'$,应有$P\notin E$,即$P$不会是$E$的内点。否则由内点定义,存在$U(P)$使得$U(P)\subseteq E$,即$U(P)$中不含$E^c$中的点,这与点$P\in (E^c)'$矛盾。由$P$的任意性有$(E^c)'\subseteq E^c$,即$E^c$是闭集。\par
	$E$是闭集:$\forall\; P\in E^c$,应有$P$是$E^c$的一个内点。否则任意$U(P)$中都存在$E$中的点,则$P$应该是$E$的一个聚点,而此时$P\in E^c$,与$E$是闭集矛盾。由$P$的任意性可得$E^c$是开集。\par
	(6)\textbf{任意个开集的并是开集:}对于在任意个开集的并集中的点$P$,$P$必属于其中一个开集,因而存在一个邻域$U(P)$使得其中的点都在那个开集中,进而$U(P)$都在任意个开集的并集中,即$P$是任意个开集的并集的内点。由$P$的任意性,任意个开集的并集是开集。\par
	\textbf{有限个开集的交是开集:}设这有限个开集为$A_i,i=1,2,\dots,n$。对任意的$ P\in\underset{i=1}{\overset{n}{\cap}}A_i$,则$P$是所有$A_i$的内点,因此存在$U_i(P)\subseteq A_i$。由\cref{prop:Neighbourhood}(2)可知存在$U(P)$满足$U(P)\subseteq\underset{i=1}{\overset{n}{\cap}}U_i(P)\subseteq\underset{i=1}{\overset{n}{\cap}}A_i$,因此$P$是$\underset{i=1}{\overset{n}{\cap}}A_i$的内点。由$P$的任意性,$\underset{i=1}{\overset{n}{\cap}}A_i$是开集。\par
	\textbf{任意个闭集的交是闭集,有限个闭集的并是闭集:}可由(5)以及\cref{prop:SetOperation}(7)在开集的基础上直接得出。
\end{proof}
\begin{definition}
	设$(X,\rho_X)$是一个度量空间。若$E\subseteq X$,且在$E$上定义距离$\rho_E$使得$E$中任意点对$x,y$之间的距离$\rho_E(x,y)=\rho_X(x,y)$,则称$(E,\rho_E)$是$(X,\rho_X)$的一个\gls{subspace}。若$E$是一个开集,则称$E$是$X$的一个\gls{OpenSubspace};$E$是一个闭集,则称$E$是$X$的一个\gls{ClosedSubspace}。
\end{definition}
\subsubsection{内外界聚孤立五点的性质}
\begin{theorem}
	设$(X,\rho)$是一个度量空间,$E\subset X$。$E$的界点不是聚点就是孤立点。
\end{theorem}
\begin{proof}
	若$P$是$E$的一个界点,则它的任意邻域中都既有$E$中的点,又有$E^c$中的点。若它的任意邻域中都只含有有限个$E$中的点,则一定存在一个邻域,使得只有自身属于$E$,那么它是一个孤立点;若它的任意邻域中都含有无限个$E$中的点,由聚点定义,它是一个聚点。
\end{proof}
\subsubsection{关于内外界聚孤立五点的总结}
\begin{equation*}
	\text{度量空间}(X,\rho)\text{中的点与$E$的关系可分为}
	\begin{cases}
		\text{内点} \\
		\text{界点} \\
		\text{外点}    
	\end{cases}\;\text{或}
	\begin{cases}
		\text{聚点}   \\
		\text{孤立点} \\
		\text{外点}    
	\end{cases}
\end{equation*}
\subsubsection{开核、导集、边界、闭包的性质}
\begin{theorem}
	设$(X,\rho)$是一个度量空间,$E\subset X$。$(\mathring{E})^c=\overline{E^c}$。
\end{theorem}
\begin{proof}
	$\overline{E^c}=E^c\cup (E^c)'$。\par
	对任意的$ P\in\overline{E^c}\cap\mathring{E}$,应有$P\in(E^c)'$,即$P$的任意邻域中都有$E^c$中的点,而这与$P\in\mathring{E}$矛盾,由$P$的任意性有$\overline{E^c}\cap\mathring{E}=\varnothing$,即$\overline{E^c}\subset(\mathring{E})^c$;\par
	对任意的$ P\in(\mathring{E})^c$,应有$P\in\mathring{(E^c)}\cup\partial E$。若$P\in\mathring{(E^c)}$,显然$P\in\mathring{(E^c)}\subset E^c\subset\overline{E^c}$;若$P\in\partial E$,要么$P\in E$,要么$P\notin E$,如果此时$P\in E$,则由界点定义它必然是$E^c$的聚点,即$P\in \overline{E^c}$;如果此时$P\notin E$,则$P\in E^c$,即$P\in \overline{E^c}$。由$P$的任意性有$(\mathring{E})^c\subset\overline{E^c}$。\par
	综上$(\mathring{E})^c=\overline{E^c}$。
\end{proof}
\begin{theorem}
	设$(X,\rho)$是一个度量空间,$E\subset X$。$\partial E = \partial E^c$。
\end{theorem}
\begin{proof}
	由定义直接可得。
\end{proof}
\begin{theorem}
	设$(X,\rho)$是一个度量空间,$E\subset X$。$\overline{\overline E}=\overline{E}$。
\end{theorem}
\begin{proof}
	$\overline{\overline E}=\overline{E}\cup (\overline{E})'$,由\cref{prop:OpenClosedSet}(3),$\overline{E}$是闭集,因此$(\overline{E})'\subset\overline{E}$,即$\overline{\overline E}=\overline{E}$。
\end{proof}
\begin{theorem}\label{theo:subsetilc}
	设$(X,\rho)$是一个度量空间,$A\cup B\subset X$。若$A\subset B$,则$\mathring{A}\subset\mathring{B}$,$A'\subset B'$,$\overline{A}\subset\overline{B}$。
\end{theorem}
\begin{proof}
	由内点、聚点的定义直接可得。
\end{proof}
\begin{theorem}\label{theo:cuplimitset eq2 limitsetcup}
	设$(X,\rho)$是一个度量空间,$A\cup B\subset X$。$(A\cup B)'=A'\cup B'$。
\end{theorem}
\begin{proof}
	$A\subset (A\cup B)$,$B\subset (A\cup B)$,由\cref{theo:subsetilc}可知,$A'\subset(A\cup B)'$,$B'\subset(A\cup B)'$,因此$A'\cup B'\subset(A\cup B)'$。\par
	另一方面,对任意的$ P\in (A\cup B)'$,应有$P\in A'\cup B'$。否则就有$P\notin A'$且$P\notin B'$,因此存在$U_1(P)$和$U_2(P)$使得$U_1(P)$中不含除了$P$以外的$A$中的点、$U_2(P)$中不含除了$P$以外的$B$中的点,由邻域的性质,存在$U_3(P)$使得其内不含除了$P$以外$A\cup B$中的点,而这与$P\in (A\cup B)'$矛盾。由$P$的任意性有$(A\cup B)'\subset A'\cup B'$。\par
	综上$(A\cup B)'=A'\cup B'$。
\end{proof}
\begin{theorem}
	设$(X,\rho)$是一个度量空间,$A\cup B\subset X$。$\overline{A\cup B}=\overline{A}\cup \overline{B}$。
\end{theorem}
\begin{proof}
	由\cref{theo:cuplimitset eq2 limitsetcup}:
	\begin{equation*}
		\overline{A}\cup \overline{B}=A\cup A'\cup B\cup B'
		=(A\cup B)\cup(A'\cup B')=(A\cup B)\cup(A\cup B)'
		=\overline{A\cup B}\qedhere
	\end{equation*}
\end{proof}
\subsection{稠密、稀疏、可分、全有界、准紧性}
\subsubsection{稠密性}
\begin{definition}
	设$(X,\rho)$是一个度量空间,$A,B\subseteq X$。若$A\subseteq\overline{B}$,则称$B$在$A$中\gls{dense}。
\end{definition}
\begin{theorem}\label{theo:Density}
	设$(X,\rho)$是一个度量空间,$A,B\subseteq X$。以下四个命题等价:
	\begin{enumerate}
		\item $B$在$A$中稠密。
		\item 对任意的$ x\in A$,$x$的任意邻域中都存在$B$中的点。
		\item 对任意的$ x\in A$,$B$中存在一个点列$\{x_n\}$收敛于$x$。
		\item 对任意的$\varepsilon>0$,$B$中每个点的$\varepsilon$开球邻域的并包含了$A$。
	\end{enumerate}
\end{theorem}
\begin{proof}
	$(1)\to(2)$因为$\overline{B}=B\cup B'$,所以若$x\in A$,则$x\in B$或$x\in B'$,此时显然(2)成立。\par
	$(2)\to(3)$显然。\par
	$(3)\to(4)$若此时(4)不成立,即存在$x\in A$对于某个$\varepsilon$,满足条件$x$不在任何$B$中点的$\varepsilon$开球邻域内,则存在$U(x)$使得$U(x)$中不含$B$中的点,那么(3)也不可能成立,矛盾。\par
	$(4)\to(1)$对任意的$ x\in A$,如果$x\notin \overline{B}$,那么存在$U(x,\delta)$使得$U(x,\delta)$中不含$B$中的点,记$x$与$B$中点之间的最小距离为$\rho$,此时只要取$\varepsilon<\rho$,$x$就不在那些邻域的并集中了,矛盾。由$x$的任意性,$A\subseteq\overline{B}$。
\end{proof}
\subsubsection{稀疏性}
\begin{definition}
	设$(X,\rho)$是一个度量空间,$A$是$X$的一个子集。若$A$在$X$的任何一个非空开集中均不稠密,则称$A$为$X$中的\gls{NowhereDenseSet}。
\end{definition}
下述定理中的邻域既可以是开邻域也可以是闭邻域:
\begin{theorem}
	度量空间$(X,\rho)$的子集$A$为稀疏集的充分必要条件是:对任意的$x\in X$和$x$的任意邻域$U(x)$,存在一个包含于邻域$U(x)$的邻域$U(y)$使得$A\cap U(y)=\varnothing$。
\end{theorem}
\begin{proof}
	\textbf{(1)开邻域:}\par
	充分性:假设此时$A$在$X$中的某非空开集$B$中稠密,则$B\subseteq\overline{A}$。任取$x\in B$和$U(x)\subseteq B$,则$x\in\overline{A}$。如果此时存在包含于$U(x)$的$U(y)$使得条件成立,则$y\notin\overline{A}$,这与$y\in U(y)\subset U(x)\subset B\subset\overline{A}$矛盾。\par
	必要性:若$A$是稀疏集,由\cref{prop:OpenClosedSet}(1)可得$A$在$U(x)$中不稠密。由\cref{theo:Density}(2)可知$\exists\;y\in U(x),\;\exists\;U(y)\subseteq U(x)$使得$A\cap U(y)=\varnothing$。\par
	\textbf{(2)闭邻域:}\par
	充分性:\par
	必要性:\info{有空证明}
\end{proof}
\subsubsection{可分性}
\begin{definition}
	设$(X,\rho)$是一个度量空间,$A\subseteq X$。若存在可列集$B\subset X$使得$B$在$A$中稠密,则称$A$是\gls{separable}。若$X$中存在一个在$X$中稠密的可列子集,则称$X$是可分的度量空间。
\end{definition}
\subsubsection{全有界性}
\begin{definition}
	设$(X,\rho)$是一个度量空间,$A$和$B$都是$X$的子集,$\varepsilon>0$。如果对任意的$x\in A$,都存在$y\in B$,使得$\rho(x,y)<\varepsilon$,则称$B$是$A$的一个$\varepsilon$-网。即:以$B$中的点为中心,$\varepsilon$为半径的所有开邻域的并包含了$A$。
\end{definition}
\begin{definition}
	设$(X,\rho)$是一个度量空间,$A$是$X$的子集。如果对任意的$\varepsilon>0$,$X$中总存在$A$的$\varepsilon$-网,且该$\varepsilon$-网只有有限个点,则称$A$是\gls{TotallyBoundedSet}。
\end{definition}
\begin{property}\label{prop:TotallyBoundedSet}
	设$(X,\rho)$是一个度量空间。全有界集具有如下性质:\par
	(1)任何有限集都是全有界集。\par
	(2)全有界集的子集也是全有界集。\par
	(3)设$A$是一个全有界集,则对任意的$\varepsilon>0$,总存在$A$的一个有限子集成为$A$的一个$\varepsilon$-网。\par
	(4)全有界集有界且可分。
\end{property}
\begin{proof}
	(1)(2)是显然的。\par
	(3)因为$A$是一个全有界集,所以对任意的$\varepsilon>0$,存在一个$A$的$\frac{\varepsilon}{2}$-网$\{x_1,x_2,\dots,x_n\}$。依次取$a_i\in A$使得$a_i\in U(x_i,\frac{\varepsilon}{2}),\;i=1,2,\dots,n$,则$\{\seq{a}{n}\}$即构成$A$的一个$\varepsilon$-网:
	\begin{equation*}
		\forall\;a\in U\left(x_i,\frac{\varepsilon}{2}\right),\;\rho(a,a_i)\leqslant\rho(a,x_i)+\rho(x_i,a_i)<\frac{\varepsilon}{2}+\frac{\varepsilon}{2}=\varepsilon,\;i=1,2,\dots,n
	\end{equation*}\par
	(4)设$A\subseteq X$是一个全有界集,则$A$应有一个$1$-网$\{x_1,x_2,\dots,x_n\}$,于是:
	\begin{equation*}
		\forall\;a\in A,\;\exists\;x_{k}\in\{x_1,x_2,\dots,x_n\},\;\rho(a,x_k)<1
	\end{equation*}
	所以(下式中$x_k$是与点$a$对应的$1$-网中的点):
	\begin{equation*}
		\forall\;a\in A,\;\rho(a,x_1)\leqslant\rho(a,x_k)+\rho(x_k,x_1)<1+\max_{k}\rho(x_k,x_1)
	\end{equation*}
	记$\max\limits_{k}\rho(x_k,x_1)=K$,则对任意的$a\in A,\;a\in U(x_1,1+K)$,所以$A$是有界的。\par
	因为$A$是一个全有界集,所以对任意的$\varepsilon=\frac{1}{n},\;n\in\mathbb{N}^+$,$X$中都存在$A$的一个只含有限个点的$\varepsilon$-网$B_n$。记:
	\begin{equation*}
		B=\underset{n=1}{\overset{+\infty}{\cup}}B_n
	\end{equation*}
	由\info{可列个可列是可列的}可知$B$是可列的。对任意的$x\in A$,存在$x_n\in B_n\subset B$满足对任意的$n\in\mathbb{N}^+$有$\rho(x,x_n)<\frac{1}{n}$,因此$\{x_n\}\to x$。由$x$的任意性和\cref{theo:Density}(3),$B$在$A$中稠密。综上,$A$是可分的。
\end{proof}
\subsubsection{准紧集}
\begin{definition}
	$(X,\rho)$是一个度量空间,$A$是$X$的一个子集。如果$A$的每个点列都有一个子列收敛于$X$中的某一点,则称$A$是\gls{PrecompactSet}。
\end{definition}
\begin{property}\label{prop:PrecompactSet}
	设$(X,\rho)$是一个度量空间。准紧集具有如下性质:
	\begin{enumerate}
		\item 准紧集的子集也是准紧集;
		\item 如果$A\subseteq X$准紧,则$A$全有界。
	\end{enumerate}
\end{property}
\begin{proof}
	(1)由定义立即可得。\par
	(2)如果$A$不是全有界集,则存在$\varepsilon>0$,使得$A$没有只有有限点的$\varepsilon$-网。任取$x_1\in A$,则存在$x_2\in A$使得$\rho(x_1,x_2)\geqslant\varepsilon$,否则$\{x_1\}$就是$A$的一个$\varepsilon$-网。同理,存在$x_3\in A$使得$\rho(x_3,x_j)\geqslant\varepsilon,\;j=1,2$,否则$\{x_1,x_2\}$就是$A$的一个$\varepsilon$-网。重复这一步骤就得到点列$\{x_n\}$,当$m\ne n$时,$\rho(x_m,x_n)\geqslant\varepsilon$。由\cref{prop:CauchySeq}(1)$\{x_n\}$没有收敛的子列,这与$A$的准紧性矛盾。\par
\end{proof}

\subsection{完备的度量空间}
\subsubsection{Cauchy点列}
\begin{definition}
	$(X,\rho)$是一个度量空间,$\{x_n\}$是$X$中的点列。若对任意的$\varepsilon>0$,$\exists\;N\in\mathbb{N}^+$,当$n,m>N$时,有:
	\begin{equation*}
		\rho(x_n,x_m)<\varepsilon
	\end{equation*}
	则称点列$\{x_n\}$是一个\gls{CauchySeq}或\gls{FoundamentalSeq}。
\end{definition}
\begin{property}\label{prop:CauchySeq}
	设$(X,\rho_X)$是一个度量空间。Cauchy点列具有如下性质:
	\begin{enumerate}
		\item 若$\{x_n\}$是$X$中的收敛点列,则$\{x_n\}$是一个Cauchy点列;
		\item 若$\{x_n\}$是$X$中的Cauchy点列,它的一个子列$\{x_{n_k}\}$收敛,则其本身也收敛,并且极限相同;
		\item Cauchy点列是有界的。
	\end{enumerate}
\end{property}
\begin{proof}
	(1)令$n<m$。因为$\{x_n\}$是$X$中的收敛点列,假设其极限为$x$,则对任意的$\varepsilon>0$,$\exists\; N_1\in\mathbb{N}^+$,当$n>N_1$时有$\rho(x_n,x)<\frac{\varepsilon}{2}$;$\exists\; N_2\in\mathbb{N}^+$,当$m>N_2$时有$\rho(x_m,x)<\frac{\varepsilon}{2}$。取$N=\max\{N_1,N_2\}$,则当$n>N$时,有
	\begin{equation*}
		\rho(x_n,x_m)\leqslant\rho(x_n,x)+\rho(x_m,x)<\frac{\varepsilon}{2}+\frac{\varepsilon}{2}=\varepsilon
	\end{equation*}
	即点列$\{x_n\}$是一个Cauchy点列。\par
	(2)设$\{x_{n_k}\}$极限为$x$,则
	\begin{equation*}
		\rho(x_n,x)\leqslant\rho(x_n,x_{n_k})+\rho(x_{n_k},x)
	\end{equation*}
	对任意的$\varepsilon>0$,因为$x_{n_k}$收敛于$x$,所以$\exists\;N_1\in\mathbb{N}^+$,使得当$k>N_1$时,有$\rho(x_{n_k},x)<\frac{\varepsilon}{2}$。又因$\{x_n\}$是$X$中的Cauchy点列,因此$\exists\;N_2\in\mathbb{N}^+$,使得当$n,k>N_2$时,有$\rho(x_n,x_{n_k})<\frac{\varepsilon}{2}$。取$N=\max\{N_1,N_2\}$,当$n,k>N$时,即有
	\begin{equation*}
		\rho(x_n,x)\leqslant\rho(x_n,x_{n_k})+\rho(x_{n_k},x)<\frac{\varepsilon}{2}+\frac{\varepsilon}{2}=\varepsilon
	\end{equation*}\par
	(3)设$(X,\rho)$是一个度量空间,$\{x_n\}$是$X$中的Cauchy点列,则对任意的$\varepsilon>0$,$\exists\;N\in\mathbb{N}^+$,当$n,m>N$时,有:
	\begin{equation*}
		\rho(x_n,x_m)<\varepsilon
	\end{equation*}
	取$m=N+1$,令$\alpha=\max\limits_{i=1,2,\dots,N}\rho(x_m,x_i)$,$\delta=\max\{\varepsilon,\alpha\}$,则$\{x_n\}$中的所有点都在闭邻域$\overline{U}(x_m,\delta)$中,所以$\{x_n\}$有界。
\end{proof}
\subsubsection{完备度量空间}
\begin{definition}
	$(X,\rho)$是一个度量空间。若$X$中的任意Cauchy点列$\{x_n\}$都收敛到$X$中的某一点,则称$X$是一个\gls{complete}度量空间。
\end{definition}
\begin{theorem}[闭球套定理]
	$(X,\rho)$是一个度量空间。$X$完备的充分必要条件为对任何满足下列条件的一列闭邻域$\{E_n=\overline{U}(x_n,\delta_n)\}$:
	\begin{enumerate}
		\item $E_1\supseteq E_2\supseteq\cdots\supseteq E_n\supseteq\cdots$
		\item $\{\delta_n\}\to0$
	\end{enumerate}
	$X$中都存在唯一的$x$满足$x\in\underset{n\in\mathbb{N}^+}{\cap}E_n$。
\end{theorem}
\begin{proof}
	\textbf{充分性:}任取$X$中的一个Cauchy点列$\{x_n\}$。因为$\{x_n\}$是一个Cauchy点列,所以对任意的$\varepsilon>0,\;\exists\;N\in\mathbb{N}^+,\;\forall\;n,m>N,\;\rho(x_m,x_n)<\varepsilon$。取$\{N_k\}$使得$N_k$是使得$\rho(x_m,x_n)<\dfrac{1}{2^k}$的临界条件,取$x_{n_k}>N_k,\;x_{n_k+1}>N_{k+1}$,即可产生一个子列 $\{x_{n_k}\}$,满足:
	\begin{equation*}
		\rho(x_{n_k},x_{n_{k+1}})<\frac{1}{2^k}
	\end{equation*}
	取闭邻域列:
	\begin{equation*}
		\left\{E_k=\bar{U}\left(x_{n_k},\frac{1}{2^{k-1}}\right)\right\}
	\end{equation*}
	于是:
	\begin{equation*}
		\forall\;y\in E_{k+1},\;\rho(y,x_{n_k})\leqslant\rho(y,x_{n_{k+1}})+\rho(x_{n_{k+1}},x_{n_k})<\frac{1}{2^k}+\frac{1}{2^k}=\frac{1}{2^{k-1}}
	\end{equation*}
	所以$E_k\supseteq E_{k+1}$。由条件,此时存在唯一的$x\in X$满足$x\in E_k,\;\forall\;k\in\mathbb{N}^+$。下证$\{x_n\}\to x$:
	\begin{equation*}
		\rho(x_n,x)\leqslant\rho(x_n,x_{n_k})+\rho(x_{n_k},x)\leqslant\rho(x_n,x_{n_k})+\frac{1}{2^{k-1}}
	\end{equation*}
	因为$\{x_n\}$是一个Cauchy点列,因此当$n$和$k$足够大时,上式右端两项均趋于$0$。因此$\{x_n\}\to x$。由$\{x_n\}$的任意性,$X$是一个完备的度量空间。\par
	\textbf{必要性中的存在性:}在每个$E_n$中取一点$y_n$构成点列$\{y_n\}$。设$m>n$,因为$E_m\subseteq E_n$,所以$y_m\in E_n$,于是有:
	\begin{equation*}
		\rho(y_n,y_m)\leqslant2\delta_n\to0
	\end{equation*}
	因此$\{y_n\}$是一个Cauchy点列。因为$X$完备,所以$\{y_n\}\to x\in X$。对任意的$n_0\in\mathbb{N}^+$,当$n>n_0$时,$y_n\in E_n\subseteq E_{n_0}$。由\cref{prop:OpenClosedSet}(2)可得闭邻域$E_{n_0}$是闭集,因此$x\in E_{n_0}$。由$n_0$的任意性,$x\in E_n,\;\forall\;n\in\mathbb{N}^+$。\par
	\textbf{必要性中的唯一性:}若还有一点$y$满足上述条件,则:
	\begin{equation*}
		\rho(x,y)\leqslant\rho(x,y_n)+\rho(y_n,y)\to0
	\end{equation*}
	唯一性得证。
\end{proof}
\begin{definition}
	设$A$是度量空间$(X,\rho)$的子集。若$A$可表示为至多可列个稀疏集的并,则称$A$是\gls{FirstSet}。反之则为\gls{SecondSet}。
\end{definition}
\begin{property}
	设$(X,\rho)$是一个完备的度量空间,则:
	\begin{enumerate}
		\item $(X,\rho)$的子空间$E$是完备度量空间的充要条件为$E$是$X$中的一个闭子空间;
		\item $X$是第二型集;
		\item $A\subseteq X$是全有界集的充要条件为$A$是准紧集;
		\item $A\subseteq X$为准紧集的充分必要条件是对任意的$\varepsilon>0$,$A$都有准紧的$\varepsilon$-网。
	\end{enumerate}
\end{property}
\begin{proof}
	(1)\textbf{必要性:}因为$E$是完备子空间,则对任意的$x\in E'$,存在$E$中的一个收敛点列$\{x_n\}\to x$。由\cref{prop:CauchySeq}(1)和$E$的完备性可知$x\in E$。由$x$的任意性,$E'\subseteq E$,故$E$是一个闭集,即$E$是$X$的一个闭子空间。\par
	\textbf{充分性:}任取$\{x_n\}$为$E$中的一个Cauchy点列,那么它也是$X$中的Cauchy点列,因此存在$x\in X$使得$\{x_n\}\to x$,即$x$是$E$的一个聚点。又因$E$是闭的,所以$x\in E$,即Cauchy点列$\{x_n\}$收敛于$E$中的一点。由$\{x_n\}$的任意性,$E$是完备的度量空间。\par
	(2)假设不成立,即存在完备的度量空间$X$使得$X$是第一型集。也就是说$X=\underset{i=1}{\overset{+\infty}{\cup}}F_i$,其中$F_i,\;\forall\;i\in\mathbb{N}^+$是稀疏集。因为$F_1$是稀疏集,由\cref{theo:Density}(2),存在$x_1\in X$,使得闭邻域$U(x_1,r_1)$中不含$F_1$中的点。对于闭邻域$U(x_1,r_1)$,由于$F_2$是稀疏集,因此存在$x_2\in U(x_1,r_1)$,使得闭邻域$U(x_2,r_2)$中不含$F_2$中的点。如此重复下去,实际上可以取$r_n=\dfrac{1}{n}$,便可以得到一个闭球套:
	\begin{equation*}
		U(x_1,r_1)\supseteq U(x_2,r_2)\supseteq\cdots\supseteq U(x_n,r_n)\supseteq\cdots
	\end{equation*}
	且$\{r_n\}\to0$。由完备度量空间的闭球套定理,存在一个$X$中的点$x\in U(x_n,r_n),\;\forall\;n\in\mathbb{N}^+$。而由闭球套的取法,$x\notin X$,矛盾。\par
	(3)根据\cref{prop:PrecompactSet}(2)可知只需证明必要性。任取$A$中的点列$\{x_n\}$。如果$\{x_n\}$中只有有限个互不相同的元素,则$\{x_n\}$显然有收敛的子列。如果$\{x_n\}$中有无限个互不相同的元素,记这些元素构成的集合为$B_0$。由\cref{prop:TotallyBoundedSet}(2),$B_0$也是全有界集。由\cref{prop:TotallyBoundedSet}(3),$B_0$中存在有限个元素使得以它们为球心、$\dfrac{1}{2}$为半径的开邻域的并包含$B_0$,显然$B_0$中至少存在一个点$y_1$使得以它为半径、$\dfrac{1}{2}$为半径的开邻域包含了无穷多个$A$中的点。记被$y_1$包含的这无穷多个点构成的集合为$B_1$,显然$B_1$的直径小于等于$1$。因为$B_1$是$B_0$的子集,所以$B_1$也是全有界的,重复上述论证,则存在$B_2\subseteq B_1$,使得$B_2$中含有$B_1$无穷多个元素且$B_2$的直径小于等于$\frac{1}{2}$。依次类推,可以得到一系列集合满足如下条件:
	\begin{enumerate}
		\item $B_1\supseteq B_2\supseteq\cdots\supseteq B_n\supseteq\cdots$。
		\item $B_n$的直径小于等于$\dfrac{1}{2^{n-1}}$。
		\item 每个$B_n$中都含有$\{x_n\}$中无限个元素。
	\end{enumerate}
	取$x_{n_k}\in B_k$,便得到$\{x_n\}$的一个子列$\{x_{n_k}\}$。显然$\{x_{n_k}\}$是一个Cauchy点列:
	\begin{equation*}
		\forall\;p>q,\;x_{n_p}\in B_p\subseteq B_q,\;\rho(x_{n_p},x_{n_q})<\frac{1}{2^{q-1}}\to0
	\end{equation*}	
	因为$X$完备,所以$\{x_{n_k}\}$在$X$中收敛。由$\{x_n\}$的任意性,$A$准紧。\par
	(4)\textbf{必要性:}$A$就是它自身的准紧的$\varepsilon$-网。\par
	\textbf{充分性:}若对任意的$\varepsilon>0$,$A$都有准紧的$\varepsilon$-网$B$。因为$B$准紧,同时$X$完备,由\cref{prop:PrecompactSet}(2)可知$B$全有界,即$B$有只有有限个元素的$\varepsilon$-网$C=\{c_1,c_2,\dots,c_n\}$。则:
	\begin{equation*}
		\forall\;a\in A,\;\rho(a,c_i)\leqslant\rho(a,b)+\rho(b,c_i),\;\forall\;i=1,2,\dots,n
	\end{equation*}
	可以选取$c_i$和$b\in B$使得$\rho(a,b)<\varepsilon,\;\rho(b,c_i)<\varepsilon$(先选择$b$,再根据$b$即可选得$c_i$),即对于任意的$a\in A$,存在$c_i\in\{c_1,c_2,\dots,c_n\}$使得$\rho(a,c_i)<2\varepsilon$,于是$\{c_1,c_2,\dots,c_n\}$中的点为中心、$2\varepsilon$为半径构成的$2\varepsilon$-网必然是$A$的一个$2\varepsilon$-网。由$\varepsilon$的任意性,$A$全有界。又因为$X$是完备的,由(3)可知$A$准紧。
\end{proof}
\section{度量空间上的映射}
\subsection{映射的定义与性质}
\begin{definition}
	设$X,Y$是任意给定的集合。如果对于任意的$x\in X$,都存在唯一的$f(x)\in Y$与之对应,则称对应关系$f$是一个从$X$到$Y$的\gls{mapping}。对任何$E\subseteq Y$,称:
	\begin{equation*}
		f^{-1}(E)=\{x:f(x)\in E\}
	\end{equation*}
	为集合$B$在映射$f$下的\gls{preimage}\footnote{原像与逆无关。}。
\end{definition}
\begin{theorem}\label{theo:PropertyOfPreimage}
	设$X,Y$是任意给定的集合,$f$是一个从$X$到$Y$的映射。集合的原像有下列性质:
	\begin{enumerate}
		\item $f^{-1}(\varnothing)=\varnothing,\;f^{-1}(Y)=X$;
		\item 若$E_1\subseteq E_2\subseteq Y$,则$f^{-1}(E_1)\subseteq f^{-1}(E_2)$;
		\item 对任意的$E\subset Y$,$[f^{-1}(E)]^c=f^{-1}(E^c)$;
		\item 设$T$是一个指标集,对$\{A_t\in\ Y:t\in T\}$,有:
		\begin{equation*}
			f^{-1}\left(\underset{t\in T}{\bigcup}A_t\right)=\underset{t\in T}{\bigcup}f^{-1}(A_t), \quad
			f^{-1}\left(\underset{t\in T}{\bigcap}A_t\right)=\underset{t\in T}{\bigcap}f^{-1}(A_t)
		\end{equation*}
	\end{enumerate}
\end{theorem}
\begin{proof}
	(1)(2)是显然的,下证(3)(4)。\par
	(3)对任意的$x\in[f^{-1}(E)]^c$,有$x\notin f^{-1}(E)$,即$f(x)\notin E$,所以$x\in f^{-1}(E^c)$。由$x$的任意性,$[f^{-1}(E)]^c\subseteq f^{-1}(E^c)$。对任意的$x\in f^{-1}(E^c)$,有$f(x)\in E^c$,所以$x\notin f^{-1}(E)$,于是$x\in[f^{-1}(E)]^c$。由$x$的任意性,$f^{-1}(E^c)\subseteq[f^{-1}(E)]^c$。综上,$[f^{-1}(E)]^c=f^{-1}(E^c)$。\par
	(4)对任意的$x\in f^{-1}\left(\underset{t\in T}{\cup}A_t\right)$,有$f(x)\in\underset{t\in T}{\cup}A_t$,即存在$t\in T$,使得$f(x)\in A_t,\;x\in f^{-1}(A_t)$,于是$x\in\underset{t\in T}{\cup}f^{-1}(A_t)$。由$x$的任意性,$f^{-1}\left(\underset{t\in T}{\cup}A_t\right)\subseteq\underset{t\in T}{\cup}f^{-1}(A_t)$。对任意的$x\in\underset{t\in T}{\cup}f^{-1}(A_t)$,则存在$t\in T$,使得$x\in f^{-1}(A_t)$,于是$x\in f^{-1}\left(\underset{t\in T}{\cup}A_t\right)$。由$x$的任意性,$\underset{t\in T}{\cup}f^{-1}(A_t)\subseteq f^{-1}\left(\underset{t\in T}{\cup}A_t\right)$。综上,$f^{-1}\left(\underset{t\in T}{\cup}A_t\right)=\underset{t\in T}{\cup}f^{-1}(A_t)$。交的情形同理可证。
\end{proof}
\begin{definition}
	设$T$是一个$X$到$Y$的映射。如果对任意的$y\in Tx$,只有唯一的$x$使得$Tx=y$,那么称映射$T$为\gls{InjectiveF}。
\end{definition}
\begin{definition}
	设$T$是一个$X$到$Y$的映射。如果对任意的$y\in Y$,都存在$X$中的$x$使得$Tx=y$,那么称映射$T$为\gls{SurjectiveF}。
\end{definition}
\begin{definition}
	设$T$是一个$X$到$Y$的映射。如果$T$既是单射,又是满射,则称之为\gls{BijectiveF}。
\end{definition}
\begin{definition}
	设$T$是一个$X$到$Y$的双射,即对任意的$y\in Y$,都存在唯一的$x\in X$使得$Tx=y$,此时可以得到一个新的映射,它将$Y$映成$X$,称这个映射为$T$的\gls{InverseMap}。
\end{definition}
\begin{theorem}
	设$T$是一个$X$到$Y$的映射。$T$存在逆映射的充要条件是$T$是一个双射。
\end{theorem}
\begin{definition}
	设$T_1$是一个$X$到$Y$的映射,$T_2$是一个$Y$到$Z$的映射。定义映射$T_3$满足$T_3x=T_2(T_1x)$,其中$x\in X$,则称映射$T_3$是映射$T_1$和映射$T_2$的\gls{CompositeMap},记作$T_2\circ T_1$或$T_2T_1$。
\end{definition}

\subsection{映射的极限}
\begin{definition}
	设$(X,\rho_X)$和$(Y,\rho_Y)$都是度量空间,$T$是一个$E\subseteq X$到$Y$的映射,$x_0\in X$是$E$的一个聚点,$y\in Y$。若对任意的$\varepsilon>0$,存在$\delta>0$,使得对$E\setminus\{x_0\}$中一切满足条件$\rho(x,x_0)<\delta$的$x$,都有$\rho_Y(Tx,y)<\varepsilon$,则称$T$沿$E$趋于$x_0$时,$T$的极限为$y$,记为:
	\begin{equation*}
		\lim_{\substack{x\to x_0 \\ E}}Tx=y
	\end{equation*}
	当$E$自明时,可简记为:
	\begin{equation*}
		\lim_{x\to x_0}Tx=y
	\end{equation*}
\end{definition}
\begin{theorem}[Sequential Characterization of Function Limits]\label{theo:SequentialCharacterizationOfFunctionLimits}
	设$(X,\rho_X)$和$(Y,\rho_Y)$都是度量空间,$T$是一个$E\subseteq X$到$Y$的映射,$x_0\in X$是$E$的一个聚点,$y\in Y$。$T$沿$E$趋于$x_0$时$T$的极限为$y$的充分必要条件为:对于任意满足$\{x_n\}\to x_0$的点列$\{x_n\}\subseteq E\setminus\{x_0\}$都有$\{Tx_n\}\to y$。
\end{theorem}
\begin{proof}
	\textbf{必要性:}若存在满足$\{x_n\}\to x_0$的点列$\{x_n\}\subseteq E\setminus\{x_0\}$不满足$\{Tx_n\}\to y$,则存在$\varepsilon>0$对任意的$N_1\in\mathbb{N}^+$都存在$n>N_1$使得$\rho_Y(Tx_n,y)\geqslant\varepsilon$。取定一个满足上述条件的$\varepsilon$,因为$T$沿$E$趋于$x_0$时极限为$y$,所以存在$\delta>0$,当$\rho_X(x,x_0)<\delta$时($x\in E\setminus\{x_0\}$)有$\rho_Y(Tx,y)<\varepsilon$。因为$\{x_n\}\to x_0$,所以存在$N_2\in\mathbb{N}^+$满足当$n>N_2$时有$\rho_X(x_n,x_0)<\delta$,此时就有$\rho_Y(Tx_n,y)<\varepsilon$。取$N_1=N_2$,矛盾,所以必要性成立。\par
	\textbf{充分性:}若此时$T$沿$E$趋于$x_0$时极限不为$y$,那么存在$\varepsilon>0$对任意的$\delta>0$都存在满足条件$\rho_X(x,x_0)<\delta$的点$x\in E\setminus\{x_0\}$使得$\rho_Y(Tx,Tx_0)\geqslant\varepsilon$,因此可取一个点列$\{x_n\}$,满足$\rho_X(x_n,x_0)<\dfrac{1}{n}$。注意到此时满足$\{x_n\}\to x$但不满足$\{Tx_n\}\to y$,矛盾。
\end{proof}
\begin{theorem}[Cauchy-Type Condition for Function Limits]\label{theo:Cauchy-TypeConditionForFunctionLimits}
	设$(X,\rho_X)$是度量空间,$(Y,\rho_Y)$是完备的度量空间,$T$是一个$E\subseteq X$到$Y$的映射,$x_0\in X$是$E$的一个聚点。$T$沿$E$趋于$x_0$时存在极限的充要条件为:对任意的$\varepsilon>0$,存在$\delta>0$使得对于任意满足条件$\rho_X(x,x_0)<\delta,\rho_X(x',x_0)<\delta$的$x,x'\in E\setminus\{x_0\}$有$\rho_Y(Tx,Tx')<\varepsilon$。
\end{theorem}
\begin{proof}
	\textbf{必要性:}设$T$沿$E$趋于$x_0$时极限为$y\in Y$,则对于任意的$\varepsilon>0$,存在$\delta_1$使得当$\rho_X(x,x_0)<\delta_1$时有$\rho_Y(Tx,y)<\dfrac{\varepsilon}{2}$,存在$\delta_1$使得当$\rho_X(x',x_0)<\delta_2$时有$\rho_Y(Tx',y)<\dfrac{\varepsilon}{2}$。取$\delta=\min\{\delta_1,\delta_2\}$则当$\rho_X(x,x_0)<\delta$且$\rho_X(x',x_0)<\delta$时有$\rho_Y(Tx,y)<\dfrac{\varepsilon}{2}$和$\rho_Y(Tx',y)<\dfrac{\varepsilon}{2}$,于是:
	\begin{equation*}
		\rho_Y(Tx,Tx')\leqslant\rho_Y(Tx,y)+\rho_Y(Tx',y)<\varepsilon
	\end{equation*}\par
	\textbf{充分性:}因为$x_0$是$E$的聚点,所以存在$\{x_n\}\subseteq E\setminus\{x_0\}$满足$\{x_n\}\to x_0$。对于任意的$\varepsilon>0$,取任意一个满足上述条件的$\{x_n\}$,则对于条件中给定的$\delta>0$,存在$N\in\mathbb{N}^+$使得当$n>N$时有$\rho_X(x_n,x_0)<\delta$,于是当$m,n>N$时就有$\rho_Y(Tx_n,Tx_m)<\varepsilon$。因为$(Y,\rho_Y)$是完备的度量空间,所以$\{Tx_n\}$有极限。下面证明所有$\{Tx_n\}$的极限都相同。假设存在不同的$\{x_n'\}$和$\{x_n''\}$满足:
	\begin{equation*}
		\{x_n'\},\{x_n''\}\subseteq E\setminus\{x_0\},\quad\{x_n'\},\{x_n''\}\to x_0,\quad\lim_{n\to+\infty}Tx_n'\ne\lim_{n\to+\infty}Tx_n''
	\end{equation*}
	定义$\{x_n\}$:
	\begin{equation*}
		x_n=
		\begin{cases}
			x_n',&n\text{为奇数} \\
			x_n'',&n\text{为偶数}
		\end{cases}
	\end{equation*}
	则$\{x_n\}\subseteq E\setminus\{x_0\}$且$\{x_n\}\to x_0$,但由\cref{prop:ConvergentSeqOfPoints}可知$\{Tx_n\}$不收敛,矛盾,所以所有$\{Tx_n\}$的极限都相同,充分性得证。
\end{proof}
\begin{property}\label{prop:MappingLimits}
	设$(X,\rho_X)$和$(Y,\rho_Y)$都是度量空间,$T$是一个$E\subseteq X$到$Y$的映射,则:
	\begin{enumerate}
		\item 若$T$沿$E$趋于某一点时有极限,则极限唯一;
		\item 若$x_0$是$E$的聚点,$T$沿$E$趋于$x_0$时极限为$y\in Y$,则存在$\delta>0$使得$T[U(x_0,\delta)\setminus\{x_0\}]$有界;
		\item 设$(Z,\rho_Z)$是度量空间,$S$是$F\subseteq Y$到$Z$的映射,$x_0\in X$是$E$的聚点,$y\in Y$是$T(E\setminus\{x_0\})$的聚点,$T(E\setminus\{x_0\})\subseteq F\setminus\{y\}$,$z\in Z$。若$T$沿$E$趋于$x_0$时极限为$y$,$S$沿$F$趋于$y$时极限为$z$,则:
		\begin{equation*}
			\lim_{\substack{x\to x_0 \\ E}}S(Tx)=z
		\end{equation*}
	\end{enumerate}
\end{property}
\begin{proof}
	(1)由\cref{theo:SequentialCharacterizationOfFunctionLimits}和\cref{prop:ConvergentSeqOfPoints}(1)立即可得。\par
	(2)由定义立即可得。\par
	(3)由映射极限的定义,$E\setminus x_0$中任意满足$\{x_n\}\to x_0$的点列$\{x_n\}$都有$\{Tx_n\}\subseteq F\setminus\{y\}$且$\{Tx_n\}\to y$,根据\cref{theo:SequentialCharacterizationOfFunctionLimits}可知$\{S(Tx_n)\}\to z$,于是结论成立。
\end{proof}
\subsection{映射的连续性}
\begin{definition}
	设$(X,\rho_X)$和$(Y,\rho_Y)$都是度量空间,$T$是一个$E\subseteq X$到$Y$的映射,$x_0\in E$是$E$的一个聚点。若对任意的$\varepsilon>0$,存在$\delta>0$使得对$E$中一切满足条件$\rho(x,x_0)<\delta$的$x$都有$\rho_Y(Tx,Tx_0)<\varepsilon$,或对于任意满足$\{x_n\}\to x_0$的点列$\{x_n\}\subseteq E$都有$\{Tx_n\}\to Tx_0$,则称$T$沿$E$在$x_0$处\gls{continuous}。
\end{definition}
两个条件的等价性类似\cref{theo:SequentialCharacterizationOfFunctionLimits}即可得到。
\begin{definition}
	设$T$是一个$(X,\rho_X)$到$(Y,\rho_Y)$的映射,若$T$在$E\subseteq X$的每一点都连续,则称$T$是$E$上的\gls{ContinuousMap}。
\end{definition}
\begin{theorem}\label{theo:ContinousMapO2OC2C}
	度量空间$(X,\rho_X)$到$(Y,\rho_Y)$上的映射$T$是$X$上的连续映射的充要条件为:
	\begin{enumerate}
		\item $Y$中任意开集$E$的原像$T^{-1}E$是$X$中的开集。
		\item $Y$中任意闭集$E$的原像$T^{-1}E$是$X$中的闭集。
	\end{enumerate}
\end{theorem}
\begin{proof}
	(1)\textbf{必要性:}设$T$是连续映射,$E\subseteq Y$是一个开集,如果$T^{-1}E=\varnothing$,则$T^{-1}E$是开集;若$T^{-1}E\ne\varnothing$,任取$x\in T^{-1}E$,令$Tx=y$,则$y\in E$,因为$E$是开集,所以存在$y$的$\varepsilon$邻域$U$,使得$U\subseteq E$,由$T$的连续性,存在$x$的$\delta$邻域$V$,使得$TV\subseteq U$,因此$V\subseteq T^{-1}U\subseteq T^{-1}E$,即$x$是$T^{-1}E$的内点。由$x$的任意性,$T^{-1}E$是$X$中的开集。\par
	\textbf{充分性:}对任意的$x\in X$及$Tx$的任意$\varepsilon$邻域$U$,由\cref{prop:OpenClosedSet}(1)可知$U$是一个开集,因此$T^{-1}U$是$X$中的开集,所以$x$是$T^{-1}U$的内点,于是存在$x$的某个$\delta$邻域$V$,使得$V\subseteq T^{-1}U$,因此$TV\subseteq U$,即$T$在$x$处连续。由$x$的任意性,$T$是$X$上的连续映射。\par
	(2)由(1)、\cref{theo:PropertyOfPreimage}(3)和\cref{prop:OpenClosedSet}(5)立即可得。
\end{proof}
\begin{theorem}\label{theo:CompositeContinuousMap}
	设$(X,\rho_X),(Y,\rho_Y),(Z,\rho_Z)$都是度量空间,$T$是一个$E\subseteq X$到$Y$的映射,$S$是$F\subseteq Y$到$Z$的映射。若$T$在$E$上连续、$S$在$F$上连续且$TE\subseteq F$,则$ST$在$E$上连续。
\end{theorem}
\begin{proof}
	任取$x\in E$和$\{x_n\}\subseteq E$满足$\{x_n\}\to x$,因为$T$在$E$上连续,所以$\{Tx_n\}\to Tx$。由$TE\subseteq F$和$S$在$F$上连续可得$\{STx_n\}\to STx$,即$ST$在$x$处连续。根据$x$的任意性可得$ST$在$E$上连续。
\end{proof}

\subsection{压缩映射原理}
\begin{definition}
	若点$\varphi$在映射$T$的作用下满足$T\varphi=\varphi$,则称$\varphi$是映射$T$的一个\gls{FixedP}。
\end{definition}
\begin{definition}
	设$(X,\rho)$是一个度量空间,$T$是$X$到$X$的一个映射,如果存在一个数$\alpha$,$0\leqslant\alpha<1$,使得对任意的$x,y\in X$,有:
	\begin{equation*}
		\rho(Tx,Ty)\leqslant\alpha\rho(x,y)
	\end{equation*}
	则称$T$是一个\gls{ContractionMap}。
\end{definition}
\begin{theorem}[Contraction Mapping Theorem]\label{theo:ContractionMapTheorem}
	设$(X,\rho)$是一个完备的度量空间,$T$是$X$到$X$的一个压缩映射,那么$T$有且只有一个不动点,该不动点为任取$x_0\in X$序列$\{x_n=T^nx_0\}$的极限$x\in X$,并有下述两种误差估计:
	\begin{equation*}
		\rho(x_n,x)\leqslant\frac{\alpha^n}{1-\alpha}\rho(x_0,x_1),\quad\rho(x_n,x)\leqslant\frac{1}{1-\alpha}\rho(x_{n+1},x_{n})
	\end{equation*}
\end{theorem}
\begin{proof}
	\textbf{存在性:}任取$x_0\in X$,令$x_n=T^nx_0$,由此产生一个点列$\{x_n\}$。下面我们来证明这个点列是一个Cauchy点列,它的极限就是一个不动点。\par
	\begin{align*}
		\rho(x_{m+1},x_m)&=\rho(Tx_m,Tx_{m-1})\leqslant\alpha\rho(x_m,x_{m+1}) \\
		&=\cdots \\
		&=\alpha^{m-1}\rho(Tx_1,Tx_0)\leqslant\alpha^m\rho(x_1,x_0)
	\end{align*}
	取$n>m$,由距离的三角不等式:
	\begin{align*}
		\rho(x_m,x_n)
		&\leqslant\rho(x_m,x_{m+1})+\cdots+\rho(x_{n-1},x_n) \\
		&\leqslant(\alpha^m+\alpha^{m+1}+\cdots+\alpha^{n-1})\rho(x_0,x_1) \\
		&=\alpha^m\frac{1-\alpha^{n-m}}{1-\alpha}\rho(x_0,x_1) \\
		&\leqslant\frac{\alpha^m}{1-\alpha}\rho(x_0,x_1)
	\end{align*}
	因为$0\leqslant\alpha<1$,所以当$m$足够大的时候,$\rho(x_m,x_n)\rightarrow 0$,即$\{x_n\}$是$X$中的Cauchy点列。又因为$X$完备,所以$\{x_n\}\rightarrow x\in X$。由三角不等式:
	\begin{equation*}
		\rho(x,Tx)\leqslant\rho(x,x_m)+\rho(x_m,Tx)\leqslant\rho(x,x_m)+\alpha\rho(x_{m-1},x)
	\end{equation*}
	当$m\to+\infty$时上式右端趋于0,因此$\rho(x,Tx)=0$,即$Tx=x$,$T$存在一个不动点。由上上式可得:
	\begin{equation*}
		\rho(x_m,x_{m+k})\leqslant\frac{\alpha^m}{1-\alpha}\rho(x_0,x_1)
	\end{equation*}
	令$k\to+\infty$,由\cref{prop:RSeq}(6)可得:
	\begin{equation*}
		\rho(x_m,x)\leqslant\frac{\alpha^m}{1-\alpha}\rho(x_0,x_1)
	\end{equation*}\par
	\textbf{唯一性:}假设$T$还有一个不动点$y$,则
	\begin{equation*}
		\rho(x,y)=\rho(Tx,Ty)\leqslant\alpha\rho(x,y)
	\end{equation*}
	因为$0\leqslant\alpha<1$,所以$\rho(x,y)=0$,即$x=y$,唯一性得证。\par
	对于第二种误差估计,只需注意到:
	\begin{equation*}
		\rho(x_n,x_{n+p})\leqslant(\alpha^{p-1}+\alpha^{p-2}+\cdots+1)\rho(x_n,x_{n+1})\leqslant\frac{1}{1-\alpha}\rho(x_n,x_{n+1})
	\end{equation*}
	令$p\to+\infty$,由\cref{prop:RSeq}(6)可得:
	\begin{equation*}
		\rho(x_n,x)\leqslant\frac{1}{1-\alpha}\rho(x_n,x_{n+1})\qedhere
	\end{equation*}
\end{proof}
压缩映射原理有一个推广:
\begin{theorem}
	设$T$是完备度量空间$X$到自身的映射,如果存在常数$\alpha$及$n\in\mathbb{N}^+$,$0\leqslant\alpha<1$,使得对任意$x,y\in X$,有:
	\begin{equation*}
		\rho(T^{n}x,T^{n}y)\leqslant\alpha\rho(x,y)
	\end{equation*}
	那么$T$在$X$中有且只有一个不动点。
\end{theorem}
\begin{proof}
	\textbf{存在性:}$T^{n}$满足\cref{theo:ContractionMapTheorem}的条件,因此$T^{n}$有且只有一个不动点$x_0$。下证$x_0$也是$T$在$X$中唯一的不动点。因为
	\begin{equation*}
		T^{n}(Tx_0)=T^{n+1}x_0=T(T^{n}x_0)=Tx_0
	\end{equation*}
	所以$Tx_0$是$T^{n}$的一个不动点,由不动点的唯一性,$Tx_0=x_0$,所以$x_0$是$T$的一个不动点。\par
	\textbf{唯一性:}若$T$存在另一个不动点$x_1$,则
	\begin{equation*}
		T^{n}x_1=T^{n-1}Tx_1=T^{n-1}x_1=\cdots=Tx_1=x_1
	\end{equation*}
	即$x_1$也是$T^{n}$的一个不动点,由$T^{n}$不动点的唯一性,$x_0=x_1$。
\end{proof}

\subsection{紧集上的连续映射}
\begin{definition}
	设$(X,\rho_X)$和$(Y,\rho_Y)$为度量空间,$T$是$E\subseteq X$到$Y$上的映射。若对于任意的$\varepsilon>0$,存在只与$\varepsilon$有关的$\delta>0$,使得对任意的$x,y\in E$,只要$\rho_X(x,y)<\delta$,就有$\rho_Y(Tx,Ty)<\varepsilon$,则称$T$在$E$上\gls{UniformlyContinuous}。
\end{definition}
\begin{theorem}[Sequential Characterization of Uniformly Continuous]
	设$(X,\rho_X)$和$(Y,\rho_Y)$为度量空间,$T$是$E\subseteq X$到$Y$上的映射。$T$在$E$上一致连续的充分必要条件为:对于任意满足条件$\lim\limits_{n\to+\infty}\rho_X(x_n,y_n)$的点列$\{x_n\},\{y_n\}\subseteq E$,都有:
	\begin{equation*}
		\lim_{n\to+\infty}\rho_Y(Tx_n,Ty_n)=0
	\end{equation*}
\end{theorem}
\begin{proof}
	\textbf{必要性:}设$T$在$E$上一致连续,则对于任意的$\varepsilon>0$,存在只与$\varepsilon$有关的$\delta>0$,使得对任意的$x,y\in E$,只要$\rho_X(x,y)<\delta$,就有$\rho_Y(Tx,Ty)<\varepsilon$。因为$\lim\limits_{n\to+\infty}\rho_X(x_n,y_n)=0$,所以存在$N\in\mathbb{N}^+$,当$n>N$时有$\rho_X(x_n,y_n)<\delta$,于是$\rho_Y(Tx_n,Ty_n)<\varepsilon$,即$\{\rho_Y(Tx_n,Ty_n)\}\to 0$。\par
	\textbf{充分性:}若此时$T$不在$E$上一致连续,则存在$\varepsilon>0$,无论$\delta=\dfrac{1}{n}$多小,总存在满足$\rho_X(x_n,y_n)<\delta$的$x_n,y_n\in E$使得$\rho_Y(Tx_n,Ty_n)\geqslant\varepsilon$。此时满足$\lim\limits_{n\to+\infty}\rho_X(x_n,y_n)$,但没有$\lim\limits_{n\to+\infty}\rho_Y(Tx_n,Ty_n)=0$,矛盾。
\end{proof}
\begin{property}\label{prop:CompactMap}
	设$(X,\rho_X)$为度量空间,$A$是$X$中的紧集,$T$是$A$到$\mathbb{R}^{}$上的映射,则:
	\begin{enumerate}
		\item $TA$是$Y$中的紧集;
		\item $T$在$A$上有界;
		\item $T$在$A$上可达到其上、下确界;
		\item $T$在$A$上一致连续。
	\end{enumerate}
\end{property}
\begin{proof}
	(1)设$\{y_n\}$为$TA$中的一个点列,则有$X$中的点列$\{x_n\}$使得$y_n=Tx_n,\;n\in\mathbb{N}^+$。因为$A$是紧集,所以$\{x_n\}$存在子列$\{x_{n_k}\}\to x_0\in A$。因为$T$连续,所以:
	\begin{equation*}
		\lim_{k\to+\infty}y_{n_k}=\lim_{k\to+\infty}Tx_{n_k}=T\left(\lim_{k\to+\infty}x_{n_k}\right)=Tx_0\in TA
	\end{equation*}
	所以$TA$是紧集。\par
	(2)由(1)可知$TA$是紧集,根据\cref{prop:CompactSet}(2)可得$T$在$A$上有界。\par
	(3)由(1)和\cref{theo:CompactRn}可知$TA$是有界闭集,所以$T$在$A$上可达到其上、下确界。\par
	(4)假设此时$T$不一致连续,则存在$\varepsilon_0>0$以及点列$\{x_n\},\{y_n\}\subseteq A$,使得:
	\begin{equation*}
		\lim_{n\to+\infty}\rho_X(x_n,y_n)=0,\quad\rho_Y(Tx_n,Ty_n)\geqslant\varepsilon_0,\;\forall\;n\in\mathbb{N}^+
	\end{equation*}
	因为$A$是紧集,所以$\{x_n\}$存在子列$\{x_{n_k}\}\to x_0\in A$,即$\rho_X(x_{n_k},x_0)\to 0$,于是由\cref{prop:RSeq}(8.b)可得:
	\begin{equation*}
		\rho_X(y_{n_k},x_0)\leqslant\rho_X(y_{n_k},x_{n_k})+\rho_X(x_{n_k},x_0)\to0
	\end{equation*}
	因为$T$是连续的,所以:
	\begin{equation*}
		\rho_Y(Tx_{n_k},Tx_0)\to0,\;\rho_Y(Ty_{n_k},Tx_0)\to0
	\end{equation*}
	于是由\cref{prop:RSeq}(8.b)可得:
	\begin{equation*}
		\rho_Y(Tx_{n_k},Ty_{n_k})\leqslant\rho_Y(Tx_{n_k},Tx_0)+\rho_Y(Tx_0,Ty_{n_k})\to 0
	\end{equation*}
	与第一个式子中的第二部分矛盾,所以$T$一致连续。
\end{proof}

\subsection{值域为$\mathbb{R}^{},\mathbb{R}^{n}$的映射}
请自行给出定义在$\mathbb{R}^{}$且值域为$\mathbb{R}^{}$的映射极限的32个定义(趋于的点为实数、负无穷、正无穷、无穷,极限为实数、负无穷、正无穷、无穷,组合共16种,再考虑序列式与$\varepsilon-\delta$语言,一共32种)。\par
对于定义在$\mathbb{R}^{}$上的函数,考虑映射$f$沿$E$趋于$x_0\in\mathbb{R}^{}$时的极限,当$E$在数轴上完全位于$x_0$的左侧或右侧时,我们将极限分别简记为:
\begin{equation*}
	\lim_{x\to x_0^-}f(x),\quad\lim_{x\to x_0^+}f(x)
\end{equation*}
\begin{property}\label{prop:RMap}
	设$(X,\rho_X)$是度量空间,$E\subseteq X$,$x_0$是$E$的一个聚点,$f,g,h$是$E$到$\mathbb{R}^{}$上的映射,则:
	\begin{enumerate}
		\item 若$\lim\limits_{\substack{x\to x_0 \\ E}}f(x)$存在,则其极限唯一;
		\item (Squeeze Theorem) 若存在$\delta>0$,使得对任意的$x\in U(x_0,\delta)\cap E\setminus\{x_0\}$都有如下之一成立,则对应的结论也成立:
		\begin{enumerate}
			\item $f(x)\leqslant h(x)\leqslant g(x)$,且$\lim\limits_{\substack{x\to x_0 \\ E}}f(x)=\lim\limits_{\substack{x\to x_0 \\ E}}g(x)=a\in\mathbb{R}^{}$,则$\lim\limits_{\substack{x\to x_0 \\ E}}h(x)=a$;
			\item $h(x)\geqslant f(x)$,且$\lim\limits_{\substack{x\to x_0 \\ E}}f(x)=+\infty$,则$\lim\limits_{\substack{x\to x_0 \\ E}}h(x)=+\infty$;
			\item $h(x)\leqslant f(x)$,且$\lim\limits_{\substack{x\to x_0 \\ E}}f(x)=-\infty$,则$\lim\limits_{\substack{x\to x_0 \\ E}}h(x)=-\infty$;
		\end{enumerate}
		\item 若$\lim\limits_{\substack{x\to x_0 \\ E}}f(x)$和$\lim\limits_{\substack{x\to x_0 \\ E}}g(x)$存在,且有:
		\begin{equation*}
			\lim_{\substack{x\to x_0 \\ E}}f(x)<\lim_{\substack{x\to x_0 \\ E}}g(x)
		\end{equation*}
		则存在$\delta>0$,使得当$x\in U(x_0,\delta)\cap E\setminus\{x_0\}$时,$f(x)<g(x)$;
		\item 若存在$\delta>0$使得当$x\in U(x_0,\delta)\cap E\setminus\{x_0\}$时有$f(x)\leqslant g(x)$,且$\lim\limits_{\substack{x\to x_0 \\ E}}f(x)$和$\lim\limits_{\substack{x\to x_0 \\ E}}g(x)$都存在,则:
		\begin{equation*}
			\lim_{\substack{x\to x_0 \\ E}}f(x)\leqslant\lim_{\substack{x\to x_0 \\ E}}g(x)
		\end{equation*}
		\item 若$\lim\limits_{\substack{x\to x_0 \\ E}}f(x)$和$\lim\limits_{\substack{x\to x_0 \\ E}}g(x)$存在,且下式右侧有意义,则公式成立:
		\begin{enumerate}
			\item $\lim\limits_{\substack{x\to x_0 \\ E}}|f(x)|=\Big|\lim\limits_{\substack{x\to x_0 \\ E}}f(x)\Big|$;
			\item $\lim\limits_{\substack{x\to x_0 \\ E}}[f(x)\pm g(x)]=\lim\limits_{\substack{x\to x_0 \\ E}}f(x)\pm\lim\limits_{\substack{x\to x_0 \\ E}}g(x)$;
			\item $\lim\limits_{\substack{x\to x_0 \\ E}}f(x)g(x)=\lim\limits_{\substack{x\to x_0 \\ E}}f(x)\lim\limits_{\substack{x\to x_0 \\ E}}g(x)$;
			\item $\lim\limits_{\substack{x\to x_0 \\ E}}\dfrac{f(x)}{g(x)}=\dfrac{\lim\limits_{\substack{x\to x_0 \\ E}}f(x)}{\lim\limits_{\substack{x\to x_0 \\ E}}g(x)}$;
		\end{enumerate}
		该性质内蕴了连续函数四则运算的连续性问题。
	\end{enumerate}
\end{property}
\begin{property}
	值域为$\mathbb{R}^{n}$的函数的收敛等价于按坐标收敛\info{度量空间补充笛卡尔积的情况}。
\end{property}
\begin{note}
	考虑到极限情况很多,上面的性质证明起来特别麻烦,故略去。但作一点说明:\cref{theo:SequentialCharacterizationOfFunctionLimits}和\cref{theo:Cauchy-TypeConditionForFunctionLimits}都可以推广到趋于的点坐标含无穷的情况,其中\cref{theo:SequentialCharacterizationOfFunctionLimits}还可以推广到极限为无穷,证明过程类似之前的叙述。之后我们在引用\cref{theo:SequentialCharacterizationOfFunctionLimits}和\cref{theo:Cauchy-TypeConditionForFunctionLimits}时,将包含无穷的情况,不做额外的说明。上述性质的证明在使用\cref{theo:SequentialCharacterizationOfFunctionLimits}和\cref{theo:Cauchy-TypeConditionForFunctionLimits}的推广后,都可以由\cref{prop:RSeq}和\cref{prop:RmConvergence}(2)推出。
\end{note}
\begin{definition}
	称多项式函数、有理分式函数、三角函数、反三角函数、对数函数、指数函数为\gls{BasicElementaryFunction},经基本初等函数经过有限次四则运算和复合而成的函数被称为\gls{ElementaryFunction}。
\end{definition}
\begin{theorem}
	初等函数在其定义域内连续。
\end{theorem}
\subsubsection{连通性与介值性}
\begin{definition}
	设$(X,\rho)$是一个度量空间,$E\subseteq X$,$x_0,x_1\in E$。若连续映射$\gamma:[0,1]\to E$满足$\gamma(0)=x_0,\;\gamma(1)=x_1$,则称$\gamma$为$E$中联结$x_0$和$x_1$的一条\gls{Path}。
\end{definition}
\begin{definition}
	设$(X,\rho)$是一个度量空间,$E\subseteq X$。若对于任意的$x_0,x_1\in E$,都存在一条联结它们的路径,则称$E$\gls{PathConnected}。定义$\varnothing$也是路径连通的。
\end{definition}
\begin{lemma}\label{lem:IntermediateValueR}
	设函数$f:\mathbb{R}^{}\to\mathbb{R}^{}$在区间$[a,b]$上连续。
	\begin{enumerate}
		\item 若$f(a)f(b)<0$,则存在$c\in(a,b)$使得$f(c)=0$;
		\item 若$f(a)\ne f(b)$,则$f$在$[a,b]$上能取到介于$f(a)$和$f(b)$之间的所有值。
	\end{enumerate}
\end{lemma}
\begin{proof}
	(1)仅讨论$f(a)<0<f(b)$时的情况,$f(b)<0<f(a)$时完全类似。\par
	取$c_0=\dfrac{a+b}{2}$,则$f(c_0)=0$或$f(c_0)$与$f(a),f(b)$中的一个异号。若$f(c_0)$与$f(a)$异号,则设$a_1=a,\;b_1=c_0$;若$f(c_0)$与$f(b)$异号,则设$a_1=c_0,\;b_1=b$。取$c_1=\dfrac{a_1+b_1}{2}$,则$f(c_1)=0$或$f(c_1)$与$f(a_1),f(b_1)$中的一个异号。若$f(c_1)$与$f(a_1)$异号,则设$a_2=a_1,\;b_2=c_1$;若$f(c_1)$与$f(b_1)$异号,则设$a_2=c_1,\;b_2=b_1$。\par
	不断重复上述讨论,要么存在一个$c_n$使得$f(c_n)=0$,结论成立,要么得到一个闭区间套$\{[a_n,b_n]\}$和一个点列$\{c_n\}$,满足$a_n\leqslant c_n\leqslant b_n$。由\cref{theo:ClosedCubeTheorem}和\cref{prop:RSeq}(4.a)可得:
	\begin{equation*}
		\lim_{n\to+\infty}a_n=\lim_{n\to+\infty}c_n=\lim_{n\to+\infty}b_n\coloneq c\in(a,b)
	\end{equation*}
	因为$f$在区间$[a,b]$上连续,由\cref{prop:RSeq}(6)可得:
	\begin{equation*}
		f(c)=\lim_{n\to+\infty}f(a_n)\leqslant0\leqslant\lim_{n\to+\infty}f(b_n)=f(c)
	\end{equation*}
	所以$f(c)=0$。\par
	(2)对任意介于$f(a),f(b)$之间的$\alpha$,取辅助函数$f(x)-\alpha$。由\cref{prop:RMap}(5.b)可得$f(x)-\alpha$是区间$[a,b]$上的连续函数,由(1)即可得出结论。
\end{proof}
\begin{theorem}[Intermidiate Value Theorem]\label{theo:IntermediateValue}
	设$(X,\rho)$是一个度量空间,$E\subseteq X$路径连通,$f:E\to\mathbb{R}^{}$是一个连续函数,则$f(E)$是一个区间(允许退化为单点)。
\end{theorem}
\begin{proof}
	任取$y_0,y_1\in f(E)$,设$f(x_0)=y_0,\;f(x_1)=y_1$。因为$E$路径连通,所以存在连续映射$\gamma:[0,1]\to E$满足$\gamma(0)=x_0,\;\gamma(1)=x_1$。考虑复合映射$f\circ\gamma$,由\cref{theo:CompositeContinuousMap}可知$f\circ\gamma$在$[0,1]$上连续。因为:
	\begin{equation*}
		f\circ\gamma(0)=f(x_0)=y_0,\quad f\circ\gamma(1)=f(x_1)=y_1
	\end{equation*}
	由\cref{lem:IntermediateValueR}可知$f\circ\gamma$在$[0,1]$上可以取到介于$y_0$和$y_1$之间的所有值,即$f$在$E$上可以取到介于$y_0$和$y_1$之间的所有值。由$y_0,y_1$的任意性,结论成立。
\end{proof}

\subsubsection{无穷小与有界记号}
\begin{definition}
	设$f(x)$是在$U(a,\delta)$上有定义,$a\in\overline{\mathbb{R}^{}},\;\delta>0$。若$\lim\limits_{x\to a}f(x)=0$,则称$f(x)$是$x\to a$时的\gls{Infinitesimal};若$\lim\limits_{x\to a}f(x)=\infty$,则称$f(x)$是$x\to a$时的\gls{InfiniteQuantity}。
\end{definition}
\begin{definition}
	设$f(x),g(x)$是在$U(a,\delta)$上有定义,$a\in\overline{\mathbb{R}^{}},\;\delta>0$,$g(x)$在$U(a,\delta)$上不为$0$。
	\begin{enumerate}
		\item 若$\dfrac{f(x)}{g(x)}$是$x\to a$时的有界变量,则记$f(x)=\operatorname{O}(g(x))$;
		\item 若$\dfrac{f(x)}{g(x)}$是$x\to a$时的无穷小量,则记$f(x)=\operatorname{o}(g(x))$;
		\item 若$\lim\limits_{x\to a}\dfrac{f(x)}{g(x)}=1$,则记$f(x)\sim g(x)$。
	\end{enumerate}
\end{definition}
\begin{note}
	使用无穷小量、无穷大量以及上述记号时需要说明涉及的极限过程,如$f(x)=\operatorname{O}(g(x))(x\to a)$。等式两边都存在记号时需要注意此时等号的含义并不是等于,而是也是。
\end{note}
\begin{definition}
	设$f(x),g(x)$是无穷小(大)量。若$f(x)=o(g(x))$,则称$f(x)$是比$g(x)$更高(低)阶的无穷小(大)量;若$f(x)\sim g(x)$,则称$f(x)$是与$g(x)$等价的无穷小(大)量。
\end{definition}
\begin{property}
	设$f(x),g(x),h(x),p(x)$是在$U(a,\delta)$上有定义,$a\in\overline{\mathbb{R}^{}},\;\delta>0$,$g(x)\sim h(x)$,则:
	\begin{enumerate}
		\item $\operatorname{o}(f(x))=\operatorname{O}(f(x))$;
		\item $\operatorname{o}(f(x))+\operatorname{o}(f(x))=\operatorname{o}(f(x)),\;\operatorname{O}(f(x))+\operatorname{O}(f(x))=\operatorname{O}(f(x))$;
		\item $\operatorname{o}(f(x))\operatorname{O}(1)=\operatorname{o}(f(x)),\;\operatorname{o}(1)\operatorname{O}(f(x))=\operatorname{o}(f(x))$;
		\item $\lim\limits_{x\to a}f(x)g(x)=\lim\limits_{x\to a}f(x)h(x)$;
		\item $\lim\limits_{x\to a}\dfrac{f(x)g(x)}{p(x)}=\lim\limits_{x\to a}\dfrac{f(x)h(x)}{p(x)}$;
		\item $\lim\limits_{x\to a}\dfrac{f(x)}{g(x)p(x)}=\lim\limits_{x\to a}\dfrac{f(x)}{h(x)p(x)}$。
	\end{enumerate}
\end{property}
\begin{proof}
	(1)由定义立即可得。\par
	(2)由\cref{prop:RMap}(5.b)即可得到。\par
	(3)由\cref{prop:RMap}(5.c)即可得到。\par
	(4)由\cref{prop:RMap}(5.c)可得:
	\begin{equation*}
		\lim_{x\to a}f(x)g(x)=\lim_{x\to a}f(x)\frac{h(x)}{g(x)}g(x)=\lim_{x\to a}f(x)h(x)
	\end{equation*}\par
	(5)由(4)立即可得。\par
	(6)由\cref{prop:RMap}(5.c)可得:
	\begin{equation*}
		\lim_{x\to a}\dfrac{f(x)}{g(x)p(x)}=\lim_{x\to a}\dfrac{f(x)h(x)}{g(x)h(x)p(x)}=\lim_{x\to a}\dfrac{f(x)}{h(x)p(x)}\qedhere
	\end{equation*}
\end{proof}
\section{范数}
\begin{definition}
	设$X$是实或者复线性空间,如果对于$X$中的每个元素$x$,都有一个实数与之对应,记为$||x||$,且满足:
	\begin{enumerate}
		\item 非负性:$||x||\geqslant 0$,等号成立当且仅当$x=\mathbf{0}$。
		\item 数乘:$||\alpha x||=|\alpha|\;||x||$,$\alpha\in\mathbb{C}$或$\mathbb{R}$。
		\item 三角不等式:$||x+y||\leqslant||x||+||y||$。
	\end{enumerate}
	则称$X$为实或复的\gls{NormedLS},$||x||$为元素$x$的\gls{norm}。
\end{definition}
\begin{definition}
	对于赋范线性空间$X$,我们定义下式来衡量$X$中元素$x$和$y$之间的距离:
	\begin{equation*}
		\rho(x,y)=||x-y||
	\end{equation*}
\end{definition}
\begin{proof}
	(1)非负性可由范数的非负性直接验证。\par
	(2)对称性:由范数定义中的条件(2)可得$\rho(x,y)=||x-y||=|-1|\;||x-y||=||y-x||=\rho(y,x)$。\par
	(3)三角不等式:由范数定义中的条件(3)可得$\rho(x,y)=||x-y||=||x-z+z-y||\leqslant||x-z||+||z-y||=\rho(x,z)+\rho(z,y)$。
\end{proof}
\begin{definition}
	赋范线性空间$X$中,若点列$\{x_n\}$收敛于点$x$,则称$\{x_n\}$\gls{convergenceNorm}于$x$,也称$\{x_n\}$\gls{Strongconvergence}于$x$。
\end{definition}
\begin{property}\label{prop:Norm}
	范数具有如下性质:
	\begin{enumerate}
		\item 设$X$是一个赋范线性空间,$x,y\in X$,有:
		\begin{equation*}
			|\;||x||-||y||\;|\leqslant||x-y||
		\end{equation*}
		\item 范数是连续泛函\info{泛函的定义};
		\item 设$\{x_n\}$和$\{y_n\}$都是赋范线性空间$X$中的点列,且$\{x_n\}\rightarrow x$,$\{y_n\}\rightarrow y$,$\{a_n\}$和$\{b_n\}$是$\mathbb{R}^{}$或$\mathbb{C}^{}$中的点列,且$\{a_n\}\to a,\;\{b_n\}\to b,\;|a|,|b|\in R$,则$a_nx_n+b_ny_n\to ax+bx$。 
	\end{enumerate}
\end{property}
\begin{proof}
	(1)$||x||=||x-y+y||\leqslant||x-y||+||y||$,即$||x||-||y||\leqslant||x-y||$;$||y||=||y-x+x||\leqslant||x-y||+||x||$,即$-||x-y||\leqslant||x||-||y||$。\par
	综上,$-||x-y||\leqslant||x||-||y||\leqslant||x-y||$,即$|\;||x||-||y||\;|\leqslant||x-y||$。\par
	(2)由(1)可知$|\;||x_n||-||x||\;|\leqslant||x_n-x||$。因此当$\{x_n\}$依范数收敛于$x$时,$||x_n||\to||x||$。\par
	(3)由范数的定义:
	\begin{align*}
		||a_nx_n+b_ny_n-(ax+by)||
		&\leqslant||a_nx_n-ax||+||b_ny_n-by|| \\
		&=||a_nx_n-a_nx+a_nx-ax||+||b_ny_n-b_ny+b_ny-by|| \\
		&\leqslant|a_n|\;||x_n-x||+|a_n-a|\;||x||+|b_n|\;||y_n-y||+|b_n-b|\;||y||
	\end{align*}
	由$\{a_n\}$和$\{b_n\}$的收敛性以及$\{x_n\}$和$\{y_n\}$的收敛性立即可得$a_nx_n+b_ny_n\to ax+bx$。
\end{proof}
\begin{definition}
	在$\mathbb{R}^n$中定义元素$x=(\xi_1,\xi_2,\dots,\xi_n)$的范数为:
	\begin{equation*}
		||x||_2=\left(\sum_{i=1}^n\xi_i^2\right)^{\frac{1}{2}}
	\end{equation*}
	则$\mathbb{R}^n$成为一个赋范线性空间。
\end{definition}
\begin{proof}
	(1)$\;||x||_2\in\mathbb{R}$、(2)非负性和(3)数乘显然,(4)三角不等式的证明可见欧式距离三角不等式的证明。
\end{proof}