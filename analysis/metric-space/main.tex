\chapter{度量空间}

\begin{definition}
	设$X$为一个非空集合。若对任意的$x,y\in X$,都$\exists\;\rho(x,y)\in\mathbb{R}$与$x,\;y$对应,且满足以下三个条件(即距离公理):
	\begin{enumerate}
		\item 非负性:$\rho(x,y)\geqslant0$,等号成立当且仅当$x=y$。
		\item 对称性:$\rho(x,y)=\rho(y,x)$。
		\item 三角不等式:$\rho(x,y)\leqslant\rho(x,z)+\rho(z,y),\;\forall\;z\in X$。
	\end{enumerate}
	则称$\rho$是$X$上的一个距离,$X$是以$\rho$为距离的\gls{MetricSpace},记为$(X,\rho)$。度量空间中的元素又称之为点。
\end{definition}

\section{度量空间上的点集}
\subsubsection{内、外、界点定义}
\begin{definition}
	$(X,\rho)$是一个度量空间,$E\subseteq X$。若对于点$P\in X$,存在$U(P)\subseteq E$,则称点$P$为$E$的\gls{InteriorP}。
\end{definition}
\begin{definition}
	$(X,\rho)$是一个度量空间,$E\subseteq X$。若点$P\in X$为$E^c$的内点,则称点$P$为$E$的\gls{ExteriorP}。
\end{definition}
\begin{definition}
	$(X,\rho)$是一个度量空间,$E\subseteq X$。若对于点$P\in X$,它的任意邻域中都既有$E$中的点,又有$E^c$中的点,则称点$P$为$E$的\gls{BoundaryP}。
\end{definition}
\subsubsection{聚点与孤立点定义}
\begin{definition}
	$(X,\rho)$是一个度量空间,$E\subseteq X$。若点$P\in X$满足以下任一条件:
	\begin{enumerate}
		\item $P$的任何邻域内都存在无穷个$E$中的点。
		\item $P$的任何邻域内都存在一个$E$中异于$P$的点。
		\item $E$中有一个极限为$P$的点列。
	\end{enumerate}
	则$P$是$E$的一个\gls{LimitP}。
\end{definition}
这三个条件实际上是等价的:
\begin{proof}
	$(1)\to(2)$显然,$(3)\to(1)$显然,下证$(2)\to(3)$。
	取$\delta_n=\frac{1}{n}$,在$U(P,\delta_n)$中至少有一点$P_n$,$P_n\in E$且$P_n\ne P$,显然$\{P_n\}\to P$。
\end{proof}
\begin{definition}
	$(X,\rho)$是一个度量空间,$E\subseteq X$。若点$P\in E$但不是$E$的聚点,即存在$P$的邻域$U(P)$,使得$E\cap U(P)=\{P\}$,则称$P$为$E$的\gls{IsolatedP}。
\end{definition}
\subsubsection{开核、导集、边界、闭包的定义}
\begin{definition}
	$(X,\rho)$是一个度量空间,$E\subseteq X$。$E$的全体内点所组成的集合称为$E$的\gls{interior},记为$\mathring{E}$。
\end{definition}
\begin{definition}
	$(X,\rho)$是一个度量空间,$E\subseteq X$。$E$的全体聚点所组成的集合称为$E$的\gls{DerivedSet},记为$E'$。
\end{definition}
\begin{definition}
	$(X,\rho)$是一个度量空间,$E\subseteq X$。$E$的全体界点所组成的集合称为$E$的\gls{boarder},记为$\partial E$。
\end{definition}
\begin{definition}
	$(X,\rho)$是一个度量空间,$E\subseteq X$。$E\cup E'$称为$E$的\gls{closure},记为$\overline{E}$。
\end{definition}
\subsubsection{开集、闭集的定义}
\begin{definition}
	$(X,\rho)$是一个度量空间,$E\subseteq X$。若$E=\mathring{E}$,则称E是一个\gls{OpenSet}。
\end{definition}
\begin{definition}
	$(X,\rho)$是一个度量空间,$E\subseteq X$。若$E'\subset E$,则称E是一个\gls{ClosedSet}。
\end{definition}
\subsubsection{子空间}
\begin{definition}
	$(X,\rho_X)$是一个度量空间。若$E\subset X$,且在$E$上定义距离$\rho_E$使得$E$中任意点对$x,y$之间的距离$\rho_E(x,y)=\rho_X(x,y)$,则称$E$是$X$的一个\gls{subspace}。若$E$是一个开集,则称$E$是$X$的一个\gls{OpenSubspace};$E$是一个闭集,则称$E$是$X$的一个\gls{ClosedSubspace}。
\end{definition}

\begin{property}\label{prop:OpenClosedSet}
	$(X,\rho)$是一个度量空间,$P\in X,\;E\subseteq X$,则有:
	\begin{enumerate}
		\item $\forall\;P_0\in U(P),\;\exists\;U(P_0)\subset U(P)$,即开邻域是开集。
		\item $\forall\;P_0\in\left(\bar{U}(P)\right)',\;P_0\in \bar{U}(P_0)$,即闭邻域是闭集;
		\item $\mathring{E}$是开集,$E'$和$\overline{E}$是闭集;
		\item $X$和$\varnothing$是唯二既开又闭的集合;
		\item 若$E$是开集,则$E^c$是闭集;若$E$是闭集,则$E^c$是开集;
		\item 任意个开集的并是开集,有限个开集的交是开集;任意个闭集的交是闭集,有限个闭集的并是闭集。
	\end{enumerate}
\end{property}
\begin{proof}
	(1)要使得$U(P_0)\subset U(P)$,则$\forall\;x\in U(P_0),\;\rho(x,P)<\delta_P$,而$\rho(x,P)<\rho(x,P_0)+\rho(P_0,P)<\delta_{P_0}+\rho(P_0,P)$,故只要使$\delta_{P_0}+\rho(P_0,P)<\delta_P$即可。 \par
	(2)对任意的$P_0\in\left(\bar{U}(P)\right)'$,$\exists\;\bar{U}(P)$中的点列$\{P_n\}\to P_0$。若$P_0\notin \bar{U}(P)$,则$\rho(P_0,P)>\delta_P$。取$\varepsilon=\rho(P_0,P)-\delta_P$,则必然$\exists\;N\in\mathbb{N}^+$,当$n>N$时,有$\rho(P_n,P_0)<\varepsilon$,那么$\rho(P_n,P)>\delta_P$,即$P_n\notin U(P)$,矛盾。因此$P_0\in \bar{U}(P)$。由$P_0$的任意性,闭邻域是闭集。\par
	(3)$\mathring{E}$是开集:对任意的$ P\in\mathring{E}$,存在$U(P)\subset E$,对任意的$ Q\in U(P)$,存在$U(Q)\subset U(P)\subset E$,因此$Q\in\mathring{E}$,即存在$U(P)\subset\mathring{E}$,$P$是$\mathring{E}$的内点。由$P$的任意性,$\mathring{E}$是开集。\par
	$E'$是闭集:对任意的$ P\in (E')'$,由聚点定义,任意$U(P)$中都含有$E'$中的点,任取$Q\in U(P)\cap E'$,任意$U(Q)$都含$E$中的点,而$Q\in U(P)$,由邻域的性质,存在$U(Q)\subset U(P)$,因此任意$U(P)$中都含有$E$中的点,即$P$是$E$的聚点,$P\in E'$。由$P$的任意性与$P$的选取方式,$E'$是闭集。\par
	$\overline{E}$是闭集:$\overline{E}=E\cup E'$,由\cref{theo:cuplimitset eq2 limitsetcup},$(\overline{E})'=E'\cup (E')'$,显然$E'\subset \overline{E}$,而$E'$是闭集,$(E')'\subset E'\subset\overline{E}$。因此,$(\overline{E})'\subset\overline{E}$,即$\overline{E}$是闭集。\par
	(4)\par
	(5)$E$是开集:$\forall\; P\in (E^c)'$,应有$P\notin E$,即$P$不会是$E$的内点。否则由内点定义,存在$U(P)$使得$U(P)\subset E$,即$U(P)$中不含$E^c$中的点,这与点$P\in (E^c)'$矛盾。由$P$的任意性有$E'\subset E$,即$E^c$是闭集。\par
	$E$是闭集:$\forall\; P\in E^c$,应有$P$是$E^c$的一个内点。否则任意$U(P)$中都存在$E$中的点,则$P$应该是$E$的一个聚点,而此时$P\in E^c$,与$E$是闭集矛盾。由$P$的任意性有$E^c=\mathring{(E^c)}$,即$E^c$是开集。\par
	(6)\textbf{任意个开集的并是开集:}对于在任意个开集的并集中的点$P$,$P$必属于其中一个开集,因而存在一个邻域$U(P)$使得其中的点都在那个开集中,进而$U(P)$都在任意个开集的并集中,即$P$是任意个开集的并集的内点。由$P$的任意性,任意个开集的并集是开集。\par
	\textbf{有限个开集的交是开集:}设这有限个开集为$A_i,i=1,2,\dots,n$。对任意的$ P\in\underset{i=1}{\overset{n}{\cap}}A_i$,则$P$是所有$A_i$的内点,因此存在$U_i(P)\subset A_i$。由邻域的性质,存在$U(P)$满足$U(P)\subset\underset{i=1}{\overset{n}{\cap}}U_i(P)\subset\underset{i=1}{\overset{n}{\cap}}A_i$,因此$P$是$\underset{i=1}{\overset{n}{\cap}}A_i$的内点。由$P$的任意性,$\underset{i=1}{\overset{n}{\cap}}A_i$是开集。\par
	\textbf{任意个闭集的交是闭集,有限个闭集的并是闭集:}可由(5)以及\info{De-morgan law}在(1)(2)的基础上直接得出。
\end{proof}
\subsubsection{内外界聚孤立五点的性质}
\begin{theorem}
	$(X,\rho)$是一个度量空间,$E\subset X$。$E$的界点不是聚点就是孤立点。
\end{theorem}
\begin{proof}
	若$P$是$E$的一个界点,则它的任意邻域中都既有$E$中的点,又有$E^c$中的点。若它的任意邻域中都只含有有限个$E$中的点,则一定存在一个邻域,使得只有自身属于$E$,那么它是一个孤立点;若它的任意邻域中都含有无限个$E$中的点,由聚点定义,它是一个聚点。
\end{proof}
\subsubsection{关于内外界聚孤立五点的总结}
\begin{equation*}
	\text{度量空间}(X,\rho)\text{中的点与$E$的关系可分为}
	\begin{cases}
		\text{内点} \\
		\text{界点} \\
		\text{外点}    
	\end{cases}\;\text{或}
	\begin{cases}
		\text{聚点}   \\
		\text{孤立点} \\
		\text{外点}    
	\end{cases}
\end{equation*}
\subsubsection{开核、导集、边界、闭包的性质}
\begin{theorem}
	$(X,\rho)$是一个度量空间,$E\subset X$。$(\mathring{E})^c=\overline{E^c}$。
\end{theorem}
\begin{proof}
	$\overline{E^c}=E^c\cup (E^c)'$。\par
	对任意的$ P\in\overline{E^c}\cap\mathring{E}$,应有$P\in(E^c)'$,即$P$的任意邻域中都有$E^c$中的点,而这与$P\in\mathring{E}$矛盾,由$P$的任意性有$\overline{E^c}\cap\mathring{E}=\varnothing$,即$\overline{E^c}\subset(\mathring{E})^c$;\par
	对任意的$ P\in(\mathring{E})^c$,应有$P\in\mathring{(E^c)}\cup\partial E$。若$P\in\mathring{(E^c)}$,显然$P\in\mathring{(E^c)}\subset E^c\subset\overline{E^c}$;若$P\in\partial E$,要么$P\in E$,要么$P\notin E$,如果此时$P\in E$,则由界点定义它必然是$E^c$的聚点,即$P\in \overline{E^c}$;如果此时$P\notin E$,则$P\in E^c$,即$P\in \overline{E^c}$。由$P$的任意性有$(\mathring{E})^c\subset\overline{E^c}$。\par
	综上$(\mathring{E})^c=\overline{E^c}$。
\end{proof}
\begin{theorem}
	$(X,\rho)$是一个度量空间,$E\subset X$。$\partial E = \partial E^c$。
\end{theorem}
\begin{proof}
	由定义直接可得。
\end{proof}
\begin{theorem}
	$(X,\rho)$是一个度量空间,$E\subset X$。$\overline{\overline E}=\overline{E}$。
\end{theorem}
\begin{proof}
	$\overline{\overline E}=\overline{E}\cup (\overline{E})'$,由\cref{prop:OpenClosedSet}(3),$\overline{E}$是闭集,因此$(\overline{E})'\subset\overline{E}$,即$\overline{\overline E}=\overline{E}$。
\end{proof}
\begin{theorem}\label{theo:subsetilc}
	$(X,\rho)$是一个度量空间,$A\cup B\subset X$。若$A\subset B$,则$\mathring{A}\subset\mathring{B}$,$A'\subset B'$,$\overline{A}\subset\overline{B}$。
\end{theorem}
\begin{proof}
	由内点、聚点的定义直接可得。
\end{proof}
\begin{theorem}\label{theo:cuplimitset eq2 limitsetcup}
	$(X,\rho)$是一个度量空间,$A\cup B\subset X$。$(A\cup B)'=A'\cup B'$。
\end{theorem}
\begin{proof}
	$A\subset (A\cup B)$,$B\subset (A\cup B)$,由\cref{theo:subsetilc}可知,$A'\subset(A\cup B)'$,$B'\subset(A\cup B)'$,因此$A'\cup B'\subset(A\cup B)'$。\par
	另一方面,对任意的$ P\in (A\cup B)'$,应有$P\in A'\cup B'$。否则就有$P\notin A'$且$P\notin B'$,因此存在$U_1(P)$和$U_2(P)$使得$U_1(P)$中不含除了$P$以外的$A$中的点、$U_2(P)$中不含除了$P$以外的$B$中的点,由邻域的性质,存在$U_3(P)$使得其内不含除了$P$以外$A\cup B$中的点,而这与$P\in (A\cup B)'$矛盾。由$P$的任意性有$(A\cup B)'\subset A'\cup B'$。\par
	综上$(A\cup B)'=A'\cup B'$。
\end{proof}
\begin{theorem}
	$(X,\rho)$是一个度量空间,$A\cup B\subset X$。$\overline{A\cup B}=\overline{A}\cup \overline{B}$。
\end{theorem}
\begin{proof}
	由\cref{theo:cuplimitset eq2 limitsetcup}:
	\begin{equation*}
		\overline{A}\cup \overline{B}=A\cup A'\cup B\cup B'
		=(A\cup B)\cup(A'\cup B')=(A\cup B)\cup(A\cup B)'
		=\overline{A\cup B}\qedhere
	\end{equation*}
\end{proof}
\section{度量空间上点集的性质}
\subsubsection{邻域的性质}
\begin{property}
	$(X,\rho)$是一个度量空间,$P\in X$,则有:
	\begin{enumerate}
		\item $\forall\;\delta>0,\;P\in U(P,\delta)$。
		\item $\forall\;\delta_1,\delta_2>0,\;\exists\;\delta_3>0,\;U(P,\delta_3)\subset U(P,\delta_1)\cap U(P,\delta_2)$。
		\item $\forall\;P_0\in X,\;P_0\ne P,\;\exists\;U(P),U(P_0),\;U(P)\cap U(P_0)=\varnothing$。
		\item $\forall\;P_0\in U(P),\;\exists\;U(P_0)\subset U(P)$,即开邻域是开集。
		\item $\forall\;P_0\in\left(\bar{U}(P)\right)',\;P_0\in \bar{U}(P_0)$,即闭邻域是闭集。
	\end{enumerate}
\end{property}
\begin{proof}
	(2)取$\delta_3<min(\delta_1,\delta_2)$。
	(3)取$\delta\in(0,\frac{\rho(P,P_0)}{2})$。 \par
	(4)要使得$U(P_0)\subset U(P)$,则$\forall\;x\in U(P_0),\;\rho(x,P)<\delta_P$,而$\rho(x,P)<\rho(x,P_0)+\rho(P_0,P)<\delta_{P_0}+\rho(P_0,P)$,故只要使$\delta_{P_0}+\rho(P_0,P)<\delta_P$即可。 \par
	(5)对任意的$P_0\in\left(\bar{U}(P)\right)'$,$\exists\;\bar{U}(P)$中的点列$\{P_n\}\to P_0$。若$P_0\notin \bar{U}(P)$,则$\rho(P_0,P)>\delta_P$。取$\varepsilon=\rho(P_0,P)-\delta_P$,则必然$\exists\;N\in\mathbb{N}^+$,当$n>N$时,有$\rho(P_n,P_0)<\varepsilon$,那么$\rho(P_n,P)>\delta_P$,即$P_n\notin U(P)$,矛盾。因此$P_0\in \bar{U}(P)$。由$P_0$的任意性,闭邻域是闭集。
\end{proof}
\subsubsection{内外界聚孤立五点的性质}
\begin{theorem}
	$(X,\rho)$是一个度量空间,$E\subset X$。$E$的界点不是聚点就是孤立点。
\end{theorem}
\begin{proof}
	若$P$是$E$的一个界点,则它的任意邻域中都既有$E$中的点,又有$E^c$中的点。若它的任意邻域中都只含有有限个$E$中的点,则一定存在一个邻域,使得只有自身属于$E$,那么它是一个孤立点;若它的任意邻域中都含有无限个$E$中的点,由聚点定义,它是一个聚点。
\end{proof}
\subsubsection{关于内外界聚孤立五点的总结}
\begin{equation*}
	\text{度量空间}(X,\rho)\text{中的点与$E$的关系可分为}
	\begin{cases}
		\text{内点} \\
		\text{界点} \\
		\text{外点}    
	\end{cases}\;\text{或}
	\begin{cases}
		\text{聚点}   \\
		\text{孤立点} \\
		\text{外点}    
	\end{cases}
\end{equation*}
\subsubsection{开核、导集、边界、闭包的性质}
\begin{theorem}
	$(X,\rho)$是一个度量空间,$E\subset X$。$(\mathring{E})^c=\overline{E^c}$。
\end{theorem}
\begin{proof}
	$\overline{E^c}=E^c\cup (E^c)'$。\par
	对任意的$ P\in\overline{E^c}\cap\mathring{E}$,应有$P\in(E^c)'$,即$P$的任意邻域中都有$E^c$中的点,而这与$P\in\mathring{E}$矛盾,由$P$的任意性有$\overline{E^c}\cap\mathring{E}=\varnothing$,即$\overline{E^c}\subset(\mathring{E})^c$;\par
	对任意的$ P\in(\mathring{E})^c$,应有$P\in\mathring{(E^c)}\cup\partial E$。若$P\in\mathring{(E^c)}$,显然$P\in\mathring{(E^c)}\subset E^c\subset\overline{E^c}$;若$P\in\partial E$,要么$P\in E$,要么$P\notin E$,如果此时$P\in E$,则由界点定义它必然是$E^c$的聚点,即$P\in \overline{E^c}$;如果此时$P\notin E$,则$P\in E^c$,即$P\in \overline{E^c}$。由$P$的任意性有$(\mathring{E})^c\subset\overline{E^c}$。\par
	综上$(\mathring{E})^c=\overline{E^c}$。
\end{proof}
\begin{theorem}
	$(X,\rho)$是一个度量空间,$E\subset X$。$\partial E = \partial E^c$。
\end{theorem}
\begin{proof}
	由定义直接可得。
\end{proof}
\begin{theorem}
	$(X,\rho)$是一个度量空间,$E\subset X$。$\overline{\overline E}=\overline{E}$。
\end{theorem}
\begin{proof}
	$\overline{\overline E}=\overline{E}\cup (\overline{E})'$,由\cref{theo:InteriorOpenSet LimitsetClosureClosedset},$\overline{E}$是闭集,因此$(\overline{E})'\subset\overline{E}$,即$\overline{\overline E}=\overline{E}$。
\end{proof}
\begin{theorem}\label{theo:subsetilc}
	$(X,\rho)$是一个度量空间,$A\cup B\subset X$。若$A\subset B$,则$\mathring{A}\subset\mathring{B}$,$A'\subset B'$,$\overline{A}\subset\overline{B}$。
\end{theorem}
\begin{proof}
	由内点、聚点的定义直接可得。
\end{proof}
\begin{theorem}\label{theo:cuplimitset eq2 limitsetcup}
	$(X,\rho)$是一个度量空间,$A\cup B\subset X$。$(A\cup B)'=A'\cup B'$。
\end{theorem}
\begin{proof}
	$A\subset (A\cup B)$,$B\subset (A\cup B)$,由\cref{theo:subsetilc}可知,$A'\subset(A\cup B)'$,$B'\subset(A\cup B)'$,因此$A'\cup B'\subset(A\cup B)'$。\par
	另一方面,对任意的$ P\in (A\cup B)'$,应有$P\in A'\cup B'$。否则就有$P\notin A'$且$P\notin B'$,因此存在$U_1(P)$和$U_2(P)$使得$U_1(P)$中不含除了$P$以外的$A$中的点、$U_2(P)$中不含除了$P$以外的$B$中的点,由邻域的性质,存在$U_3(P)$使得其内不含除了$P$以外$A\cup B$中的点,而这与$P\in (A\cup B)'$矛盾。由$P$的任意性有$(A\cup B)'\subset A'\cup B'$。\par
	综上$(A\cup B)'=A'\cup B'$。
\end{proof}
\begin{theorem}
	$(X,\rho)$是一个度量空间,$A\cup B\subset X$。$\overline{A\cup B}=\overline{A}\cup \overline{B}$。
\end{theorem}
\begin{proof}
	由\cref{theo:cuplimitset eq2 limitsetcup}:
	\begin{equation*}
		\overline{A}\cup \overline{B}=A\cup A'\cup B\cup B'
		=(A\cup B)\cup(A'\cup B')=(A\cup B)\cup(A\cup B)'
		=\overline{A\cup B}\qedhere
	\end{equation*}
\end{proof}
\begin{theorem}\label{theo:InteriorOpenSet LimitsetClosureClosedset}
	$(X,\rho)$是一个度量空间,$E\subset X$。$\mathring{E}$是开集,$E'$和$\overline{E}$都是闭集。
\end{theorem}
\begin{proof}
	(1)$\mathring{E}$是开集:对任意的$ P\in\mathring{E}$,存在$U(P)\subset E$,对任意的$ Q\in U(P)$,存在$U(Q)\subset U(P)\subset E$,因此$Q\in\mathring{E}$,即存在$U(P)\subset\mathring{E}$,$P$是$\mathring{E}$的内点。由$P$的任意性,$\mathring{E}$是开集。\par
	(2)$E'$是闭集:对任意的$ P\in (E')'$,由聚点定义,任意$U(P)$中都含有$E'$中的点,任取$Q\in U(P)\cap E'$,任意$U(Q)$都含$E$中的点,而$Q\in U(P)$,由邻域的性质,存在$U(Q)\subset U(P)$,因此任意$U(P)$中都含有$E$中的点,即$P$是$E$的聚点,$P\in E'$。由$P$的任意性与$P$的选取方式,$E'$是闭集。\par
	(3)$\overline{E}$是闭集:$\overline{E}=E\cup E'$,由\cref{theo:cuplimitset eq2 limitsetcup},$(\overline{E})'=E'\cup (E')'$,显然$E'\subset \overline{E}$,而$E'$是闭集,$(E')'\subset E'\subset\overline{E}$。因此,$(\overline{E})'\subset\overline{E}$,即$\overline{E}$是闭集。
\end{proof}
\subsection*{开集、闭集的性质}
\begin{theorem}
	对于度量空间$(X,\rho)$,$X$和$\varnothing$是唯二既开又闭的集合。
\end{theorem}
\subsubsection{开集与闭集的互补性}
\begin{theorem}\label{theo:duality between openset and closed set}
	$(X,\rho)$是一个度量空间,$E\subset X$。若$E$是开集,则$E^c$是闭集;若$E$是闭集,则$E^c$是开集。
\end{theorem}
\begin{proof}
	$E$是开集:$\forall\; P\in (E^c)'$,应有$P\notin E$,即$P$不会是$E$的内点。否则由内点定义,存在$U(P)$使得$U(P)\subset E$,即$U(P)$中不含$E^c$中的点,这与点$P\in (E^c)'$矛盾。由$P$的任意性有$E'\subset E$,即$E^c$是闭集。\par
	$E$是闭集:$\forall\; P\in E^c$,应有$P$是$E^c$的一个内点。否则任意$U(P)$中都存在$E$中的点,则$P$应该是$E$的一个聚点,而此时$P\in E^c$,与$E$是闭集矛盾。由$P$的任意性有$E^c=\mathring{(E^c)}$,即$E^c$是开集。
\end{proof}
\subsubsection{开集与闭集的交与并}
\begin{theorem}\label{theo:CapAndCupOfOpenSetAndClosedSet}
	$(X,\rho)$是一个度量空间。(1)任意个开集的并是开集;(2)有限个开集的交是开集;(3)任意个闭集的交是闭集;(4)有限个闭集的并是闭集。
\end{theorem}
\begin{proof}
	(1)对于在任意个开集的并集中的点$P$,$P$必属于其中一个开集,因而存在一个邻域$U(P)$使得其中的点都在那个开集中,进而$U(P)$都在任意个开集的并集中,即$P$是任意个开集的并集的内点。由$P$的任意性,任意个开集的并集是开集。\footnote{这里使用纯文字说明是因为考虑该定理对不可数个集合也成立,当集合不可数的时候是无法给集合编号的。}\par
	(2)设这有限个开集为$A_i,i=1,2,\dots,n$。对任意的$ P\in\underset{i=1}{\overset{n}{\cap}}A_i$,则$P$是所有$A_i$的内点,因此存在$U_i(P)\subset A_i$。由邻域的性质,存在$U(P)$满足$U(P)\subset\underset{i=1}{\overset{n}{\cap}}U_i(P)\subset\underset{i=1}{\overset{n}{\cap}}A_i$,因此$P$是$\underset{i=1}{\overset{n}{\cap}}A_i$的内点。由$P$的任意性,$\underset{i=1}{\overset{n}{\cap}}A_i$是开集。\par
	(3)(4)可由\cref{theo:duality between openset and closed set}以及De-Morgan law在(1)(2)的基础上直接得出。
\end{proof}
\section{稠密、稀疏、可分性、有界性}
\subsection{稠密}
\begin{definition}
	$(X,\rho)$是一个度量空间,$A\cup B\subset X$。若$A\subset\overline{B}$,则称$B$在$A$中\gls{dense}。
\end{definition}
\subsubsection{稠密性等价定义}
\begin{theorem}
	$(X,\rho)$是一个度量空间,$A\cup B\subset X$。以下四个命题等价:
	\begin{enumerate}
		\item $B$在$A$中稠密。
		\item 对任意的$ x\in A$,$x$的任意邻域中都存在$B$中的点。
		\item 对任意的$ x\in A$,$B$中存在一个点列$\{x_n\}$收敛于$x$(点列收敛的定义参考\cref{def:convergence of range of points})。
		\item 对任意的$\varepsilon>0$,$B$中每个点的$\varepsilon$开球邻域的并包含了$A$。
	\end{enumerate}
\end{theorem}
\begin{proof}
	$(1)\to(2)$因为$\overline{B}=B\cup B'$,所以若$x\in A$,则$x\in B$或$x\in B'$,此时显然(2)成立。\par
	$(2)\to(3)$显然。(点列中元素可重复出现)\par
	$(3)\to(4)$若此时(4)不成立,即存在$x\in A$对于某个$\varepsilon$,满足条件$x$不在任何$B$中点的$\varepsilon$开球邻域内,则存在$U(x)$使得$U(x)$中不含$B$中的点,那么(3)也不可能成立,矛盾。\par
	$(4)\to(1)$对任意的$ x\in A$,如果$x\notin \overline{B}$,那么存在$U(x,\delta)$使得$U(x,\delta)$中不含$B$中的点,记$x$与$B$中点之间的最小距离为$\rho$,此时只要取$\varepsilon<\rho$,$x$就不在那些邻域的并集中了,矛盾。由$x$的任意性,$A\subset\overline{B}$。
\end{proof}
\subsection{稀疏集}
\begin{definition}
	$(X,\rho)$是一个度量空间,$A$是$X$的一个子集。若$A$在$X$的任何一个非空开集中均不稠密,则称$A$为$X$中的\gls{NowhereDenseSet}。
\end{definition}
\subsubsection{稀疏集等价定义}
下述定理中的邻域既可以是开邻域也可以是闭邻域:
\begin{theorem}
	度量空间$(X,\rho)$的子集$A$为稀疏集的充分必要条件是:对任意的$x\in X$和$x$的任意邻域$U(x)$,存在一个包含于邻域$U(x)$的邻域$U(y)$使得:
	\begin{equation*}
		A\cap U(y)=\varnothing
	\end{equation*}
\end{theorem}
\begin{proof}
	\textbf{(1)开邻域:}\par
	充分性:假设此时$A$在$X$中的某非空开集$B$中稠密,则$B\subset\overline{A}$。任取$x\in B$和$U(x)\subset B$,则$x\in\overline{A}$。如果此时存在包含于$U(x)$的$U(y)$使得题目条件成立,则$y\notin\overline{A}$,这与$y\in U(y)\subset U(x)\subset B\subset\overline{A}$矛盾。\par
	必要性:若$A$是稀疏集,则$A$在$U(x)$中不稠密。由稠密的等价定义(2),$\exists\;y\in U(x),\;\exists\;U(y)\subset U(x)$使得$A\cap U(y)=\varnothing$。\par
	\textbf{(2)闭邻域:}\par
	充分性:\par
	必要性:\info{有空证明}
\end{proof}
\subsection{可分性}
\begin{definition}
	$(X,\rho)$是一个度量空间,$A\subset X$。若存在可列集$B\subset X$使得$B$在$A$中稠密,则称$A$是\gls{separable}。若$X$中存在一个在$X$中稠密的可列子集,则称$X$是可分的度量空间。
\end{definition}
\subsection{有界性}
\begin{definition}
	$(X,\rho)$是一个度量空间,$E\subset X$。定义
	\begin{equation*}
		\delta_E=\sup_{x,y\in E}\rho(x,y)
	\end{equation*}
	为点集$E$的\gls{diameter}。
\end{definition}
\begin{definition}
	$(X,\rho)$是一个度量空间,$E\subset X$。若$\delta_E<+\infty$,则称$E$是$(X,\rho)$中的\gls{BoundedSet}。
\end{definition}
\subsubsection{有界集的等价定义}
\begin{definition}\label{def:BoundedSet2}
	$(X,\rho)$是一个度量空间,$E\subset X$。如果$E$包含在$X$中某个点的闭邻域或开邻域内,则称$E$是$(X,\rho)$中的\gls{BoundedSet}。
\end{definition}
二者的等价性很直观:
\begin{proof}
	$(1)\to(2)$:任取$x\in X$,取$\delta=\delta_E$构造闭邻域$U(x,\delta)$或取$\delta=\delta_E+\varepsilon(\forall\;\varepsilon>0)$构造开邻域$U(x,\delta)$。\par
	$(2)\to(1)$:设关于该点的开邻域或闭邻域半径为$\delta$,显然$\delta_E\leqslant 2\delta$。
\end{proof}
\section{度量空间上的收敛点列}
\begin{definition}\label{def:convergence of range of points}
	$\{x_n\}$是度量空间$(X,\rho)$中的一个\gls{SeqOfPoints},如果对任意的$\varepsilon>0$,$\exists\; N\in\mathbb{N}^+$,当$n>N$时有:
	\begin{equation*}
		\rho(x_n,x)<\varepsilon
	\end{equation*}
	则称$\{x_n\}$是度量空间$(X,\rho)$中的\gls{ConvergentSeqOfPoints},$x$是点列$\{x_n\}$的\gls{limit}。
\end{definition}
\subsection*{收敛点列的性质}
\subsubsection{极限的唯一性}
\begin{theorem}
	$\{x_n\}$是度量空间$(X,\rho)$中的一个收敛点列。$\{x_n\}$的极限是唯一的。
\end{theorem}
\begin{proof}
	假设极限不唯一,$\{x_n\}$既收敛到$a$又收敛到$b$,则对任意的$\varepsilon>0$,$\exists\; N_1\in\mathbb{N}^+$,当$n>N_1$时有$\rho(x_n,a)<\frac{\varepsilon}{2}$;$\exists\; N_2\in\mathbb{N}^+$,当$n>N_2$时有$\rho(x_n,b)<\frac{\varepsilon}{2}$。取$N=max\{N1,N2\}$,则当$n>N$时,有
	\begin{equation*}
		\rho(a,b)\leqslant\rho(a,x_n)+\rho(x_n,b)<\varepsilon
	\end{equation*}
	即$a=b$。
\end{proof}
\subsubsection{有界性}
\begin{theorem}
	$\{x_n\}$是度量空间$(X,\rho)$中的一个收敛点列。对任意的$ y\in X$,数列$\{\rho(x_n,y)\}$有界。
\end{theorem}
\begin{proof}
	设$\{x_n\}\to x$,由距离的定义:
	\begin{equation*}
		\rho(x_n,y)\leqslant\rho(x_n,x)+\rho(x,y)
	\end{equation*}
	由于$\{x_n\}$收敛,所以对$\varepsilon=1$:
	\begin{equation*}
		\exists\;N\in\mathbb{N}^+,\;\forall\;n>N,\;\rho(x_n,x)<\varepsilon=1
	\end{equation*}
	取$K=\max\{\rho(x_1,x),\rho(x_2,x),\dots,\rho(x_N,x),1\}$,则有:
	\begin{equation*}
		\forall\;n\in\mathbb{N}^+,\;\rho(x_n,y)\leqslant K+\rho(x,y)
	\end{equation*}
	即数列$\{\rho(x_n,y)\}$有界。
\end{proof}
\begin{corollary}
	$\{x_n\}$是度量空间$(X,\rho)$中的一个收敛点列。若将其看作点集,则$\{x_n\}$是有界点集。
\end{corollary}
\begin{proof}
	任取$y\in X$,数列$\{\rho(x_n,y)\}$有界,即$\exists\;\delta>0,\;\forall\;n\in\mathbb{N}^+,\;\rho(x_n,y)<\delta$,则$\{x_n\}\subset U(y,\delta)$。
\end{proof}
\subsubsection{子列的收敛性}
\begin{definition}
	$\{x_n\}$是度量空间$(X,\rho)$中的一个点列,而
	\begin{equation*}
		n_1<n_2<\cdots<n_k<n_{k+1}<\cdots
	\end{equation*}
	是一串严格递增的自然数,则
	\begin{equation*}
		x_{n_1}<x_{n_2}<\cdots<x_{n_k}<x_{n_{k+1}}<\cdots
	\end{equation*}
	也形成一个$(X,\rho)$中的点列,我们把$\{x_{n_k}\}$称之为点列$\{x_n\}$的一个\gls{subsequence}。
\end{definition}
\begin{theorem}
	$\{x_n\}$是度量空间$(X,\rho)$中的一个收敛点列,则它的任何子列$\{x_{n_k}\}$都收敛,并且与$\{x_n\}$有同样的极限。反之,若$\{x_n\}$的任何子列都收敛,则$\{x_n\}$本身也收敛。
\end{theorem}
\begin{proof}
	(1)设$\{x_n\}$的极限为$x$,则对任意的$\varepsilon>0$,$\exists\; N\in\mathbb{N}^+$,当$n>N$时有
	\begin{equation*}
		\rho(x_n,x)<\varepsilon
	\end{equation*}
	当$k>N$时就有$n_k\geqslant k>N$,也就有
	\begin{equation*}
		\rho(x_{n_k},x)<\varepsilon
	\end{equation*}\par
	(2)$\{x_n\}$也是它自己的一个子列。
\end{proof}
\section{度量空间上的映射}
\subsection{一般映射的定义与性质}
\begin{definition}
	设$X,Y$是任意给定的集合。如果对于任意的$x\in X$,都存在唯一地$f(x)\in Y$与之对应,则称对应关系$f$是一个从$X$到$Y$的\gls{mapping}。对任何$E\subset Y$,称:
	\begin{equation*}
		f^{-1}(E)=\{x:f(x)\in E\}
	\end{equation*}
	为集合$B$在映射$f$下的\gls{preimage}\footnote{原像与逆无关。}。
\end{definition}
\begin{theorem}\label{theo:PropertyOfPreimage}
	设$X,Y$是任意给定的集合,$f$是一个从$X$到$Y$的映射。集合的原像有下列性质:
	\begin{enumerate}
		\item $f^{-1}(\varnothing)=\varnothing,\;f^{-1}(Y)=X$;
		\item 若$E_1\subset E_2\subset Y$,则$f^{-1}(E_1)\subset f^{-1}(E_2)$;
		\item 对任意的$E\subset Y$,$[f^{-1}(E)]^c=f^{-1}(E^c)$;
		\item 设$T$是一个指标集,对$\{A_t\in\ Y:t\in T\}$,有:
		\begin{equation*}
			f^{-1}\left(\underset{t\in T}{\bigcup}A_t\right)=\underset{t\in T}{\bigcup}f^{-1}(A_t), \quad
			f^{-1}\left(\underset{t\in T}{\bigcap}A_t\right)=\underset{t\in T}{\bigcap}f^{-1}(A_t)
		\end{equation*}
	\end{enumerate}
\end{theorem}
\begin{proof}
	(1)(2)是显然的,下证(3)(4)。\par
	(3)对任意的$x\in[f^{-1}(E)]^c$,有$x\notin f^{-1}(E)$,即$f(x)\notin E$,所以$x\in f^{-1}(E^c)$。由$x$的任意性,$[f^{-1}(E)]^c\subset f^{-1}(E^c)$。对任意的$x\in f^{-1}(E^c)$,有$f(x)\in E^c$,所以$x\notin f^{-1}(E)$,于是$x\in[f^{-1}(E)]^c$。由$x$的任意性,$f^{-1}(E^c)\subset[f^{-1}(E)]^c$。综上,$[f^{-1}(E)]^c=f^{-1}(E^c)$。\par
	(4)对任意的$x\in f^{-1}\left(\underset{t\in T}{\cup}A_t\right)$,有$f(x)\in\underset{t\in T}{\cup}A_t$,即存在$t\in T$,使得$f(x)\in A_t,\;x\in f^{-1}(A_t)$,于是$x\in\underset{t\in T}{\cup}f^{-1}(A_t)$。由$x$的任意性,$f^{-1}\left(\underset{t\in T}{\cup}A_t\right)\subset\underset{t\in T}{\cup}f^{-1}(A_t)$。对任意的$x\in\underset{t\in T}{\cup}f^{-1}(A_t)$,则存在$t\in T$,使得$x\in f^{-1}(A_t)$,于是$x\in f^{-1}\left(\underset{t\in T}{\cup}A_t\right)$。由$x$的任意性,$\underset{t\in T}{\cup}f^{-1}(A_t)\subset f^{-1}\left(\underset{t\in T}{\cup}A_t\right)$。综上,$f^{-1}\left(\underset{t\in T}{\cup}A_t\right)=\underset{t\in T}{\cup}f^{-1}(A_t)$。交的情形同理可证。
\end{proof}
\subsection{度量空间上的映射}
\begin{definition}
	$(X,\rho_X)$和$(Y,\rho_Y)$都是度量空间。若对任意的$ x\in X$,都$\exists\;y\in Y$与之对应,则称这个对应是一个$X$到$Y$的映射,用符号$T$表示。称集合
	\begin{equation*}
		\{x\in X:Tx\in E\subset Y\}
	\end{equation*}
	为集合$E$在映射$T$下的原像。
\end{definition}
\subsubsection{单射、满射、双射}
\begin{definition}
	$(X,\rho_X)$和$(Y,\rho_Y)$都是度量空间,$T$是一个$X$到$Y$的映射。如果对任意的$y\in Tx$,只有唯一的$x$使得$Tx=y$,那么称映射$T$为\gls{InjectiveF}。
\end{definition}
\begin{definition}
	$(X,\rho_X)$和$(Y,\rho_Y)$都是度量空间,$T$是一个$X$到$Y$的映射。如果对任意的$y\in Y$,都存在$X$中的$x$使得$Tx=y$,那么称映射$T$为\gls{SurjectiveF}。
\end{definition}
\begin{definition}
	$(X,\rho_X)$和$(Y,\rho_Y)$都是度量空间,$T$是一个$X$到$Y$的映射。如果$T$既是单射,又是满射,则称之为\gls{BijectiveF}。
\end{definition}
\subsubsection{逆映射}
\begin{definition}
	$(X,\rho_X)$和$(Y,\rho_Y)$都是度量空间,$T$是一个$X$到$Y$的双射。则对任意的$y\in Y$,都存在唯一的$x\in X$使得$Tx=y$,此时可以得到一个新的映射,它将$Y$映成$X$,称这个映射为$T$的\gls{InverseMap}。
\end{definition}
\begin{theorem}
	$(X,\rho_X)$和$(Y,\rho_Y)$都是度量空间,$T$是一个$X$到$Y$的映射。$T$存在逆映射的充要条件是$T$是一个双射。
\end{theorem}
\subsubsection{复合映射}
\begin{definition}
	$(X,\rho_X)$、$(Y,\rho_Y)$和$(Z,\rho_z)$都是度量空间,$T_1$是一个$X$到$Y$的映射,$T_2$是一个$Y$到$Z$的映射。定义映射$T_3$满足$T_3x=T_2(T_1x)$,其中$x\in X$,则称映射$T_3$是映射$T_1$和映射$T_2$的\gls{CompositeMap}。
\end{definition}
复合映射的概念也可以推广到多个映射的复合。
\subsubsection{等距映射}
\begin{definition}
	$(X,\rho_X)$和$(Y,\rho_Y)$都是度量空间,$T$是一个$X$到$Y$的双射。若对任意的$x,y\in X$,有$\rho_X(x,y)=\rho_Y(Tx,Ty)$,则称$T$是$X$到$Y$上的\gls{isometry}。如果存在一个$X$到$Y$上的等距映射,则称$X$和$Y$\gls{isometric}。
\end{definition}
\subsection{映射的连续性}
\begin{definition}
	$(X,\rho_X)$和$(Y,\rho_Y)$都是度量空间,$T$是一个$X$到$Y$的映射,$x_0\in X$。若对任意的$\varepsilon>0$,$\exists\;\delta>0$,使得对$X$中一切满足条件$\rho(x_0,x)<\delta$的$x$,都有$\rho_Y(Tx_0,Tx)<\varepsilon$,则称$T$在$x_0$处是\gls{continuous}。
\end{definition}
\subsubsection{邻域式定义}
\begin{definition}
	$T$在$x_0$处连续:对$Tx_0$的任意$\varepsilon$邻域$U$,必有$x_0$的某个$\delta$邻域$V$使得$TV\subset U$。
\end{definition}
\subsubsection{点列式定义}
\begin{theorem}
	$(X,\rho_X)$和$(Y,\rho_Y)$都是度量空间,$T$是一个$X$到$Y$的映射,那么$T$在$x_0\in X$处连续的充分必要条件为:当$x_n\to x_0$时,$Tx_n\to Tx_0$。
\end{theorem}
\begin{proof}
	必要性显然。\par
	充分性:若此时$T$在$x_0$不连续,那么$\exists\;\varepsilon>0$,$\forall\;\delta>0$,都存在满足条件$\rho_X(x_0,x)<\delta$的点$x$使得$\rho_Y(Tx_0,Tx)\geqslant\varepsilon$,因此可取一个点列$\{x_n\}$,满足$\rho_X(x_0,x_n)<\frac{1}{n}$。注意到此时满足$x_n\to x$但不满足$Tx_n\to Tx_0$,矛盾。
\end{proof}
\subsubsection{连续映射的定义}
\begin{definition}
	$T$是一个$(X,\rho_X)$到$(Y,\rho_Y)$的映射,若$T$在$X$的每一点都连续,则称$T$是$X$上的\gls{ContinuousMap}。
\end{definition}
\subsubsection{连续映射的拓扑式等价定义}
\begin{theorem}
	度量空间$(X,\rho_X)$到$(Y,\rho_Y)$上的映射$T$是$X$上连续映射的充要条件为:
	\begin{enumerate}
		\item $Y$中任意开集$E$的原像$T^{-1}E$是$X$中的开集。
		\item $Y$中任意闭集$E$的原像$T^{-1}E$是$X$中的闭集。
	\end{enumerate}
\end{theorem}
\begin{proof}
	(1)必要性:设$T$是连续映射,$M\subset Y$是一个开集,如果$T^{-1}M=\varnothing$,则$T^{-1}M$是开集;若$T^{-1}M\ne\varnothing$,任取$x\in T^{-1}M$,令$Tx=y$,则$y\in M$,因为$M$是开集,所以存在$y$的$\varepsilon$邻域$U$,使得$U\subset M$,由$T$的连续性,存在$x$的$\delta$邻域$V$,使得$TV\subset U$,因此$V\subset T^{-1}U\subset T^{-1}M$,即$x$是$T^{-1}M$的内点。由$x$的任意性,$T^{-1}M$是$X$中的开集。\par
	充分性:对任意的$x\in X$及$Tx$的任意$\varepsilon$邻域$U$,由邻域的性质,$U$是一个开集,因此$T^{-1}U$是$X$中的开集。因为$x$是$T^{-1}U$的内点,所以存在$x$的某个$\delta$邻域$V$,使得$V\subset T^{-1}U$,因此$TV\subset U$,即$T$在$x$处连续。由$x$的任意性,$T$是$X$上的连续映射。\par
	(2)由\cref{theo:PropertyOfPreimage}(3),利用(1)易证(2)。
\end{proof}
\subsubsection{同胚映射}
\begin{definition}
	$(X,\rho_X)$和$(Y,\rho_Y)$都是度量空间,$T$是一个$X$到$Y$的双射。若$T$和$T^{-1}$都是连续映射,则称$T$是$X$到$Y$上的\gls{HomeoMap}。如果存在一个$X$到$Y$上的同胚映射,则称$X$和$Y$\gls{homeomorphic}。
\end{definition}
\section{完备的度量空间}
\subsection{Cauchy点列}
\begin{definition}
	$(X,\rho)$是一个度量空间,$\{x_n\}$是$X$中的点列。若对任意的$\varepsilon>0$,$\exists\;N\in\mathbb{N}^+$,当$n,m>N$时,有:
	\begin{equation*}
		\rho(x_n,x_m)<\varepsilon
	\end{equation*}
	则称点列$\{x_n\}$是一个\gls{CauchySeq}或\gls{FoundamentalSeq}。
\end{definition}
\subsubsection{Cauchy点列的性质}
\begin{theorem}
	$(X,\rho_X)$是一个度量空间,$\{x_n\}$是$X$中的收敛点列,则$\{x_n\}$是一个Cauchy点列。
\end{theorem}
\begin{proof}
	令$n<m$。因为$\{x_n\}$是$X$中的收敛点列,假设其极限为$x$,则对任意的$\varepsilon>0$,$\exists\; N_1\in\mathbb{N}^+$,当$n>N_1$时有$\rho(x_n,x)<\frac{\varepsilon}{2}$;$\exists\; N_2\in\mathbb{N}^+$,当$m>N_2$时有$\rho(x_m,x)<\frac{\varepsilon}{2}$。取$N=max\{N_1,N_2\}$,则当$n>N$时,有
	\begin{equation*}
		\rho(x_n,x_m)\leqslant\rho(x_n,x)+\rho(x_m,x)<\frac{\varepsilon}{2}+\frac{\varepsilon}{2}=\varepsilon
	\end{equation*}
	即点列$\{x_n\}$是一个Cauchy点列。
\end{proof}
该定理的逆命题不正确,考虑有理数集按绝对值距离构成的度量空间中的Cauchy点列。
\begin{theorem}
	$(X,\rho)$是一个度量空间,$\{x_n\}$是$X$中的Cauchy点列。若它的一个子列$\{x_{n_k}\}$收敛,则其本身也收敛,并且极限相同。
\end{theorem}
\begin{proof}
	设$\{x_{n_k}\}$极限为$x$,则
	\begin{equation*}
		\rho(x_n,x)\leqslant\rho(x_n,x_{n_k})+\rho(x_{n_k},x)
	\end{equation*}
	对任意的$\varepsilon>0$,因为$x_{n_k}$收敛于$x$,所以$\exists\;N_1\in\mathbb{N}^+$,使得当$k>N_1$时,有$\rho(x_{n_k},x)<\frac{\varepsilon}{2}$。又因$\{x_n\}$是$X$中的Cauchy点列,因此$\exists\;N_2\in\mathbb{N}^+$,使得当$n,k>N_2$时,有$\rho(x_n,x_{n_k})<\frac{\varepsilon}{2}$。取$N=max\{N_1,N_2\}$,当$n,k>N$时,即有
	\begin{equation*}
		\rho(x_n,x)\leqslant\rho(x_n,x_{n_k})+\rho(x_{n_k},x)<\frac{\varepsilon}{2}+\frac{\varepsilon}{2}=\varepsilon\qedhere
	\end{equation*}
\end{proof}
\begin{theorem}
	度量空间中的任何Cauchy点列都是有界的。
\end{theorem}
\begin{proof}
	设$(X,\rho)$是一个度量空间,$\{x_n\}$是$X$中的Cauchy点列,则对任意的$\varepsilon>0$,$\exists\;N\in\mathbb{N}^+$,当$n,m>N$时,有:
	\begin{equation*}
		\rho(x_n,x_m)<\varepsilon
	\end{equation*}
	取$m=N+1$,令$\alpha=\max\limits_{i=1,2,\dots,N}\rho(x_m,x_i)$,$\delta=max\{\varepsilon,\alpha\}$,则$\{x_n\}$中的所有点都在闭邻域$\overline{U}(x_m,\delta)$中。由\cref{def:BoundedSet2},$\{x_n\}$有界。
\end{proof}

\subsection{完备度量空间的定义}
\begin{definition}
	$(X,\rho)$是一个度量空间。若$X$中的任意Cauchy点列$\{x_n\}$都收敛到$X$中的某一点,则称$X$是一个\gls{complete}度量空间。
\end{definition}
\subsubsection{完备度量空间的等价定义}
下列定理是Cantor's Intersection Theorem的一个变体:
\begin{theorem}[闭球套定理]
	$(X,\rho)$是一个度量空间。$X$完备的充分必要条件为对任何满足下列条件的一列闭邻域$\{E_n=\overline{U}(x_n,\delta_n)\}$:
	\begin{enumerate}
		\item $E_1\supset E_2\supset\cdots\supset E_n\supset\cdots$
		\item $\{\delta_n\}\to0$
	\end{enumerate}
	$X$中都存在唯一的$x$满足$x\in\underset{n\in\mathbb{N}^+}{\cap}E_n$。
\end{theorem}
\begin{proof}
	充分性:任取$X$中的一个Cauchy点列$\{x_n\}$。因为$\{x_n\}$是一个Cauchy点列,所以对任意的$\varepsilon>0,\;\exists\;N\in\mathbb{N}^+,\;\forall\;n,m>N,\;\rho(x_m,x_n)<\varepsilon$。取$\{N_k\}$使得$N_k$是使得$\rho(x_m,x_n)<\dfrac{1}{2^k}$的临界条件,取$x_{n_k}>N_k,\;x_{n_k+1}>N_{k+1}$,即可产生一个子列 $\{x_{n_k}\}$,满足:
	\begin{equation*}
		\rho(x_{n_k},x_{n_{k+1}})<\frac{1}{2^k}
	\end{equation*}
	取闭邻域列:
	\begin{equation*}
		\left\{E_k=\bar{U}\left(x_{n_k},\frac{1}{2^{k-1}}\right)\right\}
	\end{equation*}
	显然:
	\begin{equation*}
		\forall\;y\in E_{k+1},\;\rho(y,x_{n_k})\leqslant\rho(y,x_{n_{k+1}})+\rho(x_{n_{k+1}},x_{n_k})<\frac{1}{2^k}+\frac{1}{2^k}=\frac{1}{2^{k-1}}
	\end{equation*}
	所以$E_k\supset E_{k+1}$。由题目条件,此时存在唯一的$x\in X$满足$x\in E_k,\;\forall\;k\in\mathbb{N}^+$。下证$\{x_n\}\to x$。
	\begin{equation*}
		\rho(x_n,x)\leqslant\rho(x_n,x_{n_k})+\rho(x_{n_k},x)\leqslant\rho(x_n,x_{n_k})+\frac{1}{2^{k-1}}
	\end{equation*}
	因为$\{x_n\}$是一个Cauchy点列,因此当$n$和$k$足够大时,上式右端两项均趋于$0$。因此$\{x_n\}\to x$。由$\{x_n\}$的任意性,$X$是一个完备的度量空间。\par
	必要性中的存在性:在每个$E_n$中取一点$y_n$构成点列$\{y_n\}$。设$m>n$,因为$E_m\subset E_n$,所以$y_m\in E_n$,于是有:
	\begin{equation*}
		\rho(y_n,y_m)\leqslant\delta_n\to0
	\end{equation*}
	因此$\{y_n\}$是一个Cauchy点列。因为$X$完备,所以$\{y_n\}\to x\in X$。对任意的$n_0\in\mathbb{N}^+$,当$n>n_0$时,$y_n\in E_n\subset E_{n_0}$。又因为闭邻域$E_{n_0}$是闭集,因此$x\in E_{n_0}$。由$n_0$的任意性,$x\in E_n,\;\forall\;n\in\mathbb{N}^+$。\par
	必要性中的唯一性:若还有一点$y$满足上述条件,则:
	\begin{equation*}
		\rho(x,y)\leqslant\rho(x,y_n)+\rho(y_n,y)\to0
	\end{equation*}
	唯一性显然得证。
\end{proof}

\subsection{度量空间的完备化定理}
\begin{theorem}
	$(X,\rho_X)$是一个度量空间,则一定存在一个完备度量空间$(Y,\rho_Y)$,使得$X$与$Y$的一个稠密子空间等距同构,并且$Y$在等距同构的意义下是唯一的\footnote{这里的唯一性是指,如果存在另一个完备度量空间$(Z,\rho_Z)$使得$X$与$Z$的一个稠密子空间等距同构,则$Y$与$Z$等距同构。}。
\end{theorem}
证明太复杂,不提供。




\section{完备度量空间的性质}
\subsection{子空间的完备性}
\begin{theorem}
	完备度量空间$(X,\rho_X)$的子空间$M$是完备度量空间的充要条件为$M$是$X$中的一个闭子空间。
\end{theorem}
\begin{proof}
	必要性:因为$M$是完备子空间,则对任意的$x\in M'$,存在$M$中的一个收敛点列$\{x_n\}\to x$。因为收敛点列也是Cauchy点列,而此时Cauchy点列在$M$中收敛,所以$x\in M$。由$x$的任意性,$M'\subset M$,故$M$是一个闭集,即$M$是$X$的一个闭子空间。\par
	充分性:任取$\{x_n\}$为$M$中的一个Cauchy点列,那么它也是$X$中的Cauchy点列,因此$\exists\;x\in X$使得$\{x_n\}\to x$,即$x$是$M$的一个聚点。又因$M$是$X$的一个闭子空间,所以$x\in M$,即Cauchy点列$\{x_n\}$收敛于$M$中的一点。由$\{x_n\}$的任意性,$M$是完备的度量空间。
\end{proof}
\subsection{第一型集与第二型集}
\begin{definition}
	设$A$是度量空间$(X,\rho)$的子集。若$A$可表示为至多可列个稀疏集的并,则称$A$是\gls{FirstSet}。反之则为\gls{SecondSet}。
\end{definition}
\begin{theorem}
	任何完备的度量空间都是第二型集。
\end{theorem}
\begin{proof}
	假设不成立,即存在完备的度量空间$X$使得$X$是第一型集。也就是说$X=\underset{i=1}{\overset{+\infty}{\cup}}F_i$,其中$F_i,\;\forall\;i\in\mathbb{N}^+$是稀疏集。因为$F_1$是稀疏集,由稠密性等价定义(2),$\exists\;x_1\in X$,使得闭邻域$U(x_1,r_1)$中不含$F_1$中的点。对于闭邻域$U(x_1,r_1)$,由于$F_2$是稀疏集,因此$\exists\;x_2\in U(x_1,r_1)$,使得闭邻域$U(x_2,r_2)$中不含$F_2$中的点。如此重复下去,实际上可以取$r_n\in(0,\frac{1}{n})$(对于某个固定的半径,在这个半径内交集为空,那么在更小的半径内交集也为空),便可以得到一个闭球套:
	\begin{equation*}
		U(x_1,r_1)\supset U(x_2,r_2)\supset\cdots\supset U(x_n,r_n)\supset\cdots
	\end{equation*}
	且$\{r_n\}\to0$。由完备度量空间的闭球套定理,存在一个点$x\in U(x_n,r_n),\;\forall\;n\in\mathbb{N}^+$。而由闭球套的取法,$x\notin X$,矛盾。
\end{proof}
\subsection{准紧性与全有界性}
\subsubsection{准紧性的定义}
\begin{definition}
	$(X,\rho)$是一个度量空间,$A$是$X$的一个子集。如果$A$的每个点列都有一个收敛子列收敛于$X$中的某一点,则称$A$是\gls{PrecompactSet}。
\end{definition}
\subsubsection{准紧集的性质}
\begin{property}
	准紧集的子集也是准紧集。
\end{property}
\subsubsection{全有界集的定义}
\begin{definition}
	$(X,\rho)$是一个度量空间,$A$和$B$都是$X$的子集,$\varepsilon$是一个给定的正数。如果对任意的$x\in A$,都$\exists\;y\in B$,使得$\rho(x,y)<\varepsilon$,则称$B$是$A$的一个$\varepsilon$-网。即:以$B$中的点为中心,$\varepsilon$为半径的所有开邻域的并包含了$A$。
\end{definition}
\begin{definition}
	$(X,\rho)$是一个度量空间,$A$是$X$的子集。如果对任意的$\varepsilon>0$,$X$中总存在$A$的$\varepsilon$-网,且该$\varepsilon$-网只有有限个点,则称$A$是\gls{TotallyBoundedSet}。
\end{definition}
\subsubsection{全有界集的性质}
\begin{property}
	全有界集具有如下性质:\par
	(1)任何有限集都是全有界集。\par
	(2)全有界集的子集也是全有界集。\par
	(3)设$A$是一个全有界集,则对任意的$\varepsilon>0$,总存在$A$的一个有限子集成为$A$的一个$\varepsilon$-网。\par
	(4)全有界集有界且可分。
\end{property}
\begin{proof}
	(1)(2)是显然的。\par
	(3)因为$A$是一个全有界集,所以对任意的$\varepsilon>0$,存在一个$A$的$\frac{\varepsilon}{2}$-网$\{x_1,x_2,\dots,x_n\}$。依次取$a_i\in A$使得$a_i\in U(x_i,\frac{\varepsilon}{2}),\;i=1,2,\dots,n$,则$\{\seq{a}{n}\}$即构成$A$的一个$\varepsilon$-网:
	\begin{equation*}
		\forall\;a\in U\left(x_i,\frac{\varepsilon}{2}\right),\;\rho(a,a_i)\leqslant\rho(a,x_i)+\rho(x_i,a_i)<\frac{\varepsilon}{2}+\frac{\varepsilon}{2}=\varepsilon,\;i=1,2,\dots,n
	\end{equation*}
	\hspace{2em}(4)设$(X,\rho)$是给定的度量空间,$A$是一个全有界集。由全有界集定义,$A$应有一个$1$-网$\{x_1,x_2,\dots,x_n\}$。则:
	\begin{equation*}
		\forall\;a\in A,\;\exists\;x_{k}\in\{x_1,x_2,\dots,x_n\},\;\rho(a,x_k)<1
	\end{equation*}
	故(下式中$x_k$是与点$a$对应的$1$-网中的点):
	\begin{equation*}
		\forall\;a\in A,\;\rho(a,x_1)\leqslant\rho(a,x_k)+\rho(x_k,x_1)<1+\max_{k=1,2,\dots,n}\rho(x_k,x_1)
	\end{equation*}
	记$\max\limits_{k=1,2,\dots,n}\rho(x_k,x_1)=K$,则$\forall\;a\in A,\;a\in U(x_1,1+K)$,所以$A$是有界的。\par
	因为$A$是一个全有界集,所以对任意的$\varepsilon=\frac{1}{n},\;n\in\mathbb{N}^+$,$X$中都存在$A$的一个只含有限个点的$\varepsilon$-网$B_n$。记:
	\begin{equation*}
		B=\underset{n=1}{\overset{+\infty}{\cup}}B_n
	\end{equation*}
	显然$B$是可列的。对任意的$x\in A,\;\exists\;x_n\in B_n\subset B,\;\rho(x,x_n)<\frac{1}{n},\;\forall\;n\in\mathbb{N}^+$。因此点列$\{x_n\}$收敛于$x$。由$x$的任意性和稠密性的等价命题$3$,$B$在$A$中稠密。综上,$A$是可分的。
\end{proof}
\subsubsection{准紧性与全有界性的关系}
以下定理说明了完备度量空间中准紧性与全有界性的关系,即二者在完备度量空间中是等价的:
\begin{theorem}
	$(X,\rho)$是一个度量空间。\par
	(1)如果$A\subset X$准紧,则$A$全有界。\par
	(2)如果$X$是完备的,则当$A$全有界时,$A$也必定准紧。\par
\end{theorem}
\begin{proof}
	(1)如果$A$不是全有界集,则$\exists\;\varepsilon>0$,使得$A$没有只有有限点的$\varepsilon$-网。任取$x_1\in A$,则$\exists\;x_2\in A$使得$\rho(x_1,x_2)\geqslant\varepsilon$,否则$\{x_1\}$就是$A$的一个$\varepsilon$-网。同理,$\exists\;x_3\in A$使得$\rho(x_3,x_j)\geqslant\varepsilon,\;j=1,2$,否则$\{x_1,x_2\}$就是$A$的一个$\varepsilon$-网。重复这一步骤就得到点列$\{x_n\}$,当$m\ne n$时,$\rho(x_m,x_n)\geqslant\varepsilon$。由柯西收敛准则$\{x_n\}$显然没有收敛的子列,这与$A$的准紧性矛盾。\par
	(2)任取$A$中的点列$\{x_n\}$。如果$\{x_n\}$中只有有限个互不相同的元素,则$\{x_n\}$显然有收敛的子列。如果$\{x_n\}$中有无限个互不相同的元素,记这些元素构成的集合为$B_0$。由全有界集性质(2),$B_0$也是全有界集。由全有界集性质(3),$B_0$中存在有限个元素使得以它们为球心,$\frac{1}{2}$为半径的开邻域的并包含$B_0$,显然$B_0$中至少存在一个点$y_1$使得以它为半径,$\frac{1}{2}$为半径的开邻域包含了无穷多个$A$中的点。记被$y_1$包含的这无穷多个点构成的集合为$B_1$,显然$B_1$的直径小于$1$。因为$B_1$是$B_0$的子集,所以$B_1$也是全有界的,重复上述论证,则存在$B_2\subset B_1$,使得$B_2$中含有$B_1$无穷多个元素且$B_2$的直径小于$\frac{1}{2}$。依次类推,可以得到一系列集合满足如下条件:
	\begin{enumerate}
		\item $B_1\supset B_2\supset\cdots\supset B_n\supset\cdots$。
		\item $B_n$的直径小于$\frac{1}{2^{n-1}}$。
		\item 每个$B_n$中都含有$\{x_n\}$中无限个元素。
	\end{enumerate}
	取$x_{n_k}\in B_k,\;n_{k+1}>n_k,\;k\in\mathbb{N}^+$,便得到$\{x_n\}$的一个子列$\{x_{n_k}\}$。显然$\{x_{n_k}\}$是一个Cauchy点列:
	\begin{equation*}
		\forall\;p>q,\;x_{n_p}\in B_p\subset B_q,\;\rho(x_{n_p},x_{n_q})<\frac{1}{2^{q-1}}\to0
	\end{equation*}	
	因为$X$完备,所以$\{x_{n_k}\}$在$X$中收敛。由$\{x_n\}$的任意性,$A$准紧。
\end{proof}
\begin{corollary}
	度量空间中的准紧集是有界且可分的。
\end{corollary}
\begin{theorem}
	$(X,\rho)$是完备的度量空间,$A\subset X$。$A$为准紧集的充分必要条件是对任意的$\varepsilon>0$,$A$都有准紧的$\varepsilon$-网。
\end{theorem}
\begin{proof}
	(1)必要性:$A$就是它自身的准紧的$\varepsilon$-网。\par
	(2)充分性:若对任意的$\varepsilon>0$,$A$都有准紧的$\varepsilon$-网$B$。因为$B$准紧,同时$X$完备,所以$B$全有界,即$B$有只有有限个元素的$\varepsilon$-网$C=\{c_1,c_2,\dots,c_n\}$。则:
	\begin{equation*}
		\forall\;a\in A,\;\rho(a,c_i)\leqslant\rho(a,b)+\rho(b,c_i),\;\forall\;i=1,2,\dots,n
	\end{equation*}
	可以选取$c_i$和$b\in B$使得$\rho(a,b)<\varepsilon,\;\rho(b,c_i)<\varepsilon$(先选择$b$,再根据$b$即可选得$c_i$)。也就是说,$\forall\;a\in A,\;\exists\;c_i\in\{c_1,c_2,\dots,c_n\},\;\rho(a,c_i)<2\varepsilon$。以$\{c_1,c_2,\dots,c_n\}$中的点为中心,$2\varepsilon$为半径构成的$2\varepsilon$-网必然是$A$的一个$2\varepsilon$-网。由$\varepsilon$的任意性,$A$全有界。又因为$X$是完备的,所以$A$准紧。
\end{proof}

\subsection{压缩映射原理}
\subsubsection{不动点的定义}
\begin{definition}
	若点$\varphi$在映射$T$的作用下满足$T\varphi=\varphi$,则称$\varphi$是映射$T$的一个\gls{FixedP}。
\end{definition}
\subsubsection{压缩映射的定义}
\begin{definition}
	$(X,\rho)$是一个度量空间,$T$是$X$到$X$的一个映射,如果存在一个数$\alpha$,$0\leqslant\alpha<1$,使得对任意的$x,y\in X$,有:
	\begin{equation*}
		\rho(Tx,Ty)\leqslant\alpha\rho(x,y)
	\end{equation*}
	则称$T$是一个\gls{ContractionMap}。
\end{definition}
\subsubsection{压缩映射原理}
\begin{theorem}
	$(X,\rho)$是一个完备的度量空间,$T$是$X$到$X$的一个压缩映射,那么$T$有且只有一个不动点。
\end{theorem}
\begin{proof}
	(1)存在性:任取$x_0\in X$,令$x_n=T^nx_0$,由此产生一个点列$\{x_n\}$。下面我们来证明这个点列是一个Cauchy点列,它的极限就是一个不动点。\par
	\begin{align*}
		\rho(x_{m+1},x_m)&=\rho(Tx_m,Tx_{m-1})\leqslant\alpha\rho(x_m,x_{m+1}) \\
		&=\cdots \\
		&=\alpha^{m-1}\rho(Tx_1,Tx_0)\leqslant\alpha^m\rho(x_1,x_0)
	\end{align*}
	取$n>m$,由距离的三角不等式:
	\begin{align*}
		\rho(x_m,x_n)
		&\leqslant\rho(x_m,x_{m+1})+\cdots+\rho(x_{n-1},x_n) \\
		&\leqslant(\alpha^m+\alpha^{m+1}+\cdots+\alpha^{n-1})\rho(x_0,x_1) \\
		&=\alpha^m\frac{1-\alpha^{n-m}}{1-\alpha}\rho(x_0,x_1) \\
		&<\frac{\alpha^m}{1-\alpha}\rho(x_0,x_1)
	\end{align*}
	因为$0\leqslant\alpha<1$,所以当$m$足够大的时候,$\rho(x_m,x_n)\rightarrow 0$,即$\{x_n\}$是$X$中的Cauchy点列。又因为$X$完备,所以$\{x_n\}\rightarrow x\in X$。由三角不等式:
	\begin{equation*}
		\rho(x,Tx)\leqslant\rho(x,x_m)+\rho(x_m,Tx)\leqslant\rho(x,x_m)+\alpha\rho(x_{m-1},x)
	\end{equation*}
	当$m\to+\infty$时上式右端趋于0,因此$\rho(x,Tx)=0$,即$Tx=x$,$T$存在一个不动点。\par
	(2)唯一性:假设$T$还有一个不动点$y$,则
	\begin{equation*}
		\rho(x,y)=\rho(Tx,Ty)\leqslant\alpha\rho(x,y)
	\end{equation*}
	因为$0\leqslant\alpha<1$,所以$\rho(x,y)=0$,即$x=y$,唯一性得证。
\end{proof}
压缩映射原理有一个推广:
\begin{theorem}
	$T$是完备度量空间$X$到自身的映射,如果存在常数$\alpha$及自然数$n_0$,$0\leqslant\alpha<1$,使得对任意$x,y\in X$,有:
	\begin{equation*}
		\rho(T^{n_0}x,T^{n_0}y)\leqslant\alpha\rho(x,y)
	\end{equation*}
	那么$T$在$X$中有且只有一个不动点。
\end{theorem}
\begin{proof}
	存在性:$T^{n_0}$满足压缩映射原理的条件,因此$T^{n_0}$有且只有一个不动点$x_0$。下证$x_0$也是$T$在$X$中唯一的不动点。因为
	\begin{equation*}
		T^{n_0}(Tx_0)=T^{n_0+1}x_0=T(T^{n_0}x_0)=Tx_0
	\end{equation*}
	所以$Tx_0$是$T^{n_0}$的一个不动点,由不动点的唯一性,$Tx_0=x_0$,所以$x_0$是$T$的一个不动点。\par
	唯一性:若$T$存在另一个不动点$x_1$,则
	\begin{equation*}
		T^{n_0}x_1=T^{n_0-1}Tx_1=T^{n_0-1}x_1=\cdots=Tx_1=x_1
	\end{equation*}
	即$x_1$也是$T^{n_0}$的一个不动点,由$T^{n_0}$不动点的唯一性,$x_0=x_1$。
\end{proof}

\section{紧集与紧度量空间}
\begin{definition}
	$(X,\rho)$是一个度量空间,$A$是$X$的一个子集。如果$A$的每个点列都有一个收敛子列收敛于$A$中的某一点,则称$A$是\gls{CompactSet}。若$X$是紧集,则称$X$是\gls{CompactMetricSpace}。
\end{definition}

\subsection{紧集的性质}
\begin{theorem}
	度量空间中的紧集是有界且可分的。
\end{theorem}
\begin{proof}
	因为准紧集是有界可分的,且紧集必然是准紧集,所以紧集也是有界可分的。
\end{proof}

\subsection{紧度量空间的性质}
\subsubsection{完备性}
\begin{theorem}
	任一度量空间中的紧集都是完备的。紧度量空间是完备度量空间。
\end{theorem}
\begin{proof}
	设$X$是一个紧集,$\{x_n\}$是$X$中的一个Cauchy点列。由紧集定义可得出$\{x_n\}$存在收敛的子列$\{x_{n_k}\}$,设$\{x_{n_k}\}\to a$,则:
	\begin{equation*}
		\rho(x_n,a)\leqslant\rho(x_n,x_{n_k})+\rho(x_{n_k},a)\to0
	\end{equation*}
	于是$\{x_n\}\to a$。由$\{x_n\}$的任意性,$X$是完备的。
\end{proof}
\begin{theorem}
	$(X,\rho)$是一个度量空间,$\{E_n\}$是$X$中的一列非空紧集,满足:
	\begin{equation*}
		E_1\supset E_2\supset\dots\supset E_n\supset\dots
	\end{equation*}
	则$\underset{n=1}{\overset{+\infty}{\cap}}E_n\ne\varnothing$。
\end{theorem}
\begin{proof}
	在每个$E_n$中选择一点$x_n$,构成序列$\{x_n\}$。因为$\forall\;n\in\mathbb{N}^+,\;x_n\in E_n\subset E_1$,又$E_1$是紧集,所以$\{x_n\}$存在子列$\{x_{n_k}\}\rightarrow x_0\in E_1$。对任意的$n\in\mathbb{N}^+$,当$n_k>n$时,有$x_{n_k}\in E_{n_k}\subset E_n$,又因收敛点列必为Cauchy点列、$E_n$完备,所以$x_0\in E_n$。由$n$的任意性,$x_0\in\underset{n=1}{\overset{+\infty}{\cap}}E_n$,所以$\underset{n=1}{\overset{+\infty}{\cap}}E_n\ne\varnothing$。
\end{proof}

\subsection{紧集的充要条件}
\subsubsection{有限覆盖}
\begin{definition}
	$(X,\rho)$是一个度量空间,$A$是$X$的子集,$\{G_i\}_{i\in I}$(其中$I$是一个指标集)是$X$中某些开集组成的集族。如果:
	\begin{equation*}
		A\subset\underset{i\in I}{\cup}G_i
	\end{equation*}
	则称$\{G_i\}_{i\in I}$为$A$的\gls{OpenCover}。如果$I$是有限集,则称$\{G_i\}_{i\in I}$为$A$的\gls{FiniteOpenCover}。
\end{definition}
\begin{theorem}[有限覆盖定理]
	度量空间$(X,\rho)$的子集$A$是紧集的充分必要条件是从$A$的任一开覆盖$\{G_i\}_{i\in I}$中必可选出一个有限子覆盖。
\end{theorem}
\begin{proof}
	(1)必要性:先来证明$\exists\;\varepsilon>0,\;\forall\;x\in A,\;\exists\;i\in I,\;U(x,\varepsilon)\in G_i$。若不成立,则:
	\begin{equation*}
		\forall\;\varepsilon>0,\;\exists\;x\in A,\;\forall\;i\in I,\;U(x,\varepsilon)\notin G_i
	\end{equation*}
	依次选择$\varepsilon_n=\frac{1}{2^n},\;n\in\mathbb{N}^+$,即可构造出序列$\{x_n\}$满足其中的每个元素都不在任何$G_i$中。因为$A$是紧集,所以$\{x_n\}$存在收敛于$A$中某点$x_0$的子列$\{x_{n_k}\}$。又因为$A\subset\underset{i\in I}{\cup}G_i$,所以$\exists\;i\in I,\;x_0\in G_i$。于是可取充分大的$k$使得$U\left(x_{n_k},\dfrac{1}{2^{n_k}}\right)\subset G_i$,这与$x_{n_k}$的取法矛盾。\par
	记使上述命题成立的$\varepsilon=\varepsilon_0$。因为$A$是紧集,则$A$也是全有界集,故能选择$A$中有限个点,使得对任意的$\varepsilon>0$,这有限个点的开$\varepsilon$邻域能够包含整个$A$。取$\varepsilon=\varepsilon_0$,此时只要对这有限个点的开邻域选择对应的$G_i$,即可选出有限子覆盖。\par
	(2)充分性:设$\{x_n\}$是$A$中的一个点列。如果$\{x_n\}$没有子列在$A$中收敛,则:
	\begin{equation*}
		\forall\;y\in A,\;\exists\;\delta_y>0,\;\exists\;n_y\in\mathbb{N}^+,\;\forall\;n>n_y,\;x_n\notin U(y,\delta_y)
	\end{equation*}
	显然$\{U(y,\delta_y):y\in A\}\supset A$,因此可以选择$A$中有限个点,分别记为$y_1,y_2,\dots,y_{n_0}$,使得$\{U(y_i,\delta_y):i=1,2,\dots,n_0\}\supset A$。则当$n\geqslant
	\max\{n_{y_1},\;n_{y_2},\dots,n_{y_{n_0}}\}$时,$x_n\notin A$,与$\{x_n\}\in A$矛盾。
\end{proof}
\subsubsection{有限交}
\begin{definition}
	$\mathscr{F}$是度量空间$(X,\rho)$中的一个集族。如果从$\mathscr{F}$中选择任意有限个集合,它们都有非空的交集,则称$\mathscr{F}$具有有限交性质。
\end{definition}
\begin{theorem}
	度量空间$(X,\rho)$的闭子集$A$是紧集的充分必要条件是$A$中具有有限交性质的闭子集族$\mathscr{F}$有非空的交。
\end{theorem}
\begin{proof}
	设$\mathscr{F}=\{F_i\}_{i\in I}$是一个具有有限交性质的闭子集族。\par
	(1)必要性:假设该闭子集族的交集为空集。令$G_i=X\backslash F_i$,则$\{G_i\}_{i\in I}$为开集族。由:
	\begin{equation*}
		\underset{i\in I}{\overset{}{\cup}}G_i=\underset{i\in I}{\overset{}{\cup}}(X\backslash F_i)=X\backslash \underset{i\in I}{\overset{}{\cap}}F_i=X
	\end{equation*}
	可知$\{G_i\}_{i\in I}\supset A$。因为$A$是紧集,所以可从$\{G_i\}_{i\in I}$中选择出$A$的有限子覆盖$\{G_{i_j}\}_{j=1}^n$。所以:
	\begin{equation*}
		\underset{j=1}{\overset{n}{\cap}}F_{j}=\underset{j=1}{\overset{n}{\cap}}(X\backslash G_{i_j})=X\backslash\underset{j=1}{\overset{n}{\cup}}G_{i_j}\subset X\backslash A
	\end{equation*}
	又因为$F_{i_j}\subset A$,所以:
	\begin{equation*}
		\underset{j=1}{\overset{n}{\cap}}F_{i_j}\subset A
	\end{equation*}
	综合上两式可得:
	\begin{equation*}
		\underset{j=1}{\overset{n}{\cap}}F_{i_j}\subset(X\backslash A)\cap A=\varnothing
	\end{equation*}
	这与$\mathscr{F}$具有有限交性质矛盾。\par
	(2)充分性:设闭集$A$中任一具有有限交性质的闭子集族具有非空的交。我们用有限覆盖定理来证明$A$是一个紧集。设$\{G_i\}_{i\in I}$为$A$的任一开覆盖,令$F_i=A\backslash G_i$,因为$A$是闭集,所以$F_i$也是闭集。因为:
	\begin{equation*}
		\underset{i\in I}{\overset{}{\cap}}F_i=\underset{i\in I}{\overset{}{\cap}}(A\backslash G_i)=A\backslash\underset{i\in I}{\overset{}{\cup}}G_i=\varnothing
	\end{equation*}
	所以,$A$的闭子集族$\{F_i\}_{i\in I}$不具有有限交性质,否则的话根据假设应有$\underset{i\in I}{\overset{}{\cap}}F_i\ne\varnothing$。于是存在有限子集族$\{F_{i_j}\}_{j=1}^n$使得:
	\begin{equation*}
		\underset{j=1}{\overset{n}{\cap}}F_{i_j}=\varnothing
	\end{equation*}
	它对应的开集族$\{F_{i_j}\}_{j=1}^n$则满足:
	\begin{equation*}
		\underset{j=1}{\overset{n}{\cup}}G_{i_j}=\underset{j=1}{\overset{n}{\cup}}(A\backslash F_{i_j})=A\backslash\underset{j=1}{\overset{n}{\cap}}F_{i_j}=A
	\end{equation*}
	即$\{G_{i_j}\}$是$A$的一个有限开覆盖,于是$A$是紧集。
\end{proof}

\subsection{紧集上的连续映射}
\begin{theorem}
	设$(X,\rho_X)$和$(Y,\rho_Y)$为度量空间,$A$是$X$中的紧集,$T$是$A$到$Y$上的连续映射,则$TA$是$Y$中的紧集。
\end{theorem}
\begin{proof}
	设$\{y_n\}$为$TA$中的一个点列,则有$X$中的点列$\{x_n\}$使得$y_n=Tx_n,\;n\in\mathbb{N}^+$。因为$A$是紧集,所以$\{x_n\}$存在子列$\{x_{n_k}\}\to x_0\in A$。因为$T$连续,所以:
	\begin{equation*}
		\lim_{k\to+\infty}y_{n_k}=\lim_{k\to+\infty}Tx_{n_k}=T\left(\lim_{k\to+\infty}x_{n_k}\right)=Tx_0\in TA
	\end{equation*}
	所以$TA$是紧集。
\end{proof}
\begin{definition}
	设$(X,\rho_X)$和$(Y,\rho_Y)$为度量空间,$T$是$X$到$Y$上的映射。若对于任意的$\varepsilon>0$,存在只与$\varepsilon$有关的$\delta>0$,使得对任意的$x,y\in X$,只要$\rho_X(x,y)<\delta$,就有$\rho_Y(Tx,Ty)<\varepsilon$,则称$T$在$X$上\gls{UnifContinuous}。
\end{definition}
\begin{corollary}
	设$(X,\rho_X)$和$(Y,\rho_Y)$为度量空间,$A$是$X$中的紧集,$T$是$A$到$Y$上的连续泛函,则:
	\begin{enumerate}
		\item $T$在$A$上有界;
		\item $T$在$A$上可达到其上、下确界;
		\item $T$在$A$上一致连续。
	\end{enumerate}
\end{corollary}
\begin{proof}
	(1)由于$TA$是紧集,而紧集是全有界集,全有界集有界,所以$T$在$A$上有界。\par
	(2)因为$TA$是紧集,而紧集是完备的,所以$TA$是闭集,$T$在$A$上可达到其上、下确界。\par
	(3)假设此时$T$不一致连续,则存在$\varepsilon_0>0$以及点列$\{x_n\},\{y_n\}\subset A$,使得:
	\begin{equation*}
		\rho_X(x_n,y_n)\to0,\;\rho_Y(Tx_n,Ty_n)\geqslant\varepsilon_0,\;\forall\;n\in\mathbb{N}^+
	\end{equation*}
	因为$A$是紧集,所以$\{x_n\}$存在子列$\{x_{n_k}\}\to x_0\in A$,即$\rho_X(x_{n_k},x_0)\to 0$,于是:
	\begin{equation*}
		\rho_X(y_{n_k},x_0)\leqslant\rho_X(y_{n_k},x_{n_k})+\rho_X(x_{n_k},x_0)\to0
	\end{equation*}
	因为$T$是连续的,所以:
	\begin{equation*}
		\rho_Y(Tx_{n_k},Tx_0)\to0,\;\rho_Y(Ty_{n_k},Tx_0)\to0
	\end{equation*}
	于是
	\begin{equation*}
		\rho_Y(Tx_{n_k},Ty_{n_k})\leqslant\rho_Y(Tx_{n_k},Tx_0)+\rho_Y(Tx_0,Ty_{n_k})\to 0
	\end{equation*}
	与第一个式子中的第二部分矛盾,所以$T$一致连续。
\end{proof}



%实数序列
%有界序列
%收敛序列的定义(包含无穷大,并区分R中收敛还是包含扩充的R中收敛),几何解释,在这里定义发散序列
%
%收敛序列的证明方法:定义法(放大),夹逼定理(有穷与无穷),单调收敛原理(有界情况用到了收敛序列的有界性,无界情况则无穷,把两个都包含进去),柯西收敛原理(用到了波尔查诺-魏尔斯特拉斯,,需要先说明基本序列以及基本序列的有界性)(使用时常取n与n+p,最终还是需要把p放缩掉)
%
%收敛序列的性质:极限的唯一性(无穷与有穷),有界性(有界收敛),四则运算与绝对值(右端的式子有意义,分为两方面,一方面是极限存在,另一方面是极限的那些数值做的计算有意义),收敛序列与不等式的关系
%
%子列的定义,收敛序列子列必有极限且极限相同且唯一
%
%闭区间套原理(用到了单调收敛原理,至于c的唯一性可以用上下确界的唯一性得到),波尔查诺-魏尔斯特拉斯定理(有界序列必有子列有极限,用到了闭区间套)
%
%函数极限的序列式定义与柯西式定义,两种定义的等价性。(这里考虑极限值与极限点的有穷与无穷)
%
%函数极限的证明方法:定义法,夹逼定理(序列式易证,有穷与无穷),有穷极限时的收敛原理
%
%函数极限的性质:极限的唯一性(序列式易证),有界性(极限有界,柯西式易证),四则运算与绝对值(序列式可以直接得到有穷与无穷下的所有结果),收敛函数与不等式的关系(柯西式定义),复合函数求极限(序列式与柯西式都易证)
%
%函数的单侧极限,极限存在与单侧极限的关系,单调函数的单侧极限总是存在的
%
%函数的连续性与单侧连续的定义,一类与二类间断点。
%
%闭区间上连续函数的性质:介值定理(闭区间套与连续函数交换映射与极限顺序),区间上有界(闭区间套和函数极限的有界性),最值定理(其实是上下确界,利用上下确界的定义以及有界序列必有收敛子列),一致连续(构造序列反证,利用连续函数定义与有界序列必有收敛子列。同时还有序列式表述的一致连续性以及两个表述之间的等价性)
%
%
%乘积或商时等价因式的替换(和极限的四则运算一样,要求右侧有意义并且存在对应的极限关系)
%
%
%导数的定义。单侧导数,单侧导数与导数存在的关系。
%导数的运算法则:加减(函数四则运算证明),乘除(加一减一),复合函数求导(引入微分表示的不变性),反函数求导(注意反函数的单调性导致分母不会为0),隐函数求导(注意这里用了反函数,对某个隐函数是有要求的)(同时还要注意直接求导法),对数求导法,高阶导数莱布尼茨公式。
%常见函数的导函数。
%
%可微的定义,一元函数可导与可微的充要关系,一元函数可微与连续的关系。微分的定义,函数运算后的微分(导数乘dx)。
%
%极值点的判断:
%极值点定义(函数需要有定义,此处需要有内点的定义)
%引入无穷小增量公式(微分),费马定理(极值点的必要条件,此处引入临界点的定义,最值的搜索定理)
%引入有限增量公式(罗尔中值定理:闭区间函数必有最大值和最小值,最值必然也是极值,由费马定理推出导数为0;拉格朗日中值定理:作辅助函数变为罗尔定理的形式),函数递增与严格递增的充要条件,极值点的第一充分条件(不要求可导,但要求连续),极值点的第二充分条件(比第一充分条件严格,也是从第一推导来的)
%
%原函数的定义,原函数的表示形式,不定积分,函数线性运算后的不定积分
%不定积分的求法:换元积分法(微分形式不便性,分第一和第二),分步积分法(函数积的微分),有理函数的积分