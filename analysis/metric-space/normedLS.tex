\section{范数}
\begin{definition}
	设$X$是数域$K$上的线性空间。如果对于$X$中的每个元素$x$,都有一个实数与之对应,记为$||x||$,且满足:
	\begin{enumerate}
		\item \textbf{非负性:}$||x||\geqslant 0$,等号成立当且仅当$x=\mathbf{0}$。
		\item \textbf{数乘:}$||\alpha x||=|\alpha|\;||x||$,$\alpha\in K$。
		\item \textbf{三角不等式:}$||x+y||\leqslant||x||+||y||$。
	\end{enumerate}
	则称$X$为数域$K$上的\gls{NormedLS},$||x||$为元素$x$的\gls{norm}。
\end{definition}
\begin{definition}
	对于数域$K$上的赋范线性空间$X$,我们定义$X$中元素$x$和$y$之间的距离为它们差的范数,即:
	\begin{equation*}
		\rho(x,y)=||x-y||
	\end{equation*}
\end{definition}
\begin{proof}
	(1)非负性可由范数的非负性直接验证。\par
	(2)对称性:由\cref{theo:MinimumNumberField}可知$-1\in K$,根据范数定义中的条件(2)可得$\rho(x,y)=||x-y||=|-1|\;||x-y||=||y-x||=\rho(y,x)$。\par
	(3)三角不等式:由范数定义中的条件(3)可得$\rho(x,y)=||x-y||=||x-z+z-y||\leqslant||x-z||+||z-y||=\rho(x,z)+\rho(z,y)$。
\end{proof}
\begin{definition}
	设$X$是赋范线性空间。若点列$\{x_n\}\subseteq X$收敛于点$x$,则称$\{x_n\}$\gls{convergenceNorm}于$x$,也称$\{x_n\}$\gls{Strongconvergence}于$x$。
\end{definition}
\begin{definition}
	设$X$是一个线性空间,$||\cdot||_1,\;||\cdot||_2$是定义在$X$上的两个范数,$X$按照这两个范数均为赋范线性空间。若:
	\begin{equation*}
		\exists\;K_1,K_2>0,\;\forall\;x\in X,\;K_1||x||_1\leqslant||x_2||\leqslant K_2||x||_1
	\end{equation*}
	则称范数$||\cdot||_1,\;||\cdot||_2$\textbf{等价}。
\end{definition}
\begin{property}\label{prop:Norm}
	设$X$是数域$K$上的赋范线性空间。范数具有如下性质:
	\begin{enumerate}
		\item 若$x,y\in X$,则有$|\;||x||-||y||\;|\leqslant||x-y||$;
		\item 范数是连续映射;
		\item 设$\{x_n\}$和$\{y_n\}$都是赋范线性空间$X$中的点列,且$\{x_n\}\rightarrow x$,$\{y_n\}\rightarrow y$,$\{a_n\}$和$\{b_n\}$是$K$中的点列,且$\{a_n\}\to a,\;\{b_n\}\to b,\;a,b\in K$,则$a_nx_n+b_ny_n\to ax+bx$;
		\item 等价范数定义的收敛是等价的;
	\end{enumerate}
\end{property}
\begin{proof}
	(1)$\;||x||=||x-y+y||\leqslant||x-y||+||y||$,即$||x||-||y||\leqslant||x-y||$;$||y||=||y-x+x||\leqslant||x-y||+||x||$,即$-||x-y||\leqslant||x||-||y||$。\par
	综上,$-||x-y||\leqslant||x||-||y||\leqslant||x-y||$,即$|\;||x||-||y||\;|\leqslant||x-y||$。\par
	(2)由(1)可知$|\;||x_n||-||x||\;|\leqslant||x_n-x||$。因此当$\{x_n\}$依范数收敛于$x$时,$||x_n||\to||x||$。\par
	(3)由范数的定义:
	\begin{align*}
		||a_nx_n+b_ny_n-(ax+by)||
		&\leqslant||a_nx_n-ax||+||b_ny_n-by|| \\
		&=||a_nx_n-a_nx+a_nx-ax||+||b_ny_n-b_ny+b_ny-by|| \\
		&\leqslant|a_n|\;||x_n-x||+|a_n-a|\;||x||+|b_n|\;||y_n-y||+|b_n-b|\;||y||
	\end{align*}
	由$\{a_n\}$和$\{b_n\}$的收敛性以及$\{x_n\}$和$\{y_n\}$的收敛性立即可得$a_nx_n+b_ny_n\to ax+bx$。\par
	(4)由定义即可得到。
\end{proof}

\subsection{线性算子}
\begin{definition}
	设$X$和$Y$是数域$K$上的线性空间,$E\subseteq X$,$T:E\to Y$是一个映射。若对任意的$x,y\in E$,有:
	\begin{equation*}
		T(x+y)=Tx+Ty
	\end{equation*}
	则称$T$\textbf{可加}。若对任意的$\alpha\in K$和任意的$x\in E$,有:
	\begin{equation*}
		T(\alpha x)=\alpha Tx
	\end{equation*}
	则称$T$\textbf{齐次}。称可加齐次映射为\gls{LinearOperator}。
\end{definition}
\begin{note}
	其实就是线性映射。
\end{note}
\subsubsection{线性算子的连续性}
\begin{definition}
	设$X$和$Y$是数域$K$上的赋范线性空间,$E\subseteq X$,$T:E\to Y$是一个映射。若$T$是连续的,则称$T$为\gls{ContinuousLinearOperator}。
\end{definition}
\begin{property}\label{prop:ContinuousLinearOperator}
	设$X$和$Y$是数域$K$上的赋范线性空间,$E\subseteq X$,$T:E\to Y$是一个映射。
	\begin{enumerate}
		\item 若$K=\mathbb{R}^{}$,$T$是连续可加算子,则$T$是连续线性算子;
		\item 若$K=\mathbb{C}^{}$,$T$是连续可加算子且$T(ix)=iTx$,则$T$是连续线性算子。
	\end{enumerate}
\end{property}
\begin{proof}
	(1)由可加性可以得到:
	\begin{equation*}
		\forall\;n\in\mathbb{N}^+,\;\forall\;x\in X,\;T(nx)=nTx
	\end{equation*}	
	所以$T$对正整数集满足齐次性。\par
	取$x_1=\dfrac{x}{n}$,代入上式可得:
	\begin{equation*}
		Tx=nT\left(\frac{x}{n}\right)
	\end{equation*}
	即:
	\begin{equation*}
		T\left(\frac{x}{n}\right)=\frac{Tx}{n}
	\end{equation*}
	所以$T$对正有理数集满足齐次性。\par
	取$\mathbf{0}_X\in X$,由$T$对正有理数的齐次性,$T\mathbf{0}_X=T(2\mathbf{0}_X)=2T\mathbf{0}_X$,所以$T\mathbf{0}_X=\mathbf{0}_Y$。由$T$的可加性可得:
	\begin{equation*}
		T\mathbf{0}_X=T[x+(-x)]=Tx+T(-x)=\mathbf{0}_Y\in Y
	\end{equation*}
	即:
	\begin{equation*}
		T(-x)=-Tx
	\end{equation*}
	所以$T$对有理数集满足齐次性。\par
	因为任意无理数都可用一个有理数序列去逼近,由$T$的连续性,$T$对无理数集也满足齐次性。\par
	综上,$T$是一个连续线性算子。\par
	(2)设$a=a_1+ia_2\in\mathbb{C}$,由(1)可得:
	\begin{align*}
		T(ax)&=T[(a_1+ia_2)x]=T(a_1x+ia_2x)=T(a_1x)+T(ia_2x) \\
		&=a_1Tx+iT(a_2x)=a_1Tx+ia_2Tx=(a_1+ia_2)Tx=aTx\qedhere
	\end{align*}
\end{proof}
\subsubsection{线性算子的有界性}
\begin{definition}
	设$X$和$Y$是数域$K$上的赋范线性空间,$E\subseteq X$,$T:E\to Y$是一个映射。若$T$将$E$中的任一有界集映成$Y$中的有界集,则称$T$是\gls{BoundedLinearOperator}。如果存在$E$中的有界集$A$,使得$TA$是$Y$中的无界集,则称$T$是\gls{UnboundedLinearOperator}。
\end{definition}
\begin{property}\label{prop:BoundedLinearOperator}
	设$X$和$Y$是$\mathbb{R}^{}$上的赋范线性空间,$E\subseteq X$,$T:E\to Y$是一个映射。$T$有界的充分必要条件为:
	\begin{equation*}
		\exists\;M>0,\;\forall\;x\in E,\;||Tx||_Y\leqslant M||x||_X
	\end{equation*}
\end{property}
\begin{proof}
	\textbf{(1)充分性:}任取$A\subseteq E$是一个有界集,由赋范线性空间上距离的定义和有界集的定义可知:
	\begin{equation*}
		\exists\;K>0,\;\forall\;x\in A,\;||x||_X\leqslant K
	\end{equation*}
	因此:
	\begin{equation*}
		\exists\;M>0,\;\exists\;K>0,\;\forall\;x\in A,\;||Tx||_Y\leqslant M||x||_X\leqslant MK
	\end{equation*}
	所以$TA$是$Y$中的有界集。由$A$的任意性,$T$有界。\par
	(2)必要性:在$E$中取单位球面$S=\{x\in E:||x||_X=1\}$,$S$有界,因此$TS$有界。于是:
	\begin{equation*}
		\exists\;M>0,\;\forall\;x\in S,\;||Tx||_Y\leqslant M
	\end{equation*}
	设$x$为$E$中任意非$\mathbf{0}$的元素,则:
	\begin{equation*}
		\frac{x}{||x||_X}\in S
	\end{equation*}
	所以有:
	\begin{equation*}
		\left\|T\left(\frac{x}{||x||_X}\right)\right\|_Y\leqslant M
	\end{equation*}
	由$T$的齐次性:
	\begin{equation*}
		||Tx||_Y\leqslant M||x||_X
	\end{equation*}
	根据\cref{prop:LinearMapping}(2)可知当$x=\mathbf{0}$时,$Tx$必然是$Y$中的零元,上式也成立。
\end{proof}
\subsubsection{线性算子有界与连续的关系}
\begin{theorem}\label{theo:LinearOperatorBoundedContinuous}
	设$X$和$Y$是$\mathbb{R}^{}$上的赋范线性空间,$E\subseteq X$,$T:E\to Y$是一个映射,则以下陈述等价:
	\begin{enumerate}
		\item $T$在$E$上连续。
		\item $T$在$E$中某给定点$x_0$连续。
		\item $T$有界。
	\end{enumerate}
\end{theorem}
\begin{proof}
	$(1)\to(2)$由定义即可得到。下证$(2)\to(3)$。\par
	因为$T$在$x_0$连续,因此对$\varepsilon=1,\;\exists\;\delta>0,\;\forall\;x\in E$,当$||x-x_0||_X\leqslant\delta$时,有:
	\begin{equation*}
		||Tx-Tx_0||_Y=||T(x-x_0)||_Y\leqslant\varepsilon=1
	\end{equation*}
	任取$x_1\ne x_0$,且$x_1\in E$,因为:
	\begin{equation*}
		\left\|\frac{\delta(x_1-x_0)}{||x_1-x_0||_X}\right\|_X=\delta\leqslant\delta
	\end{equation*}
	所以:
	\begin{equation*}
		\left\|T\left[\frac{\delta(x_1-x_0)}{||x_1-x_0||_X}\right]\right\|_Y\leqslant1
	\end{equation*}
	作变形即有:
	\begin{equation*}
		||T(x_1-x_0)||_Y\leqslant\frac{1}{\delta}||x_1-x_0||_X
	\end{equation*}
	由\cref{prop:LinearMapping}(2)可知:
	\begin{equation*}
		||T(x_0-x_0)||_Y=0=\frac{1}{\delta}||x_0-x_0||_X
	\end{equation*}
	所以对任意的$x\in E$,有:
	\begin{equation*}
		||T(x-x_0)||_Y\leqslant\frac{1}{\delta}||x-x_0||_X
	\end{equation*}
	由\cref{prop:BoundedLinearOperator}可知$T$定义在$E-x_0$上的时候是有界的,而$E-x_0$实际上就是$E$,所以$T$是有界的。\par
	$(3)\to(1)$:由\cref{prop:BoundedLinearOperator}立即可得。
\end{proof}
\subsection{有限维赋范线性空间}
\subsubsection{拓扑同构}
\begin{definition}
	设$(X,\rho_X)$和$(Y,\rho_Y)$都是度量空间,$T$是一个$X$到$Y$的双射。若$T$和$T^{-1}$都是连续映射,则称$T$是$X$到$Y$上的\gls{HomeoMap}。如果存在一个$X$到$Y$上的同胚映射,则称$X$和$Y$\gls{homeomorphic}。
\end{definition}
\begin{definition}
	设$X$和$Y$都是赋范线性空间,如果满足条件:
	\begin{enumerate}
		\item $X$和$Y$作为线性空间是同构的。
		\item 从$X$到$Y$的同构映射$T$是同胚的。
	\end{enumerate}
	则称$X$和$Y$\gls{TopoIso}。
\end{definition}
\begin{definition}
	$X$是一个赋范线性空间,若$X$的任一有界闭集是紧的,则称$X$是
	\gls{LocallyCompact}。
\end{definition}
\begin{property}\label{prop:FiniteDimensionalNormedLinearSpace}
	设$X$是数域$K$上的$n$维赋范线性空间,则:
	\begin{enumerate}
		\item 若$K$是实数域或复数域,则$X$与数域$K$上的任一$n$维赋范线性空间拓扑同构;
		\item $X$是完备的;
		\item 赋范线性空间$X$是有限维的的充分必要条件为$X$是局部紧的;
	\end{enumerate}
\end{property}
\begin{proof}
	(1)由\cref{prop:IsomorphicOfLinearSpace}(7)及\cref{theo:CompositeContinuousMap}可知拓扑同构具有传递性,于是只需证明实(复)$n$维赋范线性空间与$\mathbb{R}^n$($\mathbb{C}^{n}$)拓扑同构。\par
	任取一个实$n$维赋范线性空间$X$。下证$X$和$\mathbb{R}^n$之间存在一个双射。\par
	设$X$是一个实的$n$维赋范线性空间,$\{e_1,e_2,\dots,e_n\}$是$X$的一个基。定义一个映射$T:\mathbb{R}^n\mapsto X$如下($\xi\in\mathbb{R}^n$):
	\begin{equation*}
		T\xi=\sum_{i=1}^n\xi_ie_i
	\end{equation*}
	由于$\{e_1,e_2,\dots,e_n\}$是$X$的一个基,根据\cref{prop:LinearlyDependent}(4)可知$T\xi$的表出方式唯一,即$T$是一个单射。其次对任意的$x\in X$,根据基的定义它必然能由$\{e_1,e_2,\dots,e_n\}$线性表出,因此有$\eta=(\eta_1,\eta_2,\dots,\eta_n)\in\mathbb{R}^n$,使得$x=\sum\limits_{i=1}^n\eta_ie_i=T\eta$,因此$T$是一个满射。\par
	映射$T$保持线性运算是显然的。\par
	综上,$X$和$\mathbb{R}^n$作为线性空间是同构的。\par
	下证映射$T$是连续映射。\par
	由\cref{ineq:cauchy-ineq-R}可得:
	\begin{align*}
		||T\xi-T\eta||_X
		&=\left\|\sum_{i=1}^n(\xi_i-\eta_i)e_i\right\|_X\leqslant\sum_{i=1}^n|\xi_i-\eta_i|\;||e_i||_X \\
		&\leqslant\left(\sum_{i=1}^n|\xi_i-\eta_i|^2\right)^\frac{1}{2}\cdot\left(\sum_{i=1}^n||e_i||_X^2\right)^\frac{1}{2}
	\end{align*}
	所以$T$是一个连续映射。\par
	下证$T^{-1}$是一个连续映射。\par
	要证$T^{-1}$是一个连续映射,证明$\forall\;x,y\in X$,$\exists\;\alpha\geqslant0$使得:
	\begin{equation*}
		||T^{-1}x-T^{-1}y||\leqslant\alpha||x-y||_X
	\end{equation*}
	即可。因为已经证得$T$是一个双射且保持线性运算,所以只需证$||T^{-1}x||\leqslant\alpha||x||_X$,即:
	\begin{equation*}
		\exists\;\alpha>0,\;\forall\;x\in X,\;\frac{||x||_X}{||T^{-1}x||}\geqslant\alpha
	\end{equation*}
	对$x=\sum\limits_{i=1}^n\xi_ie_i\in X$,令:
	\begin{equation*}
		f(\xi_1,\xi_2,\dots,\xi_n)=||x||_X
	\end{equation*}
	当$(\xi_1,\xi_2,\dots,\xi_n)$在$\mathbb{R}^n$的单位球面上时,即$\sum\limits_{i=1}^n\xi_i^2=1$时,由\cref{prop:IsomorphicOfLinearSpace}(1)可知$||x||_X\ne0$。又因为单位球面是一个有界闭集(因为$\{1\}$是闭集,根据\cref{prop:Norm}(2)和\cref{theo:ContinousMapO2OC2C}可知$\{x;||x||_2=1\}$是闭集。),由\cref{theo:CompactRn}和\cref{prop:CompactMap}(3)可得$f(\xi_1,\xi_2,\dots,\xi_n)$在单位球面上有正的下确界$\alpha$,即$||x||_X\geqslant\alpha$。对任意的$x\in X$进行单位化,即取$x'$满足:
	\begin{equation*}
		x'=\frac{x}{||T^{-1}x||}
	\end{equation*}
	那么就有$||x'||_X\geqslant\alpha$,即:
	\begin{equation*}
		\frac{||x||_X}{||T^{-1}x||}\geqslant\alpha
	\end{equation*}\par
	复:类似,$\mathbb{R}^{2n}$。\par
	(2)由(1)可知$X$与$\mathbb{R}^n$是拓扑同构的,那么$X$与$\mathbb{R}^n$之间存在一个双射$T:X\rightarrow\mathbb{R}^n$,且$T$和$T^{-1}$是连续的。任取$X$上的一个Cauchy点列$\{x_n\}$,$Tx_n=y_n\in\mathbb{R}^n$。\par 
	下证$\{y_n\}$也是一个柯西序列,即证对任意的$\varepsilon>0$,存在$N\in\mathbb{N}^+$,当$n,m>N$时,有$\rho(y_n,y_m)<\varepsilon$。\par
	因为$T$是连续的,所以对上述$\varepsilon$,存在$\delta>0$,当$\rho(x_n,x_m)<\delta$时,有$\rho(y_n,y_m)<\varepsilon$。而$\{x_n\}$是Cauchy点列,因此对上述$\delta$,存在$N_1\in\mathbb{N}^+$,当$n,m>N_1$时,有$\rho(x_n,x_m)<\delta$。所以取$N=N_1$即可满足条件。因此$\{y_n\}$是一个柯西序列。\par
	由\cref{theo:RnComplete}可知$\mathbb{R}^n$是完备的,所以$\{y_n\}\rightarrow y\in\mathbb{R}^n$,于是:
	\begin{equation*}
		\lim_{n\to+\infty}T^{-1}y_n=T^{-1}\left(\lim_{n\to+\infty}y_n\right)=T^{-1}y
	\end{equation*}
	因为$T$是双射,那么存在$x\in X$使得$x=T^{-1}y$。因为:
	\begin{equation*}
		\lim_{n\to+\infty}T^{-1}y_n=\lim_{n\to+\infty}x_n
	\end{equation*}
	所以$\{x_n\}\rightarrow x\in X$。由$\{x_n\}$的任意性,$X$是完备的。\par
	(3)\textbf{必要性:}由(1)可知$X$与$\mathbb{R}^n$拓扑同构,那么$X$与$\mathbb{R}^n$之间存在一个双射$T:X\rightarrow\mathbb{R}^n$,且$T$和$T^{-1}$是连续的。由\cref{theo:ContinousMapO2OC2C}(2)和\cref{theo:LinearOperatorBoundedContinuous}(3)可知$X$中的有界闭集映成$\mathbb{R}^n$中的有界闭集,反之亦然。由\cref{theo:CompactRn}和\cref{prop:CompactMap}(1)可知$X$中的任一有界闭集也是紧的,所以$X$是局部紧的。\par
	充分性:若此时$X$是无限维的。取$S=\{x:||x||=1\}$为$X$的单位球面\info{有空证明}
\end{proof}









\subsection{$\mathbb{R}^{n}$上的范数}
\begin{definition}
	在$\mathbb{R}^n$中定义元素$x=(\xi_1,\xi_2,\dots,\xi_n)$的范数为:
	\begin{equation*}
		||x||_2=\left(\sum_{i=1}^n\xi_i^2\right)^{\frac{1}{2}}
	\end{equation*}
	则$\mathbb{R}^n$成为一个赋范线性空间。称上述范数为\textbf{欧式范数}。
\end{definition}
\begin{proof}
	(1)$\;||x||_2\in\mathbb{R}$、(2)非负性和(3)数乘显然,(4)三角不等式的证明可见欧式距离三角不等式的证明。
\end{proof}
\begin{theorem}\label{theo:NormRn}
	$\mathbb{R}^{n}$上的任意两个范数都等价。
\end{theorem}
\begin{proof}
	只需证明$\mathbb{R}^{n}$上的任意一个范数都与欧式范数等价(传递性自行验证)。取$\mathbb{R}^{n}$上的范数$N(x)$。\par
	先证明$\{x:||x||_2=1\}$是有界闭集。有界性立即可得。因为$\{1\}$是闭集,根据\cref{prop:Norm}(2)和\cref{theo:ContinousMapO2OC2C}可知$\{x;||x||_2=1\}$是闭集。\par
	由\cref{theo:CompactRn}可知$\{x:||x||_2=1\}$是紧集,根据\cref{prop:Norm}(2)和\cref{prop:CompactMap}(3)可知$N(x)$在$\{x:||x||_2=1\}$上存在最小值$a$和最大值$b$。对于任意不为$\mathbf{0}$的$x\in\mathbb{R}^{n}$有$\dfrac{x}{||x||_2}\in\{x:||x||_2=1\}$,于是有:
	\begin{equation*}
		a\leqslant N\left(\frac{x}{||x||_2}\right)\leqslant b
	\end{equation*}
	即:
	\begin{equation*}
		a||x||_2\leqslant N(x)\leqslant b||x||_2
	\end{equation*}
	当$x=\mathbf{0}$时,上式也成立,所以$N(x)$与欧式范数$||x||_2$等价。
\end{proof}