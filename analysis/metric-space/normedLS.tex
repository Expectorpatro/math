\section{范数}
\begin{definition}
	设$X$是实或者复线性空间,如果对于$X$中的每个元素$x$,都有一个实数与之对应,记为$||x||$,且满足:
	\begin{enumerate}
		\item 非负性:$||x||\geqslant 0$,等号成立当且仅当$x=\mathbf{0}$。
		\item 数乘:$||\alpha x||=|\alpha|\;||x||$,$\alpha\in\mathbb{C}$或$\mathbb{R}$。
		\item 三角不等式:$||x+y||\leqslant||x||+||y||$。
	\end{enumerate}
	则称$X$为实或复的\gls{NormedLS},$||x||$为元素$x$的\gls{norm}。
\end{definition}
\begin{definition}
	对于赋范线性空间$X$,我们定义下式来衡量$X$中元素$x$和$y$之间的距离:
	\begin{equation*}
		\rho(x,y)=||x-y||
	\end{equation*}
\end{definition}
\begin{proof}
	(1)非负性可由范数的非负性直接验证。\par
	(2)对称性:由范数定义中的条件(2)可得$\rho(x,y)=||x-y||=|-1|\;||x-y||=||y-x||=\rho(y,x)$。\par
	(3)三角不等式:由范数定义中的条件(3)可得$\rho(x,y)=||x-y||=||x-z+z-y||\leqslant||x-z||+||z-y||=\rho(x,z)+\rho(z,y)$。
\end{proof}
\begin{definition}
	赋范线性空间$X$中,若点列$\{x_n\}$收敛于点$x$,则称$\{x_n\}$\gls{convergenceNorm}于$x$,也称$\{x_n\}$\gls{Strongconvergence}于$x$。
\end{definition}
\begin{property}\label{prop:Norm}
	范数具有如下性质:
	\begin{enumerate}
		\item 设$X$是一个赋范线性空间,$x,y\in X$,有:
		\begin{equation*}
			|\;||x||-||y||\;|\leqslant||x-y||
		\end{equation*}
		\item 范数是连续泛函\info{泛函的定义};
		\item 设$\{x_n\}$和$\{y_n\}$都是赋范线性空间$X$中的点列,且$\{x_n\}\rightarrow x$,$\{y_n\}\rightarrow y$,$\{a_n\}$和$\{b_n\}$是$\mathbb{R}^{}$或$\mathbb{C}^{}$中的点列,且$\{a_n\}\to a,\;\{b_n\}\to b,\;|a|,|b|\in R$,则$a_nx_n+b_ny_n\to ax+bx$。 
	\end{enumerate}
\end{property}
\begin{proof}
	(1)$||x||=||x-y+y||\leqslant||x-y||+||y||$,即$||x||-||y||\leqslant||x-y||$;$||y||=||y-x+x||\leqslant||x-y||+||x||$,即$-||x-y||\leqslant||x||-||y||$。\par
	综上,$-||x-y||\leqslant||x||-||y||\leqslant||x-y||$,即$|\;||x||-||y||\;|\leqslant||x-y||$。\par
	(2)由(1)可知$|\;||x_n||-||x||\;|\leqslant||x_n-x||$。因此当$\{x_n\}$依范数收敛于$x$时,$||x_n||\to||x||$。\par
	(3)由范数的定义:
	\begin{align*}
		||a_nx_n+b_ny_n-(ax+by)||
		&\leqslant||a_nx_n-ax||+||b_ny_n-by|| \\
		&=||a_nx_n-a_nx+a_nx-ax||+||b_ny_n-b_ny+b_ny-by|| \\
		&\leqslant|a_n|\;||x_n-x||+|a_n-a|\;||x||+|b_n|\;||y_n-y||+|b_n-b|\;||y||
	\end{align*}
	由$\{a_n\}$和$\{b_n\}$的收敛性以及$\{x_n\}$和$\{y_n\}$的收敛性立即可得$a_nx_n+b_ny_n\to ax+bx$。
\end{proof}
