\section{范数}
\begin{definition}
	设$X$是数域$K$上的线性空间。如果对于$X$中的每个元素$x$,都有一个实数与之对应,记为$||x||$,且满足:
	\begin{enumerate}
		\item \textbf{非负性:}$||x||\geqslant 0$,等号成立当且仅当$x=\mathbf{0}$。
		\item \textbf{数乘:}$||\alpha x||=|\alpha|\;||x||$,$\alpha\in K$。
		\item \textbf{三角不等式:}$||x+y||\leqslant||x||+||y||$。
	\end{enumerate}
	则称$X$为数域$K$上的\gls{NormedLS},$||x||$为元素$x$的\gls{norm}。
\end{definition}
\begin{definition}
	对于数域$K$上的赋范线性空间$X$,我们定义$X$中元素$x$和$y$之间的距离为它们差的范数,即:
	\begin{equation*}
		\rho(x,y)=||x-y||
	\end{equation*}
\end{definition}
\begin{proof}
	(1)非负性可由范数的非负性直接验证。\par
	(2)对称性:由\cref{theo:MinimumNumberField}可知$-1\in K$,根据范数定义中的条件(2)可得$\rho(x,y)=||x-y||=|-1|\;||x-y||=||y-x||=\rho(y,x)$。\par
	(3)三角不等式:由范数定义中的条件(3)可得$\rho(x,y)=||x-y||=||x-z+z-y||\leqslant||x-z||+||z-y||=\rho(x,z)+\rho(z,y)$。
\end{proof}
\begin{definition}
	设$X$是赋范线性空间。若点列$\{x_n\}\subseteq X$收敛于点$x$,则称$\{x_n\}$\gls{convergenceNorm}于$x$,也称$\{x_n\}$\gls{Strongconvergence}于$x$。
\end{definition}
\begin{definition}
	设$X$是一个线性空间,$||\cdot||_1,\;||\cdot||_2$是定义在$X$上的两个范数,$X$按照这两个范数均为赋范线性空间。若:
	\begin{equation*}
		\exists\;K_1,K_2>0,\;\forall\;x\in X,\;K_1||x||_1\leqslant||x_2||\leqslant K_2||x||_1
	\end{equation*}
	则称范数$||\cdot||_1,\;||\cdot||_2$\textbf{等价}。
\end{definition}
\begin{property}\label{prop:Norm}
	设$X$是一个赋范线性空间。范数具有如下性质:
	\begin{enumerate}
		\item 若$x,y\in X$,则有$|\;||x||-||y||\;|\leqslant||x-y||$;
		\item 范数是连续映射;
		\item 设$\{x_n\}$和$\{y_n\}$都是赋范线性空间$X$中的点列,且$\{x_n\}\rightarrow x$,$\{y_n\}\rightarrow y$,$\{a_n\}$和$\{b_n\}$是$\mathbb{R}^{}$或$\mathbb{C}^{}$中的点列,且$\{a_n\}\to a,\;\{b_n\}\to b,\;|a|,|b|\in R$,则$a_nx_n+b_ny_n\to ax+bx$;
		\item 等价范数定义的收敛是等价的;
	\end{enumerate}
\end{property}
\begin{proof}
	(1)$\;||x||=||x-y+y||\leqslant||x-y||+||y||$,即$||x||-||y||\leqslant||x-y||$;$||y||=||y-x+x||\leqslant||x-y||+||x||$,即$-||x-y||\leqslant||x||-||y||$。\par
	综上,$-||x-y||\leqslant||x||-||y||\leqslant||x-y||$,即$|\;||x||-||y||\;|\leqslant||x-y||$。\par
	(2)由(1)可知$|\;||x_n||-||x||\;|\leqslant||x_n-x||$。因此当$\{x_n\}$依范数收敛于$x$时,$||x_n||\to||x||$。\par
	(3)由范数的定义:
	\begin{align*}
		||a_nx_n+b_ny_n-(ax+by)||
		&\leqslant||a_nx_n-ax||+||b_ny_n-by|| \\
		&=||a_nx_n-a_nx+a_nx-ax||+||b_ny_n-b_ny+b_ny-by|| \\
		&\leqslant|a_n|\;||x_n-x||+|a_n-a|\;||x||+|b_n|\;||y_n-y||+|b_n-b|\;||y||
	\end{align*}
	由$\{a_n\}$和$\{b_n\}$的收敛性以及$\{x_n\}$和$\{y_n\}$的收敛性立即可得$a_nx_n+b_ny_n\to ax+bx$。\par
	(4)由定义即可得到。
\end{proof}
\begin{definition}
	在$\mathbb{R}^n$中定义元素$x=(\xi_1,\xi_2,\dots,\xi_n)$的范数为:
	\begin{equation*}
		||x||_2=\left(\sum_{i=1}^n\xi_i^2\right)^{\frac{1}{2}}
	\end{equation*}
	则$\mathbb{R}^n$成为一个赋范线性空间。称上述范数为\textbf{欧式范数}。
\end{definition}
\begin{proof}
	(1)$\;||x||_2\in\mathbb{R}$、(2)非负性和(3)数乘显然,(4)三角不等式的证明可见欧式距离三角不等式的证明。
\end{proof}
\begin{theorem}\label{theo:NormRn}
	$\mathbb{R}^{n}$上的任意两个范数都等价。
\end{theorem}
\begin{proof}
	只需证明$\mathbb{R}^{n}$上的任意一个范数都与欧式范数等价(传递性自行验证)。取$\mathbb{R}^{n}$上的范数$N(x)$。\par
	先证明$\{x:||x||_2=1\}$是有界闭集。有界性立即可得。因为$\{1\}$是闭集,根据\cref{prop:Norm}(2)和\cref{theo:ContinousMapO2OC2C}可知$\{x;||x||_2=1\}$是闭集。\par
	由\cref{theo:CompactRn}可知$\{x:||x||_2=1\}$是紧集,根据\cref{prop:Norm}(2)和\cref{prop:CompactMap}(3)可知$N(x)$在$\{x:||x||_2=1\}$上存在最小值$a$和最大值$b$。对于任意不为$\mathbf{0}$的$x\in\mathbb{R}^{n}$有$\dfrac{x}{||x||_2}\in\{x:||x||_2=1\}$,于是有:
	\begin{equation*}
		a\leqslant N\left(\frac{x}{||x||_2}\right)\leqslant b
	\end{equation*}
	即:
	\begin{equation*}
		a||x||_2\leqslant N(x)\leqslant b||x||_2
	\end{equation*}
	当$x=\mathbf{0}$时,上式也成立,所以$N(x)$与欧式范数$||x||_2$等价。
\end{proof}