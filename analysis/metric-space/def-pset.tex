\section{度量空间上的点集}
\subsection{基础知识}
\subsubsection{点的定义}
\begin{definition}
	设$(X,\rho)$是一个度量空间,$E\subseteq X$。若对于点$P\in X$,存在$U(P)\subseteq E$,则称点$P$为$E$的\gls{InteriorP}。
\end{definition}
\begin{definition}
	设$(X,\rho)$是一个度量空间,$E\subseteq X$。若点$P\in X$为$E^c$的内点,则称点$P$为$E$的\gls{ExteriorP}。
\end{definition}
\begin{definition}
	设$(X,\rho)$是一个度量空间,$E\subseteq X$。若对于点$P\in X$,它的任意邻域中都既有$E$中的点,又有$E^c$中的点,则称点$P$为$E$的\gls{BoundaryP}。
\end{definition}
\begin{definition}
	设$(X,\rho)$是一个度量空间,$E\subseteq X$。若点$P\in X$满足以下任一条件:
	\begin{enumerate}
		\item $P$的任何邻域内都存在无穷个$E$中的点。
		\item $P$的任何邻域内都存在一个$E$中异于$P$的点。
		\item $E$中有一个不包含$P$且极限为$P$的点列。
	\end{enumerate}
	则$P$是$E$的一个\gls{LimitP}。
\end{definition}
这三个条件实际上是等价的:
\begin{proof}
	$(1)\to(2)$显然,$(3)\to(1)$显然,下证$(2)\to(3)$。
	取$\delta_n=\dfrac{1}{n}$,在$U(P,\delta_n)$中至少有一点$P_n$,$P_n\in E$且$P_n\ne P$,显然$\{P_n\}\to P$。
\end{proof}
\begin{definition}
	设$(X,\rho)$是一个度量空间,$E\subseteq X$。若点$P\in E$但不是$E$的聚点,即存在$P$的邻域$U(P)$,使得$E\cap U(P)=\{P\}$,则称$P$为$E$的\gls{IsolatedP}。
\end{definition}
\subsubsection{点集的定义}
\begin{definition}
	设$(X,\rho)$是一个度量空间,$E\subseteq X$。$E$的全体内点所组成的集合称为$E$的\gls{interior},记为$\mathring{E}$。
\end{definition}
\begin{definition}
	设$(X,\rho)$是一个度量空间,$E\subseteq X$。$E$的全体聚点所组成的集合称为$E$的\gls{DerivedSet},记为$E'$。
\end{definition}
\begin{definition}
	设$(X,\rho)$是一个度量空间,$E\subseteq X$。$E$的全体界点所组成的集合称为$E$的\gls{boarder},记为$\partial E$。
\end{definition}
\begin{definition}
	设$(X,\rho)$是一个度量空间,$E\subseteq X$。$E\cup E'$称为$E$的\gls{closure},记为$\overline{E}$。
\end{definition}
\begin{definition}
	设$(X,\rho)$是一个度量空间,$E\subseteq X$。若$E=\mathring{E}$,则称E是一个\gls{OpenSet}。
\end{definition}
\begin{definition}
	设$(X,\rho)$是一个度量空间,$E\subseteq X$。若$E'\subseteq E$,则称E是一个\gls{ClosedSet}。
\end{definition}
\begin{property}\label{prop:OpenClosedSet}
	设$(X,\rho)$是一个度量空间,$P\in X,\;E\subseteq X$,则有:
	\begin{enumerate}
		\item $\forall\;P_0\in U(P),\;\exists\;U(P_0)\subset U(P)$,即开邻域是开集。
		\item $\forall\;P_0\in[\bar{U}(P)]',\;P_0\in \bar{U}(P_0)$,即闭邻域是闭集;
		\item $\mathring{E}$是开集,$E'$和$\overline{E}$是闭集;
		\item $X$和$\varnothing$是唯二既开又闭的集合;
		\item 若$E$是开集,则$E^c$是闭集;若$E$是闭集,则$E^c$是开集;
		\item 任意个开集的并是开集,有限个开集的交是开集;任意个闭集的交是闭集,有限个闭集的并是闭集。
	\end{enumerate}
\end{property}
\begin{proof}
	(1)要使得$U(P_0)\subseteq U(P)$,则$\forall\;x\in U(P_0),\;\rho(x,P)<\delta_P$,而$\rho(x,P)<\rho(x,P_0)+\rho(P_0,P)<\delta_{P_0}+\rho(P_0,P)$,故只要使$\delta_{P_0}+\rho(P_0,P)<\delta_P$即可。注意到$\rho(P_0,P)<\delta_P$,由\info{实数的稠密性}可知这样的$\delta_{P_0}$总是存在的,于是结论成立。 \par
	(2)对任意的$P_0\in\left(\bar{U}(P)\right)'$,$\exists\;\bar{U}(P)$中的点列$\{P_n\}\to P_0$。若$P_0\notin \bar{U}(P)$,则$\rho(P_0,P)>\delta_P$。取$\varepsilon=\rho(P_0,P)-\delta_P$,则必然$\exists\;N\in\mathbb{N}^+$,当$n>N$时,有$\rho(P_n,P_0)<\varepsilon$,那么$\rho(P_n,P)>\delta_P$,即$P_n\notin U(P)$,矛盾。因此$P_0\in \bar{U}(P)$。由$P_0$的任意性,闭邻域是闭集。\par
	(3)\textbf{$\;\mathring{E}$是开集:}对任意的$P\in\mathring{E}$,存在$U(P)\subseteq E$,对任意的$Q\in U(P)$,由(1)可得存在$U(Q)\subseteq U(P)\subseteq E$,因此$Q\in\mathring{E}$,即存在$U(P)\subseteq\mathring{E}$,$P$是$\mathring{E}$的内点。由$P$的任意性,$\mathring{E}$是开集。\par
	\textbf{$\;E'$是闭集:}对任意的$P\in (E')'$,由聚点定义,任意$U(P)$中都含有$E'$中的点,任取$Q\in U(P)\cap E'$,任意$U(Q)$都含$E$中的点,而$Q\in U(P)$,由(1)可知存在$U(Q)\subset U(P)$,因此任意$U(P)$中都含有$E$中的点,即$P$是$E$的聚点,$P\in E'$。由$P$的任意性,$E'$是闭集。\par
	\textbf{$\;\overline{E}$是闭集:}$\overline{E}=E\cup E'$,由\cref{theo:cuplimitset eq2 limitsetcup},$(\overline{E})'=E'\cup (E')'$,显然$E'\subseteq \overline{E}$,而$E'$是闭集,$(E')'\subseteq E'\subset\overline{E}$。因此,$(\overline{E})'\subseteq\overline{E}$,即$\overline{E}$是闭集。\par
	(4)\par
	(5)$\;E$是开集:$\forall\; P\in (E^c)'$,应有$P\notin E$,即$P$不会是$E$的内点。否则由内点定义,存在$U(P)$使得$U(P)\subseteq E$,即$U(P)$中不含$E^c$中的点,这与点$P\in (E^c)'$矛盾。由$P$的任意性有$(E^c)'\subseteq E^c$,即$E^c$是闭集。\par
	$E$是闭集:$\forall\; P\in E^c$,应有$P$是$E^c$的一个内点。否则任意$U(P)$中都存在$E$中的点,则$P$应该是$E$的一个聚点,而此时$P\in E^c$,与$E$是闭集矛盾。由$P$的任意性可得$E^c$是开集。\par
	(6)\textbf{任意个开集的并是开集:}对于在任意个开集的并集中的点$P$,$P$必属于其中一个开集,因而存在一个邻域$U(P)$使得其中的点都在那个开集中,进而$U(P)$都在任意个开集的并集中,即$P$是任意个开集的并集的内点。由$P$的任意性,任意个开集的并集是开集。\par
	\textbf{有限个开集的交是开集:}设这有限个开集为$A_i,i=1,2,\dots,n$。对任意的$ P\in\underset{i=1}{\overset{n}{\cap}}A_i$,则$P$是所有$A_i$的内点,因此存在$U_i(P)\subseteq A_i$。由\cref{prop:Neighbourhood}(2)可知存在$U(P)$满足$U(P)\subseteq\underset{i=1}{\overset{n}{\cap}}U_i(P)\subseteq\underset{i=1}{\overset{n}{\cap}}A_i$,因此$P$是$\underset{i=1}{\overset{n}{\cap}}A_i$的内点。由$P$的任意性,$\underset{i=1}{\overset{n}{\cap}}A_i$是开集。\par
	\textbf{任意个闭集的交是闭集,有限个闭集的并是闭集:}可由(5)以及\cref{prop:SetOperation}(7)在开集的基础上直接得出。
\end{proof}
\begin{definition}
	设$(X,\rho_X)$是一个度量空间。若$E\subseteq X$,且在$E$上定义距离$\rho_E$使得$E$中任意点对$x,y$之间的距离$\rho_E(x,y)=\rho_X(x,y)$,则称$(E,\rho_E)$是$(X,\rho_X)$的一个\gls{subspace}。若$E$是一个开集,则称$E$是$X$的一个\gls{OpenSubspace};$E$是一个闭集,则称$E$是$X$的一个\gls{ClosedSubspace}。
\end{definition}
\subsubsection{内外界聚孤立五点的性质}
\begin{theorem}
	设$(X,\rho)$是一个度量空间,$E\subset X$。$E$的界点不是聚点就是孤立点。
\end{theorem}
\begin{proof}
	若$P$是$E$的一个界点,则它的任意邻域中都既有$E$中的点,又有$E^c$中的点。若它的任意邻域中都只含有有限个$E$中的点,则一定存在一个邻域,使得只有自身属于$E$,那么它是一个孤立点;若它的任意邻域中都含有无限个$E$中的点,由聚点定义,它是一个聚点。
\end{proof}
\subsubsection{关于内外界聚孤立五点的总结}
\begin{equation*}
	\text{度量空间}(X,\rho)\text{中的点与$E$的关系可分为}
	\begin{cases}
		\text{内点} \\
		\text{界点} \\
		\text{外点}    
	\end{cases}\;\text{或}
	\begin{cases}
		\text{聚点}   \\
		\text{孤立点} \\
		\text{外点}    
	\end{cases}
\end{equation*}
\subsubsection{开核、导集、边界、闭包的性质}
\begin{theorem}
	设$(X,\rho)$是一个度量空间,$E\subset X$。$(\mathring{E})^c=\overline{E^c}$。
\end{theorem}
\begin{proof}
	$\overline{E^c}=E^c\cup (E^c)'$。\par
	对任意的$ P\in\overline{E^c}\cap\mathring{E}$,应有$P\in(E^c)'$,即$P$的任意邻域中都有$E^c$中的点,而这与$P\in\mathring{E}$矛盾,由$P$的任意性有$\overline{E^c}\cap\mathring{E}=\varnothing$,即$\overline{E^c}\subset(\mathring{E})^c$;\par
	对任意的$ P\in(\mathring{E})^c$,应有$P\in\mathring{(E^c)}\cup\partial E$。若$P\in\mathring{(E^c)}$,显然$P\in\mathring{(E^c)}\subset E^c\subset\overline{E^c}$;若$P\in\partial E$,要么$P\in E$,要么$P\notin E$,如果此时$P\in E$,则由界点定义它必然是$E^c$的聚点,即$P\in \overline{E^c}$;如果此时$P\notin E$,则$P\in E^c$,即$P\in \overline{E^c}$。由$P$的任意性有$(\mathring{E})^c\subset\overline{E^c}$。\par
	综上$(\mathring{E})^c=\overline{E^c}$。
\end{proof}
\begin{theorem}
	设$(X,\rho)$是一个度量空间,$E\subset X$。$\partial E = \partial E^c$。
\end{theorem}
\begin{proof}
	由定义直接可得。
\end{proof}
\begin{theorem}
	设$(X,\rho)$是一个度量空间,$E\subset X$。$\overline{\overline E}=\overline{E}$。
\end{theorem}
\begin{proof}
	$\overline{\overline E}=\overline{E}\cup (\overline{E})'$,由\cref{prop:OpenClosedSet}(3),$\overline{E}$是闭集,因此$(\overline{E})'\subset\overline{E}$,即$\overline{\overline E}=\overline{E}$。
\end{proof}
\begin{theorem}\label{theo:subsetilc}
	设$(X,\rho)$是一个度量空间,$A\cup B\subset X$。若$A\subset B$,则$\mathring{A}\subset\mathring{B}$,$A'\subset B'$,$\overline{A}\subset\overline{B}$。
\end{theorem}
\begin{proof}
	由内点、聚点的定义直接可得。
\end{proof}
\begin{theorem}\label{theo:cuplimitset eq2 limitsetcup}
	设$(X,\rho)$是一个度量空间,$A\cup B\subset X$。$(A\cup B)'=A'\cup B'$。
\end{theorem}
\begin{proof}
	$A\subset (A\cup B)$,$B\subset (A\cup B)$,由\cref{theo:subsetilc}可知,$A'\subset(A\cup B)'$,$B'\subset(A\cup B)'$,因此$A'\cup B'\subset(A\cup B)'$。\par
	另一方面,对任意的$ P\in (A\cup B)'$,应有$P\in A'\cup B'$。否则就有$P\notin A'$且$P\notin B'$,因此存在$U_1(P)$和$U_2(P)$使得$U_1(P)$中不含除了$P$以外的$A$中的点、$U_2(P)$中不含除了$P$以外的$B$中的点,由邻域的性质,存在$U_3(P)$使得其内不含除了$P$以外$A\cup B$中的点,而这与$P\in (A\cup B)'$矛盾。由$P$的任意性有$(A\cup B)'\subset A'\cup B'$。\par
	综上$(A\cup B)'=A'\cup B'$。
\end{proof}
\begin{theorem}
	设$(X,\rho)$是一个度量空间,$A\cup B\subset X$。$\overline{A\cup B}=\overline{A}\cup \overline{B}$。
\end{theorem}
\begin{proof}
	由\cref{theo:cuplimitset eq2 limitsetcup}:
	\begin{equation*}
		\overline{A}\cup \overline{B}=A\cup A'\cup B\cup B'
		=(A\cup B)\cup(A'\cup B')=(A\cup B)\cup(A\cup B)'
		=\overline{A\cup B}\qedhere
	\end{equation*}
\end{proof}
\subsection{稠密、稀疏、可分、全有界、准紧性}
\subsubsection{稠密性}
\begin{definition}
	设$(X,\rho)$是一个度量空间,$A,B\subseteq X$。若$A\subseteq\overline{B}$,则称$B$在$A$中\gls{dense}。
\end{definition}
\begin{theorem}\label{theo:Density}
	设$(X,\rho)$是一个度量空间,$A,B\subseteq X$。以下四个命题等价:
	\begin{enumerate}
		\item $B$在$A$中稠密。
		\item 对任意的$ x\in A$,$x$的任意邻域中都存在$B$中的点。
		\item 对任意的$ x\in A$,$B$中存在一个点列$\{x_n\}$收敛于$x$。
		\item 对任意的$\varepsilon>0$,$B$中每个点的$\varepsilon$开球邻域的并包含了$A$。
	\end{enumerate}
\end{theorem}
\begin{proof}
	$(1)\to(2)$因为$\overline{B}=B\cup B'$,所以若$x\in A$,则$x\in B$或$x\in B'$,此时显然(2)成立。\par
	$(2)\to(3)$显然。\par
	$(3)\to(4)$若此时(4)不成立,即存在$x\in A$对于某个$\varepsilon$,满足条件$x$不在任何$B$中点的$\varepsilon$开球邻域内,则存在$U(x)$使得$U(x)$中不含$B$中的点,那么(3)也不可能成立,矛盾。\par
	$(4)\to(1)$对任意的$ x\in A$,如果$x\notin \overline{B}$,那么存在$U(x,\delta)$使得$U(x,\delta)$中不含$B$中的点,记$x$与$B$中点之间的最小距离为$\rho$,此时只要取$\varepsilon<\rho$,$x$就不在那些邻域的并集中了,矛盾。由$x$的任意性,$A\subseteq\overline{B}$。
\end{proof}
\subsubsection{稀疏性}
\begin{definition}
	设$(X,\rho)$是一个度量空间,$A$是$X$的一个子集。若$A$在$X$的任何一个非空开集中均不稠密,则称$A$为$X$中的\gls{NowhereDenseSet}。
\end{definition}
下述定理中的邻域既可以是开邻域也可以是闭邻域:
\begin{theorem}
	度量空间$(X,\rho)$的子集$A$为稀疏集的充分必要条件是:对任意的$x\in X$和$x$的任意邻域$U(x)$,存在一个包含于邻域$U(x)$的邻域$U(y)$使得$A\cap U(y)=\varnothing$。
\end{theorem}
\begin{proof}
	\textbf{(1)开邻域:}\par
	充分性:假设此时$A$在$X$中的某非空开集$B$中稠密,则$B\subseteq\overline{A}$。任取$x\in B$和$U(x)\subseteq B$,则$x\in\overline{A}$。如果此时存在包含于$U(x)$的$U(y)$使得条件成立,则$y\notin\overline{A}$,这与$y\in U(y)\subset U(x)\subset B\subset\overline{A}$矛盾。\par
	必要性:若$A$是稀疏集,由\cref{prop:OpenClosedSet}(1)可得$A$在$U(x)$中不稠密。由\cref{theo:Density}(2)可知$\exists\;y\in U(x),\;\exists\;U(y)\subseteq U(x)$使得$A\cap U(y)=\varnothing$。\par
	\textbf{(2)闭邻域:}\par
	充分性:\par
	必要性:\info{有空证明}
\end{proof}
\subsubsection{可分性}
\begin{definition}
	设$(X,\rho)$是一个度量空间,$A\subseteq X$。若存在可列集$B\subset X$使得$B$在$A$中稠密,则称$A$是\gls{separable}。若$X$中存在一个在$X$中稠密的可列子集,则称$X$是可分的度量空间。
\end{definition}
\subsubsection{全有界性}
\begin{definition}
	设$(X,\rho)$是一个度量空间,$A$和$B$都是$X$的子集,$\varepsilon>0$。如果对任意的$x\in A$,都存在$y\in B$,使得$\rho(x,y)<\varepsilon$,则称$B$是$A$的一个$\varepsilon$-网。即:以$B$中的点为中心,$\varepsilon$为半径的所有开邻域的并包含了$A$。
\end{definition}
\begin{definition}
	设$(X,\rho)$是一个度量空间,$A$是$X$的子集。如果对任意的$\varepsilon>0$,$X$中总存在$A$的$\varepsilon$-网,且该$\varepsilon$-网只有有限个点,则称$A$是\gls{TotallyBoundedSet}。
\end{definition}
\begin{property}\label{prop:TotallyBoundedSet}
	设$(X,\rho)$是一个度量空间。全有界集具有如下性质:\par
	(1)任何有限集都是全有界集。\par
	(2)全有界集的子集也是全有界集。\par
	(3)设$A$是一个全有界集,则对任意的$\varepsilon>0$,总存在$A$的一个有限子集成为$A$的一个$\varepsilon$-网。\par
	(4)全有界集有界且可分。
\end{property}
\begin{proof}
	(1)(2)是显然的。\par
	(3)因为$A$是一个全有界集,所以对任意的$\varepsilon>0$,存在一个$A$的$\frac{\varepsilon}{2}$-网$\{x_1,x_2,\dots,x_n\}$。依次取$a_i\in A$使得$a_i\in U(x_i,\frac{\varepsilon}{2}),\;i=1,2,\dots,n$,则$\{\seq{a}{n}\}$即构成$A$的一个$\varepsilon$-网:
	\begin{equation*}
		\forall\;a\in U\left(x_i,\frac{\varepsilon}{2}\right),\;\rho(a,a_i)\leqslant\rho(a,x_i)+\rho(x_i,a_i)<\frac{\varepsilon}{2}+\frac{\varepsilon}{2}=\varepsilon,\;i=1,2,\dots,n
	\end{equation*}\par
	(4)设$A\subseteq X$是一个全有界集,则$A$应有一个$1$-网$\{x_1,x_2,\dots,x_n\}$,于是:
	\begin{equation*}
		\forall\;a\in A,\;\exists\;x_{k}\in\{x_1,x_2,\dots,x_n\},\;\rho(a,x_k)<1
	\end{equation*}
	所以(下式中$x_k$是与点$a$对应的$1$-网中的点):
	\begin{equation*}
		\forall\;a\in A,\;\rho(a,x_1)\leqslant\rho(a,x_k)+\rho(x_k,x_1)<1+\max_{k}\rho(x_k,x_1)
	\end{equation*}
	记$\max\limits_{k}\rho(x_k,x_1)=K$,则对任意的$a\in A,\;a\in U(x_1,1+K)$,所以$A$是有界的。\par
	因为$A$是一个全有界集,所以对任意的$\varepsilon=\frac{1}{n},\;n\in\mathbb{N}^+$,$X$中都存在$A$的一个只含有限个点的$\varepsilon$-网$B_n$。记:
	\begin{equation*}
		B=\underset{n=1}{\overset{+\infty}{\cup}}B_n
	\end{equation*}
	由\info{可列个可列是可列的}可知$B$是可列的。对任意的$x\in A$,存在$x_n\in B_n\subset B$满足对任意的$n\in\mathbb{N}^+$有$\rho(x,x_n)<\frac{1}{n}$,因此$\{x_n\}\to x$。由$x$的任意性和\cref{theo:Density}(3),$B$在$A$中稠密。综上,$A$是可分的。
\end{proof}
\subsubsection{准紧集}
\begin{definition}
	$(X,\rho)$是一个度量空间,$A$是$X$的一个子集。如果$A$的每个点列都有一个子列收敛于$X$中的某一点,则称$A$是\gls{PrecompactSet}。
\end{definition}
\begin{property}\label{prop:PrecompactSet}
	设$(X,\rho)$是一个度量空间。准紧集具有如下性质:
	\begin{enumerate}
		\item 准紧集的子集也是准紧集;
		\item 如果$A\subseteq X$准紧,则$A$全有界。
	\end{enumerate}
\end{property}
\begin{proof}
	(1)由定义立即可得。\par
	(2)如果$A$不是全有界集,则存在$\varepsilon>0$,使得$A$没有只有有限点的$\varepsilon$-网。任取$x_1\in A$,则存在$x_2\in A$使得$\rho(x_1,x_2)\geqslant\varepsilon$,否则$\{x_1\}$就是$A$的一个$\varepsilon$-网。同理,存在$x_3\in A$使得$\rho(x_3,x_j)\geqslant\varepsilon,\;j=1,2$,否则$\{x_1,x_2\}$就是$A$的一个$\varepsilon$-网。重复这一步骤就得到点列$\{x_n\}$,当$m\ne n$时,$\rho(x_m,x_n)\geqslant\varepsilon$。由\cref{prop:CauchySeq}(1)$\{x_n\}$没有收敛的子列,这与$A$的准紧性矛盾。\par
\end{proof}

\subsection{完备的度量空间}
\subsubsection{Cauchy点列}
\begin{definition}
	$(X,\rho)$是一个度量空间,$\{x_n\}$是$X$中的点列。若对任意的$\varepsilon>0$,$\exists\;N\in\mathbb{N}^+$,当$n,m>N$时,有:
	\begin{equation*}
		\rho(x_n,x_m)<\varepsilon
	\end{equation*}
	则称点列$\{x_n\}$是一个\gls{CauchySeq}或\gls{FoundamentalSeq}。
\end{definition}
\begin{property}\label{prop:CauchySeq}
	设$(X,\rho_X)$是一个度量空间。Cauchy点列具有如下性质:
	\begin{enumerate}
		\item 若$\{x_n\}$是$X$中的收敛点列,则$\{x_n\}$是一个Cauchy点列;
		\item 若$\{x_n\}$是$X$中的Cauchy点列,它的一个子列$\{x_{n_k}\}$收敛,则其本身也收敛,并且极限相同;
		\item Cauchy点列是有界的。
	\end{enumerate}
\end{property}
\begin{proof}
	(1)令$n<m$。因为$\{x_n\}$是$X$中的收敛点列,假设其极限为$x$,则对任意的$\varepsilon>0$,$\exists\; N_1\in\mathbb{N}^+$,当$n>N_1$时有$\rho(x_n,x)<\frac{\varepsilon}{2}$;$\exists\; N_2\in\mathbb{N}^+$,当$m>N_2$时有$\rho(x_m,x)<\frac{\varepsilon}{2}$。取$N=\max\{N_1,N_2\}$,则当$n>N$时,有
	\begin{equation*}
		\rho(x_n,x_m)\leqslant\rho(x_n,x)+\rho(x_m,x)<\frac{\varepsilon}{2}+\frac{\varepsilon}{2}=\varepsilon
	\end{equation*}
	即点列$\{x_n\}$是一个Cauchy点列。\par
	(2)设$\{x_{n_k}\}$极限为$x$,则
	\begin{equation*}
		\rho(x_n,x)\leqslant\rho(x_n,x_{n_k})+\rho(x_{n_k},x)
	\end{equation*}
	对任意的$\varepsilon>0$,因为$x_{n_k}$收敛于$x$,所以$\exists\;N_1\in\mathbb{N}^+$,使得当$k>N_1$时,有$\rho(x_{n_k},x)<\frac{\varepsilon}{2}$。又因$\{x_n\}$是$X$中的Cauchy点列,因此$\exists\;N_2\in\mathbb{N}^+$,使得当$n,k>N_2$时,有$\rho(x_n,x_{n_k})<\frac{\varepsilon}{2}$。取$N=\max\{N_1,N_2\}$,当$n,k>N$时,即有
	\begin{equation*}
		\rho(x_n,x)\leqslant\rho(x_n,x_{n_k})+\rho(x_{n_k},x)<\frac{\varepsilon}{2}+\frac{\varepsilon}{2}=\varepsilon
	\end{equation*}\par
	(3)设$(X,\rho)$是一个度量空间,$\{x_n\}$是$X$中的Cauchy点列,则对任意的$\varepsilon>0$,$\exists\;N\in\mathbb{N}^+$,当$n,m>N$时,有:
	\begin{equation*}
		\rho(x_n,x_m)<\varepsilon
	\end{equation*}
	取$m=N+1$,令$\alpha=\max\limits_{i=1,2,\dots,N}\rho(x_m,x_i)$,$\delta=\max\{\varepsilon,\alpha\}$,则$\{x_n\}$中的所有点都在闭邻域$\overline{U}(x_m,\delta)$中,所以$\{x_n\}$有界。
\end{proof}
\subsubsection{完备度量空间}
\begin{definition}
	$(X,\rho)$是一个度量空间。若$X$中的任意Cauchy点列$\{x_n\}$都收敛到$X$中的某一点,则称$X$是一个\gls{complete}度量空间。
\end{definition}
\begin{theorem}[闭球套定理]
	$(X,\rho)$是一个度量空间。$X$完备的充分必要条件为对任何满足下列条件的一列闭邻域$\{E_n=\overline{U}(x_n,\delta_n)\}$:
	\begin{enumerate}
		\item $E_1\supseteq E_2\supseteq\cdots\supseteq E_n\supseteq\cdots$
		\item $\{\delta_n\}\to0$
	\end{enumerate}
	$X$中都存在唯一的$x$满足$x\in\underset{n\in\mathbb{N}^+}{\cap}E_n$。
\end{theorem}
\begin{proof}
	\textbf{充分性:}任取$X$中的一个Cauchy点列$\{x_n\}$。因为$\{x_n\}$是一个Cauchy点列,所以对任意的$\varepsilon>0,\;\exists\;N\in\mathbb{N}^+,\;\forall\;n,m>N,\;\rho(x_m,x_n)<\varepsilon$。取$\{N_k\}$使得$N_k$是使得$\rho(x_m,x_n)<\dfrac{1}{2^k}$的临界条件,取$x_{n_k}>N_k,\;x_{n_k+1}>N_{k+1}$,即可产生一个子列 $\{x_{n_k}\}$,满足:
	\begin{equation*}
		\rho(x_{n_k},x_{n_{k+1}})<\frac{1}{2^k}
	\end{equation*}
	取闭邻域列:
	\begin{equation*}
		\left\{E_k=\bar{U}\left(x_{n_k},\frac{1}{2^{k-1}}\right)\right\}
	\end{equation*}
	于是:
	\begin{equation*}
		\forall\;y\in E_{k+1},\;\rho(y,x_{n_k})\leqslant\rho(y,x_{n_{k+1}})+\rho(x_{n_{k+1}},x_{n_k})<\frac{1}{2^k}+\frac{1}{2^k}=\frac{1}{2^{k-1}}
	\end{equation*}
	所以$E_k\supseteq E_{k+1}$。由条件,此时存在唯一的$x\in X$满足$x\in E_k,\;\forall\;k\in\mathbb{N}^+$。下证$\{x_n\}\to x$:
	\begin{equation*}
		\rho(x_n,x)\leqslant\rho(x_n,x_{n_k})+\rho(x_{n_k},x)\leqslant\rho(x_n,x_{n_k})+\frac{1}{2^{k-1}}
	\end{equation*}
	因为$\{x_n\}$是一个Cauchy点列,因此当$n$和$k$足够大时,上式右端两项均趋于$0$。因此$\{x_n\}\to x$。由$\{x_n\}$的任意性,$X$是一个完备的度量空间。\par
	\textbf{必要性中的存在性:}在每个$E_n$中取一点$y_n$构成点列$\{y_n\}$。设$m>n$,因为$E_m\subseteq E_n$,所以$y_m\in E_n$,于是有:
	\begin{equation*}
		\rho(y_n,y_m)\leqslant2\delta_n\to0
	\end{equation*}
	因此$\{y_n\}$是一个Cauchy点列。因为$X$完备,所以$\{y_n\}\to x\in X$。对任意的$n_0\in\mathbb{N}^+$,当$n>n_0$时,$y_n\in E_n\subseteq E_{n_0}$。由\cref{prop:OpenClosedSet}(2)可得闭邻域$E_{n_0}$是闭集,因此$x\in E_{n_0}$。由$n_0$的任意性,$x\in E_n,\;\forall\;n\in\mathbb{N}^+$。\par
	\textbf{必要性中的唯一性:}若还有一点$y$满足上述条件,则:
	\begin{equation*}
		\rho(x,y)\leqslant\rho(x,y_n)+\rho(y_n,y)\to0
	\end{equation*}
	唯一性得证。
\end{proof}
\begin{definition}
	设$A$是度量空间$(X,\rho)$的子集。若$A$可表示为至多可列个稀疏集的并,则称$A$是\gls{FirstSet}。反之则为\gls{SecondSet}。
\end{definition}
\begin{property}
	设$(X,\rho)$是一个完备的度量空间,则:
	\begin{enumerate}
		\item $(X,\rho)$的子空间$E$是完备度量空间的充要条件为$E$是$X$中的一个闭子空间;
		\item $X$是第二型集;
		\item $A\subseteq X$是全有界集的充要条件为$A$是准紧集;
		\item $A\subseteq X$为准紧集的充分必要条件是对任意的$\varepsilon>0$,$A$都有准紧的$\varepsilon$-网。
	\end{enumerate}
\end{property}
\begin{proof}
	(1)\textbf{必要性:}因为$E$是完备子空间,则对任意的$x\in E'$,存在$E$中的一个收敛点列$\{x_n\}\to x$。由\cref{prop:CauchySeq}(1)和$E$的完备性可知$x\in E$。由$x$的任意性,$E'\subseteq E$,故$E$是一个闭集,即$E$是$X$的一个闭子空间。\par
	\textbf{充分性:}任取$\{x_n\}$为$E$中的一个Cauchy点列,那么它也是$X$中的Cauchy点列,因此存在$x\in X$使得$\{x_n\}\to x$,即$x$是$E$的一个聚点。又因$E$是闭的,所以$x\in E$,即Cauchy点列$\{x_n\}$收敛于$E$中的一点。由$\{x_n\}$的任意性,$E$是完备的度量空间。\par
	(2)假设不成立,即存在完备的度量空间$X$使得$X$是第一型集。也就是说$X=\underset{i=1}{\overset{+\infty}{\cup}}F_i$,其中$F_i,\;\forall\;i\in\mathbb{N}^+$是稀疏集。因为$F_1$是稀疏集,由\cref{theo:Density}(2),存在$x_1\in X$,使得闭邻域$U(x_1,r_1)$中不含$F_1$中的点。对于闭邻域$U(x_1,r_1)$,由于$F_2$是稀疏集,因此存在$x_2\in U(x_1,r_1)$,使得闭邻域$U(x_2,r_2)$中不含$F_2$中的点。如此重复下去,实际上可以取$r_n=\dfrac{1}{n}$,便可以得到一个闭球套:
	\begin{equation*}
		U(x_1,r_1)\supseteq U(x_2,r_2)\supseteq\cdots\supseteq U(x_n,r_n)\supseteq\cdots
	\end{equation*}
	且$\{r_n\}\to0$。由完备度量空间的闭球套定理,存在一个$X$中的点$x\in U(x_n,r_n),\;\forall\;n\in\mathbb{N}^+$。而由闭球套的取法,$x\notin X$,矛盾。\par
	(3)根据\cref{prop:PrecompactSet}(2)可知只需证明必要性。任取$A$中的点列$\{x_n\}$。如果$\{x_n\}$中只有有限个互不相同的元素,则$\{x_n\}$显然有收敛的子列。如果$\{x_n\}$中有无限个互不相同的元素,记这些元素构成的集合为$B_0$。由\cref{prop:TotallyBoundedSet}(2),$B_0$也是全有界集。由\cref{prop:TotallyBoundedSet}(3),$B_0$中存在有限个元素使得以它们为球心、$\dfrac{1}{2}$为半径的开邻域的并包含$B_0$,显然$B_0$中至少存在一个点$y_1$使得以它为半径、$\dfrac{1}{2}$为半径的开邻域包含了无穷多个$A$中的点。记被$y_1$包含的这无穷多个点构成的集合为$B_1$,显然$B_1$的直径小于等于$1$。因为$B_1$是$B_0$的子集,所以$B_1$也是全有界的,重复上述论证,则存在$B_2\subseteq B_1$,使得$B_2$中含有$B_1$无穷多个元素且$B_2$的直径小于等于$\frac{1}{2}$。依次类推,可以得到一系列集合满足如下条件:
	\begin{enumerate}
		\item $B_1\supseteq B_2\supseteq\cdots\supseteq B_n\supseteq\cdots$。
		\item $B_n$的直径小于等于$\dfrac{1}{2^{n-1}}$。
		\item 每个$B_n$中都含有$\{x_n\}$中无限个元素。
	\end{enumerate}
	取$x_{n_k}\in B_k$,便得到$\{x_n\}$的一个子列$\{x_{n_k}\}$。显然$\{x_{n_k}\}$是一个Cauchy点列:
	\begin{equation*}
		\forall\;p>q,\;x_{n_p}\in B_p\subseteq B_q,\;\rho(x_{n_p},x_{n_q})<\frac{1}{2^{q-1}}\to0
	\end{equation*}	
	因为$X$完备,所以$\{x_{n_k}\}$在$X$中收敛。由$\{x_n\}$的任意性,$A$准紧。\par
	(4)\textbf{必要性:}$A$就是它自身的准紧的$\varepsilon$-网。\par
	\textbf{充分性:}若对任意的$\varepsilon>0$,$A$都有准紧的$\varepsilon$-网$B$。因为$B$准紧,同时$X$完备,由\cref{prop:PrecompactSet}(2)可知$B$全有界,即$B$有只有有限个元素的$\varepsilon$-网$C=\{c_1,c_2,\dots,c_n\}$。则:
	\begin{equation*}
		\forall\;a\in A,\;\rho(a,c_i)\leqslant\rho(a,b)+\rho(b,c_i),\;\forall\;i=1,2,\dots,n
	\end{equation*}
	可以选取$c_i$和$b\in B$使得$\rho(a,b)<\varepsilon,\;\rho(b,c_i)<\varepsilon$(先选择$b$,再根据$b$即可选得$c_i$),即对于任意的$a\in A$,存在$c_i\in\{c_1,c_2,\dots,c_n\}$使得$\rho(a,c_i)<2\varepsilon$,于是$\{c_1,c_2,\dots,c_n\}$中的点为中心、$2\varepsilon$为半径构成的$2\varepsilon$-网必然是$A$的一个$2\varepsilon$-网。由$\varepsilon$的任意性,$A$全有界。又因为$X$是完备的,由(3)可知$A$准紧。
\end{proof}