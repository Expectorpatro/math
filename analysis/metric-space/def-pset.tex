\section{度量空间上的点集}
\subsubsection{内、外、界点定义}
\begin{definition}
	$(X,\rho)$是一个度量空间,$E\subseteq X$。若对于点$P\in X$,存在$U(P)\subseteq E$,则称点$P$为$E$的\gls{InteriorP}。
\end{definition}
\begin{definition}
	$(X,\rho)$是一个度量空间,$E\subseteq X$。若点$P\in X$为$E^c$的内点,则称点$P$为$E$的\gls{ExteriorP}。
\end{definition}
\begin{definition}
	$(X,\rho)$是一个度量空间,$E\subseteq X$。若对于点$P\in X$,它的任意邻域中都既有$E$中的点,又有$E^c$中的点,则称点$P$为$E$的\gls{BoundaryP}。
\end{definition}
\subsubsection{聚点与孤立点定义}
\begin{definition}
	$(X,\rho)$是一个度量空间,$E\subseteq X$。若点$P\in X$满足以下任一条件:
	\begin{enumerate}
		\item $P$的任何邻域内都存在无穷个$E$中的点。
		\item $P$的任何邻域内都存在一个$E$中异于$P$的点。
		\item $E$中有一个极限为$P$的点列。
	\end{enumerate}
	则$P$是$E$的一个\gls{LimitP}。
\end{definition}
这三个条件实际上是等价的:
\begin{proof}
	$(1)\to(2)$显然,$(3)\to(1)$显然,下证$(2)\to(3)$。
	取$\delta_n=\frac{1}{n}$,在$U(P,\delta_n)$中至少有一点$P_n$,$P_n\in E$且$P_n\ne P$,显然$\{P_n\}\to P$。
\end{proof}
\begin{definition}
	$(X,\rho)$是一个度量空间,$E\subseteq X$。若点$P\in E$但不是$E$的聚点,即存在$P$的邻域$U(P)$,使得$E\cap U(P)=\{P\}$,则称$P$为$E$的\gls{IsolatedP}。
\end{definition}
\subsubsection{开核、导集、边界、闭包的定义}
\begin{definition}
	$(X,\rho)$是一个度量空间,$E\subseteq X$。$E$的全体内点所组成的集合称为$E$的\gls{interior},记为$\mathring{E}$。
\end{definition}
\begin{definition}
	$(X,\rho)$是一个度量空间,$E\subseteq X$。$E$的全体聚点所组成的集合称为$E$的\gls{DerivedSet},记为$E'$。
\end{definition}
\begin{definition}
	$(X,\rho)$是一个度量空间,$E\subseteq X$。$E$的全体界点所组成的集合称为$E$的\gls{boarder},记为$\partial E$。
\end{definition}
\begin{definition}
	$(X,\rho)$是一个度量空间,$E\subseteq X$。$E\cup E'$称为$E$的\gls{closure},记为$\overline{E}$。
\end{definition}
\subsubsection{开集、闭集的定义}
\begin{definition}
	$(X,\rho)$是一个度量空间,$E\subseteq X$。若$E=\mathring{E}$,则称E是一个\gls{OpenSet}。
\end{definition}
\begin{definition}
	$(X,\rho)$是一个度量空间,$E\subseteq X$。若$E'\subset E$,则称E是一个\gls{ClosedSet}。
\end{definition}
\subsubsection{子空间}
\begin{definition}
	$(X,\rho_X)$是一个度量空间。若$E\subset X$,且在$E$上定义距离$\rho_E$使得$E$中任意点对$x,y$之间的距离$\rho_E(x,y)=\rho_X(x,y)$,则称$E$是$X$的一个\gls{subspace}。若$E$是一个开集,则称$E$是$X$的一个\gls{OpenSubspace};$E$是一个闭集,则称$E$是$X$的一个\gls{ClosedSubspace}。
\end{definition}

\begin{property}\label{prop:OpenClosedSet}
	$(X,\rho)$是一个度量空间,$P\in X,\;E\subseteq X$,则有:
	\begin{enumerate}
		\item $\forall\;P_0\in U(P),\;\exists\;U(P_0)\subset U(P)$,即开邻域是开集。
		\item $\forall\;P_0\in\left(\bar{U}(P)\right)',\;P_0\in \bar{U}(P_0)$,即闭邻域是闭集;
		\item $\mathring{E}$是开集,$E'$和$\overline{E}$是闭集;
		\item $X$和$\varnothing$是唯二既开又闭的集合;
		\item 若$E$是开集,则$E^c$是闭集;若$E$是闭集,则$E^c$是开集;
		\item 任意个开集的并是开集,有限个开集的交是开集;任意个闭集的交是闭集,有限个闭集的并是闭集。
	\end{enumerate}
\end{property}
\begin{proof}
	(1)要使得$U(P_0)\subset U(P)$,则$\forall\;x\in U(P_0),\;\rho(x,P)<\delta_P$,而$\rho(x,P)<\rho(x,P_0)+\rho(P_0,P)<\delta_{P_0}+\rho(P_0,P)$,故只要使$\delta_{P_0}+\rho(P_0,P)<\delta_P$即可。 \par
	(2)对任意的$P_0\in\left(\bar{U}(P)\right)'$,$\exists\;\bar{U}(P)$中的点列$\{P_n\}\to P_0$。若$P_0\notin \bar{U}(P)$,则$\rho(P_0,P)>\delta_P$。取$\varepsilon=\rho(P_0,P)-\delta_P$,则必然$\exists\;N\in\mathbb{N}^+$,当$n>N$时,有$\rho(P_n,P_0)<\varepsilon$,那么$\rho(P_n,P)>\delta_P$,即$P_n\notin U(P)$,矛盾。因此$P_0\in \bar{U}(P)$。由$P_0$的任意性,闭邻域是闭集。\par
	(3)$\mathring{E}$是开集:对任意的$ P\in\mathring{E}$,存在$U(P)\subset E$,对任意的$ Q\in U(P)$,存在$U(Q)\subset U(P)\subset E$,因此$Q\in\mathring{E}$,即存在$U(P)\subset\mathring{E}$,$P$是$\mathring{E}$的内点。由$P$的任意性,$\mathring{E}$是开集。\par
	$E'$是闭集:对任意的$ P\in (E')'$,由聚点定义,任意$U(P)$中都含有$E'$中的点,任取$Q\in U(P)\cap E'$,任意$U(Q)$都含$E$中的点,而$Q\in U(P)$,由邻域的性质,存在$U(Q)\subset U(P)$,因此任意$U(P)$中都含有$E$中的点,即$P$是$E$的聚点,$P\in E'$。由$P$的任意性与$P$的选取方式,$E'$是闭集。\par
	$\overline{E}$是闭集:$\overline{E}=E\cup E'$,由\cref{theo:cuplimitset eq2 limitsetcup},$(\overline{E})'=E'\cup (E')'$,显然$E'\subset \overline{E}$,而$E'$是闭集,$(E')'\subset E'\subset\overline{E}$。因此,$(\overline{E})'\subset\overline{E}$,即$\overline{E}$是闭集。\par
	(4)\par
	(5)$E$是开集:$\forall\; P\in (E^c)'$,应有$P\notin E$,即$P$不会是$E$的内点。否则由内点定义,存在$U(P)$使得$U(P)\subset E$,即$U(P)$中不含$E^c$中的点,这与点$P\in (E^c)'$矛盾。由$P$的任意性有$E'\subset E$,即$E^c$是闭集。\par
	$E$是闭集:$\forall\; P\in E^c$,应有$P$是$E^c$的一个内点。否则任意$U(P)$中都存在$E$中的点,则$P$应该是$E$的一个聚点,而此时$P\in E^c$,与$E$是闭集矛盾。由$P$的任意性有$E^c=\mathring{(E^c)}$,即$E^c$是开集。\par
	(6)\textbf{任意个开集的并是开集:}对于在任意个开集的并集中的点$P$,$P$必属于其中一个开集,因而存在一个邻域$U(P)$使得其中的点都在那个开集中,进而$U(P)$都在任意个开集的并集中,即$P$是任意个开集的并集的内点。由$P$的任意性,任意个开集的并集是开集。\par
	\textbf{有限个开集的交是开集:}设这有限个开集为$A_i,i=1,2,\dots,n$。对任意的$ P\in\underset{i=1}{\overset{n}{\cap}}A_i$,则$P$是所有$A_i$的内点,因此存在$U_i(P)\subset A_i$。由邻域的性质,存在$U(P)$满足$U(P)\subset\underset{i=1}{\overset{n}{\cap}}U_i(P)\subset\underset{i=1}{\overset{n}{\cap}}A_i$,因此$P$是$\underset{i=1}{\overset{n}{\cap}}A_i$的内点。由$P$的任意性,$\underset{i=1}{\overset{n}{\cap}}A_i$是开集。\par
	\textbf{任意个闭集的交是闭集,有限个闭集的并是闭集:}可由(5)以及\info{De-morgan law}在(1)(2)的基础上直接得出。
\end{proof}
\subsubsection{内外界聚孤立五点的性质}
\begin{theorem}
	$(X,\rho)$是一个度量空间,$E\subset X$。$E$的界点不是聚点就是孤立点。
\end{theorem}
\begin{proof}
	若$P$是$E$的一个界点,则它的任意邻域中都既有$E$中的点,又有$E^c$中的点。若它的任意邻域中都只含有有限个$E$中的点,则一定存在一个邻域,使得只有自身属于$E$,那么它是一个孤立点;若它的任意邻域中都含有无限个$E$中的点,由聚点定义,它是一个聚点。
\end{proof}
\subsubsection{关于内外界聚孤立五点的总结}
\begin{equation*}
	\text{度量空间}(X,\rho)\text{中的点与$E$的关系可分为}
	\begin{cases}
		\text{内点} \\
		\text{界点} \\
		\text{外点}    
	\end{cases}\;\text{或}
	\begin{cases}
		\text{聚点}   \\
		\text{孤立点} \\
		\text{外点}    
	\end{cases}
\end{equation*}
\subsubsection{开核、导集、边界、闭包的性质}
\begin{theorem}
	$(X,\rho)$是一个度量空间,$E\subset X$。$(\mathring{E})^c=\overline{E^c}$。
\end{theorem}
\begin{proof}
	$\overline{E^c}=E^c\cup (E^c)'$。\par
	对任意的$ P\in\overline{E^c}\cap\mathring{E}$,应有$P\in(E^c)'$,即$P$的任意邻域中都有$E^c$中的点,而这与$P\in\mathring{E}$矛盾,由$P$的任意性有$\overline{E^c}\cap\mathring{E}=\varnothing$,即$\overline{E^c}\subset(\mathring{E})^c$;\par
	对任意的$ P\in(\mathring{E})^c$,应有$P\in\mathring{(E^c)}\cup\partial E$。若$P\in\mathring{(E^c)}$,显然$P\in\mathring{(E^c)}\subset E^c\subset\overline{E^c}$;若$P\in\partial E$,要么$P\in E$,要么$P\notin E$,如果此时$P\in E$,则由界点定义它必然是$E^c$的聚点,即$P\in \overline{E^c}$;如果此时$P\notin E$,则$P\in E^c$,即$P\in \overline{E^c}$。由$P$的任意性有$(\mathring{E})^c\subset\overline{E^c}$。\par
	综上$(\mathring{E})^c=\overline{E^c}$。
\end{proof}
\begin{theorem}
	$(X,\rho)$是一个度量空间,$E\subset X$。$\partial E = \partial E^c$。
\end{theorem}
\begin{proof}
	由定义直接可得。
\end{proof}
\begin{theorem}
	$(X,\rho)$是一个度量空间,$E\subset X$。$\overline{\overline E}=\overline{E}$。
\end{theorem}
\begin{proof}
	$\overline{\overline E}=\overline{E}\cup (\overline{E})'$,由\cref{prop:OpenClosedSet}(3),$\overline{E}$是闭集,因此$(\overline{E})'\subset\overline{E}$,即$\overline{\overline E}=\overline{E}$。
\end{proof}
\begin{theorem}\label{theo:subsetilc}
	$(X,\rho)$是一个度量空间,$A\cup B\subset X$。若$A\subset B$,则$\mathring{A}\subset\mathring{B}$,$A'\subset B'$,$\overline{A}\subset\overline{B}$。
\end{theorem}
\begin{proof}
	由内点、聚点的定义直接可得。
\end{proof}
\begin{theorem}\label{theo:cuplimitset eq2 limitsetcup}
	$(X,\rho)$是一个度量空间,$A\cup B\subset X$。$(A\cup B)'=A'\cup B'$。
\end{theorem}
\begin{proof}
	$A\subset (A\cup B)$,$B\subset (A\cup B)$,由\cref{theo:subsetilc}可知,$A'\subset(A\cup B)'$,$B'\subset(A\cup B)'$,因此$A'\cup B'\subset(A\cup B)'$。\par
	另一方面,对任意的$ P\in (A\cup B)'$,应有$P\in A'\cup B'$。否则就有$P\notin A'$且$P\notin B'$,因此存在$U_1(P)$和$U_2(P)$使得$U_1(P)$中不含除了$P$以外的$A$中的点、$U_2(P)$中不含除了$P$以外的$B$中的点,由邻域的性质,存在$U_3(P)$使得其内不含除了$P$以外$A\cup B$中的点,而这与$P\in (A\cup B)'$矛盾。由$P$的任意性有$(A\cup B)'\subset A'\cup B'$。\par
	综上$(A\cup B)'=A'\cup B'$。
\end{proof}
\begin{theorem}
	$(X,\rho)$是一个度量空间,$A\cup B\subset X$。$\overline{A\cup B}=\overline{A}\cup \overline{B}$。
\end{theorem}
\begin{proof}
	由\cref{theo:cuplimitset eq2 limitsetcup}:
	\begin{equation*}
		\overline{A}\cup \overline{B}=A\cup A'\cup B\cup B'
		=(A\cup B)\cup(A'\cup B')=(A\cup B)\cup(A\cup B)'
		=\overline{A\cup B}\qedhere
	\end{equation*}
\end{proof}