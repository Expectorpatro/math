\section{度量空间上的点集}
本节首先定义邻域的概念,然后由邻域得到内点、界点、外点、聚点、孤立点的定义,再由这五个点得到开核、导集、边界、闭包的定义,从而进一步讨论开集、闭集以及子空间。
\subsubsection{邻域的定义}
\begin{definition}
	度量空间$(X,\rho)$中的点集
	\begin{equation*}
		\{P\in X:\rho(P,P_0)<\delta\}\qquad (\delta>0)
	\end{equation*}
	被称之为点$P_0$的$\delta$\gls{neighbourhood}\footnote{此处的邻域不再具有$\mathbb{R}^n$中那样的球形几何直观。},记为$U(P_0,\delta)$。
\end{definition}
上式定义的邻域也称为开邻域。若在上式中取小于等于号,则称之为闭邻域,用$\bar{U}(P,\delta)$表示。下面如果不作特殊说明,涉及到邻域时都指开邻域。
\subsubsection{内、外、界点定义}
\begin{definition}
	$(X,\rho)$是一个度量空间,$E\subset X$。若对于点$P\in X$,存在$U(P)\subset E$,则称点$P$为$E$的\gls{InteriorP}。
\end{definition}
\begin{definition}
	$(X,\rho)$是一个度量空间,$E\subset X$。若点$P\in X$为$E^c$的内点,则称点$P$为$E$的\gls{ExteriorP}。
\end{definition}
\begin{definition}
	$(X,\rho)$是一个度量空间,$E\subset X$。若对于点$P\in X$,它的任意邻域中都既有$E$中的点,又有$E^c$中的点,则称点$P$为$E$的\gls{BoundaryP}。
\end{definition}
\subsubsection{聚点与孤立点定义}
\begin{definition}
	$(X,\rho)$是一个度量空间,$E\subset X$。若点$P\in X$满足以下任一条件:
	\begin{enumerate}
		\item $P$的任何邻域内都存在无穷个$E$中的点。
		\item $P$的任何邻域内都存在一个$E$中异于$P$的点。
		\item $E$中有一个极限为$P$的点列(极限的定义参考\cref{def:convergence of range of points})。
	\end{enumerate}
	则$P$是$E$的一个\gls{LimitP}。
\end{definition}
这三个条件实际上是等价的:
\begin{proof}
	$(1)\to(2)$显然,$(3)\to(1)$显然,下证$(2)\to(3)$。
	取$\delta_n=\frac{1}{n}$,在$U(P,\delta_n)$中至少有一点$P_n$,$P_n\in E$且$P_n\ne P$,显然$\{P_n\}\to P$。
\end{proof}
\begin{definition}
	$(X,\rho)$是一个度量空间,$E\subset X$。若点$P\in E$但不是$E$的聚点,即存在$P$的邻域$U(P)$,使得$E\cap U(P)=\{P\}$,则称$P$为$E$的\gls{IsolatedP}。
\end{definition}
\subsubsection{开核、导集、边界、闭包的定义}
\begin{definition}
	$(X,\rho)$是一个度量空间,$E\subset X$。$E$的全体内点所组成的集合称为$E$的\gls{interior},记为$\mathring{E}$。
\end{definition}
\begin{definition}
	$(X,\rho)$是一个度量空间,$E\subset X$。$E$的全体聚点所组成的集合称为$E$的\gls{DerivedSet},记为$E'$。
\end{definition}
\begin{definition}
	$(X,\rho)$是一个度量空间,$E\subset X$。$E$的全体界点所组成的集合称为$E$的\gls{boarder},记为$\partial E$。
\end{definition}
\begin{definition}
	$(X,\rho)$是一个度量空间,$E\subset X$。$E\cup E'$称为$E$的\gls{closure},记为$\overline{E}$。
\end{definition}
\subsubsection{开集、闭集的定义}
\begin{definition}
	$(X,\rho)$是一个度量空间,$E\subset X$。若$E=\mathring{E}$,则称E是一个\gls{OpenSet}。
\end{definition}
\begin{definition}
	$(X,\rho)$是一个度量空间,$E\subset X$。若$E'\subset E$,则称E是一个\gls{ClosedSet}。
\end{definition}
\subsubsection{子空间}
\begin{definition}
	$(X,\rho_X)$是一个度量空间。若$E\subset X$,且在$E$上定义距离$\rho_E$使得$E$中任意点对$x,y$之间的距离$\rho_E(x,y)=\rho_X(x,y)$,则称$E$是$X$的一个\gls{subspace}。
\end{definition}
\begin{definition}
	若$E$是度量空间$(X,\rho)$的一个子空间,且$E$是一个开集,则称$E$是$X$的一个\gls{OpenSubspace}。
\end{definition}
\begin{definition}
	若$E$是度量空间$(X,\rho)$的一个子空间,且$E$是一个闭集,则称$E$是$X$的一个\gls{ClosedSubspace}。
\end{definition}