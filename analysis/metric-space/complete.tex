\section{完备的度量空间}
\subsection{Cauchy点列}
\begin{definition}
	$(X,\rho)$是一个度量空间,$\{x_n\}$是$X$中的点列。若对任意的$\varepsilon>0$,$\exists\;N\in\mathbb{N}^+$,当$n,m>N$时,有:
	\begin{equation*}
		\rho(x_n,x_m)<\varepsilon
	\end{equation*}
	则称点列$\{x_n\}$是一个\gls{CauchySeq}或\gls{FoundamentalSeq}。
\end{definition}
\subsubsection{Cauchy点列的性质}
\begin{theorem}
	$(X,\rho_X)$是一个度量空间,$\{x_n\}$是$X$中的收敛点列,则$\{x_n\}$是一个Cauchy点列。
\end{theorem}
\begin{proof}
	令$n<m$。因为$\{x_n\}$是$X$中的收敛点列,假设其极限为$x$,则对任意的$\varepsilon>0$,$\exists\; N_1\in\mathbb{N}^+$,当$n>N_1$时有$\rho(x_n,x)<\frac{\varepsilon}{2}$;$\exists\; N_2\in\mathbb{N}^+$,当$m>N_2$时有$\rho(x_m,x)<\frac{\varepsilon}{2}$。取$N=max\{N_1,N_2\}$,则当$n>N$时,有
	\begin{equation*}
		\rho(x_n,x_m)\leqslant\rho(x_n,x)+\rho(x_m,x)<\frac{\varepsilon}{2}+\frac{\varepsilon}{2}=\varepsilon
	\end{equation*}
	即点列$\{x_n\}$是一个Cauchy点列。
\end{proof}
该定理的逆命题不正确,考虑有理数集按绝对值距离构成的度量空间中的Cauchy点列。
\begin{theorem}
	$(X,\rho)$是一个度量空间,$\{x_n\}$是$X$中的Cauchy点列。若它的一个子列$\{x_{n_k}\}$收敛,则其本身也收敛,并且极限相同。
\end{theorem}
\begin{proof}
	设$\{x_{n_k}\}$极限为$x$,则
	\begin{equation*}
		\rho(x_n,x)\leqslant\rho(x_n,x_{n_k})+\rho(x_{n_k},x)
	\end{equation*}
	对任意的$\varepsilon>0$,因为$x_{n_k}$收敛于$x$,所以$\exists\;N_1\in\mathbb{N}^+$,使得当$k>N_1$时,有$\rho(x_{n_k},x)<\frac{\varepsilon}{2}$。又因$\{x_n\}$是$X$中的Cauchy点列,因此$\exists\;N_2\in\mathbb{N}^+$,使得当$n,k>N_2$时,有$\rho(x_n,x_{n_k})<\frac{\varepsilon}{2}$。取$N=max\{N_1,N_2\}$,当$n,k>N$时,即有
	\begin{equation*}
		\rho(x_n,x)\leqslant\rho(x_n,x_{n_k})+\rho(x_{n_k},x)<\frac{\varepsilon}{2}+\frac{\varepsilon}{2}=\varepsilon\qedhere
	\end{equation*}
\end{proof}
\begin{theorem}
	度量空间中的任何Cauchy点列都是有界的。
\end{theorem}
\begin{proof}
	设$(X,\rho)$是一个度量空间,$\{x_n\}$是$X$中的Cauchy点列,则对任意的$\varepsilon>0$,$\exists\;N\in\mathbb{N}^+$,当$n,m>N$时,有:
	\begin{equation*}
		\rho(x_n,x_m)<\varepsilon
	\end{equation*}
	取$m=N+1$,令$\alpha=\max\limits_{i=1,2,\dots,N}\rho(x_m,x_i)$,$\delta=max\{\varepsilon,\alpha\}$,则$\{x_n\}$中的所有点都在闭邻域$\overline{U}(x_m,\delta)$中。由\cref{def:BoundedSet2},$\{x_n\}$有界。
\end{proof}

\subsection{完备度量空间的定义}
\begin{definition}
	$(X,\rho)$是一个度量空间。若$X$中的任意Cauchy点列$\{x_n\}$都收敛到$X$中的某一点,则称$X$是一个\gls{complete}度量空间。
\end{definition}
\subsubsection{完备度量空间的等价定义}
下列定理是Cantor's Intersection Theorem的一个变体:
\begin{theorem}[闭球套定理]
	$(X,\rho)$是一个度量空间。$X$完备的充分必要条件为对任何满足下列条件的一列闭邻域$\{E_n=\overline{U}(x_n,\delta_n)\}$:
	\begin{enumerate}
		\item $E_1\supset E_2\supset\cdots\supset E_n\supset\cdots$
		\item $\{\delta_n\}\to0$
	\end{enumerate}
	$X$中都存在唯一的$x$满足$x\in\underset{n\in\mathbb{N}^+}{\cap}E_n$。
\end{theorem}
\begin{proof}
	充分性:任取$X$中的一个Cauchy点列$\{x_n\}$。因为$\{x_n\}$是一个Cauchy点列,所以对任意的$\varepsilon>0,\;\exists\;N\in\mathbb{N}^+,\;\forall\;n,m>N,\;\rho(x_m,x_n)<\varepsilon$。取$\{N_k\}$使得$N_k$是使得$\rho(x_m,x_n)<\dfrac{1}{2^k}$的临界条件,取$x_{n_k}>N_k,\;x_{n_k+1}>N_{k+1}$,即可产生一个子列 $\{x_{n_k}\}$,满足:
	\begin{equation*}
		\rho(x_{n_k},x_{n_{k+1}})<\frac{1}{2^k}
	\end{equation*}
	取闭邻域列:
	\begin{equation*}
		\left\{E_k=\bar{U}\left(x_{n_k},\frac{1}{2^{k-1}}\right)\right\}
	\end{equation*}
	显然:
	\begin{equation*}
		\forall\;y\in E_{k+1},\;\rho(y,x_{n_k})\leqslant\rho(y,x_{n_{k+1}})+\rho(x_{n_{k+1}},x_{n_k})<\frac{1}{2^k}+\frac{1}{2^k}=\frac{1}{2^{k-1}}
	\end{equation*}
	所以$E_k\supset E_{k+1}$。由题目条件,此时存在唯一的$x\in X$满足$x\in E_k,\;\forall\;k\in\mathbb{N}^+$。下证$\{x_n\}\to x$。
	\begin{equation*}
		\rho(x_n,x)\leqslant\rho(x_n,x_{n_k})+\rho(x_{n_k},x)\leqslant\rho(x_n,x_{n_k})+\frac{1}{2^{k-1}}
	\end{equation*}
	因为$\{x_n\}$是一个Cauchy点列,因此当$n$和$k$足够大时,上式右端两项均趋于$0$。因此$\{x_n\}\to x$。由$\{x_n\}$的任意性,$X$是一个完备的度量空间。\par
	必要性中的存在性:在每个$E_n$中取一点$y_n$构成点列$\{y_n\}$。设$m>n$,因为$E_m\subset E_n$,所以$y_m\in E_n$,于是有:
	\begin{equation*}
		\rho(y_n,y_m)\leqslant\delta_n\to0
	\end{equation*}
	因此$\{y_n\}$是一个Cauchy点列。因为$X$完备,所以$\{y_n\}\to x\in X$。对任意的$n_0\in\mathbb{N}^+$,当$n>n_0$时,$y_n\in E_n\subset E_{n_0}$。又因为闭邻域$E_{n_0}$是闭集,因此$x\in E_{n_0}$。由$n_0$的任意性,$x\in E_n,\;\forall\;n\in\mathbb{N}^+$。\par
	必要性中的唯一性:若还有一点$y$满足上述条件,则:
	\begin{equation*}
		\rho(x,y)\leqslant\rho(x,y_n)+\rho(y_n,y)\to0
	\end{equation*}
	唯一性显然得证。
\end{proof}

\subsection{度量空间的完备化定理}
\begin{theorem}
	$(X,\rho_X)$是一个度量空间,则一定存在一个完备度量空间$(Y,\rho_Y)$,使得$X$与$Y$的一个稠密子空间等距同构,并且$Y$在等距同构的意义下是唯一的\footnote{这里的唯一性是指,如果存在另一个完备度量空间$(Z,\rho_Z)$使得$X$与$Z$的一个稠密子空间等距同构,则$Y$与$Z$等距同构。}。
\end{theorem}
证明太复杂,不提供。



