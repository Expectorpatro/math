\section{度量空间上的收敛点列}
\begin{definition}\label{def:convergence of range of points}
	$\{x_n\}$是度量空间$(X,\rho)$中的一个\gls{SeqOfPoints},如果对任意的$\varepsilon>0$,$\exists\; N\in\mathbb{N}^+$,当$n>N$时有:
	\begin{equation*}
		\rho(x_n,x)<\varepsilon
	\end{equation*}
	则称$\{x_n\}$是度量空间$(X,\rho)$中的\gls{ConvergentSeqOfPoints},$x$是点列$\{x_n\}$的\gls{limit}。
\end{definition}
\subsection*{收敛点列的性质}
\begin{property}
	设$\{x_n\}$是度量空间$(X,\rho)$中的一个收敛点列,则:
	\begin{enumerate}
		\item $\{x_n\}$的极限是唯一的;
		\item 对任意的$ y\in X$,数列$\{\rho(x_n,y)\}$有界;
		\item $\{x_n\}$是有界点集;
	\end{enumerate}
\end{property}
\begin{proof}
	(1)假设极限不唯一,$\{x_n\}$既收敛到$a$又收敛到$b$,则对任意的$\varepsilon>0$,$\exists\; N_1\in\mathbb{N}^+$,当$n>N_1$时有$\rho(x_n,a)<\frac{\varepsilon}{2}$;$\exists\; N_2\in\mathbb{N}^+$,当$n>N_2$时有$\rho(x_n,b)<\frac{\varepsilon}{2}$。取$N=\max\{N1,N2\}$,则当$n>N$时,有
	\begin{equation*}
		\rho(a,b)\leqslant\rho(a,x_n)+\rho(x_n,b)<\varepsilon
	\end{equation*}
	即$a=b$。\par
	(2)设$\{x_n\}\to x$,由距离的定义:
	\begin{equation*}
		\rho(x_n,y)\leqslant\rho(x_n,x)+\rho(x,y)
	\end{equation*}
	由于$\{x_n\}$收敛,所以对$\varepsilon=1$:
	\begin{equation*}
		\exists\;N\in\mathbb{N}^+,\;\forall\;n>N,\;\rho(x_n,x)<\varepsilon=1
	\end{equation*}
	取$K=\max\{\rho(x_1,x),\rho(x_2,x),\dots,\rho(x_N,x),1\}$,则有:
	\begin{equation*}
		\forall\;n\in\mathbb{N}^+,\;\rho(x_n,y)\leqslant K+\rho(x,y)
	\end{equation*}
	即数列$\{\rho(x_n,y)\}$有界。\par
	(3)任取$y\in X$,由(2)数列$\{\rho(x_n,y)\}$有界,即$\exists\;\delta>0,\;\forall\;n\in\mathbb{N}^+,\;\rho(x_n,y)<\delta$,则$\{x_n\}\subset U(y,\delta)$。
\end{proof}
\subsubsection{子列的收敛性}
\begin{definition}
	$\{x_n\}$是度量空间$(X,\rho)$中的一个点列,而
	\begin{equation*}
		n_1<n_2<\cdots<n_k<n_{k+1}<\cdots
	\end{equation*}
	是一串严格递增的自然数,则
	\begin{equation*}
		x_{n_1}<x_{n_2}<\cdots<x_{n_k}<x_{n_{k+1}}<\cdots
	\end{equation*}
	也形成一个$(X,\rho)$中的点列,我们把$\{x_{n_k}\}$称之为点列$\{x_n\}$的一个\gls{subsequence}。
\end{definition}
\begin{theorem}
	$\{x_n\}$是度量空间$(X,\rho)$中的一个收敛点列,则它的任何子列$\{x_{n_k}\}$都收敛,并且与$\{x_n\}$有同样的极限。反之,若$\{x_n\}$的任何子列都收敛,则$\{x_n\}$本身也收敛。
\end{theorem}
\begin{proof}
	(1)设$\{x_n\}$的极限为$x$,则对任意的$\varepsilon>0$,$\exists\; N\in\mathbb{N}^+$,当$n>N$时有
	\begin{equation*}
		\rho(x_n,x)<\varepsilon
	\end{equation*}
	当$k>N$时就有$n_k\geqslant k>N$,也就有
	\begin{equation*}
		\rho(x_{n_k},x)<\varepsilon
	\end{equation*}\par
	(2)$\{x_n\}$也是它自己的一个子列。
\end{proof}