\section{度量空间上的映射}
\subsection{一般映射的定义与性质}
\begin{definition}
	设$X,Y$是任意给定的集合。如果对于任意的$x\in X$,都存在唯一地$f(x)\in Y$与之对应,则称对应关系$f$是一个从$X$到$Y$的\gls{mapping}。对任何$E\subset Y$,称:
	\begin{equation*}
		f^{-1}(E)=\{x:f(x)\in E\}
	\end{equation*}
	为集合$B$在映射$f$下的\gls{preimage}\footnote{原像与逆无关。}。
\end{definition}
\begin{theorem}\label{theo:PropertyOfPreimage}
	设$X,Y$是任意给定的集合,$f$是一个从$X$到$Y$的映射。集合的原像有下列性质:
	\begin{enumerate}
		\item $f^{-1}(\varnothing)=\varnothing,\;f^{-1}(Y)=X$;
		\item 若$E_1\subset E_2\subset Y$,则$f^{-1}(E_1)\subset f^{-1}(E_2)$;
		\item 对任意的$E\subset Y$,$[f^{-1}(E)]^c=f^{-1}(E^c)$;
		\item 设$T$是一个指标集,对$\{A_t\in\ Y:t\in T\}$,有:
		\begin{equation*}
			f^{-1}\left(\underset{t\in T}{\bigcup}A_t\right)=\underset{t\in T}{\bigcup}f^{-1}(A_t), \quad
			f^{-1}\left(\underset{t\in T}{\bigcap}A_t\right)=\underset{t\in T}{\bigcap}f^{-1}(A_t)
		\end{equation*}
	\end{enumerate}
\end{theorem}
\begin{proof}
	(1)(2)是显然的,下证(3)(4)。\par
	(3)对任意的$x\in[f^{-1}(E)]^c$,有$x\notin f^{-1}(E)$,即$f(x)\notin E$,所以$x\in f^{-1}(E^c)$。由$x$的任意性,$[f^{-1}(E)]^c\subset f^{-1}(E^c)$。对任意的$x\in f^{-1}(E^c)$,有$f(x)\in E^c$,所以$x\notin f^{-1}(E)$,于是$x\in[f^{-1}(E)]^c$。由$x$的任意性,$f^{-1}(E^c)\subset[f^{-1}(E)]^c$。综上,$[f^{-1}(E)]^c=f^{-1}(E^c)$。\par
	(4)对任意的$x\in f^{-1}\left(\underset{t\in T}{\cup}A_t\right)$,有$f(x)\in\underset{t\in T}{\cup}A_t$,即存在$t\in T$,使得$f(x)\in A_t,\;x\in f^{-1}(A_t)$,于是$x\in\underset{t\in T}{\cup}f^{-1}(A_t)$。由$x$的任意性,$f^{-1}\left(\underset{t\in T}{\cup}A_t\right)\subset\underset{t\in T}{\cup}f^{-1}(A_t)$。对任意的$x\in\underset{t\in T}{\cup}f^{-1}(A_t)$,则存在$t\in T$,使得$x\in f^{-1}(A_t)$,于是$x\in f^{-1}\left(\underset{t\in T}{\cup}A_t\right)$。由$x$的任意性,$\underset{t\in T}{\cup}f^{-1}(A_t)\subset f^{-1}\left(\underset{t\in T}{\cup}A_t\right)$。综上,$f^{-1}\left(\underset{t\in T}{\cup}A_t\right)=\underset{t\in T}{\cup}f^{-1}(A_t)$。交的情形同理可证。
\end{proof}
\subsection{度量空间上的映射}
\begin{definition}
	$(X,\rho_X)$和$(Y,\rho_Y)$都是度量空间。若对任意的$ x\in X$,都$\exists\;y\in Y$与之对应,则称这个对应是一个$X$到$Y$的映射,用符号$T$表示。称集合
	\begin{equation*}
		\{x\in X:Tx\in E\subset Y\}
	\end{equation*}
	为集合$E$在映射$T$下的原像。
\end{definition}
\subsubsection{单射、满射、双射}
\begin{definition}
	$(X,\rho_X)$和$(Y,\rho_Y)$都是度量空间,$T$是一个$X$到$Y$的映射。如果对任意的$y\in Tx$,只有唯一的$x$使得$Tx=y$,那么称映射$T$为\gls{InjectiveF}。
\end{definition}
\begin{definition}
	$(X,\rho_X)$和$(Y,\rho_Y)$都是度量空间,$T$是一个$X$到$Y$的映射。如果对任意的$y\in Y$,都存在$X$中的$x$使得$Tx=y$,那么称映射$T$为\gls{SurjectiveF}。
\end{definition}
\begin{definition}
	$(X,\rho_X)$和$(Y,\rho_Y)$都是度量空间,$T$是一个$X$到$Y$的映射。如果$T$既是单射,又是满射,则称之为\gls{BijectiveF}。
\end{definition}
\subsubsection{逆映射}
\begin{definition}
	$(X,\rho_X)$和$(Y,\rho_Y)$都是度量空间,$T$是一个$X$到$Y$的双射。则对任意的$y\in Y$,都存在唯一的$x\in X$使得$Tx=y$,此时可以得到一个新的映射,它将$Y$映成$X$,称这个映射为$T$的\gls{InverseMap}。
\end{definition}
\begin{theorem}
	$(X,\rho_X)$和$(Y,\rho_Y)$都是度量空间,$T$是一个$X$到$Y$的映射。$T$存在逆映射的充要条件是$T$是一个双射。
\end{theorem}
\subsubsection{复合映射}
\begin{definition}
	$(X,\rho_X)$、$(Y,\rho_Y)$和$(Z,\rho_z)$都是度量空间,$T_1$是一个$X$到$Y$的映射,$T_2$是一个$Y$到$Z$的映射。定义映射$T_3$满足$T_3x=T_2(T_1x)$,其中$x\in X$,则称映射$T_3$是映射$T_1$和映射$T_2$的\gls{CompositeMap}。
\end{definition}
复合映射的概念也可以推广到多个映射的复合。
\subsubsection{等距映射}
\begin{definition}
	$(X,\rho_X)$和$(Y,\rho_Y)$都是度量空间,$T$是一个$X$到$Y$的双射。若对任意的$x,y\in X$,有$\rho_X(x,y)=\rho_Y(Tx,Ty)$,则称$T$是$X$到$Y$上的\gls{isometry}。如果存在一个$X$到$Y$上的等距映射,则称$X$和$Y$\gls{isometric}。
\end{definition}
\subsection{映射的连续性}
\begin{definition}
	$(X,\rho_X)$和$(Y,\rho_Y)$都是度量空间,$T$是一个$X$到$Y$的映射,$x_0\in X$。若对任意的$\varepsilon>0$,$\exists\;\delta>0$,使得对$X$中一切满足条件$\rho(x_0,x)<\delta$的$x$,都有$\rho_Y(Tx_0,Tx)<\varepsilon$,则称$T$在$x_0$处是\gls{continuous}。
\end{definition}
\subsubsection{邻域式定义}
\begin{definition}
	$T$在$x_0$处连续:对$Tx_0$的任意$\varepsilon$邻域$U$,必有$x_0$的某个$\delta$邻域$V$使得$TV\subset U$。
\end{definition}
\subsubsection{点列式定义}
\begin{theorem}
	$(X,\rho_X)$和$(Y,\rho_Y)$都是度量空间,$T$是一个$X$到$Y$的映射,那么$T$在$x_0\in X$处连续的充分必要条件为:当$x_n\to x_0$时,$Tx_n\to Tx_0$。
\end{theorem}
\begin{proof}
	必要性显然。\par
	充分性:若此时$T$在$x_0$不连续,那么$\exists\;\varepsilon>0$,$\forall\;\delta>0$,都存在满足条件$\rho_X(x_0,x)<\delta$的点$x$使得$\rho_Y(Tx_0,Tx)\geqslant\varepsilon$,因此可取一个点列$\{x_n\}$,满足$\rho_X(x_0,x_n)<\frac{1}{n}$。注意到此时满足$x_n\to x$但不满足$Tx_n\to Tx_0$,矛盾。
\end{proof}
\subsubsection{连续映射的定义}
\begin{definition}
	$T$是一个$(X,\rho_X)$到$(Y,\rho_Y)$的映射,若$T$在$X$的每一点都连续,则称$T$是$X$上的\gls{ContinuousMap}。
\end{definition}
\subsubsection{连续映射的拓扑式等价定义}
\begin{theorem}
	度量空间$(X,\rho_X)$到$(Y,\rho_Y)$上的映射$T$是$X$上连续映射的充要条件为:
	\begin{enumerate}
		\item $Y$中任意开集$E$的原像$T^{-1}E$是$X$中的开集。
		\item $Y$中任意闭集$E$的原像$T^{-1}E$是$X$中的闭集。
	\end{enumerate}
\end{theorem}
\begin{proof}
	(1)必要性:设$T$是连续映射,$M\subset Y$是一个开集,如果$T^{-1}M=\varnothing$,则$T^{-1}M$是开集;若$T^{-1}M\ne\varnothing$,任取$x\in T^{-1}M$,令$Tx=y$,则$y\in M$,因为$M$是开集,所以存在$y$的$\varepsilon$邻域$U$,使得$U\subset M$,由$T$的连续性,存在$x$的$\delta$邻域$V$,使得$TV\subset U$,因此$V\subset T^{-1}U\subset T^{-1}M$,即$x$是$T^{-1}M$的内点。由$x$的任意性,$T^{-1}M$是$X$中的开集。\par
	充分性:对任意的$x\in X$及$Tx$的任意$\varepsilon$邻域$U$,由邻域的性质,$U$是一个开集,因此$T^{-1}U$是$X$中的开集。因为$x$是$T^{-1}U$的内点,所以存在$x$的某个$\delta$邻域$V$,使得$V\subset T^{-1}U$,因此$TV\subset U$,即$T$在$x$处连续。由$x$的任意性,$T$是$X$上的连续映射。\par
	(2)由\cref{theo:PropertyOfPreimage}(3),利用(1)易证(2)。
\end{proof}
\subsubsection{同胚映射}
\begin{definition}
	$(X,\rho_X)$和$(Y,\rho_Y)$都是度量空间,$T$是一个$X$到$Y$的双射。若$T$和$T^{-1}$都是连续映射,则称$T$是$X$到$Y$上的\gls{HomeoMap}。如果存在一个$X$到$Y$上的同胚映射,则称$X$和$Y$\gls{homeomorphic}。
\end{definition}