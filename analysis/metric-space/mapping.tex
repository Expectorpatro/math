\section{度量空间上的映射}
\subsection{映射的定义与性质}
\begin{definition}
	设$X,Y$是任意给定的集合。如果对于任意的$x\in X$,都存在唯一的$f(x)\in Y$与之对应,则称对应关系$f$是一个从$X$到$Y$的\gls{mapping}。对任何$E\subseteq Y$,称:
	\begin{equation*}
		f^{-1}(E)=\{x:f(x)\in E\}
	\end{equation*}
	为集合$B$在映射$f$下的\gls{preimage}\footnote{原像与逆无关。}。
\end{definition}
\begin{theorem}\label{theo:PropertyOfPreimage}
	设$X,Y$是任意给定的集合,$f$是一个从$X$到$Y$的映射。集合的原像有下列性质:
	\begin{enumerate}
		\item $f^{-1}(\varnothing)=\varnothing,\;f^{-1}(Y)=X$;
		\item 若$E_1\subseteq E_2\subseteq Y$,则$f^{-1}(E_1)\subseteq f^{-1}(E_2)$;
		\item 对任意的$E\subset Y$,$[f^{-1}(E)]^c=f^{-1}(E^c)$;
		\item 设$T$是一个指标集,对$\{A_t\in\ Y:t\in T\}$,有:
		\begin{equation*}
			f^{-1}\left(\underset{t\in T}{\bigcup}A_t\right)=\underset{t\in T}{\bigcup}f^{-1}(A_t), \quad
			f^{-1}\left(\underset{t\in T}{\bigcap}A_t\right)=\underset{t\in T}{\bigcap}f^{-1}(A_t)
		\end{equation*}
	\end{enumerate}
\end{theorem}
\begin{proof}
	(1)(2)是显然的,下证(3)(4)。\par
	(3)对任意的$x\in[f^{-1}(E)]^c$,有$x\notin f^{-1}(E)$,即$f(x)\notin E$,所以$x\in f^{-1}(E^c)$。由$x$的任意性,$[f^{-1}(E)]^c\subseteq f^{-1}(E^c)$。对任意的$x\in f^{-1}(E^c)$,有$f(x)\in E^c$,所以$x\notin f^{-1}(E)$,于是$x\in[f^{-1}(E)]^c$。由$x$的任意性,$f^{-1}(E^c)\subseteq[f^{-1}(E)]^c$。综上,$[f^{-1}(E)]^c=f^{-1}(E^c)$。\par
	(4)对任意的$x\in f^{-1}\left(\underset{t\in T}{\cup}A_t\right)$,有$f(x)\in\underset{t\in T}{\cup}A_t$,即存在$t\in T$,使得$f(x)\in A_t,\;x\in f^{-1}(A_t)$,于是$x\in\underset{t\in T}{\cup}f^{-1}(A_t)$。由$x$的任意性,$f^{-1}\left(\underset{t\in T}{\cup}A_t\right)\subseteq\underset{t\in T}{\cup}f^{-1}(A_t)$。对任意的$x\in\underset{t\in T}{\cup}f^{-1}(A_t)$,则存在$t\in T$,使得$x\in f^{-1}(A_t)$,于是$x\in f^{-1}\left(\underset{t\in T}{\cup}A_t\right)$。由$x$的任意性,$\underset{t\in T}{\cup}f^{-1}(A_t)\subseteq f^{-1}\left(\underset{t\in T}{\cup}A_t\right)$。综上,$f^{-1}\left(\underset{t\in T}{\cup}A_t\right)=\underset{t\in T}{\cup}f^{-1}(A_t)$。交的情形同理可证。
\end{proof}
\begin{definition}
	设$T$是一个$X$到$Y$的映射。如果对任意的$y\in Tx$,只有唯一的$x$使得$Tx=y$,那么称映射$T$为\gls{InjectiveF}。
\end{definition}
\begin{definition}
	设$T$是一个$X$到$Y$的映射。如果对任意的$y\in Y$,都存在$X$中的$x$使得$Tx=y$,那么称映射$T$为\gls{SurjectiveF}。
\end{definition}
\begin{definition}
	设$T$是一个$X$到$Y$的映射。如果$T$既是单射,又是满射,则称之为\gls{BijectiveF}。
\end{definition}
\begin{definition}
	设$T$是一个$X$到$Y$的双射,即对任意的$y\in Y$,都存在唯一的$x\in X$使得$Tx=y$,此时可以得到一个新的映射,它将$Y$映成$X$,称这个映射为$T$的\gls{InverseMap}。
\end{definition}
\begin{theorem}
	设$T$是一个$X$到$Y$的映射。$T$存在逆映射的充要条件是$T$是一个双射。
\end{theorem}
\begin{definition}
	设$T_1$是一个$X$到$Y$的映射,$T_2$是一个$Y$到$Z$的映射。定义映射$T_3$满足$T_3x=T_2(T_1x)$,其中$x\in X$,则称映射$T_3$是映射$T_1$和映射$T_2$的\gls{CompositeMap},记作$T_2\circ T_1$或$T_2T_1$。
\end{definition}

\subsection{映射的极限}
\begin{definition}
	设$(X,\rho_X)$和$(Y,\rho_Y)$都是度量空间,$T$是一个$E\subseteq X$到$Y$的映射,$x_0\in X$是$E$的一个聚点,$y\in Y$。若对任意的$\varepsilon>0$,存在$\delta>0$,使得对$E\setminus\{x_0\}$中一切满足条件$\rho(x,x_0)<\delta$的$x$,都有$\rho_Y(Tx,y)<\varepsilon$,则称$T$沿$E$趋于$x_0$时,$T$的极限为$y$,记为:
	\begin{equation*}
		\lim_{\substack{x\to x_0 \\ E}}Tx=y
	\end{equation*}
	当$E$自明时,可简记为:
	\begin{equation*}
		\lim_{x\to x_0}Tx=y
	\end{equation*}
\end{definition}
\begin{theorem}[Sequential Characterization of Function Limits]\label{theo:SequentialCharacterizationOfFunctionLimits}
	设$(X,\rho_X)$和$(Y,\rho_Y)$都是度量空间,$T$是一个$E\subseteq X$到$Y$的映射,$x_0\in X$是$E$的一个聚点,$y\in Y$。$T$沿$E$趋于$x_0$时$T$的极限为$y$的充分必要条件为:对于任意满足$\{x_n\}\to x_0$的点列$\{x_n\}\subseteq E\setminus\{x_0\}$都有$\{Tx_n\}\to y$。
\end{theorem}
\begin{proof}
	\textbf{必要性:}若存在满足$\{x_n\}\to x_0$的点列$\{x_n\}\subseteq E\setminus\{x_0\}$不满足$\{Tx_n\}\to y$,则存在$\varepsilon>0$对任意的$N_1\in\mathbb{N}^+$都存在$n>N_1$使得$\rho_Y(Tx_n,y)\geqslant\varepsilon$。取定一个满足上述条件的$\varepsilon$,因为$T$沿$E$趋于$x_0$时极限为$y$,所以存在$\delta>0$,当$\rho_X(x,x_0)<\delta$时($x\in E\setminus\{x_0\}$)有$\rho_Y(Tx,y)<\varepsilon$。因为$\{x_n\}\to x_0$,所以存在$N_2\in\mathbb{N}^+$满足当$n>N_2$时有$\rho_X(x_n,x_0)<\delta$,此时就有$\rho_Y(Tx_n,y)<\varepsilon$。取$N_1=N_2$,矛盾,所以必要性成立。\par
	\textbf{充分性:}若此时$T$沿$E$趋于$x_0$时极限不为$y$,那么存在$\varepsilon>0$对任意的$\delta>0$都存在满足条件$\rho_X(x,x_0)<\delta$的点$x\in E\setminus\{x_0\}$使得$\rho_Y(Tx,Tx_0)\geqslant\varepsilon$,因此可取一个点列$\{x_n\}$,满足$\rho_X(x_n,x_0)<\dfrac{1}{n}$。注意到此时满足$\{x_n\}\to x$但不满足$\{Tx_n\}\to y$,矛盾。
\end{proof}
\begin{theorem}[Cauchy-Type Condition for Function Limits]\label{theo:Cauchy-TypeConditionForFunctionLimits}
	设$(X,\rho_X)$是度量空间,$(Y,\rho_Y)$是完备的度量空间,$T$是一个$E\subseteq X$到$Y$的映射,$x_0\in X$是$E$的一个聚点。$T$沿$E$趋于$x_0$时存在极限的充要条件为:对任意的$\varepsilon>0$,存在$\delta>0$使得对于任意满足条件$\rho_X(x,x_0)<\delta,\rho_X(x',x_0)<\delta$的$x,x'\in E\setminus\{x_0\}$有$\rho_Y(Tx,Tx')<\varepsilon$。
\end{theorem}
\begin{proof}
	\textbf{必要性:}设$T$沿$E$趋于$x_0$时极限为$y\in Y$,则对于任意的$\varepsilon>0$,存在$\delta_1$使得当$\rho_X(x,x_0)<\delta_1$时有$\rho_Y(Tx,y)<\dfrac{\varepsilon}{2}$,存在$\delta_1$使得当$\rho_X(x',x_0)<\delta_2$时有$\rho_Y(Tx',y)<\dfrac{\varepsilon}{2}$。取$\delta=\min\{\delta_1,\delta_2\}$则当$\rho_X(x,x_0)<\delta$且$\rho_X(x',x_0)<\delta$时有$\rho_Y(Tx,y)<\dfrac{\varepsilon}{2}$和$\rho_Y(Tx',y)<\dfrac{\varepsilon}{2}$,于是:
	\begin{equation*}
		\rho_Y(Tx,Tx')\leqslant\rho_Y(Tx,y)+\rho_Y(Tx',y)<\varepsilon
	\end{equation*}\par
	\textbf{充分性:}因为$x_0$是$E$的聚点,所以存在$\{x_n\}\subseteq E\setminus\{x_0\}$满足$\{x_n\}\to x_0$。对于任意的$\varepsilon>0$,取任意一个满足上述条件的$\{x_n\}$,则对于条件中给定的$\delta>0$,存在$N\in\mathbb{N}^+$使得当$n>N$时有$\rho_X(x_n,x_0)<\delta$,于是当$m,n>N$时就有$\rho_Y(Tx_n,Tx_m)<\varepsilon$。因为$(Y,\rho_Y)$是完备的度量空间,所以$\{Tx_n\}$有极限。下面证明所有$\{Tx_n\}$的极限都相同。假设存在不同的$\{x_n'\}$和$\{x_n''\}$满足:
	\begin{equation*}
		\{x_n'\},\{x_n''\}\subseteq E\setminus\{x_0\},\quad\{x_n'\},\{x_n''\}\to x_0,\quad\lim_{n\to+\infty}Tx_n'\ne\lim_{n\to+\infty}Tx_n''
	\end{equation*}
	定义$\{x_n\}$:
	\begin{equation*}
		x_n=
		\begin{cases}
			x_n',&n\text{为奇数} \\
			x_n'',&n\text{为偶数}
		\end{cases}
	\end{equation*}
	则$\{x_n\}\subseteq E\setminus\{x_0\}$且$\{x_n\}\to x_0$,但由\cref{prop:ConvergentSeqOfPoints}可知$\{Tx_n\}$不收敛,矛盾,所以所有$\{Tx_n\}$的极限都相同,充分性得证。
\end{proof}
\begin{property}\label{prop:MappingLimits}
	设$(X,\rho_X)$和$(Y,\rho_Y)$都是度量空间,$T$是一个$E\subseteq X$到$Y$的映射,则:
	\begin{enumerate}
		\item 若$T$沿$E$趋于某一点时有极限,则极限唯一;
		\item 若$x_0$是$E$的聚点,$T$沿$E$趋于$x_0$时极限为$y\in Y$,则存在$\delta>0$使得$T[U(x_0,\delta)\setminus\{x_0\}]$有界;
		\item 设$(Z,\rho_Z)$是度量空间,$S$是$F\subseteq Y$到$Z$的映射,$x_0\in X$是$E$的聚点,$y\in Y$是$T(E\setminus\{x_0\})$的聚点,$T(E\setminus\{x_0\})\subseteq F\setminus\{y\}$,$z\in Z$。若$T$沿$E$趋于$x_0$时极限为$y$,$S$沿$F$趋于$y$时极限为$z$,则:
		\begin{equation*}
			\lim_{\substack{x\to x_0 \\ E}}S(Tx)=z
		\end{equation*}
	\end{enumerate}
\end{property}
\begin{proof}
	(1)由\cref{theo:SequentialCharacterizationOfFunctionLimits}和\cref{prop:ConvergentSeqOfPoints}(1)立即可得。\par
	(2)由定义立即可得。\par
	(3)由映射极限的定义,$E\setminus x_0$中任意满足$\{x_n\}\to x_0$的点列$\{x_n\}$都有$\{Tx_n\}\subseteq F\setminus\{y\}$且$\{Tx_n\}\to y$,根据\cref{theo:SequentialCharacterizationOfFunctionLimits}可知$\{S(Tx_n)\}\to z$,于是结论成立。
\end{proof}
\subsection{映射的连续性}
\begin{definition}
	设$(X,\rho_X)$和$(Y,\rho_Y)$都是度量空间,$T$是一个$E\subseteq X$到$Y$的映射,$x_0\in E$是$E$的一个聚点。若对任意的$\varepsilon>0$,存在$\delta>0$使得对$E$中一切满足条件$\rho(x,x_0)<\delta$的$x$都有$\rho_Y(Tx,Tx_0)<\varepsilon$,或对于任意满足$\{x_n\}\to x_0$的点列$\{x_n\}\subseteq E$都有$\{Tx_n\}\to Tx_0$,则称$T$沿$E$在$x_0$处\gls{continuous}。
\end{definition}
两个条件的等价性类似\cref{theo:SequentialCharacterizationOfFunctionLimits}即可得到。
\begin{definition}
	设$T$是一个$(X,\rho_X)$到$(Y,\rho_Y)$的映射,若$T$在$E\subseteq X$的每一点都连续,则称$T$是$E$上的\gls{ContinuousMap}。
\end{definition}
\begin{theorem}\label{theo:ContinousMapO2OC2C}
	度量空间$(X,\rho_X)$到$(Y,\rho_Y)$上的映射$T$是$X$上的连续映射的充要条件为:
	\begin{enumerate}
		\item $Y$中任意开集$E$的原像$T^{-1}E$是$X$中的开集。
		\item $Y$中任意闭集$E$的原像$T^{-1}E$是$X$中的闭集。
	\end{enumerate}
\end{theorem}
\begin{proof}
	(1)\textbf{必要性:}设$T$是连续映射,$E\subseteq Y$是一个开集,如果$T^{-1}E=\varnothing$,则$T^{-1}E$是开集;若$T^{-1}E\ne\varnothing$,任取$x\in T^{-1}E$,令$Tx=y$,则$y\in E$,因为$E$是开集,所以存在$y$的$\varepsilon$邻域$U$,使得$U\subseteq E$,由$T$的连续性,存在$x$的$\delta$邻域$V$,使得$TV\subseteq U$,因此$V\subseteq T^{-1}U\subseteq T^{-1}E$,即$x$是$T^{-1}E$的内点。由$x$的任意性,$T^{-1}E$是$X$中的开集。\par
	\textbf{充分性:}对任意的$x\in X$及$Tx$的任意$\varepsilon$邻域$U$,由\cref{prop:OpenClosedSet}(1)可知$U$是一个开集,因此$T^{-1}U$是$X$中的开集,所以$x$是$T^{-1}U$的内点,于是存在$x$的某个$\delta$邻域$V$,使得$V\subseteq T^{-1}U$,因此$TV\subseteq U$,即$T$在$x$处连续。由$x$的任意性,$T$是$X$上的连续映射。\par
	(2)由(1)、\cref{theo:PropertyOfPreimage}(3)和\cref{prop:OpenClosedSet}(5)立即可得。
\end{proof}
\begin{theorem}\label{theo:CompositeContinuousMap}
	设$(X,\rho_X),(Y,\rho_Y),(Z,\rho_Z)$都是度量空间,$T$是一个$E\subseteq X$到$Y$的映射,$S$是$F\subseteq Y$到$Z$的映射。若$T$在$E$上连续、$S$在$F$上连续且$TE\subseteq F$,则$ST$在$E$上连续。
\end{theorem}
\begin{proof}
	任取$x\in E$和$\{x_n\}\subseteq E$满足$\{x_n\}\to x$,因为$T$在$E$上连续,所以$\{Tx_n\}\to Tx$。由$TE\subseteq F$和$S$在$F$上连续可得$\{STx_n\}\to STx$,即$ST$在$x$处连续。根据$x$的任意性可得$ST$在$E$上连续。
\end{proof}

\subsection{压缩映射原理}
\begin{definition}
	若点$\varphi$在映射$T$的作用下满足$T\varphi=\varphi$,则称$\varphi$是映射$T$的一个\gls{FixedP}。
\end{definition}
\begin{definition}
	设$(X,\rho)$是一个度量空间,$T$是$X$到$X$的一个映射,如果存在一个数$\alpha$,$0\leqslant\alpha<1$,使得对任意的$x,y\in X$,有:
	\begin{equation*}
		\rho(Tx,Ty)\leqslant\alpha\rho(x,y)
	\end{equation*}
	则称$T$是一个\gls{ContractionMap}。
\end{definition}
\begin{theorem}[Contraction Mapping Theorem]\label{theo:ContractionMapTheorem}
	设$(X,\rho)$是一个完备的度量空间,$T$是$X$到$X$的一个压缩映射,那么$T$有且只有一个不动点,该不动点为任取$x_0\in X$序列$\{x_n=T^nx_0\}$的极限$x\in X$,并有下述两种误差估计:
	\begin{equation*}
		\rho(x_n,x)\leqslant\frac{\alpha^n}{1-\alpha}\rho(x_0,x_1),\quad\rho(x_n,x)\leqslant\frac{1}{1-\alpha}\rho(x_{n+1},x_{n})
	\end{equation*}
\end{theorem}
\begin{proof}
	\textbf{存在性:}任取$x_0\in X$,令$x_n=T^nx_0$,由此产生一个点列$\{x_n\}$。下面我们来证明这个点列是一个Cauchy点列,它的极限就是一个不动点。\par
	\begin{align*}
		\rho(x_{m+1},x_m)&=\rho(Tx_m,Tx_{m-1})\leqslant\alpha\rho(x_m,x_{m+1}) \\
		&=\cdots \\
		&=\alpha^{m-1}\rho(Tx_1,Tx_0)\leqslant\alpha^m\rho(x_1,x_0)
	\end{align*}
	取$n>m$,由距离的三角不等式:
	\begin{align*}
		\rho(x_m,x_n)
		&\leqslant\rho(x_m,x_{m+1})+\cdots+\rho(x_{n-1},x_n) \\
		&\leqslant(\alpha^m+\alpha^{m+1}+\cdots+\alpha^{n-1})\rho(x_0,x_1) \\
		&=\alpha^m\frac{1-\alpha^{n-m}}{1-\alpha}\rho(x_0,x_1) \\
		&\leqslant\frac{\alpha^m}{1-\alpha}\rho(x_0,x_1)
	\end{align*}
	因为$0\leqslant\alpha<1$,所以当$m$足够大的时候,$\rho(x_m,x_n)\rightarrow 0$,即$\{x_n\}$是$X$中的Cauchy点列。又因为$X$完备,所以$\{x_n\}\rightarrow x\in X$。由三角不等式:
	\begin{equation*}
		\rho(x,Tx)\leqslant\rho(x,x_m)+\rho(x_m,Tx)\leqslant\rho(x,x_m)+\alpha\rho(x_{m-1},x)
	\end{equation*}
	当$m\to+\infty$时上式右端趋于0,因此$\rho(x,Tx)=0$,即$Tx=x$,$T$存在一个不动点。由上上式可得:
	\begin{equation*}
		\rho(x_m,x_{m+k})\leqslant\frac{\alpha^m}{1-\alpha}\rho(x_0,x_1)
	\end{equation*}
	令$k\to+\infty$,由\cref{prop:RSeq}(6)可得:
	\begin{equation*}
		\rho(x_m,x)\leqslant\frac{\alpha^m}{1-\alpha}\rho(x_0,x_1)
	\end{equation*}\par
	\textbf{唯一性:}假设$T$还有一个不动点$y$,则
	\begin{equation*}
		\rho(x,y)=\rho(Tx,Ty)\leqslant\alpha\rho(x,y)
	\end{equation*}
	因为$0\leqslant\alpha<1$,所以$\rho(x,y)=0$,即$x=y$,唯一性得证。\par
	对于第二种误差估计,只需注意到:
	\begin{equation*}
		\rho(x_n,x_{n+p})\leqslant(\alpha^{p-1}+\alpha^{p-2}+\cdots+1)\rho(x_n,x_{n+1})\leqslant\frac{1}{1-\alpha}\rho(x_n,x_{n+1})
	\end{equation*}
	令$p\to+\infty$,由\cref{prop:RSeq}(6)可得:
	\begin{equation*}
		\rho(x_n,x)\leqslant\frac{1}{1-\alpha}\rho(x_n,x_{n+1})\qedhere
	\end{equation*}
\end{proof}
压缩映射原理有一个推广:
\begin{theorem}
	设$T$是完备度量空间$X$到自身的映射,如果存在常数$\alpha$及$n\in\mathbb{N}^+$,$0\leqslant\alpha<1$,使得对任意$x,y\in X$,有:
	\begin{equation*}
		\rho(T^{n}x,T^{n}y)\leqslant\alpha\rho(x,y)
	\end{equation*}
	那么$T$在$X$中有且只有一个不动点。
\end{theorem}
\begin{proof}
	\textbf{存在性:}$T^{n}$满足\cref{theo:ContractionMapTheorem}的条件,因此$T^{n}$有且只有一个不动点$x_0$。下证$x_0$也是$T$在$X$中唯一的不动点。因为
	\begin{equation*}
		T^{n}(Tx_0)=T^{n+1}x_0=T(T^{n}x_0)=Tx_0
	\end{equation*}
	所以$Tx_0$是$T^{n}$的一个不动点,由不动点的唯一性,$Tx_0=x_0$,所以$x_0$是$T$的一个不动点。\par
	\textbf{唯一性:}若$T$存在另一个不动点$x_1$,则
	\begin{equation*}
		T^{n}x_1=T^{n-1}Tx_1=T^{n-1}x_1=\cdots=Tx_1=x_1
	\end{equation*}
	即$x_1$也是$T^{n}$的一个不动点,由$T^{n}$不动点的唯一性,$x_0=x_1$。
\end{proof}

\subsection{紧集上的连续映射}
\begin{definition}
	设$(X,\rho_X)$和$(Y,\rho_Y)$为度量空间,$T$是$E\subseteq X$到$Y$上的映射。若对于任意的$\varepsilon>0$,存在只与$\varepsilon$有关的$\delta>0$,使得对任意的$x,y\in E$,只要$\rho_X(x,y)<\delta$,就有$\rho_Y(Tx,Ty)<\varepsilon$,则称$T$在$E$上\gls{UniformlyContinuous}。
\end{definition}
\begin{theorem}[Sequential Characterization of Uniformly Continuous]
	设$(X,\rho_X)$和$(Y,\rho_Y)$为度量空间,$T$是$E\subseteq X$到$Y$上的映射。$T$在$E$上一致连续的充分必要条件为:对于任意满足条件$\lim\limits_{n\to+\infty}\rho_X(x_n,y_n)$的点列$\{x_n\},\{y_n\}\subseteq E$,都有:
	\begin{equation*}
		\lim_{n\to+\infty}\rho_Y(Tx_n,Ty_n)=0
	\end{equation*}
\end{theorem}
\begin{proof}
	\textbf{必要性:}设$T$在$E$上一致连续,则对于任意的$\varepsilon>0$,存在只与$\varepsilon$有关的$\delta>0$,使得对任意的$x,y\in E$,只要$\rho_X(x,y)<\delta$,就有$\rho_Y(Tx,Ty)<\varepsilon$。因为$\lim\limits_{n\to+\infty}\rho_X(x_n,y_n)=0$,所以存在$N\in\mathbb{N}^+$,当$n>N$时有$\rho_X(x_n,y_n)<\delta$,于是$\rho_Y(Tx_n,Ty_n)<\varepsilon$,即$\{\rho_Y(Tx_n,Ty_n)\}\to 0$。\par
	\textbf{充分性:}若此时$T$不在$E$上一致连续,则存在$\varepsilon>0$,无论$\delta=\dfrac{1}{n}$多小,总存在满足$\rho_X(x_n,y_n)<\delta$的$x_n,y_n\in E$使得$\rho_Y(Tx_n,Ty_n)\geqslant\varepsilon$。此时满足$\lim\limits_{n\to+\infty}\rho_X(x_n,y_n)$,但没有$\lim\limits_{n\to+\infty}\rho_Y(Tx_n,Ty_n)=0$,矛盾。
\end{proof}
\begin{property}\label{prop:CompactMap}
	设$(X,\rho_X)$为度量空间,$A$是$X$中的紧集,$T$是$A$到$\mathbb{R}^{}$上的映射,则:
	\begin{enumerate}
		\item $TA$是$Y$中的紧集;
		\item $T$在$A$上有界;
		\item $T$在$A$上可达到其上、下确界;
		\item $T$在$A$上一致连续。
	\end{enumerate}
\end{property}
\begin{proof}
	(1)设$\{y_n\}$为$TA$中的一个点列,则有$X$中的点列$\{x_n\}$使得$y_n=Tx_n,\;n\in\mathbb{N}^+$。因为$A$是紧集,所以$\{x_n\}$存在子列$\{x_{n_k}\}\to x_0\in A$。因为$T$连续,所以:
	\begin{equation*}
		\lim_{k\to+\infty}y_{n_k}=\lim_{k\to+\infty}Tx_{n_k}=T\left(\lim_{k\to+\infty}x_{n_k}\right)=Tx_0\in TA
	\end{equation*}
	所以$TA$是紧集。\par
	(2)由(1)可知$TA$是紧集,根据\cref{prop:CompactSet}(2)可得$T$在$A$上有界。\par
	(3)由(1)和\cref{theo:CompactRn}可知$TA$是有界闭集,所以$T$在$A$上可达到其上、下确界。\par
	(4)假设此时$T$不一致连续,则存在$\varepsilon_0>0$以及点列$\{x_n\},\{y_n\}\subseteq A$,使得:
	\begin{equation*}
		\lim_{n\to+\infty}\rho_X(x_n,y_n)=0,\quad\rho_Y(Tx_n,Ty_n)\geqslant\varepsilon_0,\;\forall\;n\in\mathbb{N}^+
	\end{equation*}
	因为$A$是紧集,所以$\{x_n\}$存在子列$\{x_{n_k}\}\to x_0\in A$,即$\rho_X(x_{n_k},x_0)\to 0$,于是由\cref{prop:RSeq}(8.b)可得:
	\begin{equation*}
		\rho_X(y_{n_k},x_0)\leqslant\rho_X(y_{n_k},x_{n_k})+\rho_X(x_{n_k},x_0)\to0
	\end{equation*}
	因为$T$是连续的,所以:
	\begin{equation*}
		\rho_Y(Tx_{n_k},Tx_0)\to0,\;\rho_Y(Ty_{n_k},Tx_0)\to0
	\end{equation*}
	于是由\cref{prop:RSeq}(8.b)可得:
	\begin{equation*}
		\rho_Y(Tx_{n_k},Ty_{n_k})\leqslant\rho_Y(Tx_{n_k},Tx_0)+\rho_Y(Tx_0,Ty_{n_k})\to 0
	\end{equation*}
	与第一个式子中的第二部分矛盾,所以$T$一致连续。
\end{proof}

\subsection{值域为$\mathbb{R}^{},\mathbb{R}^{n}$的映射}
请自行给出定义在$\mathbb{R}^{}$且值域为$\mathbb{R}^{}$的映射极限的32个定义(趋于的点为实数、负无穷、正无穷、无穷,极限为实数、负无穷、正无穷、无穷,组合共16种,再考虑序列式与$\varepsilon-\delta$语言,一共32种)。\par
对于定义在$\mathbb{R}^{}$上的函数,考虑映射$f$沿$E$趋于$x_0\in\mathbb{R}^{}$时的极限,当$E$在数轴上完全位于$x_0$的左侧或右侧时,我们将极限分别简记为:
\begin{equation*}
	\lim_{x\to x_0^-}f(x),\quad\lim_{x\to x_0^+}f(x)
\end{equation*}
\begin{property}\label{prop:RMap}
	设$(X,\rho_X)$是度量空间,$E\subseteq X$,$x_0$是$E$的一个聚点,$f,g,h$是$E$到$\mathbb{R}^{}$上的映射,则:
	\begin{enumerate}
		\item 若$\lim\limits_{\substack{x\to x_0 \\ E}}f(x)$存在,则其极限唯一;
		\item (Squeeze Theorem) 若存在$\delta>0$,使得对任意的$x\in U(x_0,\delta)\cap E\setminus\{x_0\}$都有如下之一成立,则对应的结论也成立:
		\begin{enumerate}
			\item $f(x)\leqslant h(x)\leqslant g(x)$,且$\lim\limits_{\substack{x\to x_0 \\ E}}f(x)=\lim\limits_{\substack{x\to x_0 \\ E}}g(x)=a\in\mathbb{R}^{}$,则$\lim\limits_{\substack{x\to x_0 \\ E}}h(x)=a$;
			\item $h(x)\geqslant f(x)$,且$\lim\limits_{\substack{x\to x_0 \\ E}}f(x)=+\infty$,则$\lim\limits_{\substack{x\to x_0 \\ E}}h(x)=+\infty$;
			\item $h(x)\leqslant f(x)$,且$\lim\limits_{\substack{x\to x_0 \\ E}}f(x)=-\infty$,则$\lim\limits_{\substack{x\to x_0 \\ E}}h(x)=-\infty$;
		\end{enumerate}
		\item 若$\lim\limits_{\substack{x\to x_0 \\ E}}f(x)$和$\lim\limits_{\substack{x\to x_0 \\ E}}g(x)$存在,且有:
		\begin{equation*}
			\lim_{\substack{x\to x_0 \\ E}}f(x)<\lim_{\substack{x\to x_0 \\ E}}g(x)
		\end{equation*}
		则存在$\delta>0$,使得当$x\in U(x_0,\delta)\cap E\setminus\{x_0\}$时,$f(x)<g(x)$;
		\item 若存在$\delta>0$使得当$x\in U(x_0,\delta)\cap E\setminus\{x_0\}$时有$f(x)\leqslant g(x)$,且$\lim\limits_{\substack{x\to x_0 \\ E}}f(x)$和$\lim\limits_{\substack{x\to x_0 \\ E}}g(x)$都存在,则:
		\begin{equation*}
			\lim_{\substack{x\to x_0 \\ E}}f(x)\leqslant\lim_{\substack{x\to x_0 \\ E}}g(x)
		\end{equation*}
		\item 若$\lim\limits_{\substack{x\to x_0 \\ E}}f(x)$和$\lim\limits_{\substack{x\to x_0 \\ E}}g(x)$存在,且下式右侧有意义,则公式成立:
		\begin{enumerate}
			\item $\lim\limits_{\substack{x\to x_0 \\ E}}|f(x)|=\Big|\lim\limits_{\substack{x\to x_0 \\ E}}f(x)\Big|$;
			\item $\lim\limits_{\substack{x\to x_0 \\ E}}[f(x)\pm g(x)]=\lim\limits_{\substack{x\to x_0 \\ E}}f(x)\pm\lim\limits_{\substack{x\to x_0 \\ E}}g(x)$;
			\item $\lim\limits_{\substack{x\to x_0 \\ E}}f(x)g(x)=\lim\limits_{\substack{x\to x_0 \\ E}}f(x)\lim\limits_{\substack{x\to x_0 \\ E}}g(x)$;
			\item $\lim\limits_{\substack{x\to x_0 \\ E}}\dfrac{f(x)}{g(x)}=\dfrac{\lim\limits_{\substack{x\to x_0 \\ E}}f(x)}{\lim\limits_{\substack{x\to x_0 \\ E}}g(x)}$;
		\end{enumerate}
		该性质内蕴了连续函数四则运算的连续性问题。
	\end{enumerate}
\end{property}
\begin{property}
	值域为$\mathbb{R}^{n}$的函数的收敛等价于按坐标收敛\info{度量空间补充笛卡尔积的情况}。
\end{property}
\begin{note}
	考虑到极限情况很多,上面的性质证明起来特别麻烦,故略去。但作一点说明:\cref{theo:SequentialCharacterizationOfFunctionLimits}和\cref{theo:Cauchy-TypeConditionForFunctionLimits}都可以推广到趋于的点坐标含无穷的情况,其中\cref{theo:SequentialCharacterizationOfFunctionLimits}还可以推广到极限为无穷,证明过程类似之前的叙述。之后我们在引用\cref{theo:SequentialCharacterizationOfFunctionLimits}和\cref{theo:Cauchy-TypeConditionForFunctionLimits}时,将包含无穷的情况,不做额外的说明。上述性质的证明在使用\cref{theo:SequentialCharacterizationOfFunctionLimits}和\cref{theo:Cauchy-TypeConditionForFunctionLimits}的推广后,都可以由\cref{prop:RSeq}和\cref{prop:RmConvergence}(2)推出。
\end{note}
\begin{definition}
	称多项式函数、有理分式函数、三角函数、反三角函数、对数函数、指数函数为\gls{BasicElementaryFunction},经基本初等函数经过有限次四则运算和复合而成的函数被称为\gls{ElementaryFunction}。
\end{definition}
\begin{theorem}
	初等函数在其定义域内连续。
\end{theorem}
\subsubsection{连通性与介值性}
\begin{definition}
	设$(X,\rho)$是一个度量空间,$E\subseteq X$,$x_0,x_1\in E$。若连续映射$\gamma:[0,1]\to E$满足$\gamma(0)=x_0,\;\gamma(1)=x_1$,则称$\gamma$为$E$中联结$x_0$和$x_1$的一条\gls{Path}。
\end{definition}
\begin{definition}
	设$(X,\rho)$是一个度量空间,$E\subseteq X$。若对于任意的$x_0,x_1\in E$,都存在一条联结它们的路径,则称$E$\gls{PathConnected}。定义$\varnothing$也是路径连通的。
\end{definition}
\begin{lemma}\label{lem:IntermediateValueR}
	设函数$f:\mathbb{R}^{}\to\mathbb{R}^{}$在区间$[a,b]$上连续。
	\begin{enumerate}
		\item 若$f(a)f(b)<0$,则存在$c\in(a,b)$使得$f(c)=0$;
		\item 若$f(a)\ne f(b)$,则$f$在$[a,b]$上能取到介于$f(a)$和$f(b)$之间的所有值。
	\end{enumerate}
\end{lemma}
\begin{proof}
	(1)仅讨论$f(a)<0<f(b)$时的情况,$f(b)<0<f(a)$时完全类似。\par
	取$c_0=\dfrac{a+b}{2}$,则$f(c_0)=0$或$f(c_0)$与$f(a),f(b)$中的一个异号。若$f(c_0)$与$f(a)$异号,则设$a_1=a,\;b_1=c_0$;若$f(c_0)$与$f(b)$异号,则设$a_1=c_0,\;b_1=b$。取$c_1=\dfrac{a_1+b_1}{2}$,则$f(c_1)=0$或$f(c_1)$与$f(a_1),f(b_1)$中的一个异号。若$f(c_1)$与$f(a_1)$异号,则设$a_2=a_1,\;b_2=c_1$;若$f(c_1)$与$f(b_1)$异号,则设$a_2=c_1,\;b_2=b_1$。\par
	不断重复上述讨论,要么存在一个$c_n$使得$f(c_n)=0$,结论成立,要么得到一个闭区间套$\{[a_n,b_n]\}$和一个点列$\{c_n\}$,满足$a_n\leqslant c_n\leqslant b_n$。由\cref{theo:ClosedCubeTheorem}和\cref{prop:RSeq}(4.a)可得:
	\begin{equation*}
		\lim_{n\to+\infty}a_n=\lim_{n\to+\infty}c_n=\lim_{n\to+\infty}b_n\coloneq c\in(a,b)
	\end{equation*}
	因为$f$在区间$[a,b]$上连续,由\cref{prop:RSeq}(6)可得:
	\begin{equation*}
		f(c)=\lim_{n\to+\infty}f(a_n)\leqslant0\leqslant\lim_{n\to+\infty}f(b_n)=f(c)
	\end{equation*}
	所以$f(c)=0$。\par
	(2)对任意介于$f(a),f(b)$之间的$\alpha$,取辅助函数$f(x)-\alpha$。由\cref{prop:RMap}(5.b)可得$f(x)-\alpha$是区间$[a,b]$上的连续函数,由(1)即可得出结论。
\end{proof}
\begin{theorem}[Intermidiate Value Theorem]\label{theo:IntermediateValue}
	设$(X,\rho)$是一个度量空间,$E\subseteq X$路径连通,$f:E\to\mathbb{R}^{}$是一个连续函数,则$f(E)$是一个区间(允许退化为单点)。
\end{theorem}
\begin{proof}
	任取$y_0,y_1\in f(E)$,设$f(x_0)=y_0,\;f(x_1)=y_1$。因为$E$路径连通,所以存在连续映射$\gamma:[0,1]\to E$满足$\gamma(0)=x_0,\;\gamma(1)=x_1$。考虑复合映射$f\circ\gamma$,由\cref{theo:CompositeContinuousMap}可知$f\circ\gamma$在$[0,1]$上连续。因为:
	\begin{equation*}
		f\circ\gamma(0)=f(x_0)=y_0,\quad f\circ\gamma(1)=f(x_1)=y_1
	\end{equation*}
	由\cref{lem:IntermediateValueR}可知$f\circ\gamma$在$[0,1]$上可以取到介于$y_0$和$y_1$之间的所有值,即$f$在$E$上可以取到介于$y_0$和$y_1$之间的所有值。由$y_0,y_1$的任意性,结论成立。
\end{proof}

\subsubsection{无穷小与有界记号}
\begin{definition}
	设$f(x)$是在$U(a,\delta)$上有定义,$a\in\overline{\mathbb{R}^{}},\;\delta>0$。若$\lim\limits_{x\to a}f(x)=0$,则称$f(x)$是$x\to a$时的\gls{Infinitesimal};若$\lim\limits_{x\to a}f(x)=\infty$,则称$f(x)$是$x\to a$时的\gls{InfiniteQuantity}。
\end{definition}
\begin{definition}
	设$f(x),g(x)$是在$U(a,\delta)$上有定义,$a\in\overline{\mathbb{R}^{}},\;\delta>0$,$g(x)$在$U(a,\delta)$上不为$0$。
	\begin{enumerate}
		\item 若$\dfrac{f(x)}{g(x)}$是$x\to a$时的有界变量,则记$f(x)=\operatorname{O}(g(x))$;
		\item 若$\dfrac{f(x)}{g(x)}$是$x\to a$时的无穷小量,则记$f(x)=\operatorname{o}(g(x))$;
		\item 若$\lim\limits_{x\to a}\dfrac{f(x)}{g(x)}=1$,则记$f(x)\sim g(x)$。
	\end{enumerate}
\end{definition}
\begin{note}
	使用无穷小量、无穷大量以及上述记号时需要说明涉及的极限过程,如$f(x)=\operatorname{O}(g(x))(x\to a)$。等式两边都存在记号时需要注意此时等号的含义并不是等于,而是也是。
\end{note}
\begin{definition}
	设$f(x),g(x)$是无穷小(大)量。若$f(x)=o(g(x))$,则称$f(x)$是比$g(x)$更高(低)阶的无穷小(大)量;若$f(x)\sim g(x)$,则称$f(x)$是与$g(x)$等价的无穷小(大)量。
\end{definition}
\begin{property}
	设$f(x),g(x),h(x),p(x)$是在$U(a,\delta)$上有定义,$a\in\overline{\mathbb{R}^{}},\;\delta>0$,$g(x)\sim h(x)$,则:
	\begin{enumerate}
		\item $\operatorname{o}(f(x))=\operatorname{O}(f(x))$;
		\item $\operatorname{o}(f(x))+\operatorname{o}(f(x))=\operatorname{o}(f(x)),\;\operatorname{O}(f(x))+\operatorname{O}(f(x))=\operatorname{O}(f(x))$;
		\item $\operatorname{o}(f(x))\operatorname{O}(1)=\operatorname{o}(f(x)),\;\operatorname{o}(1)\operatorname{O}(f(x))=\operatorname{o}(f(x))$;
		\item $\lim\limits_{x\to a}f(x)g(x)=\lim\limits_{x\to a}f(x)h(x)$;
		\item $\lim\limits_{x\to a}\dfrac{f(x)g(x)}{p(x)}=\lim\limits_{x\to a}\dfrac{f(x)h(x)}{p(x)}$;
		\item $\lim\limits_{x\to a}\dfrac{f(x)}{g(x)p(x)}=\lim\limits_{x\to a}\dfrac{f(x)}{h(x)p(x)}$。
	\end{enumerate}
\end{property}
\begin{proof}
	(1)由定义立即可得。\par
	(2)由\cref{prop:RMap}(5.b)即可得到。\par
	(3)由\cref{prop:RMap}(5.c)即可得到。\par
	(4)由\cref{prop:RMap}(5.c)可得:
	\begin{equation*}
		\lim_{x\to a}f(x)g(x)=\lim_{x\to a}f(x)\frac{h(x)}{g(x)}g(x)=\lim_{x\to a}f(x)h(x)
	\end{equation*}\par
	(5)由(4)立即可得。\par
	(6)由\cref{prop:RMap}(5.c)可得:
	\begin{equation*}
		\lim_{x\to a}\dfrac{f(x)}{g(x)p(x)}=\lim_{x\to a}\dfrac{f(x)h(x)}{g(x)h(x)p(x)}=\lim_{x\to a}\dfrac{f(x)}{h(x)p(x)}\qedhere
	\end{equation*}
\end{proof}