\section{完备度量空间的性质}
\subsection{子空间的完备性}
\begin{theorem}
	完备度量空间$(X,\rho_X)$的子空间$M$是完备度量空间的充要条件为$M$是$X$中的一个闭子空间。
\end{theorem}
\begin{proof}
	必要性:因为$M$是完备子空间,则对任意的$x\in M'$,存在$M$中的一个收敛点列$\{x_n\}\to x$。因为收敛点列也是Cauchy点列,而此时Cauchy点列在$M$中收敛,所以$x\in M$。由$x$的任意性,$M'\subset M$,故$M$是一个闭集,即$M$是$X$的一个闭子空间。\par
	充分性:任取$\{x_n\}$为$M$中的一个Cauchy点列,那么它也是$X$中的Cauchy点列,因此$\exists\;x\in X$使得$\{x_n\}\to x$,即$x$是$M$的一个聚点。又因$M$是$X$的一个闭子空间,所以$x\in M$,即Cauchy点列$\{x_n\}$收敛于$M$中的一点。由$\{x_n\}$的任意性,$M$是完备的度量空间。
\end{proof}
\subsection{第一型集与第二型集}
\begin{definition}
	设$A$是度量空间$(X,\rho)$的子集。若$A$可表示为至多可列个稀疏集的并,则称$A$是\gls{FirstSet}。反之则为\gls{SecondSet}。
\end{definition}
\begin{theorem}
	任何完备的度量空间都是第二型集。
\end{theorem}
\begin{proof}
	假设不成立,即存在完备的度量空间$X$使得$X$是第一型集。也就是说$X=\underset{i=1}{\overset{+\infty}{\cup}}F_i$,其中$F_i,\;\forall\;i\in\mathbb{N}^+$是稀疏集。因为$F_1$是稀疏集,由稠密性等价定义(2),$\exists\;x_1\in X$,使得闭邻域$U(x_1,r_1)$中不含$F_1$中的点。对于闭邻域$U(x_1,r_1)$,由于$F_2$是稀疏集,因此$\exists\;x_2\in U(x_1,r_1)$,使得闭邻域$U(x_2,r_2)$中不含$F_2$中的点。如此重复下去,实际上可以取$r_n\in(0,\frac{1}{n})$(对于某个固定的半径,在这个半径内交集为空,那么在更小的半径内交集也为空),便可以得到一个闭球套:
	\begin{equation*}
		U(x_1,r_1)\supset U(x_2,r_2)\supset\cdots\supset U(x_n,r_n)\supset\cdots
	\end{equation*}
	且$\{r_n\}\to0$。由完备度量空间的闭球套定理,存在一个点$x\in U(x_n,r_n),\;\forall\;n\in\mathbb{N}^+$。而由闭球套的取法,$x\notin X$,矛盾。
\end{proof}
\subsection{准紧性与全有界性}
\subsubsection{准紧性的定义}
\begin{definition}
	$(X,\rho)$是一个度量空间,$A$是$X$的一个子集。如果$A$的每个点列都有一个收敛子列收敛于$X$中的某一点,则称$A$是\gls{PrecompactSet}。
\end{definition}
\subsubsection{准紧集的性质}
\begin{property}
	准紧集的子集也是准紧集。
\end{property}
\subsubsection{全有界集的定义}
\begin{definition}
	$(X,\rho)$是一个度量空间,$A$和$B$都是$X$的子集,$\varepsilon$是一个给定的正数。如果对任意的$x\in A$,都$\exists\;y\in B$,使得$\rho(x,y)<\varepsilon$,则称$B$是$A$的一个$\varepsilon$-网。即:以$B$中的点为中心,$\varepsilon$为半径的所有开邻域的并包含了$A$。
\end{definition}
\begin{definition}
	$(X,\rho)$是一个度量空间,$A$是$X$的子集。如果对任意的$\varepsilon>0$,$X$中总存在$A$的$\varepsilon$-网,且该$\varepsilon$-网只有有限个点,则称$A$是\gls{TotallyBoundedSet}。
\end{definition}
\subsubsection{全有界集的性质}
\begin{property}
	全有界集具有如下性质:\par
	(1)任何有限集都是全有界集。\par
	(2)全有界集的子集也是全有界集。\par
	(3)设$A$是一个全有界集,则对任意的$\varepsilon>0$,总存在$A$的一个有限子集成为$A$的一个$\varepsilon$-网。\par
	(4)全有界集有界且可分。
\end{property}
\begin{proof}
	(1)(2)是显然的。\par
	(3)因为$A$是一个全有界集,所以对任意的$\varepsilon>0$,存在一个$A$的$\frac{\varepsilon}{2}$-网$\{x_1,x_2,\dots,x_n\}$。依次取$a_i\in A$使得$a_i\in U(x_i,\frac{\varepsilon}{2}),\;i=1,2,\dots,n$,则$\{\seq{a}{n}\}$即构成$A$的一个$\varepsilon$-网:
	\begin{equation*}
		\forall\;a\in U\left(x_i,\frac{\varepsilon}{2}\right),\;\rho(a,a_i)\leqslant\rho(a,x_i)+\rho(x_i,a_i)<\frac{\varepsilon}{2}+\frac{\varepsilon}{2}=\varepsilon,\;i=1,2,\dots,n
	\end{equation*}
	\hspace{2em}(4)设$(X,\rho)$是给定的度量空间,$A$是一个全有界集。由全有界集定义,$A$应有一个$1$-网$\{x_1,x_2,\dots,x_n\}$。则:
	\begin{equation*}
		\forall\;a\in A,\;\exists\;x_{k}\in\{x_1,x_2,\dots,x_n\},\;\rho(a,x_k)<1
	\end{equation*}
	故(下式中$x_k$是与点$a$对应的$1$-网中的点):
	\begin{equation*}
		\forall\;a\in A,\;\rho(a,x_1)\leqslant\rho(a,x_k)+\rho(x_k,x_1)<1+\max_{k=1,2,\dots,n}\rho(x_k,x_1)
	\end{equation*}
	记$\max\limits_{k=1,2,\dots,n}\rho(x_k,x_1)=K$,则$\forall\;a\in A,\;a\in U(x_1,1+K)$,所以$A$是有界的。\par
	因为$A$是一个全有界集,所以对任意的$\varepsilon=\frac{1}{n},\;n\in\mathbb{N}^+$,$X$中都存在$A$的一个只含有限个点的$\varepsilon$-网$B_n$。记:
	\begin{equation*}
		B=\underset{n=1}{\overset{+\infty}{\cup}}B_n
	\end{equation*}
	显然$B$是可列的。对任意的$x\in A,\;\exists\;x_n\in B_n\subset B,\;\rho(x,x_n)<\frac{1}{n},\;\forall\;n\in\mathbb{N}^+$。因此点列$\{x_n\}$收敛于$x$。由$x$的任意性和稠密性的等价命题$3$,$B$在$A$中稠密。综上,$A$是可分的。
\end{proof}
\subsubsection{准紧性与全有界性的关系}
以下定理说明了完备度量空间中准紧性与全有界性的关系,即二者在完备度量空间中是等价的:
\begin{theorem}
	$(X,\rho)$是一个度量空间。\par
	(1)如果$A\subset X$准紧,则$A$全有界。\par
	(2)如果$X$是完备的,则当$A$全有界时,$A$也必定准紧。\par
\end{theorem}
\begin{proof}
	(1)如果$A$不是全有界集,则$\exists\;\varepsilon>0$,使得$A$没有只有有限点的$\varepsilon$-网。任取$x_1\in A$,则$\exists\;x_2\in A$使得$\rho(x_1,x_2)\geqslant\varepsilon$,否则$\{x_1\}$就是$A$的一个$\varepsilon$-网。同理,$\exists\;x_3\in A$使得$\rho(x_3,x_j)\geqslant\varepsilon,\;j=1,2$,否则$\{x_1,x_2\}$就是$A$的一个$\varepsilon$-网。重复这一步骤就得到点列$\{x_n\}$,当$m\ne n$时,$\rho(x_m,x_n)\geqslant\varepsilon$。由柯西收敛准则$\{x_n\}$显然没有收敛的子列,这与$A$的准紧性矛盾。\par
	(2)任取$A$中的点列$\{x_n\}$。如果$\{x_n\}$中只有有限个互不相同的元素,则$\{x_n\}$显然有收敛的子列。如果$\{x_n\}$中有无限个互不相同的元素,记这些元素构成的集合为$B_0$。由全有界集性质(2),$B_0$也是全有界集。由全有界集性质(3),$B_0$中存在有限个元素使得以它们为球心,$\frac{1}{2}$为半径的开邻域的并包含$B_0$,显然$B_0$中至少存在一个点$y_1$使得以它为半径,$\frac{1}{2}$为半径的开邻域包含了无穷多个$A$中的点。记被$y_1$包含的这无穷多个点构成的集合为$B_1$,显然$B_1$的直径小于$1$。因为$B_1$是$B_0$的子集,所以$B_1$也是全有界的,重复上述论证,则存在$B_2\subset B_1$,使得$B_2$中含有$B_1$无穷多个元素且$B_2$的直径小于$\frac{1}{2}$。依次类推,可以得到一系列集合满足如下条件:
	\begin{enumerate}
		\item $B_1\supset B_2\supset\cdots\supset B_n\supset\cdots$。
		\item $B_n$的直径小于$\frac{1}{2^{n-1}}$。
		\item 每个$B_n$中都含有$\{x_n\}$中无限个元素。
	\end{enumerate}
	取$x_{n_k}\in B_k,\;n_{k+1}>n_k,\;k\in\mathbb{N}^+$,便得到$\{x_n\}$的一个子列$\{x_{n_k}\}$。显然$\{x_{n_k}\}$是一个Cauchy点列:
	\begin{equation*}
		\forall\;p>q,\;x_{n_p}\in B_p\subset B_q,\;\rho(x_{n_p},x_{n_q})<\frac{1}{2^{q-1}}\to0
	\end{equation*}	
	因为$X$完备,所以$\{x_{n_k}\}$在$X$中收敛。由$\{x_n\}$的任意性,$A$准紧。
\end{proof}
\begin{corollary}
	度量空间中的准紧集是有界且可分的。
\end{corollary}
\begin{theorem}
	$(X,\rho)$是完备的度量空间,$A\subset X$。$A$为准紧集的充分必要条件是对任意的$\varepsilon>0$,$A$都有准紧的$\varepsilon$-网。
\end{theorem}
\begin{proof}
	(1)必要性:$A$就是它自身的准紧的$\varepsilon$-网。\par
	(2)充分性:若对任意的$\varepsilon>0$,$A$都有准紧的$\varepsilon$-网$B$。因为$B$准紧,同时$X$完备,所以$B$全有界,即$B$有只有有限个元素的$\varepsilon$-网$C=\{c_1,c_2,\dots,c_n\}$。则:
	\begin{equation*}
		\forall\;a\in A,\;\rho(a,c_i)\leqslant\rho(a,b)+\rho(b,c_i),\;\forall\;i=1,2,\dots,n
	\end{equation*}
	可以选取$c_i$和$b\in B$使得$\rho(a,b)<\varepsilon,\;\rho(b,c_i)<\varepsilon$(先选择$b$,再根据$b$即可选得$c_i$)。也就是说,$\forall\;a\in A,\;\exists\;c_i\in\{c_1,c_2,\dots,c_n\},\;\rho(a,c_i)<2\varepsilon$。以$\{c_1,c_2,\dots,c_n\}$中的点为中心,$2\varepsilon$为半径构成的$2\varepsilon$-网必然是$A$的一个$2\varepsilon$-网。由$\varepsilon$的任意性,$A$全有界。又因为$X$是完备的,所以$A$准紧。
\end{proof}

\subsection{压缩映射原理}
\subsubsection{不动点的定义}
\begin{definition}
	若点$\varphi$在映射$T$的作用下满足$T\varphi=\varphi$,则称$\varphi$是映射$T$的一个\gls{FixedP}。
\end{definition}
\subsubsection{压缩映射的定义}
\begin{definition}
	$(X,\rho)$是一个度量空间,$T$是$X$到$X$的一个映射,如果存在一个数$\alpha$,$0\leqslant\alpha<1$,使得对任意的$x,y\in X$,有:
	\begin{equation*}
		\rho(Tx,Ty)\leqslant\alpha\rho(x,y)
	\end{equation*}
	则称$T$是一个\gls{ContractionMap}。
\end{definition}
\subsubsection{压缩映射原理}
\begin{theorem}
	$(X,\rho)$是一个完备的度量空间,$T$是$X$到$X$的一个压缩映射,那么$T$有且只有一个不动点。
\end{theorem}
\begin{proof}
	(1)存在性:任取$x_0\in X$,令$x_n=T^nx_0$,由此产生一个点列$\{x_n\}$。下面我们来证明这个点列是一个Cauchy点列,它的极限就是一个不动点。\par
	\begin{align*}
		\rho(x_{m+1},x_m)&=\rho(Tx_m,Tx_{m-1})\leqslant\alpha\rho(x_m,x_{m+1}) \\
		&=\cdots \\
		&=\alpha^{m-1}\rho(Tx_1,Tx_0)\leqslant\alpha^m\rho(x_1,x_0)
	\end{align*}
	取$n>m$,由距离的三角不等式:
	\begin{align*}
		\rho(x_m,x_n)
		&\leqslant\rho(x_m,x_{m+1})+\cdots+\rho(x_{n-1},x_n) \\
		&\leqslant(\alpha^m+\alpha^{m+1}+\cdots+\alpha^{n-1})\rho(x_0,x_1) \\
		&=\alpha^m\frac{1-\alpha^{n-m}}{1-\alpha}\rho(x_0,x_1) \\
		&<\frac{\alpha^m}{1-\alpha}\rho(x_0,x_1)
	\end{align*}
	因为$0\leqslant\alpha<1$,所以当$m$足够大的时候,$\rho(x_m,x_n)\rightarrow 0$,即$\{x_n\}$是$X$中的Cauchy点列。又因为$X$完备,所以$\{x_n\}\rightarrow x\in X$。由三角不等式:
	\begin{equation*}
		\rho(x,Tx)\leqslant\rho(x,x_m)+\rho(x_m,Tx)\leqslant\rho(x,x_m)+\alpha\rho(x_{m-1},x)
	\end{equation*}
	当$m\to+\infty$时上式右端趋于0,因此$\rho(x,Tx)=0$,即$Tx=x$,$T$存在一个不动点。\par
	(2)唯一性:假设$T$还有一个不动点$y$,则
	\begin{equation*}
		\rho(x,y)=\rho(Tx,Ty)\leqslant\alpha\rho(x,y)
	\end{equation*}
	因为$0\leqslant\alpha<1$,所以$\rho(x,y)=0$,即$x=y$,唯一性得证。
\end{proof}
压缩映射原理有一个推广:
\begin{theorem}
	$T$是完备度量空间$X$到自身的映射,如果存在常数$\alpha$及自然数$n_0$,$0\leqslant\alpha<1$,使得对任意$x,y\in X$,有:
	\begin{equation*}
		\rho(T^{n_0}x,T^{n_0}y)\leqslant\alpha\rho(x,y)
	\end{equation*}
	那么$T$在$X$中有且只有一个不动点。
\end{theorem}
\begin{proof}
	存在性:$T^{n_0}$满足压缩映射原理的条件,因此$T^{n_0}$有且只有一个不动点$x_0$。下证$x_0$也是$T$在$X$中唯一的不动点。因为
	\begin{equation*}
		T^{n_0}(Tx_0)=T^{n_0+1}x_0=T(T^{n_0}x_0)=Tx_0
	\end{equation*}
	所以$Tx_0$是$T^{n_0}$的一个不动点,由不动点的唯一性,$Tx_0=x_0$,所以$x_0$是$T$的一个不动点。\par
	唯一性:若$T$存在另一个不动点$x_1$,则
	\begin{equation*}
		T^{n_0}x_1=T^{n_0-1}Tx_1=T^{n_0-1}x_1=\cdots=Tx_1=x_1
	\end{equation*}
	即$x_1$也是$T^{n_0}$的一个不动点,由$T^{n_0}$不动点的唯一性,$x_0=x_1$。
\end{proof}
