\section{Page检验}

\subsubsection{目的}
检验水平的位置参数是否呈现出上升趋势,若想检验是否呈现出下降趋势,改变水平顺序就行了。
\subsubsection{适用条件}
各水平间并不独立(因为还有区组的影响),采用完全区组设计,水平之间分布是相似的,离散型数据与连续型数据都可以。
\subsubsection{假设}
假设$k$个水平有分布函数$F_i(x)=F(x-\theta_i),\;i=1,2,\dots,k$,则检验假设可写为:
\begin{equation}
	H_0:\theta_1=\theta_2=\cdots=\theta_n\Leftrightarrow
	H_1:\theta_1\leqslant\theta_2\leqslant\cdots\leqslant\theta_n\notag
\end{equation}
\subsubsection{原理}
假设有$k$个水平、$b$个区组。\par
因为各区组之间是有影响的,无法把各响应值混在一起排序。选择在各个区组内计算所有响应值的秩,$R_{ij}$表示在第$j$个区组中水平$i$的秩,$R_i=\sum\limits_{j=1}^bR_{ij},\;i=1,2,\dots,k$,定义如下Page统计量:
\begin{equation}
	L=\sum_{i=1}^kiR_i\notag
\end{equation}
如果备择假设是正确的,那么对$R_i$进行加权求和可以对统计量起到一个放大的作用,那么$L$就会很大。因此在$L$比较大的时候,有理由怀疑零假设。由此可看出这里只考虑上侧的单侧检验问题。\par
若想求精确检验的结果,需要满足数据中没有结(即每个区组中都没有相同的数值),然后对每一种秩分配情况计算$L$的值(秩的大小关系便反映了数据之间的大小关系),即可得到$L$的精确分布。
\subsubsection{大样本近似}
在大样本的情况下($b\to+\infty$),有如下正态近似:
\begin{gather*}
	Z=\frac{L-\mu_L}{\sigma_L}\sim N(0,\;1) \\
	\mu_L=\frac{bk(k+1)^2}{4},\;\sigma_L^2=\frac{b(k^3-k)^2}{144(k-1)}
\end{gather*}
\subsubsection{打结}
若存在打结的情况,需要对正态近似的$\sigma_L^2$作如下修正(其中$\tau_{ij}$表示第$j$个区组的第$i$个结统计量):
\begin{equation}
	\sigma_L^2=k(k^2-1)\frac{bk(k^2-1)-\sum_i\sum_j(\tau_{ij}^3-\tau_{ij})}{144(k-1)}\notag
\end{equation}
\subsubsection{对区组、水平进行重复时的page检验}
在区组和水平之间不存在交互作用,并且所有$(i,\;j)$位置的重复数都相同时(假设都是$n$),有如下正态近似(也考虑了到打结的修正):
\begin{gather*}
	Z=\frac{L-\mu_L}{\sigma_L}\sim N(0,\;1) \\
	\mu_L=\frac{nbk(k+1)(nk+1)}{4} \\
	\sigma_L^2=nk(k^2-1)\frac{nbk(n^2k^2-1)-\sum_i\sum_j(\tau_{ij}^3-\tau_{ij})}{144(nk-1)}
\end{gather*}
\subsubsection{代码}
以下是自编代码,提供重复、精确计算、大样本近似、连续性修正与打结校正功能。当对区组与水平进行重复时,x必须是一个数据框,每一列是一次重复的结果,也必须传入factor与block,此时只提供大样本近似。当不进行重复时,x可以是一个三列的数据框,第一列表示response值,第二列表示factor,第三列表示block。请注意,检验顺序与输入的factor顺序是一致的,建议对照二维表\textbf{逐行}输入(行是水平,检验顺序即为行的顺序,列是区组)。如果出现打结的情况,结统计量按照计算公式进行计算,$L$统计量的秩部分取结的平均秩(例:数据为1、1、3、4,有重复值1,则秩为1.5、1.5、3、4)。
\inputminted[bgcolor=white, linenos, frame=single, numbersep=5pt, breaklines]{r}{nonparametric-statistics/chapter3/page.R}