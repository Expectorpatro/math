\chapter{多样本位置检验}
\section{目的}
利用样本中各响应值的位置信息,判断各水平间是否存在差异或某种趋势。\info{在完成试验设计之后,把多样本位置检验与方差分析部分结合起来}
\section{本章各方法的简单总结}
\begin{enumerate}
	\item 所有方法都要求各水平的总体分布之间是相似的,仅仅是位置参数可能有些差异。
	\item Kruskal-Wallis与Friedman、Durbin对称,都是检验各水平响应值的位置参数是否一致(即各水平在该响应值下是否有差异)。Kruskal是在各水平不相关的情况下使用的,Freidman是在完全区组设计下、各水平之间由于区组问题不独立时使用的,而Durbin是在不完全区组设计下、各区组之间独立时使用的。同时Kruskal-Wallis、Durbin要求数据是连续型的,Friedman则无此类要求。
	\item Jonckheere-Terpstra与page对称,都是检验各水平响应值的位置参数是否有顺序关系,Jonckheere-Terpstra是在各水平不相关的情况下使用的,page是在完全区组设计下、各水平之间由于区组问题不独立时使用的。同时Jonckheere-Terpstra要求数据是连续型的,page则无此类要求。
	\item Cochran与Friedman对应,Cochran是完全区组设计、二元响应情况下检验各水平是否存在差异的方法。
	\item Friedman与下一章Kendall检验是一致的。Kendall一般用在判断评估者的主观评估是否一致,而Friedman则是客观的判断水平之间是否有差异。
\end{enumerate}
\section{Kruskal-Wallis检验}

\subsubsection{适用条件}
各响应值在水平间和水平内是独立的,水平之间分布是相似的,数据是连续型的。
\subsubsection{假设}
假设$k$个水平有分布函数$F_i(x)=F(x-\theta_i),\;i=1,2,\dots,k$,则检验假设可写为:
\begin{equation}
	H_0:\theta_1=\theta_2=\cdots=\theta_n\Leftrightarrow
	H_1:\text{至少有一个等号不成立}\notag
\end{equation}
\subsubsection{原理}
将所有水平的响应值混合后排序,得到每一个响应值对应于所有数据的秩,类似于方差分析中MSA的构成,若水平间的秩和差异大,则应怀疑零假设。由此构建以下Kruskal-Wallis统计量(其中$\bar{R_i}$表示第$i$个水平秩的平均值,$\bar{R}$表示所有响应值秩的平均值):
\begin{equation}
	H=\frac{12}{N(N+1)}\sum_{i=1}^kn_i(\bar{R_i}-\bar{R})^2=\frac{12}{N(N+1)}\sum_{i=1}^k\frac{R_i^2}{n_i}-3(N+1)\notag
\end{equation}
\hspace{2em}如果备择假设成立,那么统计量的值应该是偏大的,因此只考虑上侧的单侧检验问题。\par
若想求精确结果,需要满足数据中不存在结(即数据中不存在相同的数值),对每种秩分配计算对应的$H$值,便能得到此时统计量$H$的分布。\par
\subsubsection{大样本近似}
在大样本的情况下,若零假设成立,有如下近似分布:
\begin{equation}
	H\sim\chi^2_{(k-1)}\notag
\end{equation}
\subsubsection{打结}
若存在打结的情况,将H作以下修正,然后利用大样本近似公式进行计算:
\begin{equation}
	H_C=\frac{H}{1-\sum\limits_{i=1}^g(\tau_i^3-\tau_i)/(N^3-N)}\notag
\end{equation}
\subsubsection{代码}
这是R语言stats包中的函数,只提供大样本近似,并且不包含连续性修正。x表示各水平的response值,g对应于factor标签。也可以只传入x,此时x需要是一个包含所有水平响应值的列表,各水平之间分隔开。
\begin{minted}[bgcolor=white, linenos, frame=single, numbersep=5pt, breaklines, mathescape]{r}
kruskal.test(x, g)
\end{minted}
\hspace{2em}以下是自编代码,x表示各水平的response值,y对应于factor标签。也可以只传入x,此时x需要是一个数据框,第一列是response值,第二列是factor标签。提供精确检验、大样本近似、连续性修正与打结校正功能。精确检验需要gtools包,同时测试了一下一个水平8个数据一共24个数据permutation就要占到14个G的内存以上,慎用精确检验。
\inputminted[bgcolor=white, linenos, frame=single, numbersep=5pt, breaklines]{r}{nonparametric-statistics/chapter3/kruskal-wallis.R}


\section{Jonckheere-Terpstra检验}

\subsubsection{目的}
检验水平的位置参数是否呈现出上升趋势,若想检验是否呈现出下降趋势,改变水平顺序就行了。
\subsubsection{适用条件}
各响应值在水平间和水平内是独立的,水平之间分布是相似的,数据是连续型的。这里与Kruskal-Wallis检验的条件是一样的。
\subsubsection{假设}
假设$k$个水平有分布函数$F_i(x)=F(x-\theta_i),\;i=1,2,\dots,k$,则检验假设可写为:
\begin{equation}
	H_0:\theta_1=\theta_2=\cdots=\theta_n\Leftrightarrow
	H_1:\theta_1\leqslant\theta_2\leqslant\cdots\leqslant\theta_n\notag
\end{equation}
\subsubsection{原理}
假设一共有$k$个水平,每个水平中有$n_i$个响应值($i=1,2,\dots,k$),响应值用$X_{ij}$表示($i=1,2,\dots,k,\;j=1,2,\dots,n_i$),由此构建以下JT统计量(J):
\begin{gather*}
	U_{ij}=\left|X_{ia}<X_{jb},\;a=1,2,\dots,n_i,\;b=1,2,\dots,n_j\right| \\
	J=\sum_{i<j}U_{ij}
\end{gather*}
\hspace{2em}在备择假设成立的情况下,某个水平中的观测值会比后面水平中的观测值小,水平间的$U_{ij}$会比较大,$J$也会比较大。因此在$J$比较大的时候,有理由怀疑零假设。由此可看出这里只考虑上侧的单侧检验问题。\par
若想求精确检验的结果,需要满足数据中没有结(即所有数据中没有相同的数值),然后对每一种秩分配情况计算$J$的值(秩的大小关系便反映了数据之间的大小关系),即可得到$J$的精确分布。
\subsubsection{大样本近似}
在$\min\limits_in_i\to+\infty$时,有以下近似公式:
\begin{equation}
	Z=\frac{J-(N^2-\sum\limits_{i=1}^k)/4}{\sqrt{[N^2(2N+3)-\sum\limits_{i=1}^kn_i^2(2n_i+3)]/72}}\sim N(0,\;1)\notag
\end{equation}
\subsubsection{打结}
当数据中存在相同数值时,要进行修正:
\begin{gather*}
	U_{ij}=\left|X_{ia}<X_{jb},\;a=1,2,\dots,n_i,\;b=1,2,\dots,n_j\right|+ \\
	\frac{1}{2}\left|X_{ia}=X_{jb},\;a=1,2,\dots,n_i,\;b=1,2,\dots,n_j\right| \\
	J=\sum_{i<j}U_{ij}
\end{gather*}
\subsubsection{代码}
x表示各水平的response值,y对应于factor标签。也可以只传入x,此时x需要是一个数据框,第一列是response值,第二列是factor标签。提供精确检验、大样本近似、连续性修正与打结校正功能。精确检验需要gtools包,同时测试了一下一个水平8个数据一共24个数据permutation就要占到14个G的内存以上,慎用精确检验。
\inputminted[bgcolor=white, linenos, frame=single, numbersep=5pt, breaklines]{r}{nonparametric-statistics/chapter3/jt.R}
\section{Friedman秩和检验}

\subsubsection{适用条件}
各水平间并不独立(因为还有区组的影响),采用完全区组设计,水平之间分布是相似的,连续型与离散型数据都可以。
\subsubsection{假设}
假设$k$个水平有分布函数$F_i(x)=F(x-\theta_i),\;i=1,2,\dots,k$,则检验假设可写为:
\begin{equation}
	H_0:\theta_1=\theta_2=\cdots=\theta_n\Leftrightarrow
	H_1:\text{至少有一个等号不成立}\notag
\end{equation}
\subsubsection{原理}
假设有$k$个水平、$b$个区组。\label{sec:friedman检验原理}\par
因为各区组之间是有影响的,无法把各响应值混在一起排序。选择在各个区组内计算所有响应值的秩,$R_{ij}$表示在第$j$个区组中水平$i$的秩,$R_i=\sum\limits_{j=1}^bR_{ij},\;i=1,2,\dots,k$,定义如下Friedman统计量:
\begin{equation}
	Q=\frac{12}{bk(k+1)}\sum_{i=1}^k(R_i-\frac{b(k+1)}{2})^2=\frac{12}{bk(k+1)}\sum_{i=1}^kR_i^2-3b(k+1)\notag
\end{equation}
\hspace{2em}易证$\frac{b(k+1)}{2}=\bar{R_i}$(只需注意此时秩是针对区组内而言)。在零假设成立的情况下,各水平之间的秩和与均值相比不应相差过大,也就是$Q$值不应太大,若$Q$值过大,则有理由怀疑零假设。由此可看出这里只考虑上侧的单侧检验问题。
\subsubsection{大样本近似}
在大样本的情况下($b\to+\infty$),若零假设成立,有如下近似分布:
\begin{equation}
	Q\sim\chi^2_{(k-1)}\notag
\end{equation}
\subsubsection{打结}
在某个区组存在结的时候,利用下式进行修正(其中$\tau_{ij}$表示第$j$个区组的第$i$个结统计量):
\begin{equation}
	Q_C=\frac{Q}{1-C},\;C=\frac{\sum\limits_{i,\;j}(\tau_{ij}^3-\tau_{ij})}{bk(k^2-1)}\notag
\end{equation}
\subsubsection{成对数据的比较}
类似于邓肯多重比较法,有时需要比较某两个水平之间是否存在差异,那么在大样本的情况下,如果零假设为:$i$水平与$j$水平之间没有差异,那么如果下式成立(其中$\alpha$是检验的显著性水平):
\begin{gather*}
	\left|R_i-R_j>Z_{\frac{\alpha^*}{2}}\sqrt{b(k+1)k/6}\right| \\
	\alpha^*=\frac{\alpha}{k(k-1)/2}
\end{gather*}
则可拒绝零假设。可以看出这是一个很保守的检验,$\alpha^*$其实是做了多重假设检验的校正。\info{记得以后要写多重假设检验的校正问题}
\subsubsection{代码}
以下是自编代码,提供精确计算、大样本近似、连续性修正与打结校正功能。x可以是一个三列的数据框,第一列表示response值,第二列表示factor,第三列表示block。x也可以是一个向量,表示response,此时必须传入factor和block。
\inputminted[bgcolor=white, linenos, frame=single, numbersep=5pt, breaklines]{r}{nonparametric-statistics/chapter3/friedman-test.R}
\section{Page检验}

\subsubsection{目的}
检验水平的位置参数是否呈现出上升趋势,若想检验是否呈现出下降趋势,改变水平顺序就行了。
\subsubsection{适用条件}
各水平间并不独立(因为还有区组的影响),采用完全区组设计,水平之间分布是相似的,离散型数据与连续型数据都可以。
\subsubsection{假设}
假设$k$个水平有分布函数$F_i(x)=F(x-\theta_i),\;i=1,2,\dots,k$,则检验假设可写为:
\begin{equation}
	H_0:\theta_1=\theta_2=\cdots=\theta_n\Leftrightarrow
	H_1:\theta_1\leqslant\theta_2\leqslant\cdots\leqslant\theta_n\notag
\end{equation}
\subsubsection{原理}
假设有$k$个水平、$b$个区组。\par
因为各区组之间是有影响的,无法把各响应值混在一起排序。选择在各个区组内计算所有响应值的秩,$R_{ij}$表示在第$j$个区组中水平$i$的秩,$R_i=\sum\limits_{j=1}^bR_{ij},\;i=1,2,\dots,k$,定义如下Page统计量:
\begin{equation}
	L=\sum_{i=1}^kiR_i\notag
\end{equation}
如果备择假设是正确的,那么对$R_i$进行加权求和可以对统计量起到一个放大的作用,那么$L$就会很大。因此在$L$比较大的时候,有理由怀疑零假设。由此可看出这里只考虑上侧的单侧检验问题。\par
若想求精确检验的结果,需要满足数据中没有结(即每个区组中都没有相同的数值),然后对每一种秩分配情况计算$L$的值(秩的大小关系便反映了数据之间的大小关系),即可得到$L$的精确分布。
\subsubsection{大样本近似}
在大样本的情况下($b\to+\infty$),有如下正态近似:
\begin{gather*}
	Z=\frac{L-\mu_L}{\sigma_L}\sim N(0,\;1) \\
	\mu_L=\frac{bk(k+1)^2}{4},\;\sigma_L^2=\frac{b(k^3-k)^2}{144(k-1)}
\end{gather*}
\subsubsection{打结}
若存在打结的情况,需要对正态近似的$\sigma_L^2$作如下修正(其中$\tau_{ij}$表示第$j$个区组的第$i$个结统计量):
\begin{equation}
	\sigma_L^2=k(k^2-1)\frac{bk(k^2-1)-\sum_i\sum_j(\tau_{ij}^3-\tau_{ij})}{144(k-1)}\notag
\end{equation}
\subsubsection{对区组、水平进行重复时的page检验}
在区组和水平之间不存在交互作用,并且所有$(i,\;j)$位置的重复数都相同时(假设都是$n$),有如下正态近似(也考虑了到打结的修正):
\begin{gather*}
	Z=\frac{L-\mu_L}{\sigma_L}\sim N(0,\;1) \\
	\mu_L=\frac{nbk(k+1)(nk+1)}{4} \\
	\sigma_L^2=nk(k^2-1)\frac{nbk(n^2k^2-1)-\sum_i\sum_j(\tau_{ij}^3-\tau_{ij})}{144(nk-1)}
\end{gather*}
\subsubsection{代码}
以下是自编代码,提供重复、精确计算、大样本近似、连续性修正与打结校正功能。当对区组与水平进行重复时,x必须是一个数据框,每一列是一次重复的结果,也必须传入factor与block,此时只提供大样本近似。当不进行重复时,x可以是一个三列的数据框,第一列表示response值,第二列表示factor,第三列表示block。请注意,检验顺序与输入的factor顺序是一致的,建议对照二维表\textbf{逐行}输入(行是水平,检验顺序即为行的顺序,列是区组)。如果出现打结的情况,结统计量按照计算公式进行计算,$L$统计量的秩部分取结的平均秩(例:数据为1、1、3、4,有重复值1,则秩为1.5、1.5、3、4)。
\inputminted[bgcolor=white, linenos, frame=single, numbersep=5pt, breaklines]{r}{nonparametric-statistics/chapter3/page.R}
\section{Cochran检验}

\subsubsection{目的}
检验完全区组设计、二元响应情况下,各水平之间是否存在差异。
\subsubsection{适用条件}
完全区组设计,二元响应。
\subsubsection{假设}
假设$k$个水平有分布函数$F_i(x)=F(x-\theta_i),\;i=1,2,\dots,k$,则检验假设可写为:
\begin{equation}
	H_0:\theta_1=\theta_2=\cdots=\theta_n\Leftrightarrow
	H_1:\text{至少有一个等号不成立}\notag
\end{equation}
\subsubsection{原理}
假设有$k$个水平、$b$个区组。\par
令$L_j$表示第$j$个区组中为$1$的数目的总和,$N_i$表示第$i$个水平中为$1$的数目的总和,即$L_j=\sum\limits_{i=1}^kx_{ij},\;j=1,2,\;b,\;N_i=\sum\limits_{j=1}^bx_{ij},\;i=1,2,\;k$。\par
在零假设成立的情况下,各水平之间无差异,那么对于每个$j$而言,$L_j$个$1$在
所有水平中出现的概率是相同的,这个概率依赖于具体的$L_j$。由此定义如下Cochran统计量:
\begin{equation}
	Q=\frac{k(k-1)\sum\limits_{i=1}^k(N_i-\bar{N})^2}{kN-\sum\limits_{j=1}^bL_j^2}=\frac{k(k-1)\sum\limits_{i=1}^kN_i^2-(k-1)N^2}{kN-\sum\limits_{j=1}^bL_j^2}\notag
\end{equation}
其中$\bar{N}=\dfrac{1}{k}\sum\limits_{i=1}^kN_i,\;N=\sum\limits_{i=1}^kN_i$。易证,若$L_j=0$或$L_j=k$,那么这个区组的数据可以删除,对$Q$值没有影响(与前面的原理是相合的)。\par
若想进行精确检验\info{需要找论文,Patil(1975),同时需要看统计量的推导过程,为什么单侧检验}
\subsubsection{大样本近似}
在大样本的情况下($b\to+\infty$),若零假设成立,有如下近似分布:
\begin{equation}
	Q\sim\chi^2_{(k-1)}\notag
\end{equation}
\subsubsection{代码}
\info{现在只提供了大样本近似,回头看完论文和原理记得补}以下是自编代码,需要传入一个数据框,行为水平列为区组。提供大样本近似与连续性修正。
\inputminted[bgcolor=white, linenos, frame=single, numbersep=5pt, breaklines]{r}{nonparametric-statistics/chapter3/cochran.R}
\section{Durbin检验}

\subsubsection{适用条件}
不完全区组设计,水平之间分布是相似的,数据是连续型的。
\subsubsection{假设}
假设$k$个水平有分布函数$F_i(x)=F(x-\theta_i),\;i=1,2,\dots,k$,则检验假设可写为:
\begin{equation}
	H_0:\theta_1=\theta_2=\cdots=\theta_n\Leftrightarrow
	H_1:\text{至少有一个等号不成立}\notag
\end{equation}
\subsubsection{原理}
假设有$k$个水平、$b$个区组,每个区组中含$t$个处理,每个处理出现在$r$个区组中\info{有机会看看这里平衡与不平衡的不完全区组设计有没有什么区别}。\par
因为各区组之间是有影响的,无法把各响应值混在一起排序。选择在各个区组内计算所有响应值的秩,$R_{ij}$表示在第$j$个区组中水平$i$的秩,$R_i=\sum\limits_{j}R_{ij},\;i=1,2,\dots,k$,定义如下Durbin统计量:
\begin{equation}
	D=\frac{12(k-1)}{rk(t^2-1)}\sum_{i=1}^k(R_i-\frac{r(t+1)}{2})^2=\frac{12(k-1)}{rk(t^2-1)}\sum_{i=1}^kR_i^2-\frac{3r(k-1)(t+1)}{t-1}\notag
\end{equation}
\hspace{2em}在零假设成立的情况下,各水平之间的秩和与均值相比不应相差过大,也就是$D$值不应太大,若$D$值过大,则有理由怀疑零假设。由此可看出这里只考虑上侧的单侧检验问题。同时,可以看出Durbin统计量在完全区组设计($t=k,\;r=b$)的时候和Friedman统计量是完全一样的。
\subsubsection{大样本近似}
在大样本的情况下($r\to+\infty$),若零假设成立,有如下近似分布:
\begin{equation}
	D\sim\chi^2_{(k-1)}\notag
\end{equation}
在某个区组存在结的时候,利用下式进行修正(其中$\tau_{ij}$表示第$j$个区组的第$i$个结统计量):
\begin{gather*}
	D_C=\frac{(k-1)\sum_{i=1}^k(R_i-\frac{r(t+1)}{2})^2}{A-C} \\
	A=\sum_i\sum_jR_{ij}^2,\;C=\frac{bt(t+1)^2}{4}
\end{gather*}

\subsubsection{代码}
以下是自编代码,会检查输入的数据是否满足不完全区组设计的平衡性。\info{代码现在是只考虑平衡的}提供精确计算、大样本近似、连续性修正与打结校正功能。x可以是一个三列的数据框,第一列表示response值,第二列表示factor,第三列表示block。x也可以是一个向量,表示response,此时必须传入factor和block。
\inputminted[bgcolor=white, linenos, frame=single, numbersep=5pt, breaklines]{r}{nonparametric-statistics/chapter3/durbin.R}