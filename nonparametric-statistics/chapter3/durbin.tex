\section{Durbin检验}

\subsubsection{适用条件}
不完全区组设计,水平之间分布是相似的,数据是连续型的。
\subsubsection{假设}
假设$k$个水平有分布函数$F_i(x)=F(x-\theta_i),\;i=1,2,\dots,k$,则检验假设可写为:
\begin{equation}
	H_0:\theta_1=\theta_2=\cdots=\theta_n\Leftrightarrow
	H_1:\text{至少有一个等号不成立}\notag
\end{equation}
\subsubsection{原理}
假设有$k$个水平、$b$个区组,每个区组中含$t$个处理,每个处理出现在$r$个区组中\info{有机会看看这里平衡与不平衡的不完全区组设计有没有什么区别}。\par
因为各区组之间是有影响的,无法把各响应值混在一起排序。选择在各个区组内计算所有响应值的秩,$R_{ij}$表示在第$j$个区组中水平$i$的秩,$R_i=\sum\limits_{j}R_{ij},\;i=1,2,\dots,k$,定义如下Durbin统计量:
\begin{equation}
	D=\frac{12(k-1)}{rk(t^2-1)}\sum_{i=1}^k(R_i-\frac{r(t+1)}{2})^2=\frac{12(k-1)}{rk(t^2-1)}\sum_{i=1}^kR_i^2-\frac{3r(k-1)(t+1)}{t-1}\notag
\end{equation}
\hspace{2em}在零假设成立的情况下,各水平之间的秩和与均值相比不应相差过大,也就是$D$值不应太大,若$D$值过大,则有理由怀疑零假设。由此可看出这里只考虑上侧的单侧检验问题。同时,可以看出Durbin统计量在完全区组设计($t=k,\;r=b$)的时候和Friedman统计量是完全一样的。
\subsubsection{大样本近似}
在大样本的情况下($r\to+\infty$),若零假设成立,有如下近似分布:
\begin{equation}
	D\sim\chi^2_{(k-1)}\notag
\end{equation}
在某个区组存在结的时候,利用下式进行修正(其中$\tau_{ij}$表示第$j$个区组的第$i$个结统计量):
\begin{gather*}
	D_C=\frac{(k-1)\sum_{i=1}^k(R_i-\frac{r(t+1)}{2})^2}{A-C} \\
	A=\sum_i\sum_jR_{ij}^2,\;C=\frac{bt(t+1)^2}{4}
\end{gather*}

\subsubsection{代码}
以下是自编代码,会检查输入的数据是否满足不完全区组设计的平衡性。\info{代码现在是只考虑平衡的}提供精确计算、大样本近似、连续性修正与打结校正功能。x可以是一个三列的数据框,第一列表示response值,第二列表示factor,第三列表示block。x也可以是一个向量,表示response,此时必须传入factor和block。
\inputminted[bgcolor=white, linenos, frame=single, numbersep=5pt, breaklines]{r}{nonparametric-statistics/chapter3/durbin.R}