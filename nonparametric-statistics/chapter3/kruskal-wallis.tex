\section{Kruskal-Wallis检验}

\subsubsection{适用条件}
各响应值在水平间和水平内是独立的,水平之间分布是相似的,数据是连续型的。
\subsubsection{假设}
假设$k$个水平有分布函数$F_i(x)=F(x-\theta_i),\;i=1,2,\dots,k$,则检验假设可写为:
\begin{equation}
	H_0:\theta_1=\theta_2=\cdots=\theta_n\Leftrightarrow
	H_1:\text{至少有一个等号不成立}\notag
\end{equation}
\subsubsection{原理}
将所有水平的响应值混合后排序,得到每一个响应值对应于所有数据的秩,类似于方差分析中MSA的构成,若水平间的秩和差异大,则应怀疑零假设。由此构建以下Kruskal-Wallis统计量(其中$\bar{R_i}$表示第$i$个水平秩的平均值,$\bar{R}$表示所有响应值秩的平均值):
\begin{equation}
	H=\frac{12}{N(N+1)}\sum_{i=1}^kn_i(\bar{R_i}-\bar{R})^2=\frac{12}{N(N+1)}\sum_{i=1}^k\frac{R_i^2}{n_i}-3(N+1)\notag
\end{equation}
\hspace{2em}如果备择假设成立,那么统计量的值应该是偏大的,因此只考虑上侧的单侧检验问题。\par
若想求精确结果,需要满足数据中不存在结(即数据中不存在相同的数值),对每种秩分配计算对应的$H$值,便能得到此时统计量$H$的分布。\par
\subsubsection{大样本近似}
在大样本的情况下,若零假设成立,有如下近似分布:
\begin{equation}
	H\sim\chi^2_{(k-1)}\notag
\end{equation}
\subsubsection{打结}
若存在打结的情况,将H作以下修正,然后利用大样本近似公式进行计算:
\begin{equation}
	H_C=\frac{H}{1-\sum\limits_{i=1}^g(\tau_i^3-\tau_i)/(N^3-N)}\notag
\end{equation}
\subsubsection{代码}
这是R语言stats包中的函数,只提供大样本近似,并且不包含连续性修正。x表示各水平的response值,g对应于factor标签。也可以只传入x,此时x需要是一个包含所有水平响应值的列表,各水平之间分隔开。
\begin{minted}[bgcolor=white, linenos, frame=single, numbersep=5pt, breaklines, mathescape]{r}
kruskal.test(x, g)
\end{minted}
\hspace{2em}以下是自编代码,x表示各水平的response值,y对应于factor标签。也可以只传入x,此时x需要是一个数据框,第一列是response值,第二列是factor标签。提供精确检验、大样本近似、连续性修正与打结校正功能。精确检验需要gtools包,同时测试了一下一个水平8个数据一共24个数据permutation就要占到14个G的内存以上,慎用精确检验。
\inputminted[bgcolor=white, linenos, frame=single, numbersep=5pt, breaklines]{r}{nonparametric-statistics/chapter3/kruskal-wallis.R}

