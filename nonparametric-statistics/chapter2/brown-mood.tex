\section{Brown-Mood中位数检验}

\subsubsection{目的}
检验两个样本X和Y的中位数$M_X,\;M_Y$是否相同。
\subsubsection{原理}
记样本X与Y分别为$X_1,X_2,\dots,X_m,\;Y_1,Y_2,\dots,Y_n$。
将两样本混合求得混合后的中位数$M_{XY}$。若两样本中位数相同,则混合后的中位数应也等于二者的中位数,那么两个样本中大于$M_{XY}$的单元的个数应该大致一样。可构建如下表格:

\begin{table}[htbp]
	\centering
	\begin{tabular}{cccc}
		\toprule 
		 & X样本 & Y样本 & 总和 \\
		\midrule 
		单元大于$M_{XY}$的个数 & a & b & p \\
		单元小于$M_{XY}$的个数 & c & d & q \\
		总和                  & m & n & N \\
		\bottomrule 
	\end{tabular}
	\caption{$2*2$列联表矩阵}
\end{table}

令A表示表中a背后的随机变量,在m、n、p固定的情况下,A服从超几何分布H$(p,m,N)$(在四个边际和中任意三个固定的情况下,都有$2*2$列联表中的一个值其背后的随机变量服从超几何分布),有如下概率:
\begin{equation}
	P(A=k)=\frac{\binom{m}{k}\binom{n}{p-k}}{\binom{N}{p}}\notag
\end{equation}
\hspace{2em}若A过大或过小,考虑到此时p是固定的,那么X样本中大于$M_{XY}$单元的个数离Y样本中大于$M_{XY}$单元的个数就会很远,就应怀疑零假设。
\subsubsection{零假设}
$H_0:M_X=M_Y$

\begin{table}[htbp]
	\centering
	\begin{tabular}{cc}
		\toprule
		备择假设 & $p$值 \\
		\midrule 
		$H_1:M_X>M_Y$ & $P(A\geqslant a)$ \\
		$H_1:M_X<M_Y$ & $P(A\leqslant a)$ \\
		$H_1:M_X\ne M_Y$ & 2\text{min}\{$P(A\geqslant a)$,\;$P(A\leqslant a)$\} \\
		\bottomrule 
	\end{tabular}
	\caption{对$H_0:M_X=M_Y$的检验}
\end{table}

\subsubsection{大样本近似}
由超几何分布的期望与方差,在大样本情况下可认为有如下关系:
\begin{equation}
	Z=\frac{A - \frac{pm}{N}}{\sqrt{\frac{pm(N-p)(N-m)}{N^3}}} \sim N(0,1)\notag
\end{equation}
\hspace{2em}对于双边假设检验,在大样本情况下,可选取检验统计量与$p$值如下:
\begin{equation}
	K=\frac{(2a-m)^2(m+n)}{mn}\sim\chi^2(1),\;
	p=P(K\geqslant k)\notag
\end{equation}
\subsubsection{注意事项}
若有和$M_{XY}$值相同的单元,可将其去除,也可将其归于任何一类使得检验变得保守一点。
\subsubsection{代码}
以下代码需要输入x满足一定的格式:数据是两列,第一列为观测值,第二列用$1,2$来表示其属于X或Y样本,1表示属于X样本。以下代码也提供了样本中有和$M_{XY}$值相同的单元情况下的处理方案,omit表示去除与$M_{XY}$值相同的单元,less表示将其归于小于$M_{XY}$值的一类,greater表示将其归于大于$M_{XY}$值值的一类。
\inputminted[bgcolor=white, linenos, frame=single, numbersep=5pt, breaklines]{r}{nonparametric-statistics/chapter2/brown-mood.R}