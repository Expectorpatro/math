\section{单样本位置检验}

\begin{method}[Sign Test]
	设$(X,\mathscr{A},\mathscr{P})$是统计结构,$\mathbf{X}=(\seq{X}{n})$为从总体$F$中抽取的简单样本,$\mathscr{P}\mathbf{X}^{-1}$为连续型随机变量的概率分布族,检验问题为:
	\begin{equation*}
		H_0:Q_\pi=q_0,\quad H_1:
		\begin{cases}
			Q_\pi>q_0 \\
			Q_\pi<q_0 \\
			Q_\pi\ne q_0
		\end{cases}
	\end{equation*}
	记$\mathbf{X}$中大于$q_0$的样本单元的个数为$S^+$,小于$q_0$的个数为$S^-$。令$K\sim\operatorname{Binom}(S^++S^-,\pi)$,则上述检验问题的$p$值分别为:
	\begin{equation*}
		P_{H_0}(K\leqslant S^-),\quad P_{H_0}(K\geqslant S^-),\quad 2\min\{P_{H_0}(K\leqslant S^-),\;P_{H_0}(K\geqslant S^-)\}
	\end{equation*}
	当$n\to+\infty$时,令$Z=\dfrac{K-(S^++S^-)\pi}{\sqrt{(S^++S^-)\pi(1-\pi)}}$,$u$为标准正态分布y则有如下的近似$p$值:
	\begin{equation*}
		P(u)
	\end{equation*}
\end{method}
\begin{derivation}
	若零假设成立,设$K$为$s^-$的随机变量形式,则应有$K\sim\operatorname{Binom}(S^++S^-,\pi)$。直观上来讲,如果$Q_{\pi}>q_0$,那么理论计算的$K$的实现值将比实际原假设的拒绝域应具有如下形式:
	\begin{equation*}
		\{K\leqslant c\},\quad\{K\geqslant c\},\quad
	\end{equation*}
\end{derivation}

\subsubsection{适用条件}
总体是连续型的。离散型分位点的定义会导致下述统计量不服从对应的二项分布。
\subsubsection{原理}
。若零假设成立(即$q_0$确实为总体的$\pi$分位点),则从总体中任取一个元素,它小于$q_0$的概率应为$\pi$,那么对于任意一组样本来讲,

\subsubsection{大样本近似}
当$n$较大时,可认为$Z\sim N(0,1)$,其中$n$为样本中不等于$q_0$的单元的个数,由此可在求$p$值时近似计算标准正态分布的累积概率值。
\subsubsection{注意事项}
需要去除样本中值为$q_0$的单元,这部分单元对推断没有帮助。
\subsubsection{代码}
\label{sec:sign.test.code}
\inputminted[bgcolor=white, linenos, frame=single, numbersep=5pt, breaklines]{r}{nonparametric-statistics/chapter1/sign-test.R}