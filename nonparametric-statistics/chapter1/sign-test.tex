\section{广义符号检验}

\subsubsection{目的}
给定某样本,检验给定值$q_0$与总体的$\pi$分位点$Q_\pi$之间的关系。
\subsubsection{适用条件}
总体是连续型的。离散型分位点的定义会导致下述统计量不服从对应的二项分布。
\subsubsection{原理}
记一组样本中大于$q_0$的单元的个数为$s^+$,小于$q_0$的单元的个数为$s^-$。若零假设成立(即$q_0$确实为总体的$\pi$分位点),则从总体中任取一个元素,它小于$q_0$的概率应为$\pi$,那么对于任意一组样本来讲,应有$K\sim\text{Binom}(s^++s^-,\pi)$,其中$K$为$s^-$背后的随机变量。
\subsubsection{零假设}
$H_0:Q_\pi=q_0$

\begin{table}[htbp]
	\centering
	\begin{tabular}{c>{\centering\arraybackslash}p{6cm}>{\centering\arraybackslash}p{4cm}}
		\toprule 
		备择假设 & $p$值 & 解释 \\
		\midrule 
		$H_1:Q_\pi>q_0$ & $P_{H_0}(K\leqslant s^-)$ &
		在零假设下,二项分布变量小于当前样本$s^-$的概率太小,说明$s^-$太小应变大,即分位点的值应比 N大 \\
		$H_1:Q_\pi<q_0$ & $P_{H_0}(K\geqslant s^-)$ & 略 \\
		$H_1:Q_\pi\ne q_0$ & 
		2\text{min}\{$P_{H_0}(K\leqslant s^-)$,\;$P_{H_0}(K\geqslant s^-)$\} & 略 \\
		\bottomrule 
	\end{tabular}
	\caption{对$H_0:Q_\pi=q_0$的检验}
\end{table}

\subsubsection{大样本近似}
当$n$较大时,可认为$Z\sim N(0,1)$,其中$Z=\dfrac{K-n\pi}{\sqrt{n\pi(1-\pi)}}$,其中$n$为样本中不等于$q_0$的单元的个数,由此可在求$p$值时近似计算标准正态分布的累积概率值。
\subsubsection{注意事项}
需要去除样本中值为$q_0$的单元,这部分单元对推断没有帮助。
\subsubsection{尚存疑惑}
为什么$p$值是这样的呢?
\subsubsection{代码}
\label{sec:sign.test.code}
\inputminted[bgcolor=white, linenos, frame=single, numbersep=5pt, breaklines]{r}{nonparametric-statistics/chapter1/sign-test.R}