\section{Wilcoxon符号秩检验}

\subsubsection{目的}
给定某样本,检验给定值$M_0$与总体的中位数$M$之间的关系。
\subsubsection{适用条件}
总体分布是连续且对称的。连续是为了避免样本中出现重复值影响秩分配,对称则是秩推断成立的必要条件,见下文\ref{sec:符号秩检验要求总体对称的原因}。
\subsubsection{符号秩的计算}
设样本中每个单元的取值为$X_i$,$i=1,2,\dots,n$。
\begin{enumerate}[leftmargin=*, labelindent=2em]
	\item 对$i=1,2,\dots,n$,计算$d_i=|X_i-M_0|$。
	\item 对$d_i$进行排序,找出它们的秩$R_i$。
	\item 计算$W^+=\sum\limits_{i\in\{i:X_i-M_0>0\}}R_i,\;W^-=\sum\limits_{i\in\{i:X_i-M_0<0\}}R_i$。
\end{enumerate}
\subsubsection{原理}
由于总体是对称的,那么对于任意一个元素,首先它出现在$M_0$两边的可能性是相等的,其次它出现在$M_0$两边对称位置的可能性是相等的\label{sec:符号秩检验要求总体对称的原因},由此在零假设成立的条件下,$W^+$和$W^-$应相差不大,且二者满足关系:$W^+ + W^- = \frac{n(n+1)}{2}$,因此,当其中之一很小时,应怀疑零假设。若想对$p$值精确求解,见\info{在概率论里完善Wilcoxon秩统计量的分布相关理论,然后链接过来}。
\subsubsection{零假设}
$H_0: M = M_0$

\begin{table}[htbp]
	\centering
	\begin{tabular}{ccc}
		\toprule
		备择假设 & 统计量$W$ & $p$值 \\
		\midrule 
		$H_1:M>M_0$ & $W^-$ & $P(W\leqslant w)$ \\
		$H_1:M<M_0$ & $W^+$ & $P(W\leqslant w)$ \\
		$H_1:M\ne M_0$ & \text{min}\{$W^-$,\;$W^+$\} & $2P(W\leqslant w)$ \\
		\bottomrule 
	\end{tabular}
	\caption{Wilcoxon符号秩检验}
\end{table}
	
\subsubsection{大样本近似}
可求出Wilcoxon符号秩检验统计量的期望和方差分别为:
\begin{equation}
	E(W)=\frac{n(n+1)}{4}\qquad Var(W)=\frac{n(n+1)(2n+1)}{24}\notag
\end{equation}
\hspace{2em}当$n$较大时,由中心极限定理,可认为$Z\sim N(0,1)$,其中$Z=\dfrac{W-E(W)}{\sqrt{Var(W)}}$,由此可在求$p$值时近似计算标准正态分布的累积分布值。
\subsubsection{打结}
若数据中含有相同的数字,称之为打结的情况。结的个数为重复出现的数值的个数,结中数值的秩为它们升序排序后位置的平均值。如果结多了,大样本近似公式就不准,要进行修正,同时,如果数据中有结则必须使用大样本近似。修正后的公式为:
\begin{equation}
	Z=\frac{W-E(W)}{\sqrt{Var(W)-\frac{\sum_{i=1}^{g}(\tau_i^3-\tau_i)}{48}}}\sim N(0,1)\notag
\end{equation}
\hspace{2em}举例:对于数据$\{2,2,5,8,8.8,1,1,1,1\}$,一共有$3$个结,$\tau_1=2$(两个$2$),$\tau_2=3$(三个$8$),$\tau_3=4$(四个$1$)。称$\tau$为结统计量。
\subsubsection{代码}
\begin{minted}[bgcolor=white, linenos, frame=single, numbersep=5pt, breaklines, mathescape]{r}
wilcox.test(x, alternative = c("two.sided", "less", "greater"), mu = M0, exact = NULL, correct = TRUE)
\end{minted}