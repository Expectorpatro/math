\chapter{单调关联性}
\info{写完概率论把Pearson线性相关系数链接过来。}
单调关联性(association)与单调相关性(correlation)是否是同一个东西呢?\par
它们并不一样。单调关联性仅从秩的角度来衡量两个变量是否同增同减或一增一减,单调相关性除了增减性以外,还要求增减之间存在线性关系,即一个变量增大或减小一个单位,另一个变量是否会固定地增大或减小某个单位的数值。考虑语言习惯问题,本章仍使用相关性来称呼关联性。
\section{单调关联性的零假设与备择假设}
\begin{equation}
	H_0:X\text{和}Y\text{不相关},\;
	\begin{cases}
		H_1:X\text{和}Y\text{正相关} \\
		H_1:X\text{和}Y\text{负相关} \\
		H_1:X\text{和}Y\text{相关} \\
	\end{cases}\notag
\end{equation}
\section{本章各方法的简单总结}
\begin{enumerate}
	\item Spearman秩相关检验与Kendall秩相关检验中的$\tau_a$都可以用作连续变量的相关性检验。
	\item Kendell's$\;\tau_b$与$\tau_c$都可以用作有序分类变量的相关性检验,但在列联表行列数目$r$和$c$差别较大时,使用$\tau_c$更合适。
	\item Goodman-Kruskal's$\;\tau$检验针对分类有序变量。
	\item Somers'$\;d$检验针对分类有序变量,与前面几种方法的不同是:前几种方法的两个变量$X$和$Y$是对称的,而Somers'$\d$检验把其中一个变量看作自变量,另一个看作因变量,它可以度量自变量对因变量的影响。
\end{enumerate}
\section{Spearman秩关联检验}

\subsubsection{原理}
记$x_i$在$X$样本中的秩为$R_i$,$y_i$在$Y$样本中的秩为$S_i$,$d_i^2=(R_i-S_i)^2$,$\bar{R}=E(R)=\dfrac{1}{n}\sum_{i=1}^nR_i,\;bar{S}=E(S)=\dfrac{1}{n}\sum_{i=1}^nS_i$。\par
显然,若很多$d_i^2$很大,那么两个变量之间可能是负相关;若很多$d_i^2$很小,那么两个变量之间可能是正相关。类似Pearson相关系数,定义以下Spearman检验统计量:
\begin{equation}
	r_s=\frac{\sum_{i=1}^n(R_i-\bar{R})(S_i-\bar{S})}{\sqrt{\sum_{i=1}^n(R_i-\bar{R})^2\sum_{i=1}^n(S_i-\bar{S})^2}}=1-\frac{6\sum_{i=1}^nd_i^2}{n(n^2-1)}\notag
\end{equation}
由Cauchy不等式,显然有$-1\leqslant r_s\leqslant1$。\par
在样本不大且没有结的时候,可以使用精确检验:固定$R_i$从小到大,此时$S_i$的排序情况共有$n!$种可能,对每一种可能计算$r_s$,即可得到$r_s$的精确分布。
\subsubsection{大样本的情况}
大样本时没有近似分布,采用Monte Carlo模拟,固定随机数种子,随机抽取$m$个$S_i$可能的排序情况,对这$m$个情况计算$r_s$值,得到近似的分布。
\subsubsection{打结}
若$X$或$Y$样本中存在相同的数据,则称之为打结的情况。记$u_j,\;j=1,2,\dots,p$和$v_j,\;j=1,2,\dots,q$分别为$X$和$Y$样本中结统计量的值,记:
\begin{equation}
	U=\sum_{j=1}^p(u_j^3-u_j),\;V=\sum_{j=1}^q(v_j^3-v_j)\notag
\end{equation}
则此时修正过的Spearman检验统计量定义为:
\begin{equation}
	r_s=\frac{n(n^2-1)-6\sum_{i=1}^n(R_i-S_i)^2-6(U+V)}{\sqrt{\left[n(n^2-1)-12U \right]\left[n(n^2-1)-12V \right]}}\notag
\end{equation}
在样本量比较大时,有:
\begin{equation}
	Z=r_s\sqrt{n-1}\sim N(0,\;1)\notag
\end{equation}
有结的时候没有精确分布,只能使用上式的大样本近似。

\subsubsection{代码}
\begin{minted}[bgcolor=white, linenos, frame=single, numbersep=5pt, breaklines, mathescape]{r}
cor.test(x, y,
alternative = c("two.sided", "less", "greater"),
method = "spearman"
exact = NULL, continuity = FALSE) 
\end{minted}
\section{Kendall$\;\tau$关联检验}

\subsubsection{协同}
对于样本$X_i,Y_i,\;i=1,2,\dots,n$,从中任取两对作积$(X_i-X_j)(Y_i-Y_j)$,若乘积大于0,则称对子$(X_i,\;Y_i)$和$(X_j,\;Y_j)$是协同的(concordant),它们具有相同的倾向,若乘积小于0,则称对子是不协同的(disconcordant),它们有相反的倾向。令:
\begin{equation}
	\Psi(X_i,X_j,Y_i,Y_j)=
	\begin{cases}
		1,\quad(X_i-X_j)(Y_i-Y_j)>0 \\
		0,\quad(X_i-X_j)(Y_i-Y_j)=0 \\
		-1,\;(X_i-X_j)(Y_i-Y_j)<0 
	\end{cases}\notag
\end{equation}
\subsubsection{原理}
定义Kendall$\;\tau$相关系数为:
\begin{equation}
	\tau_a=\frac{2}{n(n-1)}\sum_{1\leqslant i<j\leqslant n}\Psi(X_i,X_j,Y_i,Y_j)=\frac{K}{\binom{n}{2}}=\frac{n_c-n_d}{\binom{n}{2}}\notag
\end{equation}
其中$n_c$表示协同的对子的数目,而$n_d$表示不协同的对子的数目。\par
由定义可以看出,$\tau_a$取值在$-1\sim 1$之间,$\tau_a$越大,协同的对子数目越多,两个变量越有可能正相关;$\tau_a$越小,不协同的对子数目越多,两个变量越有可能负相关。\par
在计算时,可以把成对数据$(X_i,\;Y_i)$按第一个变量从小到大排序,然后就可以只用$Y_i$的大小关系或者秩来计算$n_c$和$n_d$了。\par
在样本量不太大并且没有结的时候,可以按如下方法求精确检验结果:把成对数据$(X_i,\;Y_i)$按第一个变量从小到大排序,此时$Y_i$的排序情况共有$n!$种可能,对每一种可能计算$\tau_a$,即可得到$\tau_a$的精确分布。
\subsubsection{大样本近似}
在零假设成立的情况下,当$n\rightarrow +\infty$时,有如下近似分布:
\begin{equation}
	Z=K\sqrt{\frac{18}{n(n-1)(2n+5)}}\sim N(0,\;1)\notag
\end{equation}
\subsubsection{打结}
若$X$或$Y$样本中存在相同的数据,则称之为打结的情况。记$u_i,\;i=1,2,\dots,p$和$v_i,\;i=1,2,\dots,q$分别为$X$和$Y$样本中结统计量的值,此时有如下修正后的检验统计量:
\begin{equation}
	\tau_b=\frac{n_c-n_d}{\sqrt{\left[\frac{n(n-1)}{2}-\sum_iu_i(u_i-1)/2\right]\left[\frac{n(n-1)}{2}-\sum_iv_i(v_i-1)/2\right]}}\notag
\end{equation}
有结的时候没有精确分布,只能使用下式的大样本近似:
\begin{gather*}
Z=\frac{n_c-n_d}{\sqrt{\left[n(n-1)(2n+5)-t_u-t_v\right]/18+t_1+t_2}}\sim N(0,\;1) \\
t_u=\sum_iu_i(u_i-1)(2u_i+5) \\
t_v=\sum_iv_i(v_i-1)(2v_i+5) \\
t_1=\frac{1}{2n(n-1)}\sum_iu_i(u_i-1)\sum_jv_j(v_j-1) \\
t_2=\frac{1}{9n(n-1)(n-2)}\sum_iu_i(u_i-1)(u_i-2)\sum_jv_j(v_j-1)(v_j-2)
\end{gather*}
\subsubsection{有序分类变量情况下的$\tau_c$}
假设$X$和$Y$是有序分类变量,分别由$r$个和$c$个有序水平,将观测数据的频数放入一个列联表中,令$n_ij,\;i=1,2,\dots,r,\;j=1,2,\dots,c$为列联表中的对应元素,则Kendall's$\;\tau_c$的定义和其渐进均方差为:
\begin{gather*}
	\tau_c=\frac{2q(n_c-n_d)}{n^2(q-1)} \\
	\text{ASE}=\frac{2q}{(q-1)n^2}\sqrt{\sum_{ij}n_{ij}(C_{ij}-D_{ij})^2-4(n_c-n_d)^2/n} \\
	q=\text{min}(r,c) \\
	C_{ij}=\sum_{i'>i}\sum_{j'>j}n_{i'j'}+\sum_{i'<i}\sum_{j'<j}n_{i'j'} \\
	D_{ij}=\sum_{i'>i}\sum_{j'<j}n_{i'j'}+\sum_{i'<i}\sum_{j'>j}n_{i'j'}
\end{gather*}
其取值范围也在$-1\sim 1$之间,同时有如下大样本近似:
\begin{equation}
	\frac{\tau_c}{\text{ASE}}\sim N(0,\;1)\notag
\end{equation}
\subsubsection{代码}
\begin{minted}[bgcolor=white, linenos, frame=single, numbersep=5pt, breaklines, mathescape]{r}
cor.test(x, y,
alternative = c("two.sided", "less", "greater"),
method = "kendall"
exact = NULL, continuity = FALSE) 
\end{minted}
\section{Goodman-Kruskal's$\;\gamma$关联检验}

\subsubsection{原理}
假设$X$和$Y$是有序分类变量,分别由$r$个和$c$个有序水平,将观测数据的频数放入一个列联表中,令$n_ij,\;i=1,2,\dots,r,\;j=1,2,\dots,c$为列联表中的对应元素,则Goodman-Kruskal关联检验统计量$G$(也是相关系数的一个点估计)的定义和其近似分布为:
\begin{gather*}
	G=\frac{n_c-n_d}{n_c+n_d} \\
	\frac{G}{Var(G)}\sim N(0,\;1) \\
	Var(G)\approx\frac{16}{(P+Q)^4}\sum_{i,j}n_{ij}(PC_{ij}-QD_{ij})^2 \\
	P=2n_c,\;Q=2n_d \\
	C_{ij}=\sum_{i'>i}\sum_{j'>j}n_{i'j'}+\sum_{i'<i}\sum_{j'<j}n_{i'j'},\;D_{ij}=\sum_{i'>i}\sum_{j'<j}n_{i'j'}+\sum_{i'<i}\sum_{j'>j}n_{i'j'}
\end{gather*}

\subsubsection{代码}
\begin{minted}[bgcolor=white, linenos, frame=single, numbersep=5pt, breaklines, mathescape]{r}
library(DescTools)
GoodmanKruskalGamma(x, y, )
\end{minted}
\section{Sumers'$\;d$关联检验}
\subsubsection{原理}
假设$X$和$Y$是有序分类变量,分别由$r$个和$c$个有序水平,将观测数据的频数放入一个列联表中,令$n_ij,\;i=1,2,\dots,r,\;j=1,2,\dots,c$为列联表中的对应元素。\par
Somers定义了$d(C|R)$与$d(R|C)$,前者将行变量$X$看作自变量,后者把列变量$Y$看作自变量,下面仅介绍$d(C|R)$,$d(R|C)$的情况只需要把列联表转置即可得到相应的结果。$d(C|R)$的定义及其渐近均方差为:
\begin{gather*}
	d(C|R)=\frac{2(n_c-n_d)}{n(n-1)-\sum_i^rR_i(R_i-1)}=\frac{P-Q}{D_r} \\
	\text{ASE}=\frac{2}{D_r^2}\sqrt{\sum_{i,j}n_{ij}\left[D_r(C_{ij}-D_{ij})-(P-Q)(n-R_i)\right]^2} \\
	R_i=\sum_{j=1}^c{n_{ij}},\;D_r=n^2-\sum_i^rR_i^2 \\
	P=2n_c,\;Q=2n_d \\
	C_{ij}=\sum_{i'>i}\sum_{j'>j}n_{i'j'}+\sum_{i'<i}\sum_{j'<j}n_{i'j'},\;D_{ij}=\sum_{i'>i}\sum_{j'<j}n_{i'j'}+\sum_{i'<i}\sum_{j'>j}n_{i'j'}
\end{gather*}
在大样本情况下有如下近似:
\begin{equation}
	\frac{d(C|R)}{\text{ASE}}\sim N(0,\;1)\notag
\end{equation}