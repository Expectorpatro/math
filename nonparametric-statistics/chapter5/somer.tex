\section{Sumers'$\;d$关联检验}
\subsubsection{原理}
假设$X$和$Y$是有序分类变量,分别由$r$个和$c$个有序水平,将观测数据的频数放入一个列联表中,令$n_ij,\;i=1,2,\dots,r,\;j=1,2,\dots,c$为列联表中的对应元素。\par
Somers定义了$d(C|R)$与$d(R|C)$,前者将行变量$X$看作自变量,后者把列变量$Y$看作自变量,下面仅介绍$d(C|R)$,$d(R|C)$的情况只需要把列联表转置即可得到相应的结果。$d(C|R)$的定义及其渐近均方差为:
\begin{gather*}
	d(C|R)=\frac{2(n_c-n_d)}{n(n-1)-\sum_i^rR_i(R_i-1)}=\frac{P-Q}{D_r} \\
	\text{ASE}=\frac{2}{D_r^2}\sqrt{\sum_{i,j}n_{ij}\left[D_r(C_{ij}-D_{ij})-(P-Q)(n-R_i)\right]^2} \\
	R_i=\sum_{j=1}^c{n_{ij}},\;D_r=n^2-\sum_i^rR_i^2 \\
	P=2n_c,\;Q=2n_d \\
	C_{ij}=\sum_{i'>i}\sum_{j'>j}n_{i'j'}+\sum_{i'<i}\sum_{j'<j}n_{i'j'},\;D_{ij}=\sum_{i'>i}\sum_{j'<j}n_{i'j'}+\sum_{i'<i}\sum_{j'>j}n_{i'j'}
\end{gather*}
在大样本情况下有如下近似:
\begin{equation}
	\frac{d(C|R)}{\text{ASE}}\sim N(0,\;1)\notag
\end{equation}