\section{Goodman-Kruskal's$\;\gamma$关联检验}

\subsubsection{原理}
假设$X$和$Y$是有序分类变量,分别由$r$个和$c$个有序水平,将观测数据的频数放入一个列联表中,令$n_ij,\;i=1,2,\dots,r,\;j=1,2,\dots,c$为列联表中的对应元素,则Goodman-Kruskal关联检验统计量$G$(也是相关系数的一个点估计)的定义和其近似分布为:
\begin{gather*}
	G=\frac{n_c-n_d}{n_c+n_d} \\
	\frac{G}{Var(G)}\sim N(0,\;1) \\
	Var(G)\approx\frac{16}{(P+Q)^4}\sum_{i,j}n_{ij}(PC_{ij}-QD_{ij})^2 \\
	P=2n_c,\;Q=2n_d \\
	C_{ij}=\sum_{i'>i}\sum_{j'>j}n_{i'j'}+\sum_{i'<i}\sum_{j'<j}n_{i'j'},\;D_{ij}=\sum_{i'>i}\sum_{j'<j}n_{i'j'}+\sum_{i'<i}\sum_{j'>j}n_{i'j'}
\end{gather*}

\subsubsection{代码}
\begin{minted}[bgcolor=white, linenos, frame=single, numbersep=5pt, breaklines, mathescape]{r}
library(DescTools)
GoodmanKruskalGamma(x, y, )
\end{minted}