\section{Spearman秩关联检验}

\subsubsection{原理}
记$x_i$在$X$样本中的秩为$R_i$,$y_i$在$Y$样本中的秩为$S_i$,$d_i^2=(R_i-S_i)^2$,$\bar{R}=E(R)=\dfrac{1}{n}\sum_{i=1}^nR_i,\;bar{S}=E(S)=\dfrac{1}{n}\sum_{i=1}^nS_i$。\par
显然,若很多$d_i^2$很大,那么两个变量之间可能是负相关;若很多$d_i^2$很小,那么两个变量之间可能是正相关。类似Pearson相关系数,定义以下Spearman检验统计量:
\begin{equation}
	r_s=\frac{\sum_{i=1}^n(R_i-\bar{R})(S_i-\bar{S})}{\sqrt{\sum_{i=1}^n(R_i-\bar{R})^2\sum_{i=1}^n(S_i-\bar{S})^2}}=1-\frac{6\sum_{i=1}^nd_i^2}{n(n^2-1)}\notag
\end{equation}
由Cauchy不等式,显然有$-1\leqslant r_s\leqslant1$。\par
在样本不大且没有结的时候,可以使用精确检验:固定$R_i$从小到大,此时$S_i$的排序情况共有$n!$种可能,对每一种可能计算$r_s$,即可得到$r_s$的精确分布。
\subsubsection{大样本的情况}
大样本时没有近似分布,采用Monte Carlo模拟,固定随机数种子,随机抽取$m$个$S_i$可能的排序情况,对这$m$个情况计算$r_s$值,得到近似的分布。
\subsubsection{打结}
若$X$或$Y$样本中存在相同的数据,则称之为打结的情况。记$u_j,\;j=1,2,\dots,p$和$v_j,\;j=1,2,\dots,q$分别为$X$和$Y$样本中结统计量的值,记:
\begin{equation}
	U=\sum_{j=1}^p(u_j^3-u_j),\;V=\sum_{j=1}^q(v_j^3-v_j)\notag
\end{equation}
则此时修正过的Spearman检验统计量定义为:
\begin{equation}
	r_s=\frac{n(n^2-1)-6\sum_{i=1}^n(R_i-S_i)^2-6(U+V)}{\sqrt{\left[n(n^2-1)-12U \right]\left[n(n^2-1)-12V \right]}}\notag
\end{equation}
在样本量比较大时,有:
\begin{equation}
	Z=r_s\sqrt{n-1}\sim N(0,\;1)\notag
\end{equation}
有结的时候没有精确分布,只能使用上式的大样本近似。

\subsubsection{代码}
\begin{minted}[bgcolor=white, linenos, frame=single, numbersep=5pt, breaklines, mathescape]{r}
cor.test(x, y,
alternative = c("two.sided", "less", "greater"),
method = "spearman"
exact = NULL, continuity = FALSE) 
\end{minted}