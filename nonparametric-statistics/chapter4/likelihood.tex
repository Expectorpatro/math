\section{似然比检验}
\label{method:LikelihoodTest}

\subsubsection{假设}
\begin{equation*}
	H_0:\theta\in\Theta_1\quad H_1:\theta\notin\Theta_1
\end{equation*}
\subsubsection{原理}
\begin{definition}
	称:
	\begin{equation*}
		\lambda(y)=\frac{\sup\limits_{\theta\in\Theta}L(\theta;y)}{\sup\limits_{\theta\in\Theta_1}L(\theta;y)}
	\end{equation*}
	为\gls{LikelihoodRatio},其中$L(\theta;y)$为似然函数。
\end{definition}
\begin{derivation}
	若$\lambda(y)$较大,则说明$\theta\in\Theta_1$时出现数据$y$的可能性较小,于是拒绝域应形如$\{y:\lambda(y)\geqslant c\}$。可以寻找统计量$T(y)$,它是$\lambda(y)$的单调增函数,于是检验的拒绝域可取为$\{y:T(y)\geqslant c\}$。
\end{derivation}
