\section{Cox-Stuart趋势检验}

\subsubsection{目的}
查看某组数据是否有递增、递减的趋势。
\subsubsection{检验原理}
把样本分成前后两段,若样本大小为奇数,则去除最中间的单元。把前后两段相同位置的元素组成对子,以每个对子两个元素差的正负性衡量增减。
\begin{gather}
	x_1,x_2,x_3,\cdots,x_n\notag\\
	\text{对子:}(x_i,x_{i+c}),\;c=
	\begin{cases}
		n/2     & n\equiv0\mod2    \\
		(n+1)/2 & n\not\equiv0\mod2 \\
	\end{cases}\notag\\
	D_i=x_i-x_{i+c}\notag\\
	\text{令}s^+=\left|\{D_i:D_i>0\}\right|,\;s^-=\left|\{D_i:D_i<0\}\right|\notag
\end{gather}
\hspace{2em}若数据无趋势,$s^-$与$s^+$应服从$\text{Binom}(s^++s^-,0.5)$,则检验退化为符号检验。以下$p$值计算与广义符号检验略有差异,这是因为$p$取$0.5$时二项分布是对称分布。
\subsubsection{检验的数学语言(与表 \ref{table:cox-stuart}  对应)}
假设独立观测值序列$X_i\sim F(x-\theta_i),\;i=1,2,\dots,n$:
\begin{enumerate}[leftmargin=*, labelindent=2em]
	\item $H_0:\theta_1=\theta_2=\cdots=\theta_n\Leftrightarrow
	H_1:\theta_1\leqslant\theta_2\leqslant\cdots\leqslant\theta_n$
	\item $H_0:\theta_1=\theta_2=\cdots=\theta_n\Leftrightarrow
	H_1:\theta_1\geqslant\theta_2\geqslant\cdots\geqslant\theta_n$
	\item $H_0:\theta_1=\theta_2=\cdots=\theta_n\Leftrightarrow
	H_1:\theta_1\leqslant\theta_2\leqslant\cdots\leqslant\theta_n\;\text{或}\\\;\theta_1\geqslant\theta_2\geqslant\cdots\geqslant\theta_n$
\end{enumerate}

\begin{table}[htbp]
	\centering
	\begin{tabular}{ccc}
		\toprule
		备择假设 & 统计量$K$ & $p$值 \\
		\midrule 
		$H_1:\text{有增长趋势}$ & $s^+$ & $P(K\leqslant k)$ \\
		$H_1:\text{有减少趋势}$ & $s^-$ & $P(K\leqslant k)$ \\
		$H_1:\text{有趋势}$ & \text{min}\{$s^+$,\;$s^-$\} & $2P(K\leqslant k)$ \\
		\bottomrule 
	\end{tabular}
	\caption{Cox-stuart趋势检验}\label{table:cox-stuart}
\end{table}
\subsubsection{代码}
\begin{minted}[bgcolor=white, linenos, frame=single, numbersep=5pt, breaklines, mathescape]{r}
Cox_Stuart <- function(y) {
    n <- length(y)
    c <- ifelse(n %% 2 == 0, n / 2, (n + 1) / 2)
    D <- y[(c + 1):n] - y[1:c]
    s_neg <- sum(D < 0)
    s_pos <- sum(D > 0)
    s <- min(s_neg, s_pos)
    text <- ifelse(s_pos - s_neg > 0, "Increasing with", "Decreasing with")
    cat(text, 'p_value =', pbinom(s, c, 0.5))
}
\end{minted}

