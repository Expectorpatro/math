\section{分类评分结果一致性判断}

\subsection{Cohen's Kappa系数}
\info{根本没懂原理}
\subsubsection{评判标准}
\begin{enumerate}
	\item $<0.4$ 一致性很差
	\item $0.4-0.6$ 中度一致
	\item $0.6-0.8$ 具有较高的一致性
	\item $>0.8$ 具有极高的一致性
\end{enumerate}

\subsubsection{代码}
\begin{minted}[bgcolor=white, linenos, frame=single, numbersep=5pt, breaklines, mathescape]{r}
library(psych)
cohen.kappa(x)
\end{minted}

\subsection{Kendall协同系数检验}
\subsubsection{目的}
检验$b$次对$k$个个体的评估是随机的(不相关的)还是一致的。
\subsubsection{适用条件}
完全区组设计,离散型数据(打分,排序),主观性的数据一般使用Kendall检验。
\subsubsection{假设}
\begin{equation}
	H_0:\text{这些评估是不相关的或是随机的},\;H_1:\text{这些评估是一致的}\notag
\end{equation}
\subsubsection{原理}
这里其实是把$b$个评估者看作是区组,把被评分的个体看作了水平。所有的计算和Friedman秩和检验一样(参考第\pageref{sec:friedman检验原理}页),只是Kendall协同系数$W$要在Friedman统计量$Q$的基础上再除$b(k-1)$(大样本近似与打结的情况都可参考Friedman秩和检验)\info{查资料看Kendall的打结是否和Friedman一致}。\par
为什么这么做呢?我们想看的是对个体的评估是否是随机的,那么如果是随机的,各个体获得的评分总秩和应该是较相近的。其实还是看各水平之间是否有差异,但并不是它们本身的差异,而是在被评估者打分以后,在打分上是否呈现出各水平的差异。其实我们在这里是默认个体之间是有差异的,如果评估结果有差异那么就不随机,评估结果没有差异那就是随机的评估。
\subsubsection{代码}
以下是自编代码,提供精确计算、大样本近似、连续性修正与打结校正功能。x可以是一个三列的数据框,第一列表示response值,第二列表示factor,第三列表示block。x也可以是一个向量,表示response,此时必须传入factor和block。
\inputminted[bgcolor=white, linenos, frame=single, numbersep=5pt, breaklines]{r}{nonparametric-statistics/chapter4/kendall.R}