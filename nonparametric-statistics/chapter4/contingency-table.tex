\section{列联表独立性问题}

检验列联表的行列变量之间是否是独立的。
\subsubsection{假设}
\begin{equation}
	H_0:\text{行变量与列变量是独立的}\Leftrightarrow H_1:\text{行变量与列变量是相关的}\notag
\end{equation}

\subsection{Pearson近似检验}
\label{method:PearsonChisqTest}
\subsubsection{原理}
假设行变量有$r$个取值,列变量有$c$个取值,列联表中频数记为$n_{ij},\;i=1,2,\dots,r,\;j=1,2,\dots,c$,行频数总和$n_{i\cdot}=\sum_jn_{ij}$,列频数总和$n_{\cdot j}=\sum_in_{ij}$,频数总和$n=\sum_{i,j}n_{ij}$,列联表中第$ij$个格子的理论频数为$E_{ij}$,行变量取第$i$个值的概率为$p_{i\cdot}$,列变量取第$j$个值的概率为$p_{\cdot j}$,一个观测值被分配到列联表中第$ij$个格子的理论概率为$p_{ij}$。\info{写完概率论把独立性链接到这里来}\par
若行变量与列变量独立,由随机变量的独立性,有:
\begin{equation}
	p_{ij}=p_{i\cdot}p_{\cdot j},\;E_{ij}=p_{\cdot j}n_{i\cdot}\notag
\end{equation}
但由于$p_{\cdot j}$是理论值无法预知,用$\dfrac{n_{\cdot j}}{n}$来代替,那么$E_{ij}=\hat{p_{\cdot j}}n_{i\cdot}=\dfrac{n_{i\cdot}n_{\cdot j}}{n}$。由此构建以下Pearson$\chi^2$统计量:
\begin{equation}
	Q=\sum_{i=1}^r\sum_{j=1}^c\frac{(n_{ij}-E_{ij})^2}{E_{ij}}\notag
\end{equation}
\hspace{2em}如果零假设不成立,那么$n_{ij}$与$E_{ij}$的值相差就会比较大,统计量的值也会偏大。若统计量的值过大,则有里有怀疑零假设。由此可看出这里只考虑上侧的单侧检验问题。
\subsubsection{大样本近似}
在大样本的情况下,若零假设成立,有如下近似分布:
\begin{equation}
	Q\sim\chi^2_{(r-1)(c-1)}\notag
\end{equation}

\subsection{低维列联表的Fisher精确检验}
假定五个边际频数都是固定的,在零假设成立的情况下,这个具体的列联表出现的条件概率(给定边际频数的情况下,因此是条件概率)只依赖于四个频数中的任意一个,且该概率满足超几何分布:
\begin{equation}
	P=\frac{\binom{n_{1\cdot}}{n_{11}}\binom{n_{2\cdot}}{n_{21}}}{\binom{n}{n_{\cdot 1}}}\notag
\end{equation}

\begin{table}[htbp]
	\centering
	\begin{tabular}{@{}lllll@{}}
		\toprule
		     & B1           & B2           & 总和      \\ 
		\midrule
		A1   & $n_{11}$       & $n_{12}$       & $n_{1\cdot}$\\
		A2   & $n_{21}$       & $n_{22}$       & $n_{2\cdot}$\\
		总和 & $n_{\cdot 1}$  & $n_{\cdot 2}$  & $n$          \\ 
		\bottomrule
	\end{tabular}
	\caption{二维列联表}
\end{table}

若零假设成立,那么任何一个关于$n_{ij}$的尾概率$P(n_{ij}\leqslant a),\;P(n_{ij}\geqslant b),\;1-P(a<n_{ij}<b)$都不应该太小或太大。若尾概率太小或太大,则有理由怀疑零假设。由此可以看出这里其实是涉及备择假设的方向问题的,可以单侧也可以双侧。\info{有机会看看单侧的实际意义,不懂为什么会考虑单侧}\par
\subsubsection{代码}
\subsubsection{Pearson近似检验}
直接将列联表矩阵输入以下函数即可。
\begin{minted}[bgcolor=white, linenos, frame=single, numbersep=5pt, breaklines, mathescape]{r}
chisq.test(x)
\end{minted}
\subsubsection{Fisher精确检验}
直接将列联表矩阵输入以下函数即可。
\begin{minted}[bgcolor=white, linenos, frame=single, numbersep=5pt, breaklines, mathescape]{r}
fisher.test(x, alternative="two.sided")
\end{minted}