\section{游程检验}

\subsubsection{目的}
检验一组数据是否是随机出现的。
\subsubsection{适用条件}
数据的顺序有意义且是离散的,若不离散,可以中位数为间隔,利用符号函数将数据转换为二元数据。
\subsubsection{游程检验原理}
若数据是随机的,那么游程数不应过多也不应过少,若过多,则呈现出混合倾向,若过少,则呈现出聚集倾向。

\begin{table}[htbp]
	\centering
	\begin{tabular}{ccc}
		\toprule
		备择假设 & 统计量$K$ & $p$值 \\
		\midrule 
		$H_1:\text{数据有聚集趋势}$ & $R$ & $P(K\leqslant k)$ \\
		$H_1:\text{数据有混合趋势}$ & $R$ & $P(K\geqslant k)$ \\
		$H_1:\text{数据有趋势}$ & $R$ & $2\min{P(K\leqslant k),\;P(K\geqslant k)}$ \\
		\bottomrule 
	\end{tabular}
	\caption{随机性的游程检验}
\end{table}
\subsubsection{代码}
\subsubsection{tseries包中的runs.test}
需要注意,该函数只提供大样本近似,并且输入值必须转换为因子,同时,要去除经过符号函数转换后值为$0$的数据。
\begin{minted}[bgcolor=white, linenos, frame=single, numbersep=5pt, breaklines, mathescape]{r}
library(tseries)
# median <- median(x)
# x <- factor(sign(x[x != median]-median))
runs.test(x, alternative = c("two.sided", "less", "greater"))
\end{minted}
\subsubsection{自编版}
\inputminted[bgcolor=white, linenos, frame=single, numbersep=5pt, breaklines]{r}{nonparametric-statistics/chapter4/runs-test.R}