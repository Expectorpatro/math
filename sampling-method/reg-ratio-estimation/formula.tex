\section{估计量}

\subsection{回归估计量}
\begin{definition}
	比例估计量有如下计算公式(其中$\mu_X$和$\tau_X$是已知的):
	\begin{gather*}
		\hat{B}_1=\frac{\sum\limits_{i=1}^n(x_i-\bar{x})(y_i-\bar{y})}{\sum\limits_{i=1}^n(x_i-\bar{x})^2}=\frac{s_y\hat{R}}{s_x},\quad
		\hat{B}_0=\bar{y}-\hat{B}_1\bar{x} \\
		\hat{\mu}_{Y_{reg}}=\hat{B}_1\mu_X+\hat{B}_0=\hat{B}_1(\mu_X-\bar{x})+\bar{y},\quad
		\hat{\tau}_{Y_{reg}}=\hat{B}_1\tau_X+\hat{B}_0
	\end{gather*}
\end{definition}


\subsection{比例估计量}
\begin{definition}
	比例估计量有如下计算公式(其中$\mu_X$和$\tau_X$是已知的):
	\begin{gather*}
		\hat{B}=\frac{\bar{y}}{\bar{x}}=\frac{\tau_y}{\tau_x} \\
		\hat{\mu}_{Yr}=\hat{B}\mu_X,\quad\hat{\tau}_{Yr}=\hat{B}\tau_X 
	\end{gather*}
\end{definition}
其中$\hat{B}$是比例系数$B$的估计\footnote{这是一个有偏估计!!!},对于$B$的真实值,应有$B=\frac{\mu_Y}{\mu_X}=\frac{\tau_Y}{\tau_X}$。\footnote{依据辅助变量的选择原则,$X$与$Y$之间需要满足线性关系且截距为$0$,$B$其实就是线性关系中的斜率,同时证明了$Corr(\bar{x},\bar{y})=Corr(X,Y)$,因此$\hat{B}$和$B$既可以由均值来表示也可以由总体总量来表示。}