\section{总体总量$t$的估计}

\subsection{点估计}
\begin{theorem}
	标记重捕法中总体总量$\tau$的点估计如下:
	\begin{equation*}
		\hat{t}=\frac{y}{x}X 
	\end{equation*}
	它有如下性质:
	\begin{equation*}
		Var(\hat{t})=\frac{(yX)^2}{E^3(x)}\frac{(t-y)(t-X)}{t(t-1)},\quad
		\widehat{Var}(\hat{t})=\frac{Xy(X-x)(y-x)}{x^3}
	\end{equation*}
\end{theorem}
\begin{proof}
	在标记重捕法中,$x\sim H(y,X,t)$。因此可得:
	\begin{equation*}
		E(x)=\frac{yX}{t},\;
		Var(x)=\frac{yX(t-y)(t-X)}{t^2(t-1)}
	\end{equation*}
	而:
	\begin{equation*}
		\hat{t}=yX\frac{1}{x}
	\end{equation*}
	所以由\cref{sec:deltamethod}:
	\begin{align*}
		Var(\hat{t})&=Var\left(yX\frac{1}{x}\right) \\
		&=(yX)^2Var\left(\frac{1}{x}\right) \\
		&\approx(yX)^2\left[\frac{-1}{E^2(x)}\right]^2Var(x) \\
		&=\frac{(yX)^2}{E^4(x)}\frac{yX}{t}\frac{(t-y)(t-X)}{t(t-1)} \\
		&=\frac{(yX)^2}{E^3(x)}\frac{(t-y)(t-X)}{t(t-1)}
	\end{align*}
	用$x$替代$E(x)$(相当于是一次简单随机抽样,利用无偏性
	),然后考虑$t$较大时的近似,最后还剩一个$t$,用$\hat{t}$带入进行计算,即可得到:
	\begin{align*}
		\widehat{Var}(\hat{t})&=\frac{(yX)^2}{x^3}\frac{(t-y)(t-X)}{t(t-1)} \\
		&\approx\frac{(yX)^2}{x^3}\frac{(t-y)(t-X)}{t^2} \\
		&=\frac{(yX)^2}{x^3}\left(1-\frac{X}{t}\right)\left(1-\frac{y}{t}\right) \\
		&\approx\frac{Xy(X-x)(y-x)}{x^3}\qedhere
	\end{align*}
\end{proof}
\subsubsection{极端情况下的修正}
在极端情况下,$x$可能为$0$或很小,那么$\widehat{Var}(\hat{t})$就会无限大,此时作如下修正(即不带入$\hat{t}$,而是代入$\tilde{t}$):
\begin{gather*}
	\tilde{t}=\frac{(X+1)(y+1)}{x+1}-1 \\
	\widehat{Var}(\tilde{t})=\frac{(X+1)(y+1)(y-x)(X-x)}{(x+1)^2(x+2)}
\end{gather*}

\subsection{区间估计}
\subsubsection{正态近似求置信区间}
由点估计方差公式,易得如下估计的总体总量地置信区间:
\begin{equation*}
	\hat{t}\pm u_{1-\frac{\alpha}{2}}\sqrt{\widehat{Var}(\hat{t})},\quad
	\tilde{t}\pm u_{1-\frac{\alpha}{2}}\sqrt{\widehat{Var}(\tilde{t})}
\end{equation*}
\subsubsection{正态近似置信区间可能存在的问题}
正态近似置信区间可能会存在:置信区间左端点小于两次捕捉到的总数的现象,这显然是不合理的。
\subsubsection{Pearson$\chi^2$检验求置信区间}
由标记重捕法使用条件,第一次被捕到和第二次被捕到这两件事情是独立的,由此可构建如下的列联表:
\begin{table}[h!]
	\centering
	\begin{tabular}{@{}lcc@{}}
		\toprule
		& 第二次捕获: 是 & 第二次捕获: 否 \\ 
		\midrule
		第一次捕获:是    & $a$           & $b$           \\
		第一次捕获:否    & $c$           & $d$           \\ 
		\bottomrule
	\end{tabular}
	\caption{标记重捕法的列联表示意图}
\end{table}
\par 在这个表里,$a,c,b$显然都是已知的,只有$d$是未知的。可以通过给$d$赋值的方式,去检验列联表行列变量之间是否独立(参考\cref{method:PearsonChisqTest}),选择合适的$d$值(即让独立性检验结果显著的$d$值)作为置信区间。
\subsubsection{似然比检验}
由样本可计算出$\hat{t}$,然后可以构建如下假设:
\begin{equation*}
	H_0:\theta=\hat{t}\quad H_1:\theta=\theta_A
\end{equation*}
进行似然比检验(参考\cref{method:LikelihoodTest})。置信区间为拒绝零假设的$\theta_A$构成的区间,即比$\hat{t}$更适合作为模型参数的$\theta_A$构成置信区间。
\subsubsection{bootstrap求置信区间}
在第二个样本中进行bootstrap,有放回的抽取$y$个样本,计算每个样本对总体总量的估计值$\hat{t}$,重复$N$次。将$N$个$\hat{t}$从小到大排序,在此基础上取分位点即产生置信区间,
\subsubsection{代码}
以上四种方法的代码如下:
\inputminted[bgcolor=white, linenos, frame=single, numbersep=5pt, breaklines]{r}{sampling-method/tag-recapture/tag-recapture.R}