\chapter{整群抽样}
\gls{ClusterSampling}将所有个体划分为$N$个\gls{Cluster},称群为\gls{psus}或抽样单位,然后通过SRS对群进行抽样,在抽出的每个群中使用独立的抽样方法得到最终的样本,称个体为\gls{ssus}或观察单位。在整群抽样中,ssu只有在它属于的psu被选中时才会有可能被包含在样本中。

\subsubsection{整群抽样与SRS、分层抽样的比较}
\begin{enumerate}
	\item 整群抽样中抽出的样本不如SRS得到的样本有代表性。因为群的划分往往是依据于地理信息等进行的,群内的差异往往较小,群间的差异可能较大。此时因为仅从抽中的群中抽样而未在别的群中抽样,样本就可能缺少代表性,即SRS样本单元比整群抽样样本单元提供的信息更多。
	\item 整群抽样不同于分层抽样的地方是它并不选取辅助变量,虽然层的概念类似于群的概念,但群不由辅助变量产生,也就导致了样本代表性较低的结果。分层随机抽样中,$Var(\hat{\mu}_Y)$在群内差异小群间差异大的情况下较小,整群抽样中,$Var(\hat{\mu}_Y)$在群内差异大群间差异小的情况下较小(因为此时样本单元提供的信息更多),在实践中只能通过扩大群的个数来提高精确度。
	\item 整群抽样会使抽样更加便捷,单位价格上信息更多。
\end{enumerate}
\subsubsection{整群抽样的分类}
\begin{enumerate}
	\item \gls{OnestageClusterSampling}:一旦某个psu被选中,该psu中的ssu全部被选中。
	\item \gls{TwostageClusterSampling}:某个psu被选中后,还需对其中的所有ssu进行一次抽样,若此次抽中则包含在样本中。
\end{enumerate}
\subsubsection{什么情况下使用整群抽样}
\begin{enumerate}
	\item 当构建包含所有个体的抽样框架十分困难或根本做不到(此时就无法做到直接对个体进行抽样),但若把所有个体分为若干个群,构建包含所有群的抽样框架并对群进行抽样并不困难时,适合使用整群抽样。
	\item 当目标群体在个体角度来讲分布广泛,调查成本太高,但可以将样本分群使一个群内的样本分布集中,从而可以降低抽样成本时,适合使用整群抽样。
\end{enumerate}

\section{一阶整群抽样}
由一阶整群抽样的定义,选中某个psu后该psu中的所有ssu进入样本。
\subsubsection{符号说明}
\begin{table}[H]
	\centering
	\setlength{\tabcolsep}{25pt} % 调整列之间的间距,默认值为6pt
	\renewcommand{\arraystretch}{1.5}
	\begin{tabular}{cc}
		\toprule
		符号    & 说明 \\
		\midrule
		$N$ & 总群数 \\
		$n$ & 抽样群数 \\
		$M_i$ & 第$i$个psu中的ssu个数 \\
		$M=\sum_{i=1}^{N}M_i$ & ssu的总数 \\
		$y_{ij}$ & 第$i$个psu中第$j$个样本单元的数值\\
		$\mu_i=\sum_{j=1}^{M_i}\frac{y_{ij}}{M_i}$ & 第$i$个psu中的均值 \\
		$\mu=\sum_{i=1}^{N}\sum_{j=1}^{M_i}\frac{y_{ij}}{M}$ & 总体均值 \\
		$\tau_i=\sum_{j=1}^{M_i}y_{ij}$ & 第$i$个psu中的总量 \\
		$t_i=\sum_{j=1}^{M_i}y_{ij}$ & 入样的第$i$个psu中的总量 \\
		$\tau=\sum_{i=1}^{N}\tau_i$ & 总体总量 \\
		$\sigma_i^2=\sum_{j=1}^{M_i}\frac{(y_{ij}-\mu_{i})^2}{M_i-1}$ & 第$i$个psu中的方差 \\
		$\sigma_{psu}^2=\frac{1}{N-1}\sum_{i=1}^N\left(\tau_i-\frac{\tau}{N}\right)^2$ & psu间的方差 \\
		$\sigma_M^2=\frac{1}{N-1}\sum_{i=1}^{N}\left(M_i-\frac{M}{N}\right)^2$ & psu间ssu个数的方差 \\
		$\sigma^2=\sum_{i=1}^N\sum_{j=1}^{M_i}\frac{(y_{ij}-\mu)^2}{M-1}$ & 总体方差 \\
		$R=\dfrac{\sum_{i=1}^{N}(M_i-\frac{M}{N})(\tau_i-\frac{\tau}{N})}{(N-1)\sigma_{M}\sigma_{psu}}$ & $M_i$与$\tau_i$的回归系数 \\
		\bottomrule
	\end{tabular}
	\caption{符号说明表}
\end{table}
一阶整群抽样的相关参数有两种估计方式,分别为无偏估计与比例估计。虽然无偏估计具有无偏性,但经过模拟研究,当群内总体总量与群内个体数量成正比时,比例估计的方差会比无偏估计小很多,此时应选择比例估计量对参数进行估计。

\subsection{参数的无偏估计}
\subsubsection{总体总量的估计}
可给出如下关于总体总量的估计:
\begin{gather*}
	\hat{\tau}=\frac{N}{n}\sum_{i=1}^nt_i=\sum_{i=1}^{n}\sum_{j=1}^{M_i}w_{ij}y_{ij} \\
	Var(\hat{\tau})=N^2\left(1-\frac{n}{N}\right)\frac{\sigma_{psu}^2}{n}=N(N-n)\frac{\sigma_{psu}^2}{n} \\
	\widehat{Var}(\hat{\tau})=N^2\left(1-\frac{n}{N}\right)\frac{s_{psu}^2}{n}=N(N-n)\frac{s_{psu}^2}{n} \\
	s_{psu}^2=\frac{1}{n-1}\sum_{i=1}^n\left(t_i-\frac{\hat{\tau}}{N}\right)^2
\end{gather*}
\subsubsection{总体均值的估计}
可给出如下关于总体均值的估计:
\begin{gather*}
	\hat{\mu}=\frac{\hat{\tau}}{M} \\
	Var(\hat{\mu})=Var\left(\frac{\hat{\tau}}{M^2}\right)=\frac{N^2}{M^2}\left(1-\frac{n}{N}\right)\frac{\sigma_{psu}^2}{n} \\
	\widehat{Var}(\hat{\mu})=\widehat{Var}\left(\frac{\hat{\tau}}{M^2}\right)=\frac{N^2}{M^2}\left(1-\frac{n}{N}\right)\frac{s_{psu}^2}{n} \\
\end{gather*}

\subsection{参数的比例估计}
由:
\begin{equation*}
	\mu=\frac{1}{M}\sum_{i=1}^{N}\sum_{j=1}^{M_i}y_{ij}=\frac{\sum\limits_{i=1}^{N}\tau_i}{\sum\limits_{i=1}^{N}M_i}=\frac{\tau}{M}
\end{equation*}
可设:
\begin{equation*}
	\mu=\frac{\tau}{M}=B
\end{equation*}
将$\tau_i$看作样本单元值,将$M_i$看作辅助变量,可得到如下关于参数的比例估计:
\begin{gather*}
	\hat{B}=\hat{\mu}_r=\frac{\hat{\tau}}{\hat{M}}=\frac{\frac{N}{n}\sum_{i=1}^nt_i}{\frac{N}{n}\sum_{i=1}^nM_i}=\frac{\sum_{i=1}^{n}t_i}{\sum_{i=1}^{n}M_i} \\
	Var(\hat{B})=\frac{1}{M^2}Var(\hat{\tau}_r)\approx\left(1-\frac{n}{N}\right)\frac{\sigma_{psu}^2-2BR\sigma_{M}\sigma_{psu}+B^2\sigma_{M}^2}{n(\dfrac{M}{N})^2} \\
	\widehat{Var}(\hat{B})\approx\left(1-\frac{n}{N}\right)\frac{s_{psu}^2-2\hat{B}\hat{R}s_{psu}s_M+\hat{B}^2s_M^2}{n(\dfrac{\sum_{i=1}^{n}M_i}{n})^2} \\
	\widehat{Var}_1(\hat{\mu}_r)=\frac{1}{M^2}N\left(N-n\right)\frac{s_e^2}{n} \\
	\widehat{Var}_1(\hat{\tau}_r)=N\left(N-n\right)\frac{s_e^2}{n} \\
	s_e^2=\frac{1}{n-1}\sum_{i=1}^{n}\left(t_i-\hat{B}M_i\right)^2
\end{gather*}
\section{二阶抽样}
整群抽样抽到群则群中所有单元进入样本,二阶抽样需要继续对每个群进行第二次抽样,抽中则进入样本,即我们只在选中的psu中选择一部分ssu。
\subsubsection{符号说明}
\begin{table}[H]
	\centering
	\setlength{\tabcolsep}{25pt} % 调整列之间的间距,默认值为6pt
	\begin{tabular}{ccc}
		\toprule
		符号    &  & 说明 \\
		\midrule
		$N$ & & 总群数 \\
		$n$ & & 抽样群数 \\
		$J$ & & 每个群中分层的层数 \\
		$M_{ij}$ & & 抽出的第$i$个群第$j$个层个体的总数 \\
		$m_{ij}$ & & 从抽出的第$i$个群第$j$个层抽出来的样本单元总数 \\
		$M_{i}$ & & 抽出的第$i$个群个体的总数 \\
		$M$ & & 个体总数 \\
		$Y_{ijk}$ & &第$i$个群第$j$个层的第$k$个个体的值 \\
		$y_{ijk}$ & &抽出的第$i$个群第$j$个层的第$k$个样本的值 \\
		$\bar{y}_{ij}$ & & 第$i$个群第$j$个层样本的均值 \\
		$\mu_{ij}$ & & 第$i$个群第$j$个层的均值 \\
		$\tau_i$ & & 第$i$个群的总体总量 \\
		$t_i$ & & 入样的第$i$个群的总体总量 \\
		$\tau$ & & 总体总量 \\
		$\sigma_{ij}^2$ & & 第$i$个群第$j$个层的方差 \\
		$p_{ij}$ & & 第$i$个群第$j$个层的流行率 \\
		$p_i$ & & 第$i$个群的流行率 \\
		$p$ & & 总流行率 \\
		$Z_i$ & & 表示第$i$个群是否被抽中的示性变量 \\
		\bottomrule
	\end{tabular}
	\caption{符号说明表}
\end{table}

\subsection{总量的估计}
可得到如下关于总量的估计:
\begin{gather*}
	\hat{t_i}=\sum_{j=1}^{J}\frac{M_{ij}}{m_{ij}}\sum_{k=1}^{m_{ij}}y_{ijk} \\
	\hat{\tau}=\frac{N}{n}\sum_{i=1}^{n}\hat{t_i}=\frac{N}{n}\sum_{i=1}^{n}\sum_{j=1}^{J}\frac{M_{ij}}{m_{ij}}\sum_{k=1}^{m_{ij}}y_{ijk} \\
	E(\hat{t_i})=t_i,\;E(\hat{\tau})=\tau \\
	Var(\hat{\tau})=\frac{N^2}{n}\left(1-\frac{n}{N}\right)\frac{1}{N-1}\sum_{i=1}^{N}\left(\tau_i-\frac{\tau}{N}\right)^2+\frac{N}{n}\sum_{i=1}^{N}\sum_{j=1}^{J}\frac{M_{ij}^2}{m_{ij}}\left(1-\frac{m_{ij}}{M_{ij}}\right)\left(\frac{1}{M_{ij}-1}\right)\sum_{k=1}^{M_{ij}}\left(Y_{ijk}-\mu_{ij}\right)^2 \\
	\widehat{Var}(\hat{\tau})=\frac{N^2}{n}\left(1-\frac{n}{N}\right)\frac{1}{n-1}\sum_{i=1}^{n}\left(\hat{t_i}-\frac{\hat{\tau}}{N}\right)^2+\frac{N}{n}\sum_{i=1}^{n}\sum_{j=1}^{J}\frac{M_{ij}^2}{m_{ij}}\left(1-\frac{m_{ij}}{M_{ij}}\right)\left(\frac{1}{m_{ij}-1}\right)\sum_{k=1}^{m_{ij}}\left(y_{ijk}-\bar{y}_{ij}\right)^2 
\end{gather*}
\subsubsection{无偏性的证明}
由分层随机抽样总体总量估计量的无偏性,可以得到此时$\hat{\tau_i}$的无偏性,进而可以证明$\hat{\tau}$是无偏的:
\begin{align*}
	E(\hat{\tau})
	&=E[E(\hat{\tau}|\overrightarrow{Z})]
	=E\left[E\left(\frac{N}{n}\sum_{i=1}^{n}\hat{t_i}|\overrightarrow{Z}\right)\right]
	=E\left[E\left(\frac{N}{n}\sum_{i=1}^{N}Z_i\hat{\tau_i}\right)\right] \\ 
	&=E\left[\sum_{i=1}^{N}\frac{N}{n}Z_iE(\hat{\tau_i})\right]
	=E\left[\sum_{i=1}^{N}\frac{N}{n}Z_i\tau_i\right]
	=\sum_{i=1}^{N}\frac{N}{n}\tau_iE(Z_i)
	=\sum_{i=1}^{N}\frac{N}{n}\tau_i\frac{n}{N}
	=\sum_{i=1}^{N}\tau_i=\tau
\end{align*}
\subsubsection{方差公式的证明}
由方差的分解,可以得到:
\begin{equation*}
	Var(\hat{\tau})=Var[E(\hat{\tau}|\overrightarrow{Z})]+E[Var(\hat{\tau}|\overrightarrow{Z})] 
\end{equation*}
由SRS的结论,可以得到:
\begin{gather*}
	Var[E(\hat{\tau}|\overrightarrow{Z})]=Var\left[E\left(\sum_{i=1}^{N}\frac{N}{n}Z_i\hat{\tau_i}|\overrightarrow{Z}\right)\right]=Var\left(\sum_{i=1}^{N}\frac{N}{n}Z_i\tau_i\right)=N^2\left(1-\frac{n}{N}\right)\frac{\sigma_\tau^2}{n}\\
	\sigma_\tau^2=\frac{1}{N-1}\sum_{i=1}^{N}\left(\tau_i-\frac{\tau}{N}\right)^2
\end{gather*}
由方差的分解,可以得到:
\begin{equation*}
	E[Var(\hat{\tau}|\overrightarrow{Z})]
	=E\left\{E(\hat{\tau}^2|\overrightarrow{Z})-
	 E^2[\hat{\tau}|\overrightarrow{Z}]\right\}
\end{equation*}
而:
\begin{align*}
	E[\hat{\tau}^2|\overrightarrow{Z}]-E[\hat{\tau}|\overrightarrow{Z}]^2
	&=E\left[\left(\sum_{i=1}^{N}\frac{N}{n}Z_i\hat{\tau_i}\right)^2\right]-\left(\sum_{i=1}^{N}\frac{N}{n}Z_i\tau_i\right)^2 \\
	&=E\left(\sum_{i=1}^{N}\frac{N^2}{n^2}Z_i^2\hat{\tau_i}^2+\sum_{i=1}^{N}\sum_{j\ne i}\frac{N^2}{n^2}Z_iZ_j\hat{\tau_i}\hat{\tau_j}\right) \\
	&\quad-\left(\sum_{i=1}^{N}\frac{N^2}{n^2}Z_i^2\tau_i^2+\sum_{i=1}^{N}\sum_{j\ne i}\frac{N^2}{n^2}Z_iZ_j\tau_i\tau_j\right) \\
	&=\frac{N^2}{n^2}\left(\sum_{i=1}^{N}Z_i^2E(\hat{\tau_i}^2)+\sum_{i=1}^{N}\sum_{j\ne i}Z_iZ_jE(\hat{\tau_i}\hat{\tau_j})-\sum_{i=1}^{N}Z_i^2\tau_i^2-\sum_{i=1}^{N}\sum_{j\ne i}Z_iZ_j\tau_i\tau_j\right) \\
	&=\frac{N^2}{n^2}\left[\sum_{i=1}^{N}Z_i^2E^2(\hat{\tau_i})+\sum_{i=1}^{N}Z_i^2Var(\hat{\tau_i})+\sum_{i=1}^{N}\sum_{j\ne i}Z_iZ_jE(\hat{\tau_i})E(\hat{\tau_j})\right.\\
	&\quad\left.-\sum_{i=1}^{N}Z_i^2\tau_i^2-\sum_{i=1}^{N}\sum_{j\ne i}Z_iZ_j\tau_i\tau_j\right] \\
	&=\frac{N^2}{n^2}\sum_{i=1}^{N}Z_i^2Var(\hat{\tau_i})
\end{align*}
由$Z_i$的性质可得:
\begin{equation*}
	E[Var(\hat{\tau}|\overrightarrow{Z})]=E\left(\frac{N^2}{n^2}\sum_{i=1}^{N}Z_i^2Var(\hat{\tau_i})\right)=\frac{N^2}{n^2}\sum_{i=1}^{N}E(Z_i^2)Var(\hat{\tau_i})=\frac{N}{n}\sum_{i=1}^{N}Var(\hat{\tau_i})
\end{equation*}
由分层随机抽样层内方差的公式:
\begin{gather*}
	E[Var(\hat{\tau}|\overrightarrow{Z})]=\frac{N}{n}\sum_{i=1}^{N}\sum_{j=1}^{J}\frac{M_{ij}^2}{m_{ij}}\left(1-\frac{m_{ij}}{M_{ij}}\right)\sigma_{ij}^2 \\
	\sigma_{ij}^2=\left(\frac{1}{M_{ij}-1}\right)\sum_{k=1}^{M_{ij}}\left(y_{ijk}-\mu_{ij}\right)^2
\end{gather*}
所以:
\begin{equation*}
	Var(\hat{\tau})=\frac{N^2}{n}\left(1-\frac{n}{N}\right)\frac{1}{N-1}\sum_{i=1}^{N}\left(\tau_i-\frac{\tau}{N}\right)^2+\frac{N}{n}\sum_{i=1}^{N}\sum_{j=1}^{J}\frac{M_{ij}^2}{m_{ij}}\left(1-\frac{m_{ij}}{M_{ij}}\right)\left(\frac{1}{M_{ij}-1}\right)\sum_{k=1}^{M_{ij}}\left(Y_{ijk}-\mu_{ij}\right)^2
\end{equation*}

\subsection{流行率问题}
关于流行率问题,有如下结论:
\begin{gather*}
	\hat{p}_i=\frac{\hat{t_i}}{M_i} \\
	\hat{p}=\frac{\hat{\tau}}{M}=\frac{N}{nM}\sum_{i=1}^{n}\sum_{j=1}^{J}\frac{M_{ij}}{m_{ij}}\sum_{k=1}^{m_{ij}}y_{ijk} \\
	Var(\hat{\tau})=\frac{N^2}{n}\left(1-\frac{n}{N}\right)\frac{1}{N-1}\sum_{i=1}^{N}\left(\tau_i-\frac{\tau}{N}\right)^2+\frac{N}{n}\sum_{i=1}^{N}\sum_{j=1}^{J}\frac{M_{ij}^2}{m_{ij}}\left(1-\frac{m_{ij}}{M_{ij}}\right)\frac{M_{ij}p_{ij}(1-p_{ij})}{M_{ij}-1} \\
	\widehat{Var}(\hat{\tau})=\frac{N^2}{n}\left(1-\frac{n}{N}\right)\frac{1}{n-1}\sum_{i=1}^{n}\left(\hat{t_i}-\frac{\hat{\tau}}{N}\right)^2+\frac{N}{n}\sum_{i=1}^{n}\sum_{j=1}^{J}\frac{M_{ij}^2}{m_{ij}-1}\left(1-\frac{m_{ij}}{M_{ij}}\right)\hat{p}_{ij}(1-\hat{p}_{ij}) 
\end{gather*}
由$\hat{\tau_i}$和$\hat{\tau}$的无偏性,显然$\hat{p}_i$和$\hat{p}$也是无偏估计。由一般情况下$Var(\hat{\tau})$的计算公式,也容易得到流行率问题下$Var(\hat{\tau})$的计算公式。


\section{系统抽样}
抽样框绘制很困难的时候
系统抽样是一种特殊的整群抽样


