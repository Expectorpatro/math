\section{SRSWR的参数估计}
\subsubsection{SRSWR中$Q$的相关性质}
\begin{theorem}
	SRSWR中表示个体在样本中出现次数的变量$Q_i$具有如下性质:
	\begin{gather*}
		\overrightarrow{Q}=(Q_1,Q_2,\dots,Q_N)\sim \text{Multi}\left(n,\; \overbrace{\left(\frac{1}{N},\;\frac{1}{N},\dots,\frac{1}{N}\right)}^{\text{N个}\;\frac{1}{N}}\right)\\
		Q_i\sim\text{Binom}\left(n,\;\frac{1}{N}\right) \\
		E(Q_i)=\frac{n}{N},\;Var(Q_i)=\frac{n}{N}\left(1-\frac{1}{N}\right) ,\;Cov(Q_i,\;Q_j)=-\frac{n}{N^2}
	\end{gather*}
\end{theorem}
\begin{proof} 
	由多项分布性质:多项分布随机变量的一维边际分布是二项分布,因此上第二、三式成立。下证第四式,将$Q_i$分解为独立二项分布随机变量的和,有:
	\begin{align*}
		Cov(Q_i,\;Q_j)&=Cov\left[\sum_{k=1}^nI_i(k),\;\sum_{l=1}^nI_j(l)\right] \\
		&=\sum_{k=1}^n\sum_{l=1}^nCov[I_i(k),\;I_j(l)] \\
		&=\sum_{k=l}Cov[I_i(k),\;I_j(l)]+\sum_{k\ne l}Cov[I_i(k),\;I_j(l)]
	\end{align*}
	由二项分布随机变量的独立性,后一项为$0$:
	\begin{equation*}
		Cov(Q_i,\;Q_j)
		=\sum_{k=1}^nCov[I_i(k),\;I_j(k)] =\sum_{k=1}^n\left\{E[I_i(k)I_j(k)]-E[I_i(k)]E[I_j(k)]\right\}
	\end{equation*}
	由于同一次伯努利实验中不可能出现两个结果,所以前一项为$0$:
	\begin{equation*}
		Cov(Q_i,\;Q_j)
		=-\sum_{k=1}^nE[I_i(k)]E[I_j(k)]
		=-np_ip_j
		=-\frac{n}{N^2} \qedhere
	\end{equation*}
\end{proof}

\subsection{SRSWR总体总量$\tau$的估计}
\begin{theorem}
	在SRS中中利用样本均值可对总体总量$\tau$给出如下点估计:
	\begin{equation*}
		\hat{\tau}=\frac{N}{n}\sum_{i=1}^NY_iQ_i
	\end{equation*}
	该点估计具有如下性质:
	\begin{equation*}
		E(\hat{\tau})=\tau,\;Var(\hat{\tau})=\frac{N(N-1)}{n}\sigma^2
	\end{equation*}
\end{theorem}
\begin{proof}
	将方差展开可得到:
	\begin{align*}
		Var(\hat{\tau})
		&=Var\left(\frac{N}{n}\sum_{i=1}^N Q_iY_i\right) \\
		&=\frac{N^2}{n^2}\left[\sum_{i=1}^{N}Y_i^2 Var(Q_i)+2\sum_{i=1}^N\sum_{j=i+1}^NY_iY_j Cov(Q_i, Q_j)\right] \\
		&=\frac{N^2}{n^2}\left[\sum_{i=1}^{N}Y_i^2\frac{n}{N}\left(1-\frac{1}{N}\right)+2\sum_{i=1}^N\sum_{j=i+1}^NY_iY_j \frac{-n}{N^2}\right] \\
		&=\frac{N}{n}\left[\sum_{i=1}^{N}Y_i^2\left(1-\frac{1}{N}\right)-\frac{1}{N}\sum_{i=1}^N\sum_{j=i+1}^N2Y_iY_j\right]
	\end{align*}
	此处使用平方和公式(该技巧常用):
	\begin{align*}
		Var(\hat{\tau})
		&=\frac{N}{n}\left\{\sum_{i=1}^{N}Y_i^2\left(1-\frac{1}{N}\right)-\frac{1}{N}\left[\left(\sum_{i=1}^NY_i\right)^2-\sum_{i=1}^NY_i^2\right]\right\} \\
		&=\frac{N}{n}\left[\sum_{i=1}^{N}Y_i^2-\frac{1}{N}\left(\sum_{i=1}^NY_i\right)^2\right]
	\end{align*}
	此处使用$N\mu^2$并进行加减凑项(该技巧常用):
	\begin{equation*}
		Var(\hat{\tau})          =\frac{N}{n}\left(\sum_{i=1}^{N}Y_i^2-N\mu^2\right)
		=\frac{N}{n}\left(\sum_{i=1}^{N}Y_i^2-2N\mu^2+N\mu^2\right)
	\end{equation*}
	此处使用$N\mu=\sum\limits_{i=1}^NY_i$(该技巧常用):
	\begin{equation*}
		Var(\hat{\tau}) =\frac{N}{n}\left(\sum_{i=1}^{N}Y_i^2-2\mu\sum_{i=1}^NY_i+N\mu^2\right)
		=\frac{N}{n}\sum_{i=1}^N(Y_i-\mu)^2
		=\frac{N(N-1)}{n}\sigma^2 \qedhere
	\end{equation*}
\end{proof}
