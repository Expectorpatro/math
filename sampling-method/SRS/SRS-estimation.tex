\section{SRS的参数估计}

\subsubsection{SRS中$Z$的相关性质} 
\begin{theorem}
	SRS中表示个体是否入样的示性变量具有如下性质:
	\begin{gather*}
		E(Z_i)=\frac{n}{N} \\
		Var(Z_i)=\frac{n}{N}\left(1-\frac{n}{N}\right) \\
		Cov(Z_i,Z_j)=\frac{-n}{N(N-1)}\left(1-\frac{n}{N}\right) 
	\end{gather*}
\end{theorem}
\begin{proof}
	每个个体被抽到的概率为:
	\begin{equation}
		\pi_k=\frac{\binom{N-1}{n-1}}{\binom{N}{n}}=\frac{n}{N}\notag
	\end{equation}      
	由此可知个体示性变量的期望:
	\begin{equation}
		E(Z_i)=1\times P(Z_i=1)=1\times \pi_i=\frac{n}{N}\notag
	\end{equation}
	注意到$E(Z_i^2)=0\times P(Z_i^2=0)+1\times P(Z_i^2=1)=P(Z_i=1)=E(Z_i)$,即有:
	\begin{equation*}
		Var(Z_i)=E(Z_i^2)-E^2(Z_i)
		=E(Z_i)\left(1-E(Z_i)\right)
		=\frac{n}{N}\left(1-\frac{n}{N}\right)
	\end{equation*}
	注意到$E(Z_iZ_j)=P(Z_i=1,\;Z_j=1)$,所以:
	\begin{equation*}
		Cov(Z_i,Z_j)=E(Z_iZ_j)-E(Z_i)E(Z_j)
		=\frac{\binom{N-2}{n-2}}{\binom{N}{n}}-E^2(Z_i)
		=\frac{-n}{N(N-1)}\left(1-\frac{n}{N}\right) \qedhere
	\end{equation*}
\end{proof}

\subsection{SRS总体均值$\mu$的估计} 
\subsubsection{点估计及点估计的性质}
\begin{theorem}
	在SRS中中利用样本均值可对总体均值$\mu$给出如下点估计:
	\begin{equation*}
		\hat{\mu}=\frac{1}{n}\sum\limits_{i=1}^{n}y_i=\frac{1}{n}\sum\limits_{i=1}^{N}Y_iZ_i
	\end{equation*}
	该点估计具有如下性质:
	\begin{gather*}
		E(\hat{\mu})=\mu,\;Var(\hat{\mu})=\left(1-\frac{n}{N}\right)\frac{\sigma^2}{n} \\
		\widehat{Var}(\hat{\mu})=\left(1-\frac{n}{N}\right)\frac{s^2}{n}\text{是关于}Var(\hat{\mu})\text{的无偏估计。}
	\end{gather*}
\end{theorem}
其中方差的计算见下文总体总量方差的推导,由方差的性质即可推导出总体均值的方差。\footnote{$\left(1-\frac{n}{N}\right)$被称之为\gls{FPC},有放回抽样或$N$远大于$n$时不需要FPC。}
\subsubsection{区间估计}
\begin{theorem}
	由大数定律,大样本下有:
	\begin{equation*}
		\frac{\hat{\mu}-\mu}{\sqrt{Var(\hat{\mu})}}\sim N(0,\;1)
	\end{equation*}
	所以SRS中总体均值$\mu$的区间估计如下:
	\begin{equation}
		\hat{\mu}\pm u_{1-\frac{\alpha}{2}}\times\sqrt{Var(\hat{\mu})}\notag
	\end{equation}
	由于$Var(\hat{\mu})$的计算中涉及未知参数$\sigma^2$,以$\widehat{Var}(\hat{\mu})$代替,因此置信度为$(1-\alpha)$的估计的双侧置信区间为\label{sec:SRSmuci}:
	\begin{equation}
		\hat{\mu}\pm u_{1-\frac{\alpha}{2}}\times\sqrt{\widehat{Var}(\hat{\mu})}\notag
	\end{equation}
\end{theorem}

\subsection{SRS总体总量$\tau$的估计}
\subsection*{点估计}
\subsubsection{利用样本均值进行估计}
\begin{definition}
	在SRS中利用样本均值可对总体总量$\tau$给出如下点估计:
	\begin{gather*}
		\hat{\tau}=N\hat{\mu}=\frac{N}{n}\sum\limits_{i=1}^{n}y_i=\frac{N}{n}\sum\limits_{i=1}^{N}Y_iZ_i
	\end{gather*}
\end{definition}
\subsubsection{Horvitz-Thompson估计量(HT估计量)}
\begin{definition}
	在SRS中引入抽样权重可给出关于$\tau$的HT估计:
	\begin{equation*}
		\hat{\tau}=\sum\limits_{i=1}^{n}w_iy_i=\sum\limits_{i=1}^Nw_iY_iZ_i
	\end{equation*}
\end{definition}
\subsubsection{点估计的性质}
\begin{theorem}
	关于SRS总体总量$\tau$的点估计有如下性质:
	\begin{gather*}
		E(\hat{\tau})=\tau \\ Var(\hat{\tau})=N^2\left(1-\frac{n}{N}\right)\frac{\sigma^2}{n},\quad
		\widehat{Var}(\hat{\tau})=N^2\left(1-\frac{n}{N}\right)\frac{s^2}{n}
	\end{gather*}
\end{theorem}
下给出点估计估计量方差公式的证明。
\begin{proof}
	将方差展开可得到:
	\begin{align*}
		Var(\hat{\tau})
		&=Var\left(\sum\limits_{i=1}^Nw_iY_iZ_i\right) \\
		&=\sum\limits_{i=1}^NVar(w_iY_iZ_i)+2\sum\limits_{i=1}^N\sum\limits_{j=i+1}^NCov(w_iY_iZ_i,w_jY_jZ_j) \\
		&=\sum\limits_{i=1}^Nw_i^2Y_i^2Var(Z_i)+2\sum\limits_{i=1}^N\sum\limits_{j=i+1}^Nw_iY_iw_jY_jCov(Z_i,Z_j) \\	
	\end{align*}	
	注意到$w_i=\frac{N}{n}$并代入$Z_i$相关性质的公式可以得到:
	\begin{align*}
		Var(\hat{\tau})
		&=\sum\limits_{i=1}^Nw_i^2Y_i^2Var(Z_i)+2\sum\limits_{i=1}^N\sum\limits_{j=i+1}^Nw_iY_iw_jY_jCov(Z_i,Z_j) \\
		&=\frac{N}{n}\left(1-\frac{n}{N}\right)\frac{1}{N-1}\sum\limits_{i=1}^{N}\left[(N-1)Y_i^2-\sum\limits_{j=i+1}^N2Y_iY_j\right] \\
		&=\frac{N}{n}\left(1-\frac{n}{N}\right)\frac{1}{N-1}\left[\sum\limits_{i=1}^{N}(N-1)Y_i^2-\sum\limits_{i=1}^{N}\sum\limits_{j=i+1}^N2Y_iY_j\right] \\
		&=\frac{N}{n}\left(1-\frac{n}{N}\right)\frac{1}{N-1}\left[\sum\limits_{i=1}^{N}(N-1)Y_i^2-\left(\sum\limits_{i=1}^NY_i\right)^2+\sum\limits_{i=1}^NY_i^2\right] \\
		&=\frac{N}{n}\left(1-\frac{n}{N}\right)\frac{1}{N-1}\left[N\sum\limits_{i=1}^{N}Y_i^2-\left(\sum\limits_{i=1}^NY_i\right)^2\right] \\
		&=\frac{N^2}{n}\left(1-\frac{n}{N}\right)\frac{1}{N-1}\left(\sum\limits_{i=1}^{N}Y_i^2-N\mu^2\right)
	\end{align*}
	\begin{align*}
		Var(\hat{\tau})
		&=\frac{N^2}{n}\left(1-\frac{n}{N}\right)\frac{1}{N-1}\left(\sum\limits_{i=1}^{N}Y_i^2-2N\mu^2+N\mu^2\right) \\
		&=\frac{N^2}{n}\left(1-\frac{n}{N}\right)\frac{1}{N-1}\left(\sum\limits_{i=1}^{N}Y_i^2-2\mu\sum\limits_{i=1}^NY_i+N\mu^2\right) \\     
		&=\frac{N^2}{n}\left(1-\frac{n}{N}\right)\frac{1}{N-1}\sum\limits_{i=1}^N(Y_i-\mu)^2 \\ 
		&=\frac{N^2}{n}\left(1-\frac{n}{N}\right)\sigma^2 \qedhere
	\end{align*}
\end{proof}
\subsection*{区间估计}
\begin{theorem}
	由大数定律,大样本下有:
	\begin{equation}
		\frac{\hat{\tau}-\tau}{\sqrt{Var(\hat{\tau})}}\sim N(0,\;1)\notag
	\end{equation}
	所以SRS中总体总量$\tau$的区间估计如下:
	\begin{equation}
		\hat{\tau}\pm u_{1-\frac{\alpha}{2}}\times\sqrt{Var(\hat{\tau})}\notag
	\end{equation}
	由于$Var(\hat{\tau})$的计算中涉及未知参数$\sigma^2$,以$\widehat{Var}(\hat{\tau})$代替,因此置信度为$(1-\alpha)$的估计的双侧置信区间为:
	\begin{equation}
		\hat{\tau}\pm u_{1-\frac{\alpha}{2}}\times\sqrt{\widehat{Var}(\hat{\tau})}\notag
	\end{equation}
\end{theorem}

\subsection{SRS总体方差$\sigma^2$的估计}
\begin{definition}
	在SRS中利用样本方差可对总体方差$\sigma^2$给出如下点估计:
	\begin{equation}
		\hat{\sigma^2}=s^2=\frac{1}{n-1}\sum\limits_{i=1}^n(y_i-\overline{y})^2=\frac{1}{n-1}\sum\limits_{i=1}^N(Y_i-\hat{\mu})^2Z_i\notag
	\end{equation}
\end{definition}