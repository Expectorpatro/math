\section{SRS阳性率问题}
阳性率问题是前述问题的一种特殊形式,$Y_i$只能在$0$和$1$中取值。因此阳性率$p$即为总体均值$\mu$。
\subsection{阳性率$p$的估计}
\subsubsection{点估计}
\begin{definition}
	在SRS中利用样本阳性率可给出阳性率$p$的点估计如下:
	\begin{equation*}
		\hat{p}=\frac{1}{n}\sum\limits_{i=1}^{n}y_i=\frac{1}{n}\sum\limits_{i=1}^{N}Y_iZ_i
	\end{equation*}
	该点估计具有如下性质:
	\begin{gather*}
		E(\hat{p})=p,\quad
		Var(\hat{p})=\left(1-\frac{n}{N}\right)\frac{\sigma^2}{n},\quad
		\widehat{Var}(\hat{p})=\left(1-\frac{n}{N}\right)\frac{s^2}{n}
	\end{gather*}
\end{definition}

\subsubsection{区间估计}
\begin{theorem}
	由大数定律,大样本下有:
	\begin{equation}
		\frac{\hat{p}-p}{\sqrt{Var(\hat{p})}}\sim N(0,\;1)\notag
	\end{equation}
	所以SRS中阳性率$p$的区间估计如下:
	\begin{equation}
		\hat{p}\pm u_{1-\frac{\alpha}{2}}\sqrt{Var(\hat{p})}\notag
	\end{equation}
	由于$Var(\hat{p})$的计算中涉及未知参数$\sigma^2$,以$\widehat{Var}(\hat{p})$代替,因此置信度为$(1-\alpha)$的估计的双侧置信区间为:
	\begin{equation}
		\hat{p}\pm u_{1-\frac{\alpha}{2}}\sqrt{\widehat{Var}(\hat{p})}\notag
	\end{equation}
	上述计算公式需要满足$n\hat{p}\geqslant5$和$n(1-\hat{p})\geqslant 5$,即大样本条件。
\end{theorem}

\subsection{总体方差$\sigma^2$的估计}
\subsubsection{总体方差$\sigma^2$与总体均值$p$的关系}
由于$Y_i^2=Y_i$,可得:
\begin{equation}
	\sigma^2=\frac{1}{N-1}\sum_{i=1}^N(Y_i-p)^2=\frac{N}{N-1}p(1-p)\notag
\end{equation}
\subsubsection{点估计}
\begin{definition}
	在SRS中利用样本方差可对总体方差$\sigma^2$给出如下点估计:
	\begin{equation}
		\hat{\sigma^2}=s^2=\frac{1}{n-1}\sum_{i=1}^n(y_i-\overline{y})^2=\frac{1}{n-1}\sum_{i=1}^n(y_i-\hat{p})^2=\frac{n}{n-1}\hat{p}(1-\hat{p})\notag
	\end{equation}
\end{definition}

