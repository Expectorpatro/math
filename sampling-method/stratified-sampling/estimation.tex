\section{参数估计}

\subsection{亚群体特征的估计}
因为分层随机抽样在亚群体中为简单随机抽样,由简单随机抽样的估计公式即有如下公式:
\begin{gather*}
	\hat{\mu}_h=\frac{1}{n_h}\sum_{j=1}^{n_h}y_{hj} \\
	\hat{\tau}_h=\frac{N_h}{n_h}\sum_{j=1}^{n_h}y_{hj}=N_h\hat{\mu}_h \\
	\hat{\sigma^2_h}=s^2_h=\frac{1}{n_h-1}\sum_{j=1}^{n_h}\left(y_{hj}-\hat{\mu}_h\right)^2
\end{gather*}

\subsection{总体总量$\hat{\tau}$的估计}
\subsubsection{计算公式}
\begin{equation*}
	\hat{\tau}_{str}=\sum_{h=1}^H\hat{\tau}_h=\sum_{h=1}^HN_h\bar{y}_h
\end{equation*}
\subsubsection{抽样权重形式}
\begin{equation*}
	\hat{\tau}_{str}=\sum_{h=1}^HN_h\bar{y}_h=\sum_{h=1}^H\frac{N_h}{n_h}\sum_{j=1}^{n_h}y_{hj}=\sum_{h=1}^H\sum_{j=1}^{n_h}\frac{N_h}{n_h}y_{hj}=\sum_{h=1}^H\sum_{j=1}^{n_h}w_{hj}y_{hj}
\end{equation*}

\subsection{总体均值$\hat{\mu}$的估计}
\subsubsection{计算公式}
\begin{equation*}
	\hat{\mu}_{str}=\frac{\hat{\tau}_{str}}{N}=\frac{1}{N}\sum_{h=1}^HN_h\bar{y}_h
\end{equation*}
\subsubsection{抽样权重形式}
只需注意到$N=\sum\limits_{h=1}^H\sum\limits_{j=1}^{n_h}w_{hj}$:
\begin{equation*}
	\hat{\mu}_{str}=\frac{\sum\limits_{h=1}^H\sum\limits_{j=1}^{n_h}w_{hj}y_{hj}}{\sum\limits_{h=1}^H\sum\limits_{j=1}^{n_h}w_{hj}}
\end{equation*}

\subsection{估计的性质}
\subsubsection{无偏性}
无偏性是显然的:在每一层里面使用的是简单随机抽样,而简单随机抽样的估计量是无偏的,总和也自然是无偏的。
\subsubsection{方差}
由各层样本之间的独立性以及每层中的抽样实际是SRS,立即可得如下分层随机抽样估计量的方差公式:
\begin{gather*}
	Var(\hat{\tau}_{str})=\sum_{h=1}^HN_h^2\left(1-\frac{n_h}{N_h}\right)\frac{\sigma_h^2}{n_h} \\
	\widehat{Var}(\hat{\tau}_{str})=\sum_{h=1}^HN_h^2\left(1-\frac{n_h}{N_h}\right)\frac{s_h^2}{n_h} \\
	Var(\hat{\mu}_{str})=\sum_{h=1}^H\left(\frac{N_h}{N}\right)^2\left(1-\frac{n_h}{N_h}\right)\frac{\sigma_h^2}{n_h} \\
	\widehat{Var}(\hat{\mu}_{str})=\sum_{h=1}^H\left(\frac{N_h}{N}\right)^2\left(1-\frac{n_h}{N_h}\right)\frac{s_h^2}{n_h}
\end{gather*}
其中:
\begin{gather*}
	\sigma_h^2=\frac{1}{N_h-1}\sum_{j=1}^{N_h}\left(Y_{hj}-\mu_h\right)^2 \\
	s_h^2=\frac{1}{N_h-1}\sum_{j=1}^{n_h}\left(y_{hj}-\bar{y}_h\right)^2 \\
\end{gather*}

\subsubsection{置信区间}
\begin{equation*}
	\hat{\mu}_{str}\pm u_{1-\frac{\alpha}{2}}\sqrt{\widehat{Var}(\hat{\mu}_{str})},\;\hat{\tau}_{str}\pm u_{1-\frac{\alpha}{2}}\sqrt{\widehat{Var})(\hat{\tau}_{str})}
\end{equation*}
\subsubsection{自由度问题}
当使用$t$置信区间的时候,如果各层之间方差是齐的,那么自由度即为$n-H$。如果方差不齐,则使用Satterwaithe approximation来估计自由度:
\begin{equation*}
	Dof=\left(\sum_{h=1}^Ha_hs_h^2\right)^2\div\sum_{h=1}^H\frac{(a_hs_h^2)^2}{(n_h-1)}
\end{equation*}
其中:
\begin{equation*}
	a_h=\frac{N_h(N_h-n_h)}{n_h}
\end{equation*}

\subsection{群体比例问题}
\begin{gather*}
	\hat{p}_{str}=\sum_{h=1}^H\frac{N_h}{N}\hat{p}_h \\
	\widehat{Var}(\hat{p}_{str})=\sum_{h=1}^H\left(\frac{N_h}{N}\right)^2\left(1-\frac{n_h}{N_h}\right)\frac{\hat{p}_h(1-\hat{p}_h)}{n-1}
\end{gather*}