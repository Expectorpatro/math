\section{矩阵}

\subsection{Kronecker乘积}
\begin{definition}
	给定两个矩阵 \( A=(a_{ij}) \in M_{m\times n}(K) \) 和 \( B \in M_{p\times q}(K) \),它们的 Kronecker 乘积 \( A \otimes B \) 是一个大小为 \( mp \times nq \) 的矩阵,定义为:
	\[
	A \otimes B = \begin{pmatrix}
		a_{11} B & a_{12} B & \cdots & a_{1n} B \\
		a_{21} B & a_{22} B & \cdots & a_{2n} B \\
		\vdots & \vdots & \ddots & \vdots \\
		a_{m1} B & a_{m2} B & \cdots & a_{mn} B
	\end{pmatrix}
	\]
	\( a_{ij} B \) 表示矩阵 \( B \) 乘以标量 \( a_{ij},\;i=1,2,\dots,m,\;j=1,2,\dots,n \)。
\end{definition}
\begin{property}\label{prop:Kronecker}
	Kronecker乘积具有如下性质:
	\begin{enumerate}
		\item $I_m\otimes I_n=I_{mn}$;
		\item 设$A\in M_{m\times n}(K),\;B\in M_{p\times q}(K),\;C\in M_{n\times k}(K),\;D\in M_{q\times r}(K)$,则$(A\otimes B)(C\otimes D)=(AC)\otimes (BD)$;
		\item 设$A\in M_{m}(K),\;B\in M_{n}(K)$,则$A\otimes B$可逆的充分必要条件为$A,B$都可逆,此时有:
		\begin{equation*}
			(A\otimes B)^{-1}=A^{-1}\otimes B^{-1}
		\end{equation*}
		\item 设$A=(a_{ij})\in M_{m\times n}(K),\;B=(b_{kl})\in M_{p\times q}(K)$,则$(A\otimes B)^T=A^T\otimes B^T$;
	\end{enumerate}
\end{property}
\begin{proof}
	(1)由Kronecker乘积的定义:
	\begin{equation*}
		I_m\otimes I_n
		=
		\begin{pmatrix}
			I_n & 0 & \cdots & 0 \\
			0 & I_n & \cdots & 0 \\
			\vdots & \vdots & \ddots & \vdots \\
			0 & 0 & \cdots & I_n
		\end{pmatrix}=I_{mn}
	\end{equation*}\par
	(2)由Kronecker乘积的定义:
	\begin{align*}
		(A\otimes B)(C\otimes D)
		&=
		\begin{pmatrix}
			a_{11} B & a_{12} B & \cdots & a_{1n} B \\
			a_{21} B & a_{22} B & \cdots & a_{2n} B \\
			\vdots & \vdots & \ddots & \vdots \\
			a_{m1} B & a_{m2} B & \cdots & a_{mn} B
		\end{pmatrix}
		\begin{pmatrix}
			c_{11} D & c_{12} D & \cdots & c_{1k} D \\
			c_{21} D & c_{22} D & \cdots & c_{2k} D \\
			\vdots & \vdots & \ddots & \vdots \\
			c_{n1} D & c_{n2} D & \cdots & c_{nk} D
		\end{pmatrix} \\
		&=
		\begin{pmatrix}
			\sum\limits_{i=1}^{n}a_{1i}c_{i1}BD & \sum\limits_{i=1}^{n}a_{1i}c_{i2}BD & \cdots &\sum\limits_{i=1}^{n}a_{1i}c_{ik}BD \\
			\sum\limits_{i=1}^{n}a_{2i}c_{i1}BD & \sum\limits_{i=1}^{n}a_{2i}c_{i2}BD & \cdots &\sum\limits_{i=1}^{n}a_{2i}c_{ik}BD \\
			\vdots & \vdots & \ddots & \vdots \\
			\sum\limits_{i=1}^{n}a_{mi}c_{i1}BD & \sum\limits_{i=1}^{n}a_{mi}c_{i2}BD & \cdots &\sum\limits_{i=1}^{n}a_{mi}c_{ik}BD \\
		\end{pmatrix} \\
		&=(AC)\otimes (BD)
	\end{align*}\par
	(3)\textbf{必要性:}假设此时$A$不可逆,则存在非零向量$x$使得$Ax=\mathbf{0}$。取非零向量$y$,则$(x\otimes y)$不是一个零向量。由(2)可得:
	\begin{equation*}
		(A\otimes B)(x\otimes y)=(Ax)\otimes(By)=\mathbf{0}\otimes(By)=\mathbf{0}
	\end{equation*}
	因为$A\otimes B$可逆,所以不存在非零向量$z$使得$(A\otimes B)z=\mathbf{0}$,但此时有$(A\otimes B)(x\otimes y)=\mathbf{0}$,矛盾,所以$A$可逆。同理可得$B$可逆。\par
	\textbf{充分性:}由(1)(2)可得:
	\begin{gather*}
		(A\otimes B)(A^{-1}\otimes B^{-1})=(AA^{-1})\otimes(BB^{-1})=I_m\otimes I_n=I_{mn} \\
		(A^{-1}\otimes B^{-1})(A\otimes B)=(A^{-1}A)\otimes(B^{-1}B)=I_m\otimes I_n=I_{mn}
	\end{gather*}
	所以$A\otimes B$可逆,逆矩阵就是$(A^{-1}\otimes B^{-1})$。\par
	(4)对任意的$1\leqslant i\leqslant m,\;1\leqslant j\leqslant n,\;1\leqslant k\leqslant p,\;1\leqslant l\leqslant q$,元素$a_{ij}b_{kl}$在$A\otimes B$中的行标为$(i-1)p+k$,列标为$(j-1)q+l$。于是$a_{ij}b_{kl}$在$(A\otimes B)^T$中的列标为$(i-1)p+k$,行标为$(j-1)q+l$。而$A^T\otimes B^T$列标为$(i-1)p+k$、行标为$(j-1)q+l$的元素为$a_{ij}b_{kl}$,于是$(A\otimes B)^T=A^T\otimes B^T$。
\end{proof}
\subsection{向量化算子}
\begin{definition}
	设$A=(a_{ij})\in M_{s\times m}(K)$,则:
	\begin{gather*}
		\operatorname{vec}(A)=(a_{11},a_{21},\dots,a_{s1},a_{12},a_{22},\dots,a_{s2},\dots,a_{1m},a_{2m},\dots,a_{sm})^T \\
		\operatorname{rvec}(A)=(a_{11},a_{12},\dots,a_{1m},a_{21},a_{22},\dots,a_{2m},\dots,a_{s1},a_{s2},\dots,a_{sm})
	\end{gather*}
	称$\operatorname{vec}$与$\operatorname{rvec}$为向量化算子。
\end{definition}
\subsubsection{交换矩阵的定义及其性质}
\begin{definition}
	对$A\in M_{s\times m}(K)$,存在唯一的$K_{sm}\in M_{sm\times sm}(K)$,使得:
	\begin{equation*}
		K_{sm}\operatorname{vec}(A)=\operatorname{vec}(A^T)
	\end{equation*}
	称$K_{sm}$为\gls{CommutationMatrix}。
\end{definition}
%\begin{property}
%	矩阵$A\in M_{s\times m}(K)$,交换矩阵具有如下性质:
%	\begin{enumerate}
%		\item $K_{sm}\operatorname{vec}(A)=\operatorname{vec}(A^T),\;K_{ms}\operatorname{vec}(A^T)=\operatorname{vec}(A)$;
%		\item $K_{sm}^{-1}=K_{ms}$,这里指的是广义逆矩阵;
%		\item $K_{sm}^T=K_{ms}$;
%		\item $K_{1n}=K_{n1}=I$
%	\end{enumerate}
%\end{property}
\subsubsection{向量化算子的性质}
\begin{property}\label{prop:VecOperator}
	向量化算子具有如下性质:
	\begin{enumerate}
		\item 设$A,B\in M_{s\times m}(K)$,则$\forall\;k_1,k_2\in K, \operatorname{vec}(k_1A+k_2B)=k_1\operatorname{vec}(A)+k_2\operatorname{vec}(B)$,即向量化算子是线性算子;
		\item 设$A,B\in M_{s\times m}(K)$,则$\operatorname{tr}(A^TB)=\operatorname{vec}(A)^T\operatorname{vec}(B)$;
		\item 设$A\in M_{s\times m}(K),\;B\in M_{m\times n}(K),\;C\in M_{n\times p}(K)$,$\operatorname{vec}(ABC)=(C^T\otimes A)\operatorname{vec}(B)$;
	\end{enumerate}
\end{property}
\begin{proof}
	(1)是显然的;\par
	(2)因为:
	\begin{equation*}
		\operatorname{tr}(A^TB)=\sum_{i=1}^{m}\sum_{j=1}^{s}a_{ji}b_{ji}
	\end{equation*}
	所以:
	\begin{align*}
		\operatorname{vec}(A)^T\operatorname{vec}(B)
		&=(a_{11},a_{21},\dots,a_{s1},a_{12},a_{22},\dots,a_{s2},\dots,a_{1m},a_{2m},\dots,a_{sm}) \\
		&\quad\cdot(b_{11},b_{21},\dots,b_{s1},b_{12},b_{22},\dots,b_{s2},\dots,b_{1m},b_{2m},\dots,b_{sm})^T \\
		&=\sum_{i=1}^{m}\sum_{j=1}^{s}a_{ji}b_{ji} \\
		&=\operatorname{tr}(A^TB)
	\end{align*}\par
	(3)设$B=(B_1,B_2,\dots,B_n),\;C=(C_1,C_2,\dots,C_p)$,则$ABC$的第$k$列:
	\begin{align*}
		ABC[:,k]
		&=A(B_1,B_2,\dots,B_n)C_k
		=A\sum_{i=1}^{n}B_ic_{ik} \\
		&=(c_{1k}A,c_{2k}A,\dots,c_{nk}A)
		\begin{pmatrix}
			B_1 \\
			B_2 \\
			\vdots \\
			B_n
		\end{pmatrix}
		=(C_k^T\otimes A)\operatorname{vec}(B)
	\end{align*}
	于是:
	\begin{align*}
		\operatorname{vec}(ABC)
		&=
		\begin{pmatrix}
			ABC[:,1] \\
			ABC[:,2] \\
			\vdots \\
			ABC[:,p]
		\end{pmatrix}
		=
		\begin{pmatrix}
			(C_1^T\otimes A)\operatorname{vec}(B) \\
			(C_2^T\otimes A)\operatorname{vec}(B) \\
			\vdots \\
			(C_p^T\otimes A)\operatorname{vec}(B)
		\end{pmatrix} \\
		&=
		\begin{pmatrix}
			C_1^T\otimes A \\
			C_2^T\otimes A \\
			\vdots \\
			C_p^T\otimes A
		\end{pmatrix}
		\operatorname{vec}(B)
		=(C^T\otimes A)\operatorname{vec}(B)
	\end{align*}
\end{proof}

\subsection{平方根阵}
\begin{definition}
	设对称阵$A\in M_n(\mathbb{R})$,其特征值记为$\lambda_i,\;i=1,2,\dots,n$。因为$A$是一个实对称阵,由\cref{prop:RMatEigen}(3)可知存在正交矩阵$Q$使得$A=Q^T\operatorname{diag}\{\seq{\lambda}{n}\}Q$。若$A\geqslant0$,由\cref{theo:PositiveSemidefinite}(3)的第五条可知$\lambda_i\geqslant0,\;i=1,2,\dots,n$,记:
	\begin{equation*}
		\varLambda^{\frac{1}{2}}=\operatorname{diag}\{\lambda_1^{\frac{1}{2}},\lambda_2^{\frac{1}{2}},\dots,\lambda_n^{\frac{1}{2}}\}
	\end{equation*}
	称:
	\begin{equation*}
		A^{\frac{1}{2}}=Q^T\varLambda^{\frac{1}{2}} Q
	\end{equation*}
	为$A$的\textbf{平方根阵}。
\end{definition}
\begin{property}\label{prop:SquareRootMat}
	设对称阵$A\in M_n(\mathbb{R})$且$A\geqslant0$,$A^{\frac{1}{2}}$具有如下性质:
	\begin{enumerate}
		\item $(A^{\frac{1}{2}})^2=A$;
		\item $A^{\frac{1}{2}}\geqslant0$;
		\item $A^{\frac{1}{2}}$是对称阵;
	\end{enumerate}
\end{property}
\begin{proof}
	由$A^{\frac{1}{2}}$的定义,有$A^{\frac{1}{2}}=Q^T\varLambda^{\frac{1}{2}}Q$和$A=Q^T\operatorname{diag}\{\seq{\lambda}{n}\}Q$,其中$Q$是一个正交矩阵,$\varLambda^{\frac{1}{2}}=\operatorname{diag}\{\lambda_1^{\frac{1}{2}},\lambda_2^{\frac{1}{2}},\dots,\lambda_n^{\frac{1}{2}}\}$,$\lambda_i,\;i=1,2,\dots,n$为$A$的特征值。\par
	(1)显然:
	\begin{equation*}
		(A^{\frac{1}{2}})^2=Q^T\varLambda^{\frac{1}{2}}QQ^T\varLambda^{\frac{1}{2}}Q=Q^T(\varLambda^{\frac{1}{2}})^2Q=Q^T\operatorname{diag}\{\seq{\lambda}{n}\}Q=A
	\end{equation*}\par
	(2)因为$Q$是正交矩阵,所以$Q^T=Q^{-1}$。于是:
	\begin{equation*}
		QA^{\frac{1}{2}}Q^{-1}=\varLambda^{\frac{1}{2}}
	\end{equation*}
	由\cref{theo:DiagCondition1}的必要性可知,$\lambda_i^{\frac{1}{2}}$为$A^{\frac{1}{2}}$的特征值,而$\lambda_i\geqslant0$,由\cref{theo:PositiveSemidefinite}(3)的第五条可知$A^{\frac{1}{2}}\geqslant0$。\par
	(3)显然:
	\begin{equation*}
		(A^{\frac{1}{2}})^T=(Q^T\varLambda^{\frac{1}{2}} Q)^T=Q^T(\varLambda^{\frac{1}{2}})^TQ=Q^T\varLambda^{\frac{1}{2}}Q=A^{\frac{1}{2}}
	\end{equation*}
\end{proof}
\subsubsection{平方根阵的逆矩阵}
\begin{theorem}
	设对称阵$A\in M_n(\mathbb{R})$,其特征值记为$\lambda_i,\;i=1,2,\dots,n$。因为$A$是一个实对称阵,由\cref{prop:RMatEigen}(3)可知存在正交矩阵$Q$使得$A=Q^T\operatorname{diag}\{\seq{\lambda}{n}\}Q$。若$A>0$,由\cref{theo:PositiveDefinite}(3)的第五条可知$\lambda_i>0,\;i=1,2,\dots,n$,记:
	\begin{equation*}
		\varLambda^{-\frac{1}{2}}=\operatorname{diag}\{\lambda_1^{-\frac{1}{2}},\lambda_2^{-\frac{1}{2}},\dots,\lambda_n^{-\frac{1}{2}}\}
	\end{equation*}
	则:
	\begin{equation*}
		A^{-\frac{1}{2}}=Q^T\varLambda^{-\frac{1}{2}} Q
	\end{equation*}
	是$A^{\frac{1}{2}}$的逆矩阵。
\end{theorem}
\begin{proof}
	显然:
	\begin{equation*}
		A^{\frac{1}{2}}A^{-\frac{1}{2}}=Q^T\operatorname{diag}\{\lambda_1^{\frac{1}{2}},\lambda_2^{\frac{1}{2}},\dots,\lambda_n^{\frac{1}{2}}\}QQ^T\operatorname{diag}\{\lambda_1^{-\frac{1}{2}},\lambda_2^{-\frac{1}{2}},\dots,\lambda_n^{-\frac{1}{2}}\}Q=I\qedhere
	\end{equation*}
\end{proof}
\begin{property}\label{prop:ReverseSquareRootMat}
	设对称阵$A\in M_n(\mathbb{R})$且$A>0$,$A^{\frac{1}{2}}$具有如下性质:
	\begin{enumerate}
		\item $(A^{-\frac{1}{2}})^2=A^{-1}$;
		\item $A^{-\frac{1}{2}}>0$;
		\item $A^{-\frac{1}{2}}$是对称阵;
	\end{enumerate}
\end{property}
\begin{proof}
	由$A^{-\frac{1}{2}}$的定义,有$A^{-\frac{1}{2}}=Q^T\varLambda^{-\frac{1}{2}}Q$和$A=Q^T\operatorname{diag}\{\seq{\lambda}{n}\}Q$,其中$Q$是一个正交矩阵,$\varLambda^{-\frac{1}{2}}=\operatorname{diag}\{\lambda_1^{-\frac{1}{2}},\lambda_2^{-\frac{1}{2}},\dots,\lambda_n^{-\frac{1}{2}}\}$,$\lambda_i,\;i=1,2,\dots,n$为$A$的特征值。\par
	(1)显然:
	\begin{equation*}
		(A^{-\frac{1}{2}})^2=Q^T\varLambda^{-\frac{1}{2}}QQ^T\varLambda^{-\frac{1}{2}}Q=Q^T\operatorname{diag}\{\lambda_1^{-1},\lambda_2^{-1},\dots,\lambda_n^{-1}\}Q
	\end{equation*}
	而:
	\begin{align*}
		A^{-1}&=(Q^T\operatorname{diag}\{\seq{\lambda}{n}\}Q)^{-1}=Q^{-1}\operatorname{diag}\{\lambda_1^{-1},\lambda_2^{-1},\dots,\lambda_n^{-1}\}(Q^T)^{-1} \\
		&=Q^T\operatorname{diag}\{\lambda_1^{-1},\lambda_2^{-1},\dots,\lambda_n^{-1}\}Q
	\end{align*}
	所以$(A^{-\frac{1}{2}})^2=A^{-1}$。\par
	(2)因为$Q$是正交矩阵,所以$Q^T=Q^{-1}$。于是:
	\begin{equation*}
		QA^{-\frac{1}{2}}Q^{-1}=\varLambda^{-\frac{1}{2}}
	\end{equation*}
	由\cref{theo:DiagCondition1}的必要性可知,$\lambda_i^{-\frac{1}{2}}$为$A^{-\frac{1}{2}}$的特征值,而$\lambda_i>0$,由\cref{theo:PositiveDefinite}(3)的第五条可知$A^{\frac{1}{2}}>0$。\par
	(3)显然:
	\begin{equation*}
		(A^{-\frac{1}{2}})^T=(Q^T\varLambda^{-\frac{1}{2}} Q)^T=Q^T(\varLambda^{-\frac{1}{2}})^TQ=Q^T\varLambda^{-\frac{1}{2}}Q=A^{-\frac{1}{2}}
	\end{equation*}
\end{proof}