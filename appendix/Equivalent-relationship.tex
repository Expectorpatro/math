\section{等价关系}

\begin{definition}\label{CartesianProduct}
	对于任意两个非空集合$S,M$,称:
	\begin{equation*}
		\{(a,b):a\in S,\;b\in M\}
	\end{equation*}
	为集合$S$与集合$M$的\gls{CartesianProduct},记为$S\times M$。其中两个元素$(a_1,b_1)$与$(a_2,b_2)$如果满足$a_1=a_2,\;b_1=b_2$,则称二者相等,记作$(a_1,b_1)=(a_2,b_2)$。
\end{definition}
\begin{definition}
	设$S$是一个非空集合,把$S\times S$的一个子集$W$叫作$S$上的一个\gls{BinaryRelation}。如果$(a,b)\in W$,则称$a$与$b$有$W$关系;如果$(a,b)\notin W$,则称$a$与$b$没有$W$关系。当$a$与$b$有$W$关系时,记作$aWb$,或$a\sim b$。
\end{definition}
\begin{definition}
	集合$S$上的一个二元关系$\sim$如果具有如下性质:对$\forall\;a,b,c\in S$,有:
	\begin{enumerate}
		\item $a\sim a$(反身性);
		\item $a\sim b\Rightarrow b\sim a$(对称性);
		\item $a\sim b,\;b\sim c\Rightarrow a\sim c$(传递性)。
	\end{enumerate}
	那么称$\sim$是集合$S$上的一个\gls{EquivalenceRelation}。
\end{definition}
\begin{definition}
	设$\sim$是集合$S$上的一个等价关系,$a\in S$,令:
	\begin{equation*}
		\overline{a}\coloneq\{x\in S:x\sim a\}
	\end{equation*}
	称$\overline{a}$是由$a$确定的\gls{EquivalenceClass},称$a$是等价类$\overline{a}$的一个代表。
\end{definition}
\begin{property}
	等价类具有如下基本性质:
	\begin{enumerate}
		\item $a\in\overline{a}$;
		\item $x\in\overline{a}\Leftrightarrow x\sim a$;
		\item $x\sim y\Leftrightarrow\overline{x}=\overline{y}$。
	\end{enumerate}
\end{property}
\begin{proof}
	(1)等价关系具有反身性。(2)由等价类的定义可直接得出。\par
	(3)\textbf{充分性:}因为$x\in\overline{x}$且$\overline{x}=\overline{y}$,所以$x\in\overline{y}$,由$\overline{y}$的定义,$x\sim y$。\par
	\textbf{必要性:}任取$a\in\overline{x}$,则$a\sim x$。因为$x\sim y$,由等价关系的传递性,$a\sim y$,即$a\in\overline{y}$。由$a$的任意性,$\overline{x}\subseteq\overline{y}$。同理可证得$\overline{y}\subseteq\overline{x}$,所以$\overline{x}=\overline{y}$。
\end{proof}
\begin{corollary}
	用不同代表表示的等价类是一样的,即代表的选择与等价类本身无关。
\end{corollary}
\begin{proof}
	由等价类基本性质(3)可直接得到。
\end{proof}
\begin{theorem}\label{theo:EquivalentClass}
	设$\sim$是集合$S$上的一个等价关系。对$\forall\;a,b\in S$,有$\overline{a}=\overline{b}$或$\overline{a}\cap\overline{b}=\varnothing$。
\end{theorem}
\begin{proof}
	如果$\overline{a}\ne\overline{b}$,假设此时$\overline{a}\cap\overline{b}\ne\varnothing$,取$c\in\overline{a}\cap\overline{b}$,则有$c\sim a$且$c\sim b$。由等价关系的对称性与传递性可得$a\sim b$,根据等价类的基本性质(3),此时应有$\overline{a}=\overline{b}$,矛盾,所以$\overline{a}\cap\overline{b}=\varnothing$。
\end{proof}
\begin{definition}
	如果集合$S$可以表示为一些非空子集的并集,且这些子集不相交,即:
	\begin{equation*}
		\exists\;S_i\subseteq S,\;\underset{i\in I}{\overset{}{\cup}}S_i=S,\;S_i\cap S_j=\varnothing,\;i\ne j,\;i,j\in I
	\end{equation*}
	其中$I$是指标集。称集合$\{S_i:i\in I\}$是$S$的一个\gls{Partition},记作$\pi(S)$。
\end{definition}
\begin{theorem}
	设$\sim$是集合$S$上的一个等价关系,则所有等价类组成的集合是$S$的一个划分,记作$\pi_\sim(S)$。
\end{theorem}
\begin{proof}
	对$\forall\;a_i\in S$,其中$i$是指标集,有$a_i\in\overline{a_i}$,于是$S=\underset{i\in I}{\overset{}{\cup}}\overline{a_i}$。由\cref{theo:EquivalentClass}可得,若$i\ne j$,$\overline{a_i}\cap\overline{a_j}=\varnothing$,从而所有等价类组成的集合是$S$的一个划分。
\end{proof}
\begin{definition}
	设$\sim$是集合$S$上的一个等价关系,所有等价类组成的集合称为$S$对于关系$\sim$的\gls{QuotientSet}。
\end{definition}
\begin{definition}
	设$\sim$是集合$S$上的一个等价关系,一种量或者一种表达式如果对于同一个等价类里的元素是相等的,那么称这种量或表达式是一个\textbf{不变量};恰好能完全绝对等价类的一组不变量称为\textbf{完全不变量}。
\end{definition}











