% 添加前言到目录
% toc:表示要将条目添加到的列表类型,这里是目录(Table of Contents)。
% chapter:表示条目的类型,这里指的是一个章节。其他可能的值包括section(节)、subsection(小节)等,取决于你希望将条目添加到哪个层级。
% 前言:这是在目录中显示的文本内容。
\addcontentsline{toc}{chapter}{前言}

\chapter*{前言}
\hspace{2em}其实这个想法在我大二之初准备全国大学生数学建模竞赛的时候就已经产生。第一次打数模国赛时整篇论文的编排工作由组内成员负责,当时我并不了解\LaTeX{}的语法与使用,但由于IGEM校园选拔赛要求每位申请数模组的同学提交一篇自己独立完成的数学建模论文,我便开始学习\LaTeX{}。\par
本科阶段我曾学习大量数学专业课程,从基础的数学分析、高等代数到后来的随机过程、实变函数与泛函分析。我曾苦于寻找合适的教材(请允许我吐槽一下国内的本科教材,北大版高等代数与华东师大版数学分析简直不是人能看得下去的东西),感谢知乎、csdn等处各位同学、学长学姐们的分享,我才得以接触到一些适合我的教材。以数学分析为例,我主要以北京大学张筑生教授编写的《数学分析新讲》为学习教材,以苏州大学谢惠民、恽自求等教授编写的《数学分析习题课讲义》为习题册进行查漏补缺,同时也曾阅读复旦大学陈纪修教授、北京大学伍胜健教授编写的《数学分析》,对清华大学于品教授的讲义也曾匆匆一瞥,这些教材都可以称为是国内顶尖的数学分析教材了。可是每本书都有每本书自己的特点,对于任何一本书,以上所列的其他书籍都对其有着一定的补充,曾考虑使用marginnote软件将这些书籍整理成一些思维导图,但思来想去还是觉得不如自己动手重新编撰为一本属于自己的教材。首先,思维导图只能援引原著内容,难以进行个性化表述,即使可以,操作与使用起来都有一定的复杂性与不便性;其次,倘若重新整理,内容编排上可按我的个人习惯以及思维过程进行调整,也可引入适当高阶课程中的知识进行更高观点的阐释,不同课程中的交叉部分也很容易集中叙述;同时,\LaTeX{}的交叉引用(点击即跳转)的功能也十分吸引我;最后,自己编一本书,倘若确实质量不错,也可上传至Github或个人主页供后来者一阅。\par
曾想过在Bilibili上录播自己在SRT阶段学习到的机器学习、深度学习及生物信息学知识作为分享,同时也可作为自己复习的一手资料,也曾想过利用csdn、知乎这两个平台发布自己整理的笔记,但终究没有付诸实践,略有遗憾,但我现在做的本科阶段知识的整理,我一定会完成它。\par
本书打算整理我本科所学的所有数学类课程知识,将伴随着我个人的考研复习进度逐步完成,预计耗时一年半。\par
\begin{flushright}
	倪兴程\\
	2024年10月18日于南京农业大学教四楼B409
\end{flushright}

